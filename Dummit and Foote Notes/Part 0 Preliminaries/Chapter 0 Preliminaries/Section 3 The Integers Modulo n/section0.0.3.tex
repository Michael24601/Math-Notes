

\documentclass[12pt]{article}

\usepackage[margin=1in]{geometry}

% For using float option H that places figures exatcly where we want them
\usepackage{float}
% makes figure font bold
\usepackage{caption}
\captionsetup[figure]{labelfont=bf}
% For text generation
\usepackage{lipsum}
% For drawing
\usepackage{tikz}
% For smaller or equal sign and not divide sign
\usepackage{amssymb}
% For the diagonal fraction
\usepackage{xfrac}
% For enumerating exercise parts with letters instead of numbers
\usepackage{enumitem}
% For dfrac, which forces the fraction to be in display mode (large) e
% even in math mode (small)
\usepackage{amsmath}
% For degree sign
\usepackage{gensymb}
% For "\mathbb" macro
\usepackage{amsfonts}
\newcommand{\N}{\mathbb{N}}
\newcommand{\Z}{\mathbb{Z}}
\newcommand{\Q}{\mathbb{Q}}
\newcommand{\R}{\mathbb{R}}
\newcommand{\C}{\mathbb{C}}

% overline short italic
\newcommand{\olsi}[1]{\,\overline{{#1}}}

\usepackage{indentfirst}
\usetikzlibrary{shapes,positioning,fit,calc}

\usepackage{changepage} % for adjustwidth environment

\title{%
    \Huge Abstract Algebra \\
    \large by \\
    \Large Dummit and Foote \\~\\
    \huge Part 0: Preliminaries \\
    \LARGE Chapter 0: Preliminaries \\
    \Large Section 3: The Integers Modulo n
}
\date{2024-03-17}
\author{Michael Saba}

\begin{document}
    \pagenumbering{gobble}
    \maketitle
    \newpage
    \pagenumbering{arabic}

    For a fixed positive integer $n$,
    we will define an equivalency relation on $\Z$
    \[ a \sim b \text{ if and only if } n \mid (b - a) \]
    This is clearly both reflexive and symmetric.
    It is also transitive;
    if $a \sim b$ and $b \sim c$,
    then $n \mid (b-a)$ and $b \mid (c-b)$.
    If so, then $(b-a) = np$ and $(c-b) = nq$
    for some integers $n$ and $q$.
    Then the sum of the two terms $(b-a) + (c-b) = n(p+q)$,
    where $p+q$ is an integer.
    So $n \mid (b-a) + (c-b)$,
    which means that $n \mid (c-a)$,
    which means that $a \sim c$. \\
    Insead of using $\sim$,
    for the equivalence relation,
    if $a \sim b$,
    we write $a \equiv b \, (\text{mod }n)$,
    and say that $a$ is congruent to $b$ modulo $n$. \\
    The equivalence class of any integer $a$
    using this congruence equivalence relation
    is called the \textbf{congruence class} or \textbf{residue class}
    of $a$ mod $n$. \\
    We will denote this equivalence class using $\olsi{a}$
    (note that any integer $b \in \olsi{a}$ can also be used
    as a representative as mentioned in section 0.0.2,
    so $\olsi{a} = \olsi{b}$). \\

    The equivalence class $\olsi{a}$ is essentially the set 
    of integers that differ from $a$ by a mutliple of $n$.
    This is because, if we have $n \mid (b - a)$
    for some positive integer $b$,
    then by our defintion of division in section 0.0.2,
    we have $(b-a) = nm$ for an integer $m$
    determined by our choice of $b$.
    So $a + nm = b$,
    meaning any positive integer $b$ congruent to $a$ modulo $n$
    must differ from $a$ by a multiple of $n$.
    So
    \[ \olsi{a} = \{ a, a \pm n, a \pm 2n, a \pm 3n \dots \} \]
    Another way of thinking about this class is that it is the set
    of integers that have the same remainder as $a$ does
    when divided by $n$ using the division algorithm from section 0.0.2. \\

    The number of distinct equivalence classes modulo $n$
    (which partition $\Z$) is $n$.
    This is easy to see, since each equivalence classe
    determines the remainder of its elements when divided by $n$,
    and the remainder $r$ has to be an integer $0 \leqslant r < n$. \\
    
    The $n$ congruence classes modulo $n$ are
    \[ \{ \olsi{0}, \olsi{1}, \dots \olsi{n-1} \} \]
    We can use any representative (element) from the class
    to represent it,
    but we go with the smallest element,
    as it is the exact value of the remainder
    when elements of the class are divided by $n$
    (they are the integers $a$ for which $q = 0$ in $nq + r$
    as per the division algorithm of section 0.0.2).
    The process of finding the equivalence class of an integer $a$
    is sometimes called reudcing $a$ modulo $n$,
    where we attempt to find
    the smallest integer congruent to $a$ modulo $n$,
    the representative of the class that we most often use. \\
    This set of equivalence classes is called $\Z/n\Z$,
    \textbf{the integers modulo $\mathbf{\textit{n}}$}.
    The reason we use this notation will become clear
    in chapter 1.3.1, regarding Quotient Groups. \\

    Note that the set of equivalence classes changes with
    each positive integer $n$ we fix at the start,
    as a new $n$ defines a new relation,
    with new equivalence classes.
    for instance, in the set $\Z/3\Z$
    (when working in the relation with $n = 3$),
    $6$ and $9$ are part of the same class $\olsi{0}$,
    as both have remainder $0$ when divided by $3$. 
    But in $\Z/4\Z$,
    (when working in the relation with $n = 3$),
    $6 \in \olsi{2}$ and $9 \in \olsi{1}$,
    as $2$ and $1$ are the remainders of $6$ and $9$
    when divided by $n$. \\

    \textbf{Modular Arithmetic} is defined as addition and multiplication
    performed on the elements of $\Z/n\Z$:
    \[ \olsi{a} + \olsi{b} = \olsi{a + b} 
    \quad \text{ and } \quad
    \olsi{a} \cdot \olsi{b} = \olsi{a \cdot b} \]
    This may seem strange at first,
    to define addition and multiplication on sets,
    however,
    what this means is that,
    for any element of $\olsi{a}$,
    and any element of $\olsi{b}$,
    the sum of the two elements will belong to $\olsi{a + b}$,
    and the product of the two elements
    will belong to $\olsi{a \cdot b}$.
    Really all it means is that the remainder of the product
    is the product of the remainder,
    and that the sum of the remainders is the remainder of the sums,
    when dividing by $n$. \\
    Obviously this needs to be proven to be true,
    as it depends not just on $a$ and $b$,
    but on all the representatives of the sets. 
    It requires being well-defined. \\


    Being well-defined is a property of a function,
    meaning that if $a = b$, $f(a) = f(b)$
    as we saw in section 0.0.1.
    This is easy to show when our elements are integers for example. \\
    Binary operators themselves are functions that
    map the association of two elements
    (element of the cartesian product) to a third element in a set.
    We will talk about this in detail in section 1.1.1. \\
    For now, we will note that the addition and multiplication
    operations on the equivalence classes can be seen as
    functions $\Z/n\Z \times \Z/n\Z \to \Z/n\Z$. \\
    If for instance addition of residue classes is well defined,
    the function described above would also be well defined.
    This means that that if $\olsi{a} = \olsi{b}$,
    and $\olsi{c} = \olsi{d}$
    (meaning $b \in \olsi{a}$ and $d \in \olsi{c}$),
    that $f(\olsi{a}, \olsi{c}) = f(\olsi{b}, \olsi{d})$,
    and by extension, that $\olsi{a} + \olsi{c}$
    is the same class as $\olsi{b} + \olsi{d}$. \\
    However, we can't be sure that this function is well defined
    because the elements themselves
    $\olsi{a}$ and $\olsi{b}$ are sets. \\
    We defined $\olsi{a} + \olsi{c}$
    as being the residue class $\olsi{a + c}$.
    Because our elements $\olsi{a}$ and $\olsi{b}$ are sets,
    we could have easily picked different representatives
    $b$ and $d$,
    and we need to make sure that the result $\olsi{b + d}$
    remains the same as $\olsi{a+b}$,
    otherwise addition becomes ambiguous
    (the results depend on the representative we choose,
    not just the class itself).
    We need to show that it is true for all elements of the classes,
    not just $a$ and $c$.
    In other words, it must be true regardless
    of the choice of representatives. \\
    Note that addition will turn out to be well defined,
    and that's because
    $(a + c) \equiv ((a \mod n) \mod (c \mod n)) \mod n$,
    which makes $\olsi{a} + \olsi{c} = \olsi{a + c}$
    true for any $a$ or $c$.
    If we had picked some other mapping,
    such as $\olsi{a} + \olsi{c} = \olsi{a^3 + c^2 + 1}$,
    it would not have been welld defined. \\

    Now to prove the last statement,
    we show that, for any $a_1, a_1, b_1, b_2 \in \Z$,
    with $\olsi{a_1} = \olsi{b_1}$ and  $\olsi{a_2} = \olsi{b_2}$,
    we must show that $\olsi{a_1 + a_2} = \olsi{b_1 + b_2}$
    and  $\olsi{a_1a_2} = \olsi{b_1b_2}$
    even when $a_1 = b_1$ and $a_2 = b_2$: \\
    To prove this, we note that since $\olsi{a_1} = \olsi{b_1}$,
    $a_1 \equiv b \, (\text{mod } n)$,
    so $n \mid (b_1 - a_1)$. So $a_1 = b_1 + sn$ for some integer $s$.
    Likewise, $a_2 = b_2 + tn$ for some integer $t$.
    So $(a_1 + a_2) = (b_1 + b_2) + (ts)n$,
    which means that $n \mid ((b_1 + b_2) - (a_1 + a_2))$,
    so that $(a_1 + a_2) \equiv (b_1 + b_2) \, (\text{mod } n)$.
    Thus, $\olsi{a_1 + a_2} = \olsi{b_1 + b_2}$ for an arbitrary
    choice of representatives. \\
    By the same argument,
    $(a_1a_2) = (b_1 + sn) + (b_2 + tn) = (b_1b_2) + (b_1t+b_2s + stn)n$.
    So $n \mid ((b_1b_2) - (a_1a_2))$,
    which means that $(a_1a_2) \equiv (b_1b_2) \, (\text{mod } n)$,
    so that $\olsi{a_1a_2} = \olsi{b_1b_2}$ for an arbitrary
    choice of representatives. \\
    This completes the proof. \\

    Whenever we define an operation on sets
    (such as residue classes $\olsi{a}$)
    such as addition, mutliplication, inverses...
    we need to show that the result does not depend on which
    representative of the set we choose,
    meaning that the operation is well defined.
    This is not specific to just addition and multiplication,
    nor to binary functions (operators).
    For instance, the inverse of $\olsi{a}$
    and can be thought of as a unary operator,
    or as a function $f: \Z/n\Z \to \Z/n\Z$,
    which also needs to be shown to be well defined.

    An important subset of $\Z/n\Z$
    is the set consisting of the collection of residue classes
    that have a multiplicative inverse in $\Z/n\Z$:
    \[ (\Z/n\Z)^\times
    = \{ \olsi{a} \in \Z/n\Z \mid \text{gcd}(a, n) = 1 \} \]
    A \textbf{multiplicative inverse} of a class $\olsi{a}$
    is a class $\olsi{x}$ such that $\olsi{a}\olsi{x} = 1$.
    Obviously, in order for the multiplicative
    inverse to be well defined,
    since $\olsi{a}$ and $\olsi{x}$ are sets,
    we need to avoid any ambiguaty.
    This means that all representatives
    $x$ and $a$ of $\olsi{a}$ and $\olsi{x}$
    must satisfy $ax \equiv 1 \mod n$
    whenever $\gcd(a, n) = 1$
    (and vice-versa).
    Basically this means that the definition must hold
    for any representative $a$ of $\olsi{a}$. \\
    This is proven in an exercise using the Euclidian Algorithm. \\
    

\end{document}
