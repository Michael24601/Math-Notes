
\documentclass[12pt]{article}
\usepackage[margin=1in]{geometry}


%===============================================================================
%================================== PACKAGES ===================================
%===============================================================================

% For using float option H that places figures 
% exatcly where we want them
\usepackage{float}
% makes figure font bold
\usepackage{caption}
\captionsetup[figure]{labelfont=bf}
% For text generation
\usepackage{lipsum}
% For drawing
\usepackage{tikz}
% For smaller or equal sign and not divide sign
\usepackage{amssymb}
% For the diagonal fraction
\usepackage{xfrac}
% For enumerating exercise parts with letters instead of numbers
\usepackage{enumitem}
% For dfrac, which forces the fraction to be in display mode (large) e
% even in math mode (small)
\usepackage{amsmath}
% For degree sign
\usepackage{gensymb}
% For "\mathbb" macro
\usepackage{amsfonts}
% For positioning 
\usepackage{indentfirst}
\usetikzlibrary{shapes,positioning,fit,calc}
% for adjustwidth environment
\usepackage{changepage}
% for arrow on top
\usepackage{esvect}
% for mathbb 1
\usepackage{bbm}
% for mathsrc
\usepackage[mathscr]{eucal}
% For degree sign
\usepackage{gensymb}
% For quotes
\usepackage{csquotes}
% For vertical lines
\usepackage{mathtools}
% For cols
\usepackage{multicol}

% for tikz
\usepackage{pgfplots}
\pgfplotsset{compat=1.18}
\usepackage{amsmath}
\usepgfplotslibrary{groupplots}


%===============================================================================
%==================================== FONTS ====================================
%===============================================================================


% Mathcal
\newcommand{\acal}{\mathcal{A}}
\newcommand{\bcal}{\mathcal{B}}
\newcommand{\ccal}{\mathcal{C}}
\newcommand{\dcal}{\mathcal{D}}
\newcommand{\ecal}{\mathcal{E}}
\newcommand{\fcal}{\mathcal{F}}
\newcommand{\gcal}{\mathcal{G}}
\newcommand{\hcal}{\mathcal{H}}
\newcommand{\ical}{\mathcal{I}}
\newcommand{\jcal}{\mathcal{J}}
\newcommand{\kcal}{\mathcal{K}}
\newcommand{\lcal}{\mathcal{L}}
\newcommand{\mcal}{\mathcal{M}}
\newcommand{\ncal}{\mathcal{N}}
\newcommand{\ocal}{\mathcal{O}}
\newcommand{\pcal}{\mathcal{P}}
\newcommand{\qcal}{\mathcal{Q}}
\newcommand{\rcal}{\mathcal{R}}
\newcommand{\scal}{\mathcal{S}}
\newcommand{\tcal}{\mathcal{T}}
\newcommand{\ucal}{\mathcal{U}}
\newcommand{\vcal}{\mathcal{V}}
\newcommand{\wcal}{\mathcal{W}}
\newcommand{\xcal}{\mathcal{X}}
\newcommand{\ycal}{\mathcal{Y}}
\newcommand{\zcal}{\mathcal{Z}}

% Mathfrak
\newcommand{\afrak}{\mathfrak{A}}
\newcommand{\bfrak}{\mathfrak{B}}
\newcommand{\cfrak}{\mathfrak{C}}
\newcommand{\dfrak}{\mathfrak{D}}
\newcommand{\efrak}{\mathfrak{E}}
\newcommand{\ffrak}{\mathfrak{F}}
\newcommand{\gfrak}{\mathfrak{G}}
\newcommand{\hfrak}{\mathfrak{H}}
\newcommand{\ifrak}{\mathfrak{I}}
\newcommand{\jfrak}{\mathfrak{J}}
\newcommand{\kfrak}{\mathfrak{K}}
\newcommand{\lfrak}{\mathfrak{L}}
\newcommand{\mfrak}{\mathfrak{M}}
\newcommand{\nfrak}{\mathfrak{N}}
\newcommand{\ofrak}{\mathfrak{O}}
\newcommand{\pfrak}{\mathfrak{P}}
\newcommand{\qfrak}{\mathfrak{Q}}
\newcommand{\rfrak}{\mathfrak{R}}
\newcommand{\sfrak}{\mathfrak{S}}
\newcommand{\tfrak}{\mathfrak{T}}
\newcommand{\ufrak}{\mathfrak{U}}
\newcommand{\vfrak}{\mathfrak{V}}
\newcommand{\wfrak}{\mathfrak{W}}
\newcommand{\xfrak}{\mathfrak{X}}
\newcommand{\yfrak}{\mathfrak{Y}}
\newcommand{\zfrak}{\mathfrak{Z}}

% Mathscr
\newcommand{\ascr}{\mathscr{A}}
\newcommand{\bscr}{\mathscr{B}}
\newcommand{\cscr}{\mathscr{C}}
\newcommand{\dscr}{\mathscr{D}}
\newcommand{\escr}{\mathscr{E}}
\newcommand{\fscr}{\mathscr{F}}
\newcommand{\gscr}{\mathscr{G}}
\newcommand{\hscr}{\mathscr{H}}
\newcommand{\iscr}{\mathscr{I}}
\newcommand{\jscr}{\mathscr{J}}
\newcommand{\kscr}{\mathscr{K}}
\newcommand{\lscr}{\mathscr{L}}
\newcommand{\mscr}{\mathscr{M}}
\newcommand{\nscr}{\mathscr{N}}
\newcommand{\oscr}{\mathscr{O}}
\newcommand{\pscr}{\mathscr{P}}
\newcommand{\qscr}{\mathscr{Q}}
\newcommand{\rscr}{\mathscr{R}}
\newcommand{\sscr}{\mathscr{S}}
\newcommand{\tscr}{\mathscr{T}}
\newcommand{\uscr}{\mathscr{U}}
\newcommand{\vscr}{\mathscr{V}}
\newcommand{\wscr}{\mathscr{W}}
\newcommand{\xscr}{\mathscr{X}}
\newcommand{\yscr}{\mathscr{Y}}
\newcommand{\zscr}{\mathscr{Z}}

% Mathbb
\newcommand{\abb}{\mathbb{A}}
\newcommand{\bbb}{\mathbb{B}}
\newcommand{\cbb}{\mathbb{C}}
\newcommand{\dbb}{\mathbb{D}}
\newcommand{\ebb}{\mathbb{E}}
\newcommand{\fbb}{\mathbb{F}}
\newcommand{\gbb}{\mathbb{G}}
\newcommand{\hbb}{\mathbb{H}}
\newcommand{\ibb}{\mathbb{I}}
\newcommand{\jbb}{\mathbb{J}}
\newcommand{\kbb}{\mathbb{K}}
\newcommand{\lbb}{\mathbb{L}}
\newcommand{\mbb}{\mathbb{M}}
\newcommand{\nbb}{\mathbb{N}}
\newcommand{\obb}{\mathbb{O}}
\newcommand{\pbb}{\mathbb{P}}
\newcommand{\qbb}{\mathbb{Q}}
\newcommand{\rbb}{\mathbb{R}}
\newcommand{\sbb}{\mathbb{S}}
\newcommand{\tbb}{\mathbb{T}}
\newcommand{\ubb}{\mathbb{U}}
\newcommand{\vbb}{\mathbb{V}}
\newcommand{\wbb}{\mathbb{W}}
\newcommand{\xbb}{\mathbb{X}}
\newcommand{\ybb}{\mathbb{Y}}
\newcommand{\zbb}{\mathbb{Z}}


%===============================================================================
%=============================== SPECIAL SYMBOLS ===============================
%===============================================================================


% Orbit (group theory)
\newcommand{\orbit}{\mathcal{O}}
% Normal group
\newcommand{\normal}{\mathcal{N}}
% Indicator function
\newcommand{\indicator}{\mathbbm{1}}
% Laplace transform
\newcommand{\laplace}[1]{\mathcal{L}}
% Epsilon shorthand
\newcommand{\eps}{\varepsilon}
% Omega
\newcommand{\om}{\omega}
\newcommand{\Om}{\Omega}

%===============================================================================
%================================== OPERATORS ==================================
%===============================================================================


% Inverse exponent
\newcommand{\inv}[0]{^{-1}}
% Overline bar
\newcommand{\olsi}[1]{\,\overline{\!{#1}}}
% Less than or equal slanted
\newcommand{\seqs}{\leqslant}
% Greater or equal slanted
\newcommand{\geqs}{\geqslant}
% Subset or equal
\newcommand{\sub}{\subseteq}
% Proper subset
\newcommand{\prosub}{\subset}
% from
\newcommand{\from}{\leftarrow}

% Parantheses
\newcommand{\para}[1]{\left( #1 \right)}
% Curly Braces
\newcommand{\curl}[1]{\left\{ #1 \right\}}
% Brackets
\newcommand{\brac}[1]{\left[ #1 \right]}
% Angled Brackets
\newcommand{\ang}[1]{\left\langle #1 \right\rangle}
% Norm
\newcommand{\norm}[1]{\left\| #1 \right\|}

% Piece wise (use \\ between cases)
\newcommand{\piecewise}[1]{\begin{cases} #1 \end{cases}}

% Bold symbol shorthand
\newcommand{\bl}{\boldsymbol}

% Vertical space
\newcommand{\vs}[1]{\vspace{#1 pt}}
% Horizontal ertical space
\newcommand{\hs}[1]{\hspace{#1 pt}}


%===============================================================================
%============================== TEXT BASED SYMBOLS =============================
%===============================================================================


% Radians
\newcommand{\rad}{\text{rad}}
% Least Common Multiple
\newcommand{\lcm}{\text{lcm}}
% Automorphism
\newcommand{\Aut}{\text{Aut}}
% Variance
\newcommand{\var}{\text{Var}}
% Covariance
\newcommand{\cov}{\text{Cov}}
% Cofactor (matrix)
\newcommand{\cof}{\text{Cof}}
% Adjugate (matrix)
\newcommand{\adj}{\text{Adj}}
% Trace (matrix)
\newcommand{\tr}{\text{tr}}
% Standard deviation
\newcommand{\std}{\text{Std}}
% Correlation coefficient
\newcommand{\corr}{\text{Corr}}
% Sign
\newcommand{\sign}{\text{sign}}

% And text
\newcommand{\AND}{\text{ and }}
% Or text 
\newcommand{\OR}{\text{ or }}
% For text 
\newcommand{\FOR}{\text{ for }}
% If text
\newcommand{\IF}{\text{ if }}
% When text
\newcommand{\WHEN}{\text{ when }}
% Where text
\newcommand{\WHERE}{\text{ where }}
% Then text
\newcommand{\THEN}{\text{ then }}
% Such that text
\newcommand{\SUCHTHAT}{\text{ such that }}

% 1st
\newcommand{\st}[1]{#1^{\text{st}}}
% 2nd
\newcommand{\nd}[1]{#1^{\text{nd}}}
% 3rd
\newcommand{\rd}[1]{#1^{\text{rd}}}
% nth
\newcommand{\nth}[1]{#1^{\text{th}}}


%===============================================================================
%========================= PROBABILITY AND STATISTICS ==========================
%===============================================================================


% Permutation
\newcommand{\perm}[2]{{}^{#1}\!P_{#2}}
% Combination
\newcommand{\comb}[2]{{}^{#1}C_{#2}}

% Baye's risk
\newcommand{\risk}[1]{\mathscr{R}_{#1}}
% Baye's optimal risk
\newcommand{\riskOptimal}[1]{\mathscr{R}_{#1}^*}
% Baye's empirical risk
\newcommand{\riskEmpirical}[2]{\hat{\mathscr{R}}_{#1}^{#2}}


%===============================================================================
%=================================== CALCULUS ==================================
%===============================================================================


% d over d derivative
\newcommand{\dd}[2]{\dfrac{d#1}{d#2}}
% partial d over d derivative
\newcommand{\partialdd}[2]{\dfrac{\partial #1}{\partial #2}}
% delta d over d derivative
\newcommand{\deltadd}[2]{\dfrac{\Delta #1}{\Delta #2}}

% Integration between a and b
\newcommand{\integral}[4]{\int_{#1}^{#2} #3 \, #4}
% Integration in some space 
\newcommand{\boundIntegral}[2]{\int_{#1} #2 \, d#1}

% Limit
\newcommand{\limit}[3]{\lim_{#1 \to #2} #3}


%===============================================================================
%================================  BIG SYMBOLS  ================================
%===============================================================================


% Sum
\newcommand{\sumof}[2]{\sum_{#1}^{#2}}
% Product
\newcommand{\productof}[2]{\prod_{#1}^{#2}}
% Union
\newcommand{\unionof}[2]{\bigcup_{#1}^{#2}}
% Intersection
\newcommand{\intersectionof}[2]{\bigcap_{#1}^{#2}}
% Or
\newcommand{\orof}[2]{\bigvee_{#1}^{#2}}
% And
\newcommand{\andof}[2]{\bigwedge_{#1}^{#2}}


%===============================================================================
%=============================== LINEAR ALGEBRA ================================
%===============================================================================


% Bold vector arrow
\newcommand{\bv}[1]{\vec{\mathbf{#1}}}

% Matrix or vector (use // for column, & for row) 
% with brackets
\newcommand{\bmat}[1]{\begin{bmatrix} #1 \end{bmatrix}}
% Matrix or vector (use // for column, & for row) 
% with curved brackets
\newcommand{\pmat}[1]{\begin{pmatrix} #1 \end{pmatrix}}
% Matrix or vector (use // for column, & for row) 
% with lines on either side 
\newcommand{\lmat}[1]{\begin{vmatrix} #1 \end{vmatrix}}
% Matrix or vector (use // for column, & for row) 
% with curly braces
\newcommand{\cmat}[1]{\begin{Bmatrix} #1 \end{Bmatrix}}
% Matrix or vector (use // for column, & for row) 
% with no braces
\newcommand{\mat}[1]{\begin{matrix} #1 \end{matrix}}


%===============================================================================
%================================ LARGE OBJECTS ================================
%===============================================================================

% Multiple lines
\newcommand{\multiline}[1]{
\begin{align*}
    #1
\end{align*}
}

% Multiple lines with equation numbers
\newcommand{\eqmultiline}[1]{
\begin{align*}
    #1
\end{align*}
}

% Color
\newcommand{\colorText}[2]{
\begingroup
\color{#1}
    #2
\endgroup
}

% Centered figure
\newcommand{\centerFigure}[2]{
    \begin{figure}[h]
        \centering
            #1
        \caption{#2}
    \end{figure}
}

% Tikz figure
\newcommand{\tikzGraphic}[1]{
    \begin{center}
    \begin{tikzpicture}
        #1
    \end{tikzpicture}
    \end{center}
}

% Enumerate numbers (seperate by \item)
\newcommand{\numbers}{\textbf{\number*)}}

% Enumerate letters (seperate by \item)
\newcommand{\letters}{\textbf{\alph*)}}
    \setlength{\parindent}{0pt}

\title{%
    \Huge Abstract Algebra \\
    \large by \\
    \Large Dummit and Foote \\~\\
    \huge Part 1: Group Theory \\
    \LARGE Chapter 2: Subgroups \\
    \Large Section 4: Subgroups Generated by Subsets of a Group
}
\date{2023-07-14}
\author{Michael Saba}

\begin{document}
    \pagenumbering{gobble}
    \maketitle
    \newpage
    \setlength{\parindent}{0pt}
    \pagenumbering{arabic}

    \section*{Exercise 1}
    Proof that if $H \seq G$, then $\langle H \rangle = H$: \\
    We know that $\langle H \rangle$ is the smallest subgroup
    containing the subset $H$.
    Since $H$ does form a subgroup, there can be no smaller
    subgroup that contains all of $H$, so $\langle H \rangle = H$.

    
    \section*{Exercise 2}
    Proof that if $A \sub B$
    then $\langle A \rangle \seq \langle B \rangle$: \\
    We know that
    \[ \langle A \rangle
    = \bigcap_{\substack{A \sub H_i \\ H_i \seq G}} H_i \]
    and
    \[ \langle B \rangle
    = \bigcap_{\substack{B \sub H_i \\ H_i \seq G}} H_i \]
    Since $A \sub B$,
    the every subgroup that contains $B$ also contains $A$.
    Of the subgroups in the second expression,
    all contain $A$, but only some will also contain $B$.
    Consider the subset $S$ that contains both.
    This is the same set as the set of all subgroups that contain
    $A$ as $A \sub B$,
    so $A$ is in every subgroup automatically.
    Since $S$ is a subset and contains less subgroups,
    it is trivial to see that the intersection of these subgroups
    will be a larger set than the intersection of all subgroups
    that contain $A$.
    So $\langle A \rangle \sub \langle B \rangle$. \\
    We already know by exercise 1.2.1.10 that intersection
    of any subgroups is a subgroup of the same group.
    So aside from being a subset of $\langle B \rangle$
    $\langle A \rangle$ also satisfies the other
    subgroup axioms like being non-empty and being closed.
    We conclude that $\langle A \rangle \seq \langle B \rangle$. \\
    To give an example where $A \prosub B$
    ($A \sub B$ but $A \neq B$) and $A = B$,
    we can take $G = D_8$, $A = \{1, r\}$ and $B = \{1, r, r^3\}$,
    where $A \prosub B$
    and $\langle A \rangle = \langle B \rangle = \{1, r, r^2, r^3\}$.

    
    \section*{Exercise 3}
    We define $\langle A, B \rangle$ as $\langle A \cup B \rangle$.
    Proof that if $H$ is abelian subgroup of $G$,
    then $\langle H, Z(G) \rangle$ is abelian: \\
    We know that $\langle H, Z(G) \rangle$
    is a subgroup of $G$ generated by both $H$ and $Z(G)$.
    This is done by taking all possible combinations
    of elements in $H$ and $Z(G)$.
    We already know that all elements in $Z(G)$
    commute with all elements in $G$,
    so all elements in $Z(G)$ commute with elements in $Z(G)$
    and $H$,
    which are subsets of $G$.
    Moreover, all elements in $H$ commutes with themsleves
    as $H$ is abelian.
    So for any product of the union of the two subgroups $H \cup Z(G)$,
    the order of the elements doesn't matter,
    so $\langle H \cup Z(G) \rangle = \langle H, Z(G) \rangle$
    is abelian. \\
    Proof this isn't necessarily the case for $\langle H, C_G(H) \rangle$: \\
    We know that $\langle H, Z(G) \rangle$
    is a subgroup of $G$ generated by both $H$ and $C_G(H)$.
    We already know that elements in $C_G(H)$ commute with elements
    in $H$,
    and we know that elements in $H$ commute among themselves.
    However, there is no guarantee elements in $C_G(H)$
    commute with eachother.
    To complete the proof, we can give a counterexample:
    Take $G = D_8$ and $H = \{1, r^2\}$.
    $H$ is trivially abelian,
    and we have $C_G(H) = \{1, r, r^2, r^3, s\}$
    (all rotations commute, and $s$ commutes with $r^2$
    since for $2n = 8$, $2 = \sfrac{n}{2}$).
    However, products in $\langle H, C_G(H) \rangle$
    generated by multiplying elements in $C_G(H)$ aren't always abelian,
    for instance
    $s \circ r^3 = r^{-3} \circ s = r \circ s \neq r^3 \circ s$.


    \section*{Exercise 4}
    Proof that if $H \seq G$, then $H$ is generated by $H - \{1\}$: \\
    We can generate any non-identity element in $H$
    with the element itself in $H -\{1\}$.
    Since $H$ is a subgroup, it is closed under inverses and the group
    operation.
    So for the identity, we can take any element $x \in H$
    and multiply it by its inverse to get $x \cdot x^{-1} = 1$.
    So any element in $H$ is in $\langle H - \{1\} \rangle$,
    so $H \subset \langle H - \{1\} \rangle$.
    By the minimality of $\langle H - \{1\} \rangle$,
    we know that $\langle H - \{1\} \rangle$ is the smallest
    possible subgroup containing $H - \{1\}$.
    Since $H$ is a subgroup and contains $H - \{1\}$,
    that must mean that $|\langle H - \{1\} \rangle| \seq |H|$,
    so we can conclude that $H = \langle H - \{1\} \rangle$.
    (We could have also argued that as $H$ is closed under the group
    operation,
    any combination of products in $H - \{1\}$ are in $H$,
    as $H -\{1\} \sub H$,
    hence $\langle H - \{1\} \rangle \sub |H|$,
    which means $H = \langle H - \{1\} \rangle$). \\
    If $H - \{1\}$ doesn't contain any other elements,
    then $H = \{1\}$ (since $H$ is a subgroup and can't be empty),
    which makes it the trivial group,
    which in turn is generated by $H - \{1\} = \emptyset$.


    \section*{Exercise 5}
    Proof that for any distinct 2-cycles $\sigma, \tau \in S_3$,
    $\langle \{\sigma, \tau\} \rangle$ 
    (which we write as $\langle \sigma, \tau \rangle$)
    is equal to all of $S_3$: \\
    If $\sigma$ and $\tau$ are $(1\;2)$ and $(1\;3)$,
    then 
    \[(1\;3) \circ (1\;2) = (1\;2\;3) \qquad
    (1\;2) \circ (1\;3) = (1\;3\;2) \]
    \[ (1\;3) \circ (1\;2) \circ (1\;3) = (2\;3) \qquad
    (1\;2)(1\;2) = 1\]
    so we can generate any element in $S_3$ using $(1\;2)$ and $(1\;3)$.
    This means that $S_3 \sub \langle (1\;2), (1\;3) \rangle$.
    Since $S_3$ is a subgroup of itself
    that contains $(1\;2)$ and $(1\;3)$,
    by the minimality of $\langle (1\;2), (1\;3) \rangle$,
    $|\langle (1\;2), (1\;3) \rangle| \seq |S_3|$,
    so we conclude that $S_3 = \langle (1\;2), (1\;3) \rangle$. \\
    We can repeat the same proof for $\sigma, \tau = (1\;2), (2, 3)$
    and $\sigma, \tau = (1\;3), (2, 3)$.
    By computation, both pairs generate the whole group $S_3$,
    and by the same argument we used earlier (minimality),
    $S_3$ is equal to the generated group.


    \section*{Exercise 6}
    Proof that the subgroup of $S_4$,
    $\langle (1\;2), (1\;2)(3\;4) \rangle$,
    is non-cyclic and has order 4: \\
    We can combine the elements in the following ways:
    \[ (1\;2) \circ (1\;2)(3\;4) = (3\;4) \qquad
    (1\;2)(3\;4) \circ (1\;2) = (3\;4) \]
    Since the elements are all disjoint 2-cycles,
    the inverses of the elements are the elements themselves,
    and any odd power is also the element itself.
    On the other hand, any even power is the identity.
    Since these are all the elements we can obtain
    through any combination of the two given generators,
    we conclude that $\langle (1\;2), (1\;2)(3\;4) \rangle
    = \{1, (1\;2), (1\;2)(3\;4), (3\;4)\}$,
    which trivially has order 4, and is non cyclic
    since none of the elements individually can generate the group.


    \section*{Exercise 7}
    Proof that the subgroup of $S_4$,
    $\langle (1\;2), (1\;3)(2\;4) \rangle$,
    is isomorphic to $D_8$: \\
    We can combine the elements in the following ways:
    \[ (1\;2) \circ (1\;3)(2\;4) = (1\;3\;2\;4) \qquad
    (1\;3)(2\;4) \circ (1\;2) = (1\;4\;2\;3) \]
    \[ (1\;2) \circ (1\;3)(2\;4) \circ (1\;2) = (1\;4)(2\;3) \qquad
    (1\;3)(2\;4) \circ (1\;2) \circ (1\;3)(2\;4) = (3\;4) \]
    \[ (1\;3\;2\;4)^2 = (1\;2)(3\;4) \qquad
    (1\;4\;2\;3)^2 = (1\;2)(3\;4) \]
    \[ (1\;3\;2\;4)^3 = (1\;4)(2\;3) \qquad
    (1\;4\;2\;3)^3 = (1\;3\;2\;4) \]
    Since the gnerators are all disjoint 2-cycles,
    the inverses of the elements are the elements themselves,
    and any odd power is also the element itself.
    On the other hand, any even power is the identity.
    As for the the 2 4-cycles we obtained,
    they become the identiy when raised to the 4th power,
    so we only have to check their values up to a power of 3.
    Any other combination of the 4-cycles
    is a combination of the two cycles we already checked,
    or one that produces and even or odd power
    which we already resolved.
    Since these are all the elements we can obtain
    through any combination of the two given generators,
    we conclude that 
    \begin{align*}
        \langle (1\;2), (1\;2)(3\;4) \rangle
        = \{1, (1\;2), (1\;4)(2\;3), (1\;3)(2\;4), (3\;4), (1\;3\;2\;4), \\
        (1\;4\;2\;3), (1\;2)(3\;4) \}
    \end{align*}
    We have 
    \[D_{8} = \langle r, s \mid s^2 = r^4 = 1, rs = sr^{-1} \rangle\]
    In order to show that $\langle (1\;2), (1\;2)(3\;4) \rangle$
    is homomorphic to $D_8$,
    we must show that there exists a subset of the former group
    where, by mapping each generator of $D_8$ to an element in the subset,
    we can satisfy the relations in $D_8$. \\
    If we map $r$ to $(1\;3\;2\;4)$
    and $s$ to $(1\;2)$,
    we can see that $(1\;2)^2 = (1\;3\;2\;4)^4 = 1$
    since they are a 2-cycle and 4-cycle.
    Furthermore, we have that
    \[ (1\;3\;2\;4) \circ (1\;2) = (1\;2) \circ (1\;3)(2\;4) \circ (1\;2)
    = (1\;2) \circ (1\;4\;2\;3)
    = (1\;2) \circ (1\;3\;2\;4)^{-1} \]
    So $\langle (1\;2), (1\;2)(3\;4) \rangle$ is homomorphic to $D_8$.
    We trivially know that $(1\;2)$ and $(1\;2)(3\;4)$
    generate $\langle (1\;2), (1\;2)(3\;4) \rangle$.
    And we can write $(1\;3)(2\;4)$
    as $(1\;2)^{-1} \circ (1\;3\;2\;4) = (1\;2) \circ (1\;3\;2\;4)$,
    so that $(1\;2)$ and $(1\;3\;2\;4)$
    also generate $\langle (1\;2), (1\;2)(3\;4) \rangle$.
    This makes the mapping surjective as well as a homomorphism.
    Since $|\langle (1\;2), (1\;2)(3\;4) \rangle| = |D_8| = 8$,
    the mapping has to be a bijection,
    so $\langle (1\;2), (1\;2)(3\;4) \rangle \cong D_8$.


    \section*{Exercise 8}
    Proof that $S_4 = \langle (1\;2\;3\;4), (1\;2\;4\;3) \rangle$: \\
    We know that 
    \[ (1\;2\;3\;4)^2 = (1\;3\;2\;4) \qquad
    (1\;2\;3\;4)^3 = (1\;4\;3\;2) \]
    \[ (1\;2\;4\;3)^2 = (1\;4\;2\;3) \qquad
    (1\;2\;4\;3)^3 = (1\;3\;4\;2) \]
    And
    \[ (1\;2\;4\;3) \circ (1\;2\;3\;4) = (1\;4\;2) \qquad
    (1\;2\;3\;4) \circ (1\;2\;4\;3) = (1\;3\;2) \]
    \[ (1\;2\;4\;3) \circ (1\;2\;3\;4) \circ (1\;2\;4\;3) = (3\;4) \]
    \[ (1\;2\;3\;4)^2 \circ (1\;2\;4\;3) = (1\;4) \qquad
    (1\;2\;4\;3)^2 \circ (1\;2\;3\;4) = (1\;3) \]
    \[ (1\;2\;3\;4)^3 \circ (1\;2\;4\;3) = (2\;3\;4) \qquad
    (1\;2\;4\;3)^3 \circ (1\;2\;3\;4) = (2\;4\;3) \]
    Since we can generate 11 new elements using the two generators,
    that must mean that
    $|\langle (1\;2\;3\;4), (1\;2\;4\;3) \rangle| \beq 13$.
    By Lagrange's Theorem,
    since $\langle (1\;2\;3\;4), (1\;2\;4\;3) \rangle \seq S_4$,
    and $|S_4| = 4! = 24$,
    $|\langle (1\;2\;3\;4), (1\;2\;4\;3) \rangle| \mid 14$.
    The only number larger or equal to 13 that divides 24 is 24.
    So $|\langle (1\;2\;3\;4), (1\;2\;4\;3) \rangle| = 24$,
    and since $\langle (1\;2\;3\;4), (1\;2\;4\;3) \rangle \seq S_4$,
    we have $\langle (1\;2\;3\;4), (1\;2\;4\;3) \rangle = S_4$.

    
    \section*{Exercise 9}
    Proof that $SL_2(\F_3)$ is the subgourp of $SL_2(\F_3)$
    generated by $\begin{bmatrix} 1 & 1 \\ 0 & 1 \end{bmatrix}$
    and $\begin{bmatrix} 1 & 0 \\ 1 & 1 \end{bmatrix}$
    where $F_3 = \Z/3\Z$
    and $SL_2(\F_3)$ is the group of matrices of determinant 1
    as explained in exercise 1.2.1.9
    (we can as assume that the subgroup has order 24): \\
    We already know that
    \[\det\left(\begin{bmatrix} 1 & 1 \\ 0 & 1 \end{bmatrix}\right)
    = \det\left(\begin{bmatrix} 1 & 0 \\ 1 & 1 \end{bmatrix}\right)
    = 1\]
    So both are in $SL_2(\F_3)$.
    Moreover, we know from exercise 1.2.1.9 that 
    $SL_2(\F_3)$ is a subgroup,
    which makes it closed under all multiplications and inverses.
    So every element generated
    by $\begin{bmatrix} 1 & 1 \\ 0 & 1 \end{bmatrix}$
    and $\begin{bmatrix} 1 & 0 \\ 1 & 1 \end{bmatrix}$
    is in $SL_2(\F_3)$,
    so
    \[ \left\langle \begin{bmatrix} 1 & 1 \\ 0 & 1 \end{bmatrix},
    \begin{bmatrix} 1 & 0 \\ 1 & 1 \end{bmatrix} \right\rangle
    \sub SL_2(\F_3) \]
    Since the generated set is by definition a subgroup,
    we also have that 
    \[ \left\langle \begin{bmatrix} 1 & 1 \\ 0 & 1 \end{bmatrix},
    \begin{bmatrix} 1 & 0 \\ 1 & 1 \end{bmatrix} \right\rangle
    \seq SL_2(\F_3) \]
    And we also know that
    \[ \begin{bmatrix} 1 & 1 \\
        0 & 1 \end{bmatrix}^2
        = \begin{bmatrix} 1 & 2 \\
        0 & 1 \end{bmatrix}  \]
    \[ \begin{bmatrix} 1 & 0 \\
        1 & 1 \end{bmatrix}^2
        = \begin{bmatrix} 1 & 0 \\
        2 & 1 \end{bmatrix}  \]
    \[ \begin{bmatrix} 1 & 1 \\
        0 & 1 \end{bmatrix}
        \begin{bmatrix} 1 & 0 \\
            1 & 1 \end{bmatrix}
        = \begin{bmatrix} 2 & 1 \\
        1 & 1 \end{bmatrix}  \]
    \[ \left(\begin{bmatrix} 1 & 1 \\
        0 & 1 \end{bmatrix}
        \begin{bmatrix} 1 & 0 \\
            1 & 1 \end{bmatrix} \right)^2
        = \begin{bmatrix} 2 & 0 \\
        0 & 2 \end{bmatrix}  \]
    \[ \left(\begin{bmatrix} 1 & 1 \\
        0 & 1 \end{bmatrix}
        \begin{bmatrix} 1 & 0 \\
            1 & 1 \end{bmatrix} \right)^3
        = \begin{bmatrix} 1 & 2 \\
        2 & 2 \end{bmatrix}  \]
    \[ \begin{bmatrix} 1 & 1 \\
        0 & 1 \end{bmatrix}^2
        \begin{bmatrix} 1 & 0 \\
            1 & 1 \end{bmatrix}^2
        = \begin{bmatrix} 2 & 2 \\
        2 & 1 \end{bmatrix}  \]
    \[ \begin{bmatrix} 1 & 1 \\
        0 & 1 \end{bmatrix}
        \begin{bmatrix} 1 & 0 \\
            1 & 1 \end{bmatrix}
        \begin{bmatrix} 1 & 1 \\
            0 & 1 \end{bmatrix}
        = \begin{bmatrix} 2 & 0 \\
        1 & 2 \end{bmatrix}  \]
    \[ \begin{bmatrix} 1 & 0 \\
        1 & 1 \end{bmatrix}
        \begin{bmatrix} 1 & 1 \\
            0 & 1 \end{bmatrix}
        \begin{bmatrix} 1 & 0 \\
            1 & 1 \end{bmatrix}
        = \begin{bmatrix} 2 & 1 \\
        0 & 2 \end{bmatrix}  \]
    \[ \begin{bmatrix} 1 & 0 \\
        1 & 1 \end{bmatrix}
        \begin{bmatrix} 1 & 1 \\
            0 & 1 \end{bmatrix}
        = \begin{bmatrix} 1 & 1 \\
        1 & 2 \end{bmatrix}  \]
    \[ \begin{bmatrix} 1 & 1 \\
        0 & 1 \end{bmatrix}^2
        \begin{bmatrix} 1 & 0 \\
            1 & 1 \end{bmatrix}
        = \begin{bmatrix} 0 & 2 \\
        1 & 1 \end{bmatrix}  \]
    \[ \begin{bmatrix} 1 & 1 \\
        0 & 1 \end{bmatrix}
        \begin{bmatrix} 1 & 1 \\
        0 & 1 \end{bmatrix}^{-1}
        = \begin{bmatrix} 1 & 0 \\
        0 & 1 \end{bmatrix} \]
    So we have at least 13 elements in
    $\left\langle \begin{bmatrix} 1 & 1 \\ 0 & 1 \end{bmatrix},
    \begin{bmatrix} 1 & 0 \\ 1 & 1 \end{bmatrix} \right\rangle$,
    and since it's a subgroup of $SL_2(\F_3)$,
    then by Lagrange's Theorem,
    \[ \left|\left\langle \begin{bmatrix} 1 & 1 \\ 0 & 1 \end{bmatrix},
    \begin{bmatrix} 1 & 0 \\ 1 & 1 \end{bmatrix} \right\rangle \right| 
    \mid |SL_2(\F_3)| \]
    so we must at least have 24 elements in the generated subgroup
    (smallest number larger or equal to 13 that divides 24).
    By the minimality of the generated subgroup,
    we know that 
    \[ \left\langle \begin{bmatrix} 1 & 1 \\ 0 & 1 \end{bmatrix},
    \begin{bmatrix} 1 & 0 \\ 1 & 1 \end{bmatrix} \right\rangle
    \sub SL_2(\F_3) \]
    so
    \[\left| \left\langle \begin{bmatrix} 1 & 1 \\ 0 & 1 \end{bmatrix},
    \begin{bmatrix} 1 & 0 \\ 1 & 1 \end{bmatrix} \right\rangle \right|
    \seq 24\]
    which means that
    \[\left\langle \begin{bmatrix} 1 & 1 \\ 0 & 1 \end{bmatrix},
    \begin{bmatrix} 1 & 0 \\ 1 & 1 \end{bmatrix} \right\rangle = 24\]
    so
    \[\left\langle \begin{bmatrix} 1 & 1 \\ 0 & 1 \end{bmatrix},
    \begin{bmatrix} 1 & 0 \\ 1 & 1 \end{bmatrix} \right\rangle
    = SL_2(\F_3)\]


    \section*{Exercise 10}
    Proof that the subgroup of $SL_2(\F_3)$
    (where $\F_3 = \Z/3\Z$)
    generated by $\begin{bmatrix} 0 & -1 \\ 1 & 0 \end{bmatrix}$
    and $\begin{bmatrix} 1 & 1 \\ 1 & -1 \end{bmatrix}$
    is isomorphic to $Q_8$: \\
    We know that 
    \[ \begin{bmatrix} 0 & -1 \\ 1 & 0 \end{bmatrix}, \quad
        \begin{bmatrix} 1 & 1 \\ 1 & -1 \end{bmatrix}, \quad
        \begin{bmatrix} 0 & -1 \\ 1 & 0 \end{bmatrix}^2
        = \begin{bmatrix} -1 & 0 \\ 0 & -1 \end{bmatrix}
        = \begin{bmatrix} 2 & 0 \\ 0 & 2 \end{bmatrix}, \] 
    \[ \begin{bmatrix} 1 & 1 \\ 1 & -1 \end{bmatrix}^2
    = \begin{bmatrix} 2 & 0 \\ 0 & 2 \end{bmatrix}, \quad
    \begin{bmatrix} 1 & 1 \\ 1 & -1 \end{bmatrix}^3
    = \begin{bmatrix} 2 & 0 \\ 0 & 2 \end{bmatrix}, \]
    \[ \begin{bmatrix} 0 & -1 \\ 1 & 0 \end{bmatrix}^3
    = \begin{bmatrix} 0 & 1 \\ -1 & 0 \end{bmatrix}, \quad
    \begin{bmatrix} 1 & 1 \\ 1 & -1 \end{bmatrix}^4
    = \begin{bmatrix} 1 & 0 \\ 0 & 1 \end{bmatrix}, \] 
    \[ \begin{bmatrix} 0 & -1 \\ 1 & 0 \end{bmatrix}
    \begin{bmatrix} 1 & 1 \\ 1 & -1 \end{bmatrix}
    = \begin{bmatrix} -1 & 1 \\ 1 & 1 \end{bmatrix}, \quad
    \begin{bmatrix} 1 & 1 \\ 1 & -1 \end{bmatrix}
    \begin{bmatrix} 0 & -1 \\ 1 & 0 \end{bmatrix}
    = \begin{bmatrix} 1 & -1 \\ -1 & -1 \end{bmatrix} \]
    Are all in the generated group,
    and any ther combination of the generated elements
    produces one of those matrices,
    so 
    \[ \left| \left\langle \begin{bmatrix} 1 & -1 \\ 1 & 0 \end{bmatrix},
    \begin{bmatrix} 1 & 1 \\ 1 & -1 \end{bmatrix} \right\rangle \right|
    = 8 \]
    We know from exercise 1.1.5.3 that
    \[ Q_8 = \langle i, j \mid i^2 = j^2, i^4 = 1, i = jij \rangle \]
    So if we map $i$ to $\begin{bmatrix} 0 & -1 \\ 1 & 0 \end{bmatrix}$,
    and $j$ to $\begin{bmatrix} 1 & 1 \\ 1 & -1 \end{bmatrix}$,
    we find that
    \[ \begin{bmatrix} 0 & -1 \\ 1 & 0 \end{bmatrix}^2
    = \begin{bmatrix} 1 & 1 \\ 1 & -1 \end{bmatrix}^2
    = \begin{bmatrix} 2 & 0 \\ 0 & 2 \end{bmatrix}, \quad
    \begin{bmatrix} 0 & -1 \\ 1 & 0 \end{bmatrix}^4
    = \begin{bmatrix} 1 & 0 \\ 0 & 1 \end{bmatrix} = I_2 \]
    where $I_2$ is the identity, and
    \[ \begin{bmatrix} 1 & 1 \\ 1 & -1 \end{bmatrix}
    \begin{bmatrix} 0 & -1 \\ 1 & 0 \end{bmatrix}
    \begin{bmatrix} 1 & 1 \\ 1 & -1 \end{bmatrix}
    = \begin{bmatrix} 0 & 2 \\ -2 & 0 \end{bmatrix}
    = \begin{bmatrix} 0 & -1 \\ 1 & 0 \end{bmatrix} \]
    So as $Q_8$'s relations hold,
    the two groups are homomorphic.
    Furthermore,
    since $\begin{bmatrix} 0 & -1 \\ 1 & 0 \end{bmatrix}$,
    and $\begin{bmatrix} 1 & 1 \\ 1 & -1 \end{bmatrix}$
    are by definition generators,
    the mapping is surjective. \\
    Finally, since
    \[ \left| \left\langle \begin{bmatrix} 1 & -1 \\ 1 & 0 \end{bmatrix},
    \begin{bmatrix} 1 & 1 \\ 1 & -1 \end{bmatrix} \right\rangle \right|
    = |Q_8| = 8 \]
    The mapping becomes a bijection,
    which means that it is an isomorphism.


    \section*{Exercise 11}
    Proof that $S_4 \ncong SL_2(\F_3)$: \\
    We know that $S_4$'s largest cycle is a 4-cycle,
    so at most we have elements of order 4.
    However, in $SL_2(\F_3)$,
    we have $ \begin{bmatrix} 0 & -1 \\ 1 & 1 \end{bmatrix}$,
    which has order 6.
    According to exercise 1.1.6.2,
    this implies the two groups can't be isomorphic,
    as there should be the same amount of elements of the same
    order in both groups.


    \section*{Exercise 12}
    Proof that the subgroup of upper triangles in $GL_3(\F_2)$
    (in exercise 1.2.1.16) is isomorphic to $D_8$: \\
    We know that since every element in $GL_3(\F_2)$
    is of the form
    \[ \begin{bmatrix} a & b & c \\
        0 & e & f \\
        0 & 0 & g \end{bmatrix} \]
    We know that all of the varibales can either be 1 or 2.
    However, the diagonal variables can't be 0,
    since the determinant would be 0
    (and $GL_3(\F_2)$ contains matrices with a non-zero determinant).
    So we have 2 options to choose from for $b$, $c$, and $f$,
    meaning there are $2^3 = 8$ matrices in this group. \\
    We now note that 
    \[ D_{8} = \langle r, s \mid s^2 = r^4 = 1, rs = sr^{-1} \rangle \]
    And with that, we define a map
    $\varphi: D_8 \to GL_3(\F_2)$
    where
    \[ \varphi(r) = \begin{bmatrix} 1 & 1 & 1 \\
        0 & 1 & 1 \\
        0 & 0 & 1 \end{bmatrix} \qquad
        \text{ and } \qquad
    \varphi(s) \begin{bmatrix} 1 & 1 & 1 \\
        0 & 1 & 0 \\
        0 & 0 & 1 \end{bmatrix} \]
    We know that
    \[ \begin{bmatrix} 1 & 1 & 1 \\
        0 & 1 & 1 \\
        0 & 0 & 1 \end{bmatrix}^4
    = \begin{bmatrix} 1 & 1 & 1 \\
        0 & 1 & 1 \\
        0 & 0 & 1 \end{bmatrix}^2 = I_3 \] 
    and
    \[ \begin{bmatrix} 1 & 1 & 1 \\
        0 & 1 & 1 \\
        0 & 0 & 1 \end{bmatrix}
    \begin{bmatrix} 1 & 1 & 1 \\
        0 & 1 & 0 \\
        0 & 0 & 1 \end{bmatrix}
    = \begin{bmatrix} 1 & 0 & 0 \\
        0 & 1 & 1 \\
        0 & 0 & 1 \end{bmatrix}
    = \begin{bmatrix} 1 & 1 & 1 \\
        0 & 1 & 0 \\
        0 & 0 & 1 \end{bmatrix}
    \begin{bmatrix} 1 & 1 & 0 \\
        0 & 1 & 1 \\
        0 & 0 & 1 \end{bmatrix} \] 
    where
    \[ \begin{bmatrix} 1 & 1 & 1 \\
        0 & 1 & 1 \\
        0 & 0 & 1 \end{bmatrix}^{-1}
    = \begin{bmatrix} 1 & 1 & 0 \\
        0 & 1 & 1 \\
        0 & 0 & 1 \end{bmatrix} \] 
    Which means $\varphi$ is a homomorphism. \\
    Furthermore, we can show explicitely that this mapping is injective:
    \[ \varphi(r) = \begin{bmatrix} 1 & 1 & 1 \\
        0 & 1 & 1 \\
        0 & 0 & 1 \end{bmatrix}, \quad
    \varphi(s) \begin{bmatrix} 1 & 1 & 1 \\
        0 & 1 & 0 \\
        0 & 0 & 1 \end{bmatrix} \]
    \[ \varphi(r^2) = \begin{bmatrix} 1 & 0 & 1 \\
        0 & 1 & 0 \\
        0 & 0 & 1 \end{bmatrix}, \quad
    \varphi(r^3) \begin{bmatrix} 1 & 1 & 0 \\
        0 & 1 & 1 \\
        0 & 0 & 1 \end{bmatrix} \]
    \[ \varphi(sr) = \begin{bmatrix} 1 & 0 & 1 \\
        0 & 1 & 1 \\
        0 & 0 & 1 \end{bmatrix}, \quad
    \varphi(sr^2) \begin{bmatrix} 1 & 1 & 0 \\
        0 & 1 & 0 \\
        0 & 0 & 1 \end{bmatrix} \]
    \[ \varphi(sr^3) = \begin{bmatrix} 1 & 0 & 0 \\
        0 & 1 & 1 \\
        0 & 0 & 1 \end{bmatrix}, \quad
    \varphi(1) \begin{bmatrix} 1 & 0 & 0 \\
        0 & 1 & 0 \\
        0 & 0 & 1 \end{bmatrix} = I_3 \]
    Since $|GL_3(\F_2)| = 8$, $\varphi$ is also surjective,
    making it a bijection,
    and by extension, an isomorphism.


    \section*{Exercise 13}
    Proof that the Multiplicative group of positive rationals
    $(\Q^+, \times)$ is generated
    by $S = \{\sfrac{1}{p} \mid p \text{ is prime}\}$: \\
    We can write any rational number in $\Q^+$ as $\sfrac{a}{b}$,
    where $a, b \in \Z^+$.
    So by the fundemental theorem of arithmetic,
    \[ \dfrac{a}{b} = \dfrac{p_1p_2 \dots p_n}{q_1q_2 \dots q_m} \]
    where all $p_i$ and $q_i$ are (not necessarily distinct) primes.
    So we can write it as
    \[ \dfrac{a}{b}
    = \dfrac{p_1}{1}\dfrac{p_2}{1} \dots \dfrac{p_n}{1} \times
    \dfrac{1}{q_1}\dfrac{1}{q_2} \dots \dfrac{1}{q_m}
    = \left(\dfrac{1}{p_1}\right)^{-1} \left(\dfrac{1}{p_2}\right)^{-1}
    \dots \left(\dfrac{1}{p_n}\right)^{-1}
    \times \dfrac{1}{q_1}\dfrac{1}{q_2} \dots \dfrac{1}{q_m} \]
    Since all $p_i$ and $q_i$ are primes,
    and since by the definition of a generated subgroup,
    $\langle S \rangle$ is closed under all combinations of
    inverses and powers of elements in $S$,
    this means that any positive rational number belongs
    to $\langle S \rangle$,
    so $\Q^+ \sub S$. \\
    However, we also know that any way we combine
    the inverse and powers of elements in $S$,
    we alwasy end up with a fraction containing 2 positive integers,
    which is by definiton a positive rational.
    So $\langle S \rangle \sub \Q^+$,
    which tells us $\langle S \rangle = \Q^+$.


    \section*{Exercise 14 $***$}
    A \textit{finitely generated group} is a group that can be generated
    by a finite subset of elements
    (if there exists a finite subset $H$ such that $G = \langle H \rangle$).
    \begin{enumerate}[label=\textbf{\alph*.}]
        \item 
            Proof that all finite groups are finitly generated: \\
            It is trivial to see that for any group $G$,
            $G = \langle G \rangle$.
            Obviously, every element in $G$ generated $G$,
            so $G \sub \langle G \rangle$.
            Moreover, $G$ is closed under inverse and multiplication
            so any combination of its elements is in $G$,
            which means that $\langle G \rangle \sub G$,
            so $G = \langle G \rangle$.
            If we assume $G$ is finite,
            and $G$ generates $G$,
            then $G$ is finitely generated.
        \item 
            Proof $(\Z, +)$ is finitely generated: \\
            We know from exercise 1.1.2.14 that 1 alone can generate $\Z$.
            So $\Z \sub \langle 1 \rangle$.
            Since the addition and subtraction of 1 over and over again
            is an integer,
            $\langle 1 \rangle \sub \Z$.
            So $\Z = \langle 1 \rangle$.
        \item
            Proof that every finitely generated subgroup of $(\Q, +)$
            is cyclic: \\
            Assume that $H$ is a finitely generated subgroup of $\Q$.
            Then for some $S \sub H$,
            $\langle S \rangle = H$.
            Now consider the product of all denominators of rationals in $S$,
            which we will denote as $k$.
            For any rational in $H$,
            we know that 
            \[ \dfrac{a}{b} = \dfrac{c}{k} = c \cdot \dfrac{1}{k} \]
            where $c$ is the product of all numerators in $S$
            except for $a$.
            This means that all rationals in $H$
            can be generated by $\sfrac{1}{k}$.
            So $H \sub \langle \sfrac{1}{k} \rangle$.
            Since $H$ is already a subgroup by assumption,
            that means it satisfies all other subgroup axioms,
            so $H \seq \langle \sfrac{1}{k} \rangle$.
            We know that $\langle \sfrac{1}{k} \rangle$ is cyclic
            since it has 1 generator,
            and we know that all subgroups of a cyclic group are cyclic,
            so $H$ must be cyclic. 
        \item
            Proof that $(\Q, +)$ is not finitely generated: \\
            We know from the previous exercise that all subgroups
            of $\Q$ that are finitely generated are cyclic.
            Since $\Q \seq \Q$, this applies to the group itslef.
            However, we know that $\Q$ isn't cyclic.
            This is because no rational element $\sfrac{a}{b}$
            alone can generate all of $\Q$.
            For instance,
            we can't generate $\sfrac{2a}{3b}$
            as we can only use inverses and multiples of $\sfrac{a}{b}$,
            meaning that the entire fraction must be multiplied
            by a positive or negative integer,
            and $\sfrac{2}{3}$ isn't one.
            So contrapositively,
            this means that $\Q$ isn't finitely generated.
    \end{enumerate}


    \section*{Exercise 15}
    We have $H = \left\{ \dfrac{a}{2^n} \mid a, n \in \Z \right\}$
    is a subgroup of $\Q$.
    We can already see it's nonempty and a subset of $Q$.
    We also know that
    \[ \dfrac{a}{2^n} + \dfrac{b}{2^m} = \dfrac{a2^m + b2^n}{2^n2^m}
    = \dfrac{a2^m + b2^n}{2^{n + m}} \]
    and
    \[ -\dfrac{a}{2^n} = \dfrac{-a}{2^n} \]
    so $H$ is also closed under inverses and addition,
    making it a subgroup.
    It is also a proper subgroup, since $H \prosub \Q$.
    We can also see that it is not cyclic.
    If it were, then some $\sfrac{a}{2^n}$ could generate all of $H$.
    However, we can't generate
    $\sfrac{3a}{2^{n+1}} = \sfrac{3}{2} \times \sfrac{a}{2^n}$,
    as it would require 3 times the numerator
    and 2 times the denominator,
    which isn't possible
    (as we can only use inverses and multiples of $\sfrac{a}{2^n}$,
    meaning that the entire fraction must be multiplied
    by a positive or negative integer,
    and $\sfrac{3}{2}$ isn't one).
    So $H$ can't be cyclic.


    \section*{Exercise 16 $***$}
    We call $M$ a \textit{maximal subgroup} of $G$
    if $M < G$ ($M \seq G$ and $M \neq G$)
    and the only subgroups of $G$ containing $M$
    are $M$ and $G$. 
    \begin{enumerate}[label=\textbf{\alph*.}]
        \item
            Proof that if $H < G$ (proper subgroup of $G$)
            where $G$ is finite,
            then there exists a maximal subgroup of $G$
            that contains $H$: \\
            If $H < G$, and $H$ is a maximal subgroup of $G$,
            then $H$ is the maximal subgroup containing $H$. \\
            Now, if $H$ is not a maximal subgroup of $G$,
            that means that there exists a subgroup
            $M$ of $G$ other than $G$ that contains $H$
            (by the definition of a maximal subgroup).
            That tells us that $M < G$, 
            so if $M$ is a maximal subgroup of $G$,
            then $M$ is the maximal subgroup containing $H$. \\
            If $M$ is not maximal, we can repeat the same argument
            for another proper subgroup,
            all of which contain the previous,
            and by extension, $H$.
            Since $G$ is finite, this process will eventually
            end, and lead us to a maximal subgroup containing $H$.
        \item
            Proof that $H$, the subgroup of all rotations in $D_{2n}$
            is a maximal subgroup: \\
            We know that $H = \{1, r, r^2 \dots r^{n-1}\}$
            contains $n$ elements.
            By Lagrange's Theorem,
            we know that $|H| \mid |D_{2n}|$,
            so if a subgroup $M$ contains $H$,
            then $|M| = n$ or $|M| = |D_{2n}| = 2n$.
            However, since $H \sub M \sub D_{2n}$,
            that means that either $M = H$ or $M = D_{2n}$.
            So aside from $D_{2n}$ itself,
            the only subgroup that contains $H$ is $H$ itself,
            which by definition means it's a maximal subgroup.
        \item
            Proof that if $G = \langle x \rangle$
            is a cyclic group of order $n \beq 1$,
            then its subgroup $H$ is maximal if and only if
            $H = \langle x^p \rangle$ for some prime $p$
            such that $p \mid n$: \\
            If $H$ is a maximal subgroup of $\langle x \rangle$,
            then $H \neq \langle x \rangle$.
            We know that $\langle x^m \rangle = \langle x \rangle$
            if $\gcd(n, m) = 1$,
            so that must mean that if $H = \langle x^p \rangle$,
            then $\gcd(p, n) > 1$. \\
            Now assume that $p$ is not prime.
            That means that we can write $p$ as $c \cdot k$
            where $1 < c, k < p$ and $k$ is prime.
            Since $c \mid p$ by its definition,
            then $\gcd(c, n) > 1$ since $\gcd(p, n) > 1$.
            This means that $\langle x^c \rangle \neq \langle x \rangle$,
            which in turn tells us that 
            $\langle x^c \rangle < \langle x \rangle$. \\
            Moreover, since $1 < c, p < n$,
            $p \not\equiv c \mod n$,
            so $\langle x^p \rangle \neq \langle x^c \rangle$.
            We also have
            \[ |\langle x^c \rangle|
            = \dfrac{|\langle x \rangle|}{\gcd(c, n)} 
            = \dfrac{n}{\gcd(c, n)} \]
            and 
            \[ |\langle x^p \rangle|
            = \dfrac{|\langle x \rangle|}{\gcd(p, n)} 
            = \dfrac{n}{\gcd(p, n)} \]
            Since $\langle x^p \rangle \neq \langle x^c \rangle$,
            we have $|\langle x^p \rangle| \neq |\langle x^c \rangle|$,
            so $\gcd(p, n) \neq \gcd(c, n)$.
            However, since $p = ck$, and $k$ is prime by assumption,
            that must mean that $k$ divides some factors of $n$
            that aren't part of $\gcd(c, n)$.
            So this means that $k \mid \sfrac{n}{\gcd(c, n)}$,
            which means that $k \mid |\langle x^c \rangle|$.
            We know that
            $\langle x^p \rangle = \langle (x^c)^k \rangle
            \seq \langle x^c \rangle$
            if $k \mid |\langle x^c \rangle|$.
            So $\langle x^p \rangle \seq \langle x^c \rangle$,
            and since we showed they aren't equal,
            $\langle x^p \rangle < \langle x^c \rangle$. \\
            So we have $H < \langle x^p \rangle < G$,
            which contradicts $H$ being maximal,
            so $p$ must be prime. \\
            Finally, since $p$ is prime and $\gcd(p, n) > 1$,
            $p \mid n$.
            So we conclude that $p$ is a prime that divides $n$. \\
            Conversely, if $H = \langle x^p \rangle$
            where $p$ is prime and $p \mid n$,
            Then $\gcd(p, n) = p$,
            so $\langle x^p \rangle \seq \langle x^c \rangle$,
            and $|\langle x^p \rangle| < n$,
            so $\langle x^p \rangle < \langle x^c \rangle$.
            Moreover, if some other subgroup $\langle x^c \rangle$
            contains  $\langle x^p \rangle$,
            then that must mean that $p$ can be written
            as $k \cdot c$ where $k \mid |\langle x^c \rangle|$.
            However, since $p$ is prime,
            it can only be factored as $p \cdot 1$.
            So we only have $1 \mid |\langle x^p \rangle|$
            and $p \mid |\langle x^1 \rangle|$,
            which are both trivially true
            (we already showed $p \mid n$, and $|\langle x \rangle| = n$).
            This means that the only groups that contain $H$,
            for $c = 1$ and $c = p$, are $H$ and $G$,
            which by definition means that $H$ is a maximal subgroup of $G$.
    \end{enumerate}


    \section*{Exercise 17 $***$}
    Let $G$ be a finitely generated group
    where $G = \langle g_1, g_2 \dots g_n \rangle$,
    and let $\scal$ be the set of all proper subgroups of $G$
    (all by $G$ itself).
    Then $\scal$ is partially ordered by inclusion
    (we can create several chains $H_1 \seq H_2 \seq \dots \seq H_k$
    in $\scal$, but not necessarily a chain that includes all subgroups).
    Let $\ccal$ be such a chain in $\scal$.
    \begin{enumerate}[label=\textbf{\alph*.}]
        \item
            Proof that the union $H$ of all subgroups in $\ccal$,
            is a subgroup of $G$: \\
            We already know that $H_1, H_2 \dots H_k \in \ccal$
            are subgroups of $G$.
            We also know that they are ordered by inclusion
            since $\ccal$ is a chain.
            So $H_1, H_2 \dots H_{k-1} \seq H_k$.
            So $H_1, H_2 \dots H_{k-1} \sub H_k$,
            which trivially means that
            \[ H = \bigcup_{i = 1}^k H_i = H_k \]
            So $H = H_k \seq G$. 
        \item
            Proof that the union $H$ of all subgroups in $\ccal$,
            is a proper subgroup of $G$: \\
            We already showed that 
            \[ H = \bigcup_{i = 1}^k H_i = H_k \]
            Since $H \in \ccal$, $H \in \scal$,
            so $H_k$ is a proper subgroup by $\scal$'s definition,
            which means that $H < G$.
        \item
            \textit{Zorn's Lemma} states that a partially ordered
            set made up of ordered subsets must have a maximal element.
            Proof that $G$ has a maximal subgroup: \\
            We know $\scal$ is partially ordered,
            and that it is made up of chains of ordered elements.
            So $\scal$ has a maximal element 
            of which every other proper subgroup in $\scal$
            is a subgroup.
            By definition, this makes it a maximal subgroup.
            So any finitely generated group $G$ has a maximal subgroup.
    \end{enumerate}


    \section*{Exercise 18 $***$}
    For a prime $p$,
    the \textit{multiplicative group of all $p$-power roots of unity of $C$}
    $Z$ is the set $\{z \in \C \mid z^{p^n} = 1, n \in \Z^+\}$.
    For each fixed $k \in \Z^+$, let $H_k = \{z \in \C \mid z^{p^n} = 1\}$
    (the \textit{group of $p^k$th root of unity}).
    \begin{enumerate}[label=\textbf{\alph*.}]
        \item
            Proof that $H_k \seq H_m$ if and only if $k \seq m$: \\
            If $k \seq m$, then $m = k + c$ for some $c \in \Z^+$.
            So $\forall h \in H_k$,
            $h^{p^m} = h^{p^kp^c} = (h^{p^k})^{p^c} = 1^{p^c} = 1$.
            So $H_k \sub H_m$.
            As $H_k$ is a subgroup of $G$, it already satisfies all other 
            subgroup axioms.
            So $H_k \seq H_m$. \\
            Conversely, if $H_k \seq H_m$,
            then $H_k \sub H_m$.
            So $\forall h \in H_k$, $h \in H_m$.
            This means that if $h^{p^k} = 1$, then $h^{p^m} = 1$.
            If $m < k$, then $p^m < p^k$.
            However, $g = e^{\sfrac{2i\pi}{p^k}}$ is in $H_k$
            and has order $p^k$
            (smallest positive power to which the element is 1).
            So for $p^m < p^k$, $g^{p^m} \neq 1$,
            which contradicts our assumption that all elements in $H_k$
            are in $H_m$,
            so $k \seq m$. 
        \item
            Proof that for any $k$, $H_k$ is cyclic: \\
            We know that the set of all $p^k$th roots of $1$ in $\C$
            is the set
            $\{ e^{\sfrac{2ti\pi}{p^k}} \mid t = 0, 1, 2 \cdot (n-1) \}$.
            So the element $e^{\sfrac{2i\pi}{p^k}}$ (where $t = 1$)
            can generate all other elements
            since $(e^{\sfrac{2i\pi}{p^k}})^t = (e^{\sfrac{2ti\pi}{p^k}})$.
            So all subgroups $H_k$ are cyclic.
        \item
            Proof that every proper subgroup of $Z$ equals $H_k$
            for some fixed $k \in \Z^+$: \\
            Suppose $H < Z$ is a proper subgroup.
            Then for at least one element $z$, $z \in Z$ and $z \notin H$.
            Assuming that $|z| = p^n$,
            that means that $z \in H_n$.
            As we showed in part a,
            this means that $z \in H_k$ where $k \beq n$.
            However, since $H_k$ for any $k$ is cylic,
            as we showed in part b,
            that means that all the generators of each subgroup $H_k$
            with $k \beq n$
            can't be in $H$,
            as the closure axiom would mean $z$ is too.
            So only elements in $H_c$ such that $c < n$
            and not in $H_k$ such that $k \beq n$ can be in $H$.
            So elements in $H$ have order at most $p^{n-1}$.
            If $h$ is the element in $H$ with the largest order, $p^m$
            where $m < n$.
            So $h = e^{\sfrac{2i\pi}{p^m}}$,
            which means that $h$ generates $H_m$ by definition,
            so $H_m \sub G$. \\
            In part a we showed that if $n \seq m$,
            then $H_n \seq H_m$.
            So $H_m$ contains all subgroups with elements 
            of order $m$ or smaller.
            But we assumed $m$ was the largest order of any element in $G$,
            so $G \sub H_m$,
            which means that $G = H_m$.
        \item
            Proof that $Z$ is not finitely generated: \\
            Assume that $Z = \langle z_1, z_2 \dots z_n \rangle$.
            We know that all complex numbers in $Z$ have finite order,
            so assume that $z_k$, one of the generators,
            has the largest order, where $|z_k| = p^m$.
            This means that all elements in $Z$
            must have at most order $p^m$.
            We know that because any element
            in $\langle z_1, z_2 \dots z_n \rangle$
            can be written as $z_1^{r_1}z_2^{r_2} \dots z_n^{r_3}$
            (since $\C$ is abelian, making $Z$ abelian).
            So
            \[ (z_1^{r_1}z_2^{r_2} \dots z_n^{r_3})^{p^m}
            = (z_1^{p^m})^{r_1}(z_2^{p^m})^{r_2} \dots (z_n^{p^m})^{r_n}
            = 1^{r_1}1^{r_2} \dots 1^{r_n}
            = 1 \]
            (This is because $p^m$ is the largest order,
            so for any element $z_i$ with order $p^k$ where $k < m$
            $z_i^{p^m} = z_i^{p^kp^{m - k}}  = 1^{p^{m - k}}$).
            This means however that $Z$ is finite,
            as all its elements have finite order.
            This is a contradiction,
            as $Z$ is assumed to be infinite,
            so $Z$ is not finitely generated.
    \end{enumerate}

    
    \section*{Exercise 19 $***$}
    A nontrivial abelain group $A$ is called \textit{divisible}
    if for each element $a \in A$ and each $k \in \Z^+$,
    there is an element $x \in A$ such that $x^k = a$
    (So each element in $A$ must have a $k$th root for all $k \in \Z^+$).
    \begin{enumerate}[label=\textbf{\alph*.}]
        \item
            Proof that $(\Q, +)$ is divisible: \\
            We alreadt know that $\Q$ is abelian and nontrivial.
            Now, for every element $q \in \Q$,
            $q$ is the $k$th multiple of some other element 
            for any $k \in \Z^+$.
            This is because for any $q = \sfrac{a}{b}$,
            we can consider the rational $\sfrac{a}{kb}$
            that, when multiplied by $k$,
            equals $q$.
        \item
            Proof that no finite abelian groups is ever divisible: \\
            If $G$ is finite,
            then $|G| = n < \infty$.
            This means that for elements $a, x \in G$,
            $a = x^k$ for at most $n$ integers $k$
            (since there are $n$ elements in $G$,
            $x^k$ can have at most $n$ different values,
            as $G$ is closed under multiplication).
            This isn't all positive integers,
            so the group isn't divisible.
    \end{enumerate}


    \section*{Exercise 20}
    Proof that for nontrivial and abelian groups $A$ and $B$,
    $A \times B$ is divisible if and only if $A$ and $B$ are divisible: \\
    First assume that $A$ and $B$ are divisible.
    We know that for any $k \in \Z^+$,
    there exists elements $x \in A$ and $y \in B$
    such that $x^k = a$ and $y^k = b$.
    So for any $k \in \Z^+$ and for any element $(a, b) \in A \times B$,
    we know that $(x, y)^k = (x^k, y^k) = (a, b)$,
    making $A \times B$ divisible. \\
    Conversely, if $A \times B$ is divisble,
    then for any $a \in A$ and $b \in B$,
    and any $k \in \Z^+$,
    we have some element $(x, y) \in A \times B$
    such that $(x, y)^k = (a, b)$.
    This applies to all combinations of elements $a$ and $b$
    in both groups,
    and it implies that there exists elements $x$ and $y$
    in each group such that $x^k = a$ and $y^k = b$.
    So $A$ and $B$ must also be divisible.

    
\end{document}
