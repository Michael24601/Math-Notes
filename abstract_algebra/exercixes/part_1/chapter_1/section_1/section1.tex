
\documentclass[12pt]{article}

\usepackage[margin=1in]{geometry}

% For using float option H that places figures exatcly where we want them
\usepackage{float}
% makes figure font bold
\usepackage{caption}
\captionsetup[figure]{labelfont=bf}
% For text generation
\usepackage{lipsum}
% For drawing
\usepackage{tikz}
% For smaller or equal sign and not divide sign
\usepackage{amssymb}
% For the diagonal fraction
\usepackage{xfrac}
% For enumerating exercise parts with letters instead of numbers
\usepackage{enumitem}
% For dfrac, which forces the fraction to be in display mode (large) e
% even in math mode (small)
\usepackage{amsmath}
% For degree sign
\usepackage{gensymb}
% For "\mathbb" macro
\usepackage{amsfonts}
\newcommand{\N}{\mathbb{N}}
\newcommand{\Z}{\mathbb{Z}}
\newcommand{\Q}{\mathbb{Q}}
\newcommand{\R}{\mathbb{R}}
\newcommand{\C}{\mathbb{C}}

% overline short italic
\newcommand{\olsi}[1]{\,\overline{{#1}}}

\title{%
    \Huge Abstract Algebra \\
    \large by \\
    \Large Dummit and Foote \\~\\
    \huge Part 1: Group Theory \\
    \LARGE Chapter 1: Introduction to Groups \\
    \Large Section 1: Basic Axioms and Examples
}
\date{2023-07-14}
\author{Michael Saba}

\begin{document}
    \pagenumbering{gobble}
    \maketitle
    \newpage
    \pagenumbering{arabic}


    \section*{Exercise 1}
    To prove a group $(G, \ast)$ is associative,
    we need to show that $ \forall a, b, c \in G$,
    we have $\ast(\ast(a, b), c) = \ast(a, \ast(b, c))$.
    % Defines new way to enumerate using bold lowercase letters with a period 
    % at the end (use Alph for uppercase)
    \begin{enumerate}[label=\textbf{\alph*.}]
        \item
            For $G = (\Z, \ast)$ where $\ast(a, b) = a - b$: \\
            We have $\ast(\ast(a, b), c) = (a - b) - c$
            and $\ast(a, \ast(b, c)) = a - (b - c) = a - (b - c)$
            (subtraction is associative in $\Z$).
            So $G$ is associative. 
        \item
            For $G = (\R, \ast)$ where $\ast(a, b) = a + b + ab$: \\
            We have $\ast(\ast(a, b), c)
            = (a + b + ab) + c + (a + b + ab)c
            = a + b + c + ab + ac + bc + abc$
            and $\ast(a, \ast(b, c))
            = a + (b + c + bc) + a(b + c + bc)
            = a + b + c + ab + ac + bc + abc$.
            So $G$ is associative.
        \item
            For $G = (\Q, \ast)$ where $\ast(a, b) = \dfrac{a+b}{5}$: \\
            We have \[\ast(\ast(a, b), c)
            = \dfrac{\dfrac{a + b}{5} + c}{5}
            = \dfrac{a}{25} + \dfrac{b}{25} + \dfrac{c}{5}\]
            and \[\ast(a, \ast(b, c))
            = \dfrac{a + \dfrac{b + c}{5}}{5}
            = \dfrac{a}{5} + \dfrac{b}{25} + \dfrac{c}{25}\]
            So $G$ isn't associative.
        \item
            For $G = (\Z \times \Z, \ast)$
            where $\ast((a, b), (c, d)) = (ad + bc, bd)$: \\
            We have $\ast(\ast((a, b), (c, d)), (e, f)) 
            = \ast((ad + bc, bd), (e, f)) 
            = ((ad + bc)f + (bd)e, (bd)f) = (adf + bcf + bde, bdf)$
            and $\ast((a, b), \ast((c, d), (e, f)))
            = \ast((a, b), (cf + de, df)) 
            = (a(df) + b(cf + de), b(df)) 
            = (adf + bcf + bde, bdf)$.
            So $G$ is associative.
        \item
            For $G = (\Q - \{0\}, \ast)$ where $\ast(a, b) = \dfrac{a}{b}$: \\
            We have \[\ast(\ast(a, b), c)
            = \dfrac{\sfrac{a}{b}}{c} = \dfrac{ac}{b}\]
            and \[\ast(a, \ast(b, c))
            = \dfrac{a}{\sfrac{b}{c}} = \dfrac{ac}{b}\]
            So $G$ is associative.
    \end{enumerate}
   

    \section*{Exercise 2}
    To prove a group $(G, \ast)$ is commutative,
    we need to show that $ \forall a, b \in G$, $\ast(a, b) = \ast(b, a)$.
    \begin{enumerate}[label=\textbf{\alph*.}]
        \item
            For $G = (\Z, \ast)$ where $\ast(a, b) = a - b$: \\
            We have $\ast(a, b) = a - b$ and $\ast(b, a) = b - a$.
            So $G$ isn't commutative.
        \item
            For $G = (\R, \ast)$ where $\ast(a, b) = a + b + ab$: \\
            We have $\ast(a, b) = a + b + ab$ and $\ast(b, a) = b + a + ba$.
            So $G$ is commutative.
        \item
            For $G = (\Q, \ast)$ where $\ast(a, b) = \dfrac{a+b}{5}$: \\
            We have \[\ast(a, b) = \dfrac{a + b}{5}\]
            And \[\ast(b, a) = \dfrac{b + a}{5}\]
            So $G$ is commutative.
        \item
            For $G = (\Z \times \Z, \ast)$
            where $\ast((a, b), (c, d)) = (ad + bc, bd)$: \\
            We have $\ast((a, b), (c, d)) = (ad + bc, bd)$
            and $\ast((c, d), (a, b)) = (cb + da, db)$.
            So $G$ is commutative.
        \item
            For $G = (\Q - \{0\}, \ast)$ where $\ast(a, b) = \dfrac{a}{b}$: \\
            We have $\ast(a, b) = \dfrac{a}{b}$ and $\ast(b, a) = \dfrac{b}{a}$.
            So $G$ isn't commutative. 
    \end{enumerate}


    \section*{Exercise 3}
    Proof that the addition of residue classes in $\Z/n\Z$ is associative: \\
    We have, $\forall \olsi{a}, \olsi{b}, \olsi{c} \in \Z/n\Z$,
    $(\olsi{a} + \olsi{b}) + \olsi{c}
    = \olsi{a + b} + \olsi{c}
    = \olsi{a + b + c}$
    and $\olsi{a} + (\olsi{b} + \olsi{c})
    = \olsi{a} + \olsi{b + c} 
    = \olsi{a + b + c}$.
    So addition is associative in $\Z/n\Z$.


    \section*{Exercise 4}
    Proof that the multiplication of residue classes
    in $\Z/n\Z$ is associative: \\
    We have, $\forall \olsi{a}, \olsi{b}, \olsi{c} \in \Z/n\Z$,
    $(\olsi{a}\olsi{b})\olsi{c}
    = (\olsi{ab})\olsi{c} 
    = \olsi{abc}$ 
    and $\olsi{a}(\olsi{b}\olsi{c})
    = \olsi{a}(\olsi{bc})
    = \olsi{abc}$.
    So multiplication is associative in $\Z/n\Z$.
    

    \section*{Exercise 5}
    Proof that the residue classes $\Z/n\Z$ aren't groups under multiplication
    when $n > 1$: \\
    The issue is that $\olsi{0} \in (\Z/n\Z, \times)$ has more than one inverse
    when $n > 1$.
    This is because $\forall \olsi{a} \in \Z/n\Z$,
    $\olsi{0}\olsi{a} = \olsi{0 \cdot a} = \olsi{0}$.
    So every $\olsi{a} \in \Z/n\Z$ is an inverse of $\olsi{0}$,
    which contradicts the uniqueness of the inverse.
    So $\Z/n\Z$ isn't a group under multiplication.


    \section*{Exercise 6}
    In this exercise, we determine whether the given set $G$ forms a group
    under addition:
    \begin{enumerate}[label=\textbf{\alph*.}]
        \item
            For $G = \{q \in \Q \mid$ where for $q = \dfrac{a}{b}$
            written in lowest terms, $b$ is odd$\}$: \\
            We have $0 = \sfrac{0}{1}$ is the idenity,
            $q^{-1} = -q$ $\forall q \in \Q$,
            and we know that addition is associative in $\Q$.
            So all we have left to show is closure under addition.
            For any $p, q \in \Q$,
            where $p = \dfrac{a}{b}$ and $q = \dfrac{c}{d}$
            are written in lowest terms, $b$, and $d$ are odd,
            and \[p + q = \dfrac{a}{b} + \dfrac{c}{d} = \dfrac{ad + cb}{bd}\]
            As both b and d are odd, $bd$ is odd.
            So even if we reduce the fraction further,
            the denominator will remain odd (no even factor in $bd$).
            So $G$ is closed under addition,
            so $G$ is a group under addition.
        \item
            For $G = \{q \in \Q \mid$ where for $q = \dfrac{a}{b}$
            written in lowest terms, $b$ is even$\}$: \\
            This is not a group as it is not closed under addition.
            To give a counterexample,
            we have $\sfrac{1}{2} + \sfrac{1}{2} = \sfrac{1}{1}$
            where $\sfrac{1}{1} \notin G$.
            So $G$ is not a group under addition.
        \item
            For $G = \{q \in \Q\ \mid |q| < 1\}$: \\
            This is not a group as it is not closed under addition.
            To give a counterexample,
            we have $0.8 + 0.8 = 1.6$, and $1.6 \notin G$.
            So $G$ is not a group under addition.
        \item
            For $G = \{q \in \Q\ \mid |q| > 1\} + \{0\}$: \\
            This is not a group as it is not closed under addition.
            To give a counterexample,
            we have $1.2 - 1.1 = 0.1$, and $0.1 \notin G$.
            So $G$ is not a group under addition.
        \item
            For $G = \{q \in \Q \mid$ where for $q = \dfrac{a}{b}$
            written in lowest terms, $b = 1$ or $2\}$: \\
            We have $0 = \sfrac{0}{1}$ is the idenity,
            $q^{-1} = -q$ $\forall q \in \Q$,
            and we know that addition is associative in $\Q$.
            So all we have left to show is closure under addition.
            For any $p, q \in \Q$,
            where $p = \dfrac{a}{b}$
            and $p = \dfrac{c}{d}$ are written in lowest terms,
            $b$ and $d$ are either both equal to $1$ or $2$,
            or one is $1$ and the other $2$.
            For the last case,
            we can assume with no loss of generality that $b = 1$ and $d = 2$.
            \begin{enumerate}[label=\textbf{case \arabic*:}]
                \item
                    If $b$ and $d$ are both 1,
                    then $\dfrac{a}{1} + \dfrac{c}{1} = \dfrac{a + c}{1}$,
                    where $\dfrac{a+c}{1} \in G$.
                \item
                    If b and d are both 2,
                    then $\dfrac{a}{2} + \dfrac{c}{2} = \dfrac{a + c}{2}$.
                    Since the fractions are written in lowest terms,
                    $a$ and $c$ are odd,
                    so $a + c$ is even.
                    So $a + c = 2n$ for some $n \in \Z$.
                    So we can reduce the fraction to $\dfrac{n}{1}$
                    where $\dfrac{n}{1} \in G$.
                \item
                    If $b = 1$ and $d = 2$,
                    then $\dfrac{a}{1} + \dfrac{c}{2} = \dfrac{a + 2c}{2}$.
                    Since the  fractions are written in lowest terms,
                    $a$ and $c$ are odd,
                    and $2c$ is obviosuly even,
                    so $a + 2c$ is even.
                    So $a + 2c$ is odd,
                    so as the sum is written in lowest terms,
                    we can check that $\dfrac{a + 2c}{2} \in G$.
                    The same would apply for $b = 2$ and $d = 1$.
            \end{enumerate}
            So $G$ is closed under addition,
            which means $G$ is a group under addition.
        \item
        For $G = \{q \in \Q \mid$ where for $q = \dfrac{a}{b}$
        written in lowest terms, $b = 1$ or $2$ or $b = 3\}$: \\
        We can show that G is not closed under addition.
        To give a counterexample,
        take $\dfrac{1}{3} + \dfrac{1}{2} = \dfrac{2 + 3}{6} = \dfrac{5}{6}$,
        which is written in lowest terms as is.
        Since $\dfrac{5}{6} \notin G$,
        $G$ is not closed under addition,
        and is therefore not a group under addition.
    \end{enumerate}


    \section*{Exercise 7}
    For $(G, \ast)$, where $G = \{r \in \R \mid 0 \leqslant r < 1\}$,
    and $\forall x, y \in G$, 
    $x \ast y \equiv (x + y) \mod{1}$
    or $x \ast y = (x + y - \lfloor x + y \rfloor)$
    (The fractional part of $x + y$).
    First we have to show that $\ast$ is well defined in $G$. 
    This means we need to show that the function $\ast:G \times G \to G$
    is well defined.
    For $(x, y), (z, w) \in G \times G$,
    if $(x, y) = (z, w)$,
    then $x = z$ and $y = w$,
    so $\ast(x, y) 
    = (x + y - \lfloor x + y \rfloor) 
    = (z + w - \lfloor z + w \rfloor) 
    = \ast(z, w)$.
    So $\ast$ is well defined.
    Now, to show that $G$ is a group under $\ast$.
    We first note that $\forall r \in G$, $0 + r = r$
    since $0 \leqslant r < 1$,
    so $0$ is the identity.
    Moreover, $r^{-1} = 1 - r$
    when $r \neq 0$.
    This is because when $0 < r < 1$, $0 < 1 - r < 1$
    and $r \ast (1 - r) 
    = r + 1 - r - \lfloor r + 1 - r \rfloor 
    = 1 - \lfloor 1 \rfloor 
    = 1 - 1 = 0$.
    As for associativity, $\forall x, y, z \in G$,
    we have $(x \ast y) \ast z = $
    \[(x + y - \lfloor x + y \rfloor) + c -
    \lfloor (x + y - \lfloor x + y \rfloor) + c \rfloor\]
    and $(x \ast y) \ast z = $
    \[x + (y + z - \lfloor y + z \rfloor)
    - \lfloor x + (y + z - \lfloor y + z \rfloor) \rfloor\] 
    \begin{enumerate}[label=\textbf{case \arabic*:}]
        \item
            If $x + y < 1$ and $y + z < 1$,
            then $(x \ast y) \ast z 
            = (x + y - 0) + c - \lfloor (x + y - 0) + c \rfloor
            = x + y + c - \lfloor x + y + c \rfloor$
            and $(x \ast y) \ast z 
            = x + (y + z - 0) - \lfloor x + (y + z - 0) \rfloor
            = x + y + c - \lfloor x + y + c \rfloor$.
        \item
            If $x + y < 1$ and $y + z \leqslant 1$,
            then $(x \ast y) \ast z 
            = (x + y - 0) + c - \lfloor (x + y - 0) + c \rfloor
            = x + y + c - \lfloor x + y + c \rfloor$
            and $(x \ast y) \ast z 
            = x + (y + z - 1) - \lfloor x + (y + z - 1) \rfloor
            = x + y + c - 1 - (\lfloor x + y + c \rfloor - 1)$ 
            (since $\lfloor r - 1 \rfloor = \lfloor r \rfloor - 1$)
            $ = x + y + c - \lfloor x + y + c \rfloor$.
        \item
            If $x + y \leqslant 1$ and $y + z \leqslant 1$,
            then $(x \ast y) \ast z
            = (x + y - 1) + c - \lfloor (x + y - 1) + c
            = x + y + c - 1 - (\lfloor x + y + c \rfloor - 1)
            = x + y + c - \lfloor x + y + c \rfloor$
            and $(x \ast y) \ast z
            = x + (y + z - 1) - \lfloor x + (y + z - 1) \rfloor
            = x + y + c + 1 - (\lfloor x + y + c \rfloor - 1)
            = x + y + c - \lfloor x + y + c \rfloor$.
    \end{enumerate}
    So as $(x \ast y) \ast z = (x \ast y) \ast z$,
    $G$ is associative.
    Finally, $\forall x, y \in G, x, y < 1$,
    so $x + y < 2$.
    This means that if $x + y < 1$,
    then  $\lfloor x + y \rfloor = 0$,
    which means that $x \ast y = x + y - \lfloor x + y \rfloor = x + y - 0 < 1$.
    And if $x + y > 1$,
    then $\lfloor x + y \rfloor = 1$,
    which means that $x \ast y = x + y - \lfloor x + y \rfloor = x + y - 1 < 1$
    since $x + y < 2$.
    So either way $0 \leqslant x \ast y < 1$.
    Which means that G is closed under $\ast$.
    So $G$ is a group under $\ast$.

    
    \section*{Exercise 8}
    For $G = \{c \in \C \mid c^{n} = 1$ for some $n \in \Z^{+}\}$:
    \begin{enumerate}[label=\textbf{\alph*.}]
        \item 
            To show that $G$ is a group under multilication
            (called the \textit{group of roots of unity} in $\C$),
            we first note that trivially, the identity is $1$,
            and $1 \in G$ because $1^{1} = 1$.
            Moreover, $\forall c \in G$ and $n \in \Z^{+}$,
            $c^{-1} = \dfrac{1}{c}$
            since $c^{n} = 1$
            implies $(\dfrac{1}{c})^{n} = \dfrac{1}{c^{n}} = \dfrac{1}{1} = 1$,
            meaning $\dfrac{1}{c} \in G$.
            Furthermore, $\forall a, b \in G$
            such that $a^{n} = b^{m} = 1$,
            $(a \cdot b)^{nm}
            = a^{nm} \cdot b^{nm}
            = (a^{n})^{m} \cdot (b^{m})^{n}
            = 1^{m} + 1^{n}
            = 1$.
            So $G$ is closed under multiplication.
            Since multiplication is associative in the complex plane,
            this means $G$ is a group under multiplication.
        \item 
            To show that $G$ is not a group under addition,
            we will first note that the only number $x \in \C$
            that satisfies the equation $x + a = a, \forall a \in G$ is $0$.
            But $0 \notin \C$
            as $\nexists n \in \Z^{+}$
            such that $0^n = 1$.
            As $G$ can't have an idenity,
            it doesn't form a group under addition. 
    \end{enumerate}


    \section*{Exercise 9}
    Let $S = \{a + b\sqrt{2} \mid a, b \in \Q\}$
    \begin{enumerate}[label=\textbf{\alph*.}]
        \item 
            Proof that $S$ is a group under addition.
            We know addition is associative for all reals.
            Moreover, we know that $\forall x \in S$, $x^{-1} = -x$
            and 0 is the identity.
            So all we have to show is closure.
            For $x = a + b\sqrt{2}$ and $y = c + d\sqrt{2}$,
            $x + y = (a + c) + (b + d)\sqrt{2}$
            where $(a + c), (b, d) \in \Q$.
            So $x + y \in S$,
            which means S is closed under addition, making it a group. 
        \item
            Proof that $S - \{0\}$ is a group under multiplication.
            We know 1 is the identity,
            and we know multiplication is associative for all the reals.
            Moreover, $\dfrac{1}{a + b\sqrt{2}} \times (a + b\sqrt{2}) = 1$,
            and that $\dfrac{1}{a + b\sqrt{2}}
            = \dfrac{a - b\sqrt{2}}{(a + b\sqrt{2})(a - b\sqrt{2})}
            = \dfrac{a - b\sqrt{2}}{a^{2} + 2b^{2}}
            = \dfrac{a}{n} + \dfrac{-b}{n}\sqrt{2}$
            where $n \in \Q$,
            so $\dfrac{1}{a + b\sqrt{2}} \in S - \{0\}$,
            making it the inverse of $a + b\sqrt{2}$.
            Finally, $\forall a, b, c, d \in \Q$,
            we have $(a + b\sqrt{2})(c + d\sqrt{2})
            = (ac + 2) + (ad + cb)\sqrt{2}$
            where $(ac + 2), (ad + cb) \in \Q$,
            so $S$ is closed under multiplication,
            making it a group under multiplication.
    \end{enumerate}


    \section*{Exercise 10}
    For the group table of size $n$:

    % Triangle figure
    \begin{figure}[H]
        \centering

        \[\vbox{\tabskip0.5em\offinterlineskip
        \halign{\strut$#$\hfil\ \tabskip1em\vrule&&$#$\hfil\cr
        ~   & a_1   & a_2   & a_3 & a_4   & ...   & a_n   \cr
        \noalign{\hrule}\vrule height 12pt width 0pt
        a_1   & a_{11}   & a_{12}   & a_{13} & a_{14} & \dots & a_{1n} \cr
        a_2   & a_{21}   & a_{22} & \ddots \cr
        a_3   & a_{31}   & \ddots & \ddots \cr
        a_4   & a_{41} \cr
        \vdots & \vdots & & & & & \vdots \cr
        a_n   & a_{n1} & & & & \dots & a_{nn} \cr
        }}\]

        \caption{\label{fig:figure1} Group table of size $n$.}
    \end{figure}

    By definition, if the group table is symmetric,
    then $\forall a_i, a_j$ in the group,
    we have $a_i \cdot a_j = a_{ij} = a_{ji} = a_j \cdot a_i$,
    making it abelian.
    Conversely, if the group is abelian,
    then $\forall a_i, a_j$ in the group, $a_i \cdot a_j = a_j \cdot a_i$,
    and all $a_{ij} = a_{ji}$,
    making the group table symmetric.


    \section*{Exercise 11}
    We have $\Z/12\Z = \{\olsi{0},\olsi{1}, \olsi{2}, \olsi{3}, \olsi{4},
    \olsi{5}, \olsi{6}, \olsi{7}, \olsi{8}, \olsi{9}, \olsi{10}, \olsi{11}\}$.
    In the group $(\Z/12\Z, +)$:
    $|\olsi{0}| = 1$,
    $|\olsi{1}| = 12$,
    $|\olsi{2}| = 6$,
    $|\olsi{3}| = 4$,
    $|\olsi{4}| = 3$, 
    $|\olsi{5}| = 12$,
    $|\olsi{6}| = 2$,
    $|\olsi{7}| = 12$,
    $|\olsi{8}| = 3$,
    $|\olsi{9}| = 4$, 
    $|\olsi{10}| = 6$,
    $|\olsi{11}| = 12$. \\
    In other words, it's the $lcm(\olsi{n}, 12)$.


    \section*{Exercise 12}
    As for the group $(\Z/12\Z, \times)$
    and the elements $\{\olsi{1}, \olsi{-1}, \olsi{5}, \olsi{7}, \olsi{-7},
    \olsi{13}\}$:
    $|\olsi{1}| = 1$,
    $|\olsi{-1}| = 2$,
    $|\olsi{5}| = 2$,
    $|\olsi{7}| = 2$,
    $|\olsi{-7}| = 2$, 
    $|\olsi{13}| = |\olsi{1}| = 1$.


    \section*{Exercise 13}
    As for the group $(\Z/36\Z, +)$
    and the elements $\{\olsi{1}, \olsi{2}, \olsi{6}, \olsi{9}, \olsi{10},
    \olsi{12}, \olsi{-1}, \olsi{-10}, \olsi{-18}\}$:
    $|\olsi{1}| = 36$,
    $|\olsi{2}| = 18$,
    $|\olsi{6}| = 6$,
    $|\olsi{9}| = 4$,
    $|\olsi{10}| = 18$, 
    $|\olsi{12}| = 3$,
    $|\olsi{-1}| = |\olsi{37}| = 36$,
    $|\olsi{-10}| = |\olsi{26}| = 18$,
    $|\olsi{-18}| = |\olsi{-18}| = 2$.



    \section*{Exercise 14}
    As for the group $(\Z/36\Z, \times)$
    and the elements $\{\olsi{1}, \olsi{-1}, \olsi{5}, \olsi{13}, \olsi{-13},
    \olsi{17}\}$:
    $|\olsi{1}| = 1$,
    $|\olsi{-1}| = 2$,
    $|\olsi{5}| = 6$,
    $|\olsi{13}| = 3$,
    $|\olsi{-13}| = 6$, 
    $|\olsi{17}| = 2$.


    \section*{Exercise 15}
    Proof that $(a_1a_2...a_n)^{-1} = a_n^{-1}...a_2^{-1}a_1^{-1},
    \forall a_i$ in a group $G$: \\
    We can use induction to show this is the case:\\
    \textbf{Basis step:}
    We know that for $n = 2$, $(ab)^{-1}
    = b^{-1}a^{-1}$.\\
    \textbf{Inductive hypothesis:}
    Assume that for $n = k$, $(a_1a_2...a_n)^{-1}
    = a_n^{-1}...a_2^{-1}a_1^{-1}$. \\ 
    \textbf{Inductive step:}
    For $n = k + 1$,
    take $(a_1a_2...a_na_{n+1})^{-1}
    = (ba_{n+1})^{-1} = a_{n-1}^{-1}b^{-1}
    = a_{n-1}^{-1}a_n^{-1}...a_2^{-1}a_1^{-1}$.

    \section*{Exercise 16}
    Proof that $x^2 = 1$ if and only if $|x| = 1$ or $|x| = 2$: \\
    First, if $x^2 = 1$
    where 1 is the identity,
    then $|x| < 1$ as the order is defined as the smallest such number.
    Since the order is a positive integer,
    that means $|x| = 1$ or $|x| = 2$.
    Conversely, if $|x| = 1$,
    then $x = 1$ so $x^2 = 1^2 = 1$.
    And if $|x| = 2$,
    then $x^2 = 1$.


    \section*{Exercise 17}
    Proof that if $x^n = 1$ then $x^{-1} = x^{n-1}$: \\
    If we multiply both sides by $x^{-1}$,
    we get  $x^n = 1 = x^{n-1} = x^{-1}$.


    \section*{Exercise 18}
    Proof that $xy = yx$ if and only if $y^{-1}xy = x$
    if and only if $x^{-1}y^{-1}xy = 1$: \\
    If $xy = yx$
    (multiply both sides by $y^{-1}$)
    $\implies y^{-1}xy = x$
    (multiply both sides by $x^{-1}$)
    $\implies x^{-1}y^{-1}xy = 1$. \\
    Conversely, if $x^{-1}y^{-1}xy = 1$
    (multiply both sides by $x$)
    $\implies y^{-1}xy = x$
    (multiply both sides by $y$)
    $\implies xy = yx$.


    \section*{Exercise 19 $***$}
    Proof that for $x \in G$ and $a, b \in \Z^{+}:$
    \begin{enumerate}[label=\textbf{\alph*.}]
        \item 
            Proof that $x^{a+b} = x^ax^b$: \\
            \[x^{a+b}
            = \underbrace{x \cdot x \cdot x ... x}_\text{a} \cdot
            \underbrace{x \cdot x \cdot x ... x}_\text{b}
            = (\underbrace{x \cdot x \cdot x ... x}_\text{a})(
                \underbrace{x \cdot x \cdot x ... x}_\text{b})
            = x^ax^b\] \\
            Proof that $x^{ab} = (x^a)^b$: \\
            \[x^{ab}
            = \underbrace{\underbrace{x \cdot x ... x}_\text{a} \cdot
            \underbrace{x \cdot x ... x}_\text{a} ...
            \underbrace{x \cdot x ... x}_\text{a}}_{b}
            = \underbrace{x^a \cdot x^a \cdot ... x^a}_{b}
            = (x^a)^b\]
        \item 
            Proof that $(x^a)^{-1} = x^{-a}$:
            \[(x^a)^{-1} = (\underbrace{x \cdot x \cdot x ... x}_\text{a})^{-1}
            = \underbrace{x^{-1} \cdot x^{-1} \cdot x^{-1} ... x^{-1}}_\text{a}
            = x^{-a}\]
        \item
            Proof that $x^{a+b} = x^ax^b$ applies even when $a, b \leqslant 0$:
            \begin{enumerate}[label=\textbf{case \arabic*:}]
                \item
                    If $b = 0$ and $a$ is an arbitrary integer
                    then $-b > 0$. So $x^{-b} = (x^{b})^{-1}$ by part b.
                    So $x^{a+b} = x^{a-} = x^a \cdot 1 = x^ax^0 = x^ax^b$.
                    Same for $a = 0$ and $b$ as an arbitrary integer.
                \item
                    If $a, b < 0$,
                    then we have $a + b < 0$
                    and $-a, -b, -(a+b) > 0$.
                    So $x^{-(a+b)} = x^{-b - a} = x^{-b}x^{-a}$.
                    Since $-a, -b, -(a+b) > 0$, we can say, by part b,
                    that $(x^{-(a+b)})^{-1} = x^{-(a+b) \times -1} = x^{a+b}$.
                    Same for $(x^{-a})^{-1} = x^a$ and $(x^{-b})^{-1} = x^b$.
                    So $(x^{-(a+b)})^{-1} = (x^{-b}x^{-a})^{-1}$
                    implies $x^{a+b} = (x^{-a})^{-1}(x^{-b})^{-1}$,
                    which gets us $x^{a+b} = x^ax^b$.
                \item
                    If $a > 0$ and $b < 0$ and $a + b < 0$,
                    then since $a, -(a+b) > 0$, using part a,
                    we get that $x^{-(a+b)} \cdot x^{a} = x^{-b}$.
                    We know by part b that as $a > 0$, $x^{-a} = (x^{a})^{-1}$,
                    so by multiplying both sides by $x^{-a}$,
                    we get $x^{-(a+b)} \cdot x^{a}x^{-a} = x^{-b}x^{-a}
                    \implies x^{-(a+b)} = x^{-b}x^{-a}$.
                    Now, as $-b, -(a+b) > 0$, using part b, we get that
                    $(x^{-(a+b)})^{-1} = x^{-(a+b) \times -1} = x^{a+b}.$
                    Same for $(x^{-b})^{-1} = x^b$.
                    As for $a > 0$, $x^{-a}$ is the inverse of $x^{a}$,
                    so $(x^{-a})^{-1} = x^a$.
                    So by inverting both sides,
                    we get that $x^{-(a+b)} = x^{-b}x^{-a}
                    \implies (x^{-(a+b)})^{-1} = (x^{-b}x^{-a})^{-1}
                    \implies x^{(a+b)} = (x^{-a})^{-1}(x^{-b})^{-1}
                    \implies x^{(a+b)} = x^{a}x^{b}$.
                    The same would apply if we had $a < 0$ and $b > 0$.
                \item
                    If $a > 0$ and $b < 0$ and $a + b > 0$,
                    then since $-b > 0$, using part a,
                    we get that $x^{a+b} \cdot x^{-b} = x^{a}$.
                    We know by part b that as $-b > 0$, $x^{-b}
                    = (x^{-b})^{-1} = x^{-b \times -1} = x^{b}$.
                    So even when $b < 0$, $x^{-b} = (x^{b})^{-1}$. 
                    By multiplying both sides by $x^{b}$,
                    we arrive at $x^{a+b} \cdot x^{-b}x^{b} = x^{a}x^{b}
                    \implies x^{a+b} = x^{a}x^{b}$.
                    The same would apply if we had $a < 0$ and $b > 0$.
            \end{enumerate}
            Proof that $x^{ab} = (x^a)^b$ applies even when $a, b \leqslant 0$:
            \begin{enumerate}[label=\textbf{case \arabic*:}]
                \item
                    Assume that $a = 0$ and $b$ is an arbitrary integer.
                    Then $x^{ab} = x^{0 \times b}
                    = x^{0} = 1 = 1^b = (x^{0})^{b}
                    = (x^{a})^{b}$.
                    The same would apply for $b = 0$
                    and $a$ as an arbitrary integer.
                \item
                    Assume that $a, b < 0$.
                    Then $x^{ab} = x^{-a \times -b}$
                    where $-a, -b > 0$.
                    So by part a, we have $(x^{-a})^{-b}$.
                    Now, We know by part b that as $-a > 0$, $x^{-a}
                    = (x^{-a})^{-1} = x^{-a \times -1}
                    = x^{a}$.
                    So even when $a < 0$, $x^{-a} = (x^{a})^{-1}$. 
                    So $(x^{-a})^{-b} = ((x^{a})^{-1})^{-b}$ with $-b > 0$,
                    so by part a, $((x^{a})^{-1})^{-b}
                    = (x^{a})^{-1 \times -b}
                    = (x^{a})^{b}$.
                \item
                    Now, assume that $a < 0$ and $b > 0$.
                    Then $-ab > 0$.
                    This means that, by part b, $x^{-ab} = (x^{ab})^{-1}$.
                    Since $-a, b > 0$, by part a, $x^{ab}
                    = (x^{-ab})^{-1}
                    = ((x^{-a})^{b})^{-1}$
                    and as $b > 0$, by part b, $(x^{b})^{-1} = x^{-b}$,
                    so $((x^{-a})^{b})^{-1}
                    = (x^{-a})^{-b}
                    = ((x^{-a})^{-1})^{b}$.
                    Again, by part b, since $-a > 0$, $x^{-a} = (x^{a})^{-1}$,
                    so $((x^{-a})^{-1})^{b}
                    = (x^{-a \times -1})^{b}
                    = (x^{a})^{b}$.
                    The same would apply if $a > 0$ and $b < 0$.
                \end{enumerate}
    \end{enumerate}


    \section*{Exercise 20 $***$}
    Proof that for $x \in G$, $|x| = |x^{-1}|$:
    According to the exercise 1.1.1.19, $(x^{a})^{-1} = (x^{-1})^{a}$.
    We have $x^{a}(x^{a})^{-1} = 1$.
    Now assume that $x^a = 1$.
    Then $1 \cdot (x^{a})^{-1} = 1$, so $(x^{a})^{-1} = 1$.
    So$(x^{-1})^{a} = 1$.
    This means that $|x^{-1}| \leqslant a$
    since the order is the smallest such number.
    Now assume that $|x| = a$, and that $|x^{-1}| = b$ for $b < a$.
    Then we have $(x^{-1})^{b} = 1$.
    So $x^{a} \cdot (x^{-1})^{b} = 1$,
    so $\underbrace{x \cdot x ... x}_\text{a} \cdot
    \underbrace{x^{-1} \cdot x^{-1} ... x^{-1}}_\text{b} = 1$.
    Since $b < a$, we can cancel out $b$ terms and get
    $\underbrace{x \cdot x \cdot x ... x}_\text{a - b} = 1$,
    which implies that $x^{a-b} = 1$.
    Since $a > a - b$, This contradicts our assumption that $|x| = a$,
    as the order is the smallest such number.
    So $|x^{-1}|$ isn't smaller than $a$.
    This only leaves that $|x| = a$,
    which means $|x| = |x^{-1}|$.


    \section*{Exercise 21}
    Proof that for $x \in G$,
    if $|x| = n$ and $2 \nmid n$,
    then $(x^2)^k = x$ for some integer $k$:
    We know that $(x^2)^k = x^{2k}$.
    If $|x| = n$ and $n$ is odd,
    then $2m - 1$ for some $m \in \Z$.
    So $x^n = x^{2m-1} = 1
    \implies x^{2m-1} \cdot x = x
    \implies x^{2m}x^{-1} \cdot x = x
    \implies x^{2m} = 1
    \implies (x^{2})^{k}$ (by exercise 1.1.1.20).
    Which proves the statement for $k = m$.


    \section*{Exercise 22}
    Proof that $\forall g, x \in G$, $|x| = |gxg^{-1}|$:
    Take $|x| = n$.
    Now consider the product 
    \[(gxg^{-1})^n
    = \underbrace{gxg^{-1} \cdot gxg^{-1} \cdot gxg^{-1} ... gxg^{-1}}_{n}\]
    And because $\cdot$ is associative: 
    \[(gxg^{-1})^n
    = \underbrace{gx(g^{-1}g)x(g^{-1}g)xg^{-1} ... (g^{-1}g)xg^{-1}}_{n}
    = \underbrace{gx(1)x(1)xg^{-1} ... (1)xg^{-1}}_{n}\]
    \[= g\underbrace{x \cdot x \cdot x ... x}_{n}g
    = gx^ng^{-1}
    = g1g^{-1}
    = 1\]
    So $|gxg^{-1}| \leqslant n$.
    Now assume that $|gxg^{-1}| = k < n$.
    Then 
    \[(gxg^{-1})^{k} = 1
    \implies
    \underbrace{gx(g^{-1}g)x(g^{-1}g)xg^{-1} ... (g^{-1}g)xg^{-1}}_{k} = 1
    \implies\]
    \[\underbrace{gx(1)x(1)xg^{-1} ... (1)xg^{-1}}_{k} = 1
    \implies g\underbrace{x \cdot x \cdot x ... x}_{k}g = 1
    \implies gx^kg^{-1} = 1
    \implies\]
    \[gx^k = g
    \implies x^k = gg^{-1} = 1\]
    This is a contradiction is as the order $n$
    is defined as the smallest such number,
    So $|gxg^{-1}|$ isn't smaller than $n$.
    The only remaining option is for $|gxg^{-1}| = n$.
    The proposition we proved applies $\forall x, g \in G$.
    So $\forall a, b \in G$,
    take $g = b$ and $x = (ab)$.
    Then $|b(ab)b^{-1}| = |ab|$,
    so  $|ba(bb^{-1})| = |ba| = |ab|$.


    \section*{Exercise 23}
    Proof that for $x \in G$,
    if $|x| = n < \infty$ and $n = s \cdot t$ where $s, t \in \Z$,
    then $|x^s| = t$: \\
    We have $x^n = 1$,
    so $x^{st} = 1$,
    so $(x^{s}){t} = 1$ (by exercise 1.1.1.20).
    This means $|x^s| \leqslant t$ as it is the smallest such number.
    Now assume that $|x^s| = u < t$.
    This means that $su < st = n$.
    Then $x^{su} = 1$,
    which is a contradiction since we assumed $n$ was the order,
    by definition the smallest such number.
    So $|x^s|$ isn't smaller than $t$,
    so the only remaining option is for $|x^s| = t$.
    
    
    \section*{Exercise 24}
    Proof that for $a, b \in G$,
    if $a$ and $b$ commute ($ab = ba$),
    then $(ab)^n = a^nb^n \forall n \in \Z$: \\
    We first start with proving the proposition for $n > 0$ using induction: \\
    \textbf{Basis step:}
    We know that for $n = 1$, then $(ab)^1 = ab = a^1b^1$. \\
    \textbf{Inductive hypothesis:}
    Assume that for $n = k$, that $(ab)^k = a^kb^k$. \\
    \textbf{Inductive step:}
    For $n = k + 1$, we have $(ab)^{k+1} = (ab)^k(ab)$.
    Then $(ab)^k(ab) = a^kb^k(ab) = a^kab^kb = a^{k+1}b^{k+1}$
    (Since $a$ and $b$ commute,
    then by extension, $a$ and $b^k$ must commute). \\
    Now, for $n = 0$, it is trivial:
    $(ab)^n = (ab)^0 = 1 = 1 \cdot 1 = a^0 \cdot b^0$. \\
    Now assume that $n < 0$.
    Then $-n > 0$,
    so $(ab)^{-n} = a^{-n}b^{-n}$.
    We know from exercise 1.1.1.20 that $(ab)^{-n}
    = ((ab)^{-1})^n
    = ((ba)^{-1})^n
    = (a^{-1}b^{-1})^n$,
    and that $a^{-n}b^{-n} = (a^{-1})^n(b^{-1})^n$.
    So $(a^{-1}b^{-1})^n = (a^{-1})^n(b^{-1})^n$.
    We know that if $ab = ba$,
    then $b^{-1}a = ab^{-1}$,
    so their inverses commute $a^{-1}b^{-1} = b^{-1}a^{-1}$.
    And we know that each element in $G$ has a unique inverse.
    So if by starting with $a$ and $b$
    we get that the second proposition to apply to their inverses,
    we can instead start with $a^{-1}$ and $b^{-1}$ which exist and are unique
    (to which the initial proposition applies as they commute),
    and end up with their inverses, $a$ and $b$, satisfying the second
    proposition, $\forall a, b \in G$.


    \section*{Exercise 25}
    Proof that if $x^2 = 1, \forall x \in G$,
    then $G$ is abelian: \\
    Take any two elements $a, b \in G$.
    Then $ab \in G$,
    so $(ab)^2 = (ab)(ab) = 1$.
    This implies that $(ab)(ab) = 1 = 1 \cdot 1 = a^2b^2$,
    so $abab = aabb$,
    and $a^{-1}ababb^{-1} = a^{-1}aabbb^{-1}$,
    which in turn tells us that $ba = ab$.
    As this applies to any two elements in $G$, $G$ is abelian.

    \section*{Exercise 26 $***$}
    Proof that for a group $(G, \ast)$ and a set $S \neq \emptyset$,
    if $S \subseteq G$ and $S$ is closed under inverses and $\ast$,
    then $S$ is a group: \\
    In $S$, every element has an inverse,
    and the group is closed under $\ast$.
    This means that the identity 1 of $G$ must be in $S$.
    Since $S \neq \emptyset$,
    either 1 is the only element in $S$, completing the proof,
    or at least one element $x \in S$,
    so $x^{-1} \in S$,
    which means $x \ast x^{-1} \in S$,
    so $1 \in S$.
    Furthermore, since $\ast$ is associative in $G$,
    and $S \subseteq G$,
    then $\ast$ is also associative in $S$.
    So $S$ must be a group
    (Called a \textit{subgroup} of $G$ and denoted $S \leqslant G$).


    \section*{Exercise 27}
    Proof that if $x \in G$ is a fixed element,
    and $S = \{x^n \mid n \in \Z\}$,
    $(S, \cdot)$ is a subrgoup of $(G, \cdot)$: \\
    Every element $x$'s inverse $x^{-1} \in S$.
    This is because if $x^n = 1$,
    then $x^{-n} = 1$ where $-n \in \Z$.
    Likewise, for some $n, m \in \Z$, $x^n, x^m \in G$,
    and $x^n \cdot x^m = x^{n+m}$ where $n, m \in \Z$.
    So $x^n \cdot x^m \in G$, which makes $G$ closed under multiplication. 
    Moreover, the multiplicative identity of $S$ is that of $G$,
    as it is in $S$: $x^0 = 1 \in S$.
    Moreover, $\cdot$ is associative in $G$,
    so it is associative in $S$, as $S \subseteq G$.
    So $(S, \cdot) \leqslant (G, \cdot)$.


    \section*{Exercise 28}
    Let $G = A \times B$ be the direct product of $(A, \ast)$ and $(B, \circ)$:
    \begin{enumerate}[label=\textbf{\alph*.}]
        \item 
            Proof of associativity: \\
            For $(a, b), (c, d), (e, f) \in G$,
            \[((a, b)(c, d)) \cdot (e, f)
            = (a \ast c, b \circ d) \cdot (e, f)
            = ((a \ast c) \ast e, (b \circ d) \circ f)\]
            And \[(a, b) \cdot ((c, d)(e, f))
            = (a, b) \cdot ((c \ast e, d \circ f))
            = (a \ast (c \ast e), b \circ (d \circ f))\]
            Since $\ast$ and $\circ$ are associative in $A$ and $B$,
            then $(a \ast c) \ast e = a \ast (c \ast e)$
            and $(b \circ d) \circ f = b \circ (d \circ f)$.
            So $G$ is associative.
        \item     
            Proof that if $1_A$ and $1_B$ are $A$ and $B$'s identities,
            then $(1_A, 1_B)$ is the identity of $G$: \\
            For all $(a, b) \in G$,
            we have $(a, b) \cdot (1_A, 1_B)
            = (a \ast 1_A, b \circ 1_B)
            = (a, b)$.
            Same for $(1_A, 1_B) \cdot (a, b)$.
        \item 
            Proof that for all $(a, b) \in G$,
            $(a, b)^{-1} = (a^{-1}, b^{-1})$: \\
            W have $(a, b) \cdot (a^{-1}, b{-1})
            = (a \ast a^{-1}, b \circ b^{-1})
            = (1_A, 1_B)$.
            Same for $(a^{-1}, b^{-1}) \cdot (a, b)$.
    \end{enumerate}


    \section*{Exercise 29}
    Proof that $A \times B$ is abelian
    if and only if $(A, \ast)$ and $(B, \circ)$ are abelian: \\
    If $A \times B $ is abelian,
    then $\forall a, c \in A$ and $b, d \in B$,
    $(a, b) \cdot (c, d) = (c, d) \cdot (a, b)$,
    so, $(a \ast c, b \circ d) \cdot (c \ast a, d \ast b)$,
    which means $ac = ca$ and $bd = db$,
    meaning $A$ is abelian, and $B$ is abelian.
    Conversely, if we assume that that $A$ and $B$ are abelian, 
    then $(a, b) \cdot (c, d)
    = (a \ast c, b \circ d)
    = (c \ast a, d \circ b)
    = (c, d) \cdot (a, b)$.
    So $A \times B$ is abelian. 


    \section*{Exercise 30}
    Proof that in $A \times B$, $(a, 1_B)$ and $(1_a, b)$ commute: \\
    We know that the identities of $A$ and $B$ commute with every element
    in the groups.
    So since $(a, 1_B)(1_A, b)
    = (a1_A, 1_Bb)
    = (1_Aa, b1_B)
    = (1_A, b)(a, 1_B)$.
    So they commute. \\
    Now we know that $(a, b)^l = (a^l, b^l) = (1_A, 1_B)$
    implies that $n \mid l$ and $m \mid l$,
    where $|a| = n$ and $|b| = m$,
    so any power to which $a$ or $b$ is the identity must be
    a multiple of $n$ and $m$ respectively,
    so one that does so for both $a$ and $b$ must be a multiple of both.
    Since the $lcm(n, m)$ is by definition the smallest number that both
    $n$ and $m$ divide, that makes it the order of $(a, b)$. 

    \section*{Exercise 31 $***$}
    Proof that if $G$ is a finite group with an even order,
    then there is at least one element $x \in G$ such that $|x| = 2$: \\
    Consider the set $t(G) = \{g \in G \mid g \neq g^{-1}\}$.
    Since no $g \in t(G)$ is equal to its own inverse by definition,
    and since each element has a unique inverse to which it is the inverse,
    that means that for each $g \in t(G)$,
    $g^{-1} \in t(G)$, and $g^{-1}$ is distinct from $g$.
    So the elements and their inverses exist in pairs inside of $t(G)$.
    So $t(G)$ must have an even number of elements.
    As for the elements $x \in G - t(G)$ (in $G$ but not $t(G)$),
    we know that $x = x^{-1}$,
    so $x^2 = 1$.
    This means that $|x| \leqslant 2$.
    So $|x| = 1$ or $|x| = 2$.
    Since the identity is the only element in any group with order 1, 
    then the set $t(G) \cup {1}$ has an odd number of elements.
    That means that we have to at least one more element for the set to be
    even, and since the remaining elements all have order 2, then there is
    at least one more element with order 2.


    \section*{Exercise 32}
    Proof that if $x \in G$ and $|x| = n < \infty$,
    then $x^1, x^2, x^3 ... x^n$ are all unique:
    Assume that the opposite is true.
    Then $\exists i, j \in Z$, $1 \leqslant i, j \leqslant n$
    such that $x^i = x^j$.
    Assume with no loss of generality that $i \geqslant j$.
    Then $x^{i - j} = 0$.
    Since $i - j < n$, this contradicts our assumption that $|x| = n$,
    which is by definition the smallest such number.
    So all elements $x^1, x^2, x^3 ... x^n$ are unique. 


    \section*{Exercise 33}
    Consider $x \in G$ such that $|x| = n < \infty$:
    \begin{enumerate}[label=\textbf{\alph*.}]
        \item 
            Proof that if $n$ is odd,
            then $x^i \neq x^{-1}$ when $1 \leqslant i < n$: \\
            Assume that $x^i = x^{-i}$.
            This implies that $x^ix^i = x^{2i} = 1$.
            This is impossible. We know that after $n$, the next largest
            number to the power which $x$ turns to 1 is $2n$.
            We know that $i < n$,
            so $2i < 2n$,
            and $2i \neq n$ as n is odd.
            So this is a contradiction,
            which means that $x^i \neq x^{-i}$ under the given conditions. 
        \item 
            Proof that if $n$ is even such that $n = 2k, k \in \Z$,
            then $x^i = x^{-i}$ for $1 \leqslant i < n$ only when $i = k$: \\
            We know that after $n$, the next largest number to the power 
            which $x$ turns to 1 is $2n$.
            We know that $i < n$,
            so $2i < 2n$.
            There is only one number smaller than $2n$ for which the
            proposition holds, and that is $n$.
            So $2i = n = 2k$ as n is even.
            So this $n = k$.
    \end{enumerate}


    \section*{Exercise 34}
    Proof that if $x \in G$, and $|x| = \infty$,
    then all $x^n \forall n \in \Z$ are unique:
    For $i, j \in \Z$ and $i \neq j$, 
    assume by contradiction that $x^i = x^j$.
    Then $x^{i - j} = 1$.
    But this contradicts our assumption that $|x| = \infty$.
    So $x^n$ must be unique $\forall n$.


    \section*{Exercise 35}
    Proof that for $x \in G$ where $|x| = n < \infty$,
    $\forall n \in \Z$, $x^n = x^i$ for some integer $i$
    such that $0 \leqslant i < n$: \\
    If $0 \leqslant m < n$, then we've already proven it.
    If $m < 0$ or $m \geqslant n$,
    then we can start by using the division algorithm to write $m = qn + r$
    where by definition $q, r \in \Z$ and $0 \leqslant r < n$.
    So we have $x^m = x^{qn + r} = x^{qn}x^r = (x^n)^qx^r = 1^nx^r = x^r$.
    Since $0 \leqslant r < n$, this proves the proposition.


    \section*{Exercise 36 $***$}
    Proof that the group $G = {1, a, b, c}$ is unique and is abelian
    if we assume that none of the elements have order 4: \\
    We define two groups as being the same when a bijective mapping
    of one group's element to another produces a group table with the same
    structure. \\
    Since $|G| = 4$ then all elements in $G$ have an order smaller or equal
    to 4. We've assumed however that no elements have order 4. \\
    Let's start by assuming that $|a| = 3$.
    Then we have $G = {1, a, a^2, b}$.
    This implies that $a^2a = a^3 = 1$.
    So $a^{-1} = a^2$ and vice-versa.
    This also implies that $b$ has no inverse.
    Which is a contradiction as we assumed $G$ was a group.
    So $a$, $b$, and $c$ can't have order 3. \\
    They also can't have order 1 as it is a unique property of the identity. \\
    That means that $|a| = |b| = |c| = 2$.
    So $a = a^{-1}$, $b = b^{-1}$, and $b = b^{-1}$.
    So $aa = bb = cc = 1$.
    Since $a,b \neq 1$, and since the identity is unique,
    $ab \neq a$ and $ab \neq b$,
    so $ab = ba = c$.
    Likewise, $ac = ca = b$ and $bc = cb = a$.
    Since these values are fixed, this means the group is unique, with the
    following group table.

    \begin{figure}[H]
        \centering

        \[\vbox{\tabskip0.5em\offinterlineskip
        \halign{\strut$#$\hfil\ \tabskip1em\vrule&&$#$\hfil\cr
        ~   & 1   & a   & b & c \cr
        \noalign{\hrule}\vrule height 12pt width 0pt
        1   & 1 & a & b & c \cr 
        a   & a & 1 & c & b \cr 
        b   & b & c & 1 & a \cr 
        c   & c & b & a & 1 \cr
        }}\]

        \caption{\label{fig:figure1} The table of the described group.}
    \end{figure}

    Since the group table is symmetric, by exercise 1.1.1.10,
    the group is abelian.

\end{document}