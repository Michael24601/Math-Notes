
\documentclass[12pt]{article}

\usepackage[margin=1in]{geometry}

% For using float option H that places figures exatcly where we want them
\usepackage{float}
% makes figure font bold
\usepackage{caption}
\captionsetup[figure]{labelfont=bf}
% For text generation
\usepackage{lipsum}
% For drawing
\usepackage{tikz}
% For smaller or equal sign and not divide sign
\usepackage{amssymb}
% For the diagonal fraction
\usepackage{xfrac}
% For enumerating exercise parts with letters instead of numbers
\usepackage{enumitem}
% For dfrac, which forces the fraction to be in display mode (large) e
% even in math mode (small)
\usepackage{amsmath}
% For degree sign
\usepackage{gensymb}
% For "\mathbb" macro
\usepackage{amsfonts}
\newcommand{\N}{\mathbb{N}}
\newcommand{\Z}{\mathbb{Z}}
\newcommand{\Q}{\mathbb{Q}}
\newcommand{\R}{\mathbb{R}}
\newcommand{\C}{\mathbb{C}}

% overline short italic
\newcommand{\olsi}[1]{\,\overline{\!{#1}}}

\usetikzlibrary{shapes,positioning,fit,calc}

\usepackage{changepage} % for adjustwidth environment

\title{%
    \Huge Abstract Algebra \\
    \large by \\
    \Large Dummit and Foote \\~\\
    \huge Part 0: Preliminaries \\
    \LARGE Chapter 0: Preliminaries \\
    \Large Section 1: Basics
}
\date{2024-03-30}
\author{Michael Saba}

\begin{document}
    \pagenumbering{gobble}
    \maketitle
    \newpage
    \pagenumbering{arabic}

    \section*{Exercise 1}
    Given a matrix 
    \[ M = \begin{bmatrix}
        1 & 1 \\
        0 & 1 \\
    \end{bmatrix} \]
    and a the set $\mathcal{A}$
    the set of $2 \times 2$ matrices with real entries.\\
    Let's then define 
    \[ \mathcal{B} = \{ X \in \mathcal{A} \mid MX = XM \} \]
    We have
    \[\begin{bmatrix}
        1 & 1 \\
        0 & 1
    \end{bmatrix}
    \begin{bmatrix}
        1 & 1 \\
        0 & 1
    \end{bmatrix}
    = M^2
    = \begin{bmatrix}
        1 & 2 \\
        0 & 1
    \end{bmatrix} \]
    \[ \begin{bmatrix}
        1 & 1 \\
        0 & 1
    \end{bmatrix}
    \begin{bmatrix}
        1 & 1 \\
        1 & 1
    \end{bmatrix}
    = \begin{bmatrix}
        2 & 2 \\
        1 & 1
    \end{bmatrix}
    \quad \neq \quad
    \begin{bmatrix}
        1 & 1 \\
        1 & 1
    \end{bmatrix}
    \begin{bmatrix}
        1 & 1 \\
        0 & 1
    \end{bmatrix}
    = \begin{bmatrix}
        1 & 2 \\
        1 & 2
    \end{bmatrix} \]
    \[ \begin{bmatrix}
        1 & 1 \\
        0 & 1
    \end{bmatrix}
    \begin{bmatrix}
        0 & 0 \\
        0 & 0
    \end{bmatrix}
    = \begin{bmatrix}
        0 & 0 \\
        0 & 0
    \end{bmatrix}
    \quad = \quad
    \begin{bmatrix}
        0 & 0 \\
        0 & 0
    \end{bmatrix}
    \begin{bmatrix}
        1 & 1 \\
        0 & 1
    \end{bmatrix}
    = \begin{bmatrix}
        0 & 0 \\
        0 & 0
    \end{bmatrix} \]
    \[ \begin{bmatrix}
        1 & 1 \\
        0 & 1
    \end{bmatrix}
    \begin{bmatrix}
        1 & 1 \\
        1 & 0
    \end{bmatrix}
    = \begin{bmatrix}
        1 & 2 \\
        1 & 1
    \end{bmatrix}
    \quad \neq \quad
    \begin{bmatrix}
        1 & 1 \\
        1 & 0
    \end{bmatrix}
    \begin{bmatrix}
        1 & 1 \\
        0 & 1
    \end{bmatrix}
    = \begin{bmatrix}
        2 & 1 \\
        1 & 0
    \end{bmatrix} \]
    \[ \begin{bmatrix}
        1 & 0 \\
        0 & 1
    \end{bmatrix}
    = I_2 \text{ (the identity matrix), which means that }
    I_2M = MI_2 = M \]
    \[ \begin{bmatrix}
        1 & 1 \\
        0 & 1
    \end{bmatrix}
    \begin{bmatrix}
        0 & 1 \\
        1 & 0
    \end{bmatrix}
    = \begin{bmatrix}
        0 & 1 \\
        1 & 1
    \end{bmatrix}
    \quad \neq \quad
    \begin{bmatrix}
        0 & 1 \\
        1 & 0
    \end{bmatrix}
    \begin{bmatrix}
        1 & 1 \\
        0 & 1
    \end{bmatrix}
    = \begin{bmatrix}
        1 & 1 \\
        1 & 0
    \end{bmatrix} \]
    So we conclude that
    \[
        \begin{bmatrix}
            1 & 1 \\
            0 & 1
        \end{bmatrix},
        \begin{bmatrix}
            0 & 0 \\
            0 & 0
        \end{bmatrix},
        \begin{bmatrix}
            1 & 0 \\
            0 & 1
        \end{bmatrix}
        \in \mathcal{B}
    \]
    while the rest aren't. \\

    \section*{Exercise 2}
    Proof that if the matrices $P, Q, \in \mathcal{B}$
    (from exercise 0.0.1.1),
    then $P + Q \in \mathcal{B}$: \\
    We have $M(P + Q) = MP + MQ$, and $(P + Q)M = PM + PQ$
    (by the ditributive law of matrix addition and multiplication). \\
    Since $P, Q \in \mathcal{B}$,
    we have $MP = PM$ and $MQ = QM$,
    which means that $MP + MQ = PM + PQ$,
    and by extension, that $M(P + Q) = (P + Q)M$.
    Thus $P + Q \in \mathcal{B}$. \\ 

    \section*{Exercise 3}
    Proof that if the matrices $P, Q, \in \mathcal{B}$
    (from exercise 0.0.1.1),
    then $PQ \in \mathcal{B}$: \\
    We have $M(PQ) = MPQ$, and $(PQ)M = PQM$
    (by the associativity law of matrix multiplication). \\
    Since $P, Q \in \mathcal{B}$,
    we have $MP = PM$ and $MQ = QM$,
    which means that $MPQ = PMQ = PQM$,
    and by extension, that $M(PQ) = (PQ)M$.
    Thus $PQ \in \mathcal{B}$. \\ 

    \section*{Exercise 4 $***$}
    For $p, q, r, s \in \R$,
    \[ \begin{bmatrix}
        p & q \\
        r & s
    \end{bmatrix} \in \mathcal{B} \]
    is true if 
    \[  \begin{bmatrix}
        1 & 1 \\
        0 & 1
    \end{bmatrix}
    \begin{bmatrix}
        p & q \\
        r & s
    \end{bmatrix} 
    = \begin{bmatrix}
        p + r & q + s \\
        r & s
    \end{bmatrix} \]
    is equal to
    \[
        \begin{bmatrix}
        p & q \\
        r & s
    \end{bmatrix}
    \begin{bmatrix}
        1 & 1 \\
        0 & 1
    \end{bmatrix}
    =  \begin{bmatrix}
        p & p + q \\
        r & r + s
    \end{bmatrix} \]
    So we need to satisfy these equations
    \[ p = p + r \qquad p + q = q + s 
    \qquad r = r \qquad s = r + s \]
    The first equation implies $r = 0$,
    and the second implies that $p = s$.
    The third enforces no restrictions,
    and the fourth implies again that $r = 0$.
    So for a matrix to be part of $\mathcal{B}$,
    it must be of the form
    \[ \begin{bmatrix}
        p & q \\
        0 & p
    \end{bmatrix} \]
    where $p$ and $q$ can be any real numbers. \\

    \section*{Exercise 5 $***$}
    The goal is to determine which of these functions are well defined.
    To do that, it is enough to find a value $a = b$
    the mapping of $a$ is different from the mapping of $b$
    (the mapping depends on more than just the value of $a$ and $b$).
    \begin{enumerate}[label=\textbf{\alph*.}]
        \item 
            The function $f: \Q \to \Z$
            defined by $f(\sfrac{a}{b}) = a$ is not well defined.
            Consider the rationals $\sfrac{1}{2}$ and $\sfrac{2}{4}$.
            The two are equal (both are $0.5$),
            and yet $f(\sfrac{1}{2}) = 1$,
            and $f(\sfrac{2}{4}) = 2$,
            which makes the function ambiguous.
        \item
            The function $f: \Q \to \Q$
            defined by $f(\sfrac{a}{b}) = \sfrac{a^2}{b^2}$
            is well defined.
            This is because the function's mapping only depends on
            the value of the input,
            not the value of its numerator and denominator
            (which can differ for the same rational number). \\
            To prove that,
            we note that $\sfrac{a^2}{b^2} = (\sfrac{a}{b})^2$,
            which implies that the mapping only depends on
            the value of the fraction $\sfrac{a}{b}$ as a whole,
            not $a$ or $b$.
    \end{enumerate}

    \section*{Exercise 6 $***$}
    The function $f: \R^+ \to \Z$
    defined as the mapping of a positive real number $r$
    to the first digit to the right of the decimal point in its
    decimal expansion is not well defined. \\
    This is because,
    for any real number $r$,
    there may be more than one way of writing its decimal expansion.
    For instance, although $1 = 0.\bar{9}$,
    $f(1) = 0$ and $f(0.\bar{9}) = 9$. \\

    \section*{Exercise 7 $***$}
    Proof that, for the surjective map $f: A \to B$,
    the relation $a \sim b$ when $f(a) = f(b)$
    is an equivalence relation,
    whose classes are the fibers of $f$: \\
    We know that $f(a) = f(a)$. \\
    We also know that if $f(a) = f(b)$,
    then $f(b) = f(a)$ is trivially true. \\
    Likewise,
    if $f(a) = f(b)$ and $f(b) = f(c)$,
    then $f(a) = f(c)$ is again trivially true. \\
    This means that the relation is an equivalence relations
    which follows directly from $=$
    being an equivalence relation in $B$. \\
    For any $b \in B$,
    there exists a value $a \in A$ such that $f(a) = b$.
    The set
    \[ K = \{ a \mid a \in A \text{ and } f(a) = b \} \]
    is by defintion the fiber of $b$ over $f$. \\
    Since all of the elements in $K$ map to $b$,
    they are all part of the same equivalence class. \\
    Likewise, since all elements of an equivalence class
    represented by $a$ map to $b$,
    then they are all part of the fiber $K$.
    Since the fiber is a subset of the equivalence class,
    and the equivalence class is a subset of the fiber,
    it must be that they are the same set. \\
    So the fibers of the function $f$
    are the equivalence classes of the relation. \\

\end{document}
