
\documentclass[12pt]{article}
\usepackage[margin=1in]{geometry}


%===============================================================================
%================================== PACKAGES ===================================
%===============================================================================

% For using float option H that places figures 
% exatcly where we want them
\usepackage{float}
% makes figure font bold
\usepackage{caption}
\captionsetup[figure]{labelfont=bf}
% For text generation
\usepackage{lipsum}
% For drawing
\usepackage{tikz}
% For smaller or equal sign and not divide sign
\usepackage{amssymb}
% For the diagonal fraction
\usepackage{xfrac}
% For enumerating exercise parts with letters instead of numbers
\usepackage{enumitem}
% For dfrac, which forces the fraction to be in display mode (large) e
% even in math mode (small)
\usepackage{amsmath}
% For degree sign
\usepackage{gensymb}
% For "\mathbb" macro
\usepackage{amsfonts}
% For positioning 
\usepackage{indentfirst}
\usetikzlibrary{shapes,positioning,fit,calc}
% for adjustwidth environment
\usepackage{changepage}
% for arrow on top
\usepackage{esvect}
% for mathbb 1
\usepackage{bbm}
% for mathsrc
\usepackage[mathscr]{eucal}
% For degree sign
\usepackage{gensymb}
% For quotes
\usepackage{csquotes}
% For vertical lines
\usepackage{mathtools}
% For cols
\usepackage{multicol}

% for tikz
\usepackage{pgfplots}
\pgfplotsset{compat=1.18}
\usepackage{amsmath}
\usepgfplotslibrary{groupplots}


%===============================================================================
%==================================== FONTS ====================================
%===============================================================================


% Mathcal
\newcommand{\acal}{\mathcal{A}}
\newcommand{\bcal}{\mathcal{B}}
\newcommand{\ccal}{\mathcal{C}}
\newcommand{\dcal}{\mathcal{D}}
\newcommand{\ecal}{\mathcal{E}}
\newcommand{\fcal}{\mathcal{F}}
\newcommand{\gcal}{\mathcal{G}}
\newcommand{\hcal}{\mathcal{H}}
\newcommand{\ical}{\mathcal{I}}
\newcommand{\jcal}{\mathcal{J}}
\newcommand{\kcal}{\mathcal{K}}
\newcommand{\lcal}{\mathcal{L}}
\newcommand{\mcal}{\mathcal{M}}
\newcommand{\ncal}{\mathcal{N}}
\newcommand{\ocal}{\mathcal{O}}
\newcommand{\pcal}{\mathcal{P}}
\newcommand{\qcal}{\mathcal{Q}}
\newcommand{\rcal}{\mathcal{R}}
\newcommand{\scal}{\mathcal{S}}
\newcommand{\tcal}{\mathcal{T}}
\newcommand{\ucal}{\mathcal{U}}
\newcommand{\vcal}{\mathcal{V}}
\newcommand{\wcal}{\mathcal{W}}
\newcommand{\xcal}{\mathcal{X}}
\newcommand{\ycal}{\mathcal{Y}}
\newcommand{\zcal}{\mathcal{Z}}

% Mathfrak
\newcommand{\afrak}{\mathfrak{A}}
\newcommand{\bfrak}{\mathfrak{B}}
\newcommand{\cfrak}{\mathfrak{C}}
\newcommand{\dfrak}{\mathfrak{D}}
\newcommand{\efrak}{\mathfrak{E}}
\newcommand{\ffrak}{\mathfrak{F}}
\newcommand{\gfrak}{\mathfrak{G}}
\newcommand{\hfrak}{\mathfrak{H}}
\newcommand{\ifrak}{\mathfrak{I}}
\newcommand{\jfrak}{\mathfrak{J}}
\newcommand{\kfrak}{\mathfrak{K}}
\newcommand{\lfrak}{\mathfrak{L}}
\newcommand{\mfrak}{\mathfrak{M}}
\newcommand{\nfrak}{\mathfrak{N}}
\newcommand{\ofrak}{\mathfrak{O}}
\newcommand{\pfrak}{\mathfrak{P}}
\newcommand{\qfrak}{\mathfrak{Q}}
\newcommand{\rfrak}{\mathfrak{R}}
\newcommand{\sfrak}{\mathfrak{S}}
\newcommand{\tfrak}{\mathfrak{T}}
\newcommand{\ufrak}{\mathfrak{U}}
\newcommand{\vfrak}{\mathfrak{V}}
\newcommand{\wfrak}{\mathfrak{W}}
\newcommand{\xfrak}{\mathfrak{X}}
\newcommand{\yfrak}{\mathfrak{Y}}
\newcommand{\zfrak}{\mathfrak{Z}}

% Mathscr
\newcommand{\ascr}{\mathscr{A}}
\newcommand{\bscr}{\mathscr{B}}
\newcommand{\cscr}{\mathscr{C}}
\newcommand{\dscr}{\mathscr{D}}
\newcommand{\escr}{\mathscr{E}}
\newcommand{\fscr}{\mathscr{F}}
\newcommand{\gscr}{\mathscr{G}}
\newcommand{\hscr}{\mathscr{H}}
\newcommand{\iscr}{\mathscr{I}}
\newcommand{\jscr}{\mathscr{J}}
\newcommand{\kscr}{\mathscr{K}}
\newcommand{\lscr}{\mathscr{L}}
\newcommand{\mscr}{\mathscr{M}}
\newcommand{\nscr}{\mathscr{N}}
\newcommand{\oscr}{\mathscr{O}}
\newcommand{\pscr}{\mathscr{P}}
\newcommand{\qscr}{\mathscr{Q}}
\newcommand{\rscr}{\mathscr{R}}
\newcommand{\sscr}{\mathscr{S}}
\newcommand{\tscr}{\mathscr{T}}
\newcommand{\uscr}{\mathscr{U}}
\newcommand{\vscr}{\mathscr{V}}
\newcommand{\wscr}{\mathscr{W}}
\newcommand{\xscr}{\mathscr{X}}
\newcommand{\yscr}{\mathscr{Y}}
\newcommand{\zscr}{\mathscr{Z}}

% Mathbb
\newcommand{\abb}{\mathbb{A}}
\newcommand{\bbb}{\mathbb{B}}
\newcommand{\cbb}{\mathbb{C}}
\newcommand{\dbb}{\mathbb{D}}
\newcommand{\ebb}{\mathbb{E}}
\newcommand{\fbb}{\mathbb{F}}
\newcommand{\gbb}{\mathbb{G}}
\newcommand{\hbb}{\mathbb{H}}
\newcommand{\ibb}{\mathbb{I}}
\newcommand{\jbb}{\mathbb{J}}
\newcommand{\kbb}{\mathbb{K}}
\newcommand{\lbb}{\mathbb{L}}
\newcommand{\mbb}{\mathbb{M}}
\newcommand{\nbb}{\mathbb{N}}
\newcommand{\obb}{\mathbb{O}}
\newcommand{\pbb}{\mathbb{P}}
\newcommand{\qbb}{\mathbb{Q}}
\newcommand{\rbb}{\mathbb{R}}
\newcommand{\sbb}{\mathbb{S}}
\newcommand{\tbb}{\mathbb{T}}
\newcommand{\ubb}{\mathbb{U}}
\newcommand{\vbb}{\mathbb{V}}
\newcommand{\wbb}{\mathbb{W}}
\newcommand{\xbb}{\mathbb{X}}
\newcommand{\ybb}{\mathbb{Y}}
\newcommand{\zbb}{\mathbb{Z}}


%===============================================================================
%=============================== SPECIAL SYMBOLS ===============================
%===============================================================================


% Orbit (group theory)
\newcommand{\orbit}{\mathcal{O}}
% Normal group
\newcommand{\normal}{\mathcal{N}}
% Indicator function
\newcommand{\indicator}{\mathbbm{1}}
% Laplace transform
\newcommand{\laplace}[1]{\mathcal{L}}
% Epsilon shorthand
\newcommand{\eps}{\varepsilon}
% Omega
\newcommand{\om}{\omega}
\newcommand{\Om}{\Omega}

%===============================================================================
%================================== OPERATORS ==================================
%===============================================================================


% Inverse exponent
\newcommand{\inv}[0]{^{-1}}
% Overline bar
\newcommand{\olsi}[1]{\,\overline{\!{#1}}}
% Less than or equal slanted
\newcommand{\seqs}{\leqslant}
% Greater or equal slanted
\newcommand{\geqs}{\geqslant}
% Subset or equal
\newcommand{\sub}{\subseteq}
% Proper subset
\newcommand{\prosub}{\subset}
% from
\newcommand{\from}{\leftarrow}

% Parantheses
\newcommand{\para}[1]{\left( #1 \right)}
% Curly Braces
\newcommand{\curl}[1]{\left\{ #1 \right\}}
% Brackets
\newcommand{\brac}[1]{\left[ #1 \right]}
% Angled Brackets
\newcommand{\ang}[1]{\left\langle #1 \right\rangle}
% Norm
\newcommand{\norm}[1]{\left\| #1 \right\|}

% Piece wise (use \\ between cases)
\newcommand{\piecewise}[1]{\begin{cases} #1 \end{cases}}

% Bold symbol shorthand
\newcommand{\bl}{\boldsymbol}

% Vertical space
\newcommand{\vs}[1]{\vspace{#1 pt}}
% Horizontal ertical space
\newcommand{\hs}[1]{\hspace{#1 pt}}


%===============================================================================
%============================== TEXT BASED SYMBOLS =============================
%===============================================================================


% Radians
\newcommand{\rad}{\text{rad}}
% Least Common Multiple
\newcommand{\lcm}{\text{lcm}}
% Automorphism
\newcommand{\Aut}{\text{Aut}}
% Variance
\newcommand{\var}{\text{Var}}
% Covariance
\newcommand{\cov}{\text{Cov}}
% Cofactor (matrix)
\newcommand{\cof}{\text{Cof}}
% Adjugate (matrix)
\newcommand{\adj}{\text{Adj}}
% Trace (matrix)
\newcommand{\tr}{\text{tr}}
% Standard deviation
\newcommand{\std}{\text{Std}}
% Correlation coefficient
\newcommand{\corr}{\text{Corr}}
% Sign
\newcommand{\sign}{\text{sign}}

% And text
\newcommand{\AND}{\text{ and }}
% Or text 
\newcommand{\OR}{\text{ or }}
% If text
\newcommand{\IF}{\text{ if }}
% When text
\newcommand{\WHEN}{\text{ when }}
% Then text
\newcommand{\THEN}{\text{ then }}

% 1st
\newcommand{\st}[1]{#1^{\text{st}}}
% 2nd
\newcommand{\nd}[1]{#1^{\text{nd}}}
% 3rd
\newcommand{\rd}[1]{#1^{\text{rd}}}
% nth
\newcommand{\nth}[1]{#1^{\text{th}}}


%===============================================================================
%========================= PROBABILITY AND STATISTICS ==========================
%===============================================================================


% Permutation
\newcommand{\perm}[2]{{}^{#1}\!P_{#2}}
% Combination
\newcommand{\comb}[2]{{}^{#1}C_{#2}}

% Baye's risk
\newcommand{\risk}[1]{\mathscr{R}_{#1}}
% Baye's optimal risk
\newcommand{\riskOptimal}[1]{\mathscr{R}_{#1}^*}
% Baye's empirical risk
\newcommand{\riskEmpirical}[2]{\hat{\mathscr{R}}_{#1}^{#2}}


%===============================================================================
%=================================== CALCULUS ==================================
%===============================================================================


% d over d derivative
\newcommand{\dd}[2]{\dfrac{d#1}{d#2}}
% partial d over d derivative
\newcommand{\partialdd}[2]{\dfrac{\partial #1}{\partial #2}}
% delta d over d derivative
\newcommand{\deltadd}[2]{\dfrac{\Delta #1}{\Delta #2}}

% Integration between a and b
\newcommand{\integral}[4]{\int_{#1}^{#2} #3 \, #4}
% Integration in some space 
\newcommand{\boundIntegral}[2]{\int_{#1} #2 \, d#1}

% Limit
\newcommand{\limit}[3]{\lim_{#1 \to #2} #3}


%===============================================================================
%================================  BIG SYMBOLS  ================================
%===============================================================================


% Sum
\newcommand{\sumof}[2]{\sum_{#1}^{#2}}
% Product
\newcommand{\productof}[2]{\prod_{#1}^{#2}}
% Union
\newcommand{\unionof}[2]{\bigcup_{#1}^{#2}}
% Intersection
\newcommand{\intersectionof}[2]{\bigcap_{#1}^{#2}}
% Or
\newcommand{\orof}[2]{\bigvee_{#1}^{#2}}
% And
\newcommand{\andof}[2]{\bigwedge_{#1}^{#2}}


%===============================================================================
%=============================== LINEAR ALGEBRA ================================
%===============================================================================


% Bold vector arrow
\newcommand{\bv}[1]{\vec{\mathbf{#1}}}

% Matrix or vector (use // for column, & for row) 
% with brackets
\newcommand{\bmat}[1]{\begin{bmatrix} #1 \end{bmatrix}}
% Matrix or vector (use // for column, & for row) 
% with curved brackets
\newcommand{\pmat}[1]{\begin{pmatrix} #1 \end{pmatrix}}
% Matrix or vector (use // for column, & for row) 
% with lines on either side 
\newcommand{\lmat}[1]{\begin{vmatrix} #1 \end{vmatrix}}
% Matrix or vector (use // for column, & for row) 
% with curly braces
\newcommand{\cmat}[1]{\begin{Bmatrix} #1 \end{Bmatrix}}
% Matrix or vector (use // for column, & for row) 
% with no braces
\newcommand{\mat}[1]{\begin{matrix} #1 \end{matrix}}


%===============================================================================
%================================ LARGE OBJECTS ================================
%===============================================================================

% Multiple lines
\newcommand{\multiline}[1]{
\begin{align*}
    #1
\end{align*}
}

% Multiple lines with equation numbers
\newcommand{\eqmultiline}[1]{
\begin{align*}
    #1
\end{align*}
}

% Color
\newcommand{\colorText}[2]{
\begingroup
\color{#1}
    #2
\endgroup
}

% Centered figure
\newcommand{\centerFigure}[2]{
    \begin{figure}[h]
        \centering
            #1
        \caption{#2}
    \end{figure}
}

% Tikz figure
\newcommand{\tikzGraphic}[1]{
    \begin{center}
    \begin{tikzpicture}
        #1
    \end{tikzpicture}
    \end{center}
}

% Enumerate numbers (seperate by \item)
\newcommand{\numbers}{\textbf{\number*)}}

% Enumerate letters (seperate by \item)
\newcommand{\letters}{\textbf{\alph*)}}

\title{
    \Huge Abstract Algebra \\
    \large by \\
    \Large Dummit and Foote \\~\\
    \huge Part 1: Group Theory \\
    \LARGE Chapter 2: Subgroups \\
    \Large Section 2: Centralizers and Normalizers, Stabilizers and Kernels
}
\date{2024-03-26}
\author{Michael Saba}

\begin{document}
    \pagenumbering{gobble}
    \maketitle
    \newpage
    \setlength{\parindent}{0pt}
    \pagenumbering{arabic}

    This section will introduce important families of subgroups. \\

    \subsection*{Centralizers}

    Take $A$ to be a non-empty subset of a group $G$.
    The \textbf{centralizer} of $A$ in $G$
    is the set of all elements of in $G$ that commute with all
    elements in $A$. \\
    More formally, the centralizer $C_G(A)$ is the set
    \[ \{ g \in G \mid gag^{-1} = a \text{ for all } a \in A \} \]
    This is because, if $gag^{-1} = a$ for all elements $a \in A$,
    then we can multiply both sides with $g$
    and find that $ga = ag$,
    which implies that $g$ commutes with all elements of $A$. \\

    In the case where $A = \{a\}$,
    meaning it only contains a single element,
    we can call the centralizer of $A$ in $G$
    $C_G(a)$ instead of $C_G(\{a\})$
    as a shorthand. \\
    In this case we know for a fact that all elements
    of the form $a^n$ are in $C_G(A)$ for all $n \in \zbb$. \\

    For an abelian group $G$,
    for any non-empty subset $A$ of $G$,
    we trivially know that $C_G(A) = G$. \\

    All centralizers in $G$, of any subset $A$,
    are subgroups of $G$. \\
    To show that $C_G(A)$ is a subgroup of $G$,
    we first note that $C_G(A) \neq \emptyset$
    because $1 \in C_G(A)$ as the identity
    commutes with all elements in $G$,
    and by extension, $A$. \\
    Moreover, if $x, y \in A$,
    then $xax^{-1} = yay^{-1} = a$ for all $a \in A$.
    We can also multiply the sides by $y$ and $y^{-1}$
    to show that $yay^{-1} = a$ as well.
    Thus we can then show that
    \[ xy^{-1}a(xy^{-1})^{-1} = xy^{-1}a(y^{-1})^{-1}x^{-1} \]
    \[ = xy^{-1}ayx^{-1} \]
    \[ = xax^{-1} \]
    \[ = a \]
    This tells us that $x^y{-1}$ commutes with all elements of $A$,
    making it part of the centralizer. \\
    Therefore $C_G(A) \leqslant G$. \\

    The \textbf{centre} of the group $G$,
    which we denote as $Z(G)$,
    is the set of all elements in $G$
    that commute with all elements in $G$. \\
    Obviously, $Z(G) = C_G(G)$
    (since $G \subseteq G$).
    This means that $Z_(G) \leqslant G$,
    as it is a centralizer. \\

    \subsection*{Normalizers}

    For a non-empty subset $A$ of $G$,
    for any element $g \in G$,
    we define $gAg^{-1}$ to be the set $\{ gag^{-1} \mid a \in A \}$.
    Basically, it is an operation applied on a set (set-wise),
    which produces a new set. \\
    This is called the \textbf{conjugation} of the set $A$. \\
    We also use conjugation to refer to
    the operation $gag^{-1}$ done on elements of sets,
    not just the operation on the set as a whole. \\

    With that in mind,
    the \textbf{normalizer} of $A$ in $G$
    is the set
    \[ \{ g \in G \mid gAg^{-1} = A \} \]
    which we denote by $N_G(A)$. \\
    Basically, it checks that the operation $gag^{-1}$
    for each element of $A$
    permutes $A$;
    $gag^{-1}$ doesn't have to be $a$,
    but there has to be an element $b$ in $A$ such that $gbg^{-1} = a$,
    implying a surjective association.
    Since $b \in A$ too, this also implies that
    there exists a unique $b$ such that $gbg^{-1} = a$,
    making the association a bijection (it permutes $A$). \\
    
    We will later see in chapter 3
    that the normalizer of a subgroup $H$
    in $G$ is a measure of how normal it is
    (defined later). \\

    Notice that if $g \in C_G(A)$,
    $gag^{-1} = a$ for all $a \in A$.
    So $gAg^{-1}$ is trivially $A$
    (the permutation that keeps all elements in place),
    which means that if $g \in N_G(A)$, then $g \in C_G(A)$. \\
    So $N_G(A) \subseteq C_G(A)$. \\

    The normalizer of any non-empty subset $A$ in $G$
    is a subgroup of $G$.
    The proof follows that of the centralizer being a subgroup
    pretty closely, until the end. \\
    We already know that $N_G(A) \neq \emptyset$
    as $1 \in N_G(A)$. \\
    Furthermore, we can show closure under multiplication and inverses
    in the same way as the one we used with the centralizers.
    Eventually we get to the point where we have that
    for every $a \in A$,
    $(xy)a(xy)^{-1} = xy^{-1}ayx^{-1}$. 
    We know that $xAx^{-1}$ and $yAy^{-1}$ are bijections,
    as they permute $A$,
    and we will call them $\phi$ and $\varphi$ respectively.
    We also notice that if $yay^{-1} = b$,
    then $y^{-1}by = a$,
    such that $y^{-1}Ay = \varphi^{-1}$,
    the two-sided inverse of $\varphi$,
    and a bijection in itself.
    So $(xy)A(xy)^{-1} = \phi(\varphi^{-1}(A))$,
    and since the composition of bijections is a bijection,
    this means that 
    \[ \phi(\varphi^{-1}(A)) = \phi(A) = A \]
    which completes the proof. \\

    Since $N_G(A)$ and $C_G(A)$ are subgroups of $G$,
    and $C_G(A)$ happens to be a subset of $N_G(A)$,
    which have that $N_G(A) \leqslant C_G(A)$. \\
    This is because $C_G(A)$ being a subrgoup in its own right
    means that it already satisfies the axioms of a subgroup
    such as not being empty and being closed. \\

    \subsection*{Stabilizers and Kernels}

    If $G$ is a group acting on a non-empty set $S$,
    and $s$ is a fixed element of $S$,
    then $G_s$, the \textbf{stabilizer} of $s$ in $G$,
    is the set $\{ g \in G \mid g \cdot s = s \}$.
    Basically, it's the set of element's whose
    action stabilizes the element $s$. \\
    As we can see, the stabilizer is specific not only
    to the group $G$ and element $s$,
    but also to the action of $G$ on $S$. \\

    We have that $G_s \leqslant G$.
    To show that, we first note that $G_s$ is trivially
    a non-empty subset of $G$,
    since $1 \in G_s$. \\
    Furthermore, we know that if $x, y \in G_s$,
    then $x\cdot s = y\cdot s = s$,
    which means that $(xy) \cdot s = x \cdot (y \cdot s) = x \cdot s = s$
    by the axioms of group actions.
    On the other hand, if $x \in G_s$,
    then we know that $x \cdot s = s$,
    which means that
    $x^{-1} \cdot s = x^{-1} \cdot (x \cdot s) s = (x^{-1}x) \cdot s 
    = 1 \cdot s = s$. \\
    This tells us that $G_s$ is also closed under multiplication
    and inverses,
    which makes it a subgroup of $G$. \\
    
    A \textbf{kernel} of a group $G$ acting on a non-empty set $S$
    is similar to the stabilizer;
    it is the set of elements in $G$ whose action stabilizes
    all elements of $S$.
    Basically, the kernel is the set
    $\{ g \in G \mid g \cdot s = s \text{ for all } s \in S \}$.
    That that means that each element $g$ of the kernel
    must stabilize each element of $S$ individually,
    not just that the set produced by $g$ acting on $S$ is $S$ itself.
    (which is automatically true of actions by an element of a group,
    as they permute the set they act on). \\
    Note that that implies that for any $s \in G$,
    $G_s$ is a subset of the kernel
    (and since the kernel will later be shown to be a subgroup,
    $G_s$ is also a subgroup of the kernel). \\

    We can show that the kernel $K$ of a group $G$ acting on
    a non-empty set $S$ is a subrgoup of $G$. \\
    The argument is the exact same as that of a stabilizer $G_s$,
    but applied to all elements $s \in S$. \\

    The fact that normalizers are subgroups
    is a consequence of stabilizers being subgroups. \\
    Suppose we take the powerset $\mathcal{P}(G)$
    (the set of all subsets of $G$).
    Then we can let $G$ act on $\mathcal{P}(G)$
    by conjugation of its elements (which are sets).
    So for each set $B \in \mathcal{P}(G)$,
    and every element $g \in G$,
    $g \cdot B = gBg^{-1}$.
    There is no need to prove that this is a group action
    because when we defined this operation to be one,
    so all axioms apply by our assertion. \\
    Then, the stabilizer of some element $B$ in $\mathcal{P}(G)$
    is the set of elements $g \in G$ 
    such that $gBg^{-1} = B$,
    which means that any element $g \in G_B$ 
    is also an element of the normalizer of $B$ in $G$,
    meaning that $G_B \subseteq N_G(B)$.
    Likewise, each element of $N_G(B)$ permutes $B$,
    such that their action stabilize $B$,
    so they all belong to $G_B$,
    which means that $N_G(B) \subseteq G_B$.
    So that must mean that $N_G(B) = G_B$,
    so $N_G(B)$ is a subgroup. \\

    Likewise, the fact that centralizers are subgroups
    is a consequence of kernels being subgroups. \\
    Now let a group $G$ act on $A$, one of its non-empty subsets,
    by conjugation
    such that for any elements $g \in G$ and $a \in A$,
    $g \cdot a = gag^{-1}$.
    Again, there is no need to prove this is an action as we defined
    this operation to be one. \\
    The kernel of this action is the set of elements of $G$
    that stabilize all elements $a \in A$,
    so it's the set of elements $g$ such that $gag^{-1} = a$.
    This means that all elements of the stabilizer
    are also part of the centralizer of $A$ in $G$,
    which means that the kernel of $A$ is a subset of $C_G(A)$. \\
    Moreover, the centralizer is the set of elements $g \in G$
    such that $gag^{-1} = a$ for all $a \in A$,
    so they are all part of the kernel of $A$.
    This means that $C_G(A)$ is a subset of the kernel of $A$,
    which in turn tells us that $C_G(A)$ is the kernel of $A$.
    It must therefore be that $C_G(A)$ is a subgroup of $G$
    (and by extension, that $Z(G)$ is a subrgoup when $A = G$). \\
    

\end{document}
