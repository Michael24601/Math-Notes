
\documentclass[12pt]{article}
\usepackage[margin=1in]{geometry}


%===============================================================================
%================================== PACKAGES ===================================
%===============================================================================

% For using float option H that places figures 
% exatcly where we want them
\usepackage{float}
% makes figure font bold
\usepackage{caption}
\captionsetup[figure]{labelfont=bf}
% For text generation
\usepackage{lipsum}
% For drawing
\usepackage{tikz}
% For smaller or equal sign and not divide sign
\usepackage{amssymb}
% For the diagonal fraction
\usepackage{xfrac}
% For enumerating exercise parts with letters instead of numbers
\usepackage{enumitem}
% For dfrac, which forces the fraction to be in display mode (large) e
% even in math mode (small)
\usepackage{amsmath}
% For degree sign
\usepackage{gensymb}
% For "\mathbb" macro
\usepackage{amsfonts}
% For positioning 
\usepackage{indentfirst}
\usetikzlibrary{shapes,positioning,fit,calc}
% for adjustwidth environment
\usepackage{changepage}
% for arrow on top
\usepackage{esvect}
% for mathbb 1
\usepackage{bbm}
% for mathsrc
\usepackage[mathscr]{eucal}
% For degree sign
\usepackage{gensymb}
% For quotes
\usepackage{csquotes}
% For vertical lines
\usepackage{mathtools}
% For cols
\usepackage{multicol}

% for tikz
\usepackage{pgfplots}
\pgfplotsset{compat=1.18}
\usepackage{amsmath}
\usepgfplotslibrary{groupplots}


%===============================================================================
%==================================== FONTS ====================================
%===============================================================================


% Mathcal
\newcommand{\acal}{\mathcal{A}}
\newcommand{\bcal}{\mathcal{B}}
\newcommand{\ccal}{\mathcal{C}}
\newcommand{\dcal}{\mathcal{D}}
\newcommand{\ecal}{\mathcal{E}}
\newcommand{\fcal}{\mathcal{F}}
\newcommand{\gcal}{\mathcal{G}}
\newcommand{\hcal}{\mathcal{H}}
\newcommand{\ical}{\mathcal{I}}
\newcommand{\jcal}{\mathcal{J}}
\newcommand{\kcal}{\mathcal{K}}
\newcommand{\lcal}{\mathcal{L}}
\newcommand{\mcal}{\mathcal{M}}
\newcommand{\ncal}{\mathcal{N}}
\newcommand{\ocal}{\mathcal{O}}
\newcommand{\pcal}{\mathcal{P}}
\newcommand{\qcal}{\mathcal{Q}}
\newcommand{\rcal}{\mathcal{R}}
\newcommand{\scal}{\mathcal{S}}
\newcommand{\tcal}{\mathcal{T}}
\newcommand{\ucal}{\mathcal{U}}
\newcommand{\vcal}{\mathcal{V}}
\newcommand{\wcal}{\mathcal{W}}
\newcommand{\xcal}{\mathcal{X}}
\newcommand{\ycal}{\mathcal{Y}}
\newcommand{\zcal}{\mathcal{Z}}

% Mathfrak
\newcommand{\afrak}{\mathfrak{A}}
\newcommand{\bfrak}{\mathfrak{B}}
\newcommand{\cfrak}{\mathfrak{C}}
\newcommand{\dfrak}{\mathfrak{D}}
\newcommand{\efrak}{\mathfrak{E}}
\newcommand{\ffrak}{\mathfrak{F}}
\newcommand{\gfrak}{\mathfrak{G}}
\newcommand{\hfrak}{\mathfrak{H}}
\newcommand{\ifrak}{\mathfrak{I}}
\newcommand{\jfrak}{\mathfrak{J}}
\newcommand{\kfrak}{\mathfrak{K}}
\newcommand{\lfrak}{\mathfrak{L}}
\newcommand{\mfrak}{\mathfrak{M}}
\newcommand{\nfrak}{\mathfrak{N}}
\newcommand{\ofrak}{\mathfrak{O}}
\newcommand{\pfrak}{\mathfrak{P}}
\newcommand{\qfrak}{\mathfrak{Q}}
\newcommand{\rfrak}{\mathfrak{R}}
\newcommand{\sfrak}{\mathfrak{S}}
\newcommand{\tfrak}{\mathfrak{T}}
\newcommand{\ufrak}{\mathfrak{U}}
\newcommand{\vfrak}{\mathfrak{V}}
\newcommand{\wfrak}{\mathfrak{W}}
\newcommand{\xfrak}{\mathfrak{X}}
\newcommand{\yfrak}{\mathfrak{Y}}
\newcommand{\zfrak}{\mathfrak{Z}}

% Mathscr
\newcommand{\ascr}{\mathscr{A}}
\newcommand{\bscr}{\mathscr{B}}
\newcommand{\cscr}{\mathscr{C}}
\newcommand{\dscr}{\mathscr{D}}
\newcommand{\escr}{\mathscr{E}}
\newcommand{\fscr}{\mathscr{F}}
\newcommand{\gscr}{\mathscr{G}}
\newcommand{\hscr}{\mathscr{H}}
\newcommand{\iscr}{\mathscr{I}}
\newcommand{\jscr}{\mathscr{J}}
\newcommand{\kscr}{\mathscr{K}}
\newcommand{\lscr}{\mathscr{L}}
\newcommand{\mscr}{\mathscr{M}}
\newcommand{\nscr}{\mathscr{N}}
\newcommand{\oscr}{\mathscr{O}}
\newcommand{\pscr}{\mathscr{P}}
\newcommand{\qscr}{\mathscr{Q}}
\newcommand{\rscr}{\mathscr{R}}
\newcommand{\sscr}{\mathscr{S}}
\newcommand{\tscr}{\mathscr{T}}
\newcommand{\uscr}{\mathscr{U}}
\newcommand{\vscr}{\mathscr{V}}
\newcommand{\wscr}{\mathscr{W}}
\newcommand{\xscr}{\mathscr{X}}
\newcommand{\yscr}{\mathscr{Y}}
\newcommand{\zscr}{\mathscr{Z}}

% Mathbb
\newcommand{\abb}{\mathbb{A}}
\newcommand{\bbb}{\mathbb{B}}
\newcommand{\cbb}{\mathbb{C}}
\newcommand{\dbb}{\mathbb{D}}
\newcommand{\ebb}{\mathbb{E}}
\newcommand{\fbb}{\mathbb{F}}
\newcommand{\gbb}{\mathbb{G}}
\newcommand{\hbb}{\mathbb{H}}
\newcommand{\ibb}{\mathbb{I}}
\newcommand{\jbb}{\mathbb{J}}
\newcommand{\kbb}{\mathbb{K}}
\newcommand{\lbb}{\mathbb{L}}
\newcommand{\mbb}{\mathbb{M}}
\newcommand{\nbb}{\mathbb{N}}
\newcommand{\obb}{\mathbb{O}}
\newcommand{\pbb}{\mathbb{P}}
\newcommand{\qbb}{\mathbb{Q}}
\newcommand{\rbb}{\mathbb{R}}
\newcommand{\sbb}{\mathbb{S}}
\newcommand{\tbb}{\mathbb{T}}
\newcommand{\ubb}{\mathbb{U}}
\newcommand{\vbb}{\mathbb{V}}
\newcommand{\wbb}{\mathbb{W}}
\newcommand{\xbb}{\mathbb{X}}
\newcommand{\ybb}{\mathbb{Y}}
\newcommand{\zbb}{\mathbb{Z}}


%===============================================================================
%=============================== SPECIAL SYMBOLS ===============================
%===============================================================================


% Orbit (group theory)
\newcommand{\orbit}{\mathcal{O}}
% Normal group
\newcommand{\normal}{\mathcal{N}}
% Indicator function
\newcommand{\indicator}{\mathbbm{1}}
% Laplace transform
\newcommand{\laplace}[1]{\mathcal{L}}
% Epsilon shorthand
\newcommand{\eps}{\varepsilon}
% Omega
\newcommand{\om}{\omega}
\newcommand{\Om}{\Omega}

%===============================================================================
%================================== OPERATORS ==================================
%===============================================================================


% Inverse exponent
\newcommand{\inv}[0]{^{-1}}
% Overline bar
\newcommand{\olsi}[1]{\,\overline{\!{#1}}}
% Less than or equal slanted
\newcommand{\seqs}{\leqslant}
% Greater or equal slanted
\newcommand{\geqs}{\geqslant}
% Subset or equal
\newcommand{\sub}{\subseteq}
% Proper subset
\newcommand{\prosub}{\subset}
% from
\newcommand{\from}{\leftarrow}

% Parantheses
\newcommand{\para}[1]{\left( #1 \right)}
% Curly Braces
\newcommand{\curl}[1]{\left\{ #1 \right\}}
% Brackets
\newcommand{\brac}[1]{\left[ #1 \right]}
% Angled Brackets
\newcommand{\ang}[1]{\left\langle #1 \right\rangle}
% Norm
\newcommand{\norm}[1]{\left\| #1 \right\|}

% Piece wise (use \\ between cases)
\newcommand{\piecewise}[1]{\begin{cases} #1 \end{cases}}

% Bold symbol shorthand
\newcommand{\bl}{\boldsymbol}

% Vertical space
\newcommand{\vs}[1]{\vspace{#1 pt}}
% Horizontal ertical space
\newcommand{\hs}[1]{\hspace{#1 pt}}


%===============================================================================
%============================== TEXT BASED SYMBOLS =============================
%===============================================================================


% Radians
\newcommand{\rad}{\text{rad}}
% Least Common Multiple
\newcommand{\lcm}{\text{lcm}}
% Automorphism
\newcommand{\Aut}{\text{Aut}}
% Variance
\newcommand{\var}{\text{Var}}
% Covariance
\newcommand{\cov}{\text{Cov}}
% Cofactor (matrix)
\newcommand{\cof}{\text{Cof}}
% Adjugate (matrix)
\newcommand{\adj}{\text{Adj}}
% Trace (matrix)
\newcommand{\tr}{\text{tr}}
% Standard deviation
\newcommand{\std}{\text{Std}}
% Correlation coefficient
\newcommand{\corr}{\text{Corr}}
% Sign
\newcommand{\sign}{\text{sign}}

% And text
\newcommand{\AND}{\text{ and }}
% Or text 
\newcommand{\OR}{\text{ or }}
% For text 
\newcommand{\FOR}{\text{ for }}
% If text
\newcommand{\IF}{\text{ if }}
% When text
\newcommand{\WHEN}{\text{ when }}
% Where text
\newcommand{\WHERE}{\text{ where }}
% Then text
\newcommand{\THEN}{\text{ then }}
% Such that text
\newcommand{\SUCHTHAT}{\text{ such that }}

% 1st
\newcommand{\st}[1]{#1^{\text{st}}}
% 2nd
\newcommand{\nd}[1]{#1^{\text{nd}}}
% 3rd
\newcommand{\rd}[1]{#1^{\text{rd}}}
% nth
\newcommand{\nth}[1]{#1^{\text{th}}}


%===============================================================================
%========================= PROBABILITY AND STATISTICS ==========================
%===============================================================================


% Permutation
\newcommand{\perm}[2]{{}^{#1}\!P_{#2}}
% Combination
\newcommand{\comb}[2]{{}^{#1}C_{#2}}

% Baye's risk
\newcommand{\risk}[1]{\mathscr{R}_{#1}}
% Baye's optimal risk
\newcommand{\riskOptimal}[1]{\mathscr{R}_{#1}^*}
% Baye's empirical risk
\newcommand{\riskEmpirical}[2]{\hat{\mathscr{R}}_{#1}^{#2}}


%===============================================================================
%=================================== CALCULUS ==================================
%===============================================================================


% d over d derivative
\newcommand{\dd}[2]{\dfrac{d#1}{d#2}}
% partial d over d derivative
\newcommand{\partialdd}[2]{\dfrac{\partial #1}{\partial #2}}
% delta d over d derivative
\newcommand{\deltadd}[2]{\dfrac{\Delta #1}{\Delta #2}}

% Integration between a and b
\newcommand{\integral}[4]{\int_{#1}^{#2} #3 \, #4}
% Integration in some space 
\newcommand{\boundIntegral}[2]{\int_{#1} #2 \, d#1}

% Limit
\newcommand{\limit}[3]{\lim_{#1 \to #2} #3}


%===============================================================================
%================================  BIG SYMBOLS  ================================
%===============================================================================


% Sum
\newcommand{\sumof}[2]{\sum_{#1}^{#2}}
% Product
\newcommand{\productof}[2]{\prod_{#1}^{#2}}
% Union
\newcommand{\unionof}[2]{\bigcup_{#1}^{#2}}
% Intersection
\newcommand{\intersectionof}[2]{\bigcap_{#1}^{#2}}
% Or
\newcommand{\orof}[2]{\bigvee_{#1}^{#2}}
% And
\newcommand{\andof}[2]{\bigwedge_{#1}^{#2}}


%===============================================================================
%=============================== LINEAR ALGEBRA ================================
%===============================================================================


% Bold vector arrow
\newcommand{\bv}[1]{\vec{\mathbf{#1}}}

% Matrix or vector (use // for column, & for row) 
% with brackets
\newcommand{\bmat}[1]{\begin{bmatrix} #1 \end{bmatrix}}
% Matrix or vector (use // for column, & for row) 
% with curved brackets
\newcommand{\pmat}[1]{\begin{pmatrix} #1 \end{pmatrix}}
% Matrix or vector (use // for column, & for row) 
% with lines on either side 
\newcommand{\lmat}[1]{\begin{vmatrix} #1 \end{vmatrix}}
% Matrix or vector (use // for column, & for row) 
% with curly braces
\newcommand{\cmat}[1]{\begin{Bmatrix} #1 \end{Bmatrix}}
% Matrix or vector (use // for column, & for row) 
% with no braces
\newcommand{\mat}[1]{\begin{matrix} #1 \end{matrix}}


%===============================================================================
%================================ LARGE OBJECTS ================================
%===============================================================================

% Multiple lines
\newcommand{\multiline}[1]{
\begin{align*}
    #1
\end{align*}
}

% Multiple lines with equation numbers
\newcommand{\eqmultiline}[1]{
\begin{align*}
    #1
\end{align*}
}

% Color
\newcommand{\colorText}[2]{
\begingroup
\color{#1}
    #2
\endgroup
}

% Centered figure
\newcommand{\centerFigure}[2]{
    \begin{figure}[h]
        \centering
            #1
        \caption{#2}
    \end{figure}
}

% Tikz figure
\newcommand{\tikzGraphic}[1]{
    \begin{center}
    \begin{tikzpicture}
        #1
    \end{tikzpicture}
    \end{center}
}

% Enumerate numbers (seperate by \item)
\newcommand{\numbers}{\textbf{\number*)}}

% Enumerate letters (seperate by \item)
\newcommand{\letters}{\textbf{\alph*)}}

\title{%
    \Huge Abstract Algebra \\
    \large by \\
    \Large Dummit and Foote \\~\\
    \huge Part 1: Group Theory \\
    \LARGE Chapter 2: Subgroups \\
    \Large Section 3: Cyclic Groups and Cyclic Subgroups
}
\date{2024-03-28}
\author{Michael Saba}

\begin{document}
    \pagenumbering{gobble}
    \maketitle
    \newpage
    \setlength{\parindent}{0pt}
    \pagenumbering{arabic}

   
    Another family of groups and subgroups worth studying
    is that of cyclic groups,
    which is comprised of groups that are generated
    by a single element
    (all elements being powers of a single element). \\
    One way to form a cyclic group is to take an element $x$
    from an arbitrary group $G$
    and letting $H$ contain all the unique powers of $x$
    (such that $x$ generates all of it). \\


    \subsection*{Cyclic Groups}

    A group $H$ is \textbf{cyclic}
    if it can be generated by a single element 
    $x \in H$ such that $H = \{ x^n \mid n \in \Z \}$. \\
    This clearly forms a group as closure under multiplication
    and inverses is guaranteed.
    The identity is also present as it is the multiplication of
    no element $x$ (e.g. $x^0$). \\
    Since $H$ is generated by $x$,
    we can write $H$ as $\ang{x}$.
    Note however that cyclic groups can have more than one generator;
    for instance, $(x^{-1})^n = (x^n)^{-1}$ for any $n \in \Z$,
    so we can generate $H$ using $x^{-1}$ too,
    which means that
    \[ \{ x^n \mid n \in \Z \} = \{ (x^{-1})^n \mid n \in \Z \} \]
    In other words $H = \ang{x^{-1}}$.
    Other elements my generate the group as we will see later. \\

    All cyclic groups are abelian by the laws of exponents. \\

    If $G$ is a group, and $x \in G$,
    then we can define a cyclic group $\ang{x}$
    such that $\ang{x} \leqslant G$. \\
    This is because $\ang{x}$ is clearly closed under
    multiplication and inverses
    by virtue of being a group in its own right,
    and it is clearly a subset of $G$
    (since $G$ is closed under inverse and multiplication too,
    all elements generated by $x$ are in $G$). \\
    Thus, by taking any element $x$ in any group $G$,
    we can define a subgroup $\ang{x}$ of $G$.
    We can find the contents of $\ang{x}$
    by just repeatedly multiplying by $x$ until we get $1$. \\
    One such example is the subgroup of rotations of $D_{2n}$
    generated by taking $r$,
    which we can denote $\ang{r}$,
    and which will contain
    $\{1, r, \dots r^{n-1}\}$ since $r^n = 1$. \\

    \subsection*{Properties}

    If $H$ is a cyclic group $\ang{x}$,
    then $|H| = |x|$.
    More specifically
    \begin{itemize}[label=$\diamond$]
        \item 
            If $|H| = n < \infty$,
            then $1$, $x$, $x^2 \dots$ $x^{n-1}$ are all distinct,
            and form the entirety of $H$. \\
            To prove this statement,
            let $|x| = n$, then by definition,
            $n$ is the smallest positive integer for which $x^n = 1$.
            This implies that $1$, $x$, $x^2 \dots$ $x^{n-1}$
            are all distinct, because if $x^a = x^b$ where $a > b$,
            that would imply that $x^{b-a} = 0$,
            where $b-a < 0$. 
            This means that $H$ has at least these $n$ elements.
            Now if $i > n$, then by the division algorithm,
            we could write
            \[ x^i \text{ as } x^{nq + k} = x^{nq} + x^k = (x^q)^{n} + x^k 
            = 1 + x^k = x^k \]
            where $0 \leqslant k < n$.
            So any element $x^i$ reduces to being one of the $n$
            aforementioned elements, making them the only elements of $H$.
        \item 
            If $|H| = \infty$,
            then $x^n \neq 1$ for any $n \neq 0$
            (meaning that $x$ has infinite order).
            This means that there can't be any $x^a = x^b$
            unless $a = b$,
            which means that $|H| = \infty$. \\
            We can prove this statement as well.
            If $|x| = \infty$,
            then no positive power of $x$ is the identity.
            If we have $x^a = x^b$ where $a \neq b$,
            then that would mean that $x^{a-b} = 1$ where $a-b \neq 0$,
            which is a contradiction.
            So all distinct powers of $x$ are distinct elements of $H$,
            such that $|H| = \infty$.
    \end{itemize}
    
    Notice that when we reduce powers of elements in $\ang{x}$
    to the least residue powers, we are using modular arithmetic
    ($x^a = x^c$ where $c$ is the remainder after dividing by $n$,
    the order of $x$).
    This is no coincidence as
    cyclic groups are isomorphic to $\Z/n\Z$. \\

    First, we show that for $G$, an arbitrary group,
    and any element $x \in G$ as well as any integers $n, m \in \Z$,
    if $x^n = 1$,
    and $x^m = 1$,
    it must be that $x^d = 1$,
    where $d = \gcd(m, n)$. \\ 
    To prove this, we can use the Euclidian Algorithm;
    there exists integers $r$ and $s$ such that $d = mr + ns$.
    We then get
    \[ x^d = x^{mr + ns} = x^{mr} + x^{ns} = 1^r + 1^s = 1 \]
    completing the proof. \\

    We can now use this result to show that
    if $x^m = 1$ and $n = |x|$, then $n \mid m$.
    We first note that if $m = 0$, it is trivially true,
    so we may assume that $m \neq 0$. 
    If we take $d$ to be $\gcd(m, n)$,
    then $x^d = 1$.
    Since $0 < d \leqslant n$
    by the definition of the greatest commin divisor,
    and $n$ is the smallest positive power of $x$ such that it is $1$,
    it must be that $d = n$. So $n \mid m$. \\
    
    Any two cyclic groups of the same order are isomorphic.
    \begin{itemize}[label=$\diamond$]
        \item 
            First we consider the case where
            $n \in \Z^+$ is a finite value
            such that $|\ang{x}| = |\ang{y}| = n$.
            Then we can define a map
            \[ \varphi: \ang{x} \to \ang{y} \qquad x^k \mapsto y^k \]
            which will turn out to be an isomorphism. \\
            To prove this,
            we first need to show that the function is well defined,
            meaning that if $x^r = x^s$,
            then $\varphi(x^r) = \varphi(x^s)$
            (the mapping is independant of the exponent we choose
            as long as the element is the same).
            If $x^r = x^s$, then $x^{r-s} = 1$.
            So $n$, the order of $\ang{x}$ and by extension,
            the order of $x$ (also the order of $y$),
            divides $r - s$.
            So $r-s = tn$ for some $t \in \Z$,
            which we can write as $r = tn - s$,
            leading us to
            \[
                \varphi(x^r) = \varphi(x^{tn + s})
                = y^{tn + s}
                = y^{tn}y^{s}
                = 1^ty^s
                = y^s
                = \varphi(y^s)
            \]
            This map is thus well defined.
            By the laws of exponents,
            $\varphi(x^ax^b) = \varphi(x^{a+b})
            = y^{a+b} = y^ay^b = \varphi(x^a)\varphi(x^b)$,
            which makes it a homomorphism. 
            Since we know each element $y^k$
            is associated with $x^k$,
            the map is surjective.
            Because the two have the same order,
            the surjection becomes a bijection,
            which turns the homomorphism into an isomorphism.
        \item 
            Likewise,
            any two infinite cyclic groups are isomorphic.
            We note that if $\ang{x}$ has infinite order,
            then
            \[ \varphi: \Z \to \ang{x} \qquad k \mapsto x^k \]  
            is an isomorphism,
            where $\Z$ is the additive group of integers.
            There is no ambiguity concerning integers being mapped
            to a power of $x$, so it is well defined by default.
            It is clearly a homomorphism,
            since $\varphi(a+b) = y^{a+b} = y^ay^b = \varphi(a)\varphi(b)$.
            Moreover, it is a surjection,
            as each $x^k$ is associated with an integer.
            Likewise,
            since $x^a \neq x^b$ for $a \neq b$
            (as the group is infinite),
            the mapping is also injective,
            which in turn makes it bijective.
            Thus the mapping is an isomorphism. 
            So, since all infinite cyclic groups are isomorphic
            to the same group $\Z$,
            it must be that they are all isomorphic to each other.
    \end{itemize}

    The cyclic group $\ang{x}$ or order $n$
    from now on will be denoted using $Z_n$. \\
    When representing infinite cyclic groups,
    we can just use $\Z$, teh additive group of integers. \\

    Since $\Z/n\Z$ is cyclic and of order $n$
    for any positive integer $n$,
    it is isomorphic to $Z_n$. \\

    For some more properties about groups in general
    (not just cyclic groups),
    suppose $G$ is a group and $x \in G$ and $a \in \Z - \{0\}$.
    \begin{itemize}[label=$\diamond$]
        \item 
            If $|x| = \infty$, then 
            then $|x^a| = \infty$ for any non-zero power of $x$. \\
            We can prove this by contradiction.
            Suppose that $|x^a| = m$ where $m \in \Z^+$.
            Then $x^{am} = (x^a)^m = 1$,
            and $x^{-am} = (x^{am})^{-1} = 1^{-1} = 1$.
            Since either $am$ or $-am$ is positive,
            we conclude that some positive power of $x$
            is the identity,
            which contradicts our assumption that $|x| = \infty$,
            completing the prood.
        \item
            If $|x| = n < \infty$
            then $x^a = \dfrac{n}{\gcd(n, a)}$. \\
            This intuitively makes perfect sense.
            We know that $x^n = 1$ and that $x^n$
            is the smallest positive power where this is true.
            In order for a power of $x^a$ to be the identity,
            it needs to be raised to a power $b$
            such that $ab$ contains the factor $n$.
            So for instance, $b$ could be $n$ itself,
            since $x^{an} = 1$. 
            For $b$ to the be its order however,
            it also needs to be the smallest such number.
            So we need to divide $an$ by the largest
            number possible without removing $n$ as a factor
            and while keeping it so that we have
            an integer $b$ multiplying $a$.
            This means we can divide by and get rid of 
            any factors in common between $a$ and $n$.
            To minimize the number,
            we can do this for all factors in common,
            which when taken together form $\gcd(a, n)$. \\
            We still need to prove the statement however.
            Taking $y = x^a$ and $d = \gcd(a, n)$,
            we can write $n = db$ and $a = dc$ 
            for some integers $b, c \in \Z$
            where $b > 0$ (since $n, b > 0$).
            We now note that $b$ and $c$ are relatively prime,
            because if they had factors in common,
            $d$ would not be the greatest common divisor of $a$ and $n$.
            We notice that
            \[ b = \dfrac{n}{\gcd(n, a)} \]
            so we need to show that $|y| = 1$.
            First we note that 
            \[ y^b = x^{ab} = x^{dcb} = x^{(db)c} = (x^n)^c = 1^c = 1 \]
            Next, we need to show that $y^b$ is the smallest power
            of $y$ equal to the identity.
            We have that $x^{a|y|} = y^k = 1$,
            so $n \mid a|y|$,
            which means that $db \mid dc|y|$,
            and by extension $b \mid c|y|$.
            Since $b$ and $c$ have no factors in common,
            it must be that $b \mid |y|$.
            But by the same argument, since $y^b = 1$, $|y| \mid b$.
            Thus $b = |y|$, since they both divide each other.
        \item 
            If $|x| = n < \infty$,
            and $a \in \Z$ is an integer such that $a \mid n$,
            then $x^a = \dfrac{n}{a}$. \\
            This is just a special case of the last proposition,
            and is only recorded here for convenience.
    \end{itemize}
    
    Now, we can use these propositions to deduce some properties
    about cyclic groups.
    Take $H = \ang{x}$:
    \begin{itemize}[label=$\diamond$]
        \item 
            If $|x| = \infty$,
            then $H = \ang{x^a}$ if and only if $a = \pm 1$.
            What this means is that an infinite cyclic group
            generated by $x$
            is always generated by its inverse $x^{-1}$,
            but no other powers of $x$. \\
            In order to show that a power of $x$, $x^a$,
            generates $\ang{x}$,
            we just need to show that $x^a$ can generate $x$,
            and that means it can generate any other element
            the same way $x$ does.
            In order to show that a power of $x^a$
            can't generate $\ang{x}$,
            we just need to find an element it can't generate,
            and in our case this will be the element $x$. \\
            Suppose that for $a \in \Z$,
            some power of $x^a$ does generate $x$.
            Then there exists an integer $b \in \Z$
            such that ${x^a}^b = x$.
            We know from a previous proposition that when $\ang{x}$
            is infinite, $x^{c} = x^d$ if only if $c = d$.
            So that must mean that $ab = 1$.
            Since $b$ is an integer,
            this is only possible if $a$ has absolute value $1$.
            and $b = a$.
            So when $a = \pm 1$, we can take $b = \pm 1$,
            and generate $x$ using $x^a$.
            Otherwise, it is impossible to generate $x$ using $x^a$,
            so $\ang{x^a}$ can't be the same as $\ang{x}$
            as it does not contain $x$.
        \item
            If $|x| = n < \infty$, then $|H| = \ang{x^a}$
            if and only if $\gcd(a, n) = 1$.
            The number of generators of $H$ is $\varphi(n)$
            where $\varphi$ is Euler's phi-function. \\
            To prove this, we need to show that the group
            generated by $x^a$, $\ang{x^a}$,
            is the same as $\ang{x}$.
            We know that $|x^a| = |x|$
            if and only if $\dfrac{n}{\gcd(n, a)} = n$,
            which is true if and only if $\gcd(n, a) = 1$.
            We know that $\ang{x^a}$ is a subgroup of $\ang{x}$
            since $x^a \in \ang{x}$
            (any group generated by an element in another group
            is its subgroup).
            If a subgroup has the same order as the group,
            then it is equal to the group.
            Since $|x^a| = |x|$, $|\ang{x^a}| = |\ang{x}|$,
            which means that $\ang{x^a} = \ang{x}$. \\
            The second statement is true because by definition,
            $\varphi(n)$ is the number of integers
            between $1$ and $n$ that are relatively prime to $n$.
    \end{itemize}

    \subsection*{Cyclic Subgroups}

    We can now make some claims about cyclic subgroups of
    cyclic groups.
    Suppose that $H = \ang{x}$ is a cyclic group:
    \begin{itemize}[label=$\diamond$]
        \item 
            Every subgroup of $H$ is cyclic,
            and if $K \leqslant H$,
            then $K = \{1\}$
            or $K = \ang{x^d}$
            where $d$ is the smallest positive integer
            such that $x^d \in K$. \\
            To prove this, we first note that
            if $K = \{1\}$, the trivial group,
            the proposition is clearly true,
            so we can assume that $K \neq \{1\}$.
            So we can assume that some integer $a \neq 0$
            exists such that $x^a \in K$.
            We know that $x^{-a} = (x^a)^{-1}$ is also in the group
            by the closure of $K$ under inverses.
            So $K$ always contains a positive power of $x$.
            We now define
            \[ \mathcal{P} = \{ b \mid b \in \Z^+ x^b \in K \} \]
            where $\mathcal{P}$ is basically the set of
            all positive powers of $x$ in $K$.
            Since this is a subset of $\Z^+$,
            by the Well Ordering Principle, 
            it must have a minimal element,
            which we will call $d$.
            Since $K$ is itself a group (since it's a subgroup),
            and $x^d \in K$,
            $\ang{x^d} \leqslant K$.
            We know that $K \leqslant H$,
            so all elements of $K$ are powers of $x$ of the form $x^a$
            for integers $a \in \Z$.
            Since $d \leqslant a$ by definition of $d$,
            by the division algorithm this means that 
            there exists some $r, q \in \Z$ such that
            \[ a = qd + r \text{ such that } 0 \leqslant r < d \]
            So $r = a - qd$,
            which means that
            \[ x^r = x^{a - qd} = x^ax^{-qd} = (x^d)^{-q} \]
            Since both $x^a$ and $x^d$ are elements of $K$,
            then $x^r$ must also be part of $K$.
            Since $d$ is the smallest positive integer
            such that $x^d \in K$,
            it must be that $r = 0$,
            which means that $a = dq$.
            So all elements $x^a \in K$ can be written as $(x^d)^q$,
            which means that $x^d$ generates $K$.
            So $K = \ang{x^d}$.
        \item 
            If $|H| = \infty$,
            then for any distinct non-negative integers $a,b \in \N$,
            $\ang{x^a} \neq \ang{x^b}$.
            Furthermore, for any integer $m \in \Z$,
            $\ang{x^m} = \ang{x^{|m|}}$.
            As such, all non trivial subgroups of $H$
            correspond bijectively
            to the positive integers $1$, $2$, $3 \dots$. \\
            What this means essentially is that every subgroup $K$
            of $H$ of the form $\ang{x^a}$,
            is always generated by only $x^a$ and $x^{-a}$.
            So we need to prove that $\ang{x^a} = \ang{x^b}$
            if and only if $a = \pm b$. \\
            We know that since $|\ang{x}| = |x| = \infty$,
            that $|x^a| = \infty$.
            We also know that if $K = \ang{y}$ and $|K| = \infty$,
            $K = \ang{y^a}$ if an only if $a = \pm 1$.
            So since $\ang{x^a}$ is an infinite cyclic group
            in its own right,
            it can always be generated by
            just $(x^a)^1$ and $(x^{a})^{-1} = x^{-a}$,
            proving that $\ang{x^a} = \ang{x^b}$
            if and only if $a = \pm b$. \\
            As such, we can associate each the unique subgroup
            $\ang{x^a} = \ang{x^{|a|}}$
            where $a \neq 0$ (non trivial subgroup)
            with the positive integer $|a|$,
            which proves the second statement.
        \item
            If $|H| = n < \infty$,
            then for each positie integer $a$ that divides $n$,
            there exists a unique subgroup of $H$ order $a$,
            which is $\ang{x^d}$ where $d = \dfrac{n}{a}$. \\
            To prove this, 
            we first note that $|\ang{x^d}| = \dfrac{n}{d} = a$,
            which proves existence.
            To further show that it is unique,
            we will suppose that $K$ is any subgroup of $H$
            of order $a$,
            and show that it is the same as $\ang{x^d}$. \\
            We know that any subgroup of a cyclic group $\ang{x}$
            is generated by the smallest power of $x$ in the subgroup.
            So $K = \ang{x^b}$,
            where $b$ is the smallest positive integer such that
            $x^b \in K$.
            Since $x^b \in H$, then $|x^b| = \dfrac{n}{\gcd(n, b)}$.
            But also, $|x^b| = |\ang{x^b}| = a = \dfrac{n}{d}$.
            So $d = \gcd(n, b)$, which means that $d \mid b$.
            Since $b$ is a multiple of $d$, $x^b \in \ang{x^d}$,
            which means that $\ang{x^b}$ is a subgrouo of $\ang{x^d}$.
            Since both have the same order $a$,
            then $K = \ang{x^d} = \ang{x^b}$.
            Therefore, all subgroups of $H$ of order $a$
            are the same as $\ang{x^d}$,
            which makes it the only subgroup of $H$
            of order $a$.
        \item
            For any integer $m$,
            $\ang{x^m} = \ang{x^{\gcd(n, m)}}$.
            From this we can conclude that all subgroups of $H$
            correspond bijectively to the positive divisors of $H$. \\
            To prove this,
            we first show that $|\ang{x^m}| = |\ang{x^{\gcd(n, m)}}|$:
            \[ |\ang{x^m}| = |x^m| = \dfrac{n}{\gcd(n, m)} \]
            and 
            \[ |\ang{x^{\gcd(m, n)}}|
            = |x^{\gcd(m, n)}|
            = \dfrac{n}{\gcd(n, \gcd(n, m))} \]
            If $a \mid n$, then $\gcd(a, n) = a$.
            So since $\gcd(n, m) \mid n$,
            $\gcd(n, \gcd(n, m)) = \gcd(n, m)$,
            which means that $|\ang{x^m}| = |\ang{x^{\gcd(n, m)}}|$. \\
            We then show that 
            that $\ang{x^m} \leqslant \ang{x^{\gcd(n, m)}}$.
            This is because $x^m \in \ang{x^{\gcd(n, m)}}$,
            since $m$ is a multiple of $\gcd(n, m)$.
            Since $|\ang{x^m}| = |\ang{x^{\gcd(n, m)}}|$,
            we can conclude that $\ang{x^m} = \ang{x^{\gcd(n, m)}}$.
            Since $\gcd(n, m)$ clearly divides $n$,
            we can conclude that all subgroups of $H$
            correspond to $|H|$'s divisors.
    \end{itemize}

    This implies that for any cyclic group $H$,
    if it has a subgroup $K$ of length $m$,
    then $K$ is the only subgroup with that length
    (though it may be generated by different elements). \\


\end{document}
