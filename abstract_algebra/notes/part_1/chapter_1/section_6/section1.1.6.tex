
\documentclass[12pt]{article}
\usepackage[margin=1in]{geometry}


%===============================================================================
%================================== PACKAGES ===================================
%===============================================================================

% For using float option H that places figures 
% exatcly where we want them
\usepackage{float}
% makes figure font bold
\usepackage{caption}
\captionsetup[figure]{labelfont=bf}
% For text generation
\usepackage{lipsum}
% For drawing
\usepackage{tikz}
% For smaller or equal sign and not divide sign
\usepackage{amssymb}
% For the diagonal fraction
\usepackage{xfrac}
% For enumerating exercise parts with letters instead of numbers
\usepackage{enumitem}
% For dfrac, which forces the fraction to be in display mode (large) e
% even in math mode (small)
\usepackage{amsmath}
% For degree sign
\usepackage{gensymb}
% For "\mathbb" macro
\usepackage{amsfonts}
% For positioning 
\usepackage{indentfirst}
\usetikzlibrary{shapes,positioning,fit,calc}
% for adjustwidth environment
\usepackage{changepage}
% for arrow on top
\usepackage{esvect}
% for mathbb 1
\usepackage{bbm}
% for mathsrc
\usepackage[mathscr]{eucal}
% For degree sign
\usepackage{gensymb}
% For quotes
\usepackage{csquotes}
% For vertical lines
\usepackage{mathtools}
% For cols
\usepackage{multicol}

% for tikz
\usepackage{pgfplots}
\pgfplotsset{compat=1.18}
\usepackage{amsmath}
\usepgfplotslibrary{groupplots}


%===============================================================================
%==================================== FONTS ====================================
%===============================================================================


% Mathcal
\newcommand{\acal}{\mathcal{A}}
\newcommand{\bcal}{\mathcal{B}}
\newcommand{\ccal}{\mathcal{C}}
\newcommand{\dcal}{\mathcal{D}}
\newcommand{\ecal}{\mathcal{E}}
\newcommand{\fcal}{\mathcal{F}}
\newcommand{\gcal}{\mathcal{G}}
\newcommand{\hcal}{\mathcal{H}}
\newcommand{\ical}{\mathcal{I}}
\newcommand{\jcal}{\mathcal{J}}
\newcommand{\kcal}{\mathcal{K}}
\newcommand{\lcal}{\mathcal{L}}
\newcommand{\mcal}{\mathcal{M}}
\newcommand{\ncal}{\mathcal{N}}
\newcommand{\ocal}{\mathcal{O}}
\newcommand{\pcal}{\mathcal{P}}
\newcommand{\qcal}{\mathcal{Q}}
\newcommand{\rcal}{\mathcal{R}}
\newcommand{\scal}{\mathcal{S}}
\newcommand{\tcal}{\mathcal{T}}
\newcommand{\ucal}{\mathcal{U}}
\newcommand{\vcal}{\mathcal{V}}
\newcommand{\wcal}{\mathcal{W}}
\newcommand{\xcal}{\mathcal{X}}
\newcommand{\ycal}{\mathcal{Y}}
\newcommand{\zcal}{\mathcal{Z}}

% Mathfrak
\newcommand{\afrak}{\mathfrak{A}}
\newcommand{\bfrak}{\mathfrak{B}}
\newcommand{\cfrak}{\mathfrak{C}}
\newcommand{\dfrak}{\mathfrak{D}}
\newcommand{\efrak}{\mathfrak{E}}
\newcommand{\ffrak}{\mathfrak{F}}
\newcommand{\gfrak}{\mathfrak{G}}
\newcommand{\hfrak}{\mathfrak{H}}
\newcommand{\ifrak}{\mathfrak{I}}
\newcommand{\jfrak}{\mathfrak{J}}
\newcommand{\kfrak}{\mathfrak{K}}
\newcommand{\lfrak}{\mathfrak{L}}
\newcommand{\mfrak}{\mathfrak{M}}
\newcommand{\nfrak}{\mathfrak{N}}
\newcommand{\ofrak}{\mathfrak{O}}
\newcommand{\pfrak}{\mathfrak{P}}
\newcommand{\qfrak}{\mathfrak{Q}}
\newcommand{\rfrak}{\mathfrak{R}}
\newcommand{\sfrak}{\mathfrak{S}}
\newcommand{\tfrak}{\mathfrak{T}}
\newcommand{\ufrak}{\mathfrak{U}}
\newcommand{\vfrak}{\mathfrak{V}}
\newcommand{\wfrak}{\mathfrak{W}}
\newcommand{\xfrak}{\mathfrak{X}}
\newcommand{\yfrak}{\mathfrak{Y}}
\newcommand{\zfrak}{\mathfrak{Z}}

% Mathscr
\newcommand{\ascr}{\mathscr{A}}
\newcommand{\bscr}{\mathscr{B}}
\newcommand{\cscr}{\mathscr{C}}
\newcommand{\dscr}{\mathscr{D}}
\newcommand{\escr}{\mathscr{E}}
\newcommand{\fscr}{\mathscr{F}}
\newcommand{\gscr}{\mathscr{G}}
\newcommand{\hscr}{\mathscr{H}}
\newcommand{\iscr}{\mathscr{I}}
\newcommand{\jscr}{\mathscr{J}}
\newcommand{\kscr}{\mathscr{K}}
\newcommand{\lscr}{\mathscr{L}}
\newcommand{\mscr}{\mathscr{M}}
\newcommand{\nscr}{\mathscr{N}}
\newcommand{\oscr}{\mathscr{O}}
\newcommand{\pscr}{\mathscr{P}}
\newcommand{\qscr}{\mathscr{Q}}
\newcommand{\rscr}{\mathscr{R}}
\newcommand{\sscr}{\mathscr{S}}
\newcommand{\tscr}{\mathscr{T}}
\newcommand{\uscr}{\mathscr{U}}
\newcommand{\vscr}{\mathscr{V}}
\newcommand{\wscr}{\mathscr{W}}
\newcommand{\xscr}{\mathscr{X}}
\newcommand{\yscr}{\mathscr{Y}}
\newcommand{\zscr}{\mathscr{Z}}

% Mathbb
\newcommand{\abb}{\mathbb{A}}
\newcommand{\bbb}{\mathbb{B}}
\newcommand{\cbb}{\mathbb{C}}
\newcommand{\dbb}{\mathbb{D}}
\newcommand{\ebb}{\mathbb{E}}
\newcommand{\fbb}{\mathbb{F}}
\newcommand{\gbb}{\mathbb{G}}
\newcommand{\hbb}{\mathbb{H}}
\newcommand{\ibb}{\mathbb{I}}
\newcommand{\jbb}{\mathbb{J}}
\newcommand{\kbb}{\mathbb{K}}
\newcommand{\lbb}{\mathbb{L}}
\newcommand{\mbb}{\mathbb{M}}
\newcommand{\nbb}{\mathbb{N}}
\newcommand{\obb}{\mathbb{O}}
\newcommand{\pbb}{\mathbb{P}}
\newcommand{\qbb}{\mathbb{Q}}
\newcommand{\rbb}{\mathbb{R}}
\newcommand{\sbb}{\mathbb{S}}
\newcommand{\tbb}{\mathbb{T}}
\newcommand{\ubb}{\mathbb{U}}
\newcommand{\vbb}{\mathbb{V}}
\newcommand{\wbb}{\mathbb{W}}
\newcommand{\xbb}{\mathbb{X}}
\newcommand{\ybb}{\mathbb{Y}}
\newcommand{\zbb}{\mathbb{Z}}


%===============================================================================
%=============================== SPECIAL SYMBOLS ===============================
%===============================================================================


% Orbit (group theory)
\newcommand{\orbit}{\mathcal{O}}
% Normal group
\newcommand{\normal}{\mathcal{N}}
% Indicator function
\newcommand{\indicator}{\mathbbm{1}}
% Laplace transform
\newcommand{\laplace}[1]{\mathcal{L}}
% Epsilon shorthand
\newcommand{\eps}{\varepsilon}
% Omega
\newcommand{\om}{\omega}
\newcommand{\Om}{\Omega}

%===============================================================================
%================================== OPERATORS ==================================
%===============================================================================


% Inverse exponent
\newcommand{\inv}[0]{^{-1}}
% Overline bar
\newcommand{\olsi}[1]{\,\overline{\!{#1}}}
% Less than or equal slanted
\newcommand{\seqs}{\leqslant}
% Greater or equal slanted
\newcommand{\geqs}{\geqslant}
% Subset or equal
\newcommand{\sub}{\subseteq}
% Proper subset
\newcommand{\prosub}{\subset}
% from
\newcommand{\from}{\leftarrow}

% Parantheses
\newcommand{\para}[1]{\left( #1 \right)}
% Curly Braces
\newcommand{\curl}[1]{\left\{ #1 \right\}}
% Brackets
\newcommand{\brac}[1]{\left[ #1 \right]}
% Angled Brackets
\newcommand{\ang}[1]{\left\langle #1 \right\rangle}
% Norm
\newcommand{\norm}[1]{\left\| #1 \right\|}

% Piece wise (use \\ between cases)
\newcommand{\piecewise}[1]{\begin{cases} #1 \end{cases}}

% Bold symbol shorthand
\newcommand{\bl}{\boldsymbol}

% Vertical space
\newcommand{\vs}[1]{\vspace{#1 pt}}
% Horizontal ertical space
\newcommand{\hs}[1]{\hspace{#1 pt}}


%===============================================================================
%============================== TEXT BASED SYMBOLS =============================
%===============================================================================


% Radians
\newcommand{\rad}{\text{rad}}
% Least Common Multiple
\newcommand{\lcm}{\text{lcm}}
% Automorphism
\newcommand{\Aut}{\text{Aut}}
% Variance
\newcommand{\var}{\text{Var}}
% Covariance
\newcommand{\cov}{\text{Cov}}
% Cofactor (matrix)
\newcommand{\cof}{\text{Cof}}
% Adjugate (matrix)
\newcommand{\adj}{\text{Adj}}
% Trace (matrix)
\newcommand{\tr}{\text{tr}}
% Standard deviation
\newcommand{\std}{\text{Std}}
% Correlation coefficient
\newcommand{\corr}{\text{Corr}}
% Sign
\newcommand{\sign}{\text{sign}}

% And text
\newcommand{\AND}{\text{ and }}
% Or text 
\newcommand{\OR}{\text{ or }}
% For text 
\newcommand{\FOR}{\text{ for }}
% If text
\newcommand{\IF}{\text{ if }}
% When text
\newcommand{\WHEN}{\text{ when }}
% Where text
\newcommand{\WHERE}{\text{ where }}
% Then text
\newcommand{\THEN}{\text{ then }}
% Such that text
\newcommand{\SUCHTHAT}{\text{ such that }}

% 1st
\newcommand{\st}[1]{#1^{\text{st}}}
% 2nd
\newcommand{\nd}[1]{#1^{\text{nd}}}
% 3rd
\newcommand{\rd}[1]{#1^{\text{rd}}}
% nth
\newcommand{\nth}[1]{#1^{\text{th}}}


%===============================================================================
%========================= PROBABILITY AND STATISTICS ==========================
%===============================================================================


% Permutation
\newcommand{\perm}[2]{{}^{#1}\!P_{#2}}
% Combination
\newcommand{\comb}[2]{{}^{#1}C_{#2}}

% Baye's risk
\newcommand{\risk}[1]{\mathscr{R}_{#1}}
% Baye's optimal risk
\newcommand{\riskOptimal}[1]{\mathscr{R}_{#1}^*}
% Baye's empirical risk
\newcommand{\riskEmpirical}[2]{\hat{\mathscr{R}}_{#1}^{#2}}


%===============================================================================
%=================================== CALCULUS ==================================
%===============================================================================


% d over d derivative
\newcommand{\dd}[2]{\dfrac{d#1}{d#2}}
% partial d over d derivative
\newcommand{\partialdd}[2]{\dfrac{\partial #1}{\partial #2}}
% delta d over d derivative
\newcommand{\deltadd}[2]{\dfrac{\Delta #1}{\Delta #2}}

% Integration between a and b
\newcommand{\integral}[4]{\int_{#1}^{#2} #3 \, #4}
% Integration in some space 
\newcommand{\boundIntegral}[2]{\int_{#1} #2 \, d#1}

% Limit
\newcommand{\limit}[3]{\lim_{#1 \to #2} #3}


%===============================================================================
%================================  BIG SYMBOLS  ================================
%===============================================================================


% Sum
\newcommand{\sumof}[2]{\sum_{#1}^{#2}}
% Product
\newcommand{\productof}[2]{\prod_{#1}^{#2}}
% Union
\newcommand{\unionof}[2]{\bigcup_{#1}^{#2}}
% Intersection
\newcommand{\intersectionof}[2]{\bigcap_{#1}^{#2}}
% Or
\newcommand{\orof}[2]{\bigvee_{#1}^{#2}}
% And
\newcommand{\andof}[2]{\bigwedge_{#1}^{#2}}


%===============================================================================
%=============================== LINEAR ALGEBRA ================================
%===============================================================================


% Bold vector arrow
\newcommand{\bv}[1]{\vec{\mathbf{#1}}}

% Matrix or vector (use // for column, & for row) 
% with brackets
\newcommand{\bmat}[1]{\begin{bmatrix} #1 \end{bmatrix}}
% Matrix or vector (use // for column, & for row) 
% with curved brackets
\newcommand{\pmat}[1]{\begin{pmatrix} #1 \end{pmatrix}}
% Matrix or vector (use // for column, & for row) 
% with lines on either side 
\newcommand{\lmat}[1]{\begin{vmatrix} #1 \end{vmatrix}}
% Matrix or vector (use // for column, & for row) 
% with curly braces
\newcommand{\cmat}[1]{\begin{Bmatrix} #1 \end{Bmatrix}}
% Matrix or vector (use // for column, & for row) 
% with no braces
\newcommand{\mat}[1]{\begin{matrix} #1 \end{matrix}}


%===============================================================================
%================================ LARGE OBJECTS ================================
%===============================================================================

% Multiple lines
\newcommand{\multiline}[1]{
\begin{align*}
    #1
\end{align*}
}

% Multiple lines with equation numbers
\newcommand{\eqmultiline}[1]{
\begin{align*}
    #1
\end{align*}
}

% Color
\newcommand{\colorText}[2]{
\begingroup
\color{#1}
    #2
\endgroup
}

% Centered figure
\newcommand{\centerFigure}[2]{
    \begin{figure}[h]
        \centering
            #1
        \caption{#2}
    \end{figure}
}

% Tikz figure
\newcommand{\tikzGraphic}[1]{
    \begin{center}
    \begin{tikzpicture}
        #1
    \end{tikzpicture}
    \end{center}
}

% Enumerate numbers (seperate by \item)
\newcommand{\numbers}{\textbf{\number*)}}

% Enumerate letters (seperate by \item)
\newcommand{\letters}{\textbf{\alph*)}}

\title{%
    \Huge Abstract Algebra \\
    \large by \\
    \Large Dummit and Foote \\~\\
    \huge Part 1: Group Theory \\
    \LARGE Chapter 1: Introduction to Groups \\
    \Large Section 6: Homomorphisms and Isomorphisms
}
\date{2024-03-17}
\author{Michael Saba}

\begin{document}
    \pagenumbering{gobble}
    \maketitle
    \newpage
    \setlength{\parindent}{0pt}
    \pagenumbering{arabic}


    In group theory, we don't usually care what a group
    \textit{looks like}, meaning what the set of elements is,
    and what the binary operator is.
    A group is instead defined entirely by its group theoretic structure;
    by studying the way elements interact with each other,
    we can abstract away the details of which elements these are,
    and instead study the structure itself,
    which can be generalized to any groups that share it.
    As such, two groups are considered the same if they have the
    same structure,
    the way the elements are combined and map to each other,
    even if they look different.
    This section will formally define this idea of two groups being
    the same. \\

    \subsection*{Homomorphisms and Isomorphisms}

    Before defining an Isomorphism,
    we will look at the definition of a Homomorphism,
    which is a weaker equivalency relationship. \\
    Suppose we have two groups $(G, \cdot)$ and $(H, \star)$.
    A mapping $\varphi: G \rightarrow H$ such that
    \[ \varphi(x \cdot y) = \varphi(x) \star \varphi(y) 
    \qquad \forall x, y \in G \]
    Is called a \textbf{homomorphism}.
    Two groups $G$ and $H$ are said to be homomorphic
    if such a mapping exists, either rom $G$ to $H$ or $H$ to $G$. \\

    Let's actually examine what this means.
    If a function maps $x$ and $y$ in $G$
    to $\varphi(x)$ and $\varphi(y)$ in $H$,
    and also maps their product $x \cdot y$ in $G$
    to the product of their mappings $\varphi(x) \star \varphi(y)$ in $H$,
    then that means we found a way
    to preserve the relationship of $x$ and $y$
    in its mapping.
    If we were to be able to do that, with the same mapping,
    to all pairs of elements in $G$,
    that would mean that the entire structure of $G$,
    the web of connections of elements,
    is also found in $H$,
    through some mapping of the elements between the two. \\
    This is what we meant earlier when we said that two groups
    can be the same;
    the structure of the two groups is the same,
    but the elements are different;
    the equivalency can be established by finding
    the elements in $H$ that have the same structure as those in $G$. \\

    However, note that homomorphic groups are not the same.
    as we said earlier, they are weaker than that.
    Since $\varphi$ is well defined,
    each element of $G$ will have a mapping,
    but not all elements of $H$ necessarily have a pre-image.
    Likewise, some elements of $H$ can have more than one pre-image,
    so a several elements in $G$ may structurally be equivalent
    to the same element in $H$. \\
    So instead of saying that homomorphic groups have the same structure
    (are the same),
    it's more accurate to say that
    $H$ contains the structure of group $G$
    (that a subset of $H$ is the same as $G$). \\

    On the other hand, an \textbf{isomorphism}
    is a mapping $\varphi: G \rightarrow H$ such that
    \begin{itemize}[label=$\diamond$]
        \item 
            The mapping $\varphi$ is a homomorphism.
        \item 
            The mapping is bijective.
    \end{itemize}
    If $H$ is isomorphic to $G$, then it is also homomorphic to it,
    so as stated earlier, the structure of $G$
    will be contained in $H$.
    However, since the mapping is now bijective,
    every element of $G$ will have a single guaranteed
    structural equivalent in $H$.
    So the group operation (structure) is preserved,
    and the mapping between elements is one-to-one,
    meaning that two isomorphic groups are the same,
    even if they are written differently. \\
    To show two groups are the same,
    we just need to find one bijective mapping
    that preserves the group structure. \\
    We write $G \cong H$ when two groups are isomorphic. \\

    Trivially, any group $G$ is isomorphic to itself,
    through the identity mapping. \\

    \textbf{Classification theorems} in Group, Ring and Field theory,
    are the class of theorems that attempt to show that
    algebraic structures are the same if they possess some
    property in common. \\
    One such theorem states that
    \[ \text{Any non-abelian group of order $3$ is isomorphic to $S_3$} \]
    From this theorem we obtain that $D_6 \cong S_3$,
    which makes sense when we think about it;
    the dihedral group $D_6$ has order $2 \times 3 = 6$,
    which means that it permutes the vertices of an equilateral triangle
    oin $6$ different ways.
    There are however,
    only a total of $6$ permutations of $3$ elements ($3!$),
    so all permutations of $S_3$
    must correspond to the permutations of $D_6$.
    Furthermore, since both groups have the composition of permutations
    as their group operation,
    the combination of the elements (structure) is the same,
    meaning the two groups are isomorphic. \\
    Likewise, $GL_2(\mathbb{F}_2) \cong S_3$. \\

    To see that two groups are not isomorphic,
    it can usually be easier than proving they are isomorphic.
    If two groups $G$ and $H$ are isomorphic,
    then they have some properties in common;
    if they don't, we can immediatly conclude they aren't isomorphic:
    \begin{itemize}[label=$\diamond$]
        \item 
            $|G| = |H|$,
            which is obviously true since the isomorphism mapping
            must be bijective.
        \item 
            Either both or neither are abelian.
            This is because they must share the same structure.
        \item 
            If $\varphi$ is the isomorphism,
            then for all $g \in G$,
            $|g| = |\varphi(g)|$.
    \end{itemize} 
    For example, $S_3$ and $\zbb/n\zbb$ are not isomorphic,
    as the latter is abelian and the former is not. \\


    \subsection*{Generators and Relations}

    We will not prove this fact now,
    but given the presentation of two groups,
    we can show that they are isomorphic. \\
    Suppose that $G$ is a finite group of order $n$
    which has a set of generators $S = \{ s_1, s_2 \dots s_n \}$. \\
    Suppose that $H$ is another group
    with elements $\{ r_1, r_2 \dots r_n\}$. \\
    If all relations in $G$ can be satisfied in $H$
    by replacing each $s_i$ with $r_i$,
    then there is a unique homomorphism $\varphi: G \to H$
    which maps $s_i$ to $r_i$.
    We only need to check the relations of the presentation,
    as they give us all other relations. \\
    Intuitively, this makes sense;
    if the generators of $G$ have mappings in $H$
    that satisfy the relations,
    then the structure of $G$ can be found
    in those elements of $H$,
    so $H$ contains a copy of $G$ in it. \\
    If the set $\{ r_1, r_2 \dots r_n\}$ generates $H$,
    the the mapping $\varphi$ is surjective;
    if each generator in $H$ has a pre-image in $G$,
    then all elements in $H$ must be the image of one or more
    elements in $G$ (this is proven later). 
    So if we can show that $|G| = |H|$,
    the surjection automatically becomes a bijection,
    and $\varphi$ becomes an isomorphism. \\

    This won't be proven now and is only recorded for later,
    when we introduce the notion of vector spaces. \\
    Suppose we a finite-dimensional vector space $V$ of dimension $n$
    and with a basis $S$.
    Also take another vector space $W$.
    We can define a linear transformation from $V$ to $W$
    by mapping the elements of $S$ to arbitrary vectors in $W$.
    If $W$ is also of dimension $n$,
    and the chosen vectors span $W$ (making them its basis),
    then this linear transformation is invertible,
    making it a vector space isomorphism. \\
    As we can see, the method follows a very similar structure
    to that of the generators and relations. \\

   
\end{document}