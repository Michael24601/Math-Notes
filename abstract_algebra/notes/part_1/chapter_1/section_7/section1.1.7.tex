  
\documentclass[12pt]{article}
\usepackage[margin=1in]{geometry}


%===============================================================================
%================================== PACKAGES ===================================
%===============================================================================

% For using float option H that places figures 
% exatcly where we want them
\usepackage{float}
% makes figure font bold
\usepackage{caption}
\captionsetup[figure]{labelfont=bf}
% For text generation
\usepackage{lipsum}
% For drawing
\usepackage{tikz}
% For smaller or equal sign and not divide sign
\usepackage{amssymb}
% For the diagonal fraction
\usepackage{xfrac}
% For enumerating exercise parts with letters instead of numbers
\usepackage{enumitem}
% For dfrac, which forces the fraction to be in display mode (large) e
% even in math mode (small)
\usepackage{amsmath}
% For degree sign
\usepackage{gensymb}
% For "\mathbb" macro
\usepackage{amsfonts}
% For positioning 
\usepackage{indentfirst}
\usetikzlibrary{shapes,positioning,fit,calc}
% for adjustwidth environment
\usepackage{changepage}
% for arrow on top
\usepackage{esvect}
% for mathbb 1
\usepackage{bbm}
% for mathsrc
\usepackage[mathscr]{eucal}
% For degree sign
\usepackage{gensymb}
% For quotes
\usepackage{csquotes}
% For vertical lines
\usepackage{mathtools}
% For cols
\usepackage{multicol}

% for tikz
\usepackage{pgfplots}
\pgfplotsset{compat=1.18}
\usepackage{amsmath}
\usepgfplotslibrary{groupplots}


%===============================================================================
%==================================== FONTS ====================================
%===============================================================================


% Mathcal
\newcommand{\acal}{\mathcal{A}}
\newcommand{\bcal}{\mathcal{B}}
\newcommand{\ccal}{\mathcal{C}}
\newcommand{\dcal}{\mathcal{D}}
\newcommand{\ecal}{\mathcal{E}}
\newcommand{\fcal}{\mathcal{F}}
\newcommand{\gcal}{\mathcal{G}}
\newcommand{\hcal}{\mathcal{H}}
\newcommand{\ical}{\mathcal{I}}
\newcommand{\jcal}{\mathcal{J}}
\newcommand{\kcal}{\mathcal{K}}
\newcommand{\lcal}{\mathcal{L}}
\newcommand{\mcal}{\mathcal{M}}
\newcommand{\ncal}{\mathcal{N}}
\newcommand{\ocal}{\mathcal{O}}
\newcommand{\pcal}{\mathcal{P}}
\newcommand{\qcal}{\mathcal{Q}}
\newcommand{\rcal}{\mathcal{R}}
\newcommand{\scal}{\mathcal{S}}
\newcommand{\tcal}{\mathcal{T}}
\newcommand{\ucal}{\mathcal{U}}
\newcommand{\vcal}{\mathcal{V}}
\newcommand{\wcal}{\mathcal{W}}
\newcommand{\xcal}{\mathcal{X}}
\newcommand{\ycal}{\mathcal{Y}}
\newcommand{\zcal}{\mathcal{Z}}

% Mathfrak
\newcommand{\afrak}{\mathfrak{A}}
\newcommand{\bfrak}{\mathfrak{B}}
\newcommand{\cfrak}{\mathfrak{C}}
\newcommand{\dfrak}{\mathfrak{D}}
\newcommand{\efrak}{\mathfrak{E}}
\newcommand{\ffrak}{\mathfrak{F}}
\newcommand{\gfrak}{\mathfrak{G}}
\newcommand{\hfrak}{\mathfrak{H}}
\newcommand{\ifrak}{\mathfrak{I}}
\newcommand{\jfrak}{\mathfrak{J}}
\newcommand{\kfrak}{\mathfrak{K}}
\newcommand{\lfrak}{\mathfrak{L}}
\newcommand{\mfrak}{\mathfrak{M}}
\newcommand{\nfrak}{\mathfrak{N}}
\newcommand{\ofrak}{\mathfrak{O}}
\newcommand{\pfrak}{\mathfrak{P}}
\newcommand{\qfrak}{\mathfrak{Q}}
\newcommand{\rfrak}{\mathfrak{R}}
\newcommand{\sfrak}{\mathfrak{S}}
\newcommand{\tfrak}{\mathfrak{T}}
\newcommand{\ufrak}{\mathfrak{U}}
\newcommand{\vfrak}{\mathfrak{V}}
\newcommand{\wfrak}{\mathfrak{W}}
\newcommand{\xfrak}{\mathfrak{X}}
\newcommand{\yfrak}{\mathfrak{Y}}
\newcommand{\zfrak}{\mathfrak{Z}}

% Mathscr
\newcommand{\ascr}{\mathscr{A}}
\newcommand{\bscr}{\mathscr{B}}
\newcommand{\cscr}{\mathscr{C}}
\newcommand{\dscr}{\mathscr{D}}
\newcommand{\escr}{\mathscr{E}}
\newcommand{\fscr}{\mathscr{F}}
\newcommand{\gscr}{\mathscr{G}}
\newcommand{\hscr}{\mathscr{H}}
\newcommand{\iscr}{\mathscr{I}}
\newcommand{\jscr}{\mathscr{J}}
\newcommand{\kscr}{\mathscr{K}}
\newcommand{\lscr}{\mathscr{L}}
\newcommand{\mscr}{\mathscr{M}}
\newcommand{\nscr}{\mathscr{N}}
\newcommand{\oscr}{\mathscr{O}}
\newcommand{\pscr}{\mathscr{P}}
\newcommand{\qscr}{\mathscr{Q}}
\newcommand{\rscr}{\mathscr{R}}
\newcommand{\sscr}{\mathscr{S}}
\newcommand{\tscr}{\mathscr{T}}
\newcommand{\uscr}{\mathscr{U}}
\newcommand{\vscr}{\mathscr{V}}
\newcommand{\wscr}{\mathscr{W}}
\newcommand{\xscr}{\mathscr{X}}
\newcommand{\yscr}{\mathscr{Y}}
\newcommand{\zscr}{\mathscr{Z}}

% Mathbb
\newcommand{\abb}{\mathbb{A}}
\newcommand{\bbb}{\mathbb{B}}
\newcommand{\cbb}{\mathbb{C}}
\newcommand{\dbb}{\mathbb{D}}
\newcommand{\ebb}{\mathbb{E}}
\newcommand{\fbb}{\mathbb{F}}
\newcommand{\gbb}{\mathbb{G}}
\newcommand{\hbb}{\mathbb{H}}
\newcommand{\ibb}{\mathbb{I}}
\newcommand{\jbb}{\mathbb{J}}
\newcommand{\kbb}{\mathbb{K}}
\newcommand{\lbb}{\mathbb{L}}
\newcommand{\mbb}{\mathbb{M}}
\newcommand{\nbb}{\mathbb{N}}
\newcommand{\obb}{\mathbb{O}}
\newcommand{\pbb}{\mathbb{P}}
\newcommand{\qbb}{\mathbb{Q}}
\newcommand{\rbb}{\mathbb{R}}
\newcommand{\sbb}{\mathbb{S}}
\newcommand{\tbb}{\mathbb{T}}
\newcommand{\ubb}{\mathbb{U}}
\newcommand{\vbb}{\mathbb{V}}
\newcommand{\wbb}{\mathbb{W}}
\newcommand{\xbb}{\mathbb{X}}
\newcommand{\ybb}{\mathbb{Y}}
\newcommand{\zbb}{\mathbb{Z}}


%===============================================================================
%=============================== SPECIAL SYMBOLS ===============================
%===============================================================================


% Orbit (group theory)
\newcommand{\orbit}{\mathcal{O}}
% Normal group
\newcommand{\normal}{\mathcal{N}}
% Indicator function
\newcommand{\indicator}{\mathbbm{1}}
% Laplace transform
\newcommand{\laplace}[1]{\mathcal{L}}
% Epsilon shorthand
\newcommand{\eps}{\varepsilon}
% Omega
\newcommand{\om}{\omega}
\newcommand{\Om}{\Omega}

%===============================================================================
%================================== OPERATORS ==================================
%===============================================================================


% Inverse exponent
\newcommand{\inv}[0]{^{-1}}
% Overline bar
\newcommand{\olsi}[1]{\,\overline{\!{#1}}}
% Less than or equal slanted
\newcommand{\seqs}{\leqslant}
% Greater or equal slanted
\newcommand{\geqs}{\geqslant}
% Subset or equal
\newcommand{\sub}{\subseteq}
% Proper subset
\newcommand{\prosub}{\subset}
% from
\newcommand{\from}{\leftarrow}

% Parantheses
\newcommand{\para}[1]{\left( #1 \right)}
% Curly Braces
\newcommand{\curl}[1]{\left\{ #1 \right\}}
% Brackets
\newcommand{\brac}[1]{\left[ #1 \right]}
% Angled Brackets
\newcommand{\ang}[1]{\left\langle #1 \right\rangle}
% Norm
\newcommand{\norm}[1]{\left\| #1 \right\|}

% Piece wise (use \\ between cases)
\newcommand{\piecewise}[1]{\begin{cases} #1 \end{cases}}

% Bold symbol shorthand
\newcommand{\bl}{\boldsymbol}

% Vertical space
\newcommand{\vs}[1]{\vspace{#1 pt}}
% Horizontal ertical space
\newcommand{\hs}[1]{\hspace{#1 pt}}


%===============================================================================
%============================== TEXT BASED SYMBOLS =============================
%===============================================================================


% Radians
\newcommand{\rad}{\text{rad}}
% Least Common Multiple
\newcommand{\lcm}{\text{lcm}}
% Automorphism
\newcommand{\Aut}{\text{Aut}}
% Variance
\newcommand{\var}{\text{Var}}
% Covariance
\newcommand{\cov}{\text{Cov}}
% Cofactor (matrix)
\newcommand{\cof}{\text{Cof}}
% Adjugate (matrix)
\newcommand{\adj}{\text{Adj}}
% Trace (matrix)
\newcommand{\tr}{\text{tr}}
% Standard deviation
\newcommand{\std}{\text{Std}}
% Correlation coefficient
\newcommand{\corr}{\text{Corr}}
% Sign
\newcommand{\sign}{\text{sign}}

% And text
\newcommand{\AND}{\text{ and }}
% Or text 
\newcommand{\OR}{\text{ or }}
% For text 
\newcommand{\FOR}{\text{ for }}
% If text
\newcommand{\IF}{\text{ if }}
% When text
\newcommand{\WHEN}{\text{ when }}
% Where text
\newcommand{\WHERE}{\text{ where }}
% Then text
\newcommand{\THEN}{\text{ then }}
% Such that text
\newcommand{\SUCHTHAT}{\text{ such that }}

% 1st
\newcommand{\st}[1]{#1^{\text{st}}}
% 2nd
\newcommand{\nd}[1]{#1^{\text{nd}}}
% 3rd
\newcommand{\rd}[1]{#1^{\text{rd}}}
% nth
\newcommand{\nth}[1]{#1^{\text{th}}}


%===============================================================================
%========================= PROBABILITY AND STATISTICS ==========================
%===============================================================================


% Permutation
\newcommand{\perm}[2]{{}^{#1}\!P_{#2}}
% Combination
\newcommand{\comb}[2]{{}^{#1}C_{#2}}

% Baye's risk
\newcommand{\risk}[1]{\mathscr{R}_{#1}}
% Baye's optimal risk
\newcommand{\riskOptimal}[1]{\mathscr{R}_{#1}^*}
% Baye's empirical risk
\newcommand{\riskEmpirical}[2]{\hat{\mathscr{R}}_{#1}^{#2}}


%===============================================================================
%=================================== CALCULUS ==================================
%===============================================================================


% d over d derivative
\newcommand{\dd}[2]{\dfrac{d#1}{d#2}}
% partial d over d derivative
\newcommand{\partialdd}[2]{\dfrac{\partial #1}{\partial #2}}
% delta d over d derivative
\newcommand{\deltadd}[2]{\dfrac{\Delta #1}{\Delta #2}}

% Integration between a and b
\newcommand{\integral}[4]{\int_{#1}^{#2} #3 \, #4}
% Integration in some space 
\newcommand{\boundIntegral}[2]{\int_{#1} #2 \, d#1}

% Limit
\newcommand{\limit}[3]{\lim_{#1 \to #2} #3}


%===============================================================================
%================================  BIG SYMBOLS  ================================
%===============================================================================


% Sum
\newcommand{\sumof}[2]{\sum_{#1}^{#2}}
% Product
\newcommand{\productof}[2]{\prod_{#1}^{#2}}
% Union
\newcommand{\unionof}[2]{\bigcup_{#1}^{#2}}
% Intersection
\newcommand{\intersectionof}[2]{\bigcap_{#1}^{#2}}
% Or
\newcommand{\orof}[2]{\bigvee_{#1}^{#2}}
% And
\newcommand{\andof}[2]{\bigwedge_{#1}^{#2}}


%===============================================================================
%=============================== LINEAR ALGEBRA ================================
%===============================================================================


% Bold vector arrow
\newcommand{\bv}[1]{\vec{\mathbf{#1}}}

% Matrix or vector (use // for column, & for row) 
% with brackets
\newcommand{\bmat}[1]{\begin{bmatrix} #1 \end{bmatrix}}
% Matrix or vector (use // for column, & for row) 
% with curved brackets
\newcommand{\pmat}[1]{\begin{pmatrix} #1 \end{pmatrix}}
% Matrix or vector (use // for column, & for row) 
% with lines on either side 
\newcommand{\lmat}[1]{\begin{vmatrix} #1 \end{vmatrix}}
% Matrix or vector (use // for column, & for row) 
% with curly braces
\newcommand{\cmat}[1]{\begin{Bmatrix} #1 \end{Bmatrix}}
% Matrix or vector (use // for column, & for row) 
% with no braces
\newcommand{\mat}[1]{\begin{matrix} #1 \end{matrix}}


%===============================================================================
%================================ LARGE OBJECTS ================================
%===============================================================================

% Multiple lines
\newcommand{\multiline}[1]{
\begin{align*}
    #1
\end{align*}
}

% Multiple lines with equation numbers
\newcommand{\eqmultiline}[1]{
\begin{align*}
    #1
\end{align*}
}

% Color
\newcommand{\colorText}[2]{
\begingroup
\color{#1}
    #2
\endgroup
}

% Centered figure
\newcommand{\centerFigure}[2]{
    \begin{figure}[h]
        \centering
            #1
        \caption{#2}
    \end{figure}
}

% Tikz figure
\newcommand{\tikzGraphic}[1]{
    \begin{center}
    \begin{tikzpicture}
        #1
    \end{tikzpicture}
    \end{center}
}

% Enumerate numbers (seperate by \item)
\newcommand{\numbers}{\textbf{\number*)}}

% Enumerate letters (seperate by \item)
\newcommand{\letters}{\textbf{\alph*)}}

\title{%
    \Huge Abstract Algebra \\
    \large by \\
    \Large Dummit and Foote \\~\\
    \huge Part 1: Group Theory \\
    \LARGE Chapter 1: Introduction to Groups \\
    \Large Section 7: Group Actions
}
\date{2024-03-17}
\author{Michael Saba}

\begin{document}
    \pagenumbering{gobble}
    \maketitle
    \newpage
    \setlength{\parindent}{0pt}
    \pagenumbering{arabic}

    In section 1.1.6 we talked about isomorphisms,
    and how two groups can be the same so long as their structure,
    the way their elements interact, is the same,
    for some mapping of the elements of one group to another. \\
    But if groups are uniquely defined by their structure,
    why bother defining groups in terms of permutations ($S_n$),
    rigid motions ($D_{2n}$) and matrices $GL_n(F)$?
    Why even bother having unique symbols for elements of different
    groups if they are arbitrary? \\
    The answer is that, while groups are themselves no more than
    the association between their elements by the group operator,
    the groups sometimes perform actions on sets that can reveal
    a lot about their internal structure,
    and allow us to study them.
    For example, when we defined $D_{2n}$,
    we were able to learn its entire structure, order,
    and relations from no more than the action it performs on the vertices
    of regular polygons in $\R^3$. 
    Likewise, we learned about the groups $S_n$ from studying
    the actions it performs on the set of integers from $1$ to $n$.
    In fact, the group action was the way we defined those groups
    to begin with. \\
    So when we talked about the group of rigid motions on $n$-gons
    in $3$ dimensions,
    that wasn't the group we were reffering to,
    but the group action on the set of vertices;
    the group itself is something more abstract,
    the way the elements of the group are associated
    by the group operator.
    That's why two groups such as $S_3$ and $D_6$ can be the same,
    their elements interact identically,
    even though the group actions we used to define them are edifferent. \\
    This section formalizes this idea of group actions,
    and allows us to define group actions from groups onto sets. \\

    Formally, a \textbf{group action} from a group $G$
    to a set $A$ (which is non-empty) is a map from
    $G \times A$ to $A$
    (which we express as $g \cdot a = b$
    for all $a, b \in A$ and $g \in G$,
    which is distint from the group operation of $G$)
    satisfying the following properties:
    \begin{itemize}[label=$\diamond$]
        \item
            $g_1 \cdot (g_2 \cdot a) = (g_1g_2) \cdot a$
            for all $g_1, g_2 \in G$ and $a \in A$.
        \item 
            $1 \cdot a = a$ for all $a \in A$.
    \end{itemize}
    As we can see, this definition tells us that each element
    in the set can be acted on by each element of the group.
    The stipulation that $g_1 \cdot (g_2 \cdot a) = (g_1g_2)a$
    ensures that the way we combine elements in the group
    translates to the way the group actions transform the set,
    allowing the structure of the group to be reflected in
    its action on the set. 
    For instance if we combine two rigid motions on $D_{2n}$
    and apply their action on an polygon,
    it should have the same effect as applying the first action
    on the polygon, then applying the second on the transformed polygon.
    Basically, the combination of two group actions 
    should be equivalent to the combination of the group elements
    responsible for them,
    such that the action of their combination is
    the same as the combination of their actions. \\
    The second rule just ensures that the identity,
    which keeps all elements of the group intact,
    does not transform the set;
    this way its role in the group and as an action is the same;
    it is the identity. \\

    We can derive some properties of groups actions now.
    Suppose $G$ is a group acting on a set $A$,
    then for each fixed $g \in G$,
    we define a map $\sigma_g$ defined as:
    \[ \sigma_g: A \to A \]
    \[ \sigma_g(a) = g \cdot a \quad \forall a \in A \]
    This mapping is a permutation of $A$,
    meaning it is bijective.
    We still need to prove it is bijective,
    and we can do this by showing that $\sigma_g$
    has a two sides inverse for any fixed $g \in G$. \\
    Suppose that $(\sigma_g)^{-1}$ exists,
    we will show that it is $\sigma_{g^{-1}}$.
    For all $a \in A$:
    \[ (\sigma_{g^{-1}} \circ \sigma_g) \cdot a \]
    \[ = \sigma_{g^{-1}} \cdot (\sigma_g \cdot a) \]
    \[ = g^{-1} \cdot (g \cdot a) \]
    \[ = (g^{-1}g) \cdot a \]
    \[ = 1 \cdot a = a \]
    Likewise
    \[ (\sigma_g \circ \sigma_{g^{-1}}) \cdot a \]
    \[ = \sigma_g \cdot (\sigma_{g^{-1}} \cdot a) \]
    \[ = g \cdot (g^{-1} \cdot a) \]
    \[ = (gg^{-1}) \cdot a \]
    \[ = 1 \cdot a = a \]
    This completes the proof that $\sigma_{g^{-1}}$ is
    the two sided inverse,
    and that by extension, $\sigma_g$ is bijective. \\

    Note that when we defined a group action,
    it may have seemed that the first condition
    implies the second to be true,
    since $1 \cdot (g \cdot a) = (1g) \cdot a = g \cdot a$
    for all $a \in A$ and all $g \in G$.
    However, this only applies for elements in $A$
    that can be written as $g \cdot a$
    for some $g \in G$ and $a \in A$. \\
    From the proof that $\sigma_g$ is bijective,
    we now know that any element in $A$ can be written as $g \cdot a$
    for any fixed $g \in G$.
    However, the proof we used to show that required
    us to use the property that $1 \cdot a = a$
    in the last line,
    so depending on this property of $\sigma_g$
    to show that $1 \cdot a = a$ creates circular logic.
    As such, we still need to explicitely state that
    $1 \cdot a = a$ in the group action definition,
    and can't infer it from the first condition. \\

    We can derive a second property of the mapping $\sigma_g$;
    since it is a permutation of $A$
    for any fixed $g \in G$,
    then $\sigma_a \in S_A$, the group of permutations of $A$. \\
    If so, the map from $\varphi: G \to S_a$
    defined by $g \mapsto \sigma_g$
    is a homomorphism. \\
    We can show this is true by using an arbitrary element
    $a \in A$,
    which will show that it is true for all elements of $A$.
    For $g_1, g_2 \in G$:
    \[ \varphi(g_1g_2)(a) \]
    \[ = \sigma_{g_1g_2}(a) \]
    \[ = (g_1g_2) \cdot a \]
    \[ = g_1 \cdot (g_2 \cdot a) \]
    \[ = \sigma_{g_1}(\sigma_{g_2}(a)) \]
    \[ = (\sigma_{g_1} \circ \sigma_{g_2})(a) \]
    This completes our proof that it is a homomorphism. \\
    This homomorphism is specific not to the group,
    but to the group action defined by $g \mapsto \sigma_g$,
    and is called the action's \textbf{permutation representation}. \\
    It makes sense that this is a homomorphism;
    it basically reinforces the idea that the group action's
    permutations of set elements ($\sigma_g$),
    and the combination of those actions,
    is consistent with the group's internal structure
    and binary operation. \\

    The permutation represetation process is reversible;
    meaning that for each homomorphism $\varphi: G \to S_A$,
    then the map from $G \times A$ to $A$ defined by
    $g \cdot a = \varphi(g)(a)$ for all $g \in G$ and $a \in A$
    satisfies the properties of a group action. \\
    Since all homomorphisms from $G$ to $S_A$ are group actions,
    and all group actions are themselves homomorphisms from $G$ to $S_A$,
    we can conclude that the two notions are the same,
    just expressed differently. \\

    If the homomorphism $\varphi: G \to S_a$ is injective,
    it means that each element of $g$ corresponds
    to a unique permutation of $A$.
    All homomorphisms from $G$ to $A$ correspond to group actions,
    and homomorphisms that are injective
    correspond to what we call \textbf{faithful} group actions,
    as each element of $g$ acts on the set uniquely. \\ 

    Note that there are \textbf{left actions} and \textbf{right actions},
    which depend on where the group element is placed
    in relation to the set element.
    So far we've only used left actions,
    but everything we said also applies to right actions. \\


\end{document}