
    
\documentclass[12pt]{article}

\usepackage[margin=1in]{geometry}

% For using float option H that places figures exatcly where we want them
\usepackage{float}
\usepackage{tcolorbox} % for framing
% makes figure font bold
\usepackage{caption}
\captionsetup[figure]{labelfont=bf}
% For text generation
\usepackage{lipsum}
% For drawing
\usepackage{tikz}
% For smaller or equal sign and not divide sign
\usepackage{amssymb}
% For the diagonal fraction
\usepackage{xfrac}
% For enumerating exercise parts with letters instead of numbers
\usepackage{enumitem}
% For dfrac, which forces the fraction to be in display mode (large) e
% even in math mode (small)
\usepackage{amsmath}
% For degree sign
\usepackage{gensymb}
% For "\mathbb" macro
\usepackage{amsfonts}
\newcommand{\N}{\mathbb{N}}
\newcommand{\Z}{\mathbb{Z}}
\newcommand{\Q}{\mathbb{Q}}
\newcommand{\R}{\mathbb{R}}
\newcommand{\C}{\mathbb{C}}

% overline short italic
\newcommand{\olsi}[1]{\,\overline{\!{#1}}}

\usepackage{indentfirst}
\usetikzlibrary{shapes,positioning,fit,calc}
\usetikzlibrary{arrows}

\usepackage{changepage} % for adjustwidth environment

\title{%
    \Huge Abstract Algebra \\
    \large by \\
    \Large Dummit and Foote \\~\\
    \huge Part 1: Group Theory \\
    \LARGE Chapter 1: Introduction to Groups \\
    \Large Section 5: The Quaternion Group
}
\date{2024-03-15}
\author{Michael Saba}

\begin{document}
    \pagenumbering{gobble}
    \maketitle
    \newpage
    \pagenumbering{arabic}


    The \textbf{quaternion group}, $Q_8$, is defined as
    \[ Q_8 = \{ 1, -1, i, -i, j, -j, k, -k \} \]
    With a binary operator $\cdot$ such that
    \[ 1 \cdot a = a \cdot 1 = a \qquad \forall a \in Q_8  \]
    \[ (-1)\cdot(-1) = 1 \qquad (-1)\cdot a = a\cdot (-1) = -a
    \qquad \forall a \in Q_8 \]
    \[ i \cdot i = j \cdot j = k \cdot k = -1 \]
    \[ i \cdot j = k \qquad j \cdot i = -k \]
    \[ j \cdot k = i \qquad k \cdot j = -i \]
    \[ k \cdot i = j \qquad i \cdot k = -j \]
    With that,
    it's clear $Q_8$ satisfies the axioms of a group,
    that it is non-abelian,
    and has order $8$. \\

    


\end{document}