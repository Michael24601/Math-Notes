
    
\documentclass[12pt]{article}
\usepackage[margin=1in]{geometry}


%===============================================================================
%================================== PACKAGES ===================================
%===============================================================================

% For using float option H that places figures 
% exatcly where we want them
\usepackage{float}
% makes figure font bold
\usepackage{caption}
\captionsetup[figure]{labelfont=bf}
% For text generation
\usepackage{lipsum}
% For drawing
\usepackage{tikz}
% For smaller or equal sign and not divide sign
\usepackage{amssymb}
% For the diagonal fraction
\usepackage{xfrac}
% For enumerating exercise parts with letters instead of numbers
\usepackage{enumitem}
% For dfrac, which forces the fraction to be in display mode (large) e
% even in math mode (small)
\usepackage{amsmath}
% For degree sign
\usepackage{gensymb}
% For "\mathbb" macro
\usepackage{amsfonts}
% For positioning 
\usepackage{indentfirst}
\usetikzlibrary{shapes,positioning,fit,calc}
% for adjustwidth environment
\usepackage{changepage}
% for arrow on top
\usepackage{esvect}
% for mathbb 1
\usepackage{bbm}
% for mathsrc
\usepackage[mathscr]{eucal}
% For degree sign
\usepackage{gensymb}
% For quotes
\usepackage{csquotes}
% For vertical lines
\usepackage{mathtools}
% For cols
\usepackage{multicol}

% for tikz
\usepackage{pgfplots}
\pgfplotsset{compat=1.18}
\usepackage{amsmath}
\usepgfplotslibrary{groupplots}


%===============================================================================
%==================================== FONTS ====================================
%===============================================================================


% Mathcal
\newcommand{\acal}{\mathcal{A}}
\newcommand{\bcal}{\mathcal{B}}
\newcommand{\ccal}{\mathcal{C}}
\newcommand{\dcal}{\mathcal{D}}
\newcommand{\ecal}{\mathcal{E}}
\newcommand{\fcal}{\mathcal{F}}
\newcommand{\gcal}{\mathcal{G}}
\newcommand{\hcal}{\mathcal{H}}
\newcommand{\ical}{\mathcal{I}}
\newcommand{\jcal}{\mathcal{J}}
\newcommand{\kcal}{\mathcal{K}}
\newcommand{\lcal}{\mathcal{L}}
\newcommand{\mcal}{\mathcal{M}}
\newcommand{\ncal}{\mathcal{N}}
\newcommand{\ocal}{\mathcal{O}}
\newcommand{\pcal}{\mathcal{P}}
\newcommand{\qcal}{\mathcal{Q}}
\newcommand{\rcal}{\mathcal{R}}
\newcommand{\scal}{\mathcal{S}}
\newcommand{\tcal}{\mathcal{T}}
\newcommand{\ucal}{\mathcal{U}}
\newcommand{\vcal}{\mathcal{V}}
\newcommand{\wcal}{\mathcal{W}}
\newcommand{\xcal}{\mathcal{X}}
\newcommand{\ycal}{\mathcal{Y}}
\newcommand{\zcal}{\mathcal{Z}}

% Mathfrak
\newcommand{\afrak}{\mathfrak{A}}
\newcommand{\bfrak}{\mathfrak{B}}
\newcommand{\cfrak}{\mathfrak{C}}
\newcommand{\dfrak}{\mathfrak{D}}
\newcommand{\efrak}{\mathfrak{E}}
\newcommand{\ffrak}{\mathfrak{F}}
\newcommand{\gfrak}{\mathfrak{G}}
\newcommand{\hfrak}{\mathfrak{H}}
\newcommand{\ifrak}{\mathfrak{I}}
\newcommand{\jfrak}{\mathfrak{J}}
\newcommand{\kfrak}{\mathfrak{K}}
\newcommand{\lfrak}{\mathfrak{L}}
\newcommand{\mfrak}{\mathfrak{M}}
\newcommand{\nfrak}{\mathfrak{N}}
\newcommand{\ofrak}{\mathfrak{O}}
\newcommand{\pfrak}{\mathfrak{P}}
\newcommand{\qfrak}{\mathfrak{Q}}
\newcommand{\rfrak}{\mathfrak{R}}
\newcommand{\sfrak}{\mathfrak{S}}
\newcommand{\tfrak}{\mathfrak{T}}
\newcommand{\ufrak}{\mathfrak{U}}
\newcommand{\vfrak}{\mathfrak{V}}
\newcommand{\wfrak}{\mathfrak{W}}
\newcommand{\xfrak}{\mathfrak{X}}
\newcommand{\yfrak}{\mathfrak{Y}}
\newcommand{\zfrak}{\mathfrak{Z}}

% Mathscr
\newcommand{\ascr}{\mathscr{A}}
\newcommand{\bscr}{\mathscr{B}}
\newcommand{\cscr}{\mathscr{C}}
\newcommand{\dscr}{\mathscr{D}}
\newcommand{\escr}{\mathscr{E}}
\newcommand{\fscr}{\mathscr{F}}
\newcommand{\gscr}{\mathscr{G}}
\newcommand{\hscr}{\mathscr{H}}
\newcommand{\iscr}{\mathscr{I}}
\newcommand{\jscr}{\mathscr{J}}
\newcommand{\kscr}{\mathscr{K}}
\newcommand{\lscr}{\mathscr{L}}
\newcommand{\mscr}{\mathscr{M}}
\newcommand{\nscr}{\mathscr{N}}
\newcommand{\oscr}{\mathscr{O}}
\newcommand{\pscr}{\mathscr{P}}
\newcommand{\qscr}{\mathscr{Q}}
\newcommand{\rscr}{\mathscr{R}}
\newcommand{\sscr}{\mathscr{S}}
\newcommand{\tscr}{\mathscr{T}}
\newcommand{\uscr}{\mathscr{U}}
\newcommand{\vscr}{\mathscr{V}}
\newcommand{\wscr}{\mathscr{W}}
\newcommand{\xscr}{\mathscr{X}}
\newcommand{\yscr}{\mathscr{Y}}
\newcommand{\zscr}{\mathscr{Z}}

% Mathbb
\newcommand{\abb}{\mathbb{A}}
\newcommand{\bbb}{\mathbb{B}}
\newcommand{\cbb}{\mathbb{C}}
\newcommand{\dbb}{\mathbb{D}}
\newcommand{\ebb}{\mathbb{E}}
\newcommand{\fbb}{\mathbb{F}}
\newcommand{\gbb}{\mathbb{G}}
\newcommand{\hbb}{\mathbb{H}}
\newcommand{\ibb}{\mathbb{I}}
\newcommand{\jbb}{\mathbb{J}}
\newcommand{\kbb}{\mathbb{K}}
\newcommand{\lbb}{\mathbb{L}}
\newcommand{\mbb}{\mathbb{M}}
\newcommand{\nbb}{\mathbb{N}}
\newcommand{\obb}{\mathbb{O}}
\newcommand{\pbb}{\mathbb{P}}
\newcommand{\qbb}{\mathbb{Q}}
\newcommand{\rbb}{\mathbb{R}}
\newcommand{\sbb}{\mathbb{S}}
\newcommand{\tbb}{\mathbb{T}}
\newcommand{\ubb}{\mathbb{U}}
\newcommand{\vbb}{\mathbb{V}}
\newcommand{\wbb}{\mathbb{W}}
\newcommand{\xbb}{\mathbb{X}}
\newcommand{\ybb}{\mathbb{Y}}
\newcommand{\zbb}{\mathbb{Z}}


%===============================================================================
%=============================== SPECIAL SYMBOLS ===============================
%===============================================================================


% Orbit (group theory)
\newcommand{\orbit}{\mathcal{O}}
% Normal group
\newcommand{\normal}{\mathcal{N}}
% Indicator function
\newcommand{\indicator}{\mathbbm{1}}
% Laplace transform
\newcommand{\laplace}[1]{\mathcal{L}}
% Epsilon shorthand
\newcommand{\eps}{\varepsilon}
% Omega
\newcommand{\om}{\omega}
\newcommand{\Om}{\Omega}

%===============================================================================
%================================== OPERATORS ==================================
%===============================================================================


% Inverse exponent
\newcommand{\inv}[0]{^{-1}}
% Overline bar
\newcommand{\olsi}[1]{\,\overline{\!{#1}}}
% Less than or equal slanted
\newcommand{\seqs}{\leqslant}
% Greater or equal slanted
\newcommand{\geqs}{\geqslant}
% Subset or equal
\newcommand{\sub}{\subseteq}
% Proper subset
\newcommand{\prosub}{\subset}
% from
\newcommand{\from}{\leftarrow}

% Parantheses
\newcommand{\para}[1]{\left( #1 \right)}
% Curly Braces
\newcommand{\curl}[1]{\left\{ #1 \right\}}
% Brackets
\newcommand{\brac}[1]{\left[ #1 \right]}
% Angled Brackets
\newcommand{\ang}[1]{\left\langle #1 \right\rangle}
% Norm
\newcommand{\norm}[1]{\left\| #1 \right\|}

% Piece wise (use \\ between cases)
\newcommand{\piecewise}[1]{\begin{cases} #1 \end{cases}}

% Bold symbol shorthand
\newcommand{\bl}{\boldsymbol}

% Vertical space
\newcommand{\vs}[1]{\vspace{#1 pt}}
% Horizontal ertical space
\newcommand{\hs}[1]{\hspace{#1 pt}}


%===============================================================================
%============================== TEXT BASED SYMBOLS =============================
%===============================================================================


% Radians
\newcommand{\rad}{\text{rad}}
% Least Common Multiple
\newcommand{\lcm}{\text{lcm}}
% Automorphism
\newcommand{\Aut}{\text{Aut}}
% Variance
\newcommand{\var}{\text{Var}}
% Covariance
\newcommand{\cov}{\text{Cov}}
% Cofactor (matrix)
\newcommand{\cof}{\text{Cof}}
% Adjugate (matrix)
\newcommand{\adj}{\text{Adj}}
% Trace (matrix)
\newcommand{\tr}{\text{tr}}
% Standard deviation
\newcommand{\std}{\text{Std}}
% Correlation coefficient
\newcommand{\corr}{\text{Corr}}
% Sign
\newcommand{\sign}{\text{sign}}

% And text
\newcommand{\AND}{\text{ and }}
% Or text 
\newcommand{\OR}{\text{ or }}
% For text 
\newcommand{\FOR}{\text{ for }}
% If text
\newcommand{\IF}{\text{ if }}
% When text
\newcommand{\WHEN}{\text{ when }}
% Where text
\newcommand{\WHERE}{\text{ where }}
% Then text
\newcommand{\THEN}{\text{ then }}
% Such that text
\newcommand{\SUCHTHAT}{\text{ such that }}

% 1st
\newcommand{\st}[1]{#1^{\text{st}}}
% 2nd
\newcommand{\nd}[1]{#1^{\text{nd}}}
% 3rd
\newcommand{\rd}[1]{#1^{\text{rd}}}
% nth
\newcommand{\nth}[1]{#1^{\text{th}}}


%===============================================================================
%========================= PROBABILITY AND STATISTICS ==========================
%===============================================================================


% Permutation
\newcommand{\perm}[2]{{}^{#1}\!P_{#2}}
% Combination
\newcommand{\comb}[2]{{}^{#1}C_{#2}}

% Baye's risk
\newcommand{\risk}[1]{\mathscr{R}_{#1}}
% Baye's optimal risk
\newcommand{\riskOptimal}[1]{\mathscr{R}_{#1}^*}
% Baye's empirical risk
\newcommand{\riskEmpirical}[2]{\hat{\mathscr{R}}_{#1}^{#2}}


%===============================================================================
%=================================== CALCULUS ==================================
%===============================================================================


% d over d derivative
\newcommand{\dd}[2]{\dfrac{d#1}{d#2}}
% partial d over d derivative
\newcommand{\partialdd}[2]{\dfrac{\partial #1}{\partial #2}}
% delta d over d derivative
\newcommand{\deltadd}[2]{\dfrac{\Delta #1}{\Delta #2}}

% Integration between a and b
\newcommand{\integral}[4]{\int_{#1}^{#2} #3 \, #4}
% Integration in some space 
\newcommand{\boundIntegral}[2]{\int_{#1} #2 \, d#1}

% Limit
\newcommand{\limit}[3]{\lim_{#1 \to #2} #3}


%===============================================================================
%================================  BIG SYMBOLS  ================================
%===============================================================================


% Sum
\newcommand{\sumof}[2]{\sum_{#1}^{#2}}
% Product
\newcommand{\productof}[2]{\prod_{#1}^{#2}}
% Union
\newcommand{\unionof}[2]{\bigcup_{#1}^{#2}}
% Intersection
\newcommand{\intersectionof}[2]{\bigcap_{#1}^{#2}}
% Or
\newcommand{\orof}[2]{\bigvee_{#1}^{#2}}
% And
\newcommand{\andof}[2]{\bigwedge_{#1}^{#2}}


%===============================================================================
%=============================== LINEAR ALGEBRA ================================
%===============================================================================


% Bold vector arrow
\newcommand{\bv}[1]{\vec{\mathbf{#1}}}

% Matrix or vector (use // for column, & for row) 
% with brackets
\newcommand{\bmat}[1]{\begin{bmatrix} #1 \end{bmatrix}}
% Matrix or vector (use // for column, & for row) 
% with curved brackets
\newcommand{\pmat}[1]{\begin{pmatrix} #1 \end{pmatrix}}
% Matrix or vector (use // for column, & for row) 
% with lines on either side 
\newcommand{\lmat}[1]{\begin{vmatrix} #1 \end{vmatrix}}
% Matrix or vector (use // for column, & for row) 
% with curly braces
\newcommand{\cmat}[1]{\begin{Bmatrix} #1 \end{Bmatrix}}
% Matrix or vector (use // for column, & for row) 
% with no braces
\newcommand{\mat}[1]{\begin{matrix} #1 \end{matrix}}


%===============================================================================
%================================ LARGE OBJECTS ================================
%===============================================================================

% Multiple lines
\newcommand{\multiline}[1]{
\begin{align*}
    #1
\end{align*}
}

% Multiple lines with equation numbers
\newcommand{\eqmultiline}[1]{
\begin{align*}
    #1
\end{align*}
}

% Color
\newcommand{\colorText}[2]{
\begingroup
\color{#1}
    #2
\endgroup
}

% Centered figure
\newcommand{\centerFigure}[2]{
    \begin{figure}[h]
        \centering
            #1
        \caption{#2}
    \end{figure}
}

% Tikz figure
\newcommand{\tikzGraphic}[1]{
    \begin{center}
    \begin{tikzpicture}
        #1
    \end{tikzpicture}
    \end{center}
}

% Enumerate numbers (seperate by \item)
\newcommand{\numbers}{\textbf{\number*)}}

% Enumerate letters (seperate by \item)
\newcommand{\letters}{\textbf{\alph*)}}


\title{%
    \Huge Abstract Algebra \\
    \large by \\
    \Large Dummit and Foote \\~\\
    \huge Part 1: Group Theory \\
    \LARGE Chapter 1: Introduction to Groups \\
    \Large Section 3: Symmetric Groups
}
\date{2024-03-10}
\author{Michael Saba}

\begin{document}
    \pagenumbering{gobble}
    \maketitle
    \newpage
    \setlength{\parindent}{0pt}
    \pagenumbering{arabic}

    Another family of groups that can be studied
    is the family of Symmetric Groups. \\
    These groups are both interesting in their own right,
    and useful in motivating general theory. \\

    \subsection*{Symmetric Groups}

    Given a non-empty set $\Omega$,
    let's define $S_\Omega$ to be the set of its permutations.
    The set $S_\Omega$ will be a group under function composition,
    and we will call it a \textbf{symmetric group}.
    We can now show that $S_\Omega$ forms a group:
    \begin{itemize}[label=$\diamond$]
        \item 
            We talked about how composition is a binary operator
            in section 1.1.2, but we still haven't shown it to be true.
            We know that any permutation of a set $\Omega$
            is just a bijective function
            that maps elements of $\Omega$ to elements of itself
            (e.g. in a bijection,
            each input is associated with exactly one output).
            So if $\sigma: \Omega \rightarrow \Omega$ is bijective
            (e.g. each input is associated with exactly one output),
            and $\tau: \Omega \rightarrow \Omega$ is also bijective,
            then $\sigma \circ \tau$ is necessarily also a bijection,
            which means it is a permutation as well,
            making $\circ$ closed under the set of permutations.
        \item 
            The identity of the group is clearly the identity mapping
            that associates elements of $\Omega$ with themselves,
            which we called $ID_\Omega$.
        \item
            We also know that function composition is associative in general.
        \item
            Finally, we know that since each permutation $\sigma$
            is a bijective function  $\sigma: \Omega \rightarrow \Omega$, 
            then it must have a two-sided inverse
            $\sigma^{-1}: \Omega \rightarrow \Omega$.
            This means that
            $\sigma \circ \sigma^{-1} = \sigma^{-1} \circ \sigma = ID_\Omega$.
            We can thus clearly see that a two-sided inverse
            satisfies the role of a group inverse.

    \end{itemize}
    Thus $(S_\Omega, \circ)$ satisfies all axioms of a group. \\
    Note that the permuattions themselves are the elements 
    of the group,
    not the elements of $\Omega$, which is just any arbitrary set. \\

    In the special case where $\Omega = \{ 1, 2, 3 \dots n\}$,
    the symmetric group on $\Omega$ is denoted $S_n$. \\

    Note that if two sets $A$ and $B$ have equal cardinality,
    then $S_A$ and $S_B$ are somewhat equivalent.
    This is because the permutations themselves,
    and the action they perform on the set (the way they permute it)
    is independent of the set itself. \\
    In the last two sections of the chapter,
    we will expand on the idea of group equivalency,
    and on the idea of the actions of groups on sets. \\

    The order of $S_n$ is always $n!$ (the factorial of $n$).
    We can show that that is the case with a simple proof: \\
    Suppose that $\sigma \in S_n$.
    Then $\sigma$ is a bijective function, making it also an injection.
    This permutation can map $1$ to any other element
    in $\{1, 2, 3 \dots n\}$,
    for a total of $n$ mappings.
    It can also map $2$ to any element in 
    $\{1, 2, 3 \dots n\} - \sigma(1)$,
    not including the mapping of $1$ as $\sigma$ must be an injection,
    for a total of $(n-1)$ mappings.
    We can continue with this until we get to $n$,
    which will have only a single option for a mapping.
    The total number of permutations $\sigma$ that can be formed
    is thus $(n)(n-1)(n-2)\dots(2)(1) = n!$. \\
    Another way of thinking about this;
    every permuation in $S_n$ will turn $\{1, 2, 3 \dots n\}$
    into a set containing the same elements, but shuffled differently.
    The number of ways we can order $n$ elements
    is $P_n^n = \sfrac{n!}{(n-n)!} = \sfrac{n!}{1} = n!$. \\

    
    \subsection*{Cycle Decompositions}

    One way to efficiently write elements of $S_n$
    is by using what we call a \textbf{cycle decomposition},
    which we will define shortly. \\

    A \textbf{cycle} is a string of integers representing the permutation
    that cyclically shifts the integers.
    Basically, a cycle $(a_1 \quad a_2 \quad a_3 \dots \quad a_k)$
    maps $a_1$ to $a_2$, $a_2$ to $a_3$, and so on,
    up until $a_{n-1}$ to $a_n$ and, finally, $a_n$ to $a_1$.
    In other words,
    a cycle is an element of $S_n$ that
    sends every integer $a_i$ to $a_{(i \mod k) + 1}$
    for $1 \leqslant i \leqslant k$,
    and maps every other integer to itself
    (as they are not part of the cycle). \\

    Of course, not all permutations of $S_n$
    necessarily permute integers in one cycle.
    However, any permutation $\sigma \in S_n$
    can be written as the combination of several cycles.
    For instance, if in $S_4$, we have a permutation $\sigma$:
    \[ \sigma(1) = 2
    \qquad \sigma(2) = 1
    \qquad \sigma(3) = 4
    \qquad \sigma(4) = 3 \]
    then we are basically cyclically permuting $1$ and $2$,
    then $3$ and $4$,
    so that $\sigma$ can be written as $(1 \quad 2)(3 \quad 4)$. 
    This is a cycle decomposition of $\sigma$. \\
    Note that each of $(1 \quad 2)$ and $(3 \quad 4)$
    are elements of $S_n$ in their own right,
    and that saying $\sigma = (1 \quad 2)(3 \quad 4)$
    is equivalent to saying $\sigma = (1 \quad 2) \circ (3 \quad 4)$
    or  $\sigma = (3 \quad 4) \circ (1 \quad 2)$.
    We perform one cycle, then the other.
    Note that in this case, the cycles are \textbf{disjoint},
    meaning that no element repeats in more than once cycle.
    This makes it easier to stack them together (composition)
    as each one affects a disjoint set of integers,
    and leaves all other integers intact,
    meaning that we can perfrom their respective cycles
    without worrying that they will intertwine.
    This is not the case with all cycle decompositions,
    although we can still stack non-disjoint cycles
    in the same way; by composition. \\

    Note that a cycle with its elements shifted is still the same
    cycle since it perfroms the same cyclical permutation.
    So for example, $(1 \quad 2 \quad 3) =
    (3 \quad 1 \quad 2) = (2 \quad 3 \quad 1)$. \\
    By convention, the smallest number is written first. \\

    Given an element $\sigma \in S_{12}$ with all of its permutations
    layed out like this:
    \begin{align*}
        & \sigma(1) = 12 & &  \sigma(2) = 2
        & & \sigma(3) = 4   & & \sigma(4) = 5  \\
        & \sigma(5) = 8 & & \sigma(6) = 11
        & &\sigma(7) = 9   & & \sigma(8) = 10  \\
        & \sigma(9) = 7 & & \sigma(10) = 1
        & & \sigma(11) = 3   & & \sigma(12) = 6  \\
    \end{align*}
    The algorithm for finding the cycle decomposition is as follows:
    \begin{enumerate}
        \item 
            First we pick the smallest element not yet in any cycle.
            Suppose it's $a$.
            We then place it in a new cycle
            by opening a paranthesis and keeping it open.
            If there are none, then we are done.
        \item
            After that, we check the value of $\sigma(a)$,
            which we will call $b$.
            If it's the same as $a$, we close the paranthesis
            and end the cycle.
            We then go back to step $1$.
            If $b$ is a different integer,
            we add it to the cycle
            and keep the paranthesis open.
        \item
            After that we repeat the same idea with $c = \sigma(b)$.
            If $c = a$, we close the paranthesis after $b$.
            Otherwise we add $c$ to the cycle decomposition.
            Note that it is impossible for $c$ to be the same as $b$
            as we already have $\sigma(a) = b$,
            and $\sigma$ is injective.
            We then repeat this step with $\sigma(c)$,
            and the one after it, until the cycle closes.
            We then go back to step $1$.
        \item
            The final step is to remove all cycles
            that contain only one element.
            As we mentioned earlier,
            a cycle shifts the elements inside it,
            and leaves the rest intact.
            So if a cycle has only one integer in it,
            it does nothing to it (maps it to itself),
            so we may as well omit it.
    \end{enumerate}
    With that, we end up with:
    \[ \sigma = (1 \quad 12 \quad 6 \quad 11 \quad 3 \quad
    4 \quad 5 \quad 8 \quad 10)(2)(7 \quad 9) \]
    Note that as a consequence of the way the algorithm is done,
    all cycles are automatically disjoint 
    (no elements are present in more than one cycle).
    We mentioned this in step $3$ when we noted that if $b = \sigma(a)$,
    then $c = \sigma(b)$ can't be $b$. \\

    The length of a cycle is the number of integers that appear in it.
    A cycle with length $t$ is called a $t$-cycle. \\
    The identity of $S_n$ can be written as a cycle decomposition
    with $n$ disjoint $1$-cycles,
    which by step $4$ of the algorithm is the same as
    having no cycles (as it keeps all elements mapped to themselves). \\

    Non-disjoint cycles, with elements in common,
    can also be stacked
    (again we are just doing function composition)
    but they will affect each other,
    and as such, the order in which they are performed matters.
    If we have a composition of non-disjoint cycles,
    like all function compositions,
    they are read from right to left:
    \[ (1 \quad 2) \circ (1 \quad 3) = (1 \quad 3 \quad 2)
    \qquad (1 \quad 3) \circ (1 \quad 2) = (1 \quad 2 \quad 3) \]
    So we can conclude that symmetric groups are not abelian. \\

    If the cycles are disjoint though the order does not matter,
    and this follows from the fact that the sets of integers
    they affect are disjoint and unrelated.
    This means that disjoint cycles,
    which themselves are elements of $S_n$, commute. \\
    
    Any permutation can be written in many ways using cycle decompositions.
    There is however,
    only one unique way to write a permutation as a composition
    of disjoint cycles
    (this is later proven).
    Note that changing the order of the cycles,
    or shifting the integers inside each cycle does not constitute
    a new disjoint decomposition. \\
    Because disjoint cycles commute,
    and because its is far easier to predict
    how they permute a set of integers without any intertwining,
    we prefer to represent permutations as disjoint cycle decompositions.
    By default, the algorithm we described always gives us
    the unique disjoint decomposition we want. \\

    Another useful property of the disjoint cycle decompositions
    is that,
    due to its uniqueness, it allows to tell if two permutations
    are the same. \\

    Moreover, we can tell the order of a permutation
    by its cycle decomposition.
    The order is simply the least common multiple of the lengths
    of the cycle decompositions.
    This is proven in the exercises,
    but we can show here why it is true. \\
    Each cycle is disjoint,
    so we consider them independently.
    Each cycle shifts the integers in it,
    and for a cycle of length $t$,
    it will take $t$ shifts to return them in place
    (a complete cycle).
    However $t \times i$ shifts
    for any positive integer $i$ will also do that
    (it's $i$ complete cycles)
    So for any value $m$ such that $m$ is a multiple of
    all of the lengths of cycles,
    repeating all the cycles $m$ times,
    which corresponds to repeating the whole permutation $m$ times
    (because the cycles are disjoint),
    returns all the integers present in the set to their original place,
    turning it to the identity.
    The order is by definition,
    the smallest value $m$ for which $\sigma^m = 1$,
    so it is by definition the least common multiple
    of the cycle lengths. \\


\end{document}