

\documentclass[12pt]{article}
\usepackage[margin=1in]{geometry}


%===============================================================================
%================================== PACKAGES ===================================
%===============================================================================

% For using float option H that places figures 
% exatcly where we want them
\usepackage{float}
% makes figure font bold
\usepackage{caption}
\captionsetup[figure]{labelfont=bf}
% For text generation
\usepackage{lipsum}
% For drawing
\usepackage{tikz}
% For smaller or equal sign and not divide sign
\usepackage{amssymb}
% For the diagonal fraction
\usepackage{xfrac}
% For enumerating exercise parts with letters instead of numbers
\usepackage{enumitem}
% For dfrac, which forces the fraction to be in display mode (large) e
% even in math mode (small)
\usepackage{amsmath}
% For degree sign
\usepackage{gensymb}
% For "\mathbb" macro
\usepackage{amsfonts}
% For positioning 
\usepackage{indentfirst}
\usetikzlibrary{shapes,positioning,fit,calc}
% for adjustwidth environment
\usepackage{changepage}
% for arrow on top
\usepackage{esvect}
% for mathbb 1
\usepackage{bbm}
% for mathsrc
\usepackage[mathscr]{eucal}
% For degree sign
\usepackage{gensymb}
% For quotes
\usepackage{csquotes}
% For vertical lines
\usepackage{mathtools}
% For cols
\usepackage{multicol}

% for tikz
\usepackage{pgfplots}
\pgfplotsset{compat=1.18}
\usepackage{amsmath}
\usepgfplotslibrary{groupplots}


%===============================================================================
%==================================== FONTS ====================================
%===============================================================================


% Mathcal
\newcommand{\acal}{\mathcal{A}}
\newcommand{\bcal}{\mathcal{B}}
\newcommand{\ccal}{\mathcal{C}}
\newcommand{\dcal}{\mathcal{D}}
\newcommand{\ecal}{\mathcal{E}}
\newcommand{\fcal}{\mathcal{F}}
\newcommand{\gcal}{\mathcal{G}}
\newcommand{\hcal}{\mathcal{H}}
\newcommand{\ical}{\mathcal{I}}
\newcommand{\jcal}{\mathcal{J}}
\newcommand{\kcal}{\mathcal{K}}
\newcommand{\lcal}{\mathcal{L}}
\newcommand{\mcal}{\mathcal{M}}
\newcommand{\ncal}{\mathcal{N}}
\newcommand{\ocal}{\mathcal{O}}
\newcommand{\pcal}{\mathcal{P}}
\newcommand{\qcal}{\mathcal{Q}}
\newcommand{\rcal}{\mathcal{R}}
\newcommand{\scal}{\mathcal{S}}
\newcommand{\tcal}{\mathcal{T}}
\newcommand{\ucal}{\mathcal{U}}
\newcommand{\vcal}{\mathcal{V}}
\newcommand{\wcal}{\mathcal{W}}
\newcommand{\xcal}{\mathcal{X}}
\newcommand{\ycal}{\mathcal{Y}}
\newcommand{\zcal}{\mathcal{Z}}

% Mathfrak
\newcommand{\afrak}{\mathfrak{A}}
\newcommand{\bfrak}{\mathfrak{B}}
\newcommand{\cfrak}{\mathfrak{C}}
\newcommand{\dfrak}{\mathfrak{D}}
\newcommand{\efrak}{\mathfrak{E}}
\newcommand{\ffrak}{\mathfrak{F}}
\newcommand{\gfrak}{\mathfrak{G}}
\newcommand{\hfrak}{\mathfrak{H}}
\newcommand{\ifrak}{\mathfrak{I}}
\newcommand{\jfrak}{\mathfrak{J}}
\newcommand{\kfrak}{\mathfrak{K}}
\newcommand{\lfrak}{\mathfrak{L}}
\newcommand{\mfrak}{\mathfrak{M}}
\newcommand{\nfrak}{\mathfrak{N}}
\newcommand{\ofrak}{\mathfrak{O}}
\newcommand{\pfrak}{\mathfrak{P}}
\newcommand{\qfrak}{\mathfrak{Q}}
\newcommand{\rfrak}{\mathfrak{R}}
\newcommand{\sfrak}{\mathfrak{S}}
\newcommand{\tfrak}{\mathfrak{T}}
\newcommand{\ufrak}{\mathfrak{U}}
\newcommand{\vfrak}{\mathfrak{V}}
\newcommand{\wfrak}{\mathfrak{W}}
\newcommand{\xfrak}{\mathfrak{X}}
\newcommand{\yfrak}{\mathfrak{Y}}
\newcommand{\zfrak}{\mathfrak{Z}}

% Mathscr
\newcommand{\ascr}{\mathscr{A}}
\newcommand{\bscr}{\mathscr{B}}
\newcommand{\cscr}{\mathscr{C}}
\newcommand{\dscr}{\mathscr{D}}
\newcommand{\escr}{\mathscr{E}}
\newcommand{\fscr}{\mathscr{F}}
\newcommand{\gscr}{\mathscr{G}}
\newcommand{\hscr}{\mathscr{H}}
\newcommand{\iscr}{\mathscr{I}}
\newcommand{\jscr}{\mathscr{J}}
\newcommand{\kscr}{\mathscr{K}}
\newcommand{\lscr}{\mathscr{L}}
\newcommand{\mscr}{\mathscr{M}}
\newcommand{\nscr}{\mathscr{N}}
\newcommand{\oscr}{\mathscr{O}}
\newcommand{\pscr}{\mathscr{P}}
\newcommand{\qscr}{\mathscr{Q}}
\newcommand{\rscr}{\mathscr{R}}
\newcommand{\sscr}{\mathscr{S}}
\newcommand{\tscr}{\mathscr{T}}
\newcommand{\uscr}{\mathscr{U}}
\newcommand{\vscr}{\mathscr{V}}
\newcommand{\wscr}{\mathscr{W}}
\newcommand{\xscr}{\mathscr{X}}
\newcommand{\yscr}{\mathscr{Y}}
\newcommand{\zscr}{\mathscr{Z}}

% Mathbb
\newcommand{\abb}{\mathbb{A}}
\newcommand{\bbb}{\mathbb{B}}
\newcommand{\cbb}{\mathbb{C}}
\newcommand{\dbb}{\mathbb{D}}
\newcommand{\ebb}{\mathbb{E}}
\newcommand{\fbb}{\mathbb{F}}
\newcommand{\gbb}{\mathbb{G}}
\newcommand{\hbb}{\mathbb{H}}
\newcommand{\ibb}{\mathbb{I}}
\newcommand{\jbb}{\mathbb{J}}
\newcommand{\kbb}{\mathbb{K}}
\newcommand{\lbb}{\mathbb{L}}
\newcommand{\mbb}{\mathbb{M}}
\newcommand{\nbb}{\mathbb{N}}
\newcommand{\obb}{\mathbb{O}}
\newcommand{\pbb}{\mathbb{P}}
\newcommand{\qbb}{\mathbb{Q}}
\newcommand{\rbb}{\mathbb{R}}
\newcommand{\sbb}{\mathbb{S}}
\newcommand{\tbb}{\mathbb{T}}
\newcommand{\ubb}{\mathbb{U}}
\newcommand{\vbb}{\mathbb{V}}
\newcommand{\wbb}{\mathbb{W}}
\newcommand{\xbb}{\mathbb{X}}
\newcommand{\ybb}{\mathbb{Y}}
\newcommand{\zbb}{\mathbb{Z}}


%===============================================================================
%=============================== SPECIAL SYMBOLS ===============================
%===============================================================================


% Orbit (group theory)
\newcommand{\orbit}{\mathcal{O}}
% Normal group
\newcommand{\normal}{\mathcal{N}}
% Indicator function
\newcommand{\indicator}{\mathbbm{1}}
% Laplace transform
\newcommand{\laplace}[1]{\mathcal{L}}
% Epsilon shorthand
\newcommand{\eps}{\varepsilon}
% Omega
\newcommand{\om}{\omega}
\newcommand{\Om}{\Omega}

%===============================================================================
%================================== OPERATORS ==================================
%===============================================================================


% Inverse exponent
\newcommand{\inv}[0]{^{-1}}
% Overline bar
\newcommand{\olsi}[1]{\,\overline{\!{#1}}}
% Less than or equal slanted
\newcommand{\seqs}{\leqslant}
% Greater or equal slanted
\newcommand{\geqs}{\geqslant}
% Subset or equal
\newcommand{\sub}{\subseteq}
% Proper subset
\newcommand{\prosub}{\subset}
% from
\newcommand{\from}{\leftarrow}

% Parantheses
\newcommand{\para}[1]{\left( #1 \right)}
% Curly Braces
\newcommand{\curl}[1]{\left\{ #1 \right\}}
% Brackets
\newcommand{\brac}[1]{\left[ #1 \right]}
% Angled Brackets
\newcommand{\ang}[1]{\left\langle #1 \right\rangle}
% Norm
\newcommand{\norm}[1]{\left\| #1 \right\|}

% Piece wise (use \\ between cases)
\newcommand{\piecewise}[1]{\begin{cases} #1 \end{cases}}

% Bold symbol shorthand
\newcommand{\bl}{\boldsymbol}

% Vertical space
\newcommand{\vs}[1]{\vspace{#1 pt}}
% Horizontal ertical space
\newcommand{\hs}[1]{\hspace{#1 pt}}


%===============================================================================
%============================== TEXT BASED SYMBOLS =============================
%===============================================================================


% Radians
\newcommand{\rad}{\text{rad}}
% Least Common Multiple
\newcommand{\lcm}{\text{lcm}}
% Automorphism
\newcommand{\Aut}{\text{Aut}}
% Variance
\newcommand{\var}{\text{Var}}
% Covariance
\newcommand{\cov}{\text{Cov}}
% Cofactor (matrix)
\newcommand{\cof}{\text{Cof}}
% Adjugate (matrix)
\newcommand{\adj}{\text{Adj}}
% Trace (matrix)
\newcommand{\tr}{\text{tr}}
% Standard deviation
\newcommand{\std}{\text{Std}}
% Correlation coefficient
\newcommand{\corr}{\text{Corr}}
% Sign
\newcommand{\sign}{\text{sign}}

% And text
\newcommand{\AND}{\text{ and }}
% Or text 
\newcommand{\OR}{\text{ or }}
% For text 
\newcommand{\FOR}{\text{ for }}
% If text
\newcommand{\IF}{\text{ if }}
% When text
\newcommand{\WHEN}{\text{ when }}
% Where text
\newcommand{\WHERE}{\text{ where }}
% Then text
\newcommand{\THEN}{\text{ then }}
% Such that text
\newcommand{\SUCHTHAT}{\text{ such that }}

% 1st
\newcommand{\st}[1]{#1^{\text{st}}}
% 2nd
\newcommand{\nd}[1]{#1^{\text{nd}}}
% 3rd
\newcommand{\rd}[1]{#1^{\text{rd}}}
% nth
\newcommand{\nth}[1]{#1^{\text{th}}}


%===============================================================================
%========================= PROBABILITY AND STATISTICS ==========================
%===============================================================================


% Permutation
\newcommand{\perm}[2]{{}^{#1}\!P_{#2}}
% Combination
\newcommand{\comb}[2]{{}^{#1}C_{#2}}

% Baye's risk
\newcommand{\risk}[1]{\mathscr{R}_{#1}}
% Baye's optimal risk
\newcommand{\riskOptimal}[1]{\mathscr{R}_{#1}^*}
% Baye's empirical risk
\newcommand{\riskEmpirical}[2]{\hat{\mathscr{R}}_{#1}^{#2}}


%===============================================================================
%=================================== CALCULUS ==================================
%===============================================================================


% d over d derivative
\newcommand{\dd}[2]{\dfrac{d#1}{d#2}}
% partial d over d derivative
\newcommand{\partialdd}[2]{\dfrac{\partial #1}{\partial #2}}
% delta d over d derivative
\newcommand{\deltadd}[2]{\dfrac{\Delta #1}{\Delta #2}}

% Integration between a and b
\newcommand{\integral}[4]{\int_{#1}^{#2} #3 \, #4}
% Integration in some space 
\newcommand{\boundIntegral}[2]{\int_{#1} #2 \, d#1}

% Limit
\newcommand{\limit}[3]{\lim_{#1 \to #2} #3}


%===============================================================================
%================================  BIG SYMBOLS  ================================
%===============================================================================


% Sum
\newcommand{\sumof}[2]{\sum_{#1}^{#2}}
% Product
\newcommand{\productof}[2]{\prod_{#1}^{#2}}
% Union
\newcommand{\unionof}[2]{\bigcup_{#1}^{#2}}
% Intersection
\newcommand{\intersectionof}[2]{\bigcap_{#1}^{#2}}
% Or
\newcommand{\orof}[2]{\bigvee_{#1}^{#2}}
% And
\newcommand{\andof}[2]{\bigwedge_{#1}^{#2}}


%===============================================================================
%=============================== LINEAR ALGEBRA ================================
%===============================================================================


% Bold vector arrow
\newcommand{\bv}[1]{\vec{\mathbf{#1}}}

% Matrix or vector (use // for column, & for row) 
% with brackets
\newcommand{\bmat}[1]{\begin{bmatrix} #1 \end{bmatrix}}
% Matrix or vector (use // for column, & for row) 
% with curved brackets
\newcommand{\pmat}[1]{\begin{pmatrix} #1 \end{pmatrix}}
% Matrix or vector (use // for column, & for row) 
% with lines on either side 
\newcommand{\lmat}[1]{\begin{vmatrix} #1 \end{vmatrix}}
% Matrix or vector (use // for column, & for row) 
% with curly braces
\newcommand{\cmat}[1]{\begin{Bmatrix} #1 \end{Bmatrix}}
% Matrix or vector (use // for column, & for row) 
% with no braces
\newcommand{\mat}[1]{\begin{matrix} #1 \end{matrix}}


%===============================================================================
%================================ LARGE OBJECTS ================================
%===============================================================================

% Multiple lines
\newcommand{\multiline}[1]{
\begin{align*}
    #1
\end{align*}
}

% Multiple lines with equation numbers
\newcommand{\eqmultiline}[1]{
\begin{align*}
    #1
\end{align*}
}

% Color
\newcommand{\colorText}[2]{
\begingroup
\color{#1}
    #2
\endgroup
}

% Centered figure
\newcommand{\centerFigure}[2]{
    \begin{figure}[h]
        \centering
            #1
        \caption{#2}
    \end{figure}
}

% Tikz figure
\newcommand{\tikzGraphic}[1]{
    \begin{center}
    \begin{tikzpicture}
        #1
    \end{tikzpicture}
    \end{center}
}

% Enumerate numbers (seperate by \item)
\newcommand{\numbers}{\textbf{\number*)}}

% Enumerate letters (seperate by \item)
\newcommand{\letters}{\textbf{\alph*)}}

\title{%
    \Huge Abstract Algebra \\
    \large by \\
    \Large Dummit and Foote \\~\\
    \huge Part 0: Preliminaries \\
    \LARGE Chapter 0: Preliminaries \\
    \Large Section 2: Properties of the Integers
}
\date{2024-03-17}
\author{Michael Saba}

\begin{document}
    \pagenumbering{gobble}
    \maketitle
    \newpage    
    \setlength{\parindent}{0pt}
    \pagenumbering{arabic}

    This section introduces properties of the integer numbers.
    These properties will be proven later, in chapter $2.8$,
    in the context of Ring Theory. \\
    However, we need to know them now for Group Theory. \\
    Of course none of the Ring Theory proofs will depend
    on Group Theory. \\

    One property of integers is the \textbf{well ordering} property.
    Any non-empty subset $A$ of the positive integers $\Z^+$
    will have an element $m$
    such that $m \leqslant a$ $\forall a \in A$
    (called the \textbf{minimal element} of $A$). \\

    If $a, b \in \Z$, and $a \neq 0$,
    then $a$ \textbf{divides} $b$ if $b = na$ for some $n \in \Z$.
    We use $a \mid b$ to represent this fact
    (and $a \nmid b$ if $a$ does not divide $b$).
    In other words, $a$ divides $b$ if we can evenly divide
    $b$ $a$ times, with no remainder. \\
    Here, we say that $b$ is a \textbf{mutliple} of $a$,
    as $b = a \times n$. \\
    Note that $1$ divides all integers $a \in \Z$ (with $n = a$)
    and all integers divide themselves (with $n = 1$). \\
    

    If $a, b \in \Z - \{0\}$,
    then there is a unique integer $d$,
    called the \textbf{greatest common divisor} of $a$ and $b$,
    or $\text{gcd}(a, b)$, or sometimes just $(a, b)$.
    It's the largest integer that divides both $a$ and $b$.
    Another way of saying that is that:
    \begin{itemize}[label=$\diamond$]
        \item 
            $d \mid a$ and $d \mid b$.
        \item
            If $e \mid a$ and $e \mid b$, then $e \mid d$.
    \end{itemize}
    The first condition trivially shows that $d$ is a common divisor
    of $a$ and $b$.
    The second condition clearly shows that it is the largest
    (for now however, we will not show why $e$ divides $d$
    necessarily). \\
    If the greatest common divisior of $a$ and $b$ is $1$,
    then $1$ is their only positive integer divisor, 
    and we say $a$ and $b$ are \textbf{relatively prime}. \\
    If $a = b$, then $\text{gcd}(a, b) = a = b$. \\
 
    If $a, b \in \Z - \{0\}$,
    then there is a unique integer $l$,
    called the \textbf{least common multiple} of $a$ and $b$,
    or $\text{lcm}(a, b)$.
    It's the smallest integer that both $a$ and $b$ divide
    (that is a multiple of both).
    Another way of saying that is that:
    \begin{itemize}[label=$\diamond$]
        \item 
            $a \mid l$ and $b \mid l$.
        \item
            If $a \mid m$ and $b \mid m$, then $l \mid m$.
    \end{itemize}
    The first condition trivially shows that $l$ is a common multiple
    of $a$ and $b$.
    The second condition shows that it is the smallest
    (for now however, we will not show why $l$ divides $m$
    necessarily). \\ 
    If $a = b$, then $\text{lcm}(a, b) = a = b$. \\
 
    The connection between $d = \text{gcd}(a, b)$
    and $l = \text{lcm}(a, b)$
    is that $ld = ab$. \\

    The \textbf{division algorithm} states that
    for $a, b \in \Z - \{0\}$,
    there exists unique integers $q, r \in \Z$ such that
    \[ a = qb + r \quad \text{ and } \quad 0 \leqslant r < |b| \]
    This is reminiscent the long division algorithm,
    in elementary math. 
    If $b \nmid a$,
    we ended up with a quotient $q$,
    such that $|q|$ was the largest possible value
    with $|qb|$ not exceeding $|a|$.
    We also ended up with a remainder $r$,
    which was necessarily smaller than $b$,
    because otherwise,
    we could have found a $q$ such that $|q|$ was larger
    without exceeding $|a|$. \\
    If $b \mid a$, then $r = 0$,
    since we do have a integer $q$ for which $a = qb$. \\

    The \textbf{Euclidian algorithm} allows us
    calculate the greatest common divisor of two numbers
    using the division algorithm.
    Given $a, b \in \Z - \{0\}$,
    if $a = b$, then $\text{gcd}(a, b) = a = b$,
    and we are done. 
    So we can assume that they aren't equal,
    and, with no loss of generality,
    we also assume that $a > b$.
    Then
    \[ a = q_0b + r_0 \qquad \text{for some $q_0$ and $r_0$} \]
    \[ b = q_1r_0 + r_1 \qquad \text{for some $q_1$ and $r_1$} \]
    \[ r_0 = q_2r_1 + r_2 \qquad \text{for some $q_2$ and $r_2$} \]
    \[ \vdots \]
    \[ r_{n-2} = q_{n}r_{n-1} + r_{n}
    \qquad \text{for some $q_n$ and $r_n$} \]
    \[ r_{n-1} = q_{n+1}r_n + 0
    \qquad \text{for some $q_{n+1}$} \]
    Once we reach a division step with no remainder,
    we will have found our greatest common divisor,
    which is $r_n$. \\
    We know we will always find an answer
    because $|r_0| > |r_1| \dots > |r_n|$
    is a decreasing sequence of integers,
    which cannot continue indefinitely. \\

    As a consequence of this algorithm,
    we get that for all $a, b \in \Z - \{0\}$,
    there exists some $x, y \in \Z$ such that
    \[ \text{gcd}(a, b) = ax + by \] 
    which means that the greatest common divisior can be written
    as a \textbf{linear combination} of $a$ and $b$ in $\Z$. \\
    This is done by recursively writing $r_n$,
    the last remainder and greatest common divisor of $a$ and $b$,
    as a function of the previous remainder.
    We have $ r_{n-2} = q_{n}r_{n-1} + r_{n}$,
    so we arrive at $r_{n} = r_{n-2} - q_{n}r_{n-1}$.
    We repeat this process until we get to the first equation
    containing $a$ and $b$.
    Since all coefficients $q_i$ and $r_i$ are integers,
    we necessarily end up with $r_n = ax + by$. \\

    A positive integer $p \in \Z^+$ is called \textbf{prime}
    if the only positive divisors of $p$ are $p$ and $1$. \\
    All non-prime positive integers are called \textbf{composite}. \\
    An important property of primes,
    which can even be used to define primes
    (a number is prime if and only if it has this property)
    is that if for a prime $p$, $p \mid ab$,
    then $p \mid a$ or $p \mid b$. \\
    Intuitively, this is because we can't divide $p$
    into two integers, each of which divides either $a$ or $b$,
    so it must be that $p$ divides one, or both of them. \\
    Note that we do not consider $1$ to be prime. \\

    \textbf{The Fundamental Theorem of Arithmetic}
    states that for any integer $n \in \Z$ such that $n > 1$,
    $n$ can be factored into a unique product of primes. \\
    In other words, for any integer $n$ larger than $1$,
    there exists distinct primes $p_1, p_2 \dots p_s$
    and positive integers $\alpha_1, \alpha_2 \dots \alpha_s$
    such that
    \[ n = {p_1}^{\alpha_1}{p_2}^{\alpha_2} \dots {p_s}^{\alpha_s} \]
    This product is unique in the sense that there is only
    one way to factor any integer $n > 1$ into a product of primes. \\
    We can prove the existence of the prime factorization
    using strong induction,
    and the uniqueness using a proof by smallest counterexample
    (the well ordering principle). \\

    Suppose that $a, b \in Z$ and $a, b > 1$.
    Then we could write
    \[ a = {p_1}^{\alpha_1}{p_2}^{\alpha_2} \dots {p_s}^{\alpha_s}
    \quad \text{ and } \quad
    b = {p_1}^{\beta_1}{p_2}^{\beta_2} \dots {p_s}^{\beta_s} \]
    where all exponents $\alpha_i$ and $\beta_i$
    are larger or equal to $0$.
    This time around, we allow the exponent to be $0$
    (in the definition of the prime factorization
    we said it had to be positive),
    so that we can use the same primes in the products of $a$ and $b$,
    so that if a prime features in one and not the other,
    we just put its exponent as $0$. \\
    In this case
    \[ \text{gcd}(a, b) = 
    {p_1}^{\min(\alpha_1, \beta_1)}
    {p_2}^{\min(\alpha_2, \beta_2)} \dots
    {p_s}^{\min(\alpha_s, \beta_s)} \]
    and 
    \[ \text{lcm}(a, b) = 
    {p_1}^{\max(\alpha_1, \beta_1)}
    {p_2}^{\max(\alpha_2, \beta_2)} \dots
    {p_s}^{\max(\alpha_s, \beta_s)} \]
    This too will be proved later. \\

    The \textbf{Euler phi-function} $\varphi$ is defined as follows:
    for any non negative integer $n \in \Z^+$,
    $\phi(n)$ is the number of positive $a \leqslant n$
    that are relatively prime to $n$.
    For example, $\phi(12) = 4$,
    since $\text{gcd}(12, 11) = 1$,
    $\text{gcd}(12, 7) = 1$,
    $\text{gcd}(12, 5) = 1$,
    and $\text{gcd}(12, 1) = 1$. \\
    For primes $p$ and integers $a \geqslant 1$,
    \[ \varphi(p^a) = p^{a} - p^{a-1} = p^{a-1}(p - 1) \]
    The function is also multiplicative,
    in the sense that for integers $a, b \in \Z^+$
    \[ \varphi(a, b) = \varphi(a)\varphi(b)
    \quad \text{ if } \quad \text{gcd}(a, b) = 1 \]
    Using both formulas together,
    along with the Fundamental Theorem of Arithmetic,
    we arrive at this general formula
    for any non-negative integer $n \in \Z^+$
    with the prime factorization
    ${p_1}^{\alpha_1}{p_2}^{\alpha_2} \dots {p_s}^{\alpha_s}$:
    \[ \varphi(n) = 
    \varphi({p_1}^{\alpha_1}{p_2}^{\alpha_2} \dots {p_s}^{\alpha_s}) =
    \varphi({p_1}^{\alpha_1})\varphi({p_2}^{\alpha_2}) \dots 
    \varphi({p_s}^{\alpha_s})\]
    \[ = {p_1}^{\alpha_1 - 1}(p_1 - 1){p_2}^{\alpha_2 - 1}(p_2 - 1)
    \dots {p_s}^{\alpha_s - 1}(p_s - 1) \]
    This is true because all distinct powers of prime are relatively prime
    by the defintion of a prime numbers. \\

\end{document}
