
\documentclass[12pt]{article}

\usepackage[margin=1in]{geometry}


%===============================================================================
%================================== PACKAGES ===================================
%===============================================================================

% For using float option H that places figures 
% exatcly where we want them
\usepackage{float}
% makes figure font bold
\usepackage{caption}
\captionsetup[figure]{labelfont=bf}
% For text generation
\usepackage{lipsum}
% For drawing
\usepackage{tikz}
% For smaller or equal sign and not divide sign
\usepackage{amssymb}
% For the diagonal fraction
\usepackage{xfrac}
% For enumerating exercise parts with letters instead of numbers
\usepackage{enumitem}
% For dfrac, which forces the fraction to be in display mode (large) e
% even in math mode (small)
\usepackage{amsmath}
% For degree sign
\usepackage{gensymb}
% For "\mathbb" macro
\usepackage{amsfonts}
% For positioning 
\usepackage{indentfirst}
\usetikzlibrary{shapes,positioning,fit,calc}
% for adjustwidth environment
\usepackage{changepage}
% for arrow on top
\usepackage{esvect}
% for mathbb 1
\usepackage{bbm}
% for mathsrc
\usepackage[mathscr]{eucal}
% For degree sign
\usepackage{gensymb}
% For quotes
\usepackage{csquotes}
% For vertical lines
\usepackage{mathtools}
% For cols
\usepackage{multicol}

% for tikz
\usepackage{pgfplots}
\pgfplotsset{compat=1.18}
\usepackage{amsmath}
\usepgfplotslibrary{groupplots}


%===============================================================================
%==================================== FONTS ====================================
%===============================================================================


% Mathcal
\newcommand{\acal}{\mathcal{A}}
\newcommand{\bcal}{\mathcal{B}}
\newcommand{\ccal}{\mathcal{C}}
\newcommand{\dcal}{\mathcal{D}}
\newcommand{\ecal}{\mathcal{E}}
\newcommand{\fcal}{\mathcal{F}}
\newcommand{\gcal}{\mathcal{G}}
\newcommand{\hcal}{\mathcal{H}}
\newcommand{\ical}{\mathcal{I}}
\newcommand{\jcal}{\mathcal{J}}
\newcommand{\kcal}{\mathcal{K}}
\newcommand{\lcal}{\mathcal{L}}
\newcommand{\mcal}{\mathcal{M}}
\newcommand{\ncal}{\mathcal{N}}
\newcommand{\ocal}{\mathcal{O}}
\newcommand{\pcal}{\mathcal{P}}
\newcommand{\qcal}{\mathcal{Q}}
\newcommand{\rcal}{\mathcal{R}}
\newcommand{\scal}{\mathcal{S}}
\newcommand{\tcal}{\mathcal{T}}
\newcommand{\ucal}{\mathcal{U}}
\newcommand{\vcal}{\mathcal{V}}
\newcommand{\wcal}{\mathcal{W}}
\newcommand{\xcal}{\mathcal{X}}
\newcommand{\ycal}{\mathcal{Y}}
\newcommand{\zcal}{\mathcal{Z}}

% Mathfrak
\newcommand{\afrak}{\mathfrak{A}}
\newcommand{\bfrak}{\mathfrak{B}}
\newcommand{\cfrak}{\mathfrak{C}}
\newcommand{\dfrak}{\mathfrak{D}}
\newcommand{\efrak}{\mathfrak{E}}
\newcommand{\ffrak}{\mathfrak{F}}
\newcommand{\gfrak}{\mathfrak{G}}
\newcommand{\hfrak}{\mathfrak{H}}
\newcommand{\ifrak}{\mathfrak{I}}
\newcommand{\jfrak}{\mathfrak{J}}
\newcommand{\kfrak}{\mathfrak{K}}
\newcommand{\lfrak}{\mathfrak{L}}
\newcommand{\mfrak}{\mathfrak{M}}
\newcommand{\nfrak}{\mathfrak{N}}
\newcommand{\ofrak}{\mathfrak{O}}
\newcommand{\pfrak}{\mathfrak{P}}
\newcommand{\qfrak}{\mathfrak{Q}}
\newcommand{\rfrak}{\mathfrak{R}}
\newcommand{\sfrak}{\mathfrak{S}}
\newcommand{\tfrak}{\mathfrak{T}}
\newcommand{\ufrak}{\mathfrak{U}}
\newcommand{\vfrak}{\mathfrak{V}}
\newcommand{\wfrak}{\mathfrak{W}}
\newcommand{\xfrak}{\mathfrak{X}}
\newcommand{\yfrak}{\mathfrak{Y}}
\newcommand{\zfrak}{\mathfrak{Z}}

% Mathscr
\newcommand{\ascr}{\mathscr{A}}
\newcommand{\bscr}{\mathscr{B}}
\newcommand{\cscr}{\mathscr{C}}
\newcommand{\dscr}{\mathscr{D}}
\newcommand{\escr}{\mathscr{E}}
\newcommand{\fscr}{\mathscr{F}}
\newcommand{\gscr}{\mathscr{G}}
\newcommand{\hscr}{\mathscr{H}}
\newcommand{\iscr}{\mathscr{I}}
\newcommand{\jscr}{\mathscr{J}}
\newcommand{\kscr}{\mathscr{K}}
\newcommand{\lscr}{\mathscr{L}}
\newcommand{\mscr}{\mathscr{M}}
\newcommand{\nscr}{\mathscr{N}}
\newcommand{\oscr}{\mathscr{O}}
\newcommand{\pscr}{\mathscr{P}}
\newcommand{\qscr}{\mathscr{Q}}
\newcommand{\rscr}{\mathscr{R}}
\newcommand{\sscr}{\mathscr{S}}
\newcommand{\tscr}{\mathscr{T}}
\newcommand{\uscr}{\mathscr{U}}
\newcommand{\vscr}{\mathscr{V}}
\newcommand{\wscr}{\mathscr{W}}
\newcommand{\xscr}{\mathscr{X}}
\newcommand{\yscr}{\mathscr{Y}}
\newcommand{\zscr}{\mathscr{Z}}

% Mathbb
\newcommand{\abb}{\mathbb{A}}
\newcommand{\bbb}{\mathbb{B}}
\newcommand{\cbb}{\mathbb{C}}
\newcommand{\dbb}{\mathbb{D}}
\newcommand{\ebb}{\mathbb{E}}
\newcommand{\fbb}{\mathbb{F}}
\newcommand{\gbb}{\mathbb{G}}
\newcommand{\hbb}{\mathbb{H}}
\newcommand{\ibb}{\mathbb{I}}
\newcommand{\jbb}{\mathbb{J}}
\newcommand{\kbb}{\mathbb{K}}
\newcommand{\lbb}{\mathbb{L}}
\newcommand{\mbb}{\mathbb{M}}
\newcommand{\nbb}{\mathbb{N}}
\newcommand{\obb}{\mathbb{O}}
\newcommand{\pbb}{\mathbb{P}}
\newcommand{\qbb}{\mathbb{Q}}
\newcommand{\rbb}{\mathbb{R}}
\newcommand{\sbb}{\mathbb{S}}
\newcommand{\tbb}{\mathbb{T}}
\newcommand{\ubb}{\mathbb{U}}
\newcommand{\vbb}{\mathbb{V}}
\newcommand{\wbb}{\mathbb{W}}
\newcommand{\xbb}{\mathbb{X}}
\newcommand{\ybb}{\mathbb{Y}}
\newcommand{\zbb}{\mathbb{Z}}


%===============================================================================
%=============================== SPECIAL SYMBOLS ===============================
%===============================================================================


% Orbit (group theory)
\newcommand{\orbit}{\mathcal{O}}
% Normal group
\newcommand{\normal}{\mathcal{N}}
% Indicator function
\newcommand{\indicator}{\mathbbm{1}}
% Laplace transform
\newcommand{\laplace}[1]{\mathcal{L}}
% Epsilon shorthand
\newcommand{\eps}{\varepsilon}
% Omega
\newcommand{\om}{\omega}
\newcommand{\Om}{\Omega}

%===============================================================================
%================================== OPERATORS ==================================
%===============================================================================


% Inverse exponent
\newcommand{\inv}[0]{^{-1}}
% Overline bar
\newcommand{\olsi}[1]{\,\overline{\!{#1}}}
% Less than or equal slanted
\newcommand{\seqs}{\leqslant}
% Greater or equal slanted
\newcommand{\geqs}{\geqslant}
% Subset or equal
\newcommand{\sub}{\subseteq}
% Proper subset
\newcommand{\prosub}{\subset}
% from
\newcommand{\from}{\leftarrow}

% Parantheses
\newcommand{\para}[1]{\left( #1 \right)}
% Curly Braces
\newcommand{\curl}[1]{\left\{ #1 \right\}}
% Brackets
\newcommand{\brac}[1]{\left[ #1 \right]}
% Angled Brackets
\newcommand{\ang}[1]{\left\langle #1 \right\rangle}
% Norm
\newcommand{\norm}[1]{\left\| #1 \right\|}

% Piece wise (use \\ between cases)
\newcommand{\piecewise}[1]{\begin{cases} #1 \end{cases}}

% Bold symbol shorthand
\newcommand{\bl}{\boldsymbol}

% Vertical space
\newcommand{\vs}[1]{\vspace{#1 pt}}
% Horizontal ertical space
\newcommand{\hs}[1]{\hspace{#1 pt}}


%===============================================================================
%============================== TEXT BASED SYMBOLS =============================
%===============================================================================


% Radians
\newcommand{\rad}{\text{rad}}
% Least Common Multiple
\newcommand{\lcm}{\text{lcm}}
% Automorphism
\newcommand{\Aut}{\text{Aut}}
% Variance
\newcommand{\var}{\text{Var}}
% Covariance
\newcommand{\cov}{\text{Cov}}
% Cofactor (matrix)
\newcommand{\cof}{\text{Cof}}
% Adjugate (matrix)
\newcommand{\adj}{\text{Adj}}
% Trace (matrix)
\newcommand{\tr}{\text{tr}}
% Standard deviation
\newcommand{\std}{\text{Std}}
% Correlation coefficient
\newcommand{\corr}{\text{Corr}}
% Sign
\newcommand{\sign}{\text{sign}}

% And text
\newcommand{\AND}{\text{ and }}
% Or text 
\newcommand{\OR}{\text{ or }}
% For text 
\newcommand{\FOR}{\text{ for }}
% If text
\newcommand{\IF}{\text{ if }}
% When text
\newcommand{\WHEN}{\text{ when }}
% Where text
\newcommand{\WHERE}{\text{ where }}
% Then text
\newcommand{\THEN}{\text{ then }}
% Such that text
\newcommand{\SUCHTHAT}{\text{ such that }}

% 1st
\newcommand{\st}[1]{#1^{\text{st}}}
% 2nd
\newcommand{\nd}[1]{#1^{\text{nd}}}
% 3rd
\newcommand{\rd}[1]{#1^{\text{rd}}}
% nth
\newcommand{\nth}[1]{#1^{\text{th}}}


%===============================================================================
%========================= PROBABILITY AND STATISTICS ==========================
%===============================================================================


% Permutation
\newcommand{\perm}[2]{{}^{#1}\!P_{#2}}
% Combination
\newcommand{\comb}[2]{{}^{#1}C_{#2}}

% Baye's risk
\newcommand{\risk}[1]{\mathscr{R}_{#1}}
% Baye's optimal risk
\newcommand{\riskOptimal}[1]{\mathscr{R}_{#1}^*}
% Baye's empirical risk
\newcommand{\riskEmpirical}[2]{\hat{\mathscr{R}}_{#1}^{#2}}


%===============================================================================
%=================================== CALCULUS ==================================
%===============================================================================


% d over d derivative
\newcommand{\dd}[2]{\dfrac{d#1}{d#2}}
% partial d over d derivative
\newcommand{\partialdd}[2]{\dfrac{\partial #1}{\partial #2}}
% delta d over d derivative
\newcommand{\deltadd}[2]{\dfrac{\Delta #1}{\Delta #2}}

% Integration between a and b
\newcommand{\integral}[4]{\int_{#1}^{#2} #3 \, #4}
% Integration in some space 
\newcommand{\boundIntegral}[2]{\int_{#1} #2 \, d#1}

% Limit
\newcommand{\limit}[3]{\lim_{#1 \to #2} #3}


%===============================================================================
%================================  BIG SYMBOLS  ================================
%===============================================================================


% Sum
\newcommand{\sumof}[2]{\sum_{#1}^{#2}}
% Product
\newcommand{\productof}[2]{\prod_{#1}^{#2}}
% Union
\newcommand{\unionof}[2]{\bigcup_{#1}^{#2}}
% Intersection
\newcommand{\intersectionof}[2]{\bigcap_{#1}^{#2}}
% Or
\newcommand{\orof}[2]{\bigvee_{#1}^{#2}}
% And
\newcommand{\andof}[2]{\bigwedge_{#1}^{#2}}


%===============================================================================
%=============================== LINEAR ALGEBRA ================================
%===============================================================================


% Bold vector arrow
\newcommand{\bv}[1]{\vec{\mathbf{#1}}}

% Matrix or vector (use // for column, & for row) 
% with brackets
\newcommand{\bmat}[1]{\begin{bmatrix} #1 \end{bmatrix}}
% Matrix or vector (use // for column, & for row) 
% with curved brackets
\newcommand{\pmat}[1]{\begin{pmatrix} #1 \end{pmatrix}}
% Matrix or vector (use // for column, & for row) 
% with lines on either side 
\newcommand{\lmat}[1]{\begin{vmatrix} #1 \end{vmatrix}}
% Matrix or vector (use // for column, & for row) 
% with curly braces
\newcommand{\cmat}[1]{\begin{Bmatrix} #1 \end{Bmatrix}}
% Matrix or vector (use // for column, & for row) 
% with no braces
\newcommand{\mat}[1]{\begin{matrix} #1 \end{matrix}}


%===============================================================================
%================================ LARGE OBJECTS ================================
%===============================================================================

% Multiple lines
\newcommand{\multiline}[1]{
\begin{align*}
    #1
\end{align*}
}

% Multiple lines with equation numbers
\newcommand{\eqmultiline}[1]{
\begin{align*}
    #1
\end{align*}
}

% Color
\newcommand{\colorText}[2]{
\begingroup
\color{#1}
    #2
\endgroup
}

% Centered figure
\newcommand{\centerFigure}[2]{
    \begin{figure}[h]
        \centering
            #1
        \caption{#2}
    \end{figure}
}

% Tikz figure
\newcommand{\tikzGraphic}[1]{
    \begin{center}
    \begin{tikzpicture}
        #1
    \end{tikzpicture}
    \end{center}
}

% Enumerate numbers (seperate by \item)
\newcommand{\numbers}{\textbf{\number*)}}

% Enumerate letters (seperate by \item)
\newcommand{\letters}{\textbf{\alph*)}}

\title{%
    \Huge Numerical Algorithms \\
    \Large Lecture VI \& VII
}
\date{2025-05-07}
\author{Michael Saba}

\begin{document}
\pagenumbering{gobble}
\maketitle
\newpage
\setlength{\parindent}{0pt}
\pagenumbering{arabic}

\subsection*{Orthogonal Groups
Lie Algabra}

As we know, the general linear group:
\[ GL_n(\F) \]
is the group made up of invertible
matrices in $\R^{n \times n}$. \\
The orthogonal group $O(n)$
and the special orthogonal group $SO(n)$
are both subgroups of the general linear
group when the field is that of the reals
$GL_n(\R)$ . \\

As we mentioned earlier,
the orthogonal group $O(n)$
is the group of orthogonal matrices:
\[ O(n) = \{ M \mid M \in \R^{n \times n}
\text{ and } MM^t = M^TM = I \} \]
Each orthogonal group has a
subgroup called the special
orthogonal group, which only contains
rotations (orthogonal matrices with
determinant $1$):
\[ SO(n) = \{ M \mid MM^t = M^TM = I
\text{ and } \det(M) = 1 \} \]
All orthogonal matrices have
a determinant with absolute value
equal to $1$,
but reflections have a determinant
of $-1$. \\

We also mentioned that the Lie Algebra
of a Lie (smooth) group is just the tangent
space at the identity element. \\

Note that the Lie Algebra of the orthiogonal
and special orthogonal groups of size $n$
are actually the same. \\

First, to find the Lie Algebra of $SO(n)$. \\
First we define some curve:
\[ \gamma: \R \to SO(n) \]
Which takes as input a parameter $t$,
and outputs rotations in $SO(n)$.
Note that we should have $\gamma(0) = t$,
and that $\gamma$ should be continuous
and differentiable. Alos note that:
\[ \gamma(t)^T\gamma(t) = I_n \]
Is the constraint that defines the
special orthogonal group
(we also need the determinant constraint,
which we will discuss later). \\

Now, earlier, when trying to find the
tangent space, we had a parametric
equation $f(t)$ that takes in parameters
and maps to points on the manifold. \\
However, $SO(n)$ is defined implicitely,
using a constraint: $f(t) = c$. \\
In the case of an implicit function,
we differentiate the constraint itself
rather than the function $\gamma$:
\[ \dd{}{t}\gamma(t)^T\gamma(t) = \dd{}{t}I_n \]
\[ (\gamma(t)^T)'\gamma(t)
+ \gamma(t)^T\gamma'(t) = 0 \]
Now recall that we want the tangent space
evaluated at the identity $I_n$,
mneaning that $t = 0$:
\[ (\gamma(0)^T)'\gamma(0)
+ \gamma(0)^T\gamma'(0) = 0  \]
\[ (\gamma(0)^T)' + \gamma'(0) = 0  \]
\[ (\gamma(0)^T)' - \gamma'(0)  \]
We now have a constraint for the Lie Algebra:
\[ (\gamma(0)^T)' = -\gamma'(0)  \]
The Lie Algebra is the set of skew-symmetric
matrices $M^T = -M$:
\[ so(n) = \{M \mid M \in \R^{n \times n}
\text{ and } M^T = -M \} \]
Notice that this is an implict definition
of the tangent space or Lie Algebra. \\

But what about the second constraint $\det(M) = 1$. \\
This constraint is actually redundant,
that means that if we differentiate this constraint,
it won't add any information or restrict the Lie
Algebra further. \\

You can add an explanation here. \\

What this implies is that the Lie Algebra
of $O(n)$ and $SO(n)$ are the same:
\[ so(n) = o(n) \]
Since the constraint $\det(M) = 1$
doesn't add anything. \\

We talked about interpolating
rotations in $\R^2$ earlier.
Given two rotations $R_0$ and $R_1$,
and some $t \in [0, 1]$:
\[ R(t,R_0,R_1) = 
e^{{R_0t\ln\para{{R_1}{R_0}\inv}}} \]
Recall that in order to get this,
we had to first go to the Lie Algebra
(go to the angles $\theta_0$ and $\theta_1$),
interpolate them,
then go back to the special orthogonal group. \\
We also did the same for complex numbers
that represent rotations. \\

Can we do the same for the Orthogonal
group elements? Interpolate between two of them?
The answer is that we can't in general. \\

Think about it this way;
if we have two elements $R_0$ and $R_1$
such that both have a positive determinant,
then they are just two rotations,
and interpolating from one rotation to the other
is simple enough. \\
If both have negative determinants,
then both are also rotations,
but in the reflected space,
so we can interpolate between the two. \\
But if $R_0$ and $R_1$ have opposite determinant,
one is in the reflected space,
and the other is not, so it is impossible
to find a smooth path between the two
(we would need to relect at some point,
which we can't do smoothly). \\

Analytically,
the reason why we can't do it is because
in the interpolation formula:
\[ R(t,R_0,R_1) = 
e^{{R_0t\ln\para{{R_1}{R_0}\inv}}} \]
We have ${R_1}{R_0}\inv$.
If the two matrices $R_0$ and $R_1$ have opposite
signs, then the determinant of ${R_1}{R_0}\inv$
is negative. \\
The logarithm is not defined for matrices
with negative determinants. \\

\newpage

\subsection*{Matrix Exponential}

We have already taken the exponential of some
specific matrices with desirable properties. 
But what if we have a general matrix? \\

First note that by the Maclauarin Series
of the exponential:
\[ e^{M} = \sum_i^\infty \dfrac{M^i}{i!} \]
So we just need a way to find $M^i$. \\

Notice that if $M$ is diagonizable,
we can write its eigendecomposition as such:
\[ M = P\Sigma P^{-1} \]
Where $P$ is the eigenspace of $M$
(matrix with eigenvector columns)
and $\Sigma$ is a diagonal matrix
with the eigenvalues $\lambda_i$
on the diagonal. \\

Now, to find $M^i$, note that:
\[ M^i = (P\Sigma P^{-1})^i \]
\[ M^i = (P\Sigma P^{-1})(P\Sigma P^{-1})\dots
(P\Sigma P^{-1})\]
\[ M^i = P\Sigma (P^{-1}P)\Sigma (P^{-1}P) \dots
(P^{-1}P)\Sigma P^{-1} = P\Sigma^i P\inv \]
Also notice that since $\Sigma$
is diagonal, taking its power just means
taking the power of the scalars on its main
diagonal. \\

We can thus conclude that:
\[ e^{M} = \sum_i^\infty \dfrac{M^i}{i!} 
= \sum_i^\infty \dfrac{(P\Sigma P\inv)^i}{i!}
= P\sum_i^\infty \para{\dfrac{(\Sigma)^i}{i!}} P\inv
= Pe^{\Sigma}P\inv \]
And this is how we can find the exponential
of any diagonizable matrix. \\

\newpage

\subsection*{Rotation Around Axis}

First note that if a matrix has the property:
\[ A^T = A \]
then it is symmetric (top right triangle
is equal to bottom left). \\
And if we have:
\[ A^T = -A \]
then it is skew symmetric (top right triangle
is the negation of the bottom left). \\

Suppose we want to generate the rotation
matrix in $\R^3$ for a rotation around some
axis $a$ (given by a vector with length $1$). \\

We can imagine that the vector we want to rotate,
$u$, and the axis around which we want
to rotate, $a$,
form a triangle.
We can then define an ODE for the trajectory
in terms of the angle or time parameter:
\[ \dfrac{du}{d\theta} = a \times u \]
Which we can do by looking at the image
and doing some trigonometry on the triangle. \\
We can also express the cross product as a matrix:
\[ \dfrac{du}{d\theta} = 
\matThree{0}{-a_z}{a_y}{a_z}{0}{-a_x}{-a_y}{a_x}{0}
u \]
We can all the matrix $[a]_\times$
We already know how to solve this using the
Maclaurin series of the exponential 
$e^{[a]_\times\theta}u(0)$,
where $u(0)$ is the initial condition. \\

Now note that $([a]_\times)^3 = -[a]_\times$,
which means that $([a]_\times)^6 = [a]_\times$,
which means that the matrix is cyclic. \\
Using that to our advantage,
we can reduce the Maclaurin series to
a formula with $\sin$ and $\cos$:
\[ e^{G\theta} = \sum_k^{\infty} 
\dfrac{(G\theta)^k}{k!}
= I + \sin(\theta)[a]_\times 
+ (1-\cos(\theta))([a]_\times)^2 \]
So this is the rotation matrix
that multiplies (and rotates) $u$
around the axis $a$. \\

If $\|u\| \neq 1$, then the length
could just be factored out,
and the formula would be more or less
the same. \\ 

Now we've defined the exponentional,
and as we know, the logarithm is the inverse
operation: takes as input:
\[ I + \sin(\theta)[a]_\times 
+ (1-\cos(\theta))([a]_\times)^2 \]
And should return $[a]_\times\theta$. \\

So how to write an explicit logarithm?
Last time, with a normal rotation matrix,
it was easy to extract $J\theta$,
since $J$ was always the same. \\
Now, we want to extract $\theta[a]_x$
in some other way. \\

We would need to first do:
\[ \dfrac{R - R^T}{2} = \sin(\theta)[a]_\times \]
This gets us a skew symmetric matrix $Q$
with $\sin(\theta)$ multiplying the entries.
Now notice that we have:
\[ Q = \sin(\theta)[a]_\times
= \matThree{0}{-\sin(\theta)a_z}
{\sin(\theta)a_y}{\sin(\theta)a_z}{0}
{-\sin(\theta)a_x}{-\sin(\theta)a_y}
{\sin(\theta)a_x}{0}\]
Now, we can extract $\sin(\theta)$
by just taking:
\[ Q_{12}^2 + Q_{13}^2 + Q_{23}^2
= (a_x^2 + a_y^2 + a_z^2)\sin^2(\theta) \]
We know that
\[(a_x^2 + a_y^2 + a_z^2) = 1\]
Since it is the length squared of $a$,
which we assumed had length $1$. \\
So now we have:
\[ |\sin(\theta)| 
= \sqrt{Q_{12}^2 + Q_{13}^2 + Q_{23}^2} \]
Which gets us $\theta$ for the logarithm. \\

We still need $[a]_x$ however. \\
Now that we have $|\sin(\theta)|$,
we can divide $Q = \sin(\theta)[a]_x$
by it. \\
This works except when $\sin(\theta) = 0$. \\

We can also interpolate axis rotations
in the same way as before:
\[ R(t,R_0,R_1) = 
e^{{R_0t\ln\para{{R_1}{R_0}\inv}}}
= e^{{R_0t\ln\para{{R_1}{R_0}^T}}} \]
Note that the two rotations $R_0$
and $R_1$ we are
interpolating between may be around different
axis, so this is a bit more complex than the $\R^2$
rotation, where we just interpolated
the angles. \\

Note that any rotation in $\R^3$
using Euler angles
$R(\alpha, \beta, \gamma)$
can be expressed as a single rotation
$R(\theta)$ around some axis $a$. \\

\newpage

\subsection*{Quaternions}

Quaternions are like complex numbers:
\[ q = q_0 + iq_1 + jq_2 + kq_3 \]
Where $i$, $j$ and $k$ are imaginary.

We have:
\[ ij = k \qquad ki = j \qquad jk = i \]
And we also have:
\[ i^2 = j^2 = k^2 = -1 \]
As they are all imaginary. \\

This tells us that they are anti-commutative:
\[ ij = -ji \qquad ki = -ik \qquad jk = -kj \]
Meaning flipping the order flips the sign. \\

We can multiply quaternions as such:
\[ qp = (q_0 + iq_1 + jq_2 + kq_3)
(p_0 + ip_1 + jp_2 + kp_3) \]
\[ qp = (q_0p_0 - q_1p_1 - q_2p_2 - q_3p_3) 
+ (q_1p_0 + q_0p_1 - q_3p_2 + q_2p_3)i \]
\[ + (q_2p_0 + q_3p_1 + q_0p_2 - q_1p_3)j 
+ (q_3p_0 - q_2p_1 + q_1p_2 + q_0p_3)k \]
Note that this is non-commutative. \\

We can also define the complement of a quaternion:
\[ \olsi{q} = q_0 - iq_1 - jq_2 - kq_3 \]
Such that:
\[ \sqrt{q\olsi{q}} 
= \sqrt{q_0^2 + q_1^2 + q_2^2 + q_3^2} = |q| \]
is the length of the quaternion. \\

We can also define the quaternion inverse $q\inv$
such that:
\[ qq\inv = q\inv q = 1 \]
That means that:
\[ q\inv = \dfrac{1}{q}
= \dfrac{\olsi{q}}{q\olsi{q}}
= \dfrac{\olsi{q}}{|q|^2} \]
is the inverse. \\

Any quaternion of length $1$ can be used
to represent a rotation in $\R^3$,
just as complex numbers of length $1$
could be used to represent any 
rotation in $\R^2$. \\
In particular, quaternions are used
to represent a rotation around an axis in $\R^3$.
However, since any rotation matrix in $\R^3$
can be represented as a single rotation
around some axis,
this means that quaternions can be used
to represent any rotation in $\R^3$. \\

So, to elaborate on that point,
a quaternion that represents a rotation
by angle $\theta$
around an axis $u$ where $\|u\| = 1$
(we always assume length of the axis is $1$)
can we written as:
\[ q(\theta) = \cos \para{\dfrac{\theta}{2}}
+ \sin \para{\dfrac{\theta}{2}}
((u_x)i + (u_y)j + (u_z)k) \]
(If it had length not equal to $1$,
we would have multiplied the formula by $|q|$). \\
We can sort of see parallels
betwene this and the complex number
formula in terms of $\cos$ and $\sin$,
where $\sin$ was the imaginary part and
$\cos$ was the real part. \\

Then, in order to rotate $v$
by around the axis $u$,
we would need to perform the following
operation:
\[ qp\olsi{q} \]
Where
\[ p = (v_x)i + (v_y)j + (v_z)k \]
is the quaternion used to represent
the vector $v$. \\

In fact, in order to derive the
formula for the rotation quaternion,
we can do exactly what we did for complex numbers:
\[ \dd{v}{\theta} = \dfrac{\theta}{2}
((u_x)i + (u_y)j + (u_z)k) \]
We solve this ODE. \\
Here:
\[ \dfrac{\theta}{2}
((u_x)i + (u_y)j + (u_z)k) \]
is the quaternion analogue to $u \times v$,
which we had when we were doing axis angles. \\
It is unclear to me why this is the analogue. \\

The solution will be:
\[ e^{\sfrac{\theta}{2}((u_x)i + (u_y)j + (u_z)k)} \]
where $(u_x)i + (u_y)j + (u_z)k$
is the imaginary part of the quaternion.
Using the Maclaurin series,
we get that:
\[ e^{\sfrac{\theta}{2}((u_x)i + (u_y)j + (u_z)k)}
= \sum_{i = 1}^{\infty}
\dfrac{(\sfrac{\theta}{2}
((u_x)i + (u_y)j + (u_z)k))^i}{i!} \]
Notice that $((u_x)i + (u_y)j + (u_z)k)^2 = -1$,
since we assumed that $\|u\| = 1$. \\
So there is going to be a cycle,
the exact same as with $i$ (since $i^2 = 1$),
so we get:
\[ e^{\sfrac{\theta}{2}((u_x)i + (u_y)j + (u_z)k)}
=  \cos \para{\dfrac{\theta}{2}}
+ \sin \para{\dfrac{\theta}{2}}
((u_x)i + (u_y)j + (u_z)k) \]
Which is what we said it would be earlier. \\

We can also interpolate between quaternions:
\[ q(t) = 
e^{{q_0t\ln\para{{q_1}{q_0}\inv}}} \]
This can be equivalently written as:
\[ q(t) = \frac{\sin((1 - t)\alpha)}{\sin \alpha} 
q_0 + \frac{\sin(t\alpha)}{\sin \alpha} q_1 \]
Where $\alpha$ is the angle between $q_0$
and $q_1$. \\

It's actually more expensive to apply
quaternion rotations than matrix ones. \\

We can define the Lie group of quaternion
rotation as:
\[ S^3 = \{ q \in \mathbb{H} \cong \R^4
\mid |q| = 1 \} \]
Where each element is a quaternion of
length $1$. \\

There is no isomorphism
between $S^3$ and $SO(3)$ however. \\
Think about it this way,
if we rotate $v$ around $u$ by $\theta$,
or rotate $v$ around $-u$ by $-\theta$,
both would be the same rotation in $SO(3)$,
but would be represented by seperate
quaternions in $S^3$. \\
Instead, we have a double cover of $SO(3)$;
basically two elements from $S^3$
map to each element of $SO(3)$.
We can think of them as like two seperate
isomorphisms. \\

The group $S^3$ also has a Lie Algebra:
\[ s^3 = \{ \dfrac{\theta}{2}((u_x)i + (u_y)j + (u_z)k) 
\mid \theta \in \R, \;  \|u\| = 1 \} \]
Notice that this just defined the tangent
plane at point $p = 1$. \\
The logarithm of a quaternion in $S^3$ takes it
to this Lie Algebra. \\

Note that this explains why
the quaternion inverse is 
$\dfrac{\olsi{q}}{|q|^2}$. \\
If the quaternion has length $1$,
this is just a rotation. \\
So $\olsi{q}$
will have a negated imaginary part:
\[ \cos \para{\dfrac{\theta}{2}}
- \sin \para{\dfrac{\theta}{2}}
((u_x)i + (u_y)j + (u_z)k) \]
Which can just be written as a rotation in 
the opposite direction of $q$:
\[\cos \para{-\dfrac{\theta}{2}}
+ \sin \para{-\dfrac{\theta}{2}}
((u_x)i + (u_y)j + (u_z)k) \]
Here dividing by $|q|^2$ does nothing. \\
On the other hand, if $|q| > 1$,
then the rotation also scales at the same time,
by a factor of $|q|$. \\
This is why we divide by $|q|^2$.
The first division by $|q|$ undoes the
scaling, and the second division by $|q|$
applies the inverse scaling. \\
So $\dfrac{\olsi{q}}{|q|^2}$
represents the inverse always. \\

\newpage

\subsection*{Rigid Transformations}

The group of rigid transformations,
called the special euclidian group,
is defined as:
\[ SE(m) = \left\{ f : \R^m \to 
\R^m \mid f(p) = Ap + a,\ A 
\in SO(m),\ a \in \R^m \right\} \]
It is basically a group that only includes
matrices that move objects around without scaling
or reflecting them,
allowing only rotations and translations. \\
It includes the special orthogonal group $SO(n)$
(rotations) in it as a subgroup. \\

We can likewise define the Euclidian group,
which is similar, but also allows reflections:
\[ E(m) = \left\{ f : \R^m \to 
\R^m \mid f(p) = Ap + a,\ A 
\in O(m),\ a \in \R^m \right\} \]
It includes the orthogonal group $O(n)$
(rotations and reflections) in it as a subgroup. \\

In both cases,
the group operation is function composition:
\[ (f_2 \circ f_1)(p) = R_2(R_1p + t_1) + t_2
= R_2R_1p + (R_2t_1 + t_2) \] 
Which is associative by definition of function
composition, and clearly both groups 
are closed under it
(the result of applying the operation
is an element of the group itself). \\

The identity is just the operation:
\[ f(p) = Ip + 0 \]
Which applies no rotation and no translation. \\

The inverse of $f$ is:
\[ q = Rp + t \implies  p = R\inv q - R\inv t \]
Which we can write as:
\[ f\inv(p) = R\inv p - R\inv t =
R^T p - R^T t \]
And it holds for both groups
as both are made up of orthogonal matrices. \\

And with that, we've shown that they're
both groups. \\

We can also define a group for the same
transformation, but represented
in homogeneous coordinates. \\
Everything we said thus far would apply,
but instead of having to manually define
inverse and identity,
note that they will just correspond
to the identity $I_{n+1}$
and the inverse $M^{-1}$
of the homogeneous matrix $M$ 
in $\R^{(n+1) \times (n+1)}$,
which makes things simpler. \\

\newpage

\end{document}