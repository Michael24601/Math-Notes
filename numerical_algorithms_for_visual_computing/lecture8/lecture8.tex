
\documentclass[12pt]{article}

\usepackage[margin=1in]{geometry}


%===============================================================================
%================================== PACKAGES ===================================
%===============================================================================

% For using float option H that places figures 
% exatcly where we want them
\usepackage{float}
% makes figure font bold
\usepackage{caption}
\captionsetup[figure]{labelfont=bf}
% For text generation
\usepackage{lipsum}
% For drawing
\usepackage{tikz}
% For smaller or equal sign and not divide sign
\usepackage{amssymb}
% For the diagonal fraction
\usepackage{xfrac}
% For enumerating exercise parts with letters instead of numbers
\usepackage{enumitem}
% For dfrac, which forces the fraction to be in display mode (large) e
% even in math mode (small)
\usepackage{amsmath}
% For degree sign
\usepackage{gensymb}
% For "\mathbb" macro
\usepackage{amsfonts}
% For positioning 
\usepackage{indentfirst}
\usetikzlibrary{shapes,positioning,fit,calc}
% for adjustwidth environment
\usepackage{changepage}
% for arrow on top
\usepackage{esvect}
% for mathbb 1
\usepackage{bbm}
% for mathsrc
\usepackage[mathscr]{eucal}
% For degree sign
\usepackage{gensymb}
% For quotes
\usepackage{csquotes}
% For vertical lines
\usepackage{mathtools}
% For cols
\usepackage{multicol}

% for tikz
\usepackage{pgfplots}
\pgfplotsset{compat=1.18}
\usepackage{amsmath}
\usepgfplotslibrary{groupplots}


%===============================================================================
%==================================== FONTS ====================================
%===============================================================================


% Mathcal
\newcommand{\acal}{\mathcal{A}}
\newcommand{\bcal}{\mathcal{B}}
\newcommand{\ccal}{\mathcal{C}}
\newcommand{\dcal}{\mathcal{D}}
\newcommand{\ecal}{\mathcal{E}}
\newcommand{\fcal}{\mathcal{F}}
\newcommand{\gcal}{\mathcal{G}}
\newcommand{\hcal}{\mathcal{H}}
\newcommand{\ical}{\mathcal{I}}
\newcommand{\jcal}{\mathcal{J}}
\newcommand{\kcal}{\mathcal{K}}
\newcommand{\lcal}{\mathcal{L}}
\newcommand{\mcal}{\mathcal{M}}
\newcommand{\ncal}{\mathcal{N}}
\newcommand{\ocal}{\mathcal{O}}
\newcommand{\pcal}{\mathcal{P}}
\newcommand{\qcal}{\mathcal{Q}}
\newcommand{\rcal}{\mathcal{R}}
\newcommand{\scal}{\mathcal{S}}
\newcommand{\tcal}{\mathcal{T}}
\newcommand{\ucal}{\mathcal{U}}
\newcommand{\vcal}{\mathcal{V}}
\newcommand{\wcal}{\mathcal{W}}
\newcommand{\xcal}{\mathcal{X}}
\newcommand{\ycal}{\mathcal{Y}}
\newcommand{\zcal}{\mathcal{Z}}

% Mathfrak
\newcommand{\afrak}{\mathfrak{A}}
\newcommand{\bfrak}{\mathfrak{B}}
\newcommand{\cfrak}{\mathfrak{C}}
\newcommand{\dfrak}{\mathfrak{D}}
\newcommand{\efrak}{\mathfrak{E}}
\newcommand{\ffrak}{\mathfrak{F}}
\newcommand{\gfrak}{\mathfrak{G}}
\newcommand{\hfrak}{\mathfrak{H}}
\newcommand{\ifrak}{\mathfrak{I}}
\newcommand{\jfrak}{\mathfrak{J}}
\newcommand{\kfrak}{\mathfrak{K}}
\newcommand{\lfrak}{\mathfrak{L}}
\newcommand{\mfrak}{\mathfrak{M}}
\newcommand{\nfrak}{\mathfrak{N}}
\newcommand{\ofrak}{\mathfrak{O}}
\newcommand{\pfrak}{\mathfrak{P}}
\newcommand{\qfrak}{\mathfrak{Q}}
\newcommand{\rfrak}{\mathfrak{R}}
\newcommand{\sfrak}{\mathfrak{S}}
\newcommand{\tfrak}{\mathfrak{T}}
\newcommand{\ufrak}{\mathfrak{U}}
\newcommand{\vfrak}{\mathfrak{V}}
\newcommand{\wfrak}{\mathfrak{W}}
\newcommand{\xfrak}{\mathfrak{X}}
\newcommand{\yfrak}{\mathfrak{Y}}
\newcommand{\zfrak}{\mathfrak{Z}}

% Mathscr
\newcommand{\ascr}{\mathscr{A}}
\newcommand{\bscr}{\mathscr{B}}
\newcommand{\cscr}{\mathscr{C}}
\newcommand{\dscr}{\mathscr{D}}
\newcommand{\escr}{\mathscr{E}}
\newcommand{\fscr}{\mathscr{F}}
\newcommand{\gscr}{\mathscr{G}}
\newcommand{\hscr}{\mathscr{H}}
\newcommand{\iscr}{\mathscr{I}}
\newcommand{\jscr}{\mathscr{J}}
\newcommand{\kscr}{\mathscr{K}}
\newcommand{\lscr}{\mathscr{L}}
\newcommand{\mscr}{\mathscr{M}}
\newcommand{\nscr}{\mathscr{N}}
\newcommand{\oscr}{\mathscr{O}}
\newcommand{\pscr}{\mathscr{P}}
\newcommand{\qscr}{\mathscr{Q}}
\newcommand{\rscr}{\mathscr{R}}
\newcommand{\sscr}{\mathscr{S}}
\newcommand{\tscr}{\mathscr{T}}
\newcommand{\uscr}{\mathscr{U}}
\newcommand{\vscr}{\mathscr{V}}
\newcommand{\wscr}{\mathscr{W}}
\newcommand{\xscr}{\mathscr{X}}
\newcommand{\yscr}{\mathscr{Y}}
\newcommand{\zscr}{\mathscr{Z}}

% Mathbb
\newcommand{\abb}{\mathbb{A}}
\newcommand{\bbb}{\mathbb{B}}
\newcommand{\cbb}{\mathbb{C}}
\newcommand{\dbb}{\mathbb{D}}
\newcommand{\ebb}{\mathbb{E}}
\newcommand{\fbb}{\mathbb{F}}
\newcommand{\gbb}{\mathbb{G}}
\newcommand{\hbb}{\mathbb{H}}
\newcommand{\ibb}{\mathbb{I}}
\newcommand{\jbb}{\mathbb{J}}
\newcommand{\kbb}{\mathbb{K}}
\newcommand{\lbb}{\mathbb{L}}
\newcommand{\mbb}{\mathbb{M}}
\newcommand{\nbb}{\mathbb{N}}
\newcommand{\obb}{\mathbb{O}}
\newcommand{\pbb}{\mathbb{P}}
\newcommand{\qbb}{\mathbb{Q}}
\newcommand{\rbb}{\mathbb{R}}
\newcommand{\sbb}{\mathbb{S}}
\newcommand{\tbb}{\mathbb{T}}
\newcommand{\ubb}{\mathbb{U}}
\newcommand{\vbb}{\mathbb{V}}
\newcommand{\wbb}{\mathbb{W}}
\newcommand{\xbb}{\mathbb{X}}
\newcommand{\ybb}{\mathbb{Y}}
\newcommand{\zbb}{\mathbb{Z}}


%===============================================================================
%=============================== SPECIAL SYMBOLS ===============================
%===============================================================================


% Orbit (group theory)
\newcommand{\orbit}{\mathcal{O}}
% Normal group
\newcommand{\normal}{\mathcal{N}}
% Indicator function
\newcommand{\indicator}{\mathbbm{1}}
% Laplace transform
\newcommand{\laplace}[1]{\mathcal{L}}
% Epsilon shorthand
\newcommand{\eps}{\varepsilon}
% Omega
\newcommand{\om}{\omega}
\newcommand{\Om}{\Omega}

%===============================================================================
%================================== OPERATORS ==================================
%===============================================================================


% Inverse exponent
\newcommand{\inv}[0]{^{-1}}
% Overline bar
\newcommand{\olsi}[1]{\,\overline{\!{#1}}}
% Less than or equal slanted
\newcommand{\seqs}{\leqslant}
% Greater or equal slanted
\newcommand{\geqs}{\geqslant}
% Subset or equal
\newcommand{\sub}{\subseteq}
% Proper subset
\newcommand{\prosub}{\subset}
% from
\newcommand{\from}{\leftarrow}

% Parantheses
\newcommand{\para}[1]{\left( #1 \right)}
% Curly Braces
\newcommand{\curl}[1]{\left\{ #1 \right\}}
% Brackets
\newcommand{\brac}[1]{\left[ #1 \right]}
% Angled Brackets
\newcommand{\ang}[1]{\left\langle #1 \right\rangle}
% Norm
\newcommand{\norm}[1]{\left\| #1 \right\|}

% Piece wise (use \\ between cases)
\newcommand{\piecewise}[1]{\begin{cases} #1 \end{cases}}

% Bold symbol shorthand
\newcommand{\bl}{\boldsymbol}

% Vertical space
\newcommand{\vs}[1]{\vspace{#1 pt}}
% Horizontal ertical space
\newcommand{\hs}[1]{\hspace{#1 pt}}


%===============================================================================
%============================== TEXT BASED SYMBOLS =============================
%===============================================================================


% Radians
\newcommand{\rad}{\text{rad}}
% Least Common Multiple
\newcommand{\lcm}{\text{lcm}}
% Automorphism
\newcommand{\Aut}{\text{Aut}}
% Variance
\newcommand{\var}{\text{Var}}
% Covariance
\newcommand{\cov}{\text{Cov}}
% Cofactor (matrix)
\newcommand{\cof}{\text{Cof}}
% Adjugate (matrix)
\newcommand{\adj}{\text{Adj}}
% Trace (matrix)
\newcommand{\tr}{\text{tr}}
% Standard deviation
\newcommand{\std}{\text{Std}}
% Correlation coefficient
\newcommand{\corr}{\text{Corr}}
% Sign
\newcommand{\sign}{\text{sign}}

% And text
\newcommand{\AND}{\text{ and }}
% Or text 
\newcommand{\OR}{\text{ or }}
% If text
\newcommand{\IF}{\text{ if }}
% When text
\newcommand{\WHEN}{\text{ when }}
% Then text
\newcommand{\THEN}{\text{ then }}

% 1st
\newcommand{\st}[1]{#1^{\text{st}}}
% 2nd
\newcommand{\nd}[1]{#1^{\text{nd}}}
% 3rd
\newcommand{\rd}[1]{#1^{\text{rd}}}
% nth
\newcommand{\nth}[1]{#1^{\text{th}}}


%===============================================================================
%========================= PROBABILITY AND STATISTICS ==========================
%===============================================================================


% Permutation
\newcommand{\perm}[2]{{}^{#1}\!P_{#2}}
% Combination
\newcommand{\comb}[2]{{}^{#1}C_{#2}}

% Baye's risk
\newcommand{\risk}[1]{\mathscr{R}_{#1}}
% Baye's optimal risk
\newcommand{\riskOptimal}[1]{\mathscr{R}_{#1}^*}
% Baye's empirical risk
\newcommand{\riskEmpirical}[2]{\hat{\mathscr{R}}_{#1}^{#2}}


%===============================================================================
%=================================== CALCULUS ==================================
%===============================================================================


% d over d derivative
\newcommand{\dd}[2]{\dfrac{d#1}{d#2}}
% partial d over d derivative
\newcommand{\partialdd}[2]{\dfrac{\partial #1}{\partial #2}}
% delta d over d derivative
\newcommand{\deltadd}[2]{\dfrac{\Delta #1}{\Delta #2}}

% Integration between a and b
\newcommand{\integral}[4]{\int_{#1}^{#2} #3 \, #4}
% Integration in some space 
\newcommand{\boundIntegral}[2]{\int_{#1} #2 \, d#1}

% Limit
\newcommand{\limit}[3]{\lim_{#1 \to #2} #3}


%===============================================================================
%================================  BIG SYMBOLS  ================================
%===============================================================================


% Sum
\newcommand{\sumof}[2]{\sum_{#1}^{#2}}
% Product
\newcommand{\productof}[2]{\prod_{#1}^{#2}}
% Union
\newcommand{\unionof}[2]{\bigcup_{#1}^{#2}}
% Intersection
\newcommand{\intersectionof}[2]{\bigcap_{#1}^{#2}}
% Or
\newcommand{\orof}[2]{\bigvee_{#1}^{#2}}
% And
\newcommand{\andof}[2]{\bigwedge_{#1}^{#2}}


%===============================================================================
%=============================== LINEAR ALGEBRA ================================
%===============================================================================


% Bold vector arrow
\newcommand{\bv}[1]{\vec{\mathbf{#1}}}

% Matrix or vector (use // for column, & for row) 
% with brackets
\newcommand{\bmat}[1]{\begin{bmatrix} #1 \end{bmatrix}}
% Matrix or vector (use // for column, & for row) 
% with curved brackets
\newcommand{\pmat}[1]{\begin{pmatrix} #1 \end{pmatrix}}
% Matrix or vector (use // for column, & for row) 
% with lines on either side 
\newcommand{\lmat}[1]{\begin{vmatrix} #1 \end{vmatrix}}
% Matrix or vector (use // for column, & for row) 
% with curly braces
\newcommand{\cmat}[1]{\begin{Bmatrix} #1 \end{Bmatrix}}
% Matrix or vector (use // for column, & for row) 
% with no braces
\newcommand{\mat}[1]{\begin{matrix} #1 \end{matrix}}


%===============================================================================
%================================ LARGE OBJECTS ================================
%===============================================================================

% Multiple lines
\newcommand{\multiline}[1]{
\begin{align*}
    #1
\end{align*}
}

% Multiple lines with equation numbers
\newcommand{\eqmultiline}[1]{
\begin{align*}
    #1
\end{align*}
}

% Color
\newcommand{\colorText}[2]{
\begingroup
\color{#1}
    #2
\endgroup
}

% Centered figure
\newcommand{\centerFigure}[2]{
    \begin{figure}[h]
        \centering
            #1
        \caption{#2}
    \end{figure}
}

% Tikz figure
\newcommand{\tikzGraphic}[1]{
    \begin{center}
    \begin{tikzpicture}
        #1
    \end{tikzpicture}
    \end{center}
}

% Enumerate numbers (seperate by \item)
\newcommand{\numbers}{\textbf{\number*)}}

% Enumerate letters (seperate by \item)
\newcommand{\letters}{\textbf{\alph*)}}

\title{%
    \Huge Numerical Algorithms \\
    \Large Lecture VIII
}
\date{2025-05-14}
\author{Michael Saba}

\begin{document}
\pagenumbering{gobble}
\maketitle
\newpage
\setlength{\parindent}{0pt}
\pagenumbering{arabic}

\subsection*{The Rendering Equation}

The fundamental quantity that we are working
with is radiance, $L$, which is basically light. \\

When a ray of light in direction $\omega_i$
hits a point $x$, it will be reflected in all directions.
Likewise, the outgoing reflected ray of light in
direction $\omega_o$ is the result of many incoming
rays of light that hit $x$ and reflect. \\

The rendering equation calculates the amount of
outgoing radiance or light from a point $x$
in the outgoing direction $\omega_o$:
\[ L_{o}(x, \omega_o)
= L_{e}(x, \omega_o)
+ \int_\Omega L_{i}(x, \omega_i) \cdot
f(x, \omega_i, \omega_o) \cdot
(\omega_i \cdot n) \; d\omega_i \]
Here, the outgoing radiance is the emitted light
(if the point is a source of light)
plus the light reflected at that point (from other
light sources). \\

So $L_{e}(x, \omega_o)$ is the emitted light 
from point $0$ in direction $\omega_o$. \\
It is $0$ on a non-light source, an non-zero
on a light source.
 
On the other hand, the integral:
\[ \int_\Omega L_{i}(x, \omega_i) \cdot
f(x, \omega_i, \omega_o) \cdot
(\omega_i \cdot n) \; d\omega_i \]
integrates over all incoming rays of light
of direction $\omega_i$ into point $x$.
Here $\Omega$ is the hemisphere above $x$. \\
So $L_{i}(x, \omega_i)$ is the incoming ray
of light in direction $-\omega_i$. \\
And $f(x, \omega_i, \omega_o)$ is the BRDF, 
which is the amount of light from direction $\omega_i$
that gets reflected in direction $\omega_o$
at point $x$.
This will depend on the material that $x$ is made of. \\
Finally, $\omega_i \cdot n$ is the dot product,
which basically gives us the angle $\theta$
between the incoming light direction and the normal
$n$ at $x$, which scales how much light is reflected
(Lambert's cosine law).
It's the cosine of the angle because we 
assume both directions $n$ and $\omega_i$
have length $1$. \\

Also note that we usually have an absolute value
around the dot product (wouldn't make sense to have
negative light). \\

The BRDF $f(x, \omega_i, \omega_o)$ depends on the
material type, and is defined by the user.

The dot product term depends $\omega_i \cdot n$
depends on the geometry of the scene. \\

Finally, the light coming in $L_{i}(x, \omega_i)$
is not known; it is itself calculated
as the outgoing ray from some point $y$
whose ray of light in direction $\omega_i$
hits $x$. \\

There is a small point that needs clarification. \\
By convention, when we say $L_i(x, \omega_i)$,
we actually mean the coming into $x$
from direction $-\omega_i$
($\omega_i$ shoots out of $x$ towards the source). \\
This is by convention, and it is a bit confusing. \\
My guess is that it creates a useful symmetry.
If we have:
\[ L_o(x, \omega) \qquad \AND \qquad L_i(x, \omega) \]
Here, $L_i(x, \omega)$ will come into $x$ on one side,
but instead of coming out the other side,
this $L_o(x, \omega)$ will go out from the same
direction (but opposite) that the light came in. \\
One intuition that helped me is to think of
the hemisphere $\Omega$.
If we think of the sum of the rays of light
in all directions $\Omega$,
both incidant and exitant:
\[ \int_\Omega L_i(x, \omega_i) \; d\omega_i
\qquad \AND \qquad 
\int_\Omega L_i(x, \omega_o) \; d\omega_o \]
Then both of these sums sum over the same
hemisphere; if the light $L_i(x, \omega_i)$
were coming from $\omega_i$,
then the lights would be coming from the
opposite side (other hemisphere). \\
So defining $L_i(x, \omega_i)$
such that the light is coming in from direction
$-\omega_i$ creates this useful
symmetry that allows us to use the same
hemisphere for incoming and outgoing light. \\

The scene, or geometry,
$\mcal$, is basically made up of surfaces defined
by some equations:
\[ \mcal = \unionof{i=1}{n} S_i \]
$S$ is the set of the surfaces. \\

The function $N:\mcal \to \rbb^d$
gives the normal $N(x)$ at any point $x$ on $\mcal$. \\

We can imagine that the screen on which we view the
scene is the sensor, that captures rays of light 
$L_o(x, \omega_o)$ and then showcases the correct pixel
color depending on which rays hit which pixels. \\
The rays originate in the light sources and bounce around
until they hit the sensor. \\
However, the probability that rays will hit the sensor
is very low, so we usually do the operation in the
opposite direction, and shoot light rays from the sensor
until they hit the light sources,
and then do the calculation in reverse. \\

We have a ray tracing function $x_\mcal(x, \omega)$
or $r(x, \omega)$, which basically shoots a ray
from $x$ in direction $\omega$. We use it extensively
in order to figure out where light rays go and come from. \\

As mentioned earlier, the incoming light $L_i(x, \omega)$
is the ray of light that leaves $y$ in the direction $-\omega$
and hits $x$.
We can write $y$ as $r(x,\omega)$
since we can get $y$ by shooting a ray in the direction
$\omega$ from $x$ (recall that $L_i(x, \omega_i)$
has an incoming ray in direction $-\omega_i$). \\
So we can write:
\[ L_i(x, \omega) = L_o(r(x, \omega), -\omega) \]
Which means that we write $L_i$,
the light coming into a point $x$ from some direction,
as the light $L_o$, coming out of a point in
the opposite direction towards $x$. \\

Energy needs to be conserved, since otherwise light would
bounce around forever and the scene keeps getting brighter. \\

The property:
\[ L_i(x, \omega) = L_o(r(x, \omega), -\omega) \]
Tells us we can write the rendering equation recursively
in terms of outgoing light only (using a ray tracer). 

The recursion is not infinite, we will eventually hit a
a point where $L_o = 0$, since energy is conserved,
which means each bounce decreases the energy. \\

To solve the recursive integral, we can just pretend
that we have matrices instead of an integral, 
and then use an infinite sum. \\
The sum we get, which converges, represent each bounce
the light makes. \\

When we set a color to a specific pixel, more than
one ray arrives at this pixel, so we would have to
average out all the rays to get the final color. \\

\newpage

\subsection*{Radiometry}

This section will expand on the idea of energy
and how it related to light and the rendering
equation. \\

Radiometry is the measure of visible light. \\

We can imagine that there exists a function
$Q$ which calculates the total energy $J$
on a subset of
$\pbb = \rbb \times \mcal \times \scal^2 \times \rbb^+$,
which respectively refer to time ($\rbb$),
the geometry ($\mcal$),
all directions ($\scal^2$, a full hemisphere),
and wavelength $(\rbb^+)$. \\

So for instance, if we have some
union of time inverals (a set of times):
\[ \tcal = \cup_i[t_i, t_{i+1}] \]
Some subset of points from the scene:
\[ \dcal \sub \mcal \]
Some subset of directions:
\[ \Omega \sub \scal^2 \]
And a union of wavelength intervals (a set of wavelengths):
\[ \Lambda = \cup_i[\lambda_i, \lambda_{i+1}] \]
Then the function $Q$ measures the engery
in this space: 
\[ Q(\tcal, \dcal, \Omega, \Lambda) \in [0, \infty] \] 
Which is given in Joules. \\

In order to calculate $Q$, we would need to
integrate over each of these sets:
\[ Q(\tcal, \dcal, \Omega, \Lambda)
= \int_\tcal \int_\dcal \int_{\scal^2} \int_\Lambda
L(t, x, \omega, \lambda) 
\; d\lambda \; d\sigma(\omega) \; dA(x) \; dt \]
The term $L(t, x, \omega, \lambda)$
is the radiance (light),
which can be the light coming out of $x$ at time $t$
in direction $\omega$ on wavelength $\lambda$. \\
Or it can also be the quantity of 
light incident at $x$ at time $t$
in direction $\omega$ on wavelength 
$\lambda$ (come in from direction $-\omega$). \\
So depending on what $L$ is
and whether it is $L_i$ or $L_o$,
then $Q$ either measures
the total incident or exitant (outgoing) energy. \\

This means that in order to get the energy
at any one point $x$, or at one time $t$,
we would have to differentiate to get rid
of an integral. \\

Note that technically, $L(t, x, \omega, \lambda)$
is a spectral radiance $L_\lambda$, 
since it is a function
of a wavelength $\lambda$. \\
This won't be used much since we will
ignore wavelengths and time for the most part. \\

We will ignore time and wavelengths,
and just focus on direction and surface area. \\
This gives us flux;
the total amount of incident or exitant
energy/radiance at all points $x$
on some surface area $\dcal$ in all directions
$\Omega$ (which is some subset of the sphere $\scal^2$,
not necessarily the top hemisphere).  \\
Recall that flux is the integral of a vector
field over a surface multiplied by the dot product
with the normal of the surface. \\ 

So we have the exitant flux:
\[ \Phi^{\to}(\dcal, \Omega)
= \int_\dcal \int_\Omega L(x \to \omega) \cos(\theta) \;
d\sigma(\omega) \; dA(x) \]
Which we can more familiarly write as:
\[ \Phi^{\to}(\dcal, \Omega) 
= \int_\dcal \int_\Omega L_o(x, \omega) 
(\omega \cdot N(x)) \; d\sigma(\omega) \; dA(x)  \]
Which is the total amount of radiance exitting
the surface area $\dcal$ in all directions 
in $\Omega$. \\

We also have the incidant flux:
\[ \Phi^{\from}(\dcal, \Omega)
= \int_\dcal \int_\Omega L(x \from \omega) \cos(\theta) \;
d\sigma(\omega) \; dA(x) \]
Which we can more familiarly write as:
\[ \Phi^{\to}(\dcal, \Omega) 
= \int_\dcal \int_\Omega L_i(x, \omega) 
(\omega \cdot N(x)) \; d\sigma(\omega) \; dA(x)  \]
Which is the total amount of radiance incident
at the surface area $\dcal$ from all directions 
in $\Omega$ (recall that $L_i(x, \omega)$ 
is the radiance incident at $x$ from direction $-\omega$,
that is, $\omega$ is the vector that points out of
$x$ in the same direction light is coming in). \\

As mentioned earlier, we can differentiate in order
to find the radiance quantities in terms of one
point $x$ or one direction $\omega$. \\

For a fixed point $x$, we have:
\[ B(x, \Omega) = \int_\Omega L(x \to \omega) \cos(\theta) \;
d\sigma(\omega) = \int_\Omega L_o(x, \omega) \cdot 
(\omega \cdot N(x)) \; d\sigma(\omega) \]
which is the total exitant light from one point $x$
in all directions $\Omega$. \\

For a fixed point $x$, we also have:
\[ E(x, \Omega) = \int_\Omega L(x \from \omega) \cos(\theta) \;
d\sigma(\omega) = \int_\Omega L_i(x, \omega) \cdot 
(\omega \cdot N(x)) \; d\sigma(\omega) \]
which is the total incident light at one point $x$
from all directions $\Omega$. \\
Note that this is the total incoming light
at a point $x$ that is used in the rendering
equation (but weighted by the BRDF). \\

We can also fix the direction $\omega$,
an integrate over all points $x \in \dcal$:
\[ I^{\to}(\dcal, \omega) 
= \int_\dcal L(x \to \omega) \cos(\theta) \;
dA(x) = \int_\dcal L_o(x, \omega) \cdot 
(\omega \cdot N(x)) \; dA(x) \]
which is the total exitant radiance from all points
on a surface $\dcal$ in one directions $\omega$. \\

If we fix the direction $\omega$
and integrate over all points $x \in \dcal$,
we can also define:
\[ I^{\from}(\dcal, \omega) 
= \int_\dcal L(x \from \omega) \cos(\theta) \;
dA(x) = \int_\dcal L_i(x, \omega) \cdot 
(\omega \cdot N(x)) \; dA(x) \]
which is the total indicent radiance at all points
on a surface $\dcal$ from one direction $\omega$
(the light rays coming in have direction $-\omega$
as discussed). \\

We won't use $I^{\to}$ and $I^{\from}$ much. \\

Note that, by the conservation of energy:
\[ \dfrac{B(x)}{E(x)} \leq 1 \]
For all directions $\omega \in \Omega$. \\
This just means that the total exitant radiance $B(x)$
can't exceed the incidant radiance $E(x)$
by the conservation of engery; some of it
is absorbed by the material, and with each bounce
the light makes, some of the energy dissapates. \\

The property:
\[ L_i(x, \omega) = L_o(r(x, \omega), -\omega) \]
that we saw earlier is the conservation of energy
along a ray, that is,
the radiance that leaves $y$ towards $x$
is the same as the incidant radiance
at $x$ from $y$. \\

Note that $L_i(x, \omega) \neq L_o(x, \omega)$,
since light leaving a point $x$ isn't
necessarily equivalent to the amount of light
entering it.
Outgoing light does depend on incident light,
but it also depends on things like emitted light $L_e$
and angle and surface as we saw in 
the rendering equation. \\

\newpage

\subsection*{The Rendering Equation as an Infinte Series}

As mentioned earlier, in order to render a scene,
we can imagine that we have a quad, which is a
grid of pixels, in the scene, which acts as
a sensor. When rays of light hit a pixel,
we average these rays of lights out,
and the result is the color that the pixel
shows. \\
Those rays of lights originate at sources, bounce
around the scene, and finally land on the sensor. 
However, because the likelihood that rays of
light will hit the sensor are low, we usually
do it the other way around,
and originate rays of light at the sensor,
and then bounce them around the scene until
they hit a light source, or dissappears into
the void.
We then calculate the radiance/light
backwards. \\

There is an issue with the rendering equation:
\[ L_{o}(x, \omega_o)
= L_{e}(x, \omega_o)
+ \int_\Omega L_{i}(x, \omega_i) \cdot
f(x, \omega_i, \omega_o) \cdot
(\omega_i \cdot n) \; d\omega_i \]
We define the light sources, 
which give us $L_{e}(x, \omega_o)$.
We also define the BRDF $f(x, \omega_i, \omega_o)$.
The geometry of the scene gives us $\omega_i \cdot n$. \\
But what about $L_{i}(x, \omega_i)$?
How can we calculate the incident light
at a point $x$? \\

Now, recall that we mentioned that:
\[ L_i(x, \omega) = L_o(r(x, \omega), -\omega) \]
Which means that the incoming light at a point
is just the outgoing light from the point
we get if we just follow the light ray
in the opposite direction until we hit
an intersection with a surface (light source
or otherwise). \\
This implies that we can write the rendering
equation as such:
\[ L_{o}(x, \omega_o)
= L_{e}(x, \omega_o)
+ \int_\Omega L_o(r(x, \omega_i), -\omega_i)  \cdot
f(x, \omega_i, \omega_o) \cdot
(\omega_i \cdot n) \; d\omega_i \]
Which means that the incident light at $x$
is just the outgoing (exitant) light 
from some point $r(x, \omega_i)$.
And that outgoing light $L_o(r(x, \omega_i), -\omega_i)$
can itself be calculated using the rendering
equation, since it by definition
gives us the value of the exitant randiance
in some direction $\omega_o$. \\

And this implies a sort of recursion,
where the outgoing light at one point $x$
depends on teh outgoings lights from other points $y$,
which themselves depend on the outgoing light
from other points... \\
This creates a recursion, where each recursion
is a light bounce on a surface. \\

This recursion is not infinite however.
As light bounces, it loses energy.
As mentioned before, by the conservation of energy,
not all light that arrives at a point will exit it. \\
So after each bounce, the radiance quantity reduces,
until eventually the amount of light it contributes
to the scene is negligible. \\

In practice, we don't solve the actual integrals,
which would be very expensive.
Instead, we takes some sample of directions
from $\Omega$ of size $n$, 
and discretize the integral as a sum. \\

Generally speaking, if we have an integral:
\[ f(x) = \integral{a}{b}{K(x, y)g(y)}{dy} \]
We can approximate it as a Riemann sum,
that is, discretize it
by taking a sample of size $n$ of columns,
which we multiply by small steps $\Delta y$:
\[ f(x_j) \sim \sumof{i = 1}{n} (K(x_j, y_i)g(y_i) \Delta y) \]
Note here that if we envision
a vector $g$ that contains all values of $g(y_i)$
for each $i \in \{1, 2, \dots n\}$,
then we can rewrite this as a matrix multiplication:
\[ f = K \cdot g \]
Where:
\[ g = \bmat{g(y_1) \\ g(y_2) \\ \vdots \\ g(y_n)} \]
And $K$ is the approximation of the kernel of the
integral (basically encodes the multiplication
by $K(x_j, y_i)$ and $\Delta y$).
The result is a vector $f$ that contains
all outputs $f(x_j)$:
\[ f = \bmat{f(x_1) \\ f(x_2) \\ \vdots \\ f(x_n)} \]
And this is how we discretize an integral and write
it as a matrix multiplication for several inputs. \\

This means we can now write the integral
as a matrix multiplication:
\[ L_o = L_e + T \cdot L_i \]
What we've done here is discretize the space $\Omega$ of
incidant directions $\omega_i$. \\

Here, $L_o$ is the vector of outgoing radiances 
in several directions $\omega_o^j$ from point $x$:
\[ L_o = \bmat{L_o(x, \omega_o^1) \\ L_o(x, \omega_o^2) \\
\vdots \\ L_o(x, \omega_o^n)} \]
Here, $L_e$ is the vector of emitted radiance 
in several directions $\omega_o^j$ from point $x$:
\[ L_e = \bmat{L_e(x, \omega_o^1) \\ L_e(x, \omega_o^2) \\
\vdots \\ L_o(x, \omega_o^n)} \]
As for $L_i$, it is the vector of incidant radiances
from several directions $w_i^j$:
\[ L_i = \bmat{L_i(x, \omega_i^1) \\ L_i(x, \omega_i^2) \\
\vdots \\ L_i(x, \omega_i^n)} \]
Finally, we have the square matrix $T$ 
of dimension $n \times n$, which is the approximation
of the intergal kernel,
and basically encodes the cosine of the angle
and the BRDF for each sample point $\omega_i^k$
and $\omega_o^j$. \\

So, at some point $x$, the light output
in several directions $\omega_o^j$,
is given by 
\[ L_o = L_e + T \cdot L_i \]
As we can recall, we can write $L_i$
as an outgoing radiance, so:
\[ L_o = L_e + T \cdot L_o \]
\[ L_o - T \cdot L_o = L_e \]
\[ (I - T)L_o = L_e \]
\[ L_o = (I - T)\inv L_e \]
We can expand $(I - T)\inv$ as a Neumann series:
\[ L_o = \para{\sum_{k = 0}^{\infty} T^k} L_e
= L_e + TL_e + T^2L_e \dots \]
Where each term represents a bounce of the light. \\

This makes a lot of intuitive sense;
the light starts by being emitted at some point
$x$ in several directions, which is given by $L_e$,
and then that light propagates throughout the scence,
bouncing on the geometry. Each time it bounces,
some light is reflected, which is given by $T^k L_e$.
Each term $T^k L_e$ is essentially a bounce. \\
Note that this a geometric series. \\

For numbers, a geometric series:
\[ \sumof{k = 0}{\infty} q^k \]
converges if and only if $|q| < 1$. \\

Likewise for a matrix, the sum
\[ \sum_{k = 0}^{\infty} T^k \]
converges if and only if the norm of the matrix
$\|T\| < 1$. \\

Here, the sum will converge because $\|T\| < 1$. \\
But more pertinitly, it is because of the conservation
of enegry; as light bounces around,
it loses more an more energy, which
means that the norm of the term $\|T L_e\|$
must be smaller than $\|L_e\|$,
and the norm of the term $\|T^2 L_e\|$ is even smaller... \\
This means that the sum converges.
Otherwise, the scene would keep getting more and more
light each bounce, which would never abate,
and it would just be awash with infinite light. \\

In fact, we mentioned earlier that each bounce of
the light corresponded to one regression in the rendering
equation, rewritten recursively:
\[ L_{o}(x, \omega_o)
= L_{e}(x, \omega_o)
+ \int_\Omega L_o(r(x, \omega_i), -\omega_i)  \cdot
f(x, \omega_i, \omega_o) \cdot
(\omega_i \cdot n) \; d\omega_i \]
So intuitively, each recursion here should correspond
to one term in the Neumann expansion.
We can show that this is the case. \\
Since we have the exitant light 
$L_o(r(x, \omega_i), -\omega_i)$
in the integral, we can replace it
by the rendering equation.
We have:
\[ L_o(r(x, \omega_i), -\omega_i)
= L_{e}(r(x, \omega_i), -\omega_i)
+ \int_\Omega L_i(r(x, \omega_i), \omega_j) \cdot
f(x, \omega_j, -\omega_i) \cdot
(\omega_j \cdot n) \; d\omega_j \]
But we now have another term
$L_o(r(x, \omega_i), \omega_j)$
which we can also write as an outgoing light,
which means it will also recurse. \\
We can now write the rendering equation as such:
\multiline{
L_{o}(x, \omega_o)
& = L_{e}(x, \omega_o) \\
& + \int_\Omega \bigg( L_e(r(x, \omega_i), -\omega_i)
+ f(x, \omega_i, \omega_o) \cdot 
(\omega_i \cdot n) \\
& + \int_\Omega 
\bigg( L_e(r(r(x, \omega_i), \omega_j), -\omega_j) 
+ f(x, \omega_j, -\omega_i) \cdot 
(\omega_j \cdot n) \\
& + \int_\Omega \bigg( 
L_e(r(r(r(x, \omega_i), \omega_j), \omega_k), -\omega_k)
+ f(x, \omega_k, -\omega_j) \cdot 
(\omega_k \cdot n) \\ 
& + \int_\Omega \dots \bigg)
\; d\omega_k \bigg) \; d\omega_j \bigg) \; d\omega_i }
If we were to discretize these integrals,
and then write them as a matrix multiplication,
we would get:
\[ L_o = L_e + T( L_e + T( L_e + T (L_e + \dots))) \]
\[ L_o = L_e + TL_e + T^2L_e + T^3L_e \dots \]
Which is the Neumann series we derived earlier. \\

\newpage

\end{document}