
\documentclass[12pt]{article}

\usepackage[margin=1in]{geometry}


%===============================================================================
%================================== PACKAGES ===================================
%===============================================================================

% For using float option H that places figures 
% exatcly where we want them
\usepackage{float}
% makes figure font bold
\usepackage{caption}
\captionsetup[figure]{labelfont=bf}
% For text generation
\usepackage{lipsum}
% For drawing
\usepackage{tikz}
% For smaller or equal sign and not divide sign
\usepackage{amssymb}
% For the diagonal fraction
\usepackage{xfrac}
% For enumerating exercise parts with letters instead of numbers
\usepackage{enumitem}
% For dfrac, which forces the fraction to be in display mode (large) e
% even in math mode (small)
\usepackage{amsmath}
% For degree sign
\usepackage{gensymb}
% For "\mathbb" macro
\usepackage{amsfonts}
% For positioning 
\usepackage{indentfirst}
\usetikzlibrary{shapes,positioning,fit,calc}
% for adjustwidth environment
\usepackage{changepage}
% for arrow on top
\usepackage{esvect}
% for mathbb 1
\usepackage{bbm}
% for mathsrc
\usepackage[mathscr]{eucal}
% For degree sign
\usepackage{gensymb}
% For quotes
\usepackage{csquotes}
% For vertical lines
\usepackage{mathtools}
% For cols
\usepackage{multicol}

% for tikz
\usepackage{pgfplots}
\pgfplotsset{compat=1.18}
\usepackage{amsmath}
\usepgfplotslibrary{groupplots}


%===============================================================================
%==================================== FONTS ====================================
%===============================================================================


% Mathcal
\newcommand{\acal}{\mathcal{A}}
\newcommand{\bcal}{\mathcal{B}}
\newcommand{\ccal}{\mathcal{C}}
\newcommand{\dcal}{\mathcal{D}}
\newcommand{\ecal}{\mathcal{E}}
\newcommand{\fcal}{\mathcal{F}}
\newcommand{\gcal}{\mathcal{G}}
\newcommand{\hcal}{\mathcal{H}}
\newcommand{\ical}{\mathcal{I}}
\newcommand{\jcal}{\mathcal{J}}
\newcommand{\kcal}{\mathcal{K}}
\newcommand{\lcal}{\mathcal{L}}
\newcommand{\mcal}{\mathcal{M}}
\newcommand{\ncal}{\mathcal{N}}
\newcommand{\ocal}{\mathcal{O}}
\newcommand{\pcal}{\mathcal{P}}
\newcommand{\qcal}{\mathcal{Q}}
\newcommand{\rcal}{\mathcal{R}}
\newcommand{\scal}{\mathcal{S}}
\newcommand{\tcal}{\mathcal{T}}
\newcommand{\ucal}{\mathcal{U}}
\newcommand{\vcal}{\mathcal{V}}
\newcommand{\wcal}{\mathcal{W}}
\newcommand{\xcal}{\mathcal{X}}
\newcommand{\ycal}{\mathcal{Y}}
\newcommand{\zcal}{\mathcal{Z}}

% Mathfrak
\newcommand{\afrak}{\mathfrak{A}}
\newcommand{\bfrak}{\mathfrak{B}}
\newcommand{\cfrak}{\mathfrak{C}}
\newcommand{\dfrak}{\mathfrak{D}}
\newcommand{\efrak}{\mathfrak{E}}
\newcommand{\ffrak}{\mathfrak{F}}
\newcommand{\gfrak}{\mathfrak{G}}
\newcommand{\hfrak}{\mathfrak{H}}
\newcommand{\ifrak}{\mathfrak{I}}
\newcommand{\jfrak}{\mathfrak{J}}
\newcommand{\kfrak}{\mathfrak{K}}
\newcommand{\lfrak}{\mathfrak{L}}
\newcommand{\mfrak}{\mathfrak{M}}
\newcommand{\nfrak}{\mathfrak{N}}
\newcommand{\ofrak}{\mathfrak{O}}
\newcommand{\pfrak}{\mathfrak{P}}
\newcommand{\qfrak}{\mathfrak{Q}}
\newcommand{\rfrak}{\mathfrak{R}}
\newcommand{\sfrak}{\mathfrak{S}}
\newcommand{\tfrak}{\mathfrak{T}}
\newcommand{\ufrak}{\mathfrak{U}}
\newcommand{\vfrak}{\mathfrak{V}}
\newcommand{\wfrak}{\mathfrak{W}}
\newcommand{\xfrak}{\mathfrak{X}}
\newcommand{\yfrak}{\mathfrak{Y}}
\newcommand{\zfrak}{\mathfrak{Z}}

% Mathscr
\newcommand{\ascr}{\mathscr{A}}
\newcommand{\bscr}{\mathscr{B}}
\newcommand{\cscr}{\mathscr{C}}
\newcommand{\dscr}{\mathscr{D}}
\newcommand{\escr}{\mathscr{E}}
\newcommand{\fscr}{\mathscr{F}}
\newcommand{\gscr}{\mathscr{G}}
\newcommand{\hscr}{\mathscr{H}}
\newcommand{\iscr}{\mathscr{I}}
\newcommand{\jscr}{\mathscr{J}}
\newcommand{\kscr}{\mathscr{K}}
\newcommand{\lscr}{\mathscr{L}}
\newcommand{\mscr}{\mathscr{M}}
\newcommand{\nscr}{\mathscr{N}}
\newcommand{\oscr}{\mathscr{O}}
\newcommand{\pscr}{\mathscr{P}}
\newcommand{\qscr}{\mathscr{Q}}
\newcommand{\rscr}{\mathscr{R}}
\newcommand{\sscr}{\mathscr{S}}
\newcommand{\tscr}{\mathscr{T}}
\newcommand{\uscr}{\mathscr{U}}
\newcommand{\vscr}{\mathscr{V}}
\newcommand{\wscr}{\mathscr{W}}
\newcommand{\xscr}{\mathscr{X}}
\newcommand{\yscr}{\mathscr{Y}}
\newcommand{\zscr}{\mathscr{Z}}

% Mathbb
\newcommand{\abb}{\mathbb{A}}
\newcommand{\bbb}{\mathbb{B}}
\newcommand{\cbb}{\mathbb{C}}
\newcommand{\dbb}{\mathbb{D}}
\newcommand{\ebb}{\mathbb{E}}
\newcommand{\fbb}{\mathbb{F}}
\newcommand{\gbb}{\mathbb{G}}
\newcommand{\hbb}{\mathbb{H}}
\newcommand{\ibb}{\mathbb{I}}
\newcommand{\jbb}{\mathbb{J}}
\newcommand{\kbb}{\mathbb{K}}
\newcommand{\lbb}{\mathbb{L}}
\newcommand{\mbb}{\mathbb{M}}
\newcommand{\nbb}{\mathbb{N}}
\newcommand{\obb}{\mathbb{O}}
\newcommand{\pbb}{\mathbb{P}}
\newcommand{\qbb}{\mathbb{Q}}
\newcommand{\rbb}{\mathbb{R}}
\newcommand{\sbb}{\mathbb{S}}
\newcommand{\tbb}{\mathbb{T}}
\newcommand{\ubb}{\mathbb{U}}
\newcommand{\vbb}{\mathbb{V}}
\newcommand{\wbb}{\mathbb{W}}
\newcommand{\xbb}{\mathbb{X}}
\newcommand{\ybb}{\mathbb{Y}}
\newcommand{\zbb}{\mathbb{Z}}


%===============================================================================
%=============================== SPECIAL SYMBOLS ===============================
%===============================================================================


% Orbit (group theory)
\newcommand{\orbit}{\mathcal{O}}
% Normal group
\newcommand{\normal}{\mathcal{N}}
% Indicator function
\newcommand{\indicator}{\mathbbm{1}}
% Laplace transform
\newcommand{\laplace}[1]{\mathcal{L}}
% Epsilon shorthand
\newcommand{\eps}{\varepsilon}
% Omega
\newcommand{\om}{\omega}
\newcommand{\Om}{\Omega}

%===============================================================================
%================================== OPERATORS ==================================
%===============================================================================


% Inverse exponent
\newcommand{\inv}[0]{^{-1}}
% Overline bar
\newcommand{\olsi}[1]{\,\overline{\!{#1}}}
% Less than or equal slanted
\newcommand{\seqs}{\leqslant}
% Greater or equal slanted
\newcommand{\geqs}{\geqslant}
% Subset or equal
\newcommand{\sub}{\subseteq}
% Proper subset
\newcommand{\prosub}{\subset}
% from
\newcommand{\from}{\leftarrow}

% Parantheses
\newcommand{\para}[1]{\left( #1 \right)}
% Curly Braces
\newcommand{\curl}[1]{\left\{ #1 \right\}}
% Brackets
\newcommand{\brac}[1]{\left[ #1 \right]}
% Angled Brackets
\newcommand{\ang}[1]{\left\langle #1 \right\rangle}
% Norm
\newcommand{\norm}[1]{\left\| #1 \right\|}

% Piece wise (use \\ between cases)
\newcommand{\piecewise}[1]{\begin{cases} #1 \end{cases}}

% Bold symbol shorthand
\newcommand{\bl}{\boldsymbol}

% Vertical space
\newcommand{\vs}[1]{\vspace{#1 pt}}
% Horizontal ertical space
\newcommand{\hs}[1]{\hspace{#1 pt}}


%===============================================================================
%============================== TEXT BASED SYMBOLS =============================
%===============================================================================


% Radians
\newcommand{\rad}{\text{rad}}
% Least Common Multiple
\newcommand{\lcm}{\text{lcm}}
% Automorphism
\newcommand{\Aut}{\text{Aut}}
% Variance
\newcommand{\var}{\text{Var}}
% Covariance
\newcommand{\cov}{\text{Cov}}
% Cofactor (matrix)
\newcommand{\cof}{\text{Cof}}
% Adjugate (matrix)
\newcommand{\adj}{\text{Adj}}
% Trace (matrix)
\newcommand{\tr}{\text{tr}}
% Standard deviation
\newcommand{\std}{\text{Std}}
% Correlation coefficient
\newcommand{\corr}{\text{Corr}}
% Sign
\newcommand{\sign}{\text{sign}}

% And text
\newcommand{\AND}{\text{ and }}
% Or text 
\newcommand{\OR}{\text{ or }}
% For text 
\newcommand{\FOR}{\text{ for }}
% If text
\newcommand{\IF}{\text{ if }}
% When text
\newcommand{\WHEN}{\text{ when }}
% Where text
\newcommand{\WHERE}{\text{ where }}
% Then text
\newcommand{\THEN}{\text{ then }}
% Such that text
\newcommand{\SUCHTHAT}{\text{ such that }}

% 1st
\newcommand{\st}[1]{#1^{\text{st}}}
% 2nd
\newcommand{\nd}[1]{#1^{\text{nd}}}
% 3rd
\newcommand{\rd}[1]{#1^{\text{rd}}}
% nth
\newcommand{\nth}[1]{#1^{\text{th}}}


%===============================================================================
%========================= PROBABILITY AND STATISTICS ==========================
%===============================================================================


% Permutation
\newcommand{\perm}[2]{{}^{#1}\!P_{#2}}
% Combination
\newcommand{\comb}[2]{{}^{#1}C_{#2}}

% Baye's risk
\newcommand{\risk}[1]{\mathscr{R}_{#1}}
% Baye's optimal risk
\newcommand{\riskOptimal}[1]{\mathscr{R}_{#1}^*}
% Baye's empirical risk
\newcommand{\riskEmpirical}[2]{\hat{\mathscr{R}}_{#1}^{#2}}


%===============================================================================
%=================================== CALCULUS ==================================
%===============================================================================


% d over d derivative
\newcommand{\dd}[2]{\dfrac{d#1}{d#2}}
% partial d over d derivative
\newcommand{\partialdd}[2]{\dfrac{\partial #1}{\partial #2}}
% delta d over d derivative
\newcommand{\deltadd}[2]{\dfrac{\Delta #1}{\Delta #2}}

% Integration between a and b
\newcommand{\integral}[4]{\int_{#1}^{#2} #3 \, #4}
% Integration in some space 
\newcommand{\boundIntegral}[2]{\int_{#1} #2 \, d#1}

% Limit
\newcommand{\limit}[3]{\lim_{#1 \to #2} #3}


%===============================================================================
%================================  BIG SYMBOLS  ================================
%===============================================================================


% Sum
\newcommand{\sumof}[2]{\sum_{#1}^{#2}}
% Product
\newcommand{\productof}[2]{\prod_{#1}^{#2}}
% Union
\newcommand{\unionof}[2]{\bigcup_{#1}^{#2}}
% Intersection
\newcommand{\intersectionof}[2]{\bigcap_{#1}^{#2}}
% Or
\newcommand{\orof}[2]{\bigvee_{#1}^{#2}}
% And
\newcommand{\andof}[2]{\bigwedge_{#1}^{#2}}


%===============================================================================
%=============================== LINEAR ALGEBRA ================================
%===============================================================================


% Bold vector arrow
\newcommand{\bv}[1]{\vec{\mathbf{#1}}}

% Matrix or vector (use // for column, & for row) 
% with brackets
\newcommand{\bmat}[1]{\begin{bmatrix} #1 \end{bmatrix}}
% Matrix or vector (use // for column, & for row) 
% with curved brackets
\newcommand{\pmat}[1]{\begin{pmatrix} #1 \end{pmatrix}}
% Matrix or vector (use // for column, & for row) 
% with lines on either side 
\newcommand{\lmat}[1]{\begin{vmatrix} #1 \end{vmatrix}}
% Matrix or vector (use // for column, & for row) 
% with curly braces
\newcommand{\cmat}[1]{\begin{Bmatrix} #1 \end{Bmatrix}}
% Matrix or vector (use // for column, & for row) 
% with no braces
\newcommand{\mat}[1]{\begin{matrix} #1 \end{matrix}}


%===============================================================================
%================================ LARGE OBJECTS ================================
%===============================================================================

% Multiple lines
\newcommand{\multiline}[1]{
\begin{align*}
    #1
\end{align*}
}

% Multiple lines with equation numbers
\newcommand{\eqmultiline}[1]{
\begin{align*}
    #1
\end{align*}
}

% Color
\newcommand{\colorText}[2]{
\begingroup
\color{#1}
    #2
\endgroup
}

% Centered figure
\newcommand{\centerFigure}[2]{
    \begin{figure}[h]
        \centering
            #1
        \caption{#2}
    \end{figure}
}

% Tikz figure
\newcommand{\tikzGraphic}[1]{
    \begin{center}
    \begin{tikzpicture}
        #1
    \end{tikzpicture}
    \end{center}
}

% Enumerate numbers (seperate by \item)
\newcommand{\numbers}{\textbf{\number*)}}

% Enumerate letters (seperate by \item)
\newcommand{\letters}{\textbf{\alph*)}}

\title{%
    \Huge Numerical Algorithms \\
    \Large Lecture V
}
\date{2025-05-03}
\author{Michael Saba}

\begin{document}
\pagenumbering{gobble}
\maketitle
\newpage
\setlength{\parindent}{0pt}
\pagenumbering{arabic}

\subsection*{Complex Numbers}

We will now introduce complex numbers. \\
As we know, a complex number $z \in \cbb$
can be written as:
\[ z = a + ib \]
for some $a, b \in \rbb$. \\
We can define addition and multiplication:
\[ z_1 = a + ib \qquad z_2 = c + id \]
where:
\[ z_1 + z_2 = (a + c) + i(b + d) \]
and:
\[ z_1 \cdot z_2 = 
(a + ib)(c + id) = 
(ac - bd) + i(ad + cd) \]
Recall that $i^2 = -1$. \\

We can image complex numbers as living
in a two dimensional coordinate space
where the $y$-coordinate is the imaginary
part of the number. \\
We can thus define the norm or length
of the complex number to 
be $\|z\|$, where:
\[ \|z\|^2 = z \cdot \olsi{z}
= (a + ib)(a - ib) = a^2 + b^2 \]
We call $\olsi{z}$ the complement of $z$. \\

Finally, we want to define an inverse
$z\inv$ for a complex number, such that
$z \cdot z\inv = 1$. \\
First note that since $z \cdot \olsi{z}$
is the length squared of $z$, then:
\[ \dfrac{z \cdot \olsi{z}}{\|z\|^2} = 1 \]
Which means that:
\[ z \cdot z\inv = 
\dfrac{z \cdot \olsi{z}}{\|z\|^2} \]
Then dividing both sides by $z$:
\[ z\inv = 
\dfrac{\olsi{z}}{\|z\|^2} \]
Which is our inverse. \\

\newpage

\subsection*{Ordinary Differential Equations}

Last lecture we arrived at:
\[ \dd{u}{\theta} = Ju \]
where $J$ is a rotation matrix:
\[ J = \matTwo{0}{1}{-1}{0} \]
which rotates by $90$ degrees. \\
We will now see how to solve this. \\

First let us note that $J$,
despite being a matrix, is actually
just a constant. \\
We can build up to solving this ODE
by first solving the same ODE without
the matrix. \\

If we have:
\[ \dd{y}{x} = \alpha y \]
where $\alpha$ is just a constant number,
then the solution to the ODE is just:
\[ y = e^{\alpha x}y(0) \]
where $y(0)$ is the initial condition. \\
This makes sense, since a function
that equals a scaled version of its derivative
is just the exponential. \\

Now we want to solve this equation:
\[ \dd{z}{\theta} = i \theta \]
Again, $i$ here is just a constant,
so the solution is:
\[ y = e^{i\theta}z(0) \]
So what is $e{i\theta}$? \\

We can use the Macluarin series of the
exponential to solve that. \\
We know that:
\[ e^x = \sum_{k = 0}^{\infty} \dfrac{x^k}{k!} \]
Which means that:
\[ e^{i\theta} = \sum_{k}^{\infty} 
\dfrac{(i\theta)^k}{k!} 
= 1 + ix + \dfrac{i^2\theta^2}{2!}
+ \dfrac{i^3\theta^3}{3!} + \dfrac{i^4\theta^4}{4!}
+ \dfrac{i^5\theta^5}{5!} \dots \]
\[ e^{i\theta}
= 1 + i\theta + -\dfrac{\theta^2}{2!}
+ -i\dfrac{\theta^3}{3!} + \dfrac{\theta^4}{4!}
+ i\dfrac{\theta^5}{5!} \dots \]
We can then split the even and odd
terms since only odd terms have the $i$ factor:
\[ e^{i\theta}
= 1 -\dfrac{\theta^2}{2!}
+ \dfrac{\theta^4}{4!} - \dfrac{\theta^6}{6!} + \dots
+ i \para{\theta - \dfrac{\theta^3}{3!}
+ \dfrac{\theta^5}{5!} - \dfrac{\theta^7}{7!} + \dots} \]
\[ e^{i\theta}
= \sum_{k = 0}^{\infty}
(-1)^{k}\dfrac{\theta^{2k}}{(2k)!}
+ i\sum_{k = 0}^{\infty} (-1)^{k}
\dfrac{\theta^{2k + 1}}{(2k + 1)!} \]
Notice that these are just the Maclaurin
series of $\cos$ and $\sin$:
\[ e^{i\theta} = \cos(\theta) + i\sin(\theta) \]
Which is Euler's identity. \\

Since we imagines the imaginary part as being
the $y$-axis and the real part as being
the $x$-axis,
then $e^{i\theta} = \cos(\theta) + i\sin(\theta)$
is just the parametric definition of a circle,
where $e^{i\theta}$ is just the point
you'd get by rotating $\theta$ degrees
on the unit circle starting at $1 + i0$. \\

In fact, we can define a group:
\[ S = \{ z \mid \|z\| = 1 \} \]
Which is the set of all complex numbers
on the unit circle. \\
It is a Lie group. \\
Notice that every element
in $S$ if of the form $e^{i\theta}$. \\
It can be easy to check that this forms a group.
The group operation is complex number multiplication.
The inverses are just complex number inverses,
and the identity is the real number $1$. \\

Note that complex numbers only represent a
rotation when their length is $1$,
as the Lie group above implies. \\

Now we are ready to solve the ODE:
\[ \dd{u}{\theta} = J\theta \]
We already know the solution will be:
\[ e^{J\theta}u(0) \]
where $u(0)$ is just the starting condition. \\
But what does it mean to take the power of a
matrix? 
We again use the Taylor Series:
\[ e^{J\theta} = \sum_{k}^{\infty} 
\dfrac{(J\theta)^k}{k!} 
= J^0 + Jx + \dfrac{J^2\theta^2}{2!}
+ \dfrac{J^3\theta^3}{3!} + \dfrac{J^4\theta^4}{4!}
+ \dfrac{J^5\theta^5}{5!} \dots \]
Now notice that as we take the power of the
matrix, it cycles:
\[ J^0 = \matTwo{1}{0}{0}{1} = I \qquad
J = \matTwo{0}{1}{-1}{0}
\qquad J^2 = \matTwo{-1}{0}{0}{-1} = -I \]
\[ J^3 = \matTwo{0}{-1}{1}{0}
\qquad J^4 = \matTwo{1}{0}{0}{1} = I \]
So as we can see, $J$ cycles;
every $4$ powers, it goes back to $J$,
which means that $J^4 = I$ is the identity.
So:
\[ e^{J\theta} =
= I + Jx -I\dfrac{\theta^2}{2!}
- IJ\dfrac{\theta^3}{3!} + I\dfrac{J^4\theta^4}{4!}
+ IJ\dfrac{J^5\theta^5}{5!} \dots \]
We can then group the odd and even powers
together (since only the odd powers have $J$
as a factor):
\[ e^{J\theta}
= I\sum_{k = 0}^{\infty} 
(-1)^{k} \dfrac{\theta^{2k}}{(2k)!}
+ J\sum_{k = 0}^{\infty} (-1)^{k}
\dfrac{\theta^{2k + 1}}{(2k + 1)!} \]
\[ e^{J\theta} = I\cos(\theta) + J\sin(\theta)
= \matTwo{\cos(\theta)}{\sin(\theta)}
{-\sin(\theta)}{\cos(\theta)} \]
So the result is just a rotation matrix
by angle $\theta$. \\

Notice that the result we got is functionally
the same as the one we got when solving the
ODE with $i$ in it,
as both the rotation matrix $R(\theta)$
and $e^{i\theta}$ represent rotations. \\
In fact, the matrix $J$ and $i$
both represent rotations by $90$ degrees. \\

We can conclude that there exists
an isomorphism between the group $S$
of complex numbers $e^{i\theta}$
on the unit circle, and the group of
all rotation matrices $R(\theta)$
in $\rbb^2$ (the special orthogonal group):
\[ \phi: S \to SO(2) \]
It's easy to see why this is an isomorphism.
It's also clearly a homomorphism,
since:
\[ \phi(e^{i\theta}e^{i\gamma})
= \phi(e^{i\theta \gamma})
= R(\theta \gamma) = R(\theta)R(\gamma) \]
It's also clearly a bijection. \\

\newpage

\subsection*{Interpolation}

Given two points $z_0$ and $z_1$ in $S$,
on the unit circle, how can we interpolate
between the two (find a general formula
for a point between $z_0$ and $z_1$)? \\

First we note that both can be written as:
\[ z_0 = e^{i\theta_0} \AND
z_1 = e^{i\theta_1} \]
We can interpolate between the two
by interpolating between their angles,
which we can get by doing:
\[ \ln(z_0) = \theta_0 \AND 
\ln(z_1) = \theta_1 \]
Then to interpolate between them:
\[ (1-t)\ln(z_0) + t\ln(z_1)
= \ln(z_0) -t\ln(z_0) + t\ln(z_1) 
= \ln(z_0) + t\ln\para{\dfrac{z_1}{z_0}} \]
where $t \in [0, 1]$. \\
Note that:
\[ \dfrac{z_1}{z_0} = z_1z_0\inv \]
Now that we have an interpolated angle,
we can go back to our point:
\[ \exp\para{\ln(z_0) + t\ln\para{z_1z_0\inv}}
= \exp(\ln(z_0))
\exp\para{{t\ln\para{z_1z_0\inv}} }
= e^{{z_0t\ln\para{z_1z_0\inv}}} \]
Which is our interpolated point. \\

We can do the exact same interpolation
given two rotations $R_0$ and $R_1$:
\[ e^{{R_0t\ln\para{{R_1}{R_0}\inv}}} \]
Here we define:
\[ \ln(R(\theta)) = \theta \]
which is the natural result from having
defined the exponential as the
rotation by $\theta$. \\

\newpage

\subsection*{Tangent Spaces}

Imagine for a second the line $i\rbb$,
which is the line that sits tangent
to the unit circle at $(1 + i0)$. \\
We can imagine that the exponential map
maps reals $\theta \in \rbb$ which live on this
tangent line, to the unit circle using
$e^{i\theta}$,
and the logarithmic map maps points on the
unit circle back to the tangent line. \\
The tangent line $i\rbb$ is the Lie alegbra
of the Lie group $S$. \\

Note that when we interpolated,
we had a formula:
\[ e^{{z_0t\ln\para{z_1z_0\inv}}} \]
Here, the first $z_0$ is the offset.
Because we are interpolating between $z_0$
and $z_1$, and $z_0$
does not necessarily coincide with the
identity of the rotation/group $S$,
which is at $(1 + i0)$,
then we need to offset the interpolated point
by $z_0$. \\
Couldn't we intsead just take $z_0$ to
be the the reference point?
Starting from which everything rotates?
The answer is yes, but then the tangent
space of the unit circle at $z_0$
is no longer the Lie alegbra. \\
Calcilations done in the tangent space 
on the angles will be harder. \\
The only advantage is that we would no longer
have an offset by $z_0$,
but it isn't worth the hassle. \\

We can now formalize the idea of a tangent space.
Given an $n$-dimensional manifold $C$
($n$ dimensional
generalization of a curve or surface),
we can define a tangent space
$T_pC$ of the manifold at a point $p$. \\
The tangent space to a point is the space spanned
by the hyperplane $\rbb^{n}$
tangent to that point. \\
For example, when we have a curve,
which is a one dimensional manifold,
the tangent at any point is a line.
and the tangent space is the span
of that line. \\

The tangent space is a vector space. \\

We can define a manifold as such:
\[ C(t) = \{\gamma_1(t),\gamma_2(t), 
\dots\gamma_n(t) \mid
t \in (\alpha, \beta, \dots) \in \rbb^m \} \]
So $t$ is a set of parameters,
and $x_1(t),\gamma_2(t), \dots,\gamma_n(t)$
are the coordinates of the manifold at
a point $t$. \\

So the number of parameters $t$ is the number
of dimensions of the manifold,
and the number of coordinates of the
vector $\gamma(t)$ is the space
the manifold lives in. \\

We can for now focus on having a simple
curve, which is a one dimensional manifold
in $\rbb^n$ space:
\[ C(t) = \{\gamma(t) \in \rbb^n \mid t \in \rbb \} \]
Where:
\[ \gamma(t) = \vecThree{\gamma_1(t)}
{\vdots}{\gamma_n(t)} \]
is any point $p$ on the curve. \\

Since we have a curve, the tangent will be a line. \\
Now, as we know from calculus,
the direction of the tangent at a point $p$
is given by the derivative,
which is then anchored using the coordinates
of the function at $p$
(the direction doesn't tell us where the tangent
line is). \\

First we can take the derivative of the curve:
\[ \dd{\gamma}{t} = \curl{\dd{\gamma_1}{t},
\dd{\gamma_2}{t}, \dots
\dd{\gamma_n}{t} } \]
Which gives us our direction. \\

Now to find the tangent line at a point
\[ p = C(t_0) = \gamma(t_0) \]
The tangent line direction is just the vector:
\[ \dd{C}{t} = \curl{\dd{\gamma_1}{t},
\dd{\gamma_2}{t}, \dots
\dd{\gamma_n}{t} } \]
We can then start at the origin $\gamma(t_0)$,
and span the entirity of the vector $\dd{\gamma}{t}$,
which gives us the parametric equation of a line:
\[ \lambda \dd{\gamma}{t}(t_0) +
\gamma(t_0) \qquad \lambda \in \rbb \]
Which is our tangent. \\

If we ignore the origin and just focus
on the space spanned by the direction:
\[ T_pC = \curl{ \lambda \dd{\gamma}{t}(t_0)
\mid \lambda \in \rbb } \]
We have our tangent space,
which is just a vector space. \\

We can generalize this for any manifold
by having $t$ be more than one parameter,
and using partial derivatives. \\

For example:
\[ C(\alpha, \beta) =
\{\gamma(\alpha, \beta) \in \rbb^n 
\mid (\alpha, \beta) \in \rbb^2 \} \]
the following manifold is a two dimensional
surface in $n$-dimensional space. \\
The tangent space of this surface would be a
plane. \\
In order to find the tangent at 
$(\alpha_0, \beta_0)$, we again need the origin
$\gamma(\alpha_0, \beta_0)$,
as well as two direction vectors:
\[ \partialdd{\gamma}{\alpha}
\AND \partialdd{\gamma}{\beta} \]
Both of which can be found using the Jacobian. \\
We need two direction vectors because the
tangent is a plane, and the tangent
must span the direction vector or vectors
(the span of two vectors is a plane). \\
So the tangent plane will be:
\[
\lambda \partialdd{\gamma}{\alpha}(\alpha_0, \beta_0) +
\mu \partialdd{\gamma}{\beta}(\alpha_0, \beta_0) +
\gamma(\alpha_0, \beta_0) \qquad
\lambda, \mu \in \rbb \]
And the tangent space is the
span of this plane, which means we
can ignore the origin and just focus
on the set of all vectors in the plane
that have the form:
\[ T_pC = \curl{ 
\lambda \partialdd{\gamma}{\alpha}(\alpha_0, \beta_0)
+ \mu \partialdd{\gamma}{\beta}(\alpha_0, \beta_0) 
\mid \lambda, \mu \in \rbb } \]
Which is again a vector space. \\

Sometimes we may include the point
$\gamma(\alpha_0, \beta_0)$ to indicate 
where the tangent space lies, and to reinforce
the idea that vectors in seperate tangent spaces
can't be operated on at the same time. \\

We know that for a function $\gamma$
with an output in $\rbb^n$,
and inputs $(\alpha, \beta) \in \rbb^2$,
then the Jacobian is the $n \times 2$ matrix:
\[ J_\gamma = \begin{bmatrix}
{\partialdd{\gamma^1}{\alpha}} & 
{\partialdd{\gamma^1}{\beta}} \vspace{10pt} \\
{\vdots} & {\vdots} \vspace{10pt} \\
{\partialdd{\gamma^n}{\alpha}} & 
{\partialdd{\gamma^n}{\beta}}
\end{bmatrix} \]
However, instead of thinking of each
output variables $\gamma^1, \gamma^2, \dots$
individually, it's best to think of it
as a single output that happes to be a vector:
\[ \gamma(\alpha, \beta) = 
\vecThree{\gamma^1(\alpha, \beta)}
{\vdots}{\gamma^n (\alpha, \beta)} \]
So instead, we can think of the Jacobian
as the block component matrix:
\[ J_\gamma = \vecTwoTransposed
{\partialdd{\gamma}{\alpha}}
{\partialdd{\gamma}{\beta}} \]
Where $\partialdd{\gamma}{\alpha}$
and $\partialdd{\gamma}{\beta}$
are just the derivatives of the $\gamma$ vector. \\
As we can see from the previous jacobian definition,
the derivative of the vector $\gamma$
is just the vector of derivatives of each component
of $\gamma$. \\

We can then express the tangent space as:
\[ T_pC = \curl{ 
J_g \vecTwo{\lambda}{\mu}
\mid \lambda, \mu \in \rbb } \]
Which we know is just a linear combination of each
column in the Jacobian, which are each
a direction vector in the tangent space. \\

So we can get the derivatives of vectors
the same way we do any scalar-output functions. \\
It's easier to just think of it as a vector
than to think of each output individually. \\

The same applies for matrices;
if the output of a function is a matrix,
then the derivatives is also a matrix. 
In that case, the Jacobian will have a row
with matrix block elements (same as vectors). \\
A matrix-output function does have a tangent space;
it's easier to think of an $n \times m$ matrix
as a $nm$-dimensional vector in order to visualize
matrix vector spaces. \\

\end{document}