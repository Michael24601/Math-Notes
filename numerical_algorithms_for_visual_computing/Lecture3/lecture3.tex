
\documentclass[12pt]{article}

\usepackage[margin=1in]{geometry}


%===============================================================================
%================================== PACKAGES ===================================
%===============================================================================

% For using float option H that places figures 
% exatcly where we want them
\usepackage{float}
% makes figure font bold
\usepackage{caption}
\captionsetup[figure]{labelfont=bf}
% For text generation
\usepackage{lipsum}
% For drawing
\usepackage{tikz}
% For smaller or equal sign and not divide sign
\usepackage{amssymb}
% For the diagonal fraction
\usepackage{xfrac}
% For enumerating exercise parts with letters instead of numbers
\usepackage{enumitem}
% For dfrac, which forces the fraction to be in display mode (large) e
% even in math mode (small)
\usepackage{amsmath}
% For degree sign
\usepackage{gensymb}
% For "\mathbb" macro
\usepackage{amsfonts}
% For positioning 
\usepackage{indentfirst}
\usetikzlibrary{shapes,positioning,fit,calc}
% for adjustwidth environment
\usepackage{changepage}
% for arrow on top
\usepackage{esvect}
% for mathbb 1
\usepackage{bbm}
% for mathsrc
\usepackage[mathscr]{eucal}
% For degree sign
\usepackage{gensymb}
% For quotes
\usepackage{csquotes}
% For vertical lines
\usepackage{mathtools}
% For cols
\usepackage{multicol}

% for tikz
\usepackage{pgfplots}
\pgfplotsset{compat=1.18}
\usepackage{amsmath}
\usepgfplotslibrary{groupplots}


%===============================================================================
%==================================== FONTS ====================================
%===============================================================================


% Mathcal
\newcommand{\acal}{\mathcal{A}}
\newcommand{\bcal}{\mathcal{B}}
\newcommand{\ccal}{\mathcal{C}}
\newcommand{\dcal}{\mathcal{D}}
\newcommand{\ecal}{\mathcal{E}}
\newcommand{\fcal}{\mathcal{F}}
\newcommand{\gcal}{\mathcal{G}}
\newcommand{\hcal}{\mathcal{H}}
\newcommand{\ical}{\mathcal{I}}
\newcommand{\jcal}{\mathcal{J}}
\newcommand{\kcal}{\mathcal{K}}
\newcommand{\lcal}{\mathcal{L}}
\newcommand{\mcal}{\mathcal{M}}
\newcommand{\ncal}{\mathcal{N}}
\newcommand{\ocal}{\mathcal{O}}
\newcommand{\pcal}{\mathcal{P}}
\newcommand{\qcal}{\mathcal{Q}}
\newcommand{\rcal}{\mathcal{R}}
\newcommand{\scal}{\mathcal{S}}
\newcommand{\tcal}{\mathcal{T}}
\newcommand{\ucal}{\mathcal{U}}
\newcommand{\vcal}{\mathcal{V}}
\newcommand{\wcal}{\mathcal{W}}
\newcommand{\xcal}{\mathcal{X}}
\newcommand{\ycal}{\mathcal{Y}}
\newcommand{\zcal}{\mathcal{Z}}

% Mathfrak
\newcommand{\afrak}{\mathfrak{A}}
\newcommand{\bfrak}{\mathfrak{B}}
\newcommand{\cfrak}{\mathfrak{C}}
\newcommand{\dfrak}{\mathfrak{D}}
\newcommand{\efrak}{\mathfrak{E}}
\newcommand{\ffrak}{\mathfrak{F}}
\newcommand{\gfrak}{\mathfrak{G}}
\newcommand{\hfrak}{\mathfrak{H}}
\newcommand{\ifrak}{\mathfrak{I}}
\newcommand{\jfrak}{\mathfrak{J}}
\newcommand{\kfrak}{\mathfrak{K}}
\newcommand{\lfrak}{\mathfrak{L}}
\newcommand{\mfrak}{\mathfrak{M}}
\newcommand{\nfrak}{\mathfrak{N}}
\newcommand{\ofrak}{\mathfrak{O}}
\newcommand{\pfrak}{\mathfrak{P}}
\newcommand{\qfrak}{\mathfrak{Q}}
\newcommand{\rfrak}{\mathfrak{R}}
\newcommand{\sfrak}{\mathfrak{S}}
\newcommand{\tfrak}{\mathfrak{T}}
\newcommand{\ufrak}{\mathfrak{U}}
\newcommand{\vfrak}{\mathfrak{V}}
\newcommand{\wfrak}{\mathfrak{W}}
\newcommand{\xfrak}{\mathfrak{X}}
\newcommand{\yfrak}{\mathfrak{Y}}
\newcommand{\zfrak}{\mathfrak{Z}}

% Mathscr
\newcommand{\ascr}{\mathscr{A}}
\newcommand{\bscr}{\mathscr{B}}
\newcommand{\cscr}{\mathscr{C}}
\newcommand{\dscr}{\mathscr{D}}
\newcommand{\escr}{\mathscr{E}}
\newcommand{\fscr}{\mathscr{F}}
\newcommand{\gscr}{\mathscr{G}}
\newcommand{\hscr}{\mathscr{H}}
\newcommand{\iscr}{\mathscr{I}}
\newcommand{\jscr}{\mathscr{J}}
\newcommand{\kscr}{\mathscr{K}}
\newcommand{\lscr}{\mathscr{L}}
\newcommand{\mscr}{\mathscr{M}}
\newcommand{\nscr}{\mathscr{N}}
\newcommand{\oscr}{\mathscr{O}}
\newcommand{\pscr}{\mathscr{P}}
\newcommand{\qscr}{\mathscr{Q}}
\newcommand{\rscr}{\mathscr{R}}
\newcommand{\sscr}{\mathscr{S}}
\newcommand{\tscr}{\mathscr{T}}
\newcommand{\uscr}{\mathscr{U}}
\newcommand{\vscr}{\mathscr{V}}
\newcommand{\wscr}{\mathscr{W}}
\newcommand{\xscr}{\mathscr{X}}
\newcommand{\yscr}{\mathscr{Y}}
\newcommand{\zscr}{\mathscr{Z}}

% Mathbb
\newcommand{\abb}{\mathbb{A}}
\newcommand{\bbb}{\mathbb{B}}
\newcommand{\cbb}{\mathbb{C}}
\newcommand{\dbb}{\mathbb{D}}
\newcommand{\ebb}{\mathbb{E}}
\newcommand{\fbb}{\mathbb{F}}
\newcommand{\gbb}{\mathbb{G}}
\newcommand{\hbb}{\mathbb{H}}
\newcommand{\ibb}{\mathbb{I}}
\newcommand{\jbb}{\mathbb{J}}
\newcommand{\kbb}{\mathbb{K}}
\newcommand{\lbb}{\mathbb{L}}
\newcommand{\mbb}{\mathbb{M}}
\newcommand{\nbb}{\mathbb{N}}
\newcommand{\obb}{\mathbb{O}}
\newcommand{\pbb}{\mathbb{P}}
\newcommand{\qbb}{\mathbb{Q}}
\newcommand{\rbb}{\mathbb{R}}
\newcommand{\sbb}{\mathbb{S}}
\newcommand{\tbb}{\mathbb{T}}
\newcommand{\ubb}{\mathbb{U}}
\newcommand{\vbb}{\mathbb{V}}
\newcommand{\wbb}{\mathbb{W}}
\newcommand{\xbb}{\mathbb{X}}
\newcommand{\ybb}{\mathbb{Y}}
\newcommand{\zbb}{\mathbb{Z}}


%===============================================================================
%=============================== SPECIAL SYMBOLS ===============================
%===============================================================================


% Orbit (group theory)
\newcommand{\orbit}{\mathcal{O}}
% Normal group
\newcommand{\normal}{\mathcal{N}}
% Indicator function
\newcommand{\indicator}{\mathbbm{1}}
% Laplace transform
\newcommand{\laplace}[1]{\mathcal{L}}
% Epsilon shorthand
\newcommand{\eps}{\varepsilon}
% Omega
\newcommand{\om}{\omega}
\newcommand{\Om}{\Omega}

%===============================================================================
%================================== OPERATORS ==================================
%===============================================================================


% Inverse exponent
\newcommand{\inv}[0]{^{-1}}
% Overline bar
\newcommand{\olsi}[1]{\,\overline{\!{#1}}}
% Less than or equal slanted
\newcommand{\seqs}{\leqslant}
% Greater or equal slanted
\newcommand{\geqs}{\geqslant}
% Subset or equal
\newcommand{\sub}{\subseteq}
% Proper subset
\newcommand{\prosub}{\subset}
% from
\newcommand{\from}{\leftarrow}

% Parantheses
\newcommand{\para}[1]{\left( #1 \right)}
% Curly Braces
\newcommand{\curl}[1]{\left\{ #1 \right\}}
% Brackets
\newcommand{\brac}[1]{\left[ #1 \right]}
% Angled Brackets
\newcommand{\ang}[1]{\left\langle #1 \right\rangle}
% Norm
\newcommand{\norm}[1]{\left\| #1 \right\|}

% Piece wise (use \\ between cases)
\newcommand{\piecewise}[1]{\begin{cases} #1 \end{cases}}

% Bold symbol shorthand
\newcommand{\bl}{\boldsymbol}

% Vertical space
\newcommand{\vs}[1]{\vspace{#1 pt}}
% Horizontal ertical space
\newcommand{\hs}[1]{\hspace{#1 pt}}


%===============================================================================
%============================== TEXT BASED SYMBOLS =============================
%===============================================================================


% Radians
\newcommand{\rad}{\text{rad}}
% Least Common Multiple
\newcommand{\lcm}{\text{lcm}}
% Automorphism
\newcommand{\Aut}{\text{Aut}}
% Variance
\newcommand{\var}{\text{Var}}
% Covariance
\newcommand{\cov}{\text{Cov}}
% Cofactor (matrix)
\newcommand{\cof}{\text{Cof}}
% Adjugate (matrix)
\newcommand{\adj}{\text{Adj}}
% Trace (matrix)
\newcommand{\tr}{\text{tr}}
% Standard deviation
\newcommand{\std}{\text{Std}}
% Correlation coefficient
\newcommand{\corr}{\text{Corr}}
% Sign
\newcommand{\sign}{\text{sign}}

% And text
\newcommand{\AND}{\text{ and }}
% Or text 
\newcommand{\OR}{\text{ or }}
% For text 
\newcommand{\FOR}{\text{ for }}
% If text
\newcommand{\IF}{\text{ if }}
% When text
\newcommand{\WHEN}{\text{ when }}
% Where text
\newcommand{\WHERE}{\text{ where }}
% Then text
\newcommand{\THEN}{\text{ then }}
% Such that text
\newcommand{\SUCHTHAT}{\text{ such that }}

% 1st
\newcommand{\st}[1]{#1^{\text{st}}}
% 2nd
\newcommand{\nd}[1]{#1^{\text{nd}}}
% 3rd
\newcommand{\rd}[1]{#1^{\text{rd}}}
% nth
\newcommand{\nth}[1]{#1^{\text{th}}}


%===============================================================================
%========================= PROBABILITY AND STATISTICS ==========================
%===============================================================================


% Permutation
\newcommand{\perm}[2]{{}^{#1}\!P_{#2}}
% Combination
\newcommand{\comb}[2]{{}^{#1}C_{#2}}

% Baye's risk
\newcommand{\risk}[1]{\mathscr{R}_{#1}}
% Baye's optimal risk
\newcommand{\riskOptimal}[1]{\mathscr{R}_{#1}^*}
% Baye's empirical risk
\newcommand{\riskEmpirical}[2]{\hat{\mathscr{R}}_{#1}^{#2}}


%===============================================================================
%=================================== CALCULUS ==================================
%===============================================================================


% d over d derivative
\newcommand{\dd}[2]{\dfrac{d#1}{d#2}}
% partial d over d derivative
\newcommand{\partialdd}[2]{\dfrac{\partial #1}{\partial #2}}
% delta d over d derivative
\newcommand{\deltadd}[2]{\dfrac{\Delta #1}{\Delta #2}}

% Integration between a and b
\newcommand{\integral}[4]{\int_{#1}^{#2} #3 \, #4}
% Integration in some space 
\newcommand{\boundIntegral}[2]{\int_{#1} #2 \, d#1}

% Limit
\newcommand{\limit}[3]{\lim_{#1 \to #2} #3}


%===============================================================================
%================================  BIG SYMBOLS  ================================
%===============================================================================


% Sum
\newcommand{\sumof}[2]{\sum_{#1}^{#2}}
% Product
\newcommand{\productof}[2]{\prod_{#1}^{#2}}
% Union
\newcommand{\unionof}[2]{\bigcup_{#1}^{#2}}
% Intersection
\newcommand{\intersectionof}[2]{\bigcap_{#1}^{#2}}
% Or
\newcommand{\orof}[2]{\bigvee_{#1}^{#2}}
% And
\newcommand{\andof}[2]{\bigwedge_{#1}^{#2}}


%===============================================================================
%=============================== LINEAR ALGEBRA ================================
%===============================================================================


% Bold vector arrow
\newcommand{\bv}[1]{\vec{\mathbf{#1}}}

% Matrix or vector (use // for column, & for row) 
% with brackets
\newcommand{\bmat}[1]{\begin{bmatrix} #1 \end{bmatrix}}
% Matrix or vector (use // for column, & for row) 
% with curved brackets
\newcommand{\pmat}[1]{\begin{pmatrix} #1 \end{pmatrix}}
% Matrix or vector (use // for column, & for row) 
% with lines on either side 
\newcommand{\lmat}[1]{\begin{vmatrix} #1 \end{vmatrix}}
% Matrix or vector (use // for column, & for row) 
% with curly braces
\newcommand{\cmat}[1]{\begin{Bmatrix} #1 \end{Bmatrix}}
% Matrix or vector (use // for column, & for row) 
% with no braces
\newcommand{\mat}[1]{\begin{matrix} #1 \end{matrix}}


%===============================================================================
%================================ LARGE OBJECTS ================================
%===============================================================================

% Multiple lines
\newcommand{\multiline}[1]{
\begin{align*}
    #1
\end{align*}
}

% Multiple lines with equation numbers
\newcommand{\eqmultiline}[1]{
\begin{align*}
    #1
\end{align*}
}

% Color
\newcommand{\colorText}[2]{
\begingroup
\color{#1}
    #2
\endgroup
}

% Centered figure
\newcommand{\centerFigure}[2]{
    \begin{figure}[h]
        \centering
            #1
        \caption{#2}
    \end{figure}
}

% Tikz figure
\newcommand{\tikzGraphic}[1]{
    \begin{center}
    \begin{tikzpicture}
        #1
    \end{tikzpicture}
    \end{center}
}

% Enumerate numbers (seperate by \item)
\newcommand{\numbers}{\textbf{\number*)}}

% Enumerate letters (seperate by \item)
\newcommand{\letters}{\textbf{\alph*)}}

\title{%
    \Huge Numerical Algorithms \\
    \Large Lecture III
}
\date{2025-04-16}
\author{Michael Saba}

\begin{document}
\pagenumbering{gobble}
\maketitle
\newpage
\setlength{\parindent}{0pt}
\pagenumbering{arabic}

\subsection*{Normal Equation
and Linear Least Square}

Last lecture we saw that given a non-square
system (one that is tall and skinny):
\[ Ax = b \]
we won't be able to solve it directly. \\
Intead, what we do is
multiply both sides by $A^T$. \\
This has the effect of turning the matrix
square, which allows solving the
problem. \\
Now we have $A^TA$, which is square,
and if $A$ has linearlly independent
columns, is also invertible,
which means there will be a unique
solution. \\
But how does that work? \\
We know that because $A$
is tall and skinny, we have less
vectors than we do dimensions. 
For example, we may have a matrix
whose columns span a plane in $\rbb^3$.
But since $b$ is in $\rbb^3$,
isn't it possible for $b$ to just not
be on said plane? \\
The answer is yes, but
recall that the new system solves
for $A^Tb$,
which is equal to $b$
if $b$ were on the plane,
and the projection of $b$
on the plane otherwise. \\
So the system either solves
for $b$ or the projection of $b$. \\

How can we prove this actually works
and yields the solution described? \\
The answer is that we can use a numerical
method called linear least square,
which is good for solving problems
where we have more equations (requirements)
than vairables. \\
The idea behind it is to try and minimize
the distance between the solution
of $x$ and the target $b$. \\
In other words, we want to minimize
the distance:
\[ \min_x \| Ax - b \|^2 \]
we choose the square because we want
the absolute value of the distance
(we don't care about direction),
and because we want it to be differentiable. \\

So we have the norm squared:
\[ E(x) = \|Ax - b\|^2 \]
This is a positive convex function
\[ E : \rbb^n \to \rbb \]
which means it will have a minimum
at the point $x$ where:
\[ \nabla E(x) = 0 \]
We can calculate:
\[ \nabla \ang{Ax - b, (Ax - b)} 
= \nabla (Ax - b)(Ax - b)^T\]
which we can do by expanding the dot product,
and calculating each partial derivative
\[ \partialdd{E}{x_i} \]
individually,
which eventually tells us that
\[ \nabla E(x) = 
\nabla (Ax - b)(Ax - b)^T
= 2A^T(Ax - b)
= 2A^TAx - 2A^Tb \]
where $\nabla E(x) = 0$
at $x$ that minimized $E(x)$. \\

Notice that when we minimize $E(x)$,
we set the gradient to be $0$, so:
\[ 2A^TAx - 2A^Tb = 0 \]
\[ A^TAx = A^Tb \]
which is the system we said
solves our problem (multiplying by the
transpose). \\

So what happens when we minimize the distance
between $Ax$ and $b$? \\
As we know $Ax$ only spans the plane
defined by the columns of $A$.
If $b$ is on the plane, then $Ax$ can reach $b$,
and that is the minimal distance.
But if $b$ is not on the plane,
then $Ax$ can't reach $b$;
instead, the closest it can get to $b$
is the point directly under $b$
on the plane, which is the projection. \\
So when we minimize $\|Ax - b\|^2$,
we either solve for $Ax$ being $b$,
or $Ax$ being $b$'s projection,
which we will recall is exactly what
we described should happen
when we multiply by the transpose. \\

\newpage

\subsection*{Directional Derivative}

While $\nabla E(x)$ gives 
a vector containing the partial derivatives
in the directions of the standard basis vectors:
\[ x_1 = 
\begin{bmatrix}1 \\ 0 \\ 0 \\ \vdots \\  0\end{bmatrix}
\quad x_2 = 
\begin{bmatrix}0 \\ 1 \\ 0 \\ \vdots \\ 0\end{bmatrix}
\quad \dots \quad x_n = 
\begin{bmatrix}0 \\ \vdots \\ 0 \\ 0 \\ 1\end{bmatrix}\] 
we can actually calculate the derivative in
any arbitrary direction vector $y$:
\[ \partial_yE(x) \]
which is a scalar that we can show evaluates to:
\[ \partial_yE(x) = \ang{y, \nabla E(x)} \]
This is always true,
as long as $E(x)$ is differentiable. \\

While this is true in general, we will show
that it is true in the case of
\[ E(x) = \|Ax - b\|^2 \]
First we write the limit definition:
\[ \partial_yE(x) = \limit{\varepsilon}{0}
{\dfrac{E(x + \varepsilon y) - {E(x)}}{\varepsilon}}  \]
\[ \partial_yE(x) = \limit{\varepsilon}{0}
{\dfrac{\|A(x+\varepsilon y) - b\|^2 - \|Ax - b\|^2}
{\varepsilon}} \]
\[\partial_yE(x) = \limit{\varepsilon}{0}
{\dfrac{\|Ax+ \varepsilon Ay - b\|^2 - \|Ax - b\|^2}
{\varepsilon}} \]
\[\partial_yE(x) = \limit{\varepsilon}{0}
{\dfrac{(Ax+ \varepsilon Ay - b)^T
(Ax+ \varepsilon Ay - b)
 - (Ax - b)^T(Ax - b)}
{\varepsilon}} \]
We then distribute the dot product:
\begin{align*}
\partial_yE(x) &= \lim_{\varepsilon \to 0} 
\frac{1}{\varepsilon} \Bigg[ 
(Ax)^T(Ax) + (\varepsilon Ay)^T(\varepsilon Ay) 
+ b^Tb \\
&\quad + 2(Ax)^T(\varepsilon Ay) 
- 2(Ax)^T(b) - 2(\varepsilon Ay)^T(b) \\
&\quad - (Ax)^T(Ax) - b^Tb + 2(Ax)^T(b)
\Bigg]
\end{align*}
\[\partial_yE(x) = \limit{\varepsilon}{0}
{\dfrac{ (\varepsilon Ay)^T(\varepsilon Ay)
+ 2(Ax)^T(\varepsilon Ay) - 2(\varepsilon Ay)^T(b)}
{\varepsilon}} \]
\[\partial_yE(x) = \limit{\varepsilon}{0}
{\dfrac{ \|\varepsilon Ay\|^2
+ 2(\varepsilon Ay)^T(Ax - b)}
{\varepsilon}} \]
\[\partial_yE(x) = \limit{\varepsilon}{0}
{\dfrac{ \varepsilon^2\|Ay\|^2
+ 2\varepsilon(Ay)^T(Ax - b)}
{\varepsilon}} \]
\[\partial_yE(x) = \limit{\varepsilon}{0} 
\varepsilon\|Ay\|^2 + 2(Ay)^T(Ax - b) \]
\[\partial_yE(x) = 2(Ay)^T(Ax - b) \]
\[\partial_yE(x) = 2(y^TA^T)(Ax - b) \]
Then, using the commutativity of scalar multiplication,
and the associativity of matrix multiplication,
we can write:
\[\partial_yE(x) = (y^T2A^T)(Ax - b) \]
\[\partial_yE(x) = (y^T)(2A^T(Ax - b)) \]
\[\partial_yE(x) = \ang{y, \nabla E(x)} \]

\newpage

\subsection*{Determinant Solution to a Linear System}

The other method that we can discuss is 
one that uses determinants. \\
To solve
\[ Ax = b \]
we can just write
\[ x = A\inv b \]
which is true as long as $A$ is invertible. \\
To find the inverse of $A$,
we can either use Guass-Jordan (elimination),
or determinants (connected to Cramer's rule). \\
We will show how to do it using determinants. \\

As we know, we can calculate the determinant
of a square matrix using a cofactor expansion:
\[ \det \para{ \bmat{{a_1^1} & {a_2^1} 
& {a_3^1} \\ {a_1^2} & {a_2^2} & {a_3^2} \\ 
{a_1^3} & {a_2^3} & {a_3^3}} }
= a_1^1 \det\para{ \bmat{{a_2^2} & {a_3^2} \\
{a_2^3} & {a_3^3}}}
- a_2^1 \det\para{\bmat{{a_1^2} & {a_3^2} \\
{a_1^3} & {a_3^3}}}
+ a_3^1 \det\para{\bmat{{a_1^2} & {a_2^2} \\
{a_1^3} & {a_2^3}}} \]
We don't need to use coefficients from the top
row, it can from any row:
\[ \det \para{ \bmat{{a_1^1} & {a_2^1} & 
{a_3^1} \\ {a_1^2} & {a_2^2} & {a_3^2} \\
{a_1^3} & {a_2^3} & {a_3^3}} }
= -a_1^2 \det\para{ \bmat{{a_2^1} & {a_3^1} \\
{a_2^3} & {a_3^3}}}
+ a_2^2 \det\para{\bmat{{a_1^1} & {a_3^1} \\
{a_1^3} & {a_3^3}}}
- a_3^2 \det\para{\bmat{{a_1^1} & {a_2^1} \\
{a_1^3} & {a_2^3}}} \]
\[ \det \para{ \bmat{{a_1^1} & {a_2^1} & {a_3^1} \\
{a_1^2} & {a_2^2} & {a_3^2} \\{a_1^3} & 
{a_2^3} & {a_3^3}} }
= a_1^3 \det\para{\bmat{{a_2^1} & {a_3^1} \\
{a_2^2} & {a_3^2}}}
- a_2^3 \det\para{\bmat{{a_1^1} & {a_3^1} \\
{a_1^2} & {a_3^2}}}
+ a_3^3 \det\para{\bmat{{a_1^1} & {a_2^1} \\
{a_1^2} & {a_2^2}}} \]
We can use any row we want for the coefficients
so long as we multiply it by the determinant
of the $2 \times 2$ matrix that we get from
removing the row and column of the coefficient. \\
Also, each coefficient is multiplied
by $-1$ if $i+j$ is odd,
meaning that each coefficient is multiplied
by $-1^{i+j}$. \\

We can extend this notion to any number of dimensions.
We can also make it easier by defining
cofactor matrices. \\
An $(n-1)\times (n-1)$ matrix $C_{ij}$
is a minor matrix of the $n \times n$ matrix $A$
if $C_{ij}$ is the same as $A$ with 
the $\nth{i}$ row and $\nth{j}$ column removed. \\
Notice that
\[ \det(A) = a_1^1 \det(C_{11})
- a_2^1 \det(C_{21}) + a_3^1 \det(C_{31}) \]
\[ \det(A) = -a_1^2 \det(C_{12})
+ a_2^2 \det(C_{22}) - a_3^2 \det(C_{32}) \]
\[ \det(A) = a_1^3 \det(C_{13})
- a_2^3 \det(C_{23}) + a_3^3 \det(C_{33}) \]
So, if we were to define a matrix $M$
such that the $\nth{i}$ and $\nth{j}$
entry is the determinant of $C_{ij}$
multiplied by $-1^{i+j}$,
we would call $M$ a cofactor matrix:
\[ M = \cof(A) = 
\bmat{{\det(C_{11})} & {-\det(C_{12})} & {\det(C_{12})} \\
{-\det(C_{21})} & {\det(C_{22})} & {-\det(C_{23})} \\
{\det(C_{31})} & {-\det(C_{32})} & {\det(C_{33})}} \]

The adjugate matrix $\adj(A)$
is just $\cof(A)^T$. \\
We know that the determinant of $A$
is the dot product of the $\nth{i}$
row of $A$ and the $\nth{i}$ row of $\cof(A)$. \\
But if we were to transpose the matrix,
then the determinant would just be the dot
product of the $\nth{i}$ row of $A$
and the $\nth{i}$ column of $\adj(A)$. \\
This means that
\[ A (\adj(A)) = \bmat{{\det(A)} & {} & {} \\
{} & {\det(A)} & {} \\ {} & { } & {\det(A)}} \]
As for the other elements,
notice that when we have the dot product of the
$\nth{i}$ row of $A$
and the $\nth{j}$ column of $\adj(A)$
where $i \neq j$,
then this represents the determinant of some
singular matrix. \\
For example, the dot product of the first row
and second column is:
\[ A_1(\adj(A))^2 =
a_1^1 \det(C_{12})
- a_2^1 \det(C_{22}) + a_3^1 \det(C_{32}) \]
which is the formula for the determinant of
\[ \bmat{{a_1^1} & {a_2^1} & {a_3^1} \\
{a_1^1} & {a_2^1} & {a_3^1} \\ {a_3^1} & 
{a_3^2} & {a_3^3}} \]
which has two identical rows and is therefore
singular. \\
This means that $A_1(\adj(A))^2 = 0$,
which tells us that:
\[ A (\adj(A)) = \bmat{{\det(A)} & {0} & {0} \\
{0} & {\det(A)} & {0} \\ {0} & {0} & {\det(A)}} 
= \det(A)I \]

Now, to calculate the inverse, we just need
to get:
\[ A (\adj(A)) = \det(A)I  \]
\[ A\inv A (\adj(A)) = A\inv \det(A)I \]
\[ \adj(A) \dfrac{1}{\det(A)} = A\inv \]
So we can use the determinant and adjugate matrix
to calculate the inverse. \\
This method is actually
is being done when we solve the linear
system:
\[ Ax = b \]
using Cramer's rule.
Solving a linear system indirectly
calculates the inverse of the matrix. \\

While the method is easy, it is extremely
inefficient. \\
Suppose $f(n)$ were the time to calculate the
the determinant of an $n \times n$ matrix. \\
We know that it requires the addition of $10$
terms, where each term uses the determinant
of an $(n-1) \times (n-1)$ minor matrix. \\
So
\[ f(n) = nf(n-1) \]
which means that $f(n) = n!$ is the time
complexity of the cofactor expansion. \\

Note that the cofactor expansion is also
called the Laplacian expansion. \\

\newpage

\subsection*{Outer Product}

We know that the inner product of two vectors 
\[ u \dot v = u^Tv = v^Tu = \sum_i u_iv_i \]
is a scalar value. \\

The outer product on the other hand,
the product of two vectors
\[ uv^T \quad \text{ or } \quad vu^T \]
is a matrix, and unlike the inner product
it is not commutative. \\
The matrix we get from $uv^T$
contains in each entry $(i, j)$
the value $u_iv_j$. \\

Earlier we said we can think of vector matrix
multiplication as the dot product
of the vector and the rows or the linear combination
of the columns.
This idea can then be extended to
matrix multiplication (for each row or column
of the matrix). \\

Now, we have two new ways of thinking about matrix
multiplication $AB$. \\
Each entry can be the inner (dot) product of
a row in $A$ and a column in $B$. \\
Or instead, we can think about
calculating every outer product possible
between columns in $A$ and rows in $B$,
and then summing the resulting matrices. \\
The outer product interpretation is slower. \\

\newpage

\subsection*{Translation}

An affine transformation is a non-linear
transformation that looks like:
\[ q = Ap + t \]
where $A$ is a square matrix
(linear transformation) that first transforms $p$,
which is followed by an addition (translation) of $t$. \\
Affine transformations include
some linear Transformations
like rotations, scaling, as well as non-linear
trasnformations such as shears and translations. \\

We can apply the linear transformation
by using block element multiplication:
\[ \bmat{{A} & {t}} \bmat{{p}\\{1}} \] \\

A linear transformation $L$ must satisfy the
property that, for any two vectors $u$ and $v$,
and scalars $a$ and $b$:
\[ L(au + vb) = aL(u) + bL(v) \]
One way to check that a transformation is not linear
is to check if it moves the origin (linear
transformations don't). \\

In rendering, one of the most common transformations
that we can apply to a point is a translation
by a vector $t$:
\[ q = p + t \]
Note that this is not a linear transformation however,
since the origin is moved. \\
It cannot be represented by a matrix
(at least not one that has the same dimensionality
as $p$ and $q$). \\

We can model the translation as an affine
transformation:
\[ \bmat{{I_n}& {t}} \bmat{{p}\\{1}} \]
using the idendity matrix for $A$
so that the affine transformation only translates
$p$. \\

However, we can use homogeneous coordinates
(going up one dimension)
in order to model the transformation using a 
square matrix. \\
So suppose we have $p, q \in \rbb^2$,
then notice that:
\[ \bmat{{q_1}\\{q_2}} = \bmat{{p_1 + t_1}\\{p_2 + t_2}} \]
If we were to add a padding to the vector
(a third entry, with value $1$, 
which we know can be discarded),
Then we can write:
\[ \bmat{{1}&{0}&{t_1}\\{0}&{1}&{t_2}\\{0}&{0}&{1}}
\bmat{{p_1}\\{p_2}\\{1}} = 
\bmat{{p_1 + t_1}\\{p_2 + t_2}\\{1}}
= \bmat{{q_1}\\{q_2}\\{1}} \]
So the linear transformation gave us the translated vector,
but in homoegenous coordinates. \\
We can more generally write a translation of
vectors in $\rbb^n$ as the homogeneous matrix in
$\rbb^{(n+1) \times (n+1)}$ with the format:
\[ \bmat{{I} & {t} \\{0} & {1}} \]
where $I$ is the identity matrix in $\rbb^{n \times n}$
and $t \in \rbb^n$. \\

Now, it will also be useful to have the inverse of this
transformation.
As we know, the inverse of applying a translation
\[ q = p + t \]
is to translate by the opposite direction vector of $t$:
\[ p = q - t \]
So, the inverse transformation must be:
\[ \bmat{{I} & {-t} \\{ 0} & {1}} \]
And if we were to try and calculate the inverse of the
matrix, we would find the same answer. \\

The combination of two translation matrices
is their product:
\[ \bmat{{I} & {t_2} \\ {0} & {1}} 
\bmat{{I} & {t_1} \\ {0} & {1}} \]
Appropriately, the result of this product is
just:
\[ \bmat{{I} & {t_2 + t_1} \\{0} & {1}} \]
since we are just applying one translation
after the other,
which is to say, adding $t_1$ then $t_2$. \\

\newpage

\subsection*{Rotation and Reflection}

Another useful transformation is the rotation. \\
It preserves lengths and angles,
meaning that the transformation of a square object
for example would hold its size and shape, but be rotated
around (with the origin as the point being rotated around). \\

Suppose we have two vectors $u$ and $v$
such that the angle between $u$ and the $x$-axis
is $\phi$ and the angle between $v$ and $u$
is $\theta$ (so the angle between $v$
and $x$-axis is $\theta + \phi$). \\
We want the transformation $R$
that rotates $u$ to $v$,
meaning that it rotates by an angle $\theta$:
\[ Ru = v \]
\[ R \bmat{{\cos(\phi)} \\{\sin(\phi)}}
= \bmat{{\cos(\phi + \theta)} \\ {\sin(\phi + \theta)}} \]
We can then use the identities for
$\cos(a + b)$ and $\sin(a+b)$:
\[ R \bmat{{\cos(\phi)} \\ {\sin(\phi)}}
= \bmat{{\cos(\phi)\cos(\theta) - 
\sin(\phi)\sin(\theta)} \\
{\sin(\phi)\cos(\theta) + 
\cos(\phi)\sin(\theta)}} \]
Notice that this means that
\[R = \bmat{{\cos(\theta)} & {-\sin(\theta)} \\
{\sin(\theta)} & {\cos(\theta)}} \]
We can extend this to any dimension $\rbb^n$. \\
We can write $R(\theta)$ to indicate
a rotation of angle $\theta$. \\

The inverse of a rotation is a rotation
in the opposite direction:
\[ (R(\theta))\inv = R(-\theta) \]
which is the result we would get should we
actually calculate the inverse. \\
Note however that:
\[ (R(\theta))\inv = R(-\theta) 
= \bmat{{\cos(-\theta)} & {-\sin(-\theta)} \\
{\sin(-\theta)} & {\cos(-\theta)}}
= \bmat{{\cos(\theta)} & {\sin(\theta)} \\
{-\sin(\theta)} & {\cos(\theta)}} = R(\theta)^T \]
The reason the inverse is the transpose 
is because a rotation is orthogonal,
meaning that the inverse is the transpose. \\

The combination of two rotation patrices
is th product of the rotation matrices:
\[ R(\theta)R(\alpha) = R(\theta + \alpha) \]
which appropriately enough, is just the rotation
matrix of the sum of the rotation angles. \\

An orthogonal matrix $R$ preserves lengths and angles
(like rotation and reflection). \\
So that means that if we have $Ru = v$,
\[ \|u\|^2 = \|v\|^2 \]
This means that:
\[ \|Ru\|^2 = \|u\|^2 \]
\[ (Ru)^T(Ru) = u^Tu \]
\[ u^TR^TRu = u^Tu \]
\[ R^TR = I \]
So $R\inv = R^T$. \\

The determinant of a rotation matrix
is $1$.
This is because a rotation does not scale
shapes when it transforms them.
The basis is orthonormal. \\

A reflection matrix by comparison has a determinant
of $-1$. 
Like a rotation, the reflection doesn't scale objects,
but it does flip them, so that's where the negative
sign comes from. \\

The determinant of any orthogonal matrix
has absolute value $1$. \\

with the exception of $k\pi$ rotations,
rotation matrices have no eigenvectors. \\

A reflection matrix is similar to the identity
matrix, in that it is a diagonal matrix
with determinant that has an absolute
value of $1$. \\
However some of the entries can be $-1$
instead of $1$:
\[ \bmat{{1} & {0} \\ {0} & {-1}} \]
The transformation reflects the space $\rbb^n$
around basis vectors of the $-1$ entries.

\newpage

\subsection*{Scaling}

The matrix:
\[ S = \bmat{{\alpha_1} & {0} & {0} \\
{0} & {\alpha_2} & {0} \\ {0} & {0} & {\alpha_3}} \]
scales vectors in $\rbb^3$
by $\alpha_1$, $\alpha_2$, and $\alpha_3$
in the directions
\[ \bmat{{1} \\ {0}\\ {0}}
\qquad \bmat{{0} \\{1} \\{0}}
\qquad \bmat{{0} \\ {0}\\{1}} \]
We thus call it a scaling matrix. \\

A scaling matrix in $\rbb^n$ has a determinant equal 
to the product of its main diagonal:
\[ \sum_{i=1}^{n}\alpha_i\]
This is because the total amount of scaling
the matrix does is mad eup of the individual
scalings in each direction. \\
It's also clear this is the determinant if the cofactor
expansion is used. \\

The inverse of the scaling matrix
is clearly just the scaling matrix
with inverse scaling factors:
\[ S\inv = \bmat{{\sfrac{1} {\alpha_1}} & 
{0} & {0} \\ {0} & {\sfrac{1}{\alpha_2}} & {0} \\ 
{0} & {0} & {\sfrac{1}{\alpha_3}}} \] \\

Moreover, the product of two scaling
matrices is just the scaling matrix
that applies both scales:
\[ \bmat{{\alpha_1} & {0} & {0} \\
{0} & {\alpha_2} & {0} \\ {0} & {0} & {\alpha_3}}
\bmat{{\beta_1} & {0} & {0} \\
{0} & {\beta_2} & {0} \\ {0} & {0} & {\beta_3}}
= \bmat{{\alpha_1 \cdot \beta_1} & {0} &{0} \\
{0} & {\alpha_2 \cdot \beta_2} & {0} \\
{0} & {0} & {\alpha_3 \cdot \beta_3}} \] \\

We can write $S(s)$
as a shorthand for a scaling matrix:
\[ S(s) = \bmat{{s_1} & {0} & {0} \\
{0} & {s_2} & {0} \\ {0} & {0} & {s_3}} \]
that scales in each direction by 
the entry $s_i$ in the $s$ vector. \\

\newpage

\subsection*{Transformations}

Generally speaking, we want to apply
translations, rotation, and scaling to an object
in one operation. \\

In order to apply the operations, all transformations
need to be square matrices with the same dimensionality
as the point/vector they are being applied to. \\
Translations are only linear in homogeneous coordinates. \\
So since the vector will be written in homoegenous
coordinates,
this forces the other matrices to also be written as such. \\
So if we have a rotation $A$ and scaling matrix $B$,
then we would replace them with
\[ R = \bmat{{A} & {0} \\{0} & {1}}
\qquad S = \bmat{{B} & {0} \\ {0} & {1}} \]
which are homoegenous. \\

We apply scaling and rotation first,
because rotating an object centered is simple,
while rotating an object translated by some 
translation $T$ would move the object in an orbit. \\
So our transformation looks like:
\[ Mv = (TSR)v \]
where $v$ is any point. \\
Matrix multiplication is not commutative so
the order does matter and can't be flipped. \\
Note that $M$ is an affine transformation since:
\[ M = \bmat{{SR} & {t} \\ {0} & {1}} \]
which just applies $SR$ to $v$,
then adds the translation $t$. \\

Matrix-matrix multiplication is $O(n^3)$,
while matrix vector multiplication is $O(n^2)$. \\
So it is more efficient to multiply the vector
by each matrix indidividually:
\[ T(S(Rv)) \]
However, if we have more than one vector
that we want to transform,
then it would obviously become more efficient
to calcualte the product $M$:
\[ M = TSR \]
and then apply the transformation to the vectors. \\

If we want the inverse of the transformation $M$:
\[ M\inv = (T(t)S(s)R(\theta))\inv
= R(\theta)\inv S(s)\inv T(t)\inv 
= R(-\theta)S(\sfrac{1}{s})T(-t) \]
Recall that $(AB)\inv = B\inv A\inv$. \\

If we instead have just a a homoegenous matris $M$,
then the inverse is also homogeneous:
\[ \bmat{{A} & {t} \\ {0} & {1}}\inv
= \bmat{{A\inv} & {-A\inv t} \\ {0} & {1}} \]
This is because the matrix works by applying $A$
first to the vector, then adding $t$:
\[ q = Ap + t \]
which means inverting it first requires:
\[ p = A\inv(q - t) = A\inv q -A\inv t \]
which is the same as applying the inverse. \\

Transforming a surface or object just means Transforming
every single point on the object using the tranformation. \\

For instance, suppose we have
a parametrically defined set (surface):
\[ \mathcal{M}_\varphi = 
\{ \varphi(s) \in \rbb^n \mid 
s \in \mathcal{D} \subseteq \rbb^k \} \]
In order to transform this set (surface),
we just transform every point in it. \\
As we know, parametric equations are great at
generating the points of the surface,
so this is as simple as just transforming the
generated points. \\
We will apply some affine transformation $Ap + t$
(which is usually encoded in a square
homogeneous matrix):
\[ A\mathcal{M}_\varphi + t
= \{ A\varphi(s) + t\in \rbb^n \mid 
s \in \mathcal{D} \subseteq \rbb^k \} \]
So we just apply the transformation
to each point on the set.
We denote the transformed set or surface
$\mathcal{M}_{A\varphi + t}$. \\

Now, if we want to intersect the resulting
transformed surface with a line,
we would solve:
\[ o + \tau d = A\mathcal{M}_\varphi + t \]
Another way we could have solved the same
equation was apply the inverse transformation
to the line $o + \tau d$
and intersect it with the original surface:
\[ A\inv(o + \tau d - t)
= A\inv(o - t) A\inv \tau d 
= \mathcal{M}_\varphi \]
Notice that because $d$ is a direction,
it does not get translated (unlike $o$),
but only gets transformed by $A\inv$. \\
This is a commonly seen phenomenon.
For example, if we have some object in
local coordinates, we could 
transform it to world coordinates
in order to intersect a line with it,
or we could transform the line to the local
coordinates of the object itself. \\

On the other hand, suppose we have an implicitely
defined surface:
\[ \mathcal{M}_{f\inv(0)}
= \{ p \in \rbb^n \mid f(p) =0 \} \]
which is just the set of points $p$
such that $f(p) = 0$ is satisfied. \\
This makes it easy to check for membership,
but harder to generate points,
which makes transforming it slightly more
complicated:
\[ A\mathcal{M}_{f\inv(0)} + t
= \{ Ap + t \in \rbb^n \mid f(p) =0 \} \]
Basically it is the set of transformed points 
$Ap + t$ from the original surface definition,
so the condition to satisfy is still $f(p)$. \\
If we define $q = Ap + t$
and $p = A\inv(q - t)$ can thus rewrite
the set in the original format of
an implicit surface:
\[ A\mathcal{M}_{f\inv(0)} + t
= \{ q \in \rbb^n \mid f(A\inv(q - t)) = 0 \} \]
but this time with a new function,
which is the composition of $f$ and the inverse of the
affine transformation into one function:
\[ f \circ A\inv \circ T(-t) \]
which means that:
\[ A\mathcal{M}_{f\inv(0)} + t
= \{ q \in \rbb^n \mid 
(f \circ A\inv \circ T(-t))(q) = 0 \} \]
We can thus denote the transformed set or surface
$\mathcal{M}_{(f \circ A\inv \circ T(-t))\inv(0)}$. \\

The same thing applies regarding the
light intersection. \\
We could just replace $p$
by $o + \tau d$
inside of the transformed set 
$A\mathcal{M}_{f\inv(0)} + t$. \\
Or we could use the original set
$\mathcal{M}_{f\inv(0)}$
and replace $p$
by the line transformed by the inverse of
the affine transformation:
\[A\inv(o + \tau d - t)\] \\

\newpage

\subsection*{Determinant and Volume}

As we know, the determinant of a matrix represents
the scaling factor of its linear transformation. \\
For instance, if $A$ transforms
some box $c$,
then we expect that the volume of the box
would scale by the determinant of $A$. \\

For a matrix in $\rbb^2$,
it is the area of the parallelogram formed
by the column vectors of the matrix (basis vectors
of the matrix). \\
In $\rbb^3$, it is the volume of the parallelepiped. \\
In $\rbb^n$, it is the measure of an $n$-parallelotope. \\

So if we have some basis in $\rbb^n$,
calculating the measure of the
parallelotope is as simple as finding the
determinant. \\

But what happens if we have a basis in $\rbb^3$
that is made up of two $3$-dimensional vectors $f_1$
and $f_2$. \\
We thus have a $2$-dimensional basis,
since $f_1$ and $f_2$ span a plane
(if they are colinear). \\
But how can we calculate the parallelogram
formed by these basis vectors in $\rbb^3$?
We don't have a square matrix. \\

The first method is to use the cross product.
The cross product
\[ f_1 \times f_2 \]
returns a vector orthogonal to the plane
spanned by $f_1$ and $f_2$
with a magnitude (norm) equal to
the parallelogram formed by $f_1$ and $f_2$.
The direction of the cross product
depends on whether $f_1$ or $f_2$
comes first. \\
So we can calculate the parallelogram's
area as $\|f_1 \times f_2\|$. \\

This method does not generalize however
as it only works in $\rbb^3$. \\
As we saw earlier,
multiplying a vector by:
\[ \bmat{{f_1} & {f_2}}^T \]
has the effect of projecting it
onto the plane spanned by $f_1$ and $f_2$. \\
On the other hand,
multiplying the vectors together:
\[ F^TF = \bmat{{f_1} \\ {f_2}}
\bmat{{f_1}& {f_2}} \]
returns a matrix in $\rbb^2$
that represents the same basis
as the one defined by $f_1$ and $f_2$,
but in $\rbb^2$ (on the plane). \\
So we can use this basis to calcualte
the area of the parallelogram;
since it is square,
we can just use the determinant:
\[ \sqrt{\det(F^TF)} \]
Here we use the square root because we have
$F^TF$,
the product causes the area of the parallelogram
to square. \\
This method generalizes to any dimensions. \\

Note that this implies that
\[ \det(F^TF) = \|F_1 \times F_2\|^2 \]
in $\rbb^3$. \\

\newpage

\subsection*{Active and Passive Transformations}

In rendering, it's usually enough to transform the vertices
of the object. \\
For example, if we are rendering a triangle,
it's enough to transform $v0$, $v1$, and $v2$,
which defines the vertices of the transformed triangle.

This is called an active transformation. \\
When we use the linear transformation:
\[ Au = v \]
we intuitively think of $v$ as a different
point from $u$; it's $u$ transformed (rotated, shifted...). \\

A passive transformation on the other hand is
when we interpret $v$ as the same point as $u$,
but imagine the entire space having moved. \\
In other words, we image that $v$ is the same
point as $u$ but in the coordinates of this new,
moved space. \\
We will recall that this idea is similar to that
of expressing vectors in the coordinates of some new basis.
It can be done by multiplying by the inverse:
\[ A\inv u = v \]
So if we multiply some point $u$ by the inverse,
we image that it is the same point as $v$,
but in the coordinates of a space that had been transformed
by $A$. \\
So applying a transformation to an object
can be thought of as the inverse transformation
being applied to the space around it instead. \\

This makes intuitive sense.
For instance, applying a scaling matrix
(with a positive determinant) to an object
can be interpreted in two ways:
We are either expanding the object
(active),
or we are zooming in on the object,
which is to say,
we are making the space around the object smaller,
which means applying the inverse transform to
the space around the object. \\

Active transformations are used in rasterization. \\
Passive transformations are used in ray tracing. \\

\end{document}