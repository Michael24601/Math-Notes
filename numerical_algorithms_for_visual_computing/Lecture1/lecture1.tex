
\documentclass[12pt]{article}

\usepackage[margin=1in]{geometry}


%===============================================================================
%================================== PACKAGES ===================================
%===============================================================================

% For using float option H that places figures 
% exatcly where we want them
\usepackage{float}
% makes figure font bold
\usepackage{caption}
\captionsetup[figure]{labelfont=bf}
% For text generation
\usepackage{lipsum}
% For drawing
\usepackage{tikz}
% For smaller or equal sign and not divide sign
\usepackage{amssymb}
% For the diagonal fraction
\usepackage{xfrac}
% For enumerating exercise parts with letters instead of numbers
\usepackage{enumitem}
% For dfrac, which forces the fraction to be in display mode (large) e
% even in math mode (small)
\usepackage{amsmath}
% For degree sign
\usepackage{gensymb}
% For "\mathbb" macro
\usepackage{amsfonts}
% For positioning 
\usepackage{indentfirst}
\usetikzlibrary{shapes,positioning,fit,calc}
% for adjustwidth environment
\usepackage{changepage}
% for arrow on top
\usepackage{esvect}
% for mathbb 1
\usepackage{bbm}
% for mathsrc
\usepackage[mathscr]{eucal}
% For degree sign
\usepackage{gensymb}
% For quotes
\usepackage{csquotes}
% For vertical lines
\usepackage{mathtools}
% For cols
\usepackage{multicol}

% for tikz
\usepackage{pgfplots}
\pgfplotsset{compat=1.18}
\usepackage{amsmath}
\usepgfplotslibrary{groupplots}


%===============================================================================
%==================================== FONTS ====================================
%===============================================================================


% Mathcal
\newcommand{\acal}{\mathcal{A}}
\newcommand{\bcal}{\mathcal{B}}
\newcommand{\ccal}{\mathcal{C}}
\newcommand{\dcal}{\mathcal{D}}
\newcommand{\ecal}{\mathcal{E}}
\newcommand{\fcal}{\mathcal{F}}
\newcommand{\gcal}{\mathcal{G}}
\newcommand{\hcal}{\mathcal{H}}
\newcommand{\ical}{\mathcal{I}}
\newcommand{\jcal}{\mathcal{J}}
\newcommand{\kcal}{\mathcal{K}}
\newcommand{\lcal}{\mathcal{L}}
\newcommand{\mcal}{\mathcal{M}}
\newcommand{\ncal}{\mathcal{N}}
\newcommand{\ocal}{\mathcal{O}}
\newcommand{\pcal}{\mathcal{P}}
\newcommand{\qcal}{\mathcal{Q}}
\newcommand{\rcal}{\mathcal{R}}
\newcommand{\scal}{\mathcal{S}}
\newcommand{\tcal}{\mathcal{T}}
\newcommand{\ucal}{\mathcal{U}}
\newcommand{\vcal}{\mathcal{V}}
\newcommand{\wcal}{\mathcal{W}}
\newcommand{\xcal}{\mathcal{X}}
\newcommand{\ycal}{\mathcal{Y}}
\newcommand{\zcal}{\mathcal{Z}}

% Mathfrak
\newcommand{\afrak}{\mathfrak{A}}
\newcommand{\bfrak}{\mathfrak{B}}
\newcommand{\cfrak}{\mathfrak{C}}
\newcommand{\dfrak}{\mathfrak{D}}
\newcommand{\efrak}{\mathfrak{E}}
\newcommand{\ffrak}{\mathfrak{F}}
\newcommand{\gfrak}{\mathfrak{G}}
\newcommand{\hfrak}{\mathfrak{H}}
\newcommand{\ifrak}{\mathfrak{I}}
\newcommand{\jfrak}{\mathfrak{J}}
\newcommand{\kfrak}{\mathfrak{K}}
\newcommand{\lfrak}{\mathfrak{L}}
\newcommand{\mfrak}{\mathfrak{M}}
\newcommand{\nfrak}{\mathfrak{N}}
\newcommand{\ofrak}{\mathfrak{O}}
\newcommand{\pfrak}{\mathfrak{P}}
\newcommand{\qfrak}{\mathfrak{Q}}
\newcommand{\rfrak}{\mathfrak{R}}
\newcommand{\sfrak}{\mathfrak{S}}
\newcommand{\tfrak}{\mathfrak{T}}
\newcommand{\ufrak}{\mathfrak{U}}
\newcommand{\vfrak}{\mathfrak{V}}
\newcommand{\wfrak}{\mathfrak{W}}
\newcommand{\xfrak}{\mathfrak{X}}
\newcommand{\yfrak}{\mathfrak{Y}}
\newcommand{\zfrak}{\mathfrak{Z}}

% Mathscr
\newcommand{\ascr}{\mathscr{A}}
\newcommand{\bscr}{\mathscr{B}}
\newcommand{\cscr}{\mathscr{C}}
\newcommand{\dscr}{\mathscr{D}}
\newcommand{\escr}{\mathscr{E}}
\newcommand{\fscr}{\mathscr{F}}
\newcommand{\gscr}{\mathscr{G}}
\newcommand{\hscr}{\mathscr{H}}
\newcommand{\iscr}{\mathscr{I}}
\newcommand{\jscr}{\mathscr{J}}
\newcommand{\kscr}{\mathscr{K}}
\newcommand{\lscr}{\mathscr{L}}
\newcommand{\mscr}{\mathscr{M}}
\newcommand{\nscr}{\mathscr{N}}
\newcommand{\oscr}{\mathscr{O}}
\newcommand{\pscr}{\mathscr{P}}
\newcommand{\qscr}{\mathscr{Q}}
\newcommand{\rscr}{\mathscr{R}}
\newcommand{\sscr}{\mathscr{S}}
\newcommand{\tscr}{\mathscr{T}}
\newcommand{\uscr}{\mathscr{U}}
\newcommand{\vscr}{\mathscr{V}}
\newcommand{\wscr}{\mathscr{W}}
\newcommand{\xscr}{\mathscr{X}}
\newcommand{\yscr}{\mathscr{Y}}
\newcommand{\zscr}{\mathscr{Z}}

% Mathbb
\newcommand{\abb}{\mathbb{A}}
\newcommand{\bbb}{\mathbb{B}}
\newcommand{\cbb}{\mathbb{C}}
\newcommand{\dbb}{\mathbb{D}}
\newcommand{\ebb}{\mathbb{E}}
\newcommand{\fbb}{\mathbb{F}}
\newcommand{\gbb}{\mathbb{G}}
\newcommand{\hbb}{\mathbb{H}}
\newcommand{\ibb}{\mathbb{I}}
\newcommand{\jbb}{\mathbb{J}}
\newcommand{\kbb}{\mathbb{K}}
\newcommand{\lbb}{\mathbb{L}}
\newcommand{\mbb}{\mathbb{M}}
\newcommand{\nbb}{\mathbb{N}}
\newcommand{\obb}{\mathbb{O}}
\newcommand{\pbb}{\mathbb{P}}
\newcommand{\qbb}{\mathbb{Q}}
\newcommand{\rbb}{\mathbb{R}}
\newcommand{\sbb}{\mathbb{S}}
\newcommand{\tbb}{\mathbb{T}}
\newcommand{\ubb}{\mathbb{U}}
\newcommand{\vbb}{\mathbb{V}}
\newcommand{\wbb}{\mathbb{W}}
\newcommand{\xbb}{\mathbb{X}}
\newcommand{\ybb}{\mathbb{Y}}
\newcommand{\zbb}{\mathbb{Z}}


%===============================================================================
%=============================== SPECIAL SYMBOLS ===============================
%===============================================================================


% Orbit (group theory)
\newcommand{\orbit}{\mathcal{O}}
% Normal group
\newcommand{\normal}{\mathcal{N}}
% Indicator function
\newcommand{\indicator}{\mathbbm{1}}
% Laplace transform
\newcommand{\laplace}[1]{\mathcal{L}}
% Epsilon shorthand
\newcommand{\eps}{\varepsilon}
% Omega
\newcommand{\om}{\omega}
\newcommand{\Om}{\Omega}

%===============================================================================
%================================== OPERATORS ==================================
%===============================================================================


% Inverse exponent
\newcommand{\inv}[0]{^{-1}}
% Overline bar
\newcommand{\olsi}[1]{\,\overline{\!{#1}}}
% Less than or equal slanted
\newcommand{\seqs}{\leqslant}
% Greater or equal slanted
\newcommand{\geqs}{\geqslant}
% Subset or equal
\newcommand{\sub}{\subseteq}
% Proper subset
\newcommand{\prosub}{\subset}
% from
\newcommand{\from}{\leftarrow}

% Parantheses
\newcommand{\para}[1]{\left( #1 \right)}
% Curly Braces
\newcommand{\curl}[1]{\left\{ #1 \right\}}
% Brackets
\newcommand{\brac}[1]{\left[ #1 \right]}
% Angled Brackets
\newcommand{\ang}[1]{\left\langle #1 \right\rangle}
% Norm
\newcommand{\norm}[1]{\left\| #1 \right\|}

% Piece wise (use \\ between cases)
\newcommand{\piecewise}[1]{\begin{cases} #1 \end{cases}}

% Bold symbol shorthand
\newcommand{\bl}{\boldsymbol}

% Vertical space
\newcommand{\vs}[1]{\vspace{#1 pt}}
% Horizontal ertical space
\newcommand{\hs}[1]{\hspace{#1 pt}}


%===============================================================================
%============================== TEXT BASED SYMBOLS =============================
%===============================================================================


% Radians
\newcommand{\rad}{\text{rad}}
% Least Common Multiple
\newcommand{\lcm}{\text{lcm}}
% Automorphism
\newcommand{\Aut}{\text{Aut}}
% Variance
\newcommand{\var}{\text{Var}}
% Covariance
\newcommand{\cov}{\text{Cov}}
% Cofactor (matrix)
\newcommand{\cof}{\text{Cof}}
% Adjugate (matrix)
\newcommand{\adj}{\text{Adj}}
% Trace (matrix)
\newcommand{\tr}{\text{tr}}
% Standard deviation
\newcommand{\std}{\text{Std}}
% Correlation coefficient
\newcommand{\corr}{\text{Corr}}
% Sign
\newcommand{\sign}{\text{sign}}

% And text
\newcommand{\AND}{\text{ and }}
% Or text 
\newcommand{\OR}{\text{ or }}
% For text 
\newcommand{\FOR}{\text{ for }}
% If text
\newcommand{\IF}{\text{ if }}
% When text
\newcommand{\WHEN}{\text{ when }}
% Where text
\newcommand{\WHERE}{\text{ where }}
% Then text
\newcommand{\THEN}{\text{ then }}
% Such that text
\newcommand{\SUCHTHAT}{\text{ such that }}

% 1st
\newcommand{\st}[1]{#1^{\text{st}}}
% 2nd
\newcommand{\nd}[1]{#1^{\text{nd}}}
% 3rd
\newcommand{\rd}[1]{#1^{\text{rd}}}
% nth
\newcommand{\nth}[1]{#1^{\text{th}}}


%===============================================================================
%========================= PROBABILITY AND STATISTICS ==========================
%===============================================================================


% Permutation
\newcommand{\perm}[2]{{}^{#1}\!P_{#2}}
% Combination
\newcommand{\comb}[2]{{}^{#1}C_{#2}}

% Baye's risk
\newcommand{\risk}[1]{\mathscr{R}_{#1}}
% Baye's optimal risk
\newcommand{\riskOptimal}[1]{\mathscr{R}_{#1}^*}
% Baye's empirical risk
\newcommand{\riskEmpirical}[2]{\hat{\mathscr{R}}_{#1}^{#2}}


%===============================================================================
%=================================== CALCULUS ==================================
%===============================================================================


% d over d derivative
\newcommand{\dd}[2]{\dfrac{d#1}{d#2}}
% partial d over d derivative
\newcommand{\partialdd}[2]{\dfrac{\partial #1}{\partial #2}}
% delta d over d derivative
\newcommand{\deltadd}[2]{\dfrac{\Delta #1}{\Delta #2}}

% Integration between a and b
\newcommand{\integral}[4]{\int_{#1}^{#2} #3 \, #4}
% Integration in some space 
\newcommand{\boundIntegral}[2]{\int_{#1} #2 \, d#1}

% Limit
\newcommand{\limit}[3]{\lim_{#1 \to #2} #3}


%===============================================================================
%================================  BIG SYMBOLS  ================================
%===============================================================================


% Sum
\newcommand{\sumof}[2]{\sum_{#1}^{#2}}
% Product
\newcommand{\productof}[2]{\prod_{#1}^{#2}}
% Union
\newcommand{\unionof}[2]{\bigcup_{#1}^{#2}}
% Intersection
\newcommand{\intersectionof}[2]{\bigcap_{#1}^{#2}}
% Or
\newcommand{\orof}[2]{\bigvee_{#1}^{#2}}
% And
\newcommand{\andof}[2]{\bigwedge_{#1}^{#2}}


%===============================================================================
%=============================== LINEAR ALGEBRA ================================
%===============================================================================


% Bold vector arrow
\newcommand{\bv}[1]{\vec{\mathbf{#1}}}

% Matrix or vector (use // for column, & for row) 
% with brackets
\newcommand{\bmat}[1]{\begin{bmatrix} #1 \end{bmatrix}}
% Matrix or vector (use // for column, & for row) 
% with curved brackets
\newcommand{\pmat}[1]{\begin{pmatrix} #1 \end{pmatrix}}
% Matrix or vector (use // for column, & for row) 
% with lines on either side 
\newcommand{\lmat}[1]{\begin{vmatrix} #1 \end{vmatrix}}
% Matrix or vector (use // for column, & for row) 
% with curly braces
\newcommand{\cmat}[1]{\begin{Bmatrix} #1 \end{Bmatrix}}
% Matrix or vector (use // for column, & for row) 
% with no braces
\newcommand{\mat}[1]{\begin{matrix} #1 \end{matrix}}


%===============================================================================
%================================ LARGE OBJECTS ================================
%===============================================================================

% Multiple lines
\newcommand{\multiline}[1]{
\begin{align*}
    #1
\end{align*}
}

% Multiple lines with equation numbers
\newcommand{\eqmultiline}[1]{
\begin{align*}
    #1
\end{align*}
}

% Color
\newcommand{\colorText}[2]{
\begingroup
\color{#1}
    #2
\endgroup
}

% Centered figure
\newcommand{\centerFigure}[2]{
    \begin{figure}[h]
        \centering
            #1
        \caption{#2}
    \end{figure}
}

% Tikz figure
\newcommand{\tikzGraphic}[1]{
    \begin{center}
    \begin{tikzpicture}
        #1
    \end{tikzpicture}
    \end{center}
}

% Enumerate numbers (seperate by \item)
\newcommand{\numbers}{\textbf{\number*)}}

% Enumerate letters (seperate by \item)
\newcommand{\letters}{\textbf{\alph*)}}

\title{%
    \Huge Numerical Algorithms \\
    \Large Lecture I
}
\date{2025-04-13}
\author{Michael Saba}

\begin{document}
\pagenumbering{gobble}
\maketitle
\newpage
\setlength{\parindent}{0pt}
\pagenumbering{arabic}

    \subsection*{Introduction}

    The dot product of two vectors is defined As
    \[ \bv{u} \cdot \bv{v} = \sum u_iv_i \]
    and can be thought of as the measure of how much
    one vector travels in the direction of the other
    (the length of the projection of one vector
    on the other). \\

    Moreover, 
    \[ \bv{u} \cdot \bv{v} = \|u\|\|v\|\cos(\theta) \]
    where $\theta$ is the angle between the two vectors. \\
    We can use this definition to calculate the
    angle between vectors. \\

    The norm, or magnitude, of a vector,
    is its length.
    We can calculate it using the Pythagorean Theorem:
    \[ \| \bv{u} \| = \sqrt{\sum {u_i}^2 } \]
    which can also be expressed as: 
    \[ \| \bv{u} \| = \sqrt{u \cdot u} \] \\

    We can define surfaces in $\rbb^n$ as 
    implicit vector functions
    \[ f(\bv{p}) = 0 \]
    where $\bv{p}$ represents any point in $\rbb^n$
    and $f(\bv{p}) = 0$ if $\bv{p}$ is on the surface
    described by $f$. \\  
    
    This kind of function $ f(\bv{p}) = 0$
    is called implicit.
    An implicit function is characterized by the fact
    that it can be very easy to check for
    membership on $f$. \\
    It is relatively simple to check if a point 
    $\bv{p}$ is on the surface defined by $f$.
    We just plug $\bv{p}$ into $f$
    and check to see if the equation holds. \\
    However, it can be hard to use it to generate
    a point on $f$. \\

    The gradient of a function $f(\bv{p}) = 0$
    is a vector $\nabla f$ field defined as:
    \[
        F(\bv{p}) = \left(
            \partialdd{f}{p_1},
            \partialdd{f}{p_2},
            \dots
            \partialdd{f}{p_n}
        \right)
    \]
    This vector represents the normal of the surface,
    which is a vector orthogonal to the surface of $f$
    at $\bv{p}$. \\
    
    We can define a line using a paramtetric
    vector function:
    \[ \bv{r}(t) = \bv{o} + t\bv{d} \]
    where $\bv{o}$ is the point the vector originates from,
    $\bv{d}$ is the normalized direction of the line,
    and $t$ is a time parameter.
    As $t$ increases, the point $\bv{p}$ moves along the
    line starting at the origin $\bv{o}$. \\

    This kind of function is called parametric.
    It makes very easy to generate a new point
    on the surface of $\bv{r}$,
    by simply choosing any argument $t$. \\
    However, it can be hard to check for membership
    into $\bv{r}$. \\

    We can combine these two approaches to writing
    vector functions in order to 
    determine if and where a line intersects
    a surface $f(\bv{p}) = 0$.
    We just need to replace
    $\bv{p}$ by the line function $\bv{o} + t\bv{d}$:
    \[ f(\bv{o} + t\bv{d}) = 0 \]
    We can then solve for $t$. \\

    If we have one or more real solutions,
    we can conclude that at those times $t$,
    the line $\bv{r}(t)$ intersects the surface. \
    If we have no real solutions,
    the line does not intersect the surface. \\
    
    Later on, when chekcing if rays intersect surfaces,
    we will also require that $t$ be positive.
    This is because rays only travel in the forward
    direction
    (if a negative solution exists,
    the ray would have intersected if it were traveling
    backwards in the opposite direction).

    \newpage

    \subsection*{Intersection with a Plane}

    In $\rbb^2$, the implicit equation of a line is:
    \[
        Ax + By + C = 0
    \]
    We can express this line in vector
    notation:
    \[ \vecTwo{A}{B} \cdot \vecTwo{x}{y} 
    + \vecTwo{A}{B}  \cdot \vecTwo{q_1}{q_2} = 0 \]
    \[ \vecTwo{A}{B}  \cdot \left( \vecTwo{x}{y}
    + \vecTwo{q_1}{q_2} \right) = 0 \]
    where
    \[ \vecTwo{A}{B} \cdot \vecTwo{q_1}{q_2} = c \]
    Notice that
    \[ \vecTwo{A}{B} \]
    is the gradient $\nabla f$ of the function,
    meaning that it represents the normal $\bv{n}$
    to the surface.
    (which is constant since this is a line). \\
    Moreover, the equation 
    \[ \vecTwo{A}{B} \cdot \vecTwo{q_1}{q_2} = c \]
    can be satisfied by many points $\bv{q}$,
    as long as they are on the line. \\
    To conclude, we can define a line
    using an implicit vector function
    \[ \bv{n}(\bv{p} - \bv{q}) = 0 \]
    where the equation holds if $\bv{p}$
    is on the line. \\
    
    This defintion can easily be extended
    to higher dimensional planes and hyperplanes. \\

    Now, to analyze the function itself:
    \[
        \bv{n} \cdot (\bv{p} - \bv{q}) = 0
    \]
    As we know, each hyperplane has a normal vector that
    defines its orientation in space.
    This gives us the orientation, but not its position
    in space.
    If however we also include one point that we know
    is on the hyperplane, we are able to lock its position 
    in space. \\
    This definition does just that,
    by including the normal $\bv{n}$
    and a point $\bv{q}$ on the plane. \\
    The vector $(\bv{p} - \bv{q})$
    is the one connecting $\bv{q}$ to $\bv{p}$. \\
    If $\bv{n} \cdot (\bv{p} - \bv{q}) = 0$,
    then the normal of the hyperplane
    is perpendicular (orthogonal) to the vector
    $\bv{p} - \bv{q}$,
    meaning that $\bv{p}$ has to be on the plane. \\

    So to determine if and where a line intersects 
    any hyperplane,
    we replace the parametric equation of a line
    in the implicit equation of the hyperplane,
    and solve for $t$:
    \[
        \bv{n} \cdot ((\bv{o} + t\bv{d}) - \bv{q}) = 0
    \]
    which we can do as such
    \[
        t(\bv{n} \cdot \bv{d}) 
        + \bv{n} \cdot (\bv{o} - \bv{q}) = 0
    \]
    \[
        t(\bv{n} \cdot \bv{d}) =
        \bv{n} \cdot (\bv{q} - \bv{o})
    \]
    \[
        t = \dfrac  {\bv{n} \cdot (\bv{q} - \bv{o})}
                    {(\bv{n} \cdot \bv{d})}
    \]
    If $t$ exists, we have our intersection. \\
    If the denominator $(\bv{n} \cdot \bv{d}) = 0$,
    then there is no $t$ that satisfies the equation.
    This is because when the normal of the plane $\bv{n}$
    is orthogonal to the direction of the ray $\bv{d}$,
    the ray is parallel to the plane,
    and thus does not intersect it. \\
    However, it is possible that the numerator
    \[\bv{n} \cdot (\bv{q} - \bv{o})\]
    is also $0$,
    in which case the line is on the hyperplane
    and there are thus infinite solutions. \\

    \newpage

    \subsection*{Intersection with a Sphere}
    
    The implicit definition of a sphere
    using a scalar function $f(x, y, z) = 0$ is the following:
    \[
        (x - c_x)^2 +  (y - c_y)^2 +  (z - c_z)^2 = r^2 
    \]
    where the point $(c_x, c_y, c_z)$
    is the center of the sphere,
    and $r$ is its radius. \\

    We can rewrite this definition using an
    implicit vector function:
    \[ \| \bv{p} - \bv{c} \|^2 - r^2 = 0 \]
    where $\bv{c}$
    is the center $(c_x, c_y, c_z)$ of the sphere
    and $\|\bv{p} - \bv{c} \|^2$ is the magnitude (length)
    of the vector $\bv{p} - \bv{c}$.
    Writing $\bv{p} - \bv{c}$ is a way of expressing
    the vector pointing from $\bv{c}$ to $\bv{p}$. \\
    So, the equation 
    $\| \bv{p} - \bv{c} \|^2 - r^2 = 0$
    indicates that the point $\bv{p}$ is
    on the surface of the sphere
    if length of the vector
    from the center $\bv{c}$ of the sphere
    to $\bv{p}$ macthes the radius $r$. \\
    Note that the vector function
    is the same as the scaler function:
    \[ 
        \| \bv{p} - \bv{c} \|^2
        = (x - c_x)^2 
        + (y - c_y)^2 +  (z - c_z)^2
    \]
    only written using vector notation. \\

    In order to determine if a ray intersects a sphere,
    we solve the following equation for $t$:
    \[ 
        \| (\bv{o} + t\bv{d}) - \bv{c} \|^2 - r^2 = 0
    \]
    Which can be done by reducing it to a quadratic
    and solving for $t$. \\
    If the quadratic has one or more real solutions,
    the ray intersects the sphere. \\
    Otherwise, if the solutions are complex,
    the ray does not intersect the sphere. \\

    This can be done as such:
    \[ 
        \| (\bv{o} + t\bv{d}) - \bv{c} \|^2 - r^2 = 0
    \]
    \[ 
        \| t\bv{d} + (\bv{o} - \bv{c}) \|^2 - r^2 = 0
    \]
    \[ 
        (t\bv{d} + (\bv{o} - \bv{c})) \cdot
        (t\bv{d} + (\bv{o} - \bv{c})) - r^2 = 0
    \]
    \[ 
        t^2(\bv{d} \cdot \bv{d}) 
        + 2t\bv{d} \cdot (\bv{o} - \bv{c})
        + (\bv{o} - \bv{c})\cdot(\bv{o} - \bv{c}) - r^2 = 0
    \]
    where 
    \[ A = \bv{d} \cdot \bv{d} \]
    \[ B = 2\bv{d} \cdot (\bv{o} - \bv{c}) \]
    \[ C = (\bv{o} - \bv{c})\cdot(\bv{o} - \bv{c}) - r^2 \]
    Then, depending on whether the discriminant
    $B^2  -4AC$
    is smaller to, equal, or larger than 0,
    we may have no solutions, one solution, or two:
    \[ t = \dfrac{-B \pm \sqrt{B^2 - 4AC} }{2A} \]

    \newpage

    \subsection*{Linear System of Equations}

    Suppose we have two line
    (or any $n+1$ $n$-dimensional hyperplane)
    equations in the implicit form.
    How could we check where they intersect?
    This is not as easy as having one of the two
    in the parametric format. \\
    The answer is that we have to solve a linear
    system to find a point where they all meet.
    A point $p$ that satisfies the implicit
    equation
    \[ f(\bv{p}) = 0 \]
    for all of the lines. \\

    Suppose we have two line equations:
    \[ a_1^1x^1 + a_1^2x^2 = b^1 \]
    \[ a_2^1x^1 + a_2^2x^2 = b^2 \]
    To find a soltuion $(x^1, x^2)$ 
    that satisfies both linear equations
    means finding the point where they intersect. \\

    One way to solve them is by using substitution.
    Using the first equation, we can write:
    \[ x^1 = \dfrac{b^1 - a_1^2x^2}{a_1^1} \]
    and then substitute it in the second equation,
    and solve for $x^2$:
    \[ a_2^1\dfrac{b^1 - a_1^2x^2}{a_1^1} + a_2^2x^2 = b^2 \]
    \[ \dfrac{a_2^1b^1 - a_2^1a_1^2x^2 
    + a_1^1a_2^2x^2}{a_1^1} = b^2 \]
    \[ \dfrac{a_1^1a_2^2x^2 - a_2^1a_1^2x^2}{a_1^1} = 
    \dfrac{a_1^1b^2 - a_2^1b^1}{a_1^1} \]
    We now assume that $a_1^1 \neq 0$ and simplify:
    \[ a_1^1a_2^2x^2 - a_2^1a_1^2x^2 = 
    a_1^1b^2 - a_2^1b^1 \]
    \[ x^2(a_1^1a_2^2 - a_2^1a_1^2) = 
    a_1^1b^2 - a_2^1b^1 \]
    \[ x^2 = 
    \dfrac{a_1^1b^2 - a_2^1b^1}{a_1^1a_2^2 - a_2^1a_1^2} \]
    We can then replace $x^2$ in the first equation
    and solve for $x^1$, giving us:
    \[ x^1 = 
    \dfrac{a_2^2b^1 - a_1^2b^2}{a_1^1a_2^2 - a_2^1a_1^2} \]
    This approach motivates Cramer's Rule. \\

    Another approach is elimination. \\
    We can multiply the first row by $\dfrac{1}{a_1^1}$:
    \[ x^1 + \dfrac{a_1^2}{a_1^1}x^2 = \dfrac{b_1}{a_1^1} \]
    this assumes, like first method, that $a_1^1 \neq 0$. \\
    We then substitute $-a_2^1$ the first equation
    from the second, eliminating the first variable $x_1$:
    \[ (a_2^2 - \dfrac{a_1^2a_2^1}{a_1^1}) x^2 = 
    b^2 - \dfrac{a_2^1b_1}{a_1^1} \]
    We can then solve for $x^2$,
    and use it to get $x^1$. \\
    This approach motivates Guassian elimination,
    a much more efficient algorithm for solving
    linear systems. \\

    Cramer's rule in general is by far the slowest.
    This is because calculating the determinant of
    a matrix has time complexity $O(n!)$. \\
    Guassian-elimination is faster. \\
    Cramer's rule is however fine to use
    for a small number of entries,
    such as for 3 by 3 and 4 by 4 matrices,
    where its performance actually beats
    that of elimination, which is only asymptotically
    better (as the size grows). \\
    However, the fastest technique is using
    a numerical method, which gives an approximation
    of the solution. \\

    We can think of linear systems in two ways. \\

    The first way is to think of it as two lines
    \[ a_1^1x^1 + a_1^2x^2 = b^1 \]
    \[ a_2^1x^1 + a_2^2x^2 = b^2 \]
    intersecting each other (or $n+1$ $n$-dimensional hyperplanes)
    at the solution. \\
    We can compactly write this using dot products:
    \[ \vecTwo{a_1^1}{a_1^2} \cdot \vecTwo{x^1}{x^2}
    = b^1 \]
    \[ \vecTwo{a_2^1}{a_2^2} \cdot \vecTwo{x^1}{x^2}
    = b^2 \]
    Notice how each row is turned into a vector.
    This is the row major approach. \\
    If the two lines are parallel, we have no solutions,
    unless the two lines are the same, then we have
    infinite solutions. \\

    The second way to think of it is as two vectors
    \[ \vecTwo{a_1^1}{a_2^1} \qquad \vecTwo{a_1^2}{a_2^2} \]
    and we need to find some linear combination of them
    equal to the point $b$: 
    \[ x^1\vecTwo{a_1^1}{a_2^1} + x^2\vecTwo{a_1^2}{a_2^2}
    = \vecTwo{b^1}{b^2} \]
    Notice how each column in the system is a vector.
    This is the column major approach. \\
    If the two vectors are colinear,
    we call them linearly dependent.
    Instead of being able to span all of $\rbb^2$
    by linearly combining them, we can only span one line,
    so we most likley won't be able to each $b$,
    and we thus have no solution. \\
    Unless $b$ is on the line they span,
    then we have infinit ways to combine the two vectors
    to get to $b$, giving us infinite solutions. \\

    Notice that the linear system is uniquely
    defined by 6 numbers:
    \[ a_1^1 \quad a_1^2 \quad a_2^1 \quad a_2^2
    \quad b^1 \quad b^2 \]
    We can compactly write these using matrix notation:
    \[ \matTwo{a_1^1}{a_1^2}{a_2^1}{a_2^2}
    \vecTwo{x^1}{x^2} = \vecTwo{b^1}{b^2} \]
    Now we motivate matrix multiplication.
    Matrix multiplication of a vector needs
    to yield the original system:
    \[ \matTwo{a_1^1}{a_1^2}{a_2^1}{a_2^2}
    \vecTwo{x^1}{x^2}
    = \vecTwo{a_1^1x^1 + a_2^1x^2}{a_1^2x^1 + a_2^2x^2}
    = \vecTwo{b^1}{b^2} \]
    Notice how we can still think of matrix multiplication
    in two ways.
    The row major approach:
    \[ \vecTwo{a_1^1x^1 + a_2^1x^2}{a_1^2x^1 + a_2^2x^2}
    = \vecTwo{\vecTwo{a_1^1}{a_1^2} \cdot \vecTwo{x^1}{x^2}}
    {\vecTwo{a_2^1}{a_2^2} \cdot \vecTwo{x^1}{x^2}}
    = \vecTwo{b^1}{b^2} \]
    And the column major approach:
    \[ \vecTwo{a_1^1x^1 + a_2^1x^2}{a_1^2x^1 + a_2^2x^2}
    = x^1\vecTwo{a_1^1}{a_2^1} + x^2\vecTwo{a_1^2}{a_2^2}
    = \vecTwo{b^1}{b^2} \]
    \\

    More generally however, we can multiply
    any $m \times k$ matrix with a $k \times n$
    matrix, and get an $m\times n$ matrix as a result
    (here we think of vectors as $n \times 1$ matrices,
    or $1 \times n$ matrices when transposed).
    So for $A \times B = C$:
    \[ c_{ij} = \sum_{i}^{k}a_{ik}b_{kj} \]
    is the formula for any entry in the result matrix. \\

    We can formalize Cramer's rule,
    where:
    \[ x^1 = \dfrac{
        \det{\matTwo{b^1}{a_1^2}{b^2}{a_2^2}}
    }{\det{\matTwo{a_1^1}{a_2^1}{a_1^2}{a_2^2}}} \]
    \[ x^2 = \dfrac{
        \det{\matTwo{a_1^1}{b^1}{a_1^2}{b^2}}
    }{\det{\matTwo{a_1^1}{a_2^1}{a_1^2}{a_2^2}}} \]
    If teh determinant is 0, 
    the system has no unique solution. \\

    The determinant represents the scaling
    factor of a linear transformation. \\
    If a linear transformation has determinant 0,
    that means it flattens the entire $\rbb^2$
    vector space onto a line (making the are 0). \\
    This happens if instead of 2 linearly
    independent column vectors, we have two colinear
    vectors, causing the entire space
    to flatten onto that line. \\
    We call a matrix with determinant 0 singular,
    or non-invertible (has no inverse). \\
    It has no inverse because the inverse
    of a linear transformation reverses the effects
    of the original transformation.
    However, given a line that we get from
    applying a singular matrix to $\rbb^2$,
    there's no well defined (unique) way to get 
    from one dimensional vector space $\rbb$ to $\rbb^2$. \\

 
\end{document}7