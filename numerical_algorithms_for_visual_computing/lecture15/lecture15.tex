


\documentclass[12pt]{article}

\usepackage[margin=1in]{geometry}


%===============================================================================
%================================== PACKAGES ===================================
%===============================================================================

% For using float option H that places figures 
% exatcly where we want them
\usepackage{float}
% makes figure font bold
\usepackage{caption}
\captionsetup[figure]{labelfont=bf}
% For text generation
\usepackage{lipsum}
% For drawing
\usepackage{tikz}
% For smaller or equal sign and not divide sign
\usepackage{amssymb}
% For the diagonal fraction
\usepackage{xfrac}
% For enumerating exercise parts with letters instead of numbers
\usepackage{enumitem}
% For dfrac, which forces the fraction to be in display mode (large) e
% even in math mode (small)
\usepackage{amsmath}
% For degree sign
\usepackage{gensymb}
% For "\mathbb" macro
\usepackage{amsfonts}
% For positioning 
\usepackage{indentfirst}
\usetikzlibrary{shapes,positioning,fit,calc}
% for adjustwidth environment
\usepackage{changepage}
% for arrow on top
\usepackage{esvect}
% for mathbb 1
\usepackage{bbm}
% for mathsrc
\usepackage[mathscr]{eucal}
% For degree sign
\usepackage{gensymb}
% For quotes
\usepackage{csquotes}
% For vertical lines
\usepackage{mathtools}
% For cols
\usepackage{multicol}

% for tikz
\usepackage{pgfplots}
\pgfplotsset{compat=1.18}
\usepackage{amsmath}
\usepgfplotslibrary{groupplots}


%===============================================================================
%==================================== FONTS ====================================
%===============================================================================


% Mathcal
\newcommand{\acal}{\mathcal{A}}
\newcommand{\bcal}{\mathcal{B}}
\newcommand{\ccal}{\mathcal{C}}
\newcommand{\dcal}{\mathcal{D}}
\newcommand{\ecal}{\mathcal{E}}
\newcommand{\fcal}{\mathcal{F}}
\newcommand{\gcal}{\mathcal{G}}
\newcommand{\hcal}{\mathcal{H}}
\newcommand{\ical}{\mathcal{I}}
\newcommand{\jcal}{\mathcal{J}}
\newcommand{\kcal}{\mathcal{K}}
\newcommand{\lcal}{\mathcal{L}}
\newcommand{\mcal}{\mathcal{M}}
\newcommand{\ncal}{\mathcal{N}}
\newcommand{\ocal}{\mathcal{O}}
\newcommand{\pcal}{\mathcal{P}}
\newcommand{\qcal}{\mathcal{Q}}
\newcommand{\rcal}{\mathcal{R}}
\newcommand{\scal}{\mathcal{S}}
\newcommand{\tcal}{\mathcal{T}}
\newcommand{\ucal}{\mathcal{U}}
\newcommand{\vcal}{\mathcal{V}}
\newcommand{\wcal}{\mathcal{W}}
\newcommand{\xcal}{\mathcal{X}}
\newcommand{\ycal}{\mathcal{Y}}
\newcommand{\zcal}{\mathcal{Z}}

% Mathfrak
\newcommand{\afrak}{\mathfrak{A}}
\newcommand{\bfrak}{\mathfrak{B}}
\newcommand{\cfrak}{\mathfrak{C}}
\newcommand{\dfrak}{\mathfrak{D}}
\newcommand{\efrak}{\mathfrak{E}}
\newcommand{\ffrak}{\mathfrak{F}}
\newcommand{\gfrak}{\mathfrak{G}}
\newcommand{\hfrak}{\mathfrak{H}}
\newcommand{\ifrak}{\mathfrak{I}}
\newcommand{\jfrak}{\mathfrak{J}}
\newcommand{\kfrak}{\mathfrak{K}}
\newcommand{\lfrak}{\mathfrak{L}}
\newcommand{\mfrak}{\mathfrak{M}}
\newcommand{\nfrak}{\mathfrak{N}}
\newcommand{\ofrak}{\mathfrak{O}}
\newcommand{\pfrak}{\mathfrak{P}}
\newcommand{\qfrak}{\mathfrak{Q}}
\newcommand{\rfrak}{\mathfrak{R}}
\newcommand{\sfrak}{\mathfrak{S}}
\newcommand{\tfrak}{\mathfrak{T}}
\newcommand{\ufrak}{\mathfrak{U}}
\newcommand{\vfrak}{\mathfrak{V}}
\newcommand{\wfrak}{\mathfrak{W}}
\newcommand{\xfrak}{\mathfrak{X}}
\newcommand{\yfrak}{\mathfrak{Y}}
\newcommand{\zfrak}{\mathfrak{Z}}

% Mathscr
\newcommand{\ascr}{\mathscr{A}}
\newcommand{\bscr}{\mathscr{B}}
\newcommand{\cscr}{\mathscr{C}}
\newcommand{\dscr}{\mathscr{D}}
\newcommand{\escr}{\mathscr{E}}
\newcommand{\fscr}{\mathscr{F}}
\newcommand{\gscr}{\mathscr{G}}
\newcommand{\hscr}{\mathscr{H}}
\newcommand{\iscr}{\mathscr{I}}
\newcommand{\jscr}{\mathscr{J}}
\newcommand{\kscr}{\mathscr{K}}
\newcommand{\lscr}{\mathscr{L}}
\newcommand{\mscr}{\mathscr{M}}
\newcommand{\nscr}{\mathscr{N}}
\newcommand{\oscr}{\mathscr{O}}
\newcommand{\pscr}{\mathscr{P}}
\newcommand{\qscr}{\mathscr{Q}}
\newcommand{\rscr}{\mathscr{R}}
\newcommand{\sscr}{\mathscr{S}}
\newcommand{\tscr}{\mathscr{T}}
\newcommand{\uscr}{\mathscr{U}}
\newcommand{\vscr}{\mathscr{V}}
\newcommand{\wscr}{\mathscr{W}}
\newcommand{\xscr}{\mathscr{X}}
\newcommand{\yscr}{\mathscr{Y}}
\newcommand{\zscr}{\mathscr{Z}}

% Mathbb
\newcommand{\abb}{\mathbb{A}}
\newcommand{\bbb}{\mathbb{B}}
\newcommand{\cbb}{\mathbb{C}}
\newcommand{\dbb}{\mathbb{D}}
\newcommand{\ebb}{\mathbb{E}}
\newcommand{\fbb}{\mathbb{F}}
\newcommand{\gbb}{\mathbb{G}}
\newcommand{\hbb}{\mathbb{H}}
\newcommand{\ibb}{\mathbb{I}}
\newcommand{\jbb}{\mathbb{J}}
\newcommand{\kbb}{\mathbb{K}}
\newcommand{\lbb}{\mathbb{L}}
\newcommand{\mbb}{\mathbb{M}}
\newcommand{\nbb}{\mathbb{N}}
\newcommand{\obb}{\mathbb{O}}
\newcommand{\pbb}{\mathbb{P}}
\newcommand{\qbb}{\mathbb{Q}}
\newcommand{\rbb}{\mathbb{R}}
\newcommand{\sbb}{\mathbb{S}}
\newcommand{\tbb}{\mathbb{T}}
\newcommand{\ubb}{\mathbb{U}}
\newcommand{\vbb}{\mathbb{V}}
\newcommand{\wbb}{\mathbb{W}}
\newcommand{\xbb}{\mathbb{X}}
\newcommand{\ybb}{\mathbb{Y}}
\newcommand{\zbb}{\mathbb{Z}}


%===============================================================================
%=============================== SPECIAL SYMBOLS ===============================
%===============================================================================


% Orbit (group theory)
\newcommand{\orbit}{\mathcal{O}}
% Normal group
\newcommand{\normal}{\mathcal{N}}
% Indicator function
\newcommand{\indicator}{\mathbbm{1}}
% Laplace transform
\newcommand{\laplace}[1]{\mathcal{L}}
% Epsilon shorthand
\newcommand{\eps}{\varepsilon}
% Omega
\newcommand{\om}{\omega}
\newcommand{\Om}{\Omega}

%===============================================================================
%================================== OPERATORS ==================================
%===============================================================================


% Inverse exponent
\newcommand{\inv}[0]{^{-1}}
% Overline bar
\newcommand{\olsi}[1]{\,\overline{\!{#1}}}
% Less than or equal slanted
\newcommand{\seqs}{\leqslant}
% Greater or equal slanted
\newcommand{\geqs}{\geqslant}
% Subset or equal
\newcommand{\sub}{\subseteq}
% Proper subset
\newcommand{\prosub}{\subset}
% from
\newcommand{\from}{\leftarrow}

% Parantheses
\newcommand{\para}[1]{\left( #1 \right)}
% Curly Braces
\newcommand{\curl}[1]{\left\{ #1 \right\}}
% Brackets
\newcommand{\brac}[1]{\left[ #1 \right]}
% Angled Brackets
\newcommand{\ang}[1]{\left\langle #1 \right\rangle}
% Norm
\newcommand{\norm}[1]{\left\| #1 \right\|}

% Piece wise (use \\ between cases)
\newcommand{\piecewise}[1]{\begin{cases} #1 \end{cases}}

% Bold symbol shorthand
\newcommand{\bl}{\boldsymbol}

% Vertical space
\newcommand{\vs}[1]{\vspace{#1 pt}}
% Horizontal ertical space
\newcommand{\hs}[1]{\hspace{#1 pt}}


%===============================================================================
%============================== TEXT BASED SYMBOLS =============================
%===============================================================================


% Radians
\newcommand{\rad}{\text{rad}}
% Least Common Multiple
\newcommand{\lcm}{\text{lcm}}
% Automorphism
\newcommand{\Aut}{\text{Aut}}
% Variance
\newcommand{\var}{\text{Var}}
% Covariance
\newcommand{\cov}{\text{Cov}}
% Cofactor (matrix)
\newcommand{\cof}{\text{Cof}}
% Adjugate (matrix)
\newcommand{\adj}{\text{Adj}}
% Trace (matrix)
\newcommand{\tr}{\text{tr}}
% Standard deviation
\newcommand{\std}{\text{Std}}
% Correlation coefficient
\newcommand{\corr}{\text{Corr}}
% Sign
\newcommand{\sign}{\text{sign}}

% And text
\newcommand{\AND}{\text{ and }}
% Or text 
\newcommand{\OR}{\text{ or }}
% For text 
\newcommand{\FOR}{\text{ for }}
% If text
\newcommand{\IF}{\text{ if }}
% When text
\newcommand{\WHEN}{\text{ when }}
% Where text
\newcommand{\WHERE}{\text{ where }}
% Then text
\newcommand{\THEN}{\text{ then }}
% Such that text
\newcommand{\SUCHTHAT}{\text{ such that }}

% 1st
\newcommand{\st}[1]{#1^{\text{st}}}
% 2nd
\newcommand{\nd}[1]{#1^{\text{nd}}}
% 3rd
\newcommand{\rd}[1]{#1^{\text{rd}}}
% nth
\newcommand{\nth}[1]{#1^{\text{th}}}


%===============================================================================
%========================= PROBABILITY AND STATISTICS ==========================
%===============================================================================


% Permutation
\newcommand{\perm}[2]{{}^{#1}\!P_{#2}}
% Combination
\newcommand{\comb}[2]{{}^{#1}C_{#2}}

% Baye's risk
\newcommand{\risk}[1]{\mathscr{R}_{#1}}
% Baye's optimal risk
\newcommand{\riskOptimal}[1]{\mathscr{R}_{#1}^*}
% Baye's empirical risk
\newcommand{\riskEmpirical}[2]{\hat{\mathscr{R}}_{#1}^{#2}}


%===============================================================================
%=================================== CALCULUS ==================================
%===============================================================================


% d over d derivative
\newcommand{\dd}[2]{\dfrac{d#1}{d#2}}
% partial d over d derivative
\newcommand{\partialdd}[2]{\dfrac{\partial #1}{\partial #2}}
% delta d over d derivative
\newcommand{\deltadd}[2]{\dfrac{\Delta #1}{\Delta #2}}

% Integration between a and b
\newcommand{\integral}[4]{\int_{#1}^{#2} #3 \, #4}
% Integration in some space 
\newcommand{\boundIntegral}[2]{\int_{#1} #2 \, d#1}

% Limit
\newcommand{\limit}[3]{\lim_{#1 \to #2} #3}


%===============================================================================
%================================  BIG SYMBOLS  ================================
%===============================================================================


% Sum
\newcommand{\sumof}[2]{\sum_{#1}^{#2}}
% Product
\newcommand{\productof}[2]{\prod_{#1}^{#2}}
% Union
\newcommand{\unionof}[2]{\bigcup_{#1}^{#2}}
% Intersection
\newcommand{\intersectionof}[2]{\bigcap_{#1}^{#2}}
% Or
\newcommand{\orof}[2]{\bigvee_{#1}^{#2}}
% And
\newcommand{\andof}[2]{\bigwedge_{#1}^{#2}}


%===============================================================================
%=============================== LINEAR ALGEBRA ================================
%===============================================================================


% Bold vector arrow
\newcommand{\bv}[1]{\vec{\mathbf{#1}}}

% Matrix or vector (use // for column, & for row) 
% with brackets
\newcommand{\bmat}[1]{\begin{bmatrix} #1 \end{bmatrix}}
% Matrix or vector (use // for column, & for row) 
% with curved brackets
\newcommand{\pmat}[1]{\begin{pmatrix} #1 \end{pmatrix}}
% Matrix or vector (use // for column, & for row) 
% with lines on either side 
\newcommand{\lmat}[1]{\begin{vmatrix} #1 \end{vmatrix}}
% Matrix or vector (use // for column, & for row) 
% with curly braces
\newcommand{\cmat}[1]{\begin{Bmatrix} #1 \end{Bmatrix}}
% Matrix or vector (use // for column, & for row) 
% with no braces
\newcommand{\mat}[1]{\begin{matrix} #1 \end{matrix}}


%===============================================================================
%================================ LARGE OBJECTS ================================
%===============================================================================

% Multiple lines
\newcommand{\multiline}[1]{
\begin{align*}
    #1
\end{align*}
}

% Multiple lines with equation numbers
\newcommand{\eqmultiline}[1]{
\begin{align*}
    #1
\end{align*}
}

% Color
\newcommand{\colorText}[2]{
\begingroup
\color{#1}
    #2
\endgroup
}

% Centered figure
\newcommand{\centerFigure}[2]{
    \begin{figure}[h]
        \centering
            #1
        \caption{#2}
    \end{figure}
}

% Tikz figure
\newcommand{\tikzGraphic}[1]{
    \begin{center}
    \begin{tikzpicture}
        #1
    \end{tikzpicture}
    \end{center}
}

% Enumerate numbers (seperate by \item)
\newcommand{\numbers}{\textbf{\number*)}}

% Enumerate letters (seperate by \item)
\newcommand{\letters}{\textbf{\alph*)}}

\title{%
    \Huge Numerical Algorithms \\
    \Large Lecture XV, XVI
}
\date{2025-06-11}
\author{Michael Saba}

\begin{document}
\pagenumbering{gobble}
\maketitle
\newpage
\setlength{\parindent}{0pt}
\pagenumbering{arabic}


\subsection*{The Damped Wave Equation}

In order to ensure that the waves dampen and dissapear
eventually, we can introduce a damping term:
\[\partial_{tt} u + \gamma \partial_t u = c^2 \Delta u \]
The coefficient $\gamma$ can be space and time dependent. \\

To understand what is happening, we can reparametrize
the damping term:
\[ \gamma = \dfrac{\beta}{1 - \beta} \]
This gives us:
\[\partial_{tt} u + \dfrac{\beta}{1 - \beta} \partial_t u 
= c^2 \Delta u \]
\[(1 - \beta)\partial_{tt} u + \beta \partial_t u 
= (1 - \beta)c^2 \Delta u \]
Notice that $\beta$ multiplies $\partial_t u $,
which is the heat diffusion term. The larger $\beta$
is, the smaller the wave term $\partial_{tt} u$
is. This indicates that damping the wave is essentially
just diffusing it. \\

We can solve the equation
in the spectral domain after uncoupling it:
\[ d_{tt}\hat{u}_j(t) + \gamma d_t\hat{u}_j(t) = 
c^2 \lambda_j \hat{u}_j(t) \]
It can be solved using a harmonic oscillator.

\newpage

\subsection*{Stability}

The CFL condition is the condition placed on the
time step (when discretizing time) that gives us the
value range for which the simulation remains stable. \\

We will be using the wave equation as an example. \\

We know that, in forward Euler, we have:
\[ u^{k+1} = (I + \tau L)u^k \]
In order for such an equation to be stable, we require
the matrix multiplying $u^k$ to have a spectral
radius smaller or equal to $1$, where the spectral
radius is the largest absolute value of an eigenvalue. \\

This is because, in order for some matrix $A$ not to increase
the error introduced each iteration $\delta$,
we require that:
\[ \|A\delta\| \leq \|\delta\| \]
Which is achieved when $\rho(A) \leq 1$. \\

We know that the eigenvalues of $I + \tau L$
look like $1 + \tau \lambda_i$ where $\lambda_i$
is an eigenvalue of $L$. \\
However, $\lambda_i$ can be negative or positive,
we can assume either is true and work from there. \\

So we require that:
\[ |1 + \tau \lambda_i| \leq 1 \]
Which means that we want:
\[ 1 + \tau \lambda_i \leq 1 \qquad \AND \qquad 
1 + \tau \lambda_i \geq -1 \]
Assuming that $\lambda_i < 0$, this leads to:
\[ \tau \geq 0 \qquad \AND \qquad 
\tau \leq -\dfrac{2}{\lambda_i} \]
This means that:
\[ \tau \in \Big[0, \dfrac{2}{\rho(L)}\Big) \]
Which is our CFL condition. \\

Now, in order to find $\rho(L)$ or a bound on it,
we can use the Fourrier series to express $u(x)$. \\
Since $u$ is discretized, we can use the Discrete
Fourrier Transform, or $DFT$. \\
We will express the $u$ vector in the Fourrier domain. \\

We know that the stencil for $L$ is (in one dimension):
\[ [1 \quad -2 \quad 1] \]
We can then convert to the Fourrier domain for each
point in $u$ individually. \\
\[ \hat{u}_k = e^{2 \pi i (j-1) \sfrac{k}{n}}
-2 e^{2 \pi i (j) \sfrac{k}{n}} 
+ e^{2 \pi i (j+1) \sfrac{k}{n}}\]
Here $k$ is the frequency index, but it will
also allow us to find the $\nth{k}$ eigenvalue of $L$. \\
Since the stencil has 3 elements, we have three points
$j-1$, $j$, and $j+1$,
where we multiply each term by the stencil coefficient
that corresponds to it. \\
Here $n$ is the number of points in total. \\
We then factor out $e^{2 \pi i j \sfrac{k}{n}}$:
\[ e^{2 \pi i j \sfrac{k}{n}}(e^{-2 \pi i \sfrac{k}{n}}
-2 + e^{2 \pi i \sfrac{k}{n}}) \]
What's left is $\lambda_k$. \\
Now, we want the spectral radius: the largest absolute
value this expression can take is $4$, since 
each exponetial term is bound between $1$ and $-1$. \\

So we have:
\[ \tau \in \Big[0, \dfrac{1}{2}\Big) \]
in one dimension for the forward Euler heat
diffusion solution. \\

In more dimensions, we use the larger stencil, and
end up with a bound on $\rho(L)$ of $4 \cdot m$,
where $m$ is the dimension. \\

Note that while implicit Euler is unconditionally
stable, it does require using a linear solver to get
the solution. \\

\newpage

\subsection*{Consistency}

Consitency is what tells us how accurate our method is,
that is, the order of the error. \\

In general, when fitting a quadratic polynomial, 
we have an error of $O(h^2)$ when using a centered
derivative, and an error of $O(h)$ otherwise. \\

Now, to find the order of consistency of any stencil,
we can use the Taylor expansion. \\

Suppose we have a stencil:
\[ \dfrac{1}{h^2}[1 \quad -2 \quad 1] \]
We can call the corresponding points 
$u_{i-1}$, $u_i$, $u_{i+1}$. \\
Assuming that the step size is $h$, then:
\[ u_{i+1} = u(x_i + h) \qquad \AND \qquad 
u_{i-1} = u(x_i - h) \]
Using the taylor expansion of $u(x)$, this gives us:
\[ u_{i+1} = u(x_i + h) = u_i + h \dd{u_i}{x}
+ \dfrac{h^2}{2!} \dd{^2u_i}{x^2} 
+ \dfrac{h^3}{3!} \dd{^3u_i}{x^3} \dots \]
\[ u_{i-1} = u(x_i - h) = u_i - h \dd{u_i}{x}
+ \dfrac{h^2}{2!} \dd{^2u_i}{x^2} 
- \dfrac{h^3}{3!} \dd{^3u_i}{x^3} \dots \]
Placing them back inside the stencil, we notice 
that all of the odd terms of the two sums cancel out,
leaving us with:
\[ \dfrac{-2u_i + 2u_i + 2\dfrac{h^2}{2!}\dd{^2u_i}{x^2}
+ 2\dfrac{h^4}{4!}\dd{^4u_i}{x^4} ...}{h^2} \]
Here the $-2u_i$ from the stencil cancels out the
$2u_i$ we get from the taylor expansion sum:
\[ \dfrac{2\dfrac{h^2}{2!}\dd{^2u_i}{x^2}
+ 2\dfrac{h^4}{4!}\dd{^4u_i}{x^4} ...}{h^2} \]
Canceling out some more terms, we end up with
\[ \dd{^2u_i}{x^2} + 2\dfrac{h^2}{4!}\dd{^4u_i}{x^4} ... \]
Since the largest error is of order $2$,
that is, $dd{^2u_i}{x^2}$, we conclude that the
stencil is of order $O(h^2)$. \\

We can use this method for other stencils and in higher
dimensions. \\

\newpage

\subsection*{Finite Element Method}

While the finite difference method is easier, the finite
element method is more accurate, and allows us to more
easily perform diffusion (and other functions) on meshes. \\

Suppose we start with heat diffusion (with a source term):
\[ \partial_t u = \Delta u + f \]
We multiply the equation by a test function $v$:
\[ v \partial_t u = v\Delta u + vf \]
And then we integrate:
\[ \int_\Omega v \partial_t u = 
\int_\Omega v\Delta u +  \int_\Omega vf \]
We then use the product rule and teh divergence
theorem to split the second distribute (laplacian)
between $v$ and $u$. \\
The product rule is an extension of the one dimensional
product rule:
\[ (uv)' = u'v + v'u \]
\[ \nabla \cdot (uv) = (\nabla u) v + u (\nabla v) \]
So we may write:
\[ \nabla \cdot (v \nabla u) = \nabla u \cdot \nabla v + 
v \nabla \cdot \nabla u = 
\nabla u \cdot \nabla v + v \Delta u \]
So we have:
\[ \int_\Omega v \partial_t u = 
\int_\Omega \nabla \cdot (v \nabla u)
- \int_\Omega \nabla u \cdot \nabla v + \int_\Omega vf \]
We know that by the divergence theorem:
\[ \int_\Omega \nabla \cdot F = 
\int_{\partial \Omega} F \cdot n \]
So, we have:
\[ \int_\Omega v \partial_t u = 
\int_\Omega (v \nabla u) \cdot n
- \int_\Omega \nabla u \cdot \nabla v + \int_\Omega vf \]
\[ \int_\Omega v \partial_t u = 
\int_\Omega v \partial_n u
- \int_\Omega \nabla u \cdot \nabla v + \int_\Omega vf \]
Where $\partial_\Omega$ is the border of $\Omega$. \\

Now suppose we have Neumann bouundary condition (reflecting),
which is expressed as:
\[ \partial_n u = g \]
Which applies at the border $\partial_\Omega$
with normal $n$. \\
Then we could write:
\[ \int_\Omega v \partial_t u = 
\int_\Omega vg
- \int_\Omega \nabla u \cdot \nabla v + \int_\Omega vf \]
Which is where we can stop. \\

Now we replace $u$ and $v$ by finite dimensional
approximations:
\[ v \approx v_h = \sum_{i = 1}^m v^i_h \psi_i(x) \]
\[ u \approx u_h = \sum_{j = 1}^n u^j_h \phi_j(x) \]
So we get:
\multiline{
\int_\Omega \sum_{i = 1}^m v^i_h \psi_i(x) 
\partial_t \sum_{j = 1}^n u^j_h \phi_j(x) &= 
\int_\Omega \sum_{i = 1}^m v^i_h \psi_i(x)g \\
- \int_\Omega \nabla \sum_{j = 1}^m v^j_h \phi_j(x) 
&\cdot \nabla \sum_{i = 1}^n u^i_h \psi_i(x) + 
\int_\Omega \sum_{i = 1}^m v^i_h \psi_i(x)f }
\multiline{
\sum_{i = 1}^m \sum_{j = 1}^n v^j_h \partial_t u^j_h
\int_\Omega \psi_i(x) \phi_j(x) &= 
\sum_{i = 1}^m v^i_h \int_\Omega \psi_i(x)g \\
- \sum_{j = 1}^m \sum_{i = 1}^n v^j_h u^i_h 
\int_\Omega \nabla \phi_j(x) &\cdot \nabla \psi_i(x) + 
\sum_{i = 1}^m v^i_h \int_\Omega \psi_i(x)f }
Now, note that this expression is linear with $v_h$,
so it holds for any $v_h$, which means we can remove
the $v_h$ terms. \\
So for each $v_h^i$, where $1 \leq i \leq m$:
\[ \sum_{j = 1}^n \partial_t u^j_h
\int_\Omega \psi_i(x) \phi_j(x) = 
\int_\Omega \psi_i(x)g - \sum_{i = 1}^n u^i_h 
\int_\Omega \nabla \phi_j(x) \cdot \nabla \psi_i(x) + 
\int_\Omega \psi_i(x)f \]
Which gives us the weight matrix $W$, 
the mass matrix $M$,
and the $b$ term vector:
\[ \sum_{j = 1}^n \partial_t u^j_h
\underbrace{\int_\Omega \psi_i(x) \phi_j(x)}_{M_{ij}} = 
\underbrace{\int_\Omega 
\psi_i(x)g}_{b_{\partial \Omega, i}} 
- \sum_{i = 1}^n u^i_h 
\underbrace{\int_\Omega \nabla \phi_j(x) \cdot 
\nabla \psi_i(x)}_{W_{ij}} + 
\underbrace{\int_\Omega \psi_i(x)f}_{b_{\Omega, i}} \]
We can rewrite this using matrix notation. \\

Rewriting it using matrix notation, and grouping
the $b$ term into a single vector, we get:
\[ M d_tu_h(t) = -Wu_h(t) + b(t) \]
Which can be solved for a continuous solution. \\

The discrete solution uses forward or backwards Euler:
\[ u_h^{k+1} = u_h^k + \tau \, 
M^{-1} \para{ - W u_h^k + b^k} \]
\[ u_h^{k+1} = u_h^k + \tau \, M^{-1} 
\para{- W u_h^{k+1} + b^{k+1}} \]
As before, Explicit Euler is conditioanlly stable,
while Implicit Euler is Unconditionally stable,
but requires some solver. \\
Whether we can use conjugant gradient or some more
general solver depends on whether the weight matrix
is symmetric or not. \\

\end{document}
