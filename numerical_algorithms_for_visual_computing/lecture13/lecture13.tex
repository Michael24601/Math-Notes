


\documentclass[12pt]{article}

\usepackage[margin=1in]{geometry}


%===============================================================================
%================================== PACKAGES ===================================
%===============================================================================

% For using float option H that places figures 
% exatcly where we want them
\usepackage{float}
% makes figure font bold
\usepackage{caption}
\captionsetup[figure]{labelfont=bf}
% For text generation
\usepackage{lipsum}
% For drawing
\usepackage{tikz}
% For smaller or equal sign and not divide sign
\usepackage{amssymb}
% For the diagonal fraction
\usepackage{xfrac}
% For enumerating exercise parts with letters instead of numbers
\usepackage{enumitem}
% For dfrac, which forces the fraction to be in display mode (large) e
% even in math mode (small)
\usepackage{amsmath}
% For degree sign
\usepackage{gensymb}
% For "\mathbb" macro
\usepackage{amsfonts}
% For positioning 
\usepackage{indentfirst}
\usetikzlibrary{shapes,positioning,fit,calc}
% for adjustwidth environment
\usepackage{changepage}
% for arrow on top
\usepackage{esvect}
% for mathbb 1
\usepackage{bbm}
% for mathsrc
\usepackage[mathscr]{eucal}
% For degree sign
\usepackage{gensymb}
% For quotes
\usepackage{csquotes}
% For vertical lines
\usepackage{mathtools}
% For cols
\usepackage{multicol}

% for tikz
\usepackage{pgfplots}
\pgfplotsset{compat=1.18}
\usepackage{amsmath}
\usepgfplotslibrary{groupplots}


%===============================================================================
%==================================== FONTS ====================================
%===============================================================================


% Mathcal
\newcommand{\acal}{\mathcal{A}}
\newcommand{\bcal}{\mathcal{B}}
\newcommand{\ccal}{\mathcal{C}}
\newcommand{\dcal}{\mathcal{D}}
\newcommand{\ecal}{\mathcal{E}}
\newcommand{\fcal}{\mathcal{F}}
\newcommand{\gcal}{\mathcal{G}}
\newcommand{\hcal}{\mathcal{H}}
\newcommand{\ical}{\mathcal{I}}
\newcommand{\jcal}{\mathcal{J}}
\newcommand{\kcal}{\mathcal{K}}
\newcommand{\lcal}{\mathcal{L}}
\newcommand{\mcal}{\mathcal{M}}
\newcommand{\ncal}{\mathcal{N}}
\newcommand{\ocal}{\mathcal{O}}
\newcommand{\pcal}{\mathcal{P}}
\newcommand{\qcal}{\mathcal{Q}}
\newcommand{\rcal}{\mathcal{R}}
\newcommand{\scal}{\mathcal{S}}
\newcommand{\tcal}{\mathcal{T}}
\newcommand{\ucal}{\mathcal{U}}
\newcommand{\vcal}{\mathcal{V}}
\newcommand{\wcal}{\mathcal{W}}
\newcommand{\xcal}{\mathcal{X}}
\newcommand{\ycal}{\mathcal{Y}}
\newcommand{\zcal}{\mathcal{Z}}

% Mathfrak
\newcommand{\afrak}{\mathfrak{A}}
\newcommand{\bfrak}{\mathfrak{B}}
\newcommand{\cfrak}{\mathfrak{C}}
\newcommand{\dfrak}{\mathfrak{D}}
\newcommand{\efrak}{\mathfrak{E}}
\newcommand{\ffrak}{\mathfrak{F}}
\newcommand{\gfrak}{\mathfrak{G}}
\newcommand{\hfrak}{\mathfrak{H}}
\newcommand{\ifrak}{\mathfrak{I}}
\newcommand{\jfrak}{\mathfrak{J}}
\newcommand{\kfrak}{\mathfrak{K}}
\newcommand{\lfrak}{\mathfrak{L}}
\newcommand{\mfrak}{\mathfrak{M}}
\newcommand{\nfrak}{\mathfrak{N}}
\newcommand{\ofrak}{\mathfrak{O}}
\newcommand{\pfrak}{\mathfrak{P}}
\newcommand{\qfrak}{\mathfrak{Q}}
\newcommand{\rfrak}{\mathfrak{R}}
\newcommand{\sfrak}{\mathfrak{S}}
\newcommand{\tfrak}{\mathfrak{T}}
\newcommand{\ufrak}{\mathfrak{U}}
\newcommand{\vfrak}{\mathfrak{V}}
\newcommand{\wfrak}{\mathfrak{W}}
\newcommand{\xfrak}{\mathfrak{X}}
\newcommand{\yfrak}{\mathfrak{Y}}
\newcommand{\zfrak}{\mathfrak{Z}}

% Mathscr
\newcommand{\ascr}{\mathscr{A}}
\newcommand{\bscr}{\mathscr{B}}
\newcommand{\cscr}{\mathscr{C}}
\newcommand{\dscr}{\mathscr{D}}
\newcommand{\escr}{\mathscr{E}}
\newcommand{\fscr}{\mathscr{F}}
\newcommand{\gscr}{\mathscr{G}}
\newcommand{\hscr}{\mathscr{H}}
\newcommand{\iscr}{\mathscr{I}}
\newcommand{\jscr}{\mathscr{J}}
\newcommand{\kscr}{\mathscr{K}}
\newcommand{\lscr}{\mathscr{L}}
\newcommand{\mscr}{\mathscr{M}}
\newcommand{\nscr}{\mathscr{N}}
\newcommand{\oscr}{\mathscr{O}}
\newcommand{\pscr}{\mathscr{P}}
\newcommand{\qscr}{\mathscr{Q}}
\newcommand{\rscr}{\mathscr{R}}
\newcommand{\sscr}{\mathscr{S}}
\newcommand{\tscr}{\mathscr{T}}
\newcommand{\uscr}{\mathscr{U}}
\newcommand{\vscr}{\mathscr{V}}
\newcommand{\wscr}{\mathscr{W}}
\newcommand{\xscr}{\mathscr{X}}
\newcommand{\yscr}{\mathscr{Y}}
\newcommand{\zscr}{\mathscr{Z}}

% Mathbb
\newcommand{\abb}{\mathbb{A}}
\newcommand{\bbb}{\mathbb{B}}
\newcommand{\cbb}{\mathbb{C}}
\newcommand{\dbb}{\mathbb{D}}
\newcommand{\ebb}{\mathbb{E}}
\newcommand{\fbb}{\mathbb{F}}
\newcommand{\gbb}{\mathbb{G}}
\newcommand{\hbb}{\mathbb{H}}
\newcommand{\ibb}{\mathbb{I}}
\newcommand{\jbb}{\mathbb{J}}
\newcommand{\kbb}{\mathbb{K}}
\newcommand{\lbb}{\mathbb{L}}
\newcommand{\mbb}{\mathbb{M}}
\newcommand{\nbb}{\mathbb{N}}
\newcommand{\obb}{\mathbb{O}}
\newcommand{\pbb}{\mathbb{P}}
\newcommand{\qbb}{\mathbb{Q}}
\newcommand{\rbb}{\mathbb{R}}
\newcommand{\sbb}{\mathbb{S}}
\newcommand{\tbb}{\mathbb{T}}
\newcommand{\ubb}{\mathbb{U}}
\newcommand{\vbb}{\mathbb{V}}
\newcommand{\wbb}{\mathbb{W}}
\newcommand{\xbb}{\mathbb{X}}
\newcommand{\ybb}{\mathbb{Y}}
\newcommand{\zbb}{\mathbb{Z}}


%===============================================================================
%=============================== SPECIAL SYMBOLS ===============================
%===============================================================================


% Orbit (group theory)
\newcommand{\orbit}{\mathcal{O}}
% Normal group
\newcommand{\normal}{\mathcal{N}}
% Indicator function
\newcommand{\indicator}{\mathbbm{1}}
% Laplace transform
\newcommand{\laplace}[1]{\mathcal{L}}
% Epsilon shorthand
\newcommand{\eps}{\varepsilon}
% Omega
\newcommand{\om}{\omega}
\newcommand{\Om}{\Omega}

%===============================================================================
%================================== OPERATORS ==================================
%===============================================================================


% Inverse exponent
\newcommand{\inv}[0]{^{-1}}
% Overline bar
\newcommand{\olsi}[1]{\,\overline{\!{#1}}}
% Less than or equal slanted
\newcommand{\seqs}{\leqslant}
% Greater or equal slanted
\newcommand{\geqs}{\geqslant}
% Subset or equal
\newcommand{\sub}{\subseteq}
% Proper subset
\newcommand{\prosub}{\subset}
% from
\newcommand{\from}{\leftarrow}

% Parantheses
\newcommand{\para}[1]{\left( #1 \right)}
% Curly Braces
\newcommand{\curl}[1]{\left\{ #1 \right\}}
% Brackets
\newcommand{\brac}[1]{\left[ #1 \right]}
% Angled Brackets
\newcommand{\ang}[1]{\left\langle #1 \right\rangle}
% Norm
\newcommand{\norm}[1]{\left\| #1 \right\|}

% Piece wise (use \\ between cases)
\newcommand{\piecewise}[1]{\begin{cases} #1 \end{cases}}

% Bold symbol shorthand
\newcommand{\bl}{\boldsymbol}

% Vertical space
\newcommand{\vs}[1]{\vspace{#1 pt}}
% Horizontal ertical space
\newcommand{\hs}[1]{\hspace{#1 pt}}


%===============================================================================
%============================== TEXT BASED SYMBOLS =============================
%===============================================================================


% Radians
\newcommand{\rad}{\text{rad}}
% Least Common Multiple
\newcommand{\lcm}{\text{lcm}}
% Automorphism
\newcommand{\Aut}{\text{Aut}}
% Variance
\newcommand{\var}{\text{Var}}
% Covariance
\newcommand{\cov}{\text{Cov}}
% Cofactor (matrix)
\newcommand{\cof}{\text{Cof}}
% Adjugate (matrix)
\newcommand{\adj}{\text{Adj}}
% Trace (matrix)
\newcommand{\tr}{\text{tr}}
% Standard deviation
\newcommand{\std}{\text{Std}}
% Correlation coefficient
\newcommand{\corr}{\text{Corr}}
% Sign
\newcommand{\sign}{\text{sign}}

% And text
\newcommand{\AND}{\text{ and }}
% Or text 
\newcommand{\OR}{\text{ or }}
% If text
\newcommand{\IF}{\text{ if }}
% When text
\newcommand{\WHEN}{\text{ when }}
% Then text
\newcommand{\THEN}{\text{ then }}

% 1st
\newcommand{\st}[1]{#1^{\text{st}}}
% 2nd
\newcommand{\nd}[1]{#1^{\text{nd}}}
% 3rd
\newcommand{\rd}[1]{#1^{\text{rd}}}
% nth
\newcommand{\nth}[1]{#1^{\text{th}}}


%===============================================================================
%========================= PROBABILITY AND STATISTICS ==========================
%===============================================================================


% Permutation
\newcommand{\perm}[2]{{}^{#1}\!P_{#2}}
% Combination
\newcommand{\comb}[2]{{}^{#1}C_{#2}}

% Baye's risk
\newcommand{\risk}[1]{\mathscr{R}_{#1}}
% Baye's optimal risk
\newcommand{\riskOptimal}[1]{\mathscr{R}_{#1}^*}
% Baye's empirical risk
\newcommand{\riskEmpirical}[2]{\hat{\mathscr{R}}_{#1}^{#2}}


%===============================================================================
%=================================== CALCULUS ==================================
%===============================================================================


% d over d derivative
\newcommand{\dd}[2]{\dfrac{d#1}{d#2}}
% partial d over d derivative
\newcommand{\partialdd}[2]{\dfrac{\partial #1}{\partial #2}}
% delta d over d derivative
\newcommand{\deltadd}[2]{\dfrac{\Delta #1}{\Delta #2}}

% Integration between a and b
\newcommand{\integral}[4]{\int_{#1}^{#2} #3 \, #4}
% Integration in some space 
\newcommand{\boundIntegral}[2]{\int_{#1} #2 \, d#1}

% Limit
\newcommand{\limit}[3]{\lim_{#1 \to #2} #3}


%===============================================================================
%================================  BIG SYMBOLS  ================================
%===============================================================================


% Sum
\newcommand{\sumof}[2]{\sum_{#1}^{#2}}
% Product
\newcommand{\productof}[2]{\prod_{#1}^{#2}}
% Union
\newcommand{\unionof}[2]{\bigcup_{#1}^{#2}}
% Intersection
\newcommand{\intersectionof}[2]{\bigcap_{#1}^{#2}}
% Or
\newcommand{\orof}[2]{\bigvee_{#1}^{#2}}
% And
\newcommand{\andof}[2]{\bigwedge_{#1}^{#2}}


%===============================================================================
%=============================== LINEAR ALGEBRA ================================
%===============================================================================


% Bold vector arrow
\newcommand{\bv}[1]{\vec{\mathbf{#1}}}

% Matrix or vector (use // for column, & for row) 
% with brackets
\newcommand{\bmat}[1]{\begin{bmatrix} #1 \end{bmatrix}}
% Matrix or vector (use // for column, & for row) 
% with curved brackets
\newcommand{\pmat}[1]{\begin{pmatrix} #1 \end{pmatrix}}
% Matrix or vector (use // for column, & for row) 
% with lines on either side 
\newcommand{\lmat}[1]{\begin{vmatrix} #1 \end{vmatrix}}
% Matrix or vector (use // for column, & for row) 
% with curly braces
\newcommand{\cmat}[1]{\begin{Bmatrix} #1 \end{Bmatrix}}
% Matrix or vector (use // for column, & for row) 
% with no braces
\newcommand{\mat}[1]{\begin{matrix} #1 \end{matrix}}


%===============================================================================
%================================ LARGE OBJECTS ================================
%===============================================================================

% Multiple lines
\newcommand{\multiline}[1]{
\begin{align*}
    #1
\end{align*}
}

% Multiple lines with equation numbers
\newcommand{\eqmultiline}[1]{
\begin{align*}
    #1
\end{align*}
}

% Color
\newcommand{\colorText}[2]{
\begingroup
\color{#1}
    #2
\endgroup
}

% Centered figure
\newcommand{\centerFigure}[2]{
    \begin{figure}[h]
        \centering
            #1
        \caption{#2}
    \end{figure}
}

% Tikz figure
\newcommand{\tikzGraphic}[1]{
    \begin{center}
    \begin{tikzpicture}
        #1
    \end{tikzpicture}
    \end{center}
}

% Enumerate numbers (seperate by \item)
\newcommand{\numbers}{\textbf{\number*)}}

% Enumerate letters (seperate by \item)
\newcommand{\letters}{\textbf{\alph*)}}

\title{%
    \Huge Numerical Algorithms \\
    \Large Lecture XIV
}
\date{2025-06-04}
\author{Michael Saba}

\begin{document}
\pagenumbering{gobble}
\maketitle
\newpage
\setlength{\parindent}{0pt}
\pagenumbering{arabic}

\subsection*{Heat diffusion}

Generally particles move from higher to lower concentration
parts; as time goes on, the enture image blurs, as
the heat diffuses in space, and the heat distributes. \\
The color at the steady state
(equilibrium) is a constant, mean, color. \\

Works not just for heat, but for pressure, and for
mass... \\

We can express the heat
in some region $\Om$ at time $t$
as $Q(t, \Om)$, where:
\[ Q(t, \Om) = \int_\Om p(t, x, y, z) 
\; dx \; dy \; dz
= \int_\Om p(t, x, y, z) \; dV \]
Where $p$ is the density at a point. \\

The change in heat in the region $\Om$
with respect to the time is given by:
\[ \partialdd{}{t} \int_\Om p(t, x, y, z) \; dV \]
We can bring in the differntial
since $t$ and $V$ are not related:
\[ \int_\Om \partialdd{}{t} p(t, x, y, z) \; dV \]
It can be decomposed into the generated
heat in the region (either a sink or source
of heat depending on sign) minus the heat
leaving the region $\Om$,
which means the heat leaving in the direction
of the normal at each point
on the boundary of the region $\partial \Om$:
\[ \int_\Om  \partialdd{}{t}p(t, x, y, z) \; dV
= \int_\Om F \; dV
- \int_{\partial \Om} j \cdot n \; dA \]
Where the heat leaving the region is
given by a flux integral (dot product
of vector field with normal). \\

The divergence is just the gradient
accompanied by a dot product:
\[ \text{Div}(f(x, y, z))
= \nabla \cdot f(x, y, z)
= \partial_x f_x + \partial_y f_y
+ \partial_z f_z \]
We can think of it as a measure of how much
a vector field is a sink or source at 
some point. \\

The divergence theorem
says that the flux around the boundary
of some region $\partial \Om$ 
is equal to the sum of the divergence at each 
individual point in the region $\Om$:
\[ \int_{\partial \Om} j \cdot n \; dA
= \int_{\Om} \nabla \cdot j \; dV\]
Recall that the flux is the measure of
the vector field exitting the region
at the bounds (sum of dot product with normals),
so what we are saying is that the amount
of the vector field exitting a region
is equal to the amount of divergence
(how much of a source each point is)
at each point in the region. \\

It is simialr to Green's Theorem,
which is about circulation and curl;
both are generalized by Stoke's Theorem. \\

So using the divergence theorem, we get:
\[ \int_\Om \partialdd{}{t} p(t, x, y, z) \; dV
= \int_\Om F \; dV
- \int_{\Om} \nabla \cdot j \; dV \]
We can take out the integral since it
is over the same region, which we assume is
continuous:
\[  \partialdd{}{t} p(t, x, y, z)
= F - \nabla \cdot j \]
Which is our change of heat in terms of $t$. \\

Here we have $j$ as the vector
field of change of density $p$:
\[ j = -D\nabla p \]
Here, by Fick's Law, the matrix $D$
decides how much change we have in each
direction, and $\nabla p$ is the gradient
(rate of change in each direction). \\
However, for simplicity,
we can assume we have constant change
in each direction, which means that
$D = kI$, so we can write:
\[ j = -k\nabla p \]
There are other diffusions we can have,
but this is the simplest case. \\

This kind of diffusion, where assume that
$D = kI$, is called an isotropic homoegenous
diffusion. It does not vary over space or direction. \\
On the other hand, if we had $k(x, y)$
be a function of space, $D$ would vary in space,
and would not be homoegenous. \\
Moroever, if $D$ were a constant matrix,
but not of the form $kI$, it would be homogeneous,
but anisotropic (varies over direction). \\

Now, replacing in the original formula
using the formula for $j$ we found, we get:
\[  \partialdd{}{t} p(t, x, y, z)
= F - \nabla \cdot (-k\nabla p) \]
\[  \partialdd{}{t} p(t, x, y, z)
= F + k \nabla \cdot \nabla p \]
\[  \partialdd{}{t} p(t, x, y, z)
= F + k \Delta p \]
Where $\Delta = \nabla^2$ which is
the laplacian. \\

The laplacian is the dot product of
the gradient with itself:
\[ \Delta = \para{\partialdd{}{x}, 
\partialdd{}{y}, \partialdd{}{z}}
\cdot \para{\partialdd{}{x}, 
\partialdd{}{y}, \partialdd{}{z}} \]
\[ \Delta = \partialdd{^2}{x^2} +
\partialdd{^2}{y^2} + \partialdd{^2}{z^2} \]
Which uses second partial derivatives, so:
\[ \Delta f = \partial_{xx} f + 
\partial_{yy} f + \partial_{zz} f \]
This is the laplacian of some function $f$. \\

Now, in our equation,
we can assume there is no generated
heat, which gets us:
\[ \partialdd{}{t} p(t, x, y, z) = k \Delta p \]
This is our partial differential equation,
called the heat equation. \\

Everything we wrote in this section was
in $3$ dimensions, but can be generalized
easily to any dimension. \\ 

\newpage

\subsection*{Discretizing the Heat Equation}

We have the heat equation PDE:
\[ \partialdd{}{t} u = k \Delta u \]
This is sometimes solvable. \\

However, in general we want to discretize it
the PDE in order to solve it. \\

We will start with a one dimensional example
in order to keep it simple.
So we have a one dimensional function $u(x)$,
which gives us the PDE:
\[ \partial_t u = k \partial_{xx} u\]
We can ignore the constant $k$
now which only modulates the speed of 
the diffusion. \\

So how can we solve $\partial_t u =\partial_{xx} u$?
We will discretize $u(x)$. \\
we can imagine $u(x)$ as a line of points,
each with a value $u(x_i)$;
we can imagine this as a bunch of discrete,
non continuous points
on the line $x_1, x_2, \dots x_n$. \\
We then end up with a vector of values:
\[ (u(x_1), u(x_2), \dots u(x_n)) \]
Which we can imagine as an vector of pixels
with values $u(x_i)$ at each entry:
\[
    \begin{tabular}{|c|c|c|c|c|}
    \hline
    $u(x_1)$ & $u(x_2)$ & $u(x_3)$ & 
    $\dots$ & $u(x_n)$ \\
    \hline
    \end{tabular}
\]
If this were the 2 dimensional case, 
we would have had a 2 dimensional vector
represented by our image pixels. \\

We will use $u_i$ interchangebly
with $u(x_i)$ from here on out. \\

We now have to express $\partial_{xx} u$
using this new discretization of the $u(x)$. \\
But what is the second order rate of change of $u(x_i)$
in terms of $x$? After all this is just a set of
values that are in no way continuous or differentiable. \\

So, we have to interpolate between the points;
using linear interpolation is not especially useful,
since the derivative will just be a single
number that doesn't tell us much,
so we use quadratic interpolation. \\
So for each point $u_i$,
we want to fit the triplet $u_{i-1}$, $u_{i}$,
$u_{i+1}$ with a quadratic. \\

So we want some quadratic function:
\[ ax^2 + bx + c \]
such that:
\multiline{
a(x_{i-1})^2 + bx_{i-1} + c &= u(x_{i-1}) \\
a(x_{i})^2 + bx_{i} + c &= u(x_{i}) \\
a(x_{i+1})^2 + bx_{i+1} + c &= u(x_{i+1}) }
Notice that solving for $a$, $b$, and $c$
requires us to solve the linear system:
\[ \bmat{(x_{i-1})^2 & x_{i-1} & 1 \\
(x_{i})^2 & x_{i} & 1 \\
(x_{i+1})^2 & x_{i+1} & 1 \\} 
\bmat{a \\ b \\ c} = 
\bmat{u_{i-1} \\ u_{i} \\ u_{i+1}}\]
Solving this gets us the quadratic
at $u_i$. \\

We don't actually need to solve for $b$
and $c$ however.
The second derivative $\partial_{xx} u$
of the quadratic at $u_i$ is just $2a$,
so we only need to solve for $a$. \\
We get:
\[ \partial_{xx} u_i = 2a = 
\dfrac{u_{i-1} - 2u_{i} + u_{i+1}}{h^2} \]
Assuming that $u_{i-1}$, $u_i$, and $u_{i+1}$
are all spaced $h$ apart (the dame distance). \\

Note that we can approximate the first
derivative of a point using a linear function
with the point before or after it (since the
first derivative is just the slope, which is):
\[ \dfrac{u_{i+1} - u_{i}}{h}
\quad \OR \quad \dfrac{u_i - u_{i-1}}{h} \]
Here $h$ is the distance between the two points. \\
These will correspond to the explicit and implicit
Euler forms we will see later. \\

Notice that the coefficients for the first
derivative are $1$ and $-1$,
then for the second, are $1$, $-2$, and $1$. \\
These will follow the $\nth{n}$ row of pascal's triangle,
with the sign alternating:
\[ (1^{k}) \cdot \comb{n}{k}\]
Formula using combinations. \\

So we want to calculate this second order partial
derivative for each $u_i$;
to make things easier, we can use whats called
a stencil, which we can slide across the
one dimensional vector from earlier,
multiplying the entries with its coefficients:
\[ \dfrac{1}{h^2}\bmat{1 & -2 & 1} \]
Of course, one might naturally ask what
happens when we get to the edges;
we don't have elements $u_{0}$ and $u_{n+1}$. \\

In order to solve this issue, we will have to pad the
edges with entries $u_{0}$ and $u_{n+1}$,
which are only used for the calculation
of the second derivative at $u_1$ and $u_{n}$. \\
There are several ways to do this:
\begin{enumerate}
    \item Absorbing boundaries:
    If we were to set $u_0 = 0$
    and $u_{n+1} = 0$,
    the heat would leak out from the edges.
    At equilibrium, the heat in our scene
    would just be $0$.
    \item Reflecting boundaries:
    We set $u_0 = u_1$ and $u_{n+1} = u_{n}$.
    This prevents the heat from leaking out;
    we can imagine a boundary that prevents the heat
    from leaving the scene.
    \item Periodic boundaries:
    We set $u_0 = u_n$ and $u_{n+1} = u_{1}$.
    We can envision this as simply tying the ends
    of the vector together; heat wouldn't leak out
    from the scene, but it will diffuse from the
    left to the right side and vice-versa.
\end{enumerate}
We generally prefer using reflecting boundaries:
\[
    \begin{tabular}{|c|c|c|c|c|c|c|}
    \hline
    $u_1$ & $u_1$ & $u_2$ & $u_3$ & 
    $\dots$ & $u_n$ & $u_n$ \\
    \hline
    \end{tabular}
\]
We then slide the stencil accross to
get the second derivative at each point
from $u_1$ to $u_n$, not including the padding. \\

The vector we had was a discretization of $u$,
so the discretization of $\partial_{xx} u$
would have to be:
\[
    \begin{tabular}{|c|c|c|c|c|}
    \hline
    $\partial_{xx} u_1$ & $\partial_{xx} u_2$ 
    & $\partial_{xx} u_3$ & 
    $\dots$ & $\partial_{xx} u_n$ \\
    \hline
    \end{tabular}
\]
Where we already calculated each term using
quadratic interpolation. \\

The discretization of $\partial_{xx} u$
is the vector:
\[ \bmat{
    \partial_{xx} u_1 = \dfrac{u_1 - 2u_1 + u_2}{h^2} 
    = \dfrac{-u_1 + u_2}{h^2} \vs{10} \\
    \partial_{xx} u_2 = \dfrac{u_1 - 2u_2 + u_3}{h^2} \\
    \vdots \\
    \partial_{xx} u_n = \dfrac{u_{n-1} - 2u_n + u_n}{h^2}
    = \dfrac{u_{n-1} -u_n}{h^2}}
\]
Which we can write as a matrix multiplication
of the vector of points $u_i$:
\[ \dfrac{1}{h^2}
\bmat{-1 & 1 & 0 & 0 & \dots & 0 \\
1 & -2 & 1 & 0 & \dots & 0 \\
0 & 1 & -2 & 1 & \dots & 0 \\
\vdots & \ddots & \ddots & \ddots & \ddots & \vdots \\
0 & 0 & \dots & 0 & 1 & -1 }
\bmat{u_1 \\ u_2 \\ u_3 \\ \vdots \\ u_n}
= \bmat{\partial_{xx} u_1 \\ \partial_{xx} u_2 \\
\partial_{xx} u_3 \\ \vdots \\ \partial_{xx} u_n} \]
This matrix multiplication encodes the entire
second order derivation process. \\

So, we have:
\[ \partial_{xx} u(x) \approx Au \]
Where $A$ is the matrix from before,
and $u$ is the discretization (vector) of $u(x)$:
\[ A = \dfrac{1}{h^2}
\bmat{-1 & 1 & 0 & 0 & \dots & 0 \\
1 & -2 & 1 & 0 & \dots & 0 \\
0 & 1 & -2 & 1 & \dots & 0 \\
\vdots & \ddots & \ddots & \ddots & \ddots & \vdots \\
0 & 0 & \dots & 0 & 1 & -1 }
\quad \AND \quad u = 
\bmat{u_1 \\ u_2 \\ u_3 \\ \vdots \\ u_n} \]
This is much simpler to solve now. \\

We can now replace $\partial_{xx} u(x)$
in the original heat equation, which gets us:
\[\partial_t u = k \partial_{xx} u \]
\[ \partial_t u = kAu \]
This is just an ordinary differential equation
we can solve for $u$:
\[ u = e^{kAt}u(0) \]
Where $u(0)$ is the starting condition. \\

Note that solving this ODE requires
using the diagonalization of $A$, so it must
be diagonalizable, which it is. \\

\newpage

\subsection*{Discretizing in Higher Dimensions}

In this section, we will show how to
discretize the heat equation:
\[ \partialdd{}{t} u = k \Delta u \]
in higher dimensions than $1$. \\

We will do it in $2$ dimensions to show
what changes and what stays the same, and this
can be used as a template to generalize the
discretization to other dimensions. \\

In $2$ dimensions, we have $u(x, y)$,
which means that the laplacian term in the
heat equation evaluates to:
\[ \partial_t u = 
k (\partial_{xx}u + \partial_{yy}u) \]
We will show how to calculate 
$\partial_{xx}u + \partial_{yy}u$. \\

First note that when discretizing the space,
we end up with a $2$ dimensional array of entries:
\[ \begin{tabular}{|c|c|c|c|c|}
\hline
$u_{11}$ & $u_{12}$ & $u_{13}$ & $\cdots$ & $u_{1n}$ \\
\hline
$u_{21}$ & $u_{22}$ & $u_{23}$ & $\cdots$ & $u_{2n}$ \\
\hline
$\vdots$ & $\vdots$ & $\vdots$ & $\ddots$ & $\vdots$ \\
\hline
$u_{m1}$ & $u_{m2}$ & $u_{m3}$ & $\cdots$ & $u_{mn}$ \\
\hline \end{tabular} \]
We can treat the discretization
of the $2$D space as just several $1$D
discretizations of $x$,
and several $1$D discretizations of $y$,
which are the rows and columns respecively. \\

We can then use the same stencil as before
to calculate $\partial_{xx} u$,
by just sliding the stencil across each row
individually:
\[ \dfrac{1}{(h_x)^2}\bmat{1 & -2 & 1} \]
We can then do the same for $\partial_{yy} u$,
sliding the stencil:
\[ \dfrac{1}{(h_y)^2}\bmat{1 & -2 & 1} \]
across the columns. \\

So for each point $u_{i, j}$, we have:
\[ \partial_{xx}u_i + \partial_{yy}u_i
= \dfrac{u_{i, j-1} -2u_{i, j} + u_{i, j+1}}{(h_x)^2} 
+ \dfrac{u_{i+1, j} -2u_{i, j} + u_{i+1, j}}{(h_y)^2} \]
If we assume that the horizontal intervals between 
the points are equal to the vertical intervals,
that is, that $h_x = h_y$,
then for each point $u_i$, we can write:
\[ \partial_{xx}u_i + \partial_{yy}u_i
= \dfrac{u_{i, j-1} + u_{i-1, j} -4u_{i, j} 
+ u_{i, j+1} + u_{i+1, j}}{h^2} \]
Which induces a stencil that we can slide
across the entire $2$ dimensional space
in order to calculate both $\partial_{xx}$
and $\partial_{yy}$ for each point $u_{ij}$:
\[ \dfrac{1}{h^2}
\bmat{0 & 1 & 0 \\ 1 & -4 & 1 \\ 0 & 1 & 0} \]
We again just slide this across the $2$D array. \\

Of course, we run into the same issue as before,
where for the points on the boundaries
of the forms $u_{0, j}$, $u_{i, n+1}$,
$u_{n, j}$, $u_{i, 0}$ are not defined. \\
We can use the same techniques talked about before. \\
In the reflecting boundaries case,
we include a padding, where each elements
$u_{0, j} = u_{1, j}$, $u_{n+1, j} = u_{n, j}$,
$u_{i, 0} = u_{i, 1}$, and $u_{i, n+1} = u_{i, n}$. \\
In the periodic boundaries case,
we can imagine gluing two sides together,
yielding a tube, and then tyring the two ends together,
which gives us a torus. What this means is that
we will have:
$u_{0, j} = u_{n+1, j}$, $u_{n+1, j} = u_{0, j}$,
$u_{i, 0} = u_{i, n+1}$, and $u_{i, n+1} = u_{i, 0}$. \\

We again go with the repeating boundaries
case since it won't leak from one side to another:
\[ \begin{tabular}{|c|c|c|c|c|c|c|}
\hline
& $u_{11}$ & $u_{12}$ & $u_{13}$ & $\cdots$ 
& $u_{1n}$ & \\ \hline
$u_{11}$& $u_{11}$ & $u_{12}$ & $u_{13}$ & 
$\cdots$ & $u_{1n}$ & $u_{1n}$ \\ \hline
$u_{21}$& $u_{21}$ & $u_{22}$ & $u_{23}$ & 
$\cdots$ & $u_{2n}$ & $u_{2n}$ \\ \hline
$\vdots$ & $\vdots$ & $\vdots$ & $\vdots$ 
& $\ddots$ & $\vdots$ & $\vdots$ \\ \hline
$u_{m1}$& $u_{m1}$ & $u_{m2}$ & $u_{m3}$ & 
$\cdots$ & $u_{mn}$ & $u_{mn}$ \\ \hline
& $u_{m1}$ & $u_{m2}$ & $u_{m3}$ & $\cdots$ 
& $u_{mn}$ & \\
\hline \end{tabular} \]
This is the table we get with padding on all sides. \\

Now, the discretization of $u$, which is a $2$D
grid, can now be flattened out into a vector,
which induces a discretization of
$\partial_{xx} u + \partial_{yy} u$
which is also a vector:
\[ \bmat{
    \partial_{xx} u_{11} + \partial_{yy} u_{11} 
    = \dfrac{u_{11} + u_{11} - 4u_{11} + u_{12} 
    + u_{21}}{h^2} = 
    \dfrac{-2u_{11} + u_{12} + u_{21}}{h^2}  \vs{10} \\
    \partial_{xx} u_{12} + \partial_{yy} u_{12} 
    = \dfrac{u_{11} + u_{12} - 4u_{12} + u_{13} 
    + u_{22}}{h^2} = 
    \dfrac{u_{11} - 3u_{12} + u_{13} + u_{22}}{h^2} \\
    \vdots \\
    \partial_{xx} u_{mn} + \partial_{yy} u_{mn} 
    = \dfrac{- 2u_{mn} + u_{mn-1} + u_{m-1n}}{h^2}}
\]
Which we can write as a matrix multiplication
of the vector of points $u_ij$:
\[ \dfrac{1}{h^2}
\bmat{-2 & 1 & 0 & \dots & 1 & 0 & 0 & \dots & 0 \\
1 & -3 & 1 & 0 & \dots & 1 & 0 & \dots & 0 \\
0 & 1 & -3 & 1 & 0 & \dots & 1 & \dots & 0 \\
\vdots & \ddots & \ddots & \ddots & \ddots 
& \ddots & \ddots & \ddots & \vdots \\
0 & 0 & \dots & 1 & 0 & \dots & 0 & 1 & -2 }
\bmat{u_{11} \\ u_{12} \\ u_{13} \\ \vdots \\ u_{mn}}
= \bmat{\partial_{xx} u_{11} \\ \partial_{xx} u_{12} \\
\partial_{xx} u_{13} \\ \vdots \\ \partial_{xx} u_{mn}} \]
This matrix multiplication encodes the entire
second order derivation process. \\

And so we end up with:
\[ \partial_{xx} u(x, y) + \partial_{yy} u(x, y) = Au \]
Where $A$ and $u$
are the matrix and vector from before. \\
We can then replace this in the original heat
equation, to get:
\[\partial_t u = k (\partial_{xx} u + \partial_{yy} u) \]
\[ \partial_t u = kAu \]
Which is again solved by:
\[ u = e^{kAt}u(0) \]
Same as for the one dimensional case, 
which will again be the same for all dimensions. \\

\newpage

\subsection*{Time Discretization}

We will recall that $u$ is the density function,
which is a function of space $x, y, z, \dots$,
as well as time $t$. \\
It is continuous in terms of both. \\

Thus far we have only discretized the space
$x, y, z, \dots$, but not time. \\
The solution we got:
\[ u = e^{kAt}u(0) \]
is still continuous in terms of time. \\

We will take $L = kA$ for simplicity. \\
Notice that the ODE we arrived at last time:
\[ \partial_t u(x, t) = Lu \]
Just tells us that $u(x, t)$ changes in terms of $t$
following $Lu$, which means that we can discretize
the time $t$. \\

To start, we have:
\[ \partialdd{u}{t} = Lu \]
Which we can discretize with small time steps
$\Delta t$:
\[ \deltadd{u}{t} = Lu \]
\[ \dfrac{u^{(k+1)} - u^{(k)}}{\Delta t} = Lu^{(k)} \]
Note that all this is saying is that
$u^{(k)}$ moves following this path:
\[ u^{(k+1)} = u^{(k)} + \Delta t (Lu^{(k)}) \]
Which is clearly true since $Lu^{(k)}$
is the rate of change with respect to $t$ 
at this point. \\
We will use $\tau = \Delta t$ for our time step,
which gives us:
\[ \dfrac{u^{(k+1)} - u^{(k)}}{\tau} = Lu^{(k)} \]
Which is the explicit Euler form. \\

We can also write:
\[ u^{(k+1)} = u^{(k)} + \Delta t (Lu^{(k+1)}) \]
Using the rate of change
of the point itself that we are trying to find
(we basically work backwards).
This yields the discretization:
\[ \dfrac{u^{(k+1)} - u^{(k)}}{\tau} = Lu^{(k+1)} \]
Which is the implicit Euler form. \\

Notice that these are just the first derivative
approximations we showed in one of the last sections,
which makes sense since $\partial_t$ is
such a first order partial derivative. \\

This only shows the one dimensional case,
but the other cases are basically the same. \\

The reason why we discretized space earlier
was in order to solve the PDE. \\
The reason why we discretize time on the other hand,
is in order to update the simulation in practice.
As we know, in a computer simulation, time is 
not conitnuous.
We have discrete frames, with some time step $\tau$
based on the framerate, and the update steps:
\[ u^{(k+1)} = u^{(k)} + \Delta t (Lu^{(k)}) \]
\[ u^{(k+1)} = u^{(k)} + \Delta t (Lu^{(k+1)}) \]
Simply tell us how to update $u$ the scene each frame. \\
This is much simpler than solving the ODE
and using the exponential,
which we no longer need. \\

Note that the time step needs to be small enough
in order for the simulation to be stable, otherwise
the numerical inaccuracies build up and cause
the scene to explode. \\
For one dimension, \[ \tau \in \para{0, 
\dfrac{h^2}{2}}\]
is the stabiliy condition. \\
For two dimensions, \[\tau \in \para{0, 
\dfrac{h^2}{4}}\]
is the stabiliy condition. \\
For $n$ dimensions, \[\tau \in \para{0, 
\dfrac{h^2}{2n}}\]
is the stabiliy condition. \\

\end{document}
