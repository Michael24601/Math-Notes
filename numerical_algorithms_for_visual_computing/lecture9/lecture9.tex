

\documentclass[12pt]{article}

\usepackage[margin=1in]{geometry}


%===============================================================================
%================================== PACKAGES ===================================
%===============================================================================

% For using float option H that places figures 
% exatcly where we want them
\usepackage{float}
% makes figure font bold
\usepackage{caption}
\captionsetup[figure]{labelfont=bf}
% For text generation
\usepackage{lipsum}
% For drawing
\usepackage{tikz}
% For smaller or equal sign and not divide sign
\usepackage{amssymb}
% For the diagonal fraction
\usepackage{xfrac}
% For enumerating exercise parts with letters instead of numbers
\usepackage{enumitem}
% For dfrac, which forces the fraction to be in display mode (large) e
% even in math mode (small)
\usepackage{amsmath}
% For degree sign
\usepackage{gensymb}
% For "\mathbb" macro
\usepackage{amsfonts}
% For positioning 
\usepackage{indentfirst}
\usetikzlibrary{shapes,positioning,fit,calc}
% for adjustwidth environment
\usepackage{changepage}
% for arrow on top
\usepackage{esvect}
% for mathbb 1
\usepackage{bbm}
% for mathsrc
\usepackage[mathscr]{eucal}
% For degree sign
\usepackage{gensymb}
% For quotes
\usepackage{csquotes}
% For vertical lines
\usepackage{mathtools}
% For cols
\usepackage{multicol}

% for tikz
\usepackage{pgfplots}
\pgfplotsset{compat=1.18}
\usepackage{amsmath}
\usepgfplotslibrary{groupplots}


%===============================================================================
%==================================== FONTS ====================================
%===============================================================================


% Mathcal
\newcommand{\acal}{\mathcal{A}}
\newcommand{\bcal}{\mathcal{B}}
\newcommand{\ccal}{\mathcal{C}}
\newcommand{\dcal}{\mathcal{D}}
\newcommand{\ecal}{\mathcal{E}}
\newcommand{\fcal}{\mathcal{F}}
\newcommand{\gcal}{\mathcal{G}}
\newcommand{\hcal}{\mathcal{H}}
\newcommand{\ical}{\mathcal{I}}
\newcommand{\jcal}{\mathcal{J}}
\newcommand{\kcal}{\mathcal{K}}
\newcommand{\lcal}{\mathcal{L}}
\newcommand{\mcal}{\mathcal{M}}
\newcommand{\ncal}{\mathcal{N}}
\newcommand{\ocal}{\mathcal{O}}
\newcommand{\pcal}{\mathcal{P}}
\newcommand{\qcal}{\mathcal{Q}}
\newcommand{\rcal}{\mathcal{R}}
\newcommand{\scal}{\mathcal{S}}
\newcommand{\tcal}{\mathcal{T}}
\newcommand{\ucal}{\mathcal{U}}
\newcommand{\vcal}{\mathcal{V}}
\newcommand{\wcal}{\mathcal{W}}
\newcommand{\xcal}{\mathcal{X}}
\newcommand{\ycal}{\mathcal{Y}}
\newcommand{\zcal}{\mathcal{Z}}

% Mathfrak
\newcommand{\afrak}{\mathfrak{A}}
\newcommand{\bfrak}{\mathfrak{B}}
\newcommand{\cfrak}{\mathfrak{C}}
\newcommand{\dfrak}{\mathfrak{D}}
\newcommand{\efrak}{\mathfrak{E}}
\newcommand{\ffrak}{\mathfrak{F}}
\newcommand{\gfrak}{\mathfrak{G}}
\newcommand{\hfrak}{\mathfrak{H}}
\newcommand{\ifrak}{\mathfrak{I}}
\newcommand{\jfrak}{\mathfrak{J}}
\newcommand{\kfrak}{\mathfrak{K}}
\newcommand{\lfrak}{\mathfrak{L}}
\newcommand{\mfrak}{\mathfrak{M}}
\newcommand{\nfrak}{\mathfrak{N}}
\newcommand{\ofrak}{\mathfrak{O}}
\newcommand{\pfrak}{\mathfrak{P}}
\newcommand{\qfrak}{\mathfrak{Q}}
\newcommand{\rfrak}{\mathfrak{R}}
\newcommand{\sfrak}{\mathfrak{S}}
\newcommand{\tfrak}{\mathfrak{T}}
\newcommand{\ufrak}{\mathfrak{U}}
\newcommand{\vfrak}{\mathfrak{V}}
\newcommand{\wfrak}{\mathfrak{W}}
\newcommand{\xfrak}{\mathfrak{X}}
\newcommand{\yfrak}{\mathfrak{Y}}
\newcommand{\zfrak}{\mathfrak{Z}}

% Mathscr
\newcommand{\ascr}{\mathscr{A}}
\newcommand{\bscr}{\mathscr{B}}
\newcommand{\cscr}{\mathscr{C}}
\newcommand{\dscr}{\mathscr{D}}
\newcommand{\escr}{\mathscr{E}}
\newcommand{\fscr}{\mathscr{F}}
\newcommand{\gscr}{\mathscr{G}}
\newcommand{\hscr}{\mathscr{H}}
\newcommand{\iscr}{\mathscr{I}}
\newcommand{\jscr}{\mathscr{J}}
\newcommand{\kscr}{\mathscr{K}}
\newcommand{\lscr}{\mathscr{L}}
\newcommand{\mscr}{\mathscr{M}}
\newcommand{\nscr}{\mathscr{N}}
\newcommand{\oscr}{\mathscr{O}}
\newcommand{\pscr}{\mathscr{P}}
\newcommand{\qscr}{\mathscr{Q}}
\newcommand{\rscr}{\mathscr{R}}
\newcommand{\sscr}{\mathscr{S}}
\newcommand{\tscr}{\mathscr{T}}
\newcommand{\uscr}{\mathscr{U}}
\newcommand{\vscr}{\mathscr{V}}
\newcommand{\wscr}{\mathscr{W}}
\newcommand{\xscr}{\mathscr{X}}
\newcommand{\yscr}{\mathscr{Y}}
\newcommand{\zscr}{\mathscr{Z}}

% Mathbb
\newcommand{\abb}{\mathbb{A}}
\newcommand{\bbb}{\mathbb{B}}
\newcommand{\cbb}{\mathbb{C}}
\newcommand{\dbb}{\mathbb{D}}
\newcommand{\ebb}{\mathbb{E}}
\newcommand{\fbb}{\mathbb{F}}
\newcommand{\gbb}{\mathbb{G}}
\newcommand{\hbb}{\mathbb{H}}
\newcommand{\ibb}{\mathbb{I}}
\newcommand{\jbb}{\mathbb{J}}
\newcommand{\kbb}{\mathbb{K}}
\newcommand{\lbb}{\mathbb{L}}
\newcommand{\mbb}{\mathbb{M}}
\newcommand{\nbb}{\mathbb{N}}
\newcommand{\obb}{\mathbb{O}}
\newcommand{\pbb}{\mathbb{P}}
\newcommand{\qbb}{\mathbb{Q}}
\newcommand{\rbb}{\mathbb{R}}
\newcommand{\sbb}{\mathbb{S}}
\newcommand{\tbb}{\mathbb{T}}
\newcommand{\ubb}{\mathbb{U}}
\newcommand{\vbb}{\mathbb{V}}
\newcommand{\wbb}{\mathbb{W}}
\newcommand{\xbb}{\mathbb{X}}
\newcommand{\ybb}{\mathbb{Y}}
\newcommand{\zbb}{\mathbb{Z}}


%===============================================================================
%=============================== SPECIAL SYMBOLS ===============================
%===============================================================================


% Orbit (group theory)
\newcommand{\orbit}{\mathcal{O}}
% Normal group
\newcommand{\normal}{\mathcal{N}}
% Indicator function
\newcommand{\indicator}{\mathbbm{1}}
% Laplace transform
\newcommand{\laplace}[1]{\mathcal{L}}
% Epsilon shorthand
\newcommand{\eps}{\varepsilon}
% Omega
\newcommand{\om}{\omega}
\newcommand{\Om}{\Omega}

%===============================================================================
%================================== OPERATORS ==================================
%===============================================================================


% Inverse exponent
\newcommand{\inv}[0]{^{-1}}
% Overline bar
\newcommand{\olsi}[1]{\,\overline{\!{#1}}}
% Less than or equal slanted
\newcommand{\seqs}{\leqslant}
% Greater or equal slanted
\newcommand{\geqs}{\geqslant}
% Subset or equal
\newcommand{\sub}{\subseteq}
% Proper subset
\newcommand{\prosub}{\subset}
% from
\newcommand{\from}{\leftarrow}

% Parantheses
\newcommand{\para}[1]{\left( #1 \right)}
% Curly Braces
\newcommand{\curl}[1]{\left\{ #1 \right\}}
% Brackets
\newcommand{\brac}[1]{\left[ #1 \right]}
% Angled Brackets
\newcommand{\ang}[1]{\left\langle #1 \right\rangle}
% Norm
\newcommand{\norm}[1]{\left\| #1 \right\|}

% Piece wise (use \\ between cases)
\newcommand{\piecewise}[1]{\begin{cases} #1 \end{cases}}

% Bold symbol shorthand
\newcommand{\bl}{\boldsymbol}

% Vertical space
\newcommand{\vs}[1]{\vspace{#1 pt}}
% Horizontal ertical space
\newcommand{\hs}[1]{\hspace{#1 pt}}


%===============================================================================
%============================== TEXT BASED SYMBOLS =============================
%===============================================================================


% Radians
\newcommand{\rad}{\text{rad}}
% Least Common Multiple
\newcommand{\lcm}{\text{lcm}}
% Automorphism
\newcommand{\Aut}{\text{Aut}}
% Variance
\newcommand{\var}{\text{Var}}
% Covariance
\newcommand{\cov}{\text{Cov}}
% Cofactor (matrix)
\newcommand{\cof}{\text{Cof}}
% Adjugate (matrix)
\newcommand{\adj}{\text{Adj}}
% Trace (matrix)
\newcommand{\tr}{\text{tr}}
% Standard deviation
\newcommand{\std}{\text{Std}}
% Correlation coefficient
\newcommand{\corr}{\text{Corr}}
% Sign
\newcommand{\sign}{\text{sign}}

% And text
\newcommand{\AND}{\text{ and }}
% Or text 
\newcommand{\OR}{\text{ or }}
% For text 
\newcommand{\FOR}{\text{ for }}
% If text
\newcommand{\IF}{\text{ if }}
% When text
\newcommand{\WHEN}{\text{ when }}
% Where text
\newcommand{\WHERE}{\text{ where }}
% Then text
\newcommand{\THEN}{\text{ then }}
% Such that text
\newcommand{\SUCHTHAT}{\text{ such that }}

% 1st
\newcommand{\st}[1]{#1^{\text{st}}}
% 2nd
\newcommand{\nd}[1]{#1^{\text{nd}}}
% 3rd
\newcommand{\rd}[1]{#1^{\text{rd}}}
% nth
\newcommand{\nth}[1]{#1^{\text{th}}}


%===============================================================================
%========================= PROBABILITY AND STATISTICS ==========================
%===============================================================================


% Permutation
\newcommand{\perm}[2]{{}^{#1}\!P_{#2}}
% Combination
\newcommand{\comb}[2]{{}^{#1}C_{#2}}

% Baye's risk
\newcommand{\risk}[1]{\mathscr{R}_{#1}}
% Baye's optimal risk
\newcommand{\riskOptimal}[1]{\mathscr{R}_{#1}^*}
% Baye's empirical risk
\newcommand{\riskEmpirical}[2]{\hat{\mathscr{R}}_{#1}^{#2}}


%===============================================================================
%=================================== CALCULUS ==================================
%===============================================================================


% d over d derivative
\newcommand{\dd}[2]{\dfrac{d#1}{d#2}}
% partial d over d derivative
\newcommand{\partialdd}[2]{\dfrac{\partial #1}{\partial #2}}
% delta d over d derivative
\newcommand{\deltadd}[2]{\dfrac{\Delta #1}{\Delta #2}}

% Integration between a and b
\newcommand{\integral}[4]{\int_{#1}^{#2} #3 \, #4}
% Integration in some space 
\newcommand{\boundIntegral}[2]{\int_{#1} #2 \, d#1}

% Limit
\newcommand{\limit}[3]{\lim_{#1 \to #2} #3}


%===============================================================================
%================================  BIG SYMBOLS  ================================
%===============================================================================


% Sum
\newcommand{\sumof}[2]{\sum_{#1}^{#2}}
% Product
\newcommand{\productof}[2]{\prod_{#1}^{#2}}
% Union
\newcommand{\unionof}[2]{\bigcup_{#1}^{#2}}
% Intersection
\newcommand{\intersectionof}[2]{\bigcap_{#1}^{#2}}
% Or
\newcommand{\orof}[2]{\bigvee_{#1}^{#2}}
% And
\newcommand{\andof}[2]{\bigwedge_{#1}^{#2}}


%===============================================================================
%=============================== LINEAR ALGEBRA ================================
%===============================================================================


% Bold vector arrow
\newcommand{\bv}[1]{\vec{\mathbf{#1}}}

% Matrix or vector (use // for column, & for row) 
% with brackets
\newcommand{\bmat}[1]{\begin{bmatrix} #1 \end{bmatrix}}
% Matrix or vector (use // for column, & for row) 
% with curved brackets
\newcommand{\pmat}[1]{\begin{pmatrix} #1 \end{pmatrix}}
% Matrix or vector (use // for column, & for row) 
% with lines on either side 
\newcommand{\lmat}[1]{\begin{vmatrix} #1 \end{vmatrix}}
% Matrix or vector (use // for column, & for row) 
% with curly braces
\newcommand{\cmat}[1]{\begin{Bmatrix} #1 \end{Bmatrix}}
% Matrix or vector (use // for column, & for row) 
% with no braces
\newcommand{\mat}[1]{\begin{matrix} #1 \end{matrix}}


%===============================================================================
%================================ LARGE OBJECTS ================================
%===============================================================================

% Multiple lines
\newcommand{\multiline}[1]{
\begin{align*}
    #1
\end{align*}
}

% Multiple lines with equation numbers
\newcommand{\eqmultiline}[1]{
\begin{align*}
    #1
\end{align*}
}

% Color
\newcommand{\colorText}[2]{
\begingroup
\color{#1}
    #2
\endgroup
}

% Centered figure
\newcommand{\centerFigure}[2]{
    \begin{figure}[h]
        \centering
            #1
        \caption{#2}
    \end{figure}
}

% Tikz figure
\newcommand{\tikzGraphic}[1]{
    \begin{center}
    \begin{tikzpicture}
        #1
    \end{tikzpicture}
    \end{center}
}

% Enumerate numbers (seperate by \item)
\newcommand{\numbers}{\textbf{\number*)}}

% Enumerate letters (seperate by \item)
\newcommand{\letters}{\textbf{\alph*)}}

\title{%
    \Huge Numerical Algorithms \\
    \Large Lecture IX
}
\date{2025-05-19}
\author{Michael Saba}

\begin{document}
\pagenumbering{gobble}
\maketitle
\newpage
\setlength{\parindent}{0pt}
\pagenumbering{arabic}

\subsection*{Area Formulation}

Instead of integrating over all the directions
$\omega$ in $\Omega$, the hemisphere
above $x$,
then we can use an area formulation;
The idea behind it is to integrate
in terms of all the points $y$
from which a light ray can be projected
towards $x$. \\

We can take each small area $d\sigma(\omega)$
on the hemisphere and project them.
They will hit points on the surfaces
of the scene $\mcal$. \\

We can use a visibility function $V(x, y)$
to ensure that we have the first surface we project
onto (the rest aren't visible and are hidden
when we project). \\

We would then have $d\sigma(\omega)
= \dfrac{\cos(\theta)y}{\|x-y\|^2}dA(y)$
where $y$ is the point intersected
when we project from $x$ in the $\omega$
direction, and $dA(y)$ is a small area around it
on the surface in $\mcal$. \\

We can absorb these terms into the
term $G(x, y)$:
\[ G(x, y) = \dfrac{((y-x)\cdot N(x))((x-y)\cdot N(y))}
{\|x-y\|^4} \]
Note that we also added the term $(y-x)\cdot N(x)$
which also appears in the original integral. \\

So instead of:
\[  L_{o}(x, \omega_o)= L_{e}(x, \omega_o) +
\int_\Omega L_{i}(x, \omega_i) \cdot
f(x, \omega_i, \omega_o) \cdot
(\omega_i \cdot n) \; d\sigma(\omega_i) \]
We will have:
\[  L_{o}(x, \omega_o)= L_{e}(x, \omega_o) +
\int_\mcal L_{i}(x, y) \cdot
f(x, xy, \omega_o) G(x, y) V(x, y) \; dA(y) \]
So we now have the same integral
for the rendering equation, but instead
of thinking in terms of directions, we think
in terms of points on the scene $x$ and $y$,
and light bouncing between them. \\
So we can get the outgoing light $L_o$
at a point $x$ in the direction towards a point $y$,
which is $0$ if the visibility function
determines we can't see $x$ from $y$ and vice-versa. \\

\newpage

\subsection*{Neumann Expansion}

We know that:
\[ \sum_i^{\infty} k^i = \dfrac{1}{1 - k} \]
converges so long as $|k| < 1$. \\

Now, to generalize this idea to matrices,
we can use the Neumann series:
\[ \sum_i^{\infty} K^i = (I - K)\inv \]
Where the eigenvalues of $K$
all satisfy $|\lambda_i| < 1$. \\
Notice here that this is the same result
it converges to as before, but written in matrix
notation. \\

So how can we prove that this work?
We will use the diagonalization:
\[ K = P\Lambda P\inv \]
In both the left and right hand sides,
and then show that they are equal. \\

For the right hand side:
\[ \sum_i^{\infty} K^i
= \sum_i^{\infty} (P\Lambda P\inv)^i \]
Notice that because this is the eigen decomposition
and $P$ is orthogonal,
that $(P\Lambda P\inv)^i = P\Lambda^i P\inv$.
So we have:
\[ \sum_i^{\infty} P\Lambda^i P\inv
= P \para{\sum_i^{\infty}\Lambda^i} P\inv \]
Notice that because $\Lambda$ 
is diagonal, we can distribute both the 
sum and the power to the diagonal elements:
\[ P \bmat{ 
\sum_i^{\infty} \lambda_1^i & 0 & \dots & 0 \vs{5} \\
0 & \sum_i^{\infty} \lambda_2^i & \dots & 0 \vs{5} \\
\vdots & \ddots & \ddots & \vdots \vs{5} \\
0 & \dots & 0 & \sum_i^{\infty} \lambda_n^i } P\inv \]
Because we assumed $|\lambda| < i$,
we can write the geometric sums
as the value they converge to:
\[ P \bmat{ 
\dfrac{1}{1-\lambda_1} & 0 & \dots & 0 \vs{5} \\
0 & \dfrac{1}{1-\lambda_2} & \dots & 0 \vs{5} \\
\vdots & \ddots & \ddots & \vdots \vs{5} \\
0 & \dots & 0 & \dfrac{1}{1-\lambda_n} } P\inv \]
We can stop here. \\

Now for the left hand side, note that we have:
\multiline{ (I - K)\inv = (I - P\Lambda P\inv)\inv
& = (PIP\inv - P\Lambda P\inv)\inv 
= (P (I - \Lambda) P\inv)\inv \\
& = P (I - \Lambda)\inv P\inv }
Note that $I - \Lambda$
is a diagonal matrix with diagonal entries
$1 - \lambda_i$. This means that
we can distribute the inverse (since it is
diagonal):
\[ P \bmat{ 
\dfrac{1}{1-\lambda_1} & 0 & \dots & 0 \vs{5} \\
0 & \dfrac{1}{1-\lambda_2} & \dots & 0 \vs{5} \\
\vdots & \ddots & \ddots & \vdots \vs{5} \\
0 & \dots & 0 & \dfrac{1}{1-\lambda_n} } P\inv \]
This is the same as the result we got before. \\
This proves the Neumann series of $(I - K)\inv$
is correct. \\

Note that $\rho(K)$
is the spectral radius or largest eigenvalue
of $K$. \\

We need $|\rho(K)| < 1$ to be satisfied
in order for the Neumann series to work,
so that each geometric series in the diagonal
converges.
This means that $\|K\| < 1$. \\

In the last section, we already
showed how the recursive nature of the
rendering equation,
we could write it as an integral of an integral
of an integral...
\multiline{
L_{o}(x, \omega_o)
& = L_{e}(x, \omega_o) \\
& + \int_\Omega \bigg( L_e(r(x, \omega_i), -\omega_i)
+ f(x, \omega_i, \omega_o) \cdot 
(\omega_i \cdot n) \\
& + \int_\Omega 
\bigg( L_e(r(r(x, \omega_i), \omega_j), -\omega_j) 
+ f(x, \omega_j, -\omega_i) \cdot 
(\omega_j \cdot n) \\
& + \int_\Omega \bigg( 
L_e(r(r(r(x, \omega_i), \omega_j), \omega_k), -\omega_k)
+ f(x, \omega_k, -\omega_j) \cdot 
(\omega_k \cdot n) \\ 
& + \int_\Omega \dots \bigg)
\; d\omega_k \bigg) \; d\omega_j \bigg) \; d\omega_i }
Which is equal to the Neumann series:
\[ L_o = L_e + T( L_e + T( L_e + T (L_e + \dots))) \]
\[ L_o = L_e + TL_e + T^2L_e + T^3L_e \dots \]
But, what if we want to have a non-recursive
formula for each term $T^kL_e$? \\

As we will see later in Monte-Carlo integration,
we will need each term individually
written out as an integral. For simplcity
we will use the area formulation of the rendering
equation. \\

First, note that each term is a function
in terms of $x_0$ and $x_1$,
that is:
\[ T^kL_e = (T^kL_e)(x_0, x_1) \]
This term just outputs the light that $x_0$
gets through $x_1$, after $k$ bounces. \\

So the first term is directly
just the ligth emitted from $x_1$
towards $x_0$:
\[ L_e(x_0, x_1) = L_e(x_1 \to x_0) \]
which is to say, this is just
light that is 1 bounce away. \\
Now, $T$ is just the integral
over all the points $x_2$
from whence light might arrive.
What $TL_e$ does is output
the radiance $L_e$ that comes from
light emmitted from all the points $x_2 \in \mcal$
that arrive at $x_1$:
\[ TL_e (x_0, x_1) = \int_\mcal L_e(x_2 \to x_1)
\cdot f(x_1, x_1x_2, x_1x_0) \cdot G(x_1, x_2)
\cdot V(x_1, x_2) \; dA(x_2) \]
This is why we have $L_e(x_2 \to x_1)$
in the integral; it gives us all the radiance
at $x_1$ emitted from all points $x_2$,
which is then reflected in the direction of $x_0$
(two bounces in total). \\
Now for the third term, we have
$T^2L_e$. Note that this is just
$T(TL_e)$,
which means that we want to wrap another
integral around our earlier term.
What this does mathematically is calculate
all the emitted light from all points $x_3$
that arrive at each point $x_2$,
which then all contribute to $x_1$,
which then emits light in the direction of $x_0$.
So we are just calculating light from $2$
bounces away:
\multiline{ TL_e (x_0, x_1) = 
\int_\mcal & \para{ \int_\mcal L_e(x_3 \to x_2)
\cdot f(x_2, x_2x_3, x_2x_1) \cdot G(x_2, x_3)
\cdot V(x_2, x_3) \; dA(x_3)} \\
& f(x_1, x_1x_2, x_1x_0) \cdot G(x_1, x_2)
\cdot V(x_1, x_2) \; dA(x_2) }
The inner integral integrates over
all points $x_3$ that emit light
towards all points $x_2$,
which then reflect the light towards $x_1$,
which then relect the light towards $x_0$,
which is 3 bounces away. \\

Intuitively, this makes perfect sense;
the light that exists in a scene is
dependent only on the emitted light, as
incidant or exitant light originates 
as emitted light somewhere else. \\
Each term in the Neumann series is just
emitted light that takes $k$ bounces to arrive. \\
The larger the $k$, the fainter the light,
as it some light is usually absorbed
by the surface as per the conservation
of enegery. \\

The general formula for a term is now thus:
\multiline{TL_e (x_0, x_1) = 
& \int_{\mcal^k} L_e(x_2 \to x_1) \\
& \productof{n=1}{k} \big(
f(x_{n-1}, x_{n-1}x_{n}, x_{n-1}x_{n-2}) 
\cdot G(x_{n-1}, x_{n})
\cdot V(x_{n-1}, x_{n}) \; dA(x_{n}))}
Which has no recursion, and only
depends on emitted light.
This is the proper integral format we need
to do Monte-Carlo integration. \\
Note that $\int_{\mcal^k}$
is just $k$ integrals nested in each other. \\

\newpage

\subsection*{Monte-Carlo Integration}

In this section, we explain how to use Monte-Carlo
integration to compute each term $T^kL_e$. \\

First of all, why do we use Monte-Carlo
integration, instead of just using any
other integration method, like Riemann
Sums or the Trapezoidal rule? \\
These methods work fine for a small
dimension, but as the dimension of a scene
increases, they become less and less efficient. 
For example, Riemann sums use rectangles in
$\rbb^2$, but columns in $\rbb^3$.
Recall that some of our integrals
$T^kL_e$ have dimensions $k$,
which can grow very large. \\
Monte-Carlo integration is probabilistic
and has far less of a dependency
on the dimension. \\

Now for some motivation;
we know that the average is just the
sum divided by the number of elements:
\[ \mu(x) = \dfrac{1}{n}\sum_{i=1}^n x_i \]
Intuitively, this means that we can write
the sum as the average multilplied the number
of elements:
\[ n\mu(x) = \sum_{i=1}^n x_i \]
This is the intuition behind Monte-Carlo
integration. \\

As we know, integration is just a sum,
over an interval instead of a number of elements.
So we can do the same thing here:
\[ \mu(x) = \dfrac{1}{b-a}\int_a^b x \; dx \]
Which means that:
\[ (b-a)\mu(x) = \int_a^b x \; dx \]
Here we use the measure (length)
of the interval, since we don't have a number
of elements, in order to calculate the intergal
using the average. \\

In a more general case, if we have:
\[ \int_\dcal f(x) \; dx \]
Then we would need the generalization
of the volume or length to $\dcal$,
whatever dimension $d$ it may be,
which we can write as $v_d(\dcal)$. \\
So we can approximate the integral as:
\[ \int_\dcal f(x) dx
= v_d(\dcal)\mu(f(x)) \]
Where we can get the average by just taking
$N$ random samples and dividing by $N$:
\[ \int_\dcal f(x) dx
= \dfrac{v_d(\dcal)}{N} \sum_{i=1}^N f(x_i) \]
This is how Monte-Carlo integration works. \\

However, one trick we can use
is to map the coordinates int $\dcal$
to a $d$-dimensional unit hypercube,
which always has a measure of $1$.
In $\rbb^2$ this is a unit square,
and in $\rbb^3$ this is a unit cube. \\
We can the unit hypercube in $\rbb^n$ 
as $[0, 1]^d$. \\
So if we can transform the variables
such that they span a hypercube,
we will have:
\[ \int_{[0, 1]^d} f(u) du
= \dfrac{1}{N} \sum_{i=1}^N f(u_i)  \]
Where the volume of the space is just $1$. \\

So, how can we do this mapping?
We can show it using an example. \\
Suppose we have an integral over a hemisphere:
\[ \int_\Omega f(\omega) \; d\sigma(\omega) \]
Which uses solid angles ($\sigma(\omega)$). \\
Now, we can use a mapping:
\[ T: [0, \sfrac{\pi}{2}] \times [0, 2\pi] \to \Omega \]
Which maps the angles to parametric coordinates:
\[ T(\phi, \theta) = \pmat{
    \cos(\phi)\sin(\theta) \\
    \sin(\phi)\sin(\theta) \\
    \cos(\theta)
} \]
Notice that the new coordinate space is a rectangle,
since we have $\theta \in [0, \sfrac{\pi}{2}]$
and $\phi \in [0, 2\pi]$. \\
Now note that the infinitessimal $d\sigma(\omega)$
will change to this when we perform the transformation:
\[ d\sigma(\omega) = |\det(J_T)| \; d\phi \; d\theta \]
Where $J_T$ is the Jacobian of the transformation,
and $\det(J_T)$ is calculated the same way
we calculate the determinant of any rectangular
matrix:
\[ \det(J_T) = \sqrt{(J_T)^TJ_T}
\qquad \AND \qquad J_T = \pmat{
-\sin\theta \sin\phi & \cos\theta \cos\phi  \\
\sin\theta \cos\phi  & \cos\theta \sin\phi\\
0 & -\sin\theta } \]
So in total, we get:
\[ \int_\Omega f(\omega) \; d\sigma(\omega)
= \int_0^{2\pi} \int_0^{\sfrac{\pi}{2}} 
f(T(\phi, \theta)) \cdot |\det(J_T)| 
\; d\theta \; d\phi\]
Which is the integration with a change
of variables. \\

Now, this is a rectangle with length $2\pi$
and width $\dfrac{\pi}{2}$,
and we need a unit square,
so we can just do a second transformation,
this time:
\[ F: [0, 1]^2 \to [0, \sfrac{\pi}{2}] \times [0, 2\pi] \]
Where:
\[ F(u, v) = \pmat{2\pi u \vs{10} \\ \dfrac{\pi}{2}v} \]
Which will also have a Jacobian $J_F$,
and which gets us a new integral:
\[ \int_0^{1} \int_0^{1} 
f(T(F(u, v))) \cdot
|\det(J_T)| \cdot
|\det(J_F)| \; du \; dv \]
Which we can also write as:
\[ \int_{[0, 1]^2}
f(T(F(u, v))) \cdot
|\det(J_T)| \cdot
|\det(J_F)| \; d(u, v) \]
Where $[0, 1]^2$
has area that is just $1$. \\

Note that the transformation has to be 
differentiable for this change of variables
to work. \\

If we are asked to calculate the
Monte-Carlo estimator of:
\[ \int_\Omega f(\omega) \; d\sigma(\omega) \]
We apply the change of variable
and end up with:
\[ \int_0^{1} \int_0^{1} 
f(T(F(u, v))) \cdot
|\det(J_T)| \cdot
|\det(J_F)| \; du \; dv \]
Which we can then turn into an estimator
by averaging (average times volume,
but the volume is 1):
\[ \dfrac{1}{N} \sumof{i = 1}{N}
f(T(F(u_i, v_i))) \cdot
|\det(J_T)| \cdot |\det(J_F)| \]
This is the final result. \\

So in summary;
our goal is to take each indiviual integral
$T^kL_e$,
find a differentiable transformation
that maps the variables to the unit hypercube
$[0, 1]^d$,
then find $N$ samples in $[0, 1]^d$,
which we then average out by dividing by $N$. \\

Note that sampling points from $[0, 1]^d$
is very easy; we just find $d$ random numbers
between $0$ and $1$. \\

That's why we needed to write the rendering
equation as a series, since we can't apply
Monte-Carlo integration to an integral
which itself depends on an integral term in it. \\

Note that in practice,
when doing this to the entire series:
\[ L_e + TL_e + T^2L_e + T^3L_e \dots \]
What we do is take $N = 1$,
which means that we only sample one point
each time. \\
We then calculate each term,
though having taken $N = 1$,
this means that we are just
tracing the bounces and path of a single
light ray as it bounces around. \\
We then average out several paths. 
So instead of averaging out
on each integral of $T^kL_e$,
we calculate several paths with one ray,
and then average them out. \\

\newpage

\end{document}