
\documentclass[12pt]{article}

\usepackage[margin=1in]{geometry}


%===============================================================================
%================================== PACKAGES ===================================
%===============================================================================

% For using float option H that places figures 
% exatcly where we want them
\usepackage{float}
% makes figure font bold
\usepackage{caption}
\captionsetup[figure]{labelfont=bf}
% For text generation
\usepackage{lipsum}
% For drawing
\usepackage{tikz}
% For smaller or equal sign and not divide sign
\usepackage{amssymb}
% For the diagonal fraction
\usepackage{xfrac}
% For enumerating exercise parts with letters instead of numbers
\usepackage{enumitem}
% For dfrac, which forces the fraction to be in display mode (large) e
% even in math mode (small)
\usepackage{amsmath}
% For degree sign
\usepackage{gensymb}
% For "\mathbb" macro
\usepackage{amsfonts}
% For positioning 
\usepackage{indentfirst}
\usetikzlibrary{shapes,positioning,fit,calc}
% for adjustwidth environment
\usepackage{changepage}
% for arrow on top
\usepackage{esvect}
% for mathbb 1
\usepackage{bbm}
% for mathsrc
\usepackage[mathscr]{eucal}
% For degree sign
\usepackage{gensymb}
% For quotes
\usepackage{csquotes}
% For vertical lines
\usepackage{mathtools}
% For cols
\usepackage{multicol}

% for tikz
\usepackage{pgfplots}
\pgfplotsset{compat=1.18}
\usepackage{amsmath}
\usepgfplotslibrary{groupplots}


%===============================================================================
%==================================== FONTS ====================================
%===============================================================================


% Mathcal
\newcommand{\acal}{\mathcal{A}}
\newcommand{\bcal}{\mathcal{B}}
\newcommand{\ccal}{\mathcal{C}}
\newcommand{\dcal}{\mathcal{D}}
\newcommand{\ecal}{\mathcal{E}}
\newcommand{\fcal}{\mathcal{F}}
\newcommand{\gcal}{\mathcal{G}}
\newcommand{\hcal}{\mathcal{H}}
\newcommand{\ical}{\mathcal{I}}
\newcommand{\jcal}{\mathcal{J}}
\newcommand{\kcal}{\mathcal{K}}
\newcommand{\lcal}{\mathcal{L}}
\newcommand{\mcal}{\mathcal{M}}
\newcommand{\ncal}{\mathcal{N}}
\newcommand{\ocal}{\mathcal{O}}
\newcommand{\pcal}{\mathcal{P}}
\newcommand{\qcal}{\mathcal{Q}}
\newcommand{\rcal}{\mathcal{R}}
\newcommand{\scal}{\mathcal{S}}
\newcommand{\tcal}{\mathcal{T}}
\newcommand{\ucal}{\mathcal{U}}
\newcommand{\vcal}{\mathcal{V}}
\newcommand{\wcal}{\mathcal{W}}
\newcommand{\xcal}{\mathcal{X}}
\newcommand{\ycal}{\mathcal{Y}}
\newcommand{\zcal}{\mathcal{Z}}

% Mathfrak
\newcommand{\afrak}{\mathfrak{A}}
\newcommand{\bfrak}{\mathfrak{B}}
\newcommand{\cfrak}{\mathfrak{C}}
\newcommand{\dfrak}{\mathfrak{D}}
\newcommand{\efrak}{\mathfrak{E}}
\newcommand{\ffrak}{\mathfrak{F}}
\newcommand{\gfrak}{\mathfrak{G}}
\newcommand{\hfrak}{\mathfrak{H}}
\newcommand{\ifrak}{\mathfrak{I}}
\newcommand{\jfrak}{\mathfrak{J}}
\newcommand{\kfrak}{\mathfrak{K}}
\newcommand{\lfrak}{\mathfrak{L}}
\newcommand{\mfrak}{\mathfrak{M}}
\newcommand{\nfrak}{\mathfrak{N}}
\newcommand{\ofrak}{\mathfrak{O}}
\newcommand{\pfrak}{\mathfrak{P}}
\newcommand{\qfrak}{\mathfrak{Q}}
\newcommand{\rfrak}{\mathfrak{R}}
\newcommand{\sfrak}{\mathfrak{S}}
\newcommand{\tfrak}{\mathfrak{T}}
\newcommand{\ufrak}{\mathfrak{U}}
\newcommand{\vfrak}{\mathfrak{V}}
\newcommand{\wfrak}{\mathfrak{W}}
\newcommand{\xfrak}{\mathfrak{X}}
\newcommand{\yfrak}{\mathfrak{Y}}
\newcommand{\zfrak}{\mathfrak{Z}}

% Mathscr
\newcommand{\ascr}{\mathscr{A}}
\newcommand{\bscr}{\mathscr{B}}
\newcommand{\cscr}{\mathscr{C}}
\newcommand{\dscr}{\mathscr{D}}
\newcommand{\escr}{\mathscr{E}}
\newcommand{\fscr}{\mathscr{F}}
\newcommand{\gscr}{\mathscr{G}}
\newcommand{\hscr}{\mathscr{H}}
\newcommand{\iscr}{\mathscr{I}}
\newcommand{\jscr}{\mathscr{J}}
\newcommand{\kscr}{\mathscr{K}}
\newcommand{\lscr}{\mathscr{L}}
\newcommand{\mscr}{\mathscr{M}}
\newcommand{\nscr}{\mathscr{N}}
\newcommand{\oscr}{\mathscr{O}}
\newcommand{\pscr}{\mathscr{P}}
\newcommand{\qscr}{\mathscr{Q}}
\newcommand{\rscr}{\mathscr{R}}
\newcommand{\sscr}{\mathscr{S}}
\newcommand{\tscr}{\mathscr{T}}
\newcommand{\uscr}{\mathscr{U}}
\newcommand{\vscr}{\mathscr{V}}
\newcommand{\wscr}{\mathscr{W}}
\newcommand{\xscr}{\mathscr{X}}
\newcommand{\yscr}{\mathscr{Y}}
\newcommand{\zscr}{\mathscr{Z}}

% Mathbb
\newcommand{\abb}{\mathbb{A}}
\newcommand{\bbb}{\mathbb{B}}
\newcommand{\cbb}{\mathbb{C}}
\newcommand{\dbb}{\mathbb{D}}
\newcommand{\ebb}{\mathbb{E}}
\newcommand{\fbb}{\mathbb{F}}
\newcommand{\gbb}{\mathbb{G}}
\newcommand{\hbb}{\mathbb{H}}
\newcommand{\ibb}{\mathbb{I}}
\newcommand{\jbb}{\mathbb{J}}
\newcommand{\kbb}{\mathbb{K}}
\newcommand{\lbb}{\mathbb{L}}
\newcommand{\mbb}{\mathbb{M}}
\newcommand{\nbb}{\mathbb{N}}
\newcommand{\obb}{\mathbb{O}}
\newcommand{\pbb}{\mathbb{P}}
\newcommand{\qbb}{\mathbb{Q}}
\newcommand{\rbb}{\mathbb{R}}
\newcommand{\sbb}{\mathbb{S}}
\newcommand{\tbb}{\mathbb{T}}
\newcommand{\ubb}{\mathbb{U}}
\newcommand{\vbb}{\mathbb{V}}
\newcommand{\wbb}{\mathbb{W}}
\newcommand{\xbb}{\mathbb{X}}
\newcommand{\ybb}{\mathbb{Y}}
\newcommand{\zbb}{\mathbb{Z}}


%===============================================================================
%=============================== SPECIAL SYMBOLS ===============================
%===============================================================================


% Orbit (group theory)
\newcommand{\orbit}{\mathcal{O}}
% Normal group
\newcommand{\normal}{\mathcal{N}}
% Indicator function
\newcommand{\indicator}{\mathbbm{1}}
% Laplace transform
\newcommand{\laplace}[1]{\mathcal{L}}
% Epsilon shorthand
\newcommand{\eps}{\varepsilon}
% Omega
\newcommand{\om}{\omega}
\newcommand{\Om}{\Omega}

%===============================================================================
%================================== OPERATORS ==================================
%===============================================================================


% Inverse exponent
\newcommand{\inv}[0]{^{-1}}
% Overline bar
\newcommand{\olsi}[1]{\,\overline{\!{#1}}}
% Less than or equal slanted
\newcommand{\seqs}{\leqslant}
% Greater or equal slanted
\newcommand{\geqs}{\geqslant}
% Subset or equal
\newcommand{\sub}{\subseteq}
% Proper subset
\newcommand{\prosub}{\subset}
% from
\newcommand{\from}{\leftarrow}

% Parantheses
\newcommand{\para}[1]{\left( #1 \right)}
% Curly Braces
\newcommand{\curl}[1]{\left\{ #1 \right\}}
% Brackets
\newcommand{\brac}[1]{\left[ #1 \right]}
% Angled Brackets
\newcommand{\ang}[1]{\left\langle #1 \right\rangle}
% Norm
\newcommand{\norm}[1]{\left\| #1 \right\|}

% Piece wise (use \\ between cases)
\newcommand{\piecewise}[1]{\begin{cases} #1 \end{cases}}

% Bold symbol shorthand
\newcommand{\bl}{\boldsymbol}

% Vertical space
\newcommand{\vs}[1]{\vspace{#1 pt}}
% Horizontal ertical space
\newcommand{\hs}[1]{\hspace{#1 pt}}


%===============================================================================
%============================== TEXT BASED SYMBOLS =============================
%===============================================================================


% Radians
\newcommand{\rad}{\text{rad}}
% Least Common Multiple
\newcommand{\lcm}{\text{lcm}}
% Automorphism
\newcommand{\Aut}{\text{Aut}}
% Variance
\newcommand{\var}{\text{Var}}
% Covariance
\newcommand{\cov}{\text{Cov}}
% Cofactor (matrix)
\newcommand{\cof}{\text{Cof}}
% Adjugate (matrix)
\newcommand{\adj}{\text{Adj}}
% Trace (matrix)
\newcommand{\tr}{\text{tr}}
% Standard deviation
\newcommand{\std}{\text{Std}}
% Correlation coefficient
\newcommand{\corr}{\text{Corr}}
% Sign
\newcommand{\sign}{\text{sign}}

% And text
\newcommand{\AND}{\text{ and }}
% Or text 
\newcommand{\OR}{\text{ or }}
% For text 
\newcommand{\FOR}{\text{ for }}
% If text
\newcommand{\IF}{\text{ if }}
% When text
\newcommand{\WHEN}{\text{ when }}
% Where text
\newcommand{\WHERE}{\text{ where }}
% Then text
\newcommand{\THEN}{\text{ then }}
% Such that text
\newcommand{\SUCHTHAT}{\text{ such that }}

% 1st
\newcommand{\st}[1]{#1^{\text{st}}}
% 2nd
\newcommand{\nd}[1]{#1^{\text{nd}}}
% 3rd
\newcommand{\rd}[1]{#1^{\text{rd}}}
% nth
\newcommand{\nth}[1]{#1^{\text{th}}}


%===============================================================================
%========================= PROBABILITY AND STATISTICS ==========================
%===============================================================================


% Permutation
\newcommand{\perm}[2]{{}^{#1}\!P_{#2}}
% Combination
\newcommand{\comb}[2]{{}^{#1}C_{#2}}

% Baye's risk
\newcommand{\risk}[1]{\mathscr{R}_{#1}}
% Baye's optimal risk
\newcommand{\riskOptimal}[1]{\mathscr{R}_{#1}^*}
% Baye's empirical risk
\newcommand{\riskEmpirical}[2]{\hat{\mathscr{R}}_{#1}^{#2}}


%===============================================================================
%=================================== CALCULUS ==================================
%===============================================================================


% d over d derivative
\newcommand{\dd}[2]{\dfrac{d#1}{d#2}}
% partial d over d derivative
\newcommand{\partialdd}[2]{\dfrac{\partial #1}{\partial #2}}
% delta d over d derivative
\newcommand{\deltadd}[2]{\dfrac{\Delta #1}{\Delta #2}}

% Integration between a and b
\newcommand{\integral}[4]{\int_{#1}^{#2} #3 \, #4}
% Integration in some space 
\newcommand{\boundIntegral}[2]{\int_{#1} #2 \, d#1}

% Limit
\newcommand{\limit}[3]{\lim_{#1 \to #2} #3}


%===============================================================================
%================================  BIG SYMBOLS  ================================
%===============================================================================


% Sum
\newcommand{\sumof}[2]{\sum_{#1}^{#2}}
% Product
\newcommand{\productof}[2]{\prod_{#1}^{#2}}
% Union
\newcommand{\unionof}[2]{\bigcup_{#1}^{#2}}
% Intersection
\newcommand{\intersectionof}[2]{\bigcap_{#1}^{#2}}
% Or
\newcommand{\orof}[2]{\bigvee_{#1}^{#2}}
% And
\newcommand{\andof}[2]{\bigwedge_{#1}^{#2}}


%===============================================================================
%=============================== LINEAR ALGEBRA ================================
%===============================================================================


% Bold vector arrow
\newcommand{\bv}[1]{\vec{\mathbf{#1}}}

% Matrix or vector (use // for column, & for row) 
% with brackets
\newcommand{\bmat}[1]{\begin{bmatrix} #1 \end{bmatrix}}
% Matrix or vector (use // for column, & for row) 
% with curved brackets
\newcommand{\pmat}[1]{\begin{pmatrix} #1 \end{pmatrix}}
% Matrix or vector (use // for column, & for row) 
% with lines on either side 
\newcommand{\lmat}[1]{\begin{vmatrix} #1 \end{vmatrix}}
% Matrix or vector (use // for column, & for row) 
% with curly braces
\newcommand{\cmat}[1]{\begin{Bmatrix} #1 \end{Bmatrix}}
% Matrix or vector (use // for column, & for row) 
% with no braces
\newcommand{\mat}[1]{\begin{matrix} #1 \end{matrix}}


%===============================================================================
%================================ LARGE OBJECTS ================================
%===============================================================================

% Multiple lines
\newcommand{\multiline}[1]{
\begin{align*}
    #1
\end{align*}
}

% Multiple lines with equation numbers
\newcommand{\eqmultiline}[1]{
\begin{align*}
    #1
\end{align*}
}

% Color
\newcommand{\colorText}[2]{
\begingroup
\color{#1}
    #2
\endgroup
}

% Centered figure
\newcommand{\centerFigure}[2]{
    \begin{figure}[h]
        \centering
            #1
        \caption{#2}
    \end{figure}
}

% Tikz figure
\newcommand{\tikzGraphic}[1]{
    \begin{center}
    \begin{tikzpicture}
        #1
    \end{tikzpicture}
    \end{center}
}

% Enumerate numbers (seperate by \item)
\newcommand{\numbers}{\textbf{\number*)}}

% Enumerate letters (seperate by \item)
\newcommand{\letters}{\textbf{\alph*)}}

\title{%
    \Huge Numerical Algorithms \\
    \Large Lecture IV
}
\date{2025-05-03}
\author{Michael Saba}

\begin{document}
\pagenumbering{gobble}
\maketitle
\newpage
\setlength{\parindent}{0pt}
\pagenumbering{arabic}

\subsection*{Invariance and Variance}

We already saw how we can express vectors
in a basis other than the standard basis,
and switch between these two bases using
a transformation matrix and its inverse. \\
Now is the time to formalize this notion
for any two bases $B$ and $B'$. \\

A vector space is a 4-tuple:
\[ (V, S, +, \cdot) \]
Where $V$ is a set of vectors,
$S$ is a set of scalars,
the binary operation:
\[ +:V \times V \to V \]
defines vector addition,
and the binary operation:
\[ \cdot: S \times V \to V \]
defines scalar multiplication,
which can be thought of scaling vectors. \\
The set of vectors $V$ is closed under
both operations. \\

A linear combination of a set of vectors
is a sum of the vectors each scaled by
some scalar. \\

The basis of a vector space
is a set of linearly independent vectors
that span the entire vector space. \\
That is to say, it is a set of vectors:
\[ \{ e_1, e_2, \dots e_n \} \subseteq V \]
such that every vector in $V$
can be written as a linear combination
of the basis vectors. Moreover,
there can be no redundancy,
that is to say,
there should be a unique linear combination
that gives each vector in $V$. \\

Now to the main topic of this section.
We will show that vectors are invariant,
while their components are not invariant. \\

A vector represents a point in $\rbb^n$
space. That point is invariant,
meaning that it does not vary depending
on the basis in which the vector is expressed. \\

However, the components of the vector,
that is, the list of numbers used to represent
the vector, are variant. \\
This is because they
depend on the basis in which the vector
is being expressed, as the components are just
the coefficients of the linear combination
of the basis vectors that make up the vector. \\

We will now distinguish between covariance
and contravariance (two types of variance). \\

How can we transform a basis?
Suppose we have basis vectors of the form:
\[ \{ e_i \} \]
Then we can just apply a transformation 
using a matrix $A$ such that:
\[ e_i' = \{ Ae_i \} \]
This gives us the new basis vectors. \\

So the basis vectors are covariant;
you just transform the basis directly
using the transformation matrix. \\
We call this the forward transformation. \\

Now suppose we have a vector $v$. \\
As we said, the geometric vector (arrow in space)
is invariant, it will remain the same
no matter the basis.
But suppose that we want to express
the vector $v$,
which had previously been expressed in
the basis:
\[ \{e_i\} \]
in the new basis:
\[ \{ e_i' = Ae_i \} \]
we got from transforming the old basis. \\
How would we update the components
of the vector so they represent it
in the new basis? \\

As we know, in the original basis $\{e_i\}$
(which contains vectors $e_1, e_2, \dots e_n$),
we can express $v$ as a linear combination
of each old basis vector:
\[ v = \sum_{i = 1}^{n} v^ie_i \]
So all we have to do in order to update the
components of $v$ to the new basis is
express the old basis vectors $e_i$
in the new basis. \\
We know that:
\[ e_i' = Ae_i \]
Which means that:
\[ (A\inv)e_i' = e_i \]
So we can write:
\[ v = \sum_{i = 1}^{n} v^ie_i
= \sum_{i = 1}^{n} v^i (A\inv)e_i'
= (A\inv)\sum_{i = 1}^{n}v^i e_i' \]
So we just wrote $v$,
this time expressed as a linear combination
of the new basis vectors $e_i'$,
which means that the coefficients of the
linear combination are the components
of $v_i'$ of $v$,
expressed in the new basis $B'$. \\ 

So in the new basis, each component of $v$
will can be calculated using:
\[ (v^i)' = (A\inv)^i v \]
This is why we call the vector components
contravariant;
they vary inversely with the basis of choice.
We can call it the backward transformation. \\

We can thus write $v$ as:
\[ v = \sum_{i = 1}^{n} v^ie_i \]
in the old basis, or:
\[ v = \sum_{i = 1}^{n} (v^i)'e_i' \]
in the new basis. \\

Note that if the original basis $B$ happened
to be the standard basis, then the
transformation matrix $A$
that takes us to a new basis $B'$,
is just the matrix with the column vectors
as the basis vectors. \\
We already talked about this,
and now we've generalized the idea
to any two bases $B$ and $B'$. \\

\newpage

\subsection*{Covectors and Dual Bases}

Covectors are functions
that map vectors to the real numbers
(or any field on which the vector space
is defined). \\

We can think of covectors as just row vectors
along with the dot product operator:
\[ \alpha = \vecTwoTransposed{\alpha_2}{\alpha_2} \]
This function maps a vector $v$
to the dot product of the row vector $\alpha$
and the column vector $v$. \\

So for example, if we have a vector $v \in \rbb^2$,
then the covector $\alpha$ along with the dot product
operation form a function:
\[ \alpha(v) =
\vecTwoTransposed{\alpha_1}{\alpha_2}
\vecTwo{v^1}{v^2} = \alpha_1v^1 + \alpha_2v^2 \]
which maps the vector $v$ from $\rbb^2$
to $\alpha_1v^1 + \alpha_2v^2$ in $\rbb$. \\

More generally, we can define a covector
as the function from the vector space
onto the reals as:
\[ \alpha: V \to \rbb \]
where:
\[ \alpha(v) = \sum_{i = 1}^{n} \alpha_iv^i \] \\

We notice some things about covectors. \\
First, note that for any covector $\alpha$,
and any two vectors $v, u$:
\[ \alpha(v + u) = \alpha(v) + \alpha(u) \]
Moreover, for any scalar $\lambda$
and vector $v$:
\[ \alpha(\lambda v) = \lambda \alpha(v) \]
This is called linearity,
which will come in handy later. \\

We can visualize vectors as arrows in space,
but how can we visualize covectors?
Drawing them as arrows using their components
as the coefficients of a linear combination
of basis vectors is deceptive,
since a covector is a function. \\

Instead, a good way to visualize
a covector $\alpha$ is a set of parallel lines,
all perpendicular to the vector $\alpha^T$.
The output of the function $\alpha$
given a vector $v$ depends on how many of
these lines the vector $v$ pierces.
It's a measure of how much the vector $v$ is in
the direction of the vector $\alpha^T$. \\

Now suppose we have two covectors,
$\beta$ and $\gamma$. \\
What would the function that combines (sums)
these two covectors output?
We can use the sum formula to answer that:
\[ (\beta + \gamma)(v)
= \sum_{i = 1}^{n} (\beta_i + \gamma_i)v^i
= \sum_{i = 1}^{n} \beta_iv^i
+ \sum_{i = 1}^{n} \gamma_iv^i
= \alpha(v) + \gamma(v) \]
So it is the sum of the individual outputs. \\
Visually, we can think of this
as a set of parallel lines rotated sort of
between the set of lines of $\beta$
and $\gamma$,
with a density equal to the sum of the
density of the other two (density being
how close together the lines are).
This new set of lines will measure
how much the vector $v$ pierces
both directions $\beta^T$ and $\gamma^T$,
and adds them together. \\

We can also scale covectors. \\
For a scalar $\lambda \in \rbb$,
we can scale a covector $\alpha$:
\[ (\lambda \alpha)(v)
= \sum_{i = 1}^{n} (\lambda \alpha_i)v^i
= \lambda \sum_{i = 1}^{n}\alpha_iv^i 
= \lambda \alpha(v)\]
So what a scalar does is just scale the output,
which can be visually thought of as scaling
the density of the lines of the covector. \\

Now that we have a scaling and adding operation
for covectors, we can define the dual vector
space. \\

Given a vector space $(V, S, +, \cdot)$,
we can define the dual vector space to be:
\[ (V^*, S, +, \cdot) \]
Where $V^*$ contains all of the covectors that
act on vectors in $V$. \\
Note that the set of scalers $S$
is the same as the vector space's,
but the $+$ and $\cdot$ operators
are distinct. \\

Like vectors, covectors are invariant,
while covector components are variant. \\

The covector is a funciton represented by 
a row vector and the dot product;
so the effect of the function itself,
what it does, does not depend on the basis.
It is invariant. \\

However, the components used to represent the
covector as a row vector do depend on the
basis the covector is in. \\

However, note that when we say basis here,
we don't mean a basis of the vector space,
but rather, a basis in the dual vector space,
since this is where covectors live. \\

Given a basis $\{ e_i \}$ for the vector space $V$,
we can define a special set of covectors in $V^*$:
\[ \{ \epsilon^i \} = 
\{ \epsilon^1, \epsilon^2, \dots, \epsilon^n \} \]
which have a desirable property. \\
We can define a set of covectors such that:
\[ \epsilon^i(e_j) = 1 \iff i = j \]
That is to say, the output of applying
one of those covectors to the basis returns
$1$ if the covector
and vector share the same index,
and $0$ otherwise. \\
Notice then that if we apply every covector
$\epsilon^1, \epsilon^2, \dots \epsilon^n$
to one vector $e_i$,
we get a whole column of scalers,
which actually form a vector,
with $0$s in all entries except for one. \\
If we repeat this for every vector
$e_1, e_2, \dots, e_n$,
we get a set of vectors that form the
identity matrix. \\
Another way to write this is:
\[ \epsilon^i(e_j) = \delta_j^i \]
Where $\delta_j^i = 1$ if $i = j$,
and $\delta_j^i = 0$ if $i \neq j$. \\

So what happens if we apply these special
covectors to any vector $v$
which is expressed in the basis $\{e_i\}$?
Well, we can express $v$
as a linear combination of the basis vectors:
\[ v = \sum_{i = 1}^{n} v^ie_i \]
We can then send this sum
as input to one of the covectors $\epsilon^k$:
\[ \epsilon^k(v)
= \epsilon^k\para{\sum_{i = 1}^{n} v^ie_i} \]
And then we can decompose this into
a linear combination using the linearity of
the covector:
\[ \sum_{i = 1}^{n} v^i\epsilon^k(e_i)  \]
We already know what the output of the covector
is when the input is a basis vector.
The operands $\epsilon^k(e_i)$
in this sum will be $0$ for all $i$
except $i = k$,
where $\epsilon^k(e_k) = 1$. \\
So the result of the sum is:
\[ \epsilon^k(v) = v_k \cdot \epsilon^k(e_k)
= v^k \cdot 1 = v^k \]
So each covector $\epsilon^k$
just extracts the $\nth{k}$ component
of the vector $v$ expressed in the
basis $\{ e_i \}$. \\

This is very helpful, because it means
that we can write covectors as linear
combinations of these special covectors. \\
Again we have a vector $v$ expressed
in the basis $\{ e_i \}$,
and we have the covectors $\{\epsilon^i\}$
derived from $\{ e_i \}$. \\
So suppose we have an arbitrary covector
$\alpha \in V^*$,
which acts on some vector $v \in V$. \\
We know we can write $v$ as a linear
combination of basis vectors $\{e_i\}$, so:
\[ \alpha(v) 
= \alpha\para{\sum_{i = 1}^{n} v^ie_i} \]
So by the linearity of covectors, we have:
\[ \alpha(v) = \sum_{i = 1}^{n} v^i\alpha(e_i) \]
But we know that each component
$v_i$ can be extracted from $v$ by:
\[ v^i = \epsilon^i(v) \]
So the sum becomes:
\[ \alpha(v) = \sum_{i = 1}^{n} 
\alpha(e_i)\epsilon^i(v) \]
Notice that each $\alpha(e_i)$
is just a scalar,
which means that the sum is a linear combination
of the covectors $\{ \epsilon^i \}$. \\
To conclude, we can write a covector
as a linear combination of the special
covectors $\{ \epsilon^i \}$,
when acting on a vector expressed
in basis $\{e_i\}$
(using which we defined the special covectors). \\

This special set of covectors $\{ \epsilon^i \}$
is called the dual basis of the basis $\{e_i\}$. \\
It acts similar to a basis;
if we want to express a covector $\alpha$
in a specific dual basis,
then we just write it as a row vector
whose components are the coefficients
of the linear combination of the dual basis
covectors that give us $\alpha$. \\

Each dual basis is specific to a basis. \\

So covector components are variant,
since they depend on the dual basis
they are in. \\

However, notice what happens if we want to
transform from an old dual basis $D$ to a
new dual basis $D'$. \\
First we have the original basis $B$
with vectors $\{e_i\}$,
which we can transform into $B'$
using the matrix $A$.
The dual basis $D'$ is the one we derive
from the new basis $B'$. \\
As we know:
\[ e_j' = Ae_j 
\AND e_j = (A\inv)e_j' \]
We also know that we can derive our dual basis
$D$ with covectors $\{\epsilon^i\}$
from the basis $B$ using:
\[ \epsilon^i(e_j) = \delta_j^i \]
But this means that:
\[ \epsilon^i((A\inv) e_j')
= \epsilon^i(e_j) = \delta_j^i \]
Matrix multiplication is associative, so:
\[ \epsilon^i((A\inv) e_j')
=  (\epsilon^i(A\inv))(e_j') 
= \delta_j^i \]
But as we know, if we have a set of covectors,
that take the new basis vectors $\{ e_i' \}$
as input, and return $\delta_j^i$,
then these covectors form the dual basis of $B'$.
This means that the new dual basis $D'$
is made up of covectors:
\[ (\epsilon^i)' = \epsilon^i(A\inv) \]
Which means we can transform from the old
dual basis to the new dual basis
using the backwards (inverse) transformation. \\
This makes the dual basis contravariant. \\

We know that we can write covectors
as linear combinations of dual basis
covectors:
\[ \alpha(v) = \sum_{i = 1}^{n} 
\alpha(e_i)\epsilon^i(v) \]
We also know that:
\[ (\epsilon^i)' = \epsilon^i(A\inv) \]
Which means that:
\[ (\epsilon^i)'A = \epsilon^i \]
So now we can write the covector $\alpha$
in terms of the new dual basis covectors
by just replacing $\epsilon_i$
by $(\epsilon_i)'A$:
\[ \alpha(v) 
= \sum_{i = 1}^{n} (\alpha(e_i))((\epsilon_i)'A)(v) \]
We can then use matrix associativity to do this:
\[ \alpha(v) = 
A\sum_{i = 1}^{n} (\alpha(e_i))(\epsilon_i)'(v) \]
We know that if we can express a covector
as a linear combination of dual basis covectors,
then the coefficients of the dual basis covectors
are the components of the covector in that dual
basis. \\

So since we can express the covector $\alpha$
as a linear combination of the new dual basis
covectors $\{ (\epsilon^i)' \}$,
this means that the coefficients are the
components $\alpha_k'$ of the covector
expressed in the new dual basis:
\[ \alpha_k' = \alpha A^k \]
Which means we need to use a forward
transformation to get the covector components
expressed in a new dual basis. \\
So covector components are covariant. \\

So covectors and dual basis covectors
are the opposite of vectors and basis vectors
when it comes to which is covariant and
which is covariant. \\

Note that when we transform a covector,
we do a matrix multiplication from
the right, since it's a row vector:
\[ \alpha_k' = \alpha A^k \]
And row vectors can only be mutliplied
by matrices from the right side. \\

Also note that the same forward and backward
transformation matrices used to transform
the basis and vectors are used 
on the covectors and dual basis,
but with the backwards transform used
on the dual basis instead of the covector. \\

\newpage

\subsection*{Rotations in Three Dimensions}

In $\rbb^2$,
the only rotation we can do is in the $xy$ plane,
so we only have one angle of rotation $\theta$. \\

In $\rbb^3$ however,
we can actually rotate vectors in three planes
(around three seperate axes):
$R_x$ is the rotation in the $yz$ plane,
$R_y$ is the rotation in the $xz$ plane,
and $R_z$ is the rotation in the $xy$ plane. \\

Each one can be expressed similarly to 
a rotation in $\rbb^2$ (since they are a rotation
in a single plane), but in a $3\times 3$ matrix:
\[ R_x(\theta) = \matTwo{0}{0}{0}{R(\theta)}
\matThree{0}{0}{0}{0}{\cos(\theta)}
{-\sin(\theta)}{0}{\sin(\theta)}{\cos(\theta)} \]
\[ R_y(\alpha) =
\matThree{\cos(\alpha)}{0}{-\sin(\alpha)}{0}{0}
{0}{\sin(\alpha)}{0}{\cos(\alpha)} \]
\[ R_z(\beta) = \matTwo{R(\beta)}{0}{0}{0}
\matThree{\cos(\beta)}{-\sin(\beta)}{0}
{\sin(\beta)}{\cos(\beta)}
{0}{0}{0}{0} \] \\

We can express a full $3$-dimensional rotation
by just composing the matrices:
\[ R(\theta, \alpha, \beta) = 
R_z(\beta)R_y(\alpha)R_x(\theta) \]
Order does matter. \\
Do note that using these angles 
(Euler angles) can cause gimbal lock. \\

For a $4$-dimensional rotation, we have $6$
rotation matrices. 
This is because there are $6$ plane to rotate
around:
\[ xy \quad xz \quad xw \quad
yz \quad yw \quad zw \] 
In general, for an $n$-dimensional space,
we will have $\comb{n}{2}$ individual
rotations in a plane. \\

Rotations and reflections are both
orthogonal, meaning that they preserve
angles and lengths after transforming. \\
Both rotations and reflections
(and combinations of both)
have determinants with absolute values
equal to $1$ for this reason,
with transformations that reflect
the vector space in some way having
determinant $-1$,
and pure rotations having determinant $1$. \\

In fact, the group of all $n \times n$
matrices with determinants
that have absolute value equal to $1$,
which encompasses all rotation and reflection
matrices, as well as their combinations,
form a group:
\[ O(n) = \{ M \mid M \in \rbb^{n \times n}
\text{ and } MM^t = M^TM = I \} \]
Which is called an orthogonal group. \\

Each orthogonal group has a
subgroup called the special
orthogonal group, which only contains
rotations:
\[ SO(n) = \{ M \mid M \in O(n)
\text{ and } \det(M) = 1 \} \]
Note that since some
rotations in $SO(3)$
correspond to rotations in only
one plane $xy$, $xz$ or $yz$,
that means that these rotations
in $SO(3)$ form subgroups that are
isomorphic to $SO(2)$. \\

For example, if $\varphi$
is the isomorphism from $R_x$
to $SO(2)$:
\[ \varphi: R_x \to SO(2) \]
then we can map:
\[ \varphi\para{
\matThree{0}{0}{0}{0}{\cos(\theta)}
{-\sin(\theta)}{0}{\sin(\theta)}
{\cos(\theta)}}
= \matTwo{\cos(\theta)}
{-\sin(\theta)}{\sin(\theta)}
{\cos(\theta)} \] \\

\newpage

\subsection*{Gram-Schmidt Process and
More on Rotations}

Suppose that we are in $\rbb^3$,
and we want to rotate
a vector or point $p$ around some plane
that doesn't necessarily coincide
with the $xy$, $xz$, or $yz$ planes. \\

We can imagine that the plane in which we
want to rotate the point is given to
us by two linearlly independent vectors
$f_1$ and $f_2$ that span the plane. \\

The first issue we will encounter is that
the point may not be on the plane
since we are in $\rbb^3$. \\
In order to rotate a point in this plane,
we will first have to project the point
onto the plane,
then rotate it, then bring the point back up
(undo the projection). \\

A second problem we will encounter is
that the rotation matrices we've used
assume we have an orthonormal basis,
with orthogonal basis vectors of length $1$. \\
We can resolve this by using
the Gram-Shmidt process to turn the
two vectors $f_1$ and $f_2$ into an orthonormal
basis. \\

This is how the Gram-Shmidt process works. \\
First we will take:
\[ u_1 = f_1 \]
to be the first orthonormal basis vector. \\
Now we want the second basis vector $u_2$ to
be orthogonal to $u_1$.
What we can do is project $f_2$ onto $f_1$,
then take the vector that connects
the tip of the projection to the tip of $f_2$,
which is orthogonal to $f_1$
since it is a projection. \\
To get the vector of the projection,
we note that the dot product $f_2 \cdot f_1$
gives us the norm of $f_1$
multiplied by the norm of the projection
of $f_2$ on $f_1$.
So if we were to divide this dot product
by the norm of $f_1$, we would
get the norm of the projection:
\[ \dfrac{f_1 \cdot f_2}{\|f_1\|} \]
So if we scale $f_1$ by this length,
we would get the projection
of $f_2$ on $f_1$:
\[ \dfrac{f_1 \cdot f_2}{\|f_1\|}f_1 \]
We can then define $u_2$
to be the vector from the tip of the projection
to the tip of $f_2$:
\[ u_2 = f_2 - \dfrac{f_1 \cdot f_2}{\|f_1\|}f_1 \]
We can then normalize both $u_1$ and $u_2$,
meaning we divide by their
norm in order to ensure they have length $1$,
in order to get our final orthonormal basis vectors:
\[ e_1 = \dfrac{u_1}{\|u_1\|} \AND
e_2 = \dfrac{u_2}{\|u_2\|} \]
These two vectors form our orthonormal basis. 

We will call $A$ the orthonormal basis formed by
$e_1$ and $e_2$. \\
Note that $A$ has two $3$-dimensional vectors,
which are linearlly independent,
and span the plane that was spanned
by $f_1$ and $f_2$. \\

Now that we have our orthonormal basis $A$,
we can rotate $p$ around the plane
spanned by $A$ using some $2$-dimensional
vector $R$. \\
First we want to find the projection
of $p$ on the plane expressed in the basis $A$.
This can be done by solving the normal
form equation. The goal was to minimize 
the distance between the $Ax$
(a linear combination of the basis vectors, 
which can be anywhere on the plane),
and the point $p$. \\
Solving
\[ \nabla \|Ax - p\|^2 = 0 \]
gives us the square system:
\[ A^TAx = A^Tp \]
Where the projection will have coordinates:
\[ x = (A^TA)\inv A^Tp  \]
Now, $x$ has become a $2$-dimensional vector. \\
However, we assumed that $A$
was orthonormal in this particular case,
which means that $A^TA = I$,
so we can simplify:
\[ x = (A^TA)\inv A^Tp
= A\inv(A^T)\inv A^Tp = 
A\inv p = A^Tp \]
So $x = A^Tp$. \\
Now that the projection is expressed in
the orthonormal basis $x$,
we can rotate it by just applying $R$:
\[ RA^Tp \]
We now have the rotated point on the plane
expressed in the basis $A$. \\
We now need to reexpress the vector back
in the original basis (which was the standard
basis).
We can do that by multiplying by $A$.
This is because vector components are contravariant
(as we saw earlier),
meaning that multiplying them by a transformation,
turns the vector back to the original basis. \\
So the $2$-dimensional
vector back to a $3$-dimensional vector,
but the point still remains on the plane:
\[ ARA^Tp \]
Now we need to undo the projection. 
Recall that:
\[ x = (A^TA)\inv A^Tp = A^Tp \]
are the coordinates of the projected
point expressed in the basis $A$.
To get the coordinates of the point
in the original basis,
but after having projected it,
we multiply by $A$:
\[ AA^Tp \]
which gives us a $3$-dimensional vector
of the projected point on the plane
before rotation.
This means that:
\[ p - AA^Tp \]
Is the vector that points from the 
tip of the projection of $p$ to
the tip of the original $p$. \\
That's why we needed the projected point
in the original basis: both points
need to be in the same basis in order for
it to make sense to subtract one from the other. \\
So if we add this vector to the
rotated point, we can undo the projection:
\[ ARA^Tp + p - AA^Tp \]
This should be rewritten such that
we are performing a single matrix transformation:
\[ (ARA^T + I - AA^T)p \] \\

So to conclude:
\begin{enumerate}
    \item 
    Create an orthonormal basis $A$
    using the Gram-Schmidt process.
    \item 
    Project $p$ onto the plane
    and solve the linear system to get
    the coordinates of the projection
    in the basis $A$.
    \item 
    Rotate $p$ now that it is in a $2$-dimensional
    orthonormal basis.
    \item
    Multiply the rotated point by $A$
    to send it back to the original $3$-dimensional
    basis it was in.
    \item 
    Also multiply the projected point $p$
    by $A$ to get it in the original basis,
    which allows us to subtract the result
    from $p$ to get the vector from
    the projection to the original point
    (only works if they are in the same basis).
    \item 
    Add the vector we got in the last step
    to the rotated point (now also in the original
    basis) in order to undo the projection.
\end{enumerate}
And this concludes the algorithm. \\

There is still one small curiosity. \\
In the general case, where $A$
is not orthonormal,
we would have arrived at this system:
\[ A^TAx = A^Tp \]
Here $A^Tp$ is the projected point. \\
In order to get the projected point
to be expressed in the new basis $A$,
we can solve the system and find that:
\[ x = (A^TA)\inv A^Tp  \]
is the projection in the basis $A$. \\
We also know that multiplying
this result by $A$
takes it back to the original basis
since vectors are contravariant:
\[ x = A(A^TA)\inv A^Tp \]
But if this is the projected point
expressed in the original basis,
then what is:
\[  A^Tp \]
Yes it is $p$ projected on the plane,
but what basis is it expressed in? \\

\newpage

\subsection*{Differential Equation for Rotation}

Suppose we don't know that a $2$-dimensional rotation
matrix has the format:
\[ R(\theta) = 
\matTwo{\cos(\theta)}{-\sin(\theta)}
{\sin(\theta)}{\cos(\theta)} \]
How would we find out that this is
the formula? \\
Assume zero knowledge of trigonometry as well. \\

We are attempting to find the formula
for a function $R(\theta)$,
such that the function takes
a vector $u$ as input,
and spits out a rotated vector $R(\theta)(u)$. \\
We can use differentials, which are infinitessimal
values that make calculations easier
(and allow more hand-waviness to be applied
to small inaccuracies). \\

In order to define a linear transformation,
we generally attempt to apply it to
the basis vectors:
\[ \hat{\imath}  = \vecTwo{1}{0} \AND 
\hat{\jmath} = \vecTwo{0}{1} \]
We then take the result and use them
as column vectors for our transformation
matrix; this is because transforming
from the standard basis to a new basis
just means multiplying by the new basis
matrix. \\

So we want to define a rotation by some
small amount $d\theta$ (infinitessimal). \\
Notice that because the angle is small,
the vector $\hat{\imath}$ would move almost
purely vertically upwards,
while the vector $\hat{\jmath}$
would move almost purely horizontally
to the right. \\
This means that the rotation:
\[ R(d\theta) 
= \matTwo{1}{-d\theta}{d\theta}{1} \]
We can use this matrix to define
a differential equation,
whose result will give us a function
$R(\theta)$ whose output is
the rotation matrix. \\

First note that transforming $u$
by the rotation matrix with a small angle $d\theta$,
boils down to adding a small vector $du$
to $u$ such that $u + du$
is the rotated vector:
\[ u + du = R(d\theta)u  \]
We can thus define the differential equation:
\[ du = R(d\theta)u - u \]
\[ du = (R(d\theta) - I)u \]
\[ du = R(d\theta)u - u \]
\[ du = \para{\matTwo{1}{-d\theta}{d\theta}{1}
- \matTwo{1}{0}{0}{1}}u \]
\[ du = \matTwo{0}{-d\theta}{d\theta}{0}u \]
\[ \dfrac{du}{d\theta} 
= \matTwo{0}{-1}{1}{0}u \]
So this is our differential equation. \\
Solving it should give us the $R(\theta)$
that we defined earlier. \\

\end{document}