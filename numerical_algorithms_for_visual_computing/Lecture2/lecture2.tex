
\documentclass[12pt]{article}

\usepackage[margin=1in]{geometry}


%===============================================================================
%================================== PACKAGES ===================================
%===============================================================================

% For using float option H that places figures 
% exatcly where we want them
\usepackage{float}
% makes figure font bold
\usepackage{caption}
\captionsetup[figure]{labelfont=bf}
% For text generation
\usepackage{lipsum}
% For drawing
\usepackage{tikz}
% For smaller or equal sign and not divide sign
\usepackage{amssymb}
% For the diagonal fraction
\usepackage{xfrac}
% For enumerating exercise parts with letters instead of numbers
\usepackage{enumitem}
% For dfrac, which forces the fraction to be in display mode (large) e
% even in math mode (small)
\usepackage{amsmath}
% For degree sign
\usepackage{gensymb}
% For "\mathbb" macro
\usepackage{amsfonts}
% For positioning 
\usepackage{indentfirst}
\usetikzlibrary{shapes,positioning,fit,calc}
% for adjustwidth environment
\usepackage{changepage}
% for arrow on top
\usepackage{esvect}
% for mathbb 1
\usepackage{bbm}
% for mathsrc
\usepackage[mathscr]{eucal}
% For degree sign
\usepackage{gensymb}
% For quotes
\usepackage{csquotes}
% For vertical lines
\usepackage{mathtools}
% For cols
\usepackage{multicol}

% for tikz
\usepackage{pgfplots}
\pgfplotsset{compat=1.18}
\usepackage{amsmath}
\usepgfplotslibrary{groupplots}


%===============================================================================
%==================================== FONTS ====================================
%===============================================================================


% Mathcal
\newcommand{\acal}{\mathcal{A}}
\newcommand{\bcal}{\mathcal{B}}
\newcommand{\ccal}{\mathcal{C}}
\newcommand{\dcal}{\mathcal{D}}
\newcommand{\ecal}{\mathcal{E}}
\newcommand{\fcal}{\mathcal{F}}
\newcommand{\gcal}{\mathcal{G}}
\newcommand{\hcal}{\mathcal{H}}
\newcommand{\ical}{\mathcal{I}}
\newcommand{\jcal}{\mathcal{J}}
\newcommand{\kcal}{\mathcal{K}}
\newcommand{\lcal}{\mathcal{L}}
\newcommand{\mcal}{\mathcal{M}}
\newcommand{\ncal}{\mathcal{N}}
\newcommand{\ocal}{\mathcal{O}}
\newcommand{\pcal}{\mathcal{P}}
\newcommand{\qcal}{\mathcal{Q}}
\newcommand{\rcal}{\mathcal{R}}
\newcommand{\scal}{\mathcal{S}}
\newcommand{\tcal}{\mathcal{T}}
\newcommand{\ucal}{\mathcal{U}}
\newcommand{\vcal}{\mathcal{V}}
\newcommand{\wcal}{\mathcal{W}}
\newcommand{\xcal}{\mathcal{X}}
\newcommand{\ycal}{\mathcal{Y}}
\newcommand{\zcal}{\mathcal{Z}}

% Mathfrak
\newcommand{\afrak}{\mathfrak{A}}
\newcommand{\bfrak}{\mathfrak{B}}
\newcommand{\cfrak}{\mathfrak{C}}
\newcommand{\dfrak}{\mathfrak{D}}
\newcommand{\efrak}{\mathfrak{E}}
\newcommand{\ffrak}{\mathfrak{F}}
\newcommand{\gfrak}{\mathfrak{G}}
\newcommand{\hfrak}{\mathfrak{H}}
\newcommand{\ifrak}{\mathfrak{I}}
\newcommand{\jfrak}{\mathfrak{J}}
\newcommand{\kfrak}{\mathfrak{K}}
\newcommand{\lfrak}{\mathfrak{L}}
\newcommand{\mfrak}{\mathfrak{M}}
\newcommand{\nfrak}{\mathfrak{N}}
\newcommand{\ofrak}{\mathfrak{O}}
\newcommand{\pfrak}{\mathfrak{P}}
\newcommand{\qfrak}{\mathfrak{Q}}
\newcommand{\rfrak}{\mathfrak{R}}
\newcommand{\sfrak}{\mathfrak{S}}
\newcommand{\tfrak}{\mathfrak{T}}
\newcommand{\ufrak}{\mathfrak{U}}
\newcommand{\vfrak}{\mathfrak{V}}
\newcommand{\wfrak}{\mathfrak{W}}
\newcommand{\xfrak}{\mathfrak{X}}
\newcommand{\yfrak}{\mathfrak{Y}}
\newcommand{\zfrak}{\mathfrak{Z}}

% Mathscr
\newcommand{\ascr}{\mathscr{A}}
\newcommand{\bscr}{\mathscr{B}}
\newcommand{\cscr}{\mathscr{C}}
\newcommand{\dscr}{\mathscr{D}}
\newcommand{\escr}{\mathscr{E}}
\newcommand{\fscr}{\mathscr{F}}
\newcommand{\gscr}{\mathscr{G}}
\newcommand{\hscr}{\mathscr{H}}
\newcommand{\iscr}{\mathscr{I}}
\newcommand{\jscr}{\mathscr{J}}
\newcommand{\kscr}{\mathscr{K}}
\newcommand{\lscr}{\mathscr{L}}
\newcommand{\mscr}{\mathscr{M}}
\newcommand{\nscr}{\mathscr{N}}
\newcommand{\oscr}{\mathscr{O}}
\newcommand{\pscr}{\mathscr{P}}
\newcommand{\qscr}{\mathscr{Q}}
\newcommand{\rscr}{\mathscr{R}}
\newcommand{\sscr}{\mathscr{S}}
\newcommand{\tscr}{\mathscr{T}}
\newcommand{\uscr}{\mathscr{U}}
\newcommand{\vscr}{\mathscr{V}}
\newcommand{\wscr}{\mathscr{W}}
\newcommand{\xscr}{\mathscr{X}}
\newcommand{\yscr}{\mathscr{Y}}
\newcommand{\zscr}{\mathscr{Z}}

% Mathbb
\newcommand{\abb}{\mathbb{A}}
\newcommand{\bbb}{\mathbb{B}}
\newcommand{\cbb}{\mathbb{C}}
\newcommand{\dbb}{\mathbb{D}}
\newcommand{\ebb}{\mathbb{E}}
\newcommand{\fbb}{\mathbb{F}}
\newcommand{\gbb}{\mathbb{G}}
\newcommand{\hbb}{\mathbb{H}}
\newcommand{\ibb}{\mathbb{I}}
\newcommand{\jbb}{\mathbb{J}}
\newcommand{\kbb}{\mathbb{K}}
\newcommand{\lbb}{\mathbb{L}}
\newcommand{\mbb}{\mathbb{M}}
\newcommand{\nbb}{\mathbb{N}}
\newcommand{\obb}{\mathbb{O}}
\newcommand{\pbb}{\mathbb{P}}
\newcommand{\qbb}{\mathbb{Q}}
\newcommand{\rbb}{\mathbb{R}}
\newcommand{\sbb}{\mathbb{S}}
\newcommand{\tbb}{\mathbb{T}}
\newcommand{\ubb}{\mathbb{U}}
\newcommand{\vbb}{\mathbb{V}}
\newcommand{\wbb}{\mathbb{W}}
\newcommand{\xbb}{\mathbb{X}}
\newcommand{\ybb}{\mathbb{Y}}
\newcommand{\zbb}{\mathbb{Z}}


%===============================================================================
%=============================== SPECIAL SYMBOLS ===============================
%===============================================================================


% Orbit (group theory)
\newcommand{\orbit}{\mathcal{O}}
% Normal group
\newcommand{\normal}{\mathcal{N}}
% Indicator function
\newcommand{\indicator}{\mathbbm{1}}
% Laplace transform
\newcommand{\laplace}[1]{\mathcal{L}}
% Epsilon shorthand
\newcommand{\eps}{\varepsilon}
% Omega
\newcommand{\om}{\omega}
\newcommand{\Om}{\Omega}

%===============================================================================
%================================== OPERATORS ==================================
%===============================================================================


% Inverse exponent
\newcommand{\inv}[0]{^{-1}}
% Overline bar
\newcommand{\olsi}[1]{\,\overline{\!{#1}}}
% Less than or equal slanted
\newcommand{\seqs}{\leqslant}
% Greater or equal slanted
\newcommand{\geqs}{\geqslant}
% Subset or equal
\newcommand{\sub}{\subseteq}
% Proper subset
\newcommand{\prosub}{\subset}
% from
\newcommand{\from}{\leftarrow}

% Parantheses
\newcommand{\para}[1]{\left( #1 \right)}
% Curly Braces
\newcommand{\curl}[1]{\left\{ #1 \right\}}
% Brackets
\newcommand{\brac}[1]{\left[ #1 \right]}
% Angled Brackets
\newcommand{\ang}[1]{\left\langle #1 \right\rangle}
% Norm
\newcommand{\norm}[1]{\left\| #1 \right\|}

% Piece wise (use \\ between cases)
\newcommand{\piecewise}[1]{\begin{cases} #1 \end{cases}}

% Bold symbol shorthand
\newcommand{\bl}{\boldsymbol}

% Vertical space
\newcommand{\vs}[1]{\vspace{#1 pt}}
% Horizontal ertical space
\newcommand{\hs}[1]{\hspace{#1 pt}}


%===============================================================================
%============================== TEXT BASED SYMBOLS =============================
%===============================================================================


% Radians
\newcommand{\rad}{\text{rad}}
% Least Common Multiple
\newcommand{\lcm}{\text{lcm}}
% Automorphism
\newcommand{\Aut}{\text{Aut}}
% Variance
\newcommand{\var}{\text{Var}}
% Covariance
\newcommand{\cov}{\text{Cov}}
% Cofactor (matrix)
\newcommand{\cof}{\text{Cof}}
% Adjugate (matrix)
\newcommand{\adj}{\text{Adj}}
% Trace (matrix)
\newcommand{\tr}{\text{tr}}
% Standard deviation
\newcommand{\std}{\text{Std}}
% Correlation coefficient
\newcommand{\corr}{\text{Corr}}
% Sign
\newcommand{\sign}{\text{sign}}

% And text
\newcommand{\AND}{\text{ and }}
% Or text 
\newcommand{\OR}{\text{ or }}
% If text
\newcommand{\IF}{\text{ if }}
% When text
\newcommand{\WHEN}{\text{ when }}
% Then text
\newcommand{\THEN}{\text{ then }}

% 1st
\newcommand{\st}[1]{#1^{\text{st}}}
% 2nd
\newcommand{\nd}[1]{#1^{\text{nd}}}
% 3rd
\newcommand{\rd}[1]{#1^{\text{rd}}}
% nth
\newcommand{\nth}[1]{#1^{\text{th}}}


%===============================================================================
%========================= PROBABILITY AND STATISTICS ==========================
%===============================================================================


% Permutation
\newcommand{\perm}[2]{{}^{#1}\!P_{#2}}
% Combination
\newcommand{\comb}[2]{{}^{#1}C_{#2}}

% Baye's risk
\newcommand{\risk}[1]{\mathscr{R}_{#1}}
% Baye's optimal risk
\newcommand{\riskOptimal}[1]{\mathscr{R}_{#1}^*}
% Baye's empirical risk
\newcommand{\riskEmpirical}[2]{\hat{\mathscr{R}}_{#1}^{#2}}


%===============================================================================
%=================================== CALCULUS ==================================
%===============================================================================


% d over d derivative
\newcommand{\dd}[2]{\dfrac{d#1}{d#2}}
% partial d over d derivative
\newcommand{\partialdd}[2]{\dfrac{\partial #1}{\partial #2}}
% delta d over d derivative
\newcommand{\deltadd}[2]{\dfrac{\Delta #1}{\Delta #2}}

% Integration between a and b
\newcommand{\integral}[4]{\int_{#1}^{#2} #3 \, #4}
% Integration in some space 
\newcommand{\boundIntegral}[2]{\int_{#1} #2 \, d#1}

% Limit
\newcommand{\limit}[3]{\lim_{#1 \to #2} #3}


%===============================================================================
%================================  BIG SYMBOLS  ================================
%===============================================================================


% Sum
\newcommand{\sumof}[2]{\sum_{#1}^{#2}}
% Product
\newcommand{\productof}[2]{\prod_{#1}^{#2}}
% Union
\newcommand{\unionof}[2]{\bigcup_{#1}^{#2}}
% Intersection
\newcommand{\intersectionof}[2]{\bigcap_{#1}^{#2}}
% Or
\newcommand{\orof}[2]{\bigvee_{#1}^{#2}}
% And
\newcommand{\andof}[2]{\bigwedge_{#1}^{#2}}


%===============================================================================
%=============================== LINEAR ALGEBRA ================================
%===============================================================================


% Bold vector arrow
\newcommand{\bv}[1]{\vec{\mathbf{#1}}}

% Matrix or vector (use // for column, & for row) 
% with brackets
\newcommand{\bmat}[1]{\begin{bmatrix} #1 \end{bmatrix}}
% Matrix or vector (use // for column, & for row) 
% with curved brackets
\newcommand{\pmat}[1]{\begin{pmatrix} #1 \end{pmatrix}}
% Matrix or vector (use // for column, & for row) 
% with lines on either side 
\newcommand{\lmat}[1]{\begin{vmatrix} #1 \end{vmatrix}}
% Matrix or vector (use // for column, & for row) 
% with curly braces
\newcommand{\cmat}[1]{\begin{Bmatrix} #1 \end{Bmatrix}}
% Matrix or vector (use // for column, & for row) 
% with no braces
\newcommand{\mat}[1]{\begin{matrix} #1 \end{matrix}}


%===============================================================================
%================================ LARGE OBJECTS ================================
%===============================================================================

% Multiple lines
\newcommand{\multiline}[1]{
\begin{align*}
    #1
\end{align*}
}

% Multiple lines with equation numbers
\newcommand{\eqmultiline}[1]{
\begin{align*}
    #1
\end{align*}
}

% Color
\newcommand{\colorText}[2]{
\begingroup
\color{#1}
    #2
\endgroup
}

% Centered figure
\newcommand{\centerFigure}[2]{
    \begin{figure}[h]
        \centering
            #1
        \caption{#2}
    \end{figure}
}

% Tikz figure
\newcommand{\tikzGraphic}[1]{
    \begin{center}
    \begin{tikzpicture}
        #1
    \end{tikzpicture}
    \end{center}
}

% Enumerate numbers (seperate by \item)
\newcommand{\numbers}{\textbf{\number*)}}

% Enumerate letters (seperate by \item)
\newcommand{\letters}{\textbf{\alph*)}}

\title{
    \Huge Numerical Algorithms \\
    \Large Lecture II
}
\date{2025-04-16}
\author{Michael Saba}

\begin{document}
\pagenumbering{gobble}
\maketitle
\newpage
\setlength{\parindent}{0pt}
\pagenumbering{arabic}

\subsection*{Vector Space Bases}

Given a vector space $V$ over $\rbb$,
a subset $B$ of $V$
\[ B = \{v_1, v_2, \dots v_n \} \]
is called a basis of $V$ as long as the
vectors $v_1, v_2, \dots v_n$
are linearly independent,
and if every vector in $V$ could be written
as a linear combination of the vectors in $B$. \\
These two properties allow that each vector
can be expressed
in terms of the basis vector uniquely,
giving us both a uniqueness
and an existence property. \\

A set of vectors who do not form a basis are
called a frame (if existence is not fulfilled),
or $k-$frames (if uniqueness is not fulfilled). \\

We can think of the basis as the coordinate
system of the vector space. \\
So the entries of a vector expressed in a
certain basis refer to the scalar factors
that have to be applied to the basis vectors
such that their linear combinaton produces
the vector. \\
By default, all vectors are written in the standard
basis, which in $\rbb^2$ means the vectors:
\[ \hat{i} = \bmat{{1} \\ {0}} \qquad \text{ and }
\qquad \hat{j} = \bmat{{0} \\ {1}} \]
So, when we have a vector:
\[ \bmat{{2}\\{5}} \]
it can be reached using the linear combination:
\[ \bmat{{2}\\{5}} = 2 \bmat{{1}\\{0}} 
+ 5 \bmat{{0}\\{1}} \]
The basis vectors are essentially the reference
point using which we define all other vectors. \\

More compactly, we can express a basis using
a matrix (where the columns are basis vectors). \\
This is useful as it allows us to express
vectors in a basis using matrix multiplication:
\[ \bmat{{2}\\{5}} = 2 \bmat{{1}\\{0}} 
+ 5 \bmat{{0}\\{1}}
= \bmat{{1}&{0}\\{0}&{1}} \bmat{{5}\\{2}} \] \\

Now suppose we have a new basis
\[ A = \bmat{{a_1^1} & {a_2^1} \\ {a_1^2} & {a_2^2}} \]
If we have a vector 
\[ v = \bmat{{5}\\{2}} \]
expressed in the basis $A$,
then the vector can be formed using
\[ \bmat{{5}\\{2}} = 5 \bmat{{a_1^1}\\{a_1^2}}
+ 2 \bmat{{a_2^1}\\{a_2^2}} \]
which gives us the coordinates of the 
vector in the standard basis. \\
In other words, multiplying a vector
by a basis matrix
gives us the coordinates of the vector
expressed in the standard basis
has the original vector been
expressed in said basis. \\

Now, here is what happens if we multiply by 
the basis, then multiply by its inverse 
(which we know exists since basis column vectors
are by definition linearly independent):
\[ A\inv Av = 
\bmat{{a_1^1} & {a_2^1} \\{a_1^2}& {a_2^2}} \inv
\para{\bmat{{a_1^1} & {a_2^1} \\ {a_1^2} \\ {a_2^2}}
\bmat{{5}\\{2}}}
= \bmat{{5}\\{2}} \]
We know that $v$ is the vector
expressed in the standard basis,
and that $Av$ is the value of the vector
expressed in the standard basis,
has $v$ been expressed in the basis $A$. \\
We know that multiplying the vector $Av$ by $A\inv$
returns its value to $v$. \\ 
So multiplying by the inverse $A\inv$
has the effect of taking a vector
in the standard basis,
and giving us the value that same vector
would need to have to point to the same coordinate
it did, but expressed in the new basis. \\

Another way to think about this is that
multiplying by an inverse $A\inv v$
is the same as solving $Ax = v$,
and as we know,
solving a linear system
means finding a linear combination 
of the columns of $A$ that match $v$. \\
This is the same as finding the coordinates
a vector would need to have in the new basis
in order to reach the same coordinate $v$
does in the standard basis, which is $A\inv v$. \\

So, to conclude:
\[ Av \]
is a vector expressed in the standard basis
that points to the same coordinate as $v$
would if it had been a vector of basis $A$. \\
On the other hand,
\[ A\inv v \]
is the value a vector would need to have
in the new basis $A$ in order to point to the same
coordinate as $v$ does in the standard basis. \\

Determinants represent the scaling factor
of a linear transformation. \\
It thus makes sense why a singular matrix,
with determinant 0,
can't form a basis.
Having a scaling factor of 0 implies
the space is flattened onto a lower dimension,
like a plane being turned into a line. \\

As mentioned in the first lecture,
we can think of linear systems as
itersections of hyperplanes or linear
combinations of vectors. \\
The latter of the two can now be formalized
within the context of bases. \\
Solving a linear system $Ax = b$ entails finding
the coordinates vector $b$
in the basis $A$. \\
So if $A$ is invertible (linearly independent
column vectors),
then $A$ is a basis proper, with uniqueness
and existence properties for any vector in $\rbb^n$,
which means we will have on unique solution. \\
On the other hand, if the $A$ is not a basis proper,
then the matrix flattens the original span
onto a lower dimension 
(like flattening $\rbb^2$ onto a line).
It's possible then that $b$ is not in the new
span, which means no solution exists,
or that $b$ is in that space, but reachable
in more than one way (since we have colinear
or coplanar vectors). \\

\newpage

\subsection*{More Plane Intersections}

We've seen planes expressed as implicit
functions. \\
Now to express a plane as a parametric function,
we can mimic the parametric line formula. \\
First, we note that a plane is two-dimensional,
so we need to linearlly combine two vectors
in order to span all of it
and reach each point on it
(as explained in the Basis section):
\[ \bv{r}(\alpha, \beta) 
= \alpha \bv{f_1} + \beta \bv{f_2} \]
where $\bv{f_1}$ and $\bv{f_2}$
are any two linearlly independent vectors
on the plane. \\
This gives us the orientation of the plane,
but we still need a point $\bv{q}$ on the plane
in order to fix the plane in space.
We can call this point the offset
(similar to the line equation):
\[ \bv{r}(\alpha, \beta) 
= \alpha \bv{f_1} + \beta \bv{f_2} + \bv{q} \]
This equation defined a plane using
two parameters $\alpha$ and $\beta$
which can be any real value. \\

We already saw the case where we intersect
a line by plane using the implicit plane
equation. \\
Now, if we have a parametric plane equation
and a parametric line equation,
how can we check where and if they intersect? \\
If we have two parametric equations:
\[ \bv{r_1}(t_1) \qquad \text{ and }
\qquad \bv{r_2}(t_2) \]
all we would need to do to find the intersection
is set
\[ \bv{r_1}(t_1) = \bv{r_2}(t_2) \]
and solve for $t_1$ and $t_1$. \\
So, for a plane and a line, we solve
\[ \alpha \bv{f_1} + \beta \bv{f_2} + \bv{q}
= \bv{o} + t\bv{d} \]
for $\alpha$, $\beta$, and $t$. \\
We can do the following:
\[ \alpha \bv{f_1} + \beta \bv{f_2} - t\bv{d}
= \bv{o} - \bv{q} \]
which turns the equation into a linear system:
\[ 
    \bmat{{f_1^1}&{f_2^1}&{-d^1}\\
    {f_1^2 }&{f_2^2}&{-d^2}\\
    {f_1^3}&{f_2^3}&{-d^3}}
    \bmat{{\alpha}\\{\beta}\\{t}}
    = \bmat{{o^1 - q^1}\\{o^2 - q^2}\\{o^3 - q^3}} 
\]
Solving this entails finding
$\alpha$, $\beta$, and $t$. \\

Recall that when we had a list of implicit equations
hyperplanes,
finding an intersection also entailed
solving a linear system.
This will always be the case when we have
an intersection of linear shapes. \\
On the other hand, inetrsecting a linear shape
(line) with a quadratic shape (sphere)
entailed solving a quadratic equation
(degree 1 $\times$ degree 2). \\
If we had two quadratic shapes we wanted to 
find the intersection of (e.g. parabola and sphere),
we would then need to solve a quartic equation
(degree 2 $\times$ degree 2). \\

\newpage

\subsection*{Triangle Intersection}

All polygons in computer graphics are
rendered using triangles,
as they are simple,
and as any polygon can be divided into triangles. \\

We will now look at how to check if a line
intersects a triangle in $\rbb^3$. \\

A triangle resides on a plane,
and can be defined by three points
which we will call $\bv{v_0}$, $\bv{v_1}$,
and $\bv{v_3}$. \\
The check for a line intersection,
we can first find the plane on which the triangle
resides,
and check for an intersection between the line
and the plane. \\
If such a point $\bv{p_0}$ exists,
we can then check to see if the point is
inside the triangle. \\

First, given the points $\bv{v_0}$, $\bv{v_1}$,
and $\bv{v_3}$,
we want to define a plane. \\

First, for a parametric equation. \\
We know the parametric equation of a plane is:
\[ \bv{r}(\alpha, \beta) 
= \alpha \bv{f_1} + \beta \bv{f_2} + \bv{q} \]
where $\bv{f_1}$ and $\bv{f_2}$
are any two linearlly independent vectors
on the plane,
and $\bv{q}$ is a point on the plane. \\
If we were to set
\[ \bv{q} = \bv{v_0} \] 
and 
\[ \bv{f_1} = \bv{v_1} - \bv{v_0}   \]
\[ \bv{f_2} = \bv{v_2} - \bv{v_0}   \]
then we will have fulfilled the conditions
(here, $\bv{f_1}$ and $\bv{f_2}$ 
are the vector jutting out of $\bv{v_0}$
towards $\bv{v_1}$ and $\bv{v_2}$). \\
We can thus define
\[ \bv{r}(\alpha, \beta) 
= \alpha (\bv{v_1} - \bv{v_0})
+ \beta (\bv{v_2} - \bv{v_0}) + \bv{v_0} \]
as parametric equation of the plane
on which the triangle resides. \\

Now for an implicit equation. \\
We know a plane can be written as the implicit
equation 
\[ \bv{n} \cdot (\bv{p} - \bv{q}) = 0 \]
where $\bv{n}$ is the normal
and $\bv{q}$ is any point on the plane. \\
We can use the parametric equation as a starting
point:
to get the normal, we can use the cross
product of any two non-colinear vectors
on the plane (only works in $\rbb^3$),
which returns a vector orthogonal to the plane
defined by the two vectors.
We can use $\bv{f_1} \times \bv{f_2}$
for this purpose. \\
Then we can use $\bv{v_0}$ for the point $q$. \\
Thus:
\[ ((\bv{v_1} - \bv{v_0}) \times (\bv{v_2} - \bv{v_0}))
\cdot (\bv{p} - \bv{v_0}) = 0 \]
is the implicit plane equation on which the
triangle resides. \\

Once we have checked for an intersection with the
plane,
and gotten an intersection point $\bv{p_0}$
on the plane,
we need to check to see if the point is in the
triangle. \\
Fortunately, this is not so hard to do. \\
Since $\bv{f_1}$ and $\bv{f_2}$
are linearlly independent,
they form a basis that spans the whole plane,
with an origin at $\bv{q}$. \\
We can thus express $\bv{p_0}$ in this
new basis (as a linear combination
of $\bv{f_1}$ and $\bv{f_2}$). \\
The four points
\[ \bmat{{0}\\{0}} \qquad
\bmat{{0}\\{1}} \qquad
\bmat{{1}\\{1}} \qquad
\bmat{{1}\\{0}} \qquad \]
expressed in this new basis
define a parallelogram
(whose area is the determinant of the basis).
As we know, the vectors
\[ \bmat{{0}\\{0}} \qquad
\bmat{{0}\\{1}} \qquad
\bmat{{1}\\{0}} \qquad \]
in the new basis correspond to
$\bv{v_0}$, $\bv{v_1}$, and $\bv{v_3}$.
So in the parametric equation
\[ \bv{r}(\alpha, \beta) 
= \alpha (\bv{v_1} - \bv{v_0})
+ \beta (\bv{v_2} - \bv{v_0}) + \bv{v_0} \]
we would need
\[ \alpha + \beta < 1 \]
in order to for a point to be inside the triangle. \\

So in short, we intersect the plane by a line
using the technique highlighted in the last section
(parametric equation of line and plane),
and then check tho make sure that our solution
satisfies $\alpha + \beta < 1$. \\

We still need to clarify one thing:
how can we express the point $\bv{p_0}$
that we get from the intersection
in the basis defined by $\bv{f_1}$ and $\bv{f_2}$? \\
As we said earlier,
multiplying a vector $v$ by the inverse 
of the matrix
whose columns are basis vectors
gives us the coordinates
of that same vector in said basis:
\[ \bmat{{\bv{f_1}}&{\bv{f_2}}}\inv
\bv{p_0} = \bmat{{\alpha}\\{\beta}} \]
which is the same as saying we need to 
solve the linear system
\[ \bmat{{\bv{f_1}}&{\bv{f_2}}}
\bmat{{\alpha} \\ {\beta}}
= \bv{p_0} \]
The only issue here is that
\[\bmat{{\bv{f_1}} & {\bv{f_2}}}\]
is not square.
It is a $3 \times 2$ matrix. \\
One solution to that is to multiply both sides
by the transpose:
\[ \bmat{{\bv{f_1}}\\{\bv{f_2}}}
\bmat{{\bv{f_1}} & {\bv{f_2}}}
\bmat{{\alpha} \\ {\beta}}
= \bmat{{\bv{f_1}} \\ {\bv{f_2}}} \bv{p_0} \]
This gives us a linear system with a $2 \times 2$
square matrix that we can solve. \\

While we assume that $\bv{p_0}$ is on the plane,
since it is the result of intersecting a line
and plane,
it is worth noting that the equation still yields
a solution even if $\bv{p_0}$ were not on the plane
(since the matrix is not singular). \\
In that case, the coordinates we get are those
of the projection of $\bv{p_0}$ onto the plane,
expressed in the new basis. \\

This happening makes sense when you consider
what the matrix
$\bmat{{\bv{f_1}} & {\bv{f_2}}}$
does when it transform $\bv{p_0}$:
\[ \begin{bmatrix}
    f_1^1 & f_1^2 & f_1^3 \\
    f_2^1 & f_2^2 & 2_1^3
\end{bmatrix} \bmat{{p_0^1}\\{p_0^2}\\{p_0^3}} \]
It is a $2 \times 3$ matrix,
meaning that it flattens $\rbb^3$
space onto a plane. \\
In this case in particular,
it projects $\bv{p_0}$ onto the plane
defined by the basis
$\bmat{{\bv{f_1}} & {\bv{f_2}}}$. \\

As we know, solving a linear system
is the equivalent of multiplying by the inverse
of a matrix. \\
In order to get the coordinates of $\bv{p_0}$
in the basis $\bmat{{\bv{f_1}} & {\bv{f_2}}}$,
we can just multiply iy by the inverse
of that basis. \\
However, $\bmat{{\bv{f_1}} & {\bv{f_2}}}$
is not square, so instead, we do the following:
\[ \bmat{{\bv{f_1}} \\ {\bv{f_2}}}
\bmat{{\bv{f_1}} & {\bv{f_2}}}
\bmat{{\alpha}\\{\beta}}
= \bmat{{\bv{f_1}}\\{\bv{f_2}}}\bv{p_0} \]
\[ \bmat{{\alpha} \\ {\beta}}
= \para{\bmat{{\bv{f_1}}\\{\bv{f_2}}}
\bmat{{\bv{f_1}}&{\bv{f_2}}}}\inv
\bmat{{\bv{f_1}} \\ {\bv{f_2}}}\bv{p_0} \]
and solve to get $\alpha$ and $\beta$,
the coordinates of $\bv{p_0}$
in the new basis. \\

\newpage

\subsection*{Intersection with a Quadric}

Quadrics are the $\rbb^3$ analogue of conics in $\rbb^2$. \\
Conics form the family of curves you get if you intersect
a cone by a plane.
That includes circles, ellipses, parabolae,
and hyperbolae. \\
In $\rbb^2$, the general formula for a conic curve is
\[ Ax^2 + By^2 + Cxy + Dx + Ey + F = 0 \]
This formula can be extended to $R^3$:
\[ Ax^2 + By^2 + Cz^2 + Dxy + Exz + Fyz +
Gx + Hy + Iz + J = 0 \]
where it can represent spheres, ellipsoids,
paraboloids, and hyperboloids, saddles... \\

We can rewrite this formula using vector notation:
\[ (\bv{p})^T \boldsymbol{Q}\bv{p} 
+ (\bv{b})^T\bv{p} + J = 0 \]
where
\[ 
    \boldsymbol{Q} = 
    \bmat{{A} & {\sfrac{D}{2}} & {\sfrac{E}{2}} \\
    {\sfrac{D}{2}} & {B} & {\sfrac{F} {2}} \\ 
    {\sfrac{E}{2}} & {\sfrac{F}{2}} & {C}}
    \qquad b = \bmat{{G} \\ {H} \\{I}}
\]
Expanding the formula shows it is the same. \\

To check if a ray intersects a quadric,
we can solve this equation for $t$:
\[ (\bv{o} + t\bv{d})^T \boldsymbol{Q} (\bv{o} + t\bv{d})
    + \bv{b}^T\bv{p} + J = 0 \]
It will also lead to a quadratic, 
where having one or more positive solution
means the ray intersects the quadric. \\

For example,
suppose we have the quadric surface:
\[ 3(x-5)^2 - 2(y-1)^2 -z^2 + 1 = 0 \]
and a ray
\[ \bmat{{1} \\{0} \\{-1}}t \]
Then we can first develop the formula of the quadric:
\[ 3x^2 - 30x - 2y^2 +4y -z^2 + 74 = 0 \]
where
\[ 
    A = 3 \qquad B = -2 \qquad C = -1 \qquad G = -30
    \qquad H = 4  \qquad J = 74
\]
This can then be written in vector notation:
\[
    (\bv{p})^T
    \bmat{{3}&{0}&{0} \\ {0}&{-2}&{0} \\ {0}&{0}&{-30}}
    \bv{p}
    + \bmat{{-39}&{4}&{0}}
    \bv{p}
    + 74 = 0
\]
Then, we can replace $\bv{p}$ by the ray:
\[
    \bmat{{t}&{0}&{-t}}
    \bmat{{3}&{0}&{0}\\{0}&{-2}&{0}\\{0}&{0}&{-30}}
    \bmat{{t}\\{0}\\{-t}}
    + \bmat{{-30}&{4}&{0}}
    \bmat{{t}\\{0}\\{-t}}
    + 74 = 0
\]
\[
    \bmat{{t}&{0}&{-t}}
    \bmat{{3t}\\{0}\\{30t}}
    - 30t + 74 = 0
\]
\[
    -27t^2 - 30t + 74 = 0                                                                                                                                                                                                                                               
\]
We can then solve for $t$:
\[ t = -\dfrac{30 \pm \sqrt{900 + 7992}}{60} \]


 
\end{document}