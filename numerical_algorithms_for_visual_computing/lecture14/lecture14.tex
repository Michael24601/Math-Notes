


\documentclass[12pt]{article}

\usepackage[margin=1in]{geometry}


%===============================================================================
%================================== PACKAGES ===================================
%===============================================================================

% For using float option H that places figures 
% exatcly where we want them
\usepackage{float}
% makes figure font bold
\usepackage{caption}
\captionsetup[figure]{labelfont=bf}
% For text generation
\usepackage{lipsum}
% For drawing
\usepackage{tikz}
% For smaller or equal sign and not divide sign
\usepackage{amssymb}
% For the diagonal fraction
\usepackage{xfrac}
% For enumerating exercise parts with letters instead of numbers
\usepackage{enumitem}
% For dfrac, which forces the fraction to be in display mode (large) e
% even in math mode (small)
\usepackage{amsmath}
% For degree sign
\usepackage{gensymb}
% For "\mathbb" macro
\usepackage{amsfonts}
% For positioning 
\usepackage{indentfirst}
\usetikzlibrary{shapes,positioning,fit,calc}
% for adjustwidth environment
\usepackage{changepage}
% for arrow on top
\usepackage{esvect}
% for mathbb 1
\usepackage{bbm}
% for mathsrc
\usepackage[mathscr]{eucal}
% For degree sign
\usepackage{gensymb}
% For quotes
\usepackage{csquotes}
% For vertical lines
\usepackage{mathtools}
% For cols
\usepackage{multicol}

% for tikz
\usepackage{pgfplots}
\pgfplotsset{compat=1.18}
\usepackage{amsmath}
\usepgfplotslibrary{groupplots}


%===============================================================================
%==================================== FONTS ====================================
%===============================================================================


% Mathcal
\newcommand{\acal}{\mathcal{A}}
\newcommand{\bcal}{\mathcal{B}}
\newcommand{\ccal}{\mathcal{C}}
\newcommand{\dcal}{\mathcal{D}}
\newcommand{\ecal}{\mathcal{E}}
\newcommand{\fcal}{\mathcal{F}}
\newcommand{\gcal}{\mathcal{G}}
\newcommand{\hcal}{\mathcal{H}}
\newcommand{\ical}{\mathcal{I}}
\newcommand{\jcal}{\mathcal{J}}
\newcommand{\kcal}{\mathcal{K}}
\newcommand{\lcal}{\mathcal{L}}
\newcommand{\mcal}{\mathcal{M}}
\newcommand{\ncal}{\mathcal{N}}
\newcommand{\ocal}{\mathcal{O}}
\newcommand{\pcal}{\mathcal{P}}
\newcommand{\qcal}{\mathcal{Q}}
\newcommand{\rcal}{\mathcal{R}}
\newcommand{\scal}{\mathcal{S}}
\newcommand{\tcal}{\mathcal{T}}
\newcommand{\ucal}{\mathcal{U}}
\newcommand{\vcal}{\mathcal{V}}
\newcommand{\wcal}{\mathcal{W}}
\newcommand{\xcal}{\mathcal{X}}
\newcommand{\ycal}{\mathcal{Y}}
\newcommand{\zcal}{\mathcal{Z}}

% Mathfrak
\newcommand{\afrak}{\mathfrak{A}}
\newcommand{\bfrak}{\mathfrak{B}}
\newcommand{\cfrak}{\mathfrak{C}}
\newcommand{\dfrak}{\mathfrak{D}}
\newcommand{\efrak}{\mathfrak{E}}
\newcommand{\ffrak}{\mathfrak{F}}
\newcommand{\gfrak}{\mathfrak{G}}
\newcommand{\hfrak}{\mathfrak{H}}
\newcommand{\ifrak}{\mathfrak{I}}
\newcommand{\jfrak}{\mathfrak{J}}
\newcommand{\kfrak}{\mathfrak{K}}
\newcommand{\lfrak}{\mathfrak{L}}
\newcommand{\mfrak}{\mathfrak{M}}
\newcommand{\nfrak}{\mathfrak{N}}
\newcommand{\ofrak}{\mathfrak{O}}
\newcommand{\pfrak}{\mathfrak{P}}
\newcommand{\qfrak}{\mathfrak{Q}}
\newcommand{\rfrak}{\mathfrak{R}}
\newcommand{\sfrak}{\mathfrak{S}}
\newcommand{\tfrak}{\mathfrak{T}}
\newcommand{\ufrak}{\mathfrak{U}}
\newcommand{\vfrak}{\mathfrak{V}}
\newcommand{\wfrak}{\mathfrak{W}}
\newcommand{\xfrak}{\mathfrak{X}}
\newcommand{\yfrak}{\mathfrak{Y}}
\newcommand{\zfrak}{\mathfrak{Z}}

% Mathscr
\newcommand{\ascr}{\mathscr{A}}
\newcommand{\bscr}{\mathscr{B}}
\newcommand{\cscr}{\mathscr{C}}
\newcommand{\dscr}{\mathscr{D}}
\newcommand{\escr}{\mathscr{E}}
\newcommand{\fscr}{\mathscr{F}}
\newcommand{\gscr}{\mathscr{G}}
\newcommand{\hscr}{\mathscr{H}}
\newcommand{\iscr}{\mathscr{I}}
\newcommand{\jscr}{\mathscr{J}}
\newcommand{\kscr}{\mathscr{K}}
\newcommand{\lscr}{\mathscr{L}}
\newcommand{\mscr}{\mathscr{M}}
\newcommand{\nscr}{\mathscr{N}}
\newcommand{\oscr}{\mathscr{O}}
\newcommand{\pscr}{\mathscr{P}}
\newcommand{\qscr}{\mathscr{Q}}
\newcommand{\rscr}{\mathscr{R}}
\newcommand{\sscr}{\mathscr{S}}
\newcommand{\tscr}{\mathscr{T}}
\newcommand{\uscr}{\mathscr{U}}
\newcommand{\vscr}{\mathscr{V}}
\newcommand{\wscr}{\mathscr{W}}
\newcommand{\xscr}{\mathscr{X}}
\newcommand{\yscr}{\mathscr{Y}}
\newcommand{\zscr}{\mathscr{Z}}

% Mathbb
\newcommand{\abb}{\mathbb{A}}
\newcommand{\bbb}{\mathbb{B}}
\newcommand{\cbb}{\mathbb{C}}
\newcommand{\dbb}{\mathbb{D}}
\newcommand{\ebb}{\mathbb{E}}
\newcommand{\fbb}{\mathbb{F}}
\newcommand{\gbb}{\mathbb{G}}
\newcommand{\hbb}{\mathbb{H}}
\newcommand{\ibb}{\mathbb{I}}
\newcommand{\jbb}{\mathbb{J}}
\newcommand{\kbb}{\mathbb{K}}
\newcommand{\lbb}{\mathbb{L}}
\newcommand{\mbb}{\mathbb{M}}
\newcommand{\nbb}{\mathbb{N}}
\newcommand{\obb}{\mathbb{O}}
\newcommand{\pbb}{\mathbb{P}}
\newcommand{\qbb}{\mathbb{Q}}
\newcommand{\rbb}{\mathbb{R}}
\newcommand{\sbb}{\mathbb{S}}
\newcommand{\tbb}{\mathbb{T}}
\newcommand{\ubb}{\mathbb{U}}
\newcommand{\vbb}{\mathbb{V}}
\newcommand{\wbb}{\mathbb{W}}
\newcommand{\xbb}{\mathbb{X}}
\newcommand{\ybb}{\mathbb{Y}}
\newcommand{\zbb}{\mathbb{Z}}


%===============================================================================
%=============================== SPECIAL SYMBOLS ===============================
%===============================================================================


% Orbit (group theory)
\newcommand{\orbit}{\mathcal{O}}
% Normal group
\newcommand{\normal}{\mathcal{N}}
% Indicator function
\newcommand{\indicator}{\mathbbm{1}}
% Laplace transform
\newcommand{\laplace}[1]{\mathcal{L}}
% Epsilon shorthand
\newcommand{\eps}{\varepsilon}
% Omega
\newcommand{\om}{\omega}
\newcommand{\Om}{\Omega}

%===============================================================================
%================================== OPERATORS ==================================
%===============================================================================


% Inverse exponent
\newcommand{\inv}[0]{^{-1}}
% Overline bar
\newcommand{\olsi}[1]{\,\overline{\!{#1}}}
% Less than or equal slanted
\newcommand{\seqs}{\leqslant}
% Greater or equal slanted
\newcommand{\geqs}{\geqslant}
% Subset or equal
\newcommand{\sub}{\subseteq}
% Proper subset
\newcommand{\prosub}{\subset}
% from
\newcommand{\from}{\leftarrow}

% Parantheses
\newcommand{\para}[1]{\left( #1 \right)}
% Curly Braces
\newcommand{\curl}[1]{\left\{ #1 \right\}}
% Brackets
\newcommand{\brac}[1]{\left[ #1 \right]}
% Angled Brackets
\newcommand{\ang}[1]{\left\langle #1 \right\rangle}
% Norm
\newcommand{\norm}[1]{\left\| #1 \right\|}

% Piece wise (use \\ between cases)
\newcommand{\piecewise}[1]{\begin{cases} #1 \end{cases}}

% Bold symbol shorthand
\newcommand{\bl}{\boldsymbol}

% Vertical space
\newcommand{\vs}[1]{\vspace{#1 pt}}
% Horizontal ertical space
\newcommand{\hs}[1]{\hspace{#1 pt}}


%===============================================================================
%============================== TEXT BASED SYMBOLS =============================
%===============================================================================


% Radians
\newcommand{\rad}{\text{rad}}
% Least Common Multiple
\newcommand{\lcm}{\text{lcm}}
% Automorphism
\newcommand{\Aut}{\text{Aut}}
% Variance
\newcommand{\var}{\text{Var}}
% Covariance
\newcommand{\cov}{\text{Cov}}
% Cofactor (matrix)
\newcommand{\cof}{\text{Cof}}
% Adjugate (matrix)
\newcommand{\adj}{\text{Adj}}
% Trace (matrix)
\newcommand{\tr}{\text{tr}}
% Standard deviation
\newcommand{\std}{\text{Std}}
% Correlation coefficient
\newcommand{\corr}{\text{Corr}}
% Sign
\newcommand{\sign}{\text{sign}}

% And text
\newcommand{\AND}{\text{ and }}
% Or text 
\newcommand{\OR}{\text{ or }}
% For text 
\newcommand{\FOR}{\text{ for }}
% If text
\newcommand{\IF}{\text{ if }}
% When text
\newcommand{\WHEN}{\text{ when }}
% Where text
\newcommand{\WHERE}{\text{ where }}
% Then text
\newcommand{\THEN}{\text{ then }}
% Such that text
\newcommand{\SUCHTHAT}{\text{ such that }}

% 1st
\newcommand{\st}[1]{#1^{\text{st}}}
% 2nd
\newcommand{\nd}[1]{#1^{\text{nd}}}
% 3rd
\newcommand{\rd}[1]{#1^{\text{rd}}}
% nth
\newcommand{\nth}[1]{#1^{\text{th}}}


%===============================================================================
%========================= PROBABILITY AND STATISTICS ==========================
%===============================================================================


% Permutation
\newcommand{\perm}[2]{{}^{#1}\!P_{#2}}
% Combination
\newcommand{\comb}[2]{{}^{#1}C_{#2}}

% Baye's risk
\newcommand{\risk}[1]{\mathscr{R}_{#1}}
% Baye's optimal risk
\newcommand{\riskOptimal}[1]{\mathscr{R}_{#1}^*}
% Baye's empirical risk
\newcommand{\riskEmpirical}[2]{\hat{\mathscr{R}}_{#1}^{#2}}


%===============================================================================
%=================================== CALCULUS ==================================
%===============================================================================


% d over d derivative
\newcommand{\dd}[2]{\dfrac{d#1}{d#2}}
% partial d over d derivative
\newcommand{\partialdd}[2]{\dfrac{\partial #1}{\partial #2}}
% delta d over d derivative
\newcommand{\deltadd}[2]{\dfrac{\Delta #1}{\Delta #2}}

% Integration between a and b
\newcommand{\integral}[4]{\int_{#1}^{#2} #3 \, #4}
% Integration in some space 
\newcommand{\boundIntegral}[2]{\int_{#1} #2 \, d#1}

% Limit
\newcommand{\limit}[3]{\lim_{#1 \to #2} #3}


%===============================================================================
%================================  BIG SYMBOLS  ================================
%===============================================================================


% Sum
\newcommand{\sumof}[2]{\sum_{#1}^{#2}}
% Product
\newcommand{\productof}[2]{\prod_{#1}^{#2}}
% Union
\newcommand{\unionof}[2]{\bigcup_{#1}^{#2}}
% Intersection
\newcommand{\intersectionof}[2]{\bigcap_{#1}^{#2}}
% Or
\newcommand{\orof}[2]{\bigvee_{#1}^{#2}}
% And
\newcommand{\andof}[2]{\bigwedge_{#1}^{#2}}


%===============================================================================
%=============================== LINEAR ALGEBRA ================================
%===============================================================================


% Bold vector arrow
\newcommand{\bv}[1]{\vec{\mathbf{#1}}}

% Matrix or vector (use // for column, & for row) 
% with brackets
\newcommand{\bmat}[1]{\begin{bmatrix} #1 \end{bmatrix}}
% Matrix or vector (use // for column, & for row) 
% with curved brackets
\newcommand{\pmat}[1]{\begin{pmatrix} #1 \end{pmatrix}}
% Matrix or vector (use // for column, & for row) 
% with lines on either side 
\newcommand{\lmat}[1]{\begin{vmatrix} #1 \end{vmatrix}}
% Matrix or vector (use // for column, & for row) 
% with curly braces
\newcommand{\cmat}[1]{\begin{Bmatrix} #1 \end{Bmatrix}}
% Matrix or vector (use // for column, & for row) 
% with no braces
\newcommand{\mat}[1]{\begin{matrix} #1 \end{matrix}}


%===============================================================================
%================================ LARGE OBJECTS ================================
%===============================================================================

% Multiple lines
\newcommand{\multiline}[1]{
\begin{align*}
    #1
\end{align*}
}

% Multiple lines with equation numbers
\newcommand{\eqmultiline}[1]{
\begin{align*}
    #1
\end{align*}
}

% Color
\newcommand{\colorText}[2]{
\begingroup
\color{#1}
    #2
\endgroup
}

% Centered figure
\newcommand{\centerFigure}[2]{
    \begin{figure}[h]
        \centering
            #1
        \caption{#2}
    \end{figure}
}

% Tikz figure
\newcommand{\tikzGraphic}[1]{
    \begin{center}
    \begin{tikzpicture}
        #1
    \end{tikzpicture}
    \end{center}
}

% Enumerate numbers (seperate by \item)
\newcommand{\numbers}{\textbf{\number*)}}

% Enumerate letters (seperate by \item)
\newcommand{\letters}{\textbf{\alph*)}}

\title{%
    \Huge Numerical Algorithms \\
    \Large Lecture XIV
}
\date{2025-06-11}
\author{Michael Saba}

\begin{document}
\pagenumbering{gobble}
\maketitle
\newpage
\setlength{\parindent}{0pt}
\pagenumbering{arabic}

\subsection*{Inhomogeneous and Anisotropic Diffusion}

We show how to discretize the
non-isotropic and non-homogeneous cases
for the heat equation. \\

Recall that by Fick's law, we have a diffusion flux:
\[ j = -D\nabla u \]
Where $u$, also written $p$ is the density, and $D$ is the
structure tensor. \\

In the Non-linear (Inhomogeneous) Isotropic Diffusion
case, the diffusion varies in space but not
directions. We have a structure tensor which
is constant, but not of the form $kI$. \\
In particular this leads to aa diffusion flux:
\[ j = -g(\|\nabla u\|^2)\nabla u \]
Which leads to the heat equation
\[ \partialdd{u}{t} = \nabla \cdot (g(\|\nabla u\|^2) 
\nabla u) \]
Here $g$ is called diffusivity. \\

Typical diffusivity functions include the Charbonnier 
and Perona-Malik diffusivities:
\[ g_{\text{Ch}}(s^2) = 
\frac{1}{p \left( 1 + \frac{s^2}{\lambda^2} \right)} 
\qquad g_{\text{PM}}(s^2) = 
\frac{1}{1 + \frac{s^2}{\lambda^2}} \]
They both reduce diffusion at the edges. \\

On the other hand, we can have am Inhomogeneous Anisotropic
diffusion, where the diffusion depends on direction
but not space. 
The structure tensor $D$ is a function of $x$ and $t$,
position and time. \\
Here we will have a diffusion flux:
\[ j = -D(\nabla u)\nabla u \]
Where $D = \nabla u (\nabla u)^T$. 
In the $2$ dimesnional case for example:
\[ D(\nabla u) = \bmat{\partial_{x}\partial_{x} u 
& \partial_{x}\partial_{y}u \\
\partial_{y}\partial_{x}u &
\partial_{y}\partial_{y}u} \]
This leads to the heat equation:
\[ \partialdd{u}{t} = \nabla \cdot (D(\nabla u) \nabla u) \]
It can also be used to
discourage diffusion across edges. \\

Note that we can approximate the first derivative
$\partial_x u_i$ with a line
(here $h$ is the interval):
\[ \partial_x u_i \approx \dfrac{u_{i + \sfrac{1}{2}}
- u_{i - \sfrac{1}{2}}}{h} \]
As mentioned in the last chapter. \\

Note that if we apply the first order derivative
approximation twice, we just get the second order 
derivative we derived last time:
\[ \partial_{xx} u_i
= \partial_{x}(\partial_x u_i) 
\approx \dfrac{(\partial_x u)_{i + \sfrac{1}{2}}
- (\partial_x u)_{i - \sfrac{1}{2}}}{h} \]
Now replacing $\partial_x u$ by the
approximation, we get: 
\[ \dfrac{\para{\dfrac{u_{i + 1}
- u_{i - 0}}{h}} - \para{\dfrac{u_{i + 0}
- u_{i - 1}}{h}}}{h}
= \dfrac{u_{i-1} - 2u_{i} + u_{i+1}}{h^2} \]
This is the same quadratic formula. \\

Now to discretize a PDE of the form:
\[ \partial_t u
= \partial_x(g(x) \partial_x u) \]
We can apply the first order derivative
twice, same as we did earlier:
\multiline{
\partial_t u(x_i) &= \partial_x 
\left( g(x_i) \, \partial_x u(t, x_i) \right) \\
&\approx \frac{g_{i+\frac{1}{2}} 
(\partial_x u)_{i+\frac{1}{2}} - 
g_{i-\frac{1}{2}} (\partial_x u)_{i-\frac{1}{2}}}{h} \\
&\approx \frac{g_{i+\frac{1}{2}} 
(u_{i+1} - u_i) - g_{i-\frac{1}{2}} 
(u_i - u_{i-1})}{h^2} \\
&= \frac{g_{i+1} + g_i}{2 h^2} u_{i+1} - 
\frac{g_{i+1} + 2 g_i + g_{i-1}}{2 h^2} u_i + 
\frac{g_i + g_{i-1}}{2 h^2} u_{i-1} }
Now we have the coefficients that we can place
into a finite difference matrix. \\
This can be used for any $g(x)$,
which makes it applicable for the isotropic
inhomogeneous case,
where we have a $g(x)$,
but not a matrix. \\

\newpage

\subsection*{Wave Equation}

We can create small waves on an image.
This is done using the wave equation. \\

The wave equation is a PDE:
\[ \partialdd{^2u}{t^2} = \nabla^2 u \]
In the two dimensional case, this is:
\[ \partial_{tt}u = \partial_{xx} u + \partial_{yy} u \]
So we have a second order derivative in terms of $t$,
that is the main difference. \\

Now, we already know how to discretize the right hand
side:
\[ \partial_{tt}u = \dfrac{u_{i, j-1}-2u_{i, j}
+u_{i, j+1}}{(h_x)^2} + \dfrac{u_{i-1, j}-2u_{i, j}
+u_{i-1, j}}{(h_y)^2} \]
But now, instead of using a first derivative discretization
of the left hand side, like we did last time with
the explicit and implicit Euler form,
we just use the quadratic formula for the second derivative
again, this time interpolating three points in time:
\[ \dfrac{u^{(k-1)}_{i, j}-2u^{(k)}_{i, j}
+u^{(k+1)}_{i, j}}{\tau} = \dfrac{u^{(k)}_{i, j-1}
-2u^{(k)}_{i, j} +u^{(k)}_{i, j+1}}{(h_x)^2} 
+ \dfrac{u^{(k)}_{i-1, j}-2u^{(k)}_{i, j} 
+ u^{(k)}_{i-1, j}}{(h_y)^2} \]
And then we solve for $u^{(k+1)}_{ij}$, the
next point's update:
\[ u^{(k+1)}_{i, j}= \tau\dfrac{u^{(k)}_{i, j-1}
-2u^{(k)}_{i, j} +u^{(k)}_{i, j+1}}{(h_x)^2} 
+ \tau\dfrac{u^{(k)}_{i-1, j}-2u^{(k)}_{i, j} 
+ u^{(k)}_{i-1, j}}{(h_y)^2}
-u^{(k-1)}_{i, j} + 2u^{(k)}_{i, j} \]
Notice that we will need to save the last frame,
and the frame before that one, in order to be able
to do our update step. \\

Note that an implicit form wouldn't make sense
in the second derivative case.


Having to remember the last two frames also implies
that we will need to have two starting conditions. \\
In the heat equation, we only needed to keep track
of the first frame, and recall that the solution,
without discretizing time, was:
\[ u = e^{kAt}u(0) \]
Where $A$ is the laplacian finite difference matrix. \\
Notice that we only need one starting condition $u(0)$,
which is the initial texture before diffusion in the
code example. \\

On the other hand, if we attempt to solve 
the wave equation without discretizing time:
\[ \partialdd{^2u}{t^2} = Lu \]
Then we can solve it as such:
\[ \partialdd{}{t}\partialdd{u}{t} = Lu \]
\[ \partialdd{u}{t} = e^{Lt}u'(0) \]
Then, the solution to $\partial_t u = e^{f(t)}$
is very simple, it is the exponential itself:
\[ u = \para{\dfrac{e^{Lt}u'(0)}{L}}u(0) \]
Notice that we need two starting conditions, 
$u'(0)$, and $u(0)$. \\

To give an intuitive example, suppose we have
the acceleration (a second derivative 
of the position); to find the position after some time,
we need the initial velocity and the initial position.
These are analogs of $u'(0)$ and $u(0)$. \\

\newpage 

\subsection*{The Frequency Domain}

The issue with discretizing the the PDE
with respect to time is that the time step
has to be quite small to achieve stability. \\
Another issue is that we are locked into
waves that have the same size, or amplitude,
which does not wane with time. \\

We can instead try solving the PDE
analytically as we did earlier:
\[ \partial_{tt}u = Lu \]
We know that the laplacian finite difference
matrix is diagonalizable, so we can write:
\[ \partial_{tt}u = V\Lambda V^T u \]
Multiplying both sides by $V^T$:
\[ V^T\partial_{tt}u = V^TV\Lambda V^T u \]
Now, since $V$ is orthogonal $VV^T = I$,
and since $V^T$ is constant, we can
bring it into the second derivative
with respect to $t$:
\[ \partial_{tt} V^Tu = \Lambda V^T u \]
We will take $\hat{u} = V^T u$,
which nets us the PDE:
\[ \partial_{tt} \hat{u} = \Lambda \hat{u} \]
Which, for each $u_i$ in the discretized vector
of points, is:
\[ \partial_{tt} \hat{u}_i = \lambda_i \hat{u}_i \]
We can solve this for $\hat{u}$,
then multiply the result by $V$,
the inverse of $V^T$, in order to get $u$. \\
This $\hat{u}$ is said to be in the frequency 
domain. \\

So how can we find the eigenspace $V$, 
and its transpose $V^T$?
As we know, the finite difference Laplacian matrix
depends on the boundary conditions we are using,
be it reflecting boundaries, toroidal boundaries...
So the eigenspace of $L$
will depend on which reflecting boundary is
chosen (and on the size of the matrix of course,
which is a function of the number of dimensions
and points in the discretization). \\

The eigenspace $V^T$ of $L$ is given by the 
discrete sine transform DST, or the discrete
cosine transform DCT.
There are several DST and DCT functions;
for instance, DCT-II is the one that
corresponds to relfecting boundaries, which is
what we have been using. \\

The DCT-II of a sequence 
$x = (x_1, \dots, x_n)$ is defined as:
\[ {X_{k}=\sum _{n=1}^{N}x_{n}\cos 
\left[\,{\tfrac {\,\pi \,}{N}}
\left(n+{\tfrac {1}{2}}\right)k\,
\right]\qquad {\text{ for }}~k=1,\ 
\dots \ N}\]
So if we apply the DCT-II to the vector $u$,
we would get the eigenspace $V^T$. \\
Note that it won't be orthogonal however,
and will need to be normalized by:
\[ \piecewise{
    \sqrt{\dfrac{1}{N}} \qquad & \FOR X_1 \vs{10} \\
    \sqrt{\dfrac{2}{N}} \qquad & \text{otherwise}
} \]
So applying the DCT-II, with tyhe normalization
takes us into the frequency domain, and
applying the inverse operation gives us the $u$
we are looking for. \\

We can use this formula to creates waves
with amplitudes that decrease with time
as well, but modifying the frequency
domain differential equation. \\

We end up with:
\[ u(t) = \cos(t \lambda_j) \hat{f}(0) 
+ \frac{\sin(t \lambda_j)}{\lambda_j} \hat{g}(0) \]
Which we can then convert back to $u$. \\

Note that we can solve the continuous differential
equation without the spectral method. We will however
need to turn the higher order differential
equation into a system of first order differential
equations. \\

This same method can also be used for heat diffusion. \\

\newpage

\subsection*{Finite Differences in Higher Dimensions}

We've already seen the laplacian
finite difference matrix in both 1 and 2
dimensions. We can generalize this matrix
by using the Kronecker product. \\
The intuition is that our space, which
can be in any dimension $n$,
is discretized into an $n$-dimensional grid
of dimensions:
\[ m_1 \times m_2 \times \dots m_n \] 
with an $n$-dimensional stencil,
but can be flattened to look like a vector
of dimension:
\[ \prod_{i=1}^n m_i \]
Then, using a flattened one dimensional
stencil on said vectors would yield
a matrix of dimension:
\[ \prod_{i=1}^n m_i \times \prod_{i=1}^n m_i \]
Which will multiply our flattened 
vector of points $u$. \\

Now, we don't usually have to physically
flatten the grid into a vector, and can
instead use Kronecker sets
and products. \\

If $ A $ is an $m \times n$ matrix and $B$ 
is a $ p \times q$ matrix, 
then the Kronecker product $A \otimes B$ 
is the $ pm \times qn$ block matrix:
\[ \mathbf{A} \otimes \mathbf{B} = 
\begin{bmatrix}
a_{11} \mathbf{B} & \cdots & a_{1n} \mathbf{B} \\
\vdots & \ddots & \vdots \\
a_{m1} \mathbf{B} & \cdots & a_{mn} \mathbf{B}
\end{bmatrix} \]
More explicitly, each block $ a_{ij} \mathbf{B} $ 
is a scalar multiple of the full matrix $B$, so:
\[\mathbf{A} \otimes \mathbf{B} =
\begin{bmatrix}
a_{11} b_{11} & a_{11} b_{12} & \cdots & a_{11} b_{1q} & \cdots & a_{1n} b_{11} & \cdots & a_{1n} b_{1q} \\
a_{11} b_{21} & a_{11} b_{22} & \cdots & a_{11} b_{2q} & \cdots & a_{1n} b_{21} & \cdots & a_{1n} b_{2q} \\
\vdots & \vdots & \ddots & \vdots & & \vdots & \ddots & \vdots \\
a_{11} b_{p1} & a_{11} b_{p2} & \cdots & a_{11} b_{pq} & \cdots & a_{1n} b_{p1} & \cdots & a_{1n} b_{pq} \\
\vdots & \vdots & & \vdots & \ddots & \vdots & & \vdots \\
a_{m1} b_{11} & a_{m1} b_{12} & \cdots & a_{m1} b_{1q} & \cdots & a_{mn} b_{11} & \cdots & a_{mn} b_{1q} \\
a_{m1} b_{21} & a_{m1} b_{22} & \cdots & a_{m1} b_{2q} & \cdots & a_{mn} b_{21} & \cdots & a_{mn} b_{2q} \\
\vdots & \vdots & \ddots & \vdots & & \vdots & \ddots & \vdots \\
a_{m1} b_{p1} & a_{m1} b_{p2} & \cdots & a_{m1} b_{pq} & \cdots & a_{mn} b_{p1} & \cdots & a_{mn} b_{pq}
\end{bmatrix} \]
As we can see, it sort of ``flattens"
matrix multiplication. \\
Note that the Kronecker product
does not require both matrices
to be of the same dimensions. \\

On the other hand, a Kronecker
sum of two square matrices $A$
and $B$ of dimensions $n \times n$
and $m \times m$ respectively
is:
\[ A \oplus B = A \otimes I_m + I_n \otimes B \]
Notice that we end up with an $mn \times mn$
matrix. \\

So, if we have a two dimensional grid of points,
we can find the finite difference Laplacian
matrix for each direction (a row
and a column), and then just
use the Kronecker sum on the two. \\
Recall that when we had a $3 \times 4$ grid,
we ended up with a $12 \times 12$ matrix, 
which matches the result of the Kronecker
sum of the two finite difference matrices,
and matches what we would have gotten
had we flattened the grid first into
a vector, and then found the finite
difference matrix. \\

This allows us to generalize the finite differences
method to higher dimensions without having
to bother with $n$-dimensional stencils. \\
So if we have $n$ dimensions for our
grid of points, each with $m_i$
points and an induced finite difference
Laplacian $A_i$, our $n$-dimensional
finite difference matrix $A$ is:
\[ A = \sum_{i=1}^n \para{A_i \otimes 
\bigotimes_{\substack{j=1 \\ j \neq i}}^n I_{m_j}}  \]
This works for any dimension. \\

Note that one big advantage of this method
when it comes to using the DST and DCT is that
we don't have to bother with the DST and DCT
of higher dimensional finite difference laplacian 
matrices, and can instead just use the one 
dimensional  matrix's DCT or DST, and then 
generalize it to $n$-dimensions. \\

Suppose that we have the eigendecomposition:
\[ A = P\Lambda P\inv \qquad B = Q\Sigma Q\inv \]
Then the eigendecomposion of the Kronecker sum
of $A$ and $B$ (the larger stencil in multiple
dimensions) will be:
\multiline{
A \oplus B &= A \otimes I_H + I_W \otimes B \\
&= (P \Lambda P^{-1}) \otimes (Q I Q^{-1}) 
+ (P I P^{-1}) \otimes (Q \Sigma Q^{-1}) \\
&= (P \otimes Q) (\Lambda \otimes I_H) 
(P^{-1} \otimes_K Q^{-1}) + (P \otimes Q) 
(I_W \otimes \Sigma) (P^{-1} \otimes Q^{-1}) \\
&= (P \otimes Q) \big(\Lambda \otimes I_H 
+ I_W \otimes \Sigma \big) (P^{-1} \otimes Q^{-1}) \\
&= (P \otimes Q) \big(\Lambda \oplus \Sigma \big) 
(P^{-1} \otimes Q^{-1}) }
So the diagonal matrix with the eigenvalues
of $A \oplus B$ has entries that look like:
\[ \lambda_i + \mu_j \]
where $\lambda_i$ and $\mu_j$ are the eigenvalues of $A$
and $B$ respectively. \\
Here $\lambda_i + \mu_j$ is the eigenvalue of
the element at the point $(i, j)$ on the two dimensional
grid. \\

We can also find the eigenvectors using the Kronecker
product of $P$ and $Q$. \\

The same applies for more dimensions. \\


\newpage

\end{document}
