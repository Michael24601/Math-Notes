
\documentclass[12pt]{article}
\usepackage[margin=1in]{geometry}


%===============================================================================
%================================== PACKAGES ===================================
%===============================================================================

% For using float option H that places figures 
% exatcly where we want them
\usepackage{float}
% makes figure font bold
\usepackage{caption}
\captionsetup[figure]{labelfont=bf}
% For text generation
\usepackage{lipsum}
% For drawing
\usepackage{tikz}
% For smaller or equal sign and not divide sign
\usepackage{amssymb}
% For the diagonal fraction
\usepackage{xfrac}
% For enumerating exercise parts with letters instead of numbers
\usepackage{enumitem}
% For dfrac, which forces the fraction to be in display mode (large) e
% even in math mode (small)
\usepackage{amsmath}
% For degree sign
\usepackage{gensymb}
% For "\mathbb" macro
\usepackage{amsfonts}
% For positioning 
\usepackage{indentfirst}
\usetikzlibrary{shapes,positioning,fit,calc}
% for adjustwidth environment
\usepackage{changepage}
% for arrow on top
\usepackage{esvect}
% for mathbb 1
\usepackage{bbm}
% for mathsrc
\usepackage[mathscr]{eucal}
% For degree sign
\usepackage{gensymb}
% For quotes
\usepackage{csquotes}
% For vertical lines
\usepackage{mathtools}
% For cols
\usepackage{multicol}

% for tikz
\usepackage{pgfplots}
\pgfplotsset{compat=1.18}
\usepackage{amsmath}
\usepgfplotslibrary{groupplots}


%===============================================================================
%==================================== FONTS ====================================
%===============================================================================


% Mathcal
\newcommand{\acal}{\mathcal{A}}
\newcommand{\bcal}{\mathcal{B}}
\newcommand{\ccal}{\mathcal{C}}
\newcommand{\dcal}{\mathcal{D}}
\newcommand{\ecal}{\mathcal{E}}
\newcommand{\fcal}{\mathcal{F}}
\newcommand{\gcal}{\mathcal{G}}
\newcommand{\hcal}{\mathcal{H}}
\newcommand{\ical}{\mathcal{I}}
\newcommand{\jcal}{\mathcal{J}}
\newcommand{\kcal}{\mathcal{K}}
\newcommand{\lcal}{\mathcal{L}}
\newcommand{\mcal}{\mathcal{M}}
\newcommand{\ncal}{\mathcal{N}}
\newcommand{\ocal}{\mathcal{O}}
\newcommand{\pcal}{\mathcal{P}}
\newcommand{\qcal}{\mathcal{Q}}
\newcommand{\rcal}{\mathcal{R}}
\newcommand{\scal}{\mathcal{S}}
\newcommand{\tcal}{\mathcal{T}}
\newcommand{\ucal}{\mathcal{U}}
\newcommand{\vcal}{\mathcal{V}}
\newcommand{\wcal}{\mathcal{W}}
\newcommand{\xcal}{\mathcal{X}}
\newcommand{\ycal}{\mathcal{Y}}
\newcommand{\zcal}{\mathcal{Z}}

% Mathfrak
\newcommand{\afrak}{\mathfrak{A}}
\newcommand{\bfrak}{\mathfrak{B}}
\newcommand{\cfrak}{\mathfrak{C}}
\newcommand{\dfrak}{\mathfrak{D}}
\newcommand{\efrak}{\mathfrak{E}}
\newcommand{\ffrak}{\mathfrak{F}}
\newcommand{\gfrak}{\mathfrak{G}}
\newcommand{\hfrak}{\mathfrak{H}}
\newcommand{\ifrak}{\mathfrak{I}}
\newcommand{\jfrak}{\mathfrak{J}}
\newcommand{\kfrak}{\mathfrak{K}}
\newcommand{\lfrak}{\mathfrak{L}}
\newcommand{\mfrak}{\mathfrak{M}}
\newcommand{\nfrak}{\mathfrak{N}}
\newcommand{\ofrak}{\mathfrak{O}}
\newcommand{\pfrak}{\mathfrak{P}}
\newcommand{\qfrak}{\mathfrak{Q}}
\newcommand{\rfrak}{\mathfrak{R}}
\newcommand{\sfrak}{\mathfrak{S}}
\newcommand{\tfrak}{\mathfrak{T}}
\newcommand{\ufrak}{\mathfrak{U}}
\newcommand{\vfrak}{\mathfrak{V}}
\newcommand{\wfrak}{\mathfrak{W}}
\newcommand{\xfrak}{\mathfrak{X}}
\newcommand{\yfrak}{\mathfrak{Y}}
\newcommand{\zfrak}{\mathfrak{Z}}

% Mathscr
\newcommand{\ascr}{\mathscr{A}}
\newcommand{\bscr}{\mathscr{B}}
\newcommand{\cscr}{\mathscr{C}}
\newcommand{\dscr}{\mathscr{D}}
\newcommand{\escr}{\mathscr{E}}
\newcommand{\fscr}{\mathscr{F}}
\newcommand{\gscr}{\mathscr{G}}
\newcommand{\hscr}{\mathscr{H}}
\newcommand{\iscr}{\mathscr{I}}
\newcommand{\jscr}{\mathscr{J}}
\newcommand{\kscr}{\mathscr{K}}
\newcommand{\lscr}{\mathscr{L}}
\newcommand{\mscr}{\mathscr{M}}
\newcommand{\nscr}{\mathscr{N}}
\newcommand{\oscr}{\mathscr{O}}
\newcommand{\pscr}{\mathscr{P}}
\newcommand{\qscr}{\mathscr{Q}}
\newcommand{\rscr}{\mathscr{R}}
\newcommand{\sscr}{\mathscr{S}}
\newcommand{\tscr}{\mathscr{T}}
\newcommand{\uscr}{\mathscr{U}}
\newcommand{\vscr}{\mathscr{V}}
\newcommand{\wscr}{\mathscr{W}}
\newcommand{\xscr}{\mathscr{X}}
\newcommand{\yscr}{\mathscr{Y}}
\newcommand{\zscr}{\mathscr{Z}}

% Mathbb
\newcommand{\abb}{\mathbb{A}}
\newcommand{\bbb}{\mathbb{B}}
\newcommand{\cbb}{\mathbb{C}}
\newcommand{\dbb}{\mathbb{D}}
\newcommand{\ebb}{\mathbb{E}}
\newcommand{\fbb}{\mathbb{F}}
\newcommand{\gbb}{\mathbb{G}}
\newcommand{\hbb}{\mathbb{H}}
\newcommand{\ibb}{\mathbb{I}}
\newcommand{\jbb}{\mathbb{J}}
\newcommand{\kbb}{\mathbb{K}}
\newcommand{\lbb}{\mathbb{L}}
\newcommand{\mbb}{\mathbb{M}}
\newcommand{\nbb}{\mathbb{N}}
\newcommand{\obb}{\mathbb{O}}
\newcommand{\pbb}{\mathbb{P}}
\newcommand{\qbb}{\mathbb{Q}}
\newcommand{\rbb}{\mathbb{R}}
\newcommand{\sbb}{\mathbb{S}}
\newcommand{\tbb}{\mathbb{T}}
\newcommand{\ubb}{\mathbb{U}}
\newcommand{\vbb}{\mathbb{V}}
\newcommand{\wbb}{\mathbb{W}}
\newcommand{\xbb}{\mathbb{X}}
\newcommand{\ybb}{\mathbb{Y}}
\newcommand{\zbb}{\mathbb{Z}}


%===============================================================================
%=============================== SPECIAL SYMBOLS ===============================
%===============================================================================


% Orbit (group theory)
\newcommand{\orbit}{\mathcal{O}}
% Normal group
\newcommand{\normal}{\mathcal{N}}
% Indicator function
\newcommand{\indicator}{\mathbbm{1}}
% Laplace transform
\newcommand{\laplace}[1]{\mathcal{L}}
% Epsilon shorthand
\newcommand{\eps}{\varepsilon}
% Omega
\newcommand{\om}{\omega}
\newcommand{\Om}{\Omega}

%===============================================================================
%================================== OPERATORS ==================================
%===============================================================================


% Inverse exponent
\newcommand{\inv}[0]{^{-1}}
% Overline bar
\newcommand{\olsi}[1]{\,\overline{\!{#1}}}
% Less than or equal slanted
\newcommand{\seqs}{\leqslant}
% Greater or equal slanted
\newcommand{\geqs}{\geqslant}
% Subset or equal
\newcommand{\sub}{\subseteq}
% Proper subset
\newcommand{\prosub}{\subset}
% from
\newcommand{\from}{\leftarrow}

% Parantheses
\newcommand{\para}[1]{\left( #1 \right)}
% Curly Braces
\newcommand{\curl}[1]{\left\{ #1 \right\}}
% Brackets
\newcommand{\brac}[1]{\left[ #1 \right]}
% Angled Brackets
\newcommand{\ang}[1]{\left\langle #1 \right\rangle}
% Norm
\newcommand{\norm}[1]{\left\| #1 \right\|}

% Piece wise (use \\ between cases)
\newcommand{\piecewise}[1]{\begin{cases} #1 \end{cases}}

% Bold symbol shorthand
\newcommand{\bl}{\boldsymbol}

% Vertical space
\newcommand{\vs}[1]{\vspace{#1 pt}}
% Horizontal ertical space
\newcommand{\hs}[1]{\hspace{#1 pt}}


%===============================================================================
%============================== TEXT BASED SYMBOLS =============================
%===============================================================================


% Radians
\newcommand{\rad}{\text{rad}}
% Least Common Multiple
\newcommand{\lcm}{\text{lcm}}
% Automorphism
\newcommand{\Aut}{\text{Aut}}
% Variance
\newcommand{\var}{\text{Var}}
% Covariance
\newcommand{\cov}{\text{Cov}}
% Cofactor (matrix)
\newcommand{\cof}{\text{Cof}}
% Adjugate (matrix)
\newcommand{\adj}{\text{Adj}}
% Trace (matrix)
\newcommand{\tr}{\text{tr}}
% Standard deviation
\newcommand{\std}{\text{Std}}
% Correlation coefficient
\newcommand{\corr}{\text{Corr}}
% Sign
\newcommand{\sign}{\text{sign}}

% And text
\newcommand{\AND}{\text{ and }}
% Or text 
\newcommand{\OR}{\text{ or }}
% For text 
\newcommand{\FOR}{\text{ for }}
% If text
\newcommand{\IF}{\text{ if }}
% When text
\newcommand{\WHEN}{\text{ when }}
% Where text
\newcommand{\WHERE}{\text{ where }}
% Then text
\newcommand{\THEN}{\text{ then }}
% Such that text
\newcommand{\SUCHTHAT}{\text{ such that }}

% 1st
\newcommand{\st}[1]{#1^{\text{st}}}
% 2nd
\newcommand{\nd}[1]{#1^{\text{nd}}}
% 3rd
\newcommand{\rd}[1]{#1^{\text{rd}}}
% nth
\newcommand{\nth}[1]{#1^{\text{th}}}


%===============================================================================
%========================= PROBABILITY AND STATISTICS ==========================
%===============================================================================


% Permutation
\newcommand{\perm}[2]{{}^{#1}\!P_{#2}}
% Combination
\newcommand{\comb}[2]{{}^{#1}C_{#2}}

% Baye's risk
\newcommand{\risk}[1]{\mathscr{R}_{#1}}
% Baye's optimal risk
\newcommand{\riskOptimal}[1]{\mathscr{R}_{#1}^*}
% Baye's empirical risk
\newcommand{\riskEmpirical}[2]{\hat{\mathscr{R}}_{#1}^{#2}}


%===============================================================================
%=================================== CALCULUS ==================================
%===============================================================================


% d over d derivative
\newcommand{\dd}[2]{\dfrac{d#1}{d#2}}
% partial d over d derivative
\newcommand{\partialdd}[2]{\dfrac{\partial #1}{\partial #2}}
% delta d over d derivative
\newcommand{\deltadd}[2]{\dfrac{\Delta #1}{\Delta #2}}

% Integration between a and b
\newcommand{\integral}[4]{\int_{#1}^{#2} #3 \, #4}
% Integration in some space 
\newcommand{\boundIntegral}[2]{\int_{#1} #2 \, d#1}

% Limit
\newcommand{\limit}[3]{\lim_{#1 \to #2} #3}


%===============================================================================
%================================  BIG SYMBOLS  ================================
%===============================================================================


% Sum
\newcommand{\sumof}[2]{\sum_{#1}^{#2}}
% Product
\newcommand{\productof}[2]{\prod_{#1}^{#2}}
% Union
\newcommand{\unionof}[2]{\bigcup_{#1}^{#2}}
% Intersection
\newcommand{\intersectionof}[2]{\bigcap_{#1}^{#2}}
% Or
\newcommand{\orof}[2]{\bigvee_{#1}^{#2}}
% And
\newcommand{\andof}[2]{\bigwedge_{#1}^{#2}}


%===============================================================================
%=============================== LINEAR ALGEBRA ================================
%===============================================================================


% Bold vector arrow
\newcommand{\bv}[1]{\vec{\mathbf{#1}}}

% Matrix or vector (use // for column, & for row) 
% with brackets
\newcommand{\bmat}[1]{\begin{bmatrix} #1 \end{bmatrix}}
% Matrix or vector (use // for column, & for row) 
% with curved brackets
\newcommand{\pmat}[1]{\begin{pmatrix} #1 \end{pmatrix}}
% Matrix or vector (use // for column, & for row) 
% with lines on either side 
\newcommand{\lmat}[1]{\begin{vmatrix} #1 \end{vmatrix}}
% Matrix or vector (use // for column, & for row) 
% with curly braces
\newcommand{\cmat}[1]{\begin{Bmatrix} #1 \end{Bmatrix}}
% Matrix or vector (use // for column, & for row) 
% with no braces
\newcommand{\mat}[1]{\begin{matrix} #1 \end{matrix}}


%===============================================================================
%================================ LARGE OBJECTS ================================
%===============================================================================

% Multiple lines
\newcommand{\multiline}[1]{
\begin{align*}
    #1
\end{align*}
}

% Multiple lines with equation numbers
\newcommand{\eqmultiline}[1]{
\begin{align*}
    #1
\end{align*}
}

% Color
\newcommand{\colorText}[2]{
\begingroup
\color{#1}
    #2
\endgroup
}

% Centered figure
\newcommand{\centerFigure}[2]{
    \begin{figure}[h]
        \centering
            #1
        \caption{#2}
    \end{figure}
}

% Tikz figure
\newcommand{\tikzGraphic}[1]{
    \begin{center}
    \begin{tikzpicture}
        #1
    \end{tikzpicture}
    \end{center}
}

% Enumerate numbers (seperate by \item)
\newcommand{\numbers}{\textbf{\number*)}}

% Enumerate letters (seperate by \item)
\newcommand{\letters}{\textbf{\alph*)}}

\title{%
    \Huge Book of Proof \\
    \large by \\
    \Large Richard Hammack \\~\\
    \huge Part 2: How to Prove a Conditional Statement \\
    \LARGE Chapter 5: Contrapositive Proof
}
\date{2024-03-17}
\author{Michael Saba}

\begin{document}
    \pagenumbering{gobble}
    \maketitle
    \newpage
    \setlength{\parindent}{0pt}
    \pagenumbering{arabic}

    For questions 1 - 13, we have to use a contrapositive proof. \\

    \section*{Exercise 1}
    Proof that for $n \in \Z$,
    if $n^2$ is even,
    then $n$ is even: \\
    Suppose that $n \in Z$ 
    such that $n$ is not even.
    Then $n$ is odd which means
    that $n$ can be written as $2m + 1$ where $m \in \Z$.
    Then $n^2 = (2m+1)^2 = 4m^2 + 4m + 1 = 2(2m^2 + 2m) + 1$
    where $2m^2 + 2m$ is an integer.
    This means that $n^2$ is odd (not even),
    which proves the proposition contrapositively.

    \section*{Exercise 2}
    Proof that for $n \in \Z$,
    if $n^2$ is odd,
    then $n$ is odd: \\
    Suppose that $n \in Z$ 
    such that $n$ is not odd.
    Then $n$ is even which means
    that $n$ can be written as $2m$ where $m \in \Z$.
    Then $n^2 = (2m)^2 = 4m^2 + 4m = 2(2m^2 + 2m)$
    where $2m^2 + 2m$ is an integer.
    This means that $n^2$ is even (not odd),
    which proves the proposition contrapositively.

    \section*{Exercise 3}
    Proof that for $a, b \in \Z$,
    if $a^2(b^2 - 2b)$ is odd,
    then $a$ and $b$ are odd: \\
    Suppose that $a, b \in \Z$
    such that $a$ and $b$ aren't both odd. 
    This means that $a$ or $b$ is even (or both). \\
    If $a$ is even (and $b$ is either),
    we could write $a$ as $2n$ for some $n \in \Z$.
    Then
    \[ a^2(b^2 - 2b) = (2n)^2(b^2 - 2b)
    = 4n^2(b^2 - 2b) = 2(2n^2(b^2 - 2b)) \]
    where $2n^2(b^2 - 2b)$ is an integer,
    so $a^2(b^2 - 2b)$ is even (not odd). \\
    If $b$ is even (and $a$ is either),
    we could write $b$ as $2m$ for some $n \in \Z$.
    Then
    \[ a^2(b^2 - 2b) = a^2((2m)^2 - 2(2m))
    = a^2(4m^2 - 4m) = 2(a^2(2m^2 - 2m)) \]
    where $a^2(2m^2 - 2m)$ is an integer,
    so $a^2(b^2 - 2b)$ is even (not odd). \\

    \section*{Exercise 4}
    Proof that for $a, b, c \in \Z$,
    if $a \nmid bc$,
    then $a \nmid b$: \\
    Suppose that $a, b, c \in \Z$
    such that $a \mid b$,
    then $b = an$ for some $n \in \Z$.
    So $bc = a(cn)$ where $cn$ is an integer.
    Thus $a \mid bc$,
    proving the proposition contrapositively. \\

    \section*{Exercise 5}
    Proof that for $x \in \R$,
    if $x^2 + 5x < 0$,
    then $x < 0$: \\
    Suppose that $x \in \R$
    such that $x \geqslant 0$.
    Then $5x \geqslant 0$ and $x^2 \geqslant 0$.
    This means that $x^2 + 5x \geqslant 0$,
    which proves the proposition contrapositively. \\

    \section*{Exercise 6}
    Proof that for $x \in \R$,
    if $x^3 - x > 0$,
    then $x > -1$: \\
    Suppose that $x \in \R$
    such that $x \leqslant -1$.
    Then $x^2\geqslant 1$.
    We can then multiply the whole equation by $x$,
    which reverses the sign since $x$ is negative.
    This gives us $x^3 \leqslant x$.
    We can rearrange it to get $x^3 - x \leqslant 0$,
    which proves the proposition contrapositively. \\

    \section*{Exercise 7}
    Proof that for $a, b \in \Z$,
    if $a + b$ and $ab$ are even,
    then $a$ and $b$ are even: \\
    Suppose that $a, b \in \Z$
    such that $a$ or $b$ is odd
    (both of them are not even).
    We can assume, with no loss of generality
    that $a$ is odd, and $b$ is either.
    Then $a = 2n+1$ for some $n \in \Z$. \\
    Now if $b$ is even,
    then $b = 2m$ for some $m \in \Z$. 
    This means that $ab = (2n+1)(2m) = 2(m(2n+1))$
    where $m(2n+1)$ is an integer,
    so $ab$ is even. \\
    On ther other hand, if $b$ is odd,
    then $b = 2q + 1$ for some $q \in \Z$. 
    This means that $a + b = 2n+1 + 2q+1 = 2(n + q + 1)$
    where $n + q + 1$ is an integer,
    so $a + b$ is even. \\
    So either $a+b$ or $ab$ is even,
    which means both are not odd,
    proving the proposition contrapositively. \\

    \section*{Exercise 8}
    Proof that for $x \in \R$,
    if $x^5 - 4x^4 + 3x^3 - x^2 + 3x - 4 \geqslant 0$,
    then $x \geqslant 0$: \\
    Suppose that $x \in \R$
    such that $x < 0$.
    Then $x^5, x^3, x < 0$,
    and $x^2, x^4 > 0$.
    This means that
    \[ x^5, -4x^2, 3x^3, -x^2, 3x, -4 < 0 \]
    which means that 
    \[ x^5 - 4x^2 + 3x^3 - x^2 + 3x - 4 < 0  \]
    completing the proof. \\

    \section*{Exercise 9}
    Proof that for $n \in \Z$,
    if $3 \nmid n^2$,
    then $3 \nmid n$: \\
    Suppose that $n \in \Z$
    such that $3 \nmid n$.
    Then $n = 3m$ for some $m \in \Z$.
    This means that $n^2 = (3m)^2 = 9m^2 = 3(2m^2)$,
    which tells us that $3 \mid n^2$,
    proving the proposition contrapositively. \\
    
    \section*{Exercise 10}
    Proof that for $x, y, z \in \Z$ and $x \neq 0$,
    if $x \nmid yz$,
    then $x \nmid y$ and $x \nmid z$: \\
    Suppose that  $x, y, z \in \Z$
    such that $x \neq 0$, $x \mid y$, $x \mid z$. 
    Then $y = nx$ and $z = mx$ for some $n, m \in \Z$.
    This means that $yz = mnx^2 = x(xmn)$
    where $xmn$ is an integer.
    We thus conclude that $x \mid yz$
    which proves the proposition contrapositively. \\

    \section*{Exercise 11}
    Proof that for $x, y \in \Z$,
    if $x^2(y+3)$ is even,
    then $x$ is even or $y$ is odd: \\
    Suppose that $x, y \in \Z$
    such that $x$ is not even (odd) and $y$ is not odd (even).
    Then $x = 2n + 1$ and $y = 2m$ for some $n, m \in \Z$.
    This means that
    \[ x^2(y+3) = (2n+1)^2(2m + 3)
    = (4n^2 + 4n + 1)(2m + 3) \]
    \[ = 8n^2m + 6n^2 + 8nm + 2m + 12n^2 + 12n + 3 \]
    \[ = 2(n^2m + 3n^2  4nm + m + 6n^2 + 6n + 1) + 1 \]
    where $n^2m + 3n^2  4nm + m + 6n^2 + 6n + 1$ is an integer,
    which means that $x^2(y+3)$ is odd,
    proving the proposition contrapositively. \\

    \section*{Exercise 12}
    Proof that for $x \in \Z$,
    if $4 \nmid x^2$,
    then $x$ is odd: \\
    Suppose that $x \in \Z$
    such that $x$ is even.
    Then $x = 2n$ for some $n \in \Z$.
    This means that $x^2 = (2n)^2 = 4n^2$,
    which tells us that $4 \nmid x^2$,
    proving the proposition contrapositively. \\
New line

    \section*{Exercise 13}
    Proof that for $x \in \R$,
    if $x^5 + 7x^3 + 5x \geqslant x^4 + x^2 + 8$,
    then $x \geqslant 0$: \\
    Suppose that $x \in \R$
    such that $x < 0$.
    Then $x^5, x^3, x < 0$,
    which means that $x^5 + 7x^3 + 5x < 0$.
    Moreover,
    $x^4, x^2, 8 > 0$,
    which means that $x^4 + x^2 + 8 > 0$.
    We thus conclude that $x^5 + 7x^3 + 5x < x^4 + x^2 + 8$. \\

    For questions 14 - 31, we have to choose between a direct
    and contrapositive proof. \\

    \section*{Exercise 14}
    Proof that for $a, b \in \Z$,
    if $a$ and $b$ have the same parity,
    then $3a + 7$ and $7b - 4$ do not: \\
    For this question, it is easier to use a direct proof. \\
    Suppose that $a, b \in \Z$
    such that $a$ and $b$ have the same parity: \\
    If $a$ and $b$ are even,
    then $a = 2n$ and $b = 2m$ for some $n, m \in \Z$.
    This means that
    \[ 3a + 7 = 3(2n) + 7 = 6n + 7 = 2(3n + 3) + 1 \]
    where $3n + 3$ is an integer,
    making $3a + 7$ odd.
    This also means that
    \[ 7b - 4 = 7(2m) - 4 = 14m - 4 = 2(7m - 2) \]
    where $7m - 2$ is an integer,
    making $7b - 4$ even.
    This means that they have opposite parity. \\
    If $a$ and $b$ are odd,
    then $a = 2p + 1$ and $b = 2q + 1$ for some $p, q \in \Z$.
    This means that
    \[ 3a + 7 = 3(2p + 1) + 7 = 6p + 10 = 2(3p + 5) \]
    where $3p + 5$ is an integer,
    making $3a + 7$ even.
    This also means that
    \[ 7b - 4 = 7(2q + 1) - 4 = 14q + 3 = 2(7m + 1) + 1 \]
    where $7m + 1$ is an integer,
    making $7b - 4$ odd.
    This means that they have opposite parity. \\

    \section*{Exercise 15}
    Proof that for $x \in \Z$,
    if $x^3 - 1$ is even, 
    then $x$ is odd: \\
    For this question, it is easier to use a contrapositive proof. \\
    Suppose that $x \in \Z$
    such that $x$ is even (not odd).
    Then $x = 2n$ for some $n \in \Z$.
    This means that
    \[ x^3 - 1 = (2n^3) - 1 = 8n^3 - 2 + 1 = 2(4n^2 - 1) + 1 \]
    where $4n^2 - 1$ is an integer,
    which tells us that $x^3 - 1$ is odd (not even). \\

    \section*{Exercise 16}
    Proog that for $x, y \in \Z$,
    if $x + y$ is even,
    then $x$ and $y$ have the same parity: \\
    For this question, it is easier to use a contrapositive proof. \\
    Suppose that $x, y \in \Z$
    such that $x$ and $y$ have opposite parity. \\
    We can assume with no loss of generality
    that $x$ is even and $y$ is odd. 
    If so, we can write $x = 2n$ and $y = 2m + 1$
    for some $n, m \in \Z$.
    This means that $x + y = 2n + 2m + 1 = 2(n+m) + 1$
    where $n + m$ is an integer,
    which means that $x + y$ is odd,
    proving the proposition contrapositively. \\

    \section*{Exercise 17}
    Proof that for $n \in \Z$,
    if $n$ is odd,
    then $8 \mid (n^2 - 1)$: \\
    For this question, it is easier to use a direct proof. \\
    Suppose that $n \in \Z$
    such that $n$ is odd.
    Then $n = 2m+1$ for some $m \in \Z$.
    This means that
    \[ n^2 - 1 = (2m+1)^2 - 1 = 4m^2 + 4m = 4(m)(m+1) \]
    Since $m$ and $m+1$ are consecutive integers,
    one of them is even;
    we can assume with no loss of generality that it is $m$
    which is even.
    This means that we can write $m$ as $2a$ for some $a \in \Z$.
    By extension, this means that $n^2 - 1 = 8a(m+1)$
    where $a(m+1)$ is an integer,
    which means that $8 \nmid n^2 - 1$. \\

    \section*{Exercise 18}
    Proof that if $a, b \in \Z$,
    then $(a+b)^3 \equiv a^3 + b^3 \mod 3$: \\
    For this question, it is easier to use a direct proof. \\
    Suppose that $a, b \in \Z$.
    Then
    \[ (a+b)^3 - (a^3 + b^3) = a^3 + 3a^2b + 3ab^2 + b^3 - (a^3 + b^3) 
    = 3(a^2b + ab^2) \]
    where $a^2b + ab^2$ is an integer,
    which means that $3 \mid (a+b)^3 - (a^3 + b^3)$.
    This tells us that $(a+b)^3 \equiv a^3 + b^3 \mod 3$. \\

    \section*{Exercise 19}
    Proof that for $a, b, c \in \Z$ and $n \in \N$,
    if $a \equiv b \mod n$ and $a \equiv c \mod n$,
    then $c \equiv b \mod n$: \\
    For this question, it is easier to use a direct proof. \\
    Suppose that $a, b, c \in \Z$ and $n \in \N$
    such that $a \equiv b \mod n$ and $a \equiv c \mod n$.
    Then $n \mid (a - b)$ and $n \mid (a - c)$,
    which means that $a-b = np$ and $a-c = nq$
    for some $p, q \in \Z$.
    This means that
    \[ (a-b)-(a-c) = c - b  \quad = \quad np - nq = n(p -q) \]
    where $p-q$ is an integer.
    So $n \mid (c - b)$,
    which means that $c \equiv b \mod n$. \\

    \section*{Exercise 20}
    Proof that for $a \in \Z$,
    if $a \equiv 1 \mod 5$,
    then $a^2 \equiv 1 \mod 5$: \\
    For this question, it is easier to use a direct proof. \\
    Suppose that $a \in \Z$,
    such that $a \equiv 1 \mod 5$.
    Then $5 \mid a-1$,
    which means that $a-1 = 5n$ for some $n \in \Z$.
    Rearranging the equation gives us $a = 5n - 1$,
    which means that $a^2 = (5n-1)^2 = 25n^2 -10n + 1$.
    We can rearrange the equation again to arrive
    at $a^2 - 1 = 5(5n^2 - 2n)$
    where $5n^2 - 2n$ is an integer.
    This means that $5 \mid (a^2 - 1)$,
    which tells us that $a^2 \equiv 1  \mod 5$. \\

    \section*{Exercise 21}
    Proof that for $a, b \in \Z$ and $n \in \N$,
    if $a \equiv b \mod n$,
    then $a^3 \equiv b^3 \mod n$: \\
    For this question, it is easier to use a direct proof. \\
    Suppose that $a, b \in \Z$ and $n \in \N$
    such that $a \equiv b \mod n$.
    Then $n \mid (a - b)$,
    which means that $(a-b) = nm$ for some $m \in \Z$.
    This means that $(a-b)^3 = (nm)^3$,
    which means that $a^3 - 3a^2b + 3ab^2 - b^3 = n^3m^3$.
    We can rearrange the equation to arrive at
    \[ a^3 - b^3 = n^3m^3 + 3a^2b - 3ab^2 = n^3m^3 + 3ab(a-b) \]
    \[ = n^3m^3 + 3ab(nm) = n(n^2m^3 + 3ab) \]
    where $n^2m^3 + 3ab$ is an integer.

    \section*{Exercise 22}
    Proof that for $a \in \Z$ and $n \in \N$,
    if $r$ is the remainder of $a$ being divided by $n$,
    then $a \equiv r \mod n$: \\
    For this question, it is easier to use a direct proof. \\
    Suppose that $a \in \Z$ and $n \in \N$ such that
    if $r$ is the remainder of $a$ being divided by $n$.
    Then $a = nq + r$ for some $q \in \Z$. 
    This means that $a - r = nq$,
    which tells us that $n \mid (a-r)$.
    We thus conclude $a \equiv r \mod n$. \\

    \section*{Exercise 23}
    Proof that for $a, b, c \in \Z$ and $n \in N$,
    if $a \equiv b \mod n$,
    then $ca \equiv b \mod n$: \\
    For this question, it is easier to use a direct proof. \\
    Suppose that $a, b, c \in \Z$ and $n \in N$
    such that $a \equiv b \mod n$.
    Then $n \mid a - b$,
    which means that $a-b = np$ for some $p \in \Z$.
    This means that $c(a-b) = ca - bc = npc$
    where $pc$ is an integer.
    We can conclude that $n \mid (ca - cb)$,
    which means that $ca \equiv cb \mod n$. \\

    \section*{Exercise 24}
    Proof that for $a, b, c, d \in \Z$ and $n \in N$,
    if $a \equiv b \mod n$ and $c \equiv d \mod n$,
    then $ac \equiv bd \mod n$: \\
    For this question, it is easier to use a direct proof. \\
    Suppose that $a, b, c, d \in \Z$ and $n \in N$
    such that $a \equiv b \mod n$ and $c \equiv d \mod n$.
    Then $n \mid a - b$ and $n \mid c - d$,
    which means that $a-b = np$ and $c -d = nq$
    for some $p, q \in \Z$.
    This means that $a = np + b$ and $c = nq + d$.
    We then get
    \[ ac = (np+b)(nq+d) = n^2pq + npd + nqb + bd \]
    which we can rearrange the equation to get that
    \[ ac - bd = n(npq + pd + qb) \]
    where $npq + pd + qb$ is an integer.
    We can conclude that $n \mid (ac - bd)$,
    which means that $ac \equiv bd \mod n$. \\

    \section*{Exercise 25 $***$}
    Proof that for $n \in \N$,
    if $2^n - 1$ is prime,
    then $n$ is prime: \\
    For this question, it is easier to use a contrapositive proof. \\
    Suppose that $n \in \N$
    such that $n$ is composite (not prime).
    This means that $n = pq$ for some $p, q \in \Z$
    such that $p, q \geqslant 2$. \\
    We know that
    \[ x^k - 1 = (x-1)(x^{n-1} + x^{n-1} \dots x + 1) \]
    So we have that
    \[ 2^n - 1 = 2^{pq} - 1 = (2^p)^q - 1
    = (2^p - 1)(2^{p(q-1)} + 2^{p(q-2)} \dots 2^{2p} + 2^p) = 
    (2^{n-1} + 2^{n-1} \dots 2 + 1) \]
    Since both $(2^p - 1)$
    and $(2^{p(q-1)} + 2^{p(q-2)} \dots 2^{2p} + 2^p)$
    are integers whose value exceeds $2$,
    $2^n - 1$ is composite (not prime). \\

    \section*{Exercise 26 $***$}
    Proof that if $n = 2^k - 1$ for some $k \in \N$,
    then every entry in any row $n$
    of \textit{Pascal's Triangle} is odd: \\
    For this question, it is easier to use a direct proof. \\
    Suppose that $n = 2^k - 1$ for some $k \in \N$. \\
    We know that this is \textit{Pascal's Triangle}:
    \[
    \begin{array}{ccccccccccccc}
    &   &   &   &   & 1 &   &   &   &   &   &   &  \qquad \text{ row } 0 \\
    &   &   &   & 1 &   & 1 &   &   &   &   &   & \qquad \text{ row } 1  \\
    &   &   & 1 &   & 2 &   & 1 &   &   &   &   & \qquad \text{ row } 2  \\
    &   & 1 &   & 3 &   & 3 &   & 1 &   &   &   & \qquad \text{ row } 3 \\
    & \comb{4}{0} &   & \comb{4}{1} &   & \comb{4}{2} &   & 
    \comb{4}{1} &   & \comb{4}{0} &   &   & \qquad \text{ row } 4 \\
    &   &  &   &  & \vdots &  &  &  &   &  &   &   \\
    \comb{n}{0} &   & \comb{n}{1} &   & \comb{n}{2} & \dots &
    \comb{n}{n-2} &  & \comb{n}{n-1} &   & \comb{n}{n} &   & 
    \qquad \text{ row } n \\
    \end{array}
    \]
    When we are at row $n$,
    all entries will be of the form $\comb{n}{m}$
    where $m \in \N$ such that $0 \leqslant m \leqslant n$. \\
    First, we assert that $\comb{n}{0} = 1$
    which is obviously odd. \\
    Next, we will show that this applies
    for $1 \leqslant m \leqslant n$.
    We know that $\comb{n+1}{m} = \comb{n}{m-1} + \comb{n}{m}$
    for any $m$.
    This means that
    \[ \comb{2^k}{m} = \comb{2^k-1}{m-1} + \comb{2^k-1}{m} \]
    We know that $\comb{2^k}{m}$ is always even.
    This requires induction,
    but it is true.
    We thus know that $\comb{2^k}{1}$ is even.
    Since 
    \[ \comb{2^k}{1} = \comb{2^k-1}{0} + \comb{2^k-1}{1} \]
    and $\comb{2^k-1}{0}$ is odd,
    this means that $\comb{2^k-1}{1}$ must also be odd
    in order for $\comb{2^k}{1}$ to be even. \\
    We can repeat this argument for $\comb{2^k}{2}$,
    which will give us that $\comb{2^k-1}{2}$ is odd,
    since $\comb{2^k-1}{1}$ is odd (as we just earlier). \\
    We can complete this proof by showng that $\comb{2^k-1}{m}$
    is odd for all values $1 \leqslant m \leqslant n$
    from left to right. \\

    \section*{Exercise 27}
    Proof that for $a \in \N$,
    if $a \equiv 0 \mod 4$ or $a \equiv 1 \mod 4$,
    then $\comb{a}{2}$ is even: \\
    For this question, it is easier to use a direct proof. \\
    Suppose that $a \in \N$
    such that $a \equiv 0 \mod 4$ or $a \equiv 1 \mod 4$.
    We know that
    \[ \comb{a}{2} = \dfrac{a!}{2!(a-2)!} = \dfrac{a(a-1)}{2} \]
    If $a \equiv 0 \mod 4$,
    then $4 \mid a$,
    which means that $a = 4n$ for some $n \in \Z$.
    This tells us that 
    \[ \comb{a}{2} = \dfrac{4n(4n-1)}{2} = 2n(2n-1) \]
    where $n(2n-1)$ is an integer,
    which makes $\comb{a}{2}$ even. \\
    On the other hand, if $a \equiv 1 \mod 4$,
    then $4 \mid (a-1)$,
    which means that $a-1 = 4n$ for some $n \in \Z$,'
    which in turn means that $a = 4n + 1$.
    This tells us that 
    \[ \comb{a}{2} = \dfrac{(4n+1)(4n)}{2} = 2n(2n+1) \]
    where $n(2n+1)$ is an integer,
    which makes $\comb{a}{2}$ even. \\

    \section*{Exercise 28}
    Proof that if $n \in \Z$,
    then $4 \nmid (n^2 - 3)$: \\
    For this question, it is easier to use a direct proof. \\
    Suppose that $n \in \Z$.
    Then $n$ can be even or odd,
    so if we can show the proposition stands for both cases,
    we will have covered all of our bases. \\
    If $n$ is even,
    then $n = 2m$ for some $m \in \Z$.
    This means that
    \[ n^2 - 3 = (2m)^2 - 3 = 4m^2 - 3 = 4m^2 - 4 + 1 = 4(m^2 - 1) + 1 \]
    This means that $(n^2 - 3) - 1 = 4(m^2 - 1)$
    where $m^2 - 1$ is an integer.
    So $(n^2 - 3) \equiv 1 \mod 4$. \\
    If $n$ is odd,
    then $n = 2p+1$ for some $p \in \Z$.
    This means that
    \[ n^2 - 3 = (2p+1)^2 - 3 = 4p^2 + 4p + 1 - 3 =
    4p^2 + 4p - 4 + 2 = 4(p^2 + p - 1) + 2 \]
    This means that $(n^2 - 3) - 2 = 4(p^2 + p - 1)$
    where $p^2 + p - 1$ is an integer.
    So $(n^2 - 3) \equiv 2 \mod 4$. \\
    Either way,
    $(n^2 - 3) \not\equiv 0 \mod 4$,
    so $4 \nmid (n^2 - 3)$. \\

    \section*{Exercise 29}
    Proof that for $a, b \in \Z$,
    if $a$ and $b$ aren't both $0$,
    then $\gcd(a, b) = \gcd(a-b, b)$: \\
    For this question, it is easier to use a direct proof. \\
    Suppose that $a, b \in \Z$
    such that $a$ and $b$ aren't both $0$.
    Let's take $g_1 = \gcd(a, b)$ and $g_2 = \gcd(a - b, b)$.
    To show that they are equal,
    we only need to show that $g_1 \leqslant g_2$
    and $g_2 \leqslant g_1$. \\
    We know that $g_1 \mid a$ and $g_1 \mid b$,
    which means that $a = ng_1$ and $b = mg_1$
    for some $n, m \in \Z$.
    This means that $a - b = ng_1 - mg_1 = (n-m)g_1$,
    where $n-m$ is an integer,
    which tells us that $g_1 \mid (a-b)$.
    Since $g_2$ is the greatest integer that
    divides $b$ and $a-b$
    it must be that $g_1 \leqslant g_2$. \\
    By the same argument,
    we know that $g_2 \mid b$ and $g_2 \mid (a-b)$,
    which means that $b = pg_2$ and $a-b = qg_2$
    for some $p, q \in \Z$.
    This means that $a - b + b = a = pg_2 + qg_2 = (p+q)g_2$,
    where $p+q$ is an integer,
    which tells us that $g_2 \mid a$.
    Since $g_1$ is the greatest integer that
    divides $a$ and $b$
    it must be that $g_2 \leqslant g_1$. \\
    We can thus conclude that $g_1 = g_2$. \\

    \section*{Exercise 30}
    Proof that for $a, b \in \Z$ and $n \in \N$,
    if $a \equiv b \mod n$,
    then $\gcd(a, n) = \gcd(b, n)$: \\
    For this question, it is easier to use a direct proof. \\
    Suppose that $a, b \in \Z$ and $n \in \N$
    such that $a \equiv b \mod n$.
    Let's take $g_1 = \gcd(a, n)$ and $g_2 = \gcd(b, n)$.
    To show that they are equal,
    we only need to show that $g_1 \leqslant g_2$
    and $g_2 \leqslant g_1$. \\
    Since $a \equiv b \mod n$,
    we have that $n \mid (a-b)$,
    which means that $a-b = nm$
    for some $m \in \Z$.
    We know that $g_1 \mid a$ and $g_1 \mid n$.
    This means that $a = pg_1$ and $n = qg_1$
    for some $p, q \in \Z$.
    So we can rewrite $a-b = nm$
    as $b = pg_1 - qg_1m = g_1(p - qm)$
    where $p - qm$ is an integer.
    So $g_1 \mid b$.
    Since $g_2$ is the greatest integer that
    divides $n$ and $b$
    it must be that $g_1 \leqslant g_2$. \\
    On the other hand, 
    we know that $g_2 \mid b$ and $g_2 \mid n$,
    so we can write $b = g_2r$ and $b = g_2s$
    for some $r, s \in \Z$. 
    So we can write $a-b = nm$
    as $a = rg_2 - sg_2m = g_2(r - sm)$
    where $r - sm$ is an integer.
    So $g_2 \mid a$.
    Since $g_1$ is the greatest integer that
    divides $n$ and $a$
    it must be that $g_2 \leqslant g_1$. \\
    We can thus conclude that $g_1 = g_2$. \\

    \section*{Exercise 31}
    Proof that for $a, b \in \Z$,
    if the division algorithm is applied such that $a = qb + r$
    where $0 \leqslant r < b$ for  $r, q \in \Z$,
    then $\gcd(a, b) = \gcd(r, b)$: \\
    For this question, it is easier to use a direct proof. \\
    Suppose that $a, b \in \Z$.
    By the division algorithm, $a = qb + r$
    where $0 \leqslant r < b$ for  $r, q \in \Z$. \\
    Let's take $g_1 = \gcd(a, b)$ and $g_2 = \gcd(r, b)$.
    To show that they are equal,
    we only need to show that $g_1 \leqslant g_2$
    and $g_2 \leqslant g_1$. \\
    We know that $g_1 \mid a$ and $g_1 \mid b$.
    Then $a = ng_1$ and $b = mg_1$
    for some $n, m \in \Z$.
    We know that $r = a - qb$,
    so $r = ng_1 - qmg_1 = g_1(n - qm)$
    where $n - qm$ is an integer.
    So $g_1 \mid r$.
    Since $g_2$ is the greatest integer that
    divides $r$ and $b$
    it must be that $g_1 \leqslant g_2$. \\
    On the other hand, we know that $g_2 \mid r$ and $g_2 \mid b$.
    Then $r = pg_2$ and $b = sg_2$
    for some $p, s \in \Z$.
    We know that $a = qb + r$,
    so $a = qsg_2 + pg_2 = g_2(qs + p)$
    where $qs + p$ is an integer.
    So $g_2 \mid a$.
    Since $g_1$ is the greatest integer that
    divides $a$ and $b$
    it must be that $g_2 \leqslant g_1$. \\
    We can thus conclude that $g_1 = g_2$. \\


\end{document}
