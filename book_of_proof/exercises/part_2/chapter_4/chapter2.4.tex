
\documentclass[12pt]{article}
\usepackage[margin=1in]{geometry}


%===============================================================================
%================================== PACKAGES ===================================
%===============================================================================

% For using float option H that places figures 
% exatcly where we want them
\usepackage{float}
% makes figure font bold
\usepackage{caption}
\captionsetup[figure]{labelfont=bf}
% For text generation
\usepackage{lipsum}
% For drawing
\usepackage{tikz}
% For smaller or equal sign and not divide sign
\usepackage{amssymb}
% For the diagonal fraction
\usepackage{xfrac}
% For enumerating exercise parts with letters instead of numbers
\usepackage{enumitem}
% For dfrac, which forces the fraction to be in display mode (large) e
% even in math mode (small)
\usepackage{amsmath}
% For degree sign
\usepackage{gensymb}
% For "\mathbb" macro
\usepackage{amsfonts}
% For positioning 
\usepackage{indentfirst}
\usetikzlibrary{shapes,positioning,fit,calc}
% for adjustwidth environment
\usepackage{changepage}
% for arrow on top
\usepackage{esvect}
% for mathbb 1
\usepackage{bbm}
% for mathsrc
\usepackage[mathscr]{eucal}
% For degree sign
\usepackage{gensymb}
% For quotes
\usepackage{csquotes}
% For vertical lines
\usepackage{mathtools}
% For cols
\usepackage{multicol}

% for tikz
\usepackage{pgfplots}
\pgfplotsset{compat=1.18}
\usepackage{amsmath}
\usepgfplotslibrary{groupplots}


%===============================================================================
%==================================== FONTS ====================================
%===============================================================================


% Mathcal
\newcommand{\acal}{\mathcal{A}}
\newcommand{\bcal}{\mathcal{B}}
\newcommand{\ccal}{\mathcal{C}}
\newcommand{\dcal}{\mathcal{D}}
\newcommand{\ecal}{\mathcal{E}}
\newcommand{\fcal}{\mathcal{F}}
\newcommand{\gcal}{\mathcal{G}}
\newcommand{\hcal}{\mathcal{H}}
\newcommand{\ical}{\mathcal{I}}
\newcommand{\jcal}{\mathcal{J}}
\newcommand{\kcal}{\mathcal{K}}
\newcommand{\lcal}{\mathcal{L}}
\newcommand{\mcal}{\mathcal{M}}
\newcommand{\ncal}{\mathcal{N}}
\newcommand{\ocal}{\mathcal{O}}
\newcommand{\pcal}{\mathcal{P}}
\newcommand{\qcal}{\mathcal{Q}}
\newcommand{\rcal}{\mathcal{R}}
\newcommand{\scal}{\mathcal{S}}
\newcommand{\tcal}{\mathcal{T}}
\newcommand{\ucal}{\mathcal{U}}
\newcommand{\vcal}{\mathcal{V}}
\newcommand{\wcal}{\mathcal{W}}
\newcommand{\xcal}{\mathcal{X}}
\newcommand{\ycal}{\mathcal{Y}}
\newcommand{\zcal}{\mathcal{Z}}

% Mathfrak
\newcommand{\afrak}{\mathfrak{A}}
\newcommand{\bfrak}{\mathfrak{B}}
\newcommand{\cfrak}{\mathfrak{C}}
\newcommand{\dfrak}{\mathfrak{D}}
\newcommand{\efrak}{\mathfrak{E}}
\newcommand{\ffrak}{\mathfrak{F}}
\newcommand{\gfrak}{\mathfrak{G}}
\newcommand{\hfrak}{\mathfrak{H}}
\newcommand{\ifrak}{\mathfrak{I}}
\newcommand{\jfrak}{\mathfrak{J}}
\newcommand{\kfrak}{\mathfrak{K}}
\newcommand{\lfrak}{\mathfrak{L}}
\newcommand{\mfrak}{\mathfrak{M}}
\newcommand{\nfrak}{\mathfrak{N}}
\newcommand{\ofrak}{\mathfrak{O}}
\newcommand{\pfrak}{\mathfrak{P}}
\newcommand{\qfrak}{\mathfrak{Q}}
\newcommand{\rfrak}{\mathfrak{R}}
\newcommand{\sfrak}{\mathfrak{S}}
\newcommand{\tfrak}{\mathfrak{T}}
\newcommand{\ufrak}{\mathfrak{U}}
\newcommand{\vfrak}{\mathfrak{V}}
\newcommand{\wfrak}{\mathfrak{W}}
\newcommand{\xfrak}{\mathfrak{X}}
\newcommand{\yfrak}{\mathfrak{Y}}
\newcommand{\zfrak}{\mathfrak{Z}}

% Mathscr
\newcommand{\ascr}{\mathscr{A}}
\newcommand{\bscr}{\mathscr{B}}
\newcommand{\cscr}{\mathscr{C}}
\newcommand{\dscr}{\mathscr{D}}
\newcommand{\escr}{\mathscr{E}}
\newcommand{\fscr}{\mathscr{F}}
\newcommand{\gscr}{\mathscr{G}}
\newcommand{\hscr}{\mathscr{H}}
\newcommand{\iscr}{\mathscr{I}}
\newcommand{\jscr}{\mathscr{J}}
\newcommand{\kscr}{\mathscr{K}}
\newcommand{\lscr}{\mathscr{L}}
\newcommand{\mscr}{\mathscr{M}}
\newcommand{\nscr}{\mathscr{N}}
\newcommand{\oscr}{\mathscr{O}}
\newcommand{\pscr}{\mathscr{P}}
\newcommand{\qscr}{\mathscr{Q}}
\newcommand{\rscr}{\mathscr{R}}
\newcommand{\sscr}{\mathscr{S}}
\newcommand{\tscr}{\mathscr{T}}
\newcommand{\uscr}{\mathscr{U}}
\newcommand{\vscr}{\mathscr{V}}
\newcommand{\wscr}{\mathscr{W}}
\newcommand{\xscr}{\mathscr{X}}
\newcommand{\yscr}{\mathscr{Y}}
\newcommand{\zscr}{\mathscr{Z}}

% Mathbb
\newcommand{\abb}{\mathbb{A}}
\newcommand{\bbb}{\mathbb{B}}
\newcommand{\cbb}{\mathbb{C}}
\newcommand{\dbb}{\mathbb{D}}
\newcommand{\ebb}{\mathbb{E}}
\newcommand{\fbb}{\mathbb{F}}
\newcommand{\gbb}{\mathbb{G}}
\newcommand{\hbb}{\mathbb{H}}
\newcommand{\ibb}{\mathbb{I}}
\newcommand{\jbb}{\mathbb{J}}
\newcommand{\kbb}{\mathbb{K}}
\newcommand{\lbb}{\mathbb{L}}
\newcommand{\mbb}{\mathbb{M}}
\newcommand{\nbb}{\mathbb{N}}
\newcommand{\obb}{\mathbb{O}}
\newcommand{\pbb}{\mathbb{P}}
\newcommand{\qbb}{\mathbb{Q}}
\newcommand{\rbb}{\mathbb{R}}
\newcommand{\sbb}{\mathbb{S}}
\newcommand{\tbb}{\mathbb{T}}
\newcommand{\ubb}{\mathbb{U}}
\newcommand{\vbb}{\mathbb{V}}
\newcommand{\wbb}{\mathbb{W}}
\newcommand{\xbb}{\mathbb{X}}
\newcommand{\ybb}{\mathbb{Y}}
\newcommand{\zbb}{\mathbb{Z}}


%===============================================================================
%=============================== SPECIAL SYMBOLS ===============================
%===============================================================================


% Orbit (group theory)
\newcommand{\orbit}{\mathcal{O}}
% Normal group
\newcommand{\normal}{\mathcal{N}}
% Indicator function
\newcommand{\indicator}{\mathbbm{1}}
% Laplace transform
\newcommand{\laplace}[1]{\mathcal{L}}
% Epsilon shorthand
\newcommand{\eps}{\varepsilon}
% Omega
\newcommand{\om}{\omega}
\newcommand{\Om}{\Omega}

%===============================================================================
%================================== OPERATORS ==================================
%===============================================================================


% Inverse exponent
\newcommand{\inv}[0]{^{-1}}
% Overline bar
\newcommand{\olsi}[1]{\,\overline{\!{#1}}}
% Less than or equal slanted
\newcommand{\seqs}{\leqslant}
% Greater or equal slanted
\newcommand{\geqs}{\geqslant}
% Subset or equal
\newcommand{\sub}{\subseteq}
% Proper subset
\newcommand{\prosub}{\subset}
% from
\newcommand{\from}{\leftarrow}

% Parantheses
\newcommand{\para}[1]{\left( #1 \right)}
% Curly Braces
\newcommand{\curl}[1]{\left\{ #1 \right\}}
% Brackets
\newcommand{\brac}[1]{\left[ #1 \right]}
% Angled Brackets
\newcommand{\ang}[1]{\left\langle #1 \right\rangle}
% Norm
\newcommand{\norm}[1]{\left\| #1 \right\|}

% Piece wise (use \\ between cases)
\newcommand{\piecewise}[1]{\begin{cases} #1 \end{cases}}

% Bold symbol shorthand
\newcommand{\bl}{\boldsymbol}

% Vertical space
\newcommand{\vs}[1]{\vspace{#1 pt}}
% Horizontal ertical space
\newcommand{\hs}[1]{\hspace{#1 pt}}


%===============================================================================
%============================== TEXT BASED SYMBOLS =============================
%===============================================================================


% Radians
\newcommand{\rad}{\text{rad}}
% Least Common Multiple
\newcommand{\lcm}{\text{lcm}}
% Automorphism
\newcommand{\Aut}{\text{Aut}}
% Variance
\newcommand{\var}{\text{Var}}
% Covariance
\newcommand{\cov}{\text{Cov}}
% Cofactor (matrix)
\newcommand{\cof}{\text{Cof}}
% Adjugate (matrix)
\newcommand{\adj}{\text{Adj}}
% Trace (matrix)
\newcommand{\tr}{\text{tr}}
% Standard deviation
\newcommand{\std}{\text{Std}}
% Correlation coefficient
\newcommand{\corr}{\text{Corr}}
% Sign
\newcommand{\sign}{\text{sign}}

% And text
\newcommand{\AND}{\text{ and }}
% Or text 
\newcommand{\OR}{\text{ or }}
% For text 
\newcommand{\FOR}{\text{ for }}
% If text
\newcommand{\IF}{\text{ if }}
% When text
\newcommand{\WHEN}{\text{ when }}
% Where text
\newcommand{\WHERE}{\text{ where }}
% Then text
\newcommand{\THEN}{\text{ then }}
% Such that text
\newcommand{\SUCHTHAT}{\text{ such that }}

% 1st
\newcommand{\st}[1]{#1^{\text{st}}}
% 2nd
\newcommand{\nd}[1]{#1^{\text{nd}}}
% 3rd
\newcommand{\rd}[1]{#1^{\text{rd}}}
% nth
\newcommand{\nth}[1]{#1^{\text{th}}}


%===============================================================================
%========================= PROBABILITY AND STATISTICS ==========================
%===============================================================================


% Permutation
\newcommand{\perm}[2]{{}^{#1}\!P_{#2}}
% Combination
\newcommand{\comb}[2]{{}^{#1}C_{#2}}

% Baye's risk
\newcommand{\risk}[1]{\mathscr{R}_{#1}}
% Baye's optimal risk
\newcommand{\riskOptimal}[1]{\mathscr{R}_{#1}^*}
% Baye's empirical risk
\newcommand{\riskEmpirical}[2]{\hat{\mathscr{R}}_{#1}^{#2}}


%===============================================================================
%=================================== CALCULUS ==================================
%===============================================================================


% d over d derivative
\newcommand{\dd}[2]{\dfrac{d#1}{d#2}}
% partial d over d derivative
\newcommand{\partialdd}[2]{\dfrac{\partial #1}{\partial #2}}
% delta d over d derivative
\newcommand{\deltadd}[2]{\dfrac{\Delta #1}{\Delta #2}}

% Integration between a and b
\newcommand{\integral}[4]{\int_{#1}^{#2} #3 \, #4}
% Integration in some space 
\newcommand{\boundIntegral}[2]{\int_{#1} #2 \, d#1}

% Limit
\newcommand{\limit}[3]{\lim_{#1 \to #2} #3}


%===============================================================================
%================================  BIG SYMBOLS  ================================
%===============================================================================


% Sum
\newcommand{\sumof}[2]{\sum_{#1}^{#2}}
% Product
\newcommand{\productof}[2]{\prod_{#1}^{#2}}
% Union
\newcommand{\unionof}[2]{\bigcup_{#1}^{#2}}
% Intersection
\newcommand{\intersectionof}[2]{\bigcap_{#1}^{#2}}
% Or
\newcommand{\orof}[2]{\bigvee_{#1}^{#2}}
% And
\newcommand{\andof}[2]{\bigwedge_{#1}^{#2}}


%===============================================================================
%=============================== LINEAR ALGEBRA ================================
%===============================================================================


% Bold vector arrow
\newcommand{\bv}[1]{\vec{\mathbf{#1}}}

% Matrix or vector (use // for column, & for row) 
% with brackets
\newcommand{\bmat}[1]{\begin{bmatrix} #1 \end{bmatrix}}
% Matrix or vector (use // for column, & for row) 
% with curved brackets
\newcommand{\pmat}[1]{\begin{pmatrix} #1 \end{pmatrix}}
% Matrix or vector (use // for column, & for row) 
% with lines on either side 
\newcommand{\lmat}[1]{\begin{vmatrix} #1 \end{vmatrix}}
% Matrix or vector (use // for column, & for row) 
% with curly braces
\newcommand{\cmat}[1]{\begin{Bmatrix} #1 \end{Bmatrix}}
% Matrix or vector (use // for column, & for row) 
% with no braces
\newcommand{\mat}[1]{\begin{matrix} #1 \end{matrix}}


%===============================================================================
%================================ LARGE OBJECTS ================================
%===============================================================================

% Multiple lines
\newcommand{\multiline}[1]{
\begin{align*}
    #1
\end{align*}
}

% Multiple lines with equation numbers
\newcommand{\eqmultiline}[1]{
\begin{align*}
    #1
\end{align*}
}

% Color
\newcommand{\colorText}[2]{
\begingroup
\color{#1}
    #2
\endgroup
}

% Centered figure
\newcommand{\centerFigure}[2]{
    \begin{figure}[h]
        \centering
            #1
        \caption{#2}
    \end{figure}
}

% Tikz figure
\newcommand{\tikzGraphic}[1]{
    \begin{center}
    \begin{tikzpicture}
        #1
    \end{tikzpicture}
    \end{center}
}

% Enumerate numbers (seperate by \item)
\newcommand{\numbers}{\textbf{\number*)}}

% Enumerate letters (seperate by \item)
\newcommand{\letters}{\textbf{\alph*)}}

\title{%
    \Huge Book of Proof \\
    \large by \\
    \Large Richard Hammack \\~\\
    \huge Part 2: How to Prove a Conditional Statement \\
    \LARGE Chapter 4: Direct Proof
}
\date{2024-03-17}
\author{Michael Saba}

\begin{document}
    \pagenumbering{gobble}
    \maketitle
    \newpage
    \setlength{\parindent}{0pt}
    \pagenumbering{arabic}

    For questions 1 - 28, we have to use a direct proof. \\

    \section*{Exercise 1}
    Proof that if $x$ is even,
    then $x^2$ is even: \\
    Suppose that $x$ is an even integer.
    Then $x = 2n$ for some $n \in \Z$.
    So $x^2 = (2n)^2 = (2n)^2 = 4n^2 = 2(2n^2)$
    where $2n^2$ is an integer.
    So $x^2$ is even.

    \section*{Exercise 2}
    Proof that if $x$ is odd,
    then $x^3$ is odd: \\
    Suppose that $x$ is an odd integer.
    Then $x = 2n+1$ for some $n \in \Z$.
    So
    \[ x^3 = (2n+1)^3 = 8n^3 + 4n^2 + 2n + 1 = 2(4n^3 + 2n^2 + n) + 1 \]
    where $4n^3 + 2n^2 + n$ is an integer.
    So $x^3$ is odd. \\

    \section*{Exercise 3}
    Proof that if $a$ is odd,
    then $a^3 + 3a + 5$ is odd: \\
    Suppose that $a$ is an odd integer.
    Then $a = 2n+1$ for some $n \in \Z$.
    So 
    \[ a^3 + 3a + 5 = (2n+1)^3 + 3(2n+1) + 5 \]
    \[ = 8n^3 + 4n^2 + 8n + 9 = 2(4n^3 + 2n^2 + 4n + 4) + 1 \]
    where $4n^3 + 2n^2 + 4n + 4$ is an integer.
    So $a^3 + 3a + 5$ is odd. \\

    \section*{Exercise 4}
    Proof that for $x, y \in \Z$,
    $xy$ is odd if $x$ and $y$ are odd: \\
    Suppose that $x$ and $y$ are odd integers.
    Then $x = 2n+1$ and $y = 2m+1$ for some $n, m \in \Z$.
    So 
    \[ xy = (2n+1)(2m+1) = 4nm + 2n + 2m + 1 = 2(2nm + n + m) + 1\]
    where $2nm + n + m$ is an integer.
    So $xy$ is odd. \\

    \section*{Exercise 5}
    Proof that for $x, y \in \Z$,
    $xy$ is even if $x$ is even: \\
    Suppose that $x$ and $y$ are integers
    such that $x$ is even. 
    Then $x = 2n$ and $y = 2m+1$ for some $n, m \in \Z$.
    So 
    \[ xy = (2n)(2m+1) = 4nm + 2n = 2(2nm + n)\]
    where $2nm + n$ is an integer.
    So $xy$ is even. \\

    \section*{Exercise 6}
    Proof that for $a, b, c \in \Z$,
    if $a \mid b$ and $a \mid c$,
    then $a \mid (b+c)$: \\
    Suppose that $a, b, c$ are integers
    such that $a \mid b$ and $a \mid c$.
    Then $b = an$ and $c = am$ for some $n, m \in \Z$.
    So $b+c = an + am = a(n + m)$,
    which means that $a \mid (b+c)$. \\

    \section*{Exercise 7}
    Proof that for $a, b \in \Z$,
    if $a \mid b$,
    then $a^2 \mid b^2$: \\
    Suppose that $a, b$ are integers
    such that $a \mid b$.
    Then $b = an$ for some $n \in \Z$.
    So $b^2 = (an)^2 = a^2n^2$,
    which means that $a^2 \mid b^2$. \\

    \section*{Exercise 8}
    Proof that for $a \in \Z$,
    if $5 \mid 2a$,
    then $5 \mid a$: \\
    Suppose that $a \in \Z$
    such that $5 \mid 2a$.
    Then $2a = 5n$ for some $n \in \Z$.
    So $5n$ must be even,
    which means that it contains a factor of $2$.
    But $5$ is prime and can't be factored further,
    so $2$ must be a factor of $n$, making it even,
    such that $n = 2m$ for some $m \in \Z$.
    So $2a = 5 \cdot 2m$,
    which means that $a = 5m$.
    So $5 \mid a$. \\

    \section*{Exercise 9}
    Proof that for $a \in \Z$,
    if $7 \mid 4a$,
    then $7 \mid a$: \\
    Suppose that $a \in \Z$
    such that $7 \mid 4a$.
    Then $4a = 7n$ for some $n \in \Z$.
    So $7n$ must have a factor of $4$.
    But $7$ is prime and can't be factored further,
    so $4$ must be a factor of $n$,
    such that $n = 4m$ for some $m \in \Z$.
    So $4a = 7 \cdot 4m$,
    which means that $a = 7m$.
    So $7 \mid a$. \\

    \section*{Exercise 10}
    Proof that for $a, b \in \Z$,
    if $a \mid b$,
    then $a \mid (3b^3 - b^2 + 5b)$: \\
    Suppose that $a \in \Z$
    such that $a \mid b$.
    Then $b = an$ for some $n \in \Z$.
    So 
    \[ 3b^3 - b^2 + 5b = 3(an)^3 - (an)^2 + 5(an)
    = 3a^3n^3 - a^2n^2 + 5an = a(3a^2n^3 - an^2 + 5n) \]
    where $3a^2n^3 - an^2 + 5n$,
    which means that $a \mid (3b^3 - b^2 + 5b)$. \\

    \section*{Exercise 11}
    Proof that for $a, b, c, d \in \Z$,
    if $a \mid b$ and $c \mid d$,
    then $ac \mid bd$: \\
    Suppose that $a, b, c, d \in \Z$
    such that $a \mid b$ and $c \mid d$.
    Then $b = an$ and $d = cm$ for some $a, c \in \Z$.
    So $bd = ancm = ac(nm)$,
    which means that $ac \mid bd$. \\

    \section*{Exercise 12}
    Proof that for $x \in \R$,
    if $0 < x < 4$,
    then $\dfrac{4}{x(4-x)} \geqslant 1$: \\
    Suppose that $x \in \R$
    such that $0 < x < 4$.
    We know that $(x-2)^2 \geqslant 0$ since all squares are positive.
    We can then develop the term and get
    that $x^2 - 4x + 4 \geqslant 0$,
    which means that $4 \geqslant 4x - x^2$,
    which in turn means that $4 \geqslant x(4 - x)$.
    We can thus conclude that $\dfrac{4}{x(4-x)} \geqslant 1$
    as the numertaor is larger than the denominator.
    (Note that this ended up being the case
    regardless of the fact that $0 < x < 4$). \\

    \section*{Exercise 13}
    Proof that for $x, y \in \R$,
    if $x^2+5y = y^2+5x$,
    then $x = y$ or $x+y = 5$: \\
    Suppose that $x, y \in \R$
    such that $x^2+5y = y^2+5x$.
    Then $x^2 - y^2 = 5x - 5y$,
    which we can restructure to get $(x+y)(x-y) = 5(x - y)$.
    If $x-y$ is $0$,
    then both sides of the equations are $0$,
    and it implies that $x = y$.
    On the other hand, if $x-y \neq 0$,
    then we can simplify by $x-y$,
    and get that $x+y = 5$. \\
    So either $x = y$ or $x+y = 5$. \\

    \section*{Exercise 14}
    Proof that if $n \in \Z$,
    then $5n^2 + 3n + 7$ is odd: \\
    Suppose that $n \in \Z$.
    We know that $n$ is awlays either even or odd,
    so if we can prove the statement for both cases,
    we've then covered all of our bases. \\
    If $n$ is even,
    then $n = 2m$ for some $m \in \Z$.
    This means that 
    \[ 5n^2 + 3n + 7 = 5(2m)^2 + 3(2m) + 7 = 20m^2 + 6m + 7
    = 2(10m^2 + 3m + 3) + 1 \]
    where $10m^2 + 3m + 3$ is an integer,
    which means that $5n^2 + 3n + 7$ is odd. \\
    On the other hand, if $n$ is odd,
    then $n = 2p + 1$ for some $p \in \Z$.
    This means that 
    \[ 5n^2 + 3n + 7 = 5(2p+1)^2 + 3(2p+1) + 7
    = 20(4p^2 + 4p + 1) + 6(2p+1) + 7 \]
    \[ = 80p^2 + 92p + 33
    = 2(40p^2 + 46p + 16) + 1 \]
    where $40p^2 + 46p + 16$ is an integer,
    which means that $5n^2 + 3n + 7$ is odd. \\

    \section*{Exercise 15}
    Proof that if $n \in \Z$,
    then $n^2 + 3n + 4$ is even: \\
    Suppose that $n \in \Z$.
    We know that $n$ is awlays either even or odd,
    so if we can prove the statement for both cases,
    we've then covered all of our bases. \\
    If $n$ is even,
    then $n = 2m$ for some $m \in \Z$.
    This means that 
    \[ n^2 + 3n + 4 = (2m)^2 + 3(2m) + 4 = 4m^2 + 6m + 4
    = 1(2m^2 + 3m + 2) \]
    where $2m^2 + 3m + 2$ is an integer,
    which means that $n^2 + 3n + 4$ is even. \\
    On the other hand, if $n$ is odd,
    then $n = 2p + 1$ for some $p \in \Z$.
    This means that 
    \[ n^2 + 3n + 4 = (2p+1)^2 + 3(2p+1) + 4 = 4m^2 + 10m + 8
    = 2(2m^2 + 5m + 4) \]
    where $2m^2 + 5m + 4$ is an integer,
    which means that $n^2 + 3n + 4$ is even. \\

    \section*{Exercise 16}
    Proof that for $x, y \in \Z$,
    if $x$ and $y$ have the same parity,
    then $x+y$ is even: \\
    Suppose that $x, y \in \Z$. \\
    If $x$ and $y$ are both even,
    then $x = 2n$ and $y=2m$ for some $n, m \in \Z$.
    So $x+y = 2n + 2m = 2(n+m)$
    where $n+m$ is an integer,
    so $x+y$ is even. \\
    On the other hand, 
    if $x$ and $y$ are both odd,
    then $x = 2p+1$ and $y=2q+1$ for some $p, q \in \Z$.
    So $x+y = 2p+1 + 2q+1 = 2(p+q+1)$
    where $p+q+1$ is an integer,
    so $x+y$ is even. \\

    \section*{Exercise 17}
    Proof that for $x, y \in \Z$,
    if $x$ and $y$ have opposite parity,
    then $xy$ is even: \\
    Suppose that $x, y \in \Z$
    such that they have opposite parity.
    We can assume, without loss of generality,
    that $x$ is even and $y$ is odd
    (the prove would be the same otherwise). 
    Then $x = 2n$ and $y=2m+1$ for some $n, m \in \Z$.
    So $xy = 2n(2m+1) = 4nm + 2n = 2(2nm + n)$
    where $2nm + n$ is an integer,
    so $xy$ is even. \\

    \section*{Exercise 18}
    Proof that for $x, y \in \R$ where $x, y > 0$,
    if $x > y$,
    then $x^2 < y^2$: \\
    Suppose that $x, y \in \R$
    such that $x, y > 0$ and $x > y$.
    Since $x > 0$,
    we can multiply both sides of the equation by $x$
    without changing the direction of the operator,
    arriving at $x^2 > xy$.
    Likewise, since $y > 0$,
    we can multiply both sides of the equation by $y$
    without changing the direction of the operator,
    arriving at $xy > y^2$.
    So since $x^2 > xy > y^2$,
    it follows that $x^2 > y^2$. \\
   
    \section*{Exercise 19}
    Proof that for $a, b, c \in \Z$,
    if $a^2 \mid b$ and $b^3 \mid c$,
    then $a^6 \mid c$: \\
    Suppose that $a, b, c \in \Z$
    such that $a^2 \mid b$ and $b^3 \mid c$.
    Then $b = a^2n$ and $c = b^3m$ for some $n, m \in \Z$.
    So $c = (a^2n)^3m = a^6n^3m$
    where $n^3m$ is an integer.
    This means that $a^6 \mid c$. \\

    \section*{Exercise 20}
    Proof that for $a \in \Z$,
    if $a^2 \mid a$, 
    then $a \in \{-1, 0, 1\}$: \\
    Suppose that $a \in \Z$
    such that $a^2 \mid a$.
    Then $a = a^2n$ for some $n \in \Z$. \\
    If $a = 0$, the the equation holds. \\
    But if $a \neq 0$,
    then we could simplifty by $a$,
    and get that $1 = an$,
    which means that $n = \sfrac{1}{a}$,
    (where $a$ can be the denominator since we assumed $a \neq 0$).
    Since $n$ is an integer,
    it must be that $a = \pm 1$. \\
    So $a = 0$ or $a = 1$ or $1 = -1$,
    which means that $a \in \{-1, 0, 1\}$. \\

    \section*{Exercise 21 $***$}
    Proof that if $p$ is prime and $k \in \Z$
    such that $0 < k < p$,
    then $p \mid \comb{p}{k}$: \\
    Suppose that $p$ is prime and $k \in \Z$
    such that $0 < k < p$.
    Then 
    \[ \comb{p}{k} = \dfrac{p!}{k!(p-k)!} \]
    Since $p > k$ and $p > p - k$,
    and since $p$ is prime
    (meaning it is not the combination of several smaller factors),
    we conclude that the factor $p$ does not appear in $k!$ or $p-k!$.
    We know that $\dfrac{p!}{k!(p-k)!}$ is an integer,
    which means that $k!(p-k)!$ appear as factors in $p!$.
    But since $p$ as a factor does not appear in $k!$ or $(p-k)!$,
    we assert that $p$ and $k!(p-k)!$ have no factors in common.
    We know that $p! = p(p-1)!$,
    so since $k!(p-k)! \mid p!$
    and $p$ and $k!(p-k)!$ have no factors in common,
    it must be that $k!(p-k)!$ is a factor in $(p-1)!$,
    which means that $k!(p-k)! \mid (p-1)!$. \\
    So 
    \[ \comb{p}{k} = p\dfrac{(p-1)!}{k!(p-k)!} \]
    where
    \[ \dfrac{(p-1)!}{k!(p-k)!} \]
    is an integer.
    This means that $p \mid \comb{p}{k}$. \\

    \section*{Exercise 22}
    Proof that if $n \in \N$,
    then $n^2 = 2(\comb{n}{2}) + \comb{n}{1}$: \\
    Suppose that $n \in \N$ (a positive integer). \\
    We will consider first the case where $n = 1$,
    where $\comb{n}{1} = \comb{1}{1} = 1$
    and $\comb{n}{2} = \comb{1}{2} = 0$
    (there is no way to choose $2$ out of $1$).
    So $n^2 = 1^2 = 1$,
    and $2(\comb{n}{2}) + \comb{n}{1} = 0 + 1 = 1$,
    which means that $n^2 = 2(\comb{n}{2}) + \comb{n}{1}$. \\
    On the other hand,
    if $n \geqslant 2$,
    then 
    \[ \comb{n}{1} = \comb{n}{1} = n \]
    (there are exactly $n$ ways to choose a single element from $n$).
    and
    \[ \comb{n}{2} = \dfrac{n!}{2!(n-2)!}
    = \dfrac{n(n-1)}{2} \].
    So we have
    \[ 2(\comb{n}{2}) + \comb{n}{1} = n(n-1) + n = n^2 + n - n = n^2 \]
    which completes the proof. \\

    \section*{Exercise 23 $***$}
    Proof that if $n \in \N$,
    then $\comb{2n}{n}$ is even: \\
    Suppose that $n \in \N$.
    Thwn $\comb{2n}{n}$ is the number of ways we can choose $n$
    numbers out of $2n$.
    Everytime we choose $n$ numbers,
    we effectively build a set $X_i$ containing these numbers.
    Then all the numbers we did not pick will be in a set $\olsi{X_i}$.
    As such, we could pair each set $X_i$,
    with its compelement $\olsi{X_i}$,
    where $\olsi{X_i}$ has a length of $n$ as well.
    Since $\olsi{X_i}$ is itself
    a way to choose $n$ numbers from $2n$,
    it will contribute a combination in $\comb{2n}{n}$.
    Since each combination $X_i$ in $\comb{2n}{n}$ can be paired
    with another combination in $\comb{2n}{n}$,
    we conclude that $\comb{2n}{n}$ must be even. \\

    \section*{Exercise 24}
    Proof that for $n \in \Z$,
    if $n \geqslant 2$,
    then $n! + 2, n! + 3 \dots n! + n$ are all compositie: \\
    Suppose that $n \in \Z$
    such that $n \geqslant 2$,
    Then for $m \in \Z$ 
    such that $2 \leqslant m \leqslant n$,
    we will have
    \[ n! = 2 \cdot 3 \dots m  \dots (n-1) \cdot n \]
    Then
    \[ n! + m = 2 \cdot 3 \dots m  \dots (n-1) \cdot n + m
    =  m (2 \cdot 3 \dots m - 1 \cdot m + 1  \dots (n-1) \cdot n) \]
    Since both $m$
    and $(2 \cdot 3 \dots m - 1 \cdot m + 1  \dots (n-1) \cdot n)$
    are integers larger than $1$,
    this shows that $n! + m$ is composite. \\

    \section*{Exercise 25}
    Proof that for $a, b, c \in \Z$,
    if $c \leqslant b \leqslant a$,
    then $\comb{a}{b} \times \comb{b}{c}
    = \comb{a}{b-c} \times \comb{a-b+c}{b}$: \\
    Suppose that $a, b, c \in \Z$
    such that $c \leqslant b \leqslant a$.
    Then
    \[ \comb{a}{b} \times \comb{b}{c} \]
    \[ = \dfrac{a!}{b!(a-b)!}\dfrac{b!}{c!(b-c)!} \]
    \[ = \dfrac{a!}{(a-b)!(b-c)!c!} \]
    Which means that
    \[ \comb{a}{b-c} \times \comb{a-b+c}{c} \]
    \[ = \dfrac{a!}{(b-c)!(a-b+c)!}\dfrac{(a-b+c)!}{c!(a-c)!} \]
    \[ = \dfrac{a!}{(b-c)!(a-b)!c!} = \comb{a}{b} \times \comb{b}{c} \]
    completing the proof. \\

    \section*{Exercise 26}
    Proof that all odd integers are the difference of two squares: \\
    Suppose that $a \in \Z$
    such that $a$ is odd.
    Then $a = 2n+1$ for some $n \in \Z$.
    Consider now the squares $n^2$ and $(n+1)^2$.
    We know that $(n+1)^2 - n^2 = n^2 + 2n + 1 - n^2 = 2n+1$.
    So any arbitrary odd integer can be written
    as the difference of two squares. \\

    \section*{Exercise 27}
    Proof that for $a, b \in \N$,
    if $\gcd(a, b) \geqslant 1$,
    then $b \mid a$ or $b$ is composite: \\
    Suppose that $a, b \in \N$
    such that $\gcd(a, b) > 1$.
    We know that $b$ is either prime or composite,
    so if we check the proposition for each case
    we will have covered all of our bases. \\
    If $b$ is prime,
    then no number aside from $b$ and $1$ can divide $b$.
    Since $\gcd(a, b) \mid b$,
    and $\gcd(a, b) > 1$,
    it must be that $\gcd(a, b) = b$.
    Since $\gcd(a, b) \mid a$,
    we conclude that $b \mid a$. \\
    Otherwise, if $b$ is composite,
    then the proposition is true and we can stop here. \\

    \section*{Exercise 28 $***$}
    Proof that if $a, b, c \in \Z$,
    then $\gcd(a, b) \leqslant \gcd(ca, cb)$: \\
    Suppose that $a, b, c \in \Z$.
    Take $d = \gcd(a, b)$.
    Then $d \mid a$ and $d \mid b$,
    which means that $a = nd$ and $b = md$
    for some $n, m \in \Z$.
    This means that $ca = (cn)d$ and $cb = (cm)d$,
    which means that $d \mid ca$ and $d \mid cb$.
    By definition,
    $\gcd(ca, cb)$
    is the largest number that divides both $ca$ and $cb$,
    so since $d$ divides both,
    it is at most equal to it.
    So $\gcd(a, c) \leqslant \gcd(ca, cb)$. \\




\end{document}
