
\documentclass[12pt]{article}

\usepackage[margin=1in]{geometry}


%===============================================================================
%================================== PACKAGES ===================================
%===============================================================================

% For using float option H that places figures 
% exatcly where we want them
\usepackage{float}
% makes figure font bold
\usepackage{caption}
\captionsetup[figure]{labelfont=bf}
% For text generation
\usepackage{lipsum}
% For drawing
\usepackage{tikz}
% For smaller or equal sign and not divide sign
\usepackage{amssymb}
% For the diagonal fraction
\usepackage{xfrac}
% For enumerating exercise parts with letters instead of numbers
\usepackage{enumitem}
% For dfrac, which forces the fraction to be in display mode (large) e
% even in math mode (small)
\usepackage{amsmath}
% For degree sign
\usepackage{gensymb}
% For "\mathbb" macro
\usepackage{amsfonts}
% For positioning 
\usepackage{indentfirst}
\usetikzlibrary{shapes,positioning,fit,calc}
% for adjustwidth environment
\usepackage{changepage}
% for arrow on top
\usepackage{esvect}
% for mathbb 1
\usepackage{bbm}
% for mathsrc
\usepackage[mathscr]{eucal}
% For degree sign
\usepackage{gensymb}
% For quotes
\usepackage{csquotes}
% For vertical lines
\usepackage{mathtools}
% For cols
\usepackage{multicol}

% for tikz
\usepackage{pgfplots}
\pgfplotsset{compat=1.18}
\usepackage{amsmath}
\usepgfplotslibrary{groupplots}


%===============================================================================
%==================================== FONTS ====================================
%===============================================================================


% Mathcal
\newcommand{\acal}{\mathcal{A}}
\newcommand{\bcal}{\mathcal{B}}
\newcommand{\ccal}{\mathcal{C}}
\newcommand{\dcal}{\mathcal{D}}
\newcommand{\ecal}{\mathcal{E}}
\newcommand{\fcal}{\mathcal{F}}
\newcommand{\gcal}{\mathcal{G}}
\newcommand{\hcal}{\mathcal{H}}
\newcommand{\ical}{\mathcal{I}}
\newcommand{\jcal}{\mathcal{J}}
\newcommand{\kcal}{\mathcal{K}}
\newcommand{\lcal}{\mathcal{L}}
\newcommand{\mcal}{\mathcal{M}}
\newcommand{\ncal}{\mathcal{N}}
\newcommand{\ocal}{\mathcal{O}}
\newcommand{\pcal}{\mathcal{P}}
\newcommand{\qcal}{\mathcal{Q}}
\newcommand{\rcal}{\mathcal{R}}
\newcommand{\scal}{\mathcal{S}}
\newcommand{\tcal}{\mathcal{T}}
\newcommand{\ucal}{\mathcal{U}}
\newcommand{\vcal}{\mathcal{V}}
\newcommand{\wcal}{\mathcal{W}}
\newcommand{\xcal}{\mathcal{X}}
\newcommand{\ycal}{\mathcal{Y}}
\newcommand{\zcal}{\mathcal{Z}}

% Mathfrak
\newcommand{\afrak}{\mathfrak{A}}
\newcommand{\bfrak}{\mathfrak{B}}
\newcommand{\cfrak}{\mathfrak{C}}
\newcommand{\dfrak}{\mathfrak{D}}
\newcommand{\efrak}{\mathfrak{E}}
\newcommand{\ffrak}{\mathfrak{F}}
\newcommand{\gfrak}{\mathfrak{G}}
\newcommand{\hfrak}{\mathfrak{H}}
\newcommand{\ifrak}{\mathfrak{I}}
\newcommand{\jfrak}{\mathfrak{J}}
\newcommand{\kfrak}{\mathfrak{K}}
\newcommand{\lfrak}{\mathfrak{L}}
\newcommand{\mfrak}{\mathfrak{M}}
\newcommand{\nfrak}{\mathfrak{N}}
\newcommand{\ofrak}{\mathfrak{O}}
\newcommand{\pfrak}{\mathfrak{P}}
\newcommand{\qfrak}{\mathfrak{Q}}
\newcommand{\rfrak}{\mathfrak{R}}
\newcommand{\sfrak}{\mathfrak{S}}
\newcommand{\tfrak}{\mathfrak{T}}
\newcommand{\ufrak}{\mathfrak{U}}
\newcommand{\vfrak}{\mathfrak{V}}
\newcommand{\wfrak}{\mathfrak{W}}
\newcommand{\xfrak}{\mathfrak{X}}
\newcommand{\yfrak}{\mathfrak{Y}}
\newcommand{\zfrak}{\mathfrak{Z}}

% Mathscr
\newcommand{\ascr}{\mathscr{A}}
\newcommand{\bscr}{\mathscr{B}}
\newcommand{\cscr}{\mathscr{C}}
\newcommand{\dscr}{\mathscr{D}}
\newcommand{\escr}{\mathscr{E}}
\newcommand{\fscr}{\mathscr{F}}
\newcommand{\gscr}{\mathscr{G}}
\newcommand{\hscr}{\mathscr{H}}
\newcommand{\iscr}{\mathscr{I}}
\newcommand{\jscr}{\mathscr{J}}
\newcommand{\kscr}{\mathscr{K}}
\newcommand{\lscr}{\mathscr{L}}
\newcommand{\mscr}{\mathscr{M}}
\newcommand{\nscr}{\mathscr{N}}
\newcommand{\oscr}{\mathscr{O}}
\newcommand{\pscr}{\mathscr{P}}
\newcommand{\qscr}{\mathscr{Q}}
\newcommand{\rscr}{\mathscr{R}}
\newcommand{\sscr}{\mathscr{S}}
\newcommand{\tscr}{\mathscr{T}}
\newcommand{\uscr}{\mathscr{U}}
\newcommand{\vscr}{\mathscr{V}}
\newcommand{\wscr}{\mathscr{W}}
\newcommand{\xscr}{\mathscr{X}}
\newcommand{\yscr}{\mathscr{Y}}
\newcommand{\zscr}{\mathscr{Z}}

% Mathbb
\newcommand{\abb}{\mathbb{A}}
\newcommand{\bbb}{\mathbb{B}}
\newcommand{\cbb}{\mathbb{C}}
\newcommand{\dbb}{\mathbb{D}}
\newcommand{\ebb}{\mathbb{E}}
\newcommand{\fbb}{\mathbb{F}}
\newcommand{\gbb}{\mathbb{G}}
\newcommand{\hbb}{\mathbb{H}}
\newcommand{\ibb}{\mathbb{I}}
\newcommand{\jbb}{\mathbb{J}}
\newcommand{\kbb}{\mathbb{K}}
\newcommand{\lbb}{\mathbb{L}}
\newcommand{\mbb}{\mathbb{M}}
\newcommand{\nbb}{\mathbb{N}}
\newcommand{\obb}{\mathbb{O}}
\newcommand{\pbb}{\mathbb{P}}
\newcommand{\qbb}{\mathbb{Q}}
\newcommand{\rbb}{\mathbb{R}}
\newcommand{\sbb}{\mathbb{S}}
\newcommand{\tbb}{\mathbb{T}}
\newcommand{\ubb}{\mathbb{U}}
\newcommand{\vbb}{\mathbb{V}}
\newcommand{\wbb}{\mathbb{W}}
\newcommand{\xbb}{\mathbb{X}}
\newcommand{\ybb}{\mathbb{Y}}
\newcommand{\zbb}{\mathbb{Z}}


%===============================================================================
%=============================== SPECIAL SYMBOLS ===============================
%===============================================================================


% Orbit (group theory)
\newcommand{\orbit}{\mathcal{O}}
% Normal group
\newcommand{\normal}{\mathcal{N}}
% Indicator function
\newcommand{\indicator}{\mathbbm{1}}
% Laplace transform
\newcommand{\laplace}[1]{\mathcal{L}}
% Epsilon shorthand
\newcommand{\eps}{\varepsilon}
% Omega
\newcommand{\om}{\omega}
\newcommand{\Om}{\Omega}

%===============================================================================
%================================== OPERATORS ==================================
%===============================================================================


% Inverse exponent
\newcommand{\inv}[0]{^{-1}}
% Overline bar
\newcommand{\olsi}[1]{\,\overline{\!{#1}}}
% Less than or equal slanted
\newcommand{\seqs}{\leqslant}
% Greater or equal slanted
\newcommand{\geqs}{\geqslant}
% Subset or equal
\newcommand{\sub}{\subseteq}
% Proper subset
\newcommand{\prosub}{\subset}
% from
\newcommand{\from}{\leftarrow}

% Parantheses
\newcommand{\para}[1]{\left( #1 \right)}
% Curly Braces
\newcommand{\curl}[1]{\left\{ #1 \right\}}
% Brackets
\newcommand{\brac}[1]{\left[ #1 \right]}
% Angled Brackets
\newcommand{\ang}[1]{\left\langle #1 \right\rangle}
% Norm
\newcommand{\norm}[1]{\left\| #1 \right\|}

% Piece wise (use \\ between cases)
\newcommand{\piecewise}[1]{\begin{cases} #1 \end{cases}}

% Bold symbol shorthand
\newcommand{\bl}{\boldsymbol}

% Vertical space
\newcommand{\vs}[1]{\vspace{#1 pt}}
% Horizontal ertical space
\newcommand{\hs}[1]{\hspace{#1 pt}}


%===============================================================================
%============================== TEXT BASED SYMBOLS =============================
%===============================================================================


% Radians
\newcommand{\rad}{\text{rad}}
% Least Common Multiple
\newcommand{\lcm}{\text{lcm}}
% Automorphism
\newcommand{\Aut}{\text{Aut}}
% Variance
\newcommand{\var}{\text{Var}}
% Covariance
\newcommand{\cov}{\text{Cov}}
% Cofactor (matrix)
\newcommand{\cof}{\text{Cof}}
% Adjugate (matrix)
\newcommand{\adj}{\text{Adj}}
% Trace (matrix)
\newcommand{\tr}{\text{tr}}
% Standard deviation
\newcommand{\std}{\text{Std}}
% Correlation coefficient
\newcommand{\corr}{\text{Corr}}
% Sign
\newcommand{\sign}{\text{sign}}

% And text
\newcommand{\AND}{\text{ and }}
% Or text 
\newcommand{\OR}{\text{ or }}
% If text
\newcommand{\IF}{\text{ if }}
% When text
\newcommand{\WHEN}{\text{ when }}
% Then text
\newcommand{\THEN}{\text{ then }}

% 1st
\newcommand{\st}[1]{#1^{\text{st}}}
% 2nd
\newcommand{\nd}[1]{#1^{\text{nd}}}
% 3rd
\newcommand{\rd}[1]{#1^{\text{rd}}}
% nth
\newcommand{\nth}[1]{#1^{\text{th}}}


%===============================================================================
%========================= PROBABILITY AND STATISTICS ==========================
%===============================================================================


% Permutation
\newcommand{\perm}[2]{{}^{#1}\!P_{#2}}
% Combination
\newcommand{\comb}[2]{{}^{#1}C_{#2}}

% Baye's risk
\newcommand{\risk}[1]{\mathscr{R}_{#1}}
% Baye's optimal risk
\newcommand{\riskOptimal}[1]{\mathscr{R}_{#1}^*}
% Baye's empirical risk
\newcommand{\riskEmpirical}[2]{\hat{\mathscr{R}}_{#1}^{#2}}


%===============================================================================
%=================================== CALCULUS ==================================
%===============================================================================


% d over d derivative
\newcommand{\dd}[2]{\dfrac{d#1}{d#2}}
% partial d over d derivative
\newcommand{\partialdd}[2]{\dfrac{\partial #1}{\partial #2}}
% delta d over d derivative
\newcommand{\deltadd}[2]{\dfrac{\Delta #1}{\Delta #2}}

% Integration between a and b
\newcommand{\integral}[4]{\int_{#1}^{#2} #3 \, #4}
% Integration in some space 
\newcommand{\boundIntegral}[2]{\int_{#1} #2 \, d#1}

% Limit
\newcommand{\limit}[3]{\lim_{#1 \to #2} #3}


%===============================================================================
%================================  BIG SYMBOLS  ================================
%===============================================================================


% Sum
\newcommand{\sumof}[2]{\sum_{#1}^{#2}}
% Product
\newcommand{\productof}[2]{\prod_{#1}^{#2}}
% Union
\newcommand{\unionof}[2]{\bigcup_{#1}^{#2}}
% Intersection
\newcommand{\intersectionof}[2]{\bigcap_{#1}^{#2}}
% Or
\newcommand{\orof}[2]{\bigvee_{#1}^{#2}}
% And
\newcommand{\andof}[2]{\bigwedge_{#1}^{#2}}


%===============================================================================
%=============================== LINEAR ALGEBRA ================================
%===============================================================================


% Bold vector arrow
\newcommand{\bv}[1]{\vec{\mathbf{#1}}}

% Matrix or vector (use // for column, & for row) 
% with brackets
\newcommand{\bmat}[1]{\begin{bmatrix} #1 \end{bmatrix}}
% Matrix or vector (use // for column, & for row) 
% with curved brackets
\newcommand{\pmat}[1]{\begin{pmatrix} #1 \end{pmatrix}}
% Matrix or vector (use // for column, & for row) 
% with lines on either side 
\newcommand{\lmat}[1]{\begin{vmatrix} #1 \end{vmatrix}}
% Matrix or vector (use // for column, & for row) 
% with curly braces
\newcommand{\cmat}[1]{\begin{Bmatrix} #1 \end{Bmatrix}}
% Matrix or vector (use // for column, & for row) 
% with no braces
\newcommand{\mat}[1]{\begin{matrix} #1 \end{matrix}}


%===============================================================================
%================================ LARGE OBJECTS ================================
%===============================================================================

% Multiple lines
\newcommand{\multiline}[1]{
\begin{align*}
    #1
\end{align*}
}

% Multiple lines with equation numbers
\newcommand{\eqmultiline}[1]{
\begin{align*}
    #1
\end{align*}
}

% Color
\newcommand{\colorText}[2]{
\begingroup
\color{#1}
    #2
\endgroup
}

% Centered figure
\newcommand{\centerFigure}[2]{
    \begin{figure}[h]
        \centering
            #1
        \caption{#2}
    \end{figure}
}

% Tikz figure
\newcommand{\tikzGraphic}[1]{
    \begin{center}
    \begin{tikzpicture}
        #1
    \end{tikzpicture}
    \end{center}
}

% Enumerate numbers (seperate by \item)
\newcommand{\numbers}{\textbf{\number*)}}

% Enumerate letters (seperate by \item)
\newcommand{\letters}{\textbf{\alph*)}}

\title{%
    \Huge Machine Learning \\
    \Large Lecture II
}
\date{2025-04-28}
\author{Michael Saba}

\begin{document}
\pagenumbering{gobble}
\maketitle
\newpage
\setlength{\parindent}{0pt}
\pagenumbering{arabic}

\section*{Bayesian Learning Theory}

\subsection*{Recap}

There is an underlying probability space. \\

The data space is the joint distribution of
inputs and output.
We assume it exists for now
(in practice we have a sample). \\

We assume the distribution is given by a density
function that we wish to use. \\

There is a marginal and conditional distribution,
as well as a marginal and conditional density. \\

We can find the conditinonal using the joint
and marginal. \\

The performance criteria is evaluated
by a loss function, which could be
discrete, continuous, multivariate... \\

The risk is the expected value of the loss
function; it is a performance measure for all
the points (teh expcted value is an integral).
It is deterministic for a learning rule.
It must be integrable however. \\

The training data and testing data must
be independent so we ensure the classifier
generalizes. \\

Baye's optimal risk gives us the best possible
risk that a learning rule can have.
Our goal is to have a classifier with a risk
close to the optimal risk. \\

The total law of expectation gives us a way
to transform an expected value of a joint
distribution in terms of the expected value
of the joint distribution. \\

\newpage

\subsection*{More on Risk}

We can caluclate the risk by
using the total expectation law to decompose
the expectation of the joint distribution
into the expectation of the marginal:
\[ \rscr_\ell(f)
= \ebb[\ell(f(X), Y)]
= \ebb[\ebb[\ell(f(X), Y) \mid X]] \]
This will be important. \\

Here
\[ \ebb[\ell(f(X), Y)]
= \ebb[\ebb[\ell(f(X), Y) \mid X]] \]
The inner expectation is a sum over
all the values of $Y$ (the output),
while the outer expectation is a sum over
all the values of $X$. \\
The initial expectation we saw was an expectation
over a joint distribution,
which is a sum over the joint space
meaning over both $x$ and $y$ at once.
Decomposing simplifies it since we now
have two distinct sums each over one variable. \\

For instance, the conditional risk,
which is like the risk
but for a single input $x$,
can be written as:
\[ r(f(x) \mid x)
= \ebb[\ell(f(x), Y) \mid X = x] \]
which makes sense, since this is a sum
over all values of $Y$,
for one input $x$,
that evaluates how well we predicted
the output with $f(x)$.
We want to find the output $f(x)$
that minimizes the conditional
risk in order to have an optimal
prediction for $x$ (just this input). \\

And the risk if just the sum of all individual
losses (over $X$)
times the probability that they take place:
\[ \rscr_\ell(f) = 
\sum_{x} r(f(x) \mid x)\cdot \pbb(X = x) \]
which makes:
The total risk is the risk of each 
indidividual input,
weighted by the probability that we get this input.
If we are very bad at predicting some input,
then the risk is high,
but if that input rarely every shows up,
then the total risk $\rscr_\ell(f)$
won't necessarily be affected much. \\ 

Note that
\[ \rscr_\ell(f) = \ebb[\ell(f(X), Y)] \]
And
\[ \rscr_\ell(f) = \ebb[\ebb[\ell(f(X), Y) \mid X]] 
= \ebb[r(f(x) \mid x)] \]
which seems to imply that the conditional
risk $r(f(x) \mid x)$
is equal to the loss $\ell(f(X), Y)$. \\
However this is NOT case! \\
The first expected value:
\[ \ebb[\ell(f(X), Y)] \] 
is a sum of the joint distribution
over the joint space,
which means a sum over both $x$ and $y$
(basically a double integral). \\
On the other hand:
\[ \ebb[r(f(x) \mid x)] \]
Here the expected value is the outer
expected value,
which a sum over $x$ only (the inner expected
value was the sum over $y$). \\
So these two expected values are
different sums,
and the value inside each is therefore distinct. \\

In other words, just because we can decompose
this integral like this:
\[\int_{\xcal \times \ycal} f(x, y)
\cdot \pbb_{(X, Y)}(dx, dy) 
= \int_{\xcal} \int_{\ycal} f(x, y)
\cdot \pbb_{Y \mid X = x}(dy)
\pbb_X(dx) \]
does not mean that the contents of the left integral
is equal to the contents of the outer integral
on the right:
\[\int_{\xcal \times \ycal} 
(\text{function of $x$ and $y$}) dxdy 
= \int_{\xcal} (\text{function of only $x$}) dx \]
Clearly the contents of the sums are not equal! \\
If two sums are equal, we can only say
that what they are summing is equal if they are
summing over the same space! \\

We can actually think of the risk in two ways:
A single sum over the joint space (weighted joint
distribution) for all pairs $(x, y)$,
or a double sum over $y$ and $x$,
one weighted by the conditional,
and another by the marginal. \\
We can see it more clearly with sums:
The first has every combination of $x$
and $y$ weighted by the joint probability:
\[ \ebb[\ell(f(X), Y)]
= \sum_x \sum_y \ell(f(x), y) \pbb(X = x, Y = y) \]
The second has the same formula but decomposed:
\[ \ebb[\ebb[\ell(f(X), Y) \mid X = x]]
= \sum_x \pbb(X = x) \sum_y \ell(f(x), y) 
\pbb(Y = y \mid X = x) \] \\

We know that the optimal risk is:
\[ \rscr_\ell^*
= \inf_f \ebb[\ebb[\ell(f(X), Y) \mid X]] \]
first we try to find the optimal
output $f(x)$ for each input $x$
(by summing over all outputs $Y$ for the fixed $x$),
then we do the same for all inputs $x$,
so we can get the best predictor for
all inputs $x$. \\

Notice that if we just want the optimal
prediction for just one input $x$,
we can skip the outer summation,
and just do the inner one with that $x$
(the optimal conditional risk):
\[ f^*(x) = \text{min}_{f(x)} \;
\ebb[\ell(f(x), Y) \mid X = x] \]
which means that the optimal output $f^*(x)$
is the one that minimizes the expected value
of the loss knowing $X = x$
(which is just one sum over all $Y$). \\

In fact, if we find Baye's optimal classifier
$f*$,
then we can prove that it satisfies this optimization
\[ f^*(x) = \text{min}_{f(x)} \;
\ebb[\ell(f(x), Y) \mid X = x] \]
for all $x$ over the distribution $\pbb_x$.
On the other hand, if we prove it only
for a specific $x$ (using only the conditional
probabilities given that $x$),
then we would only have the optimal
prediction for one input. \\

We can prove that the optimal classifier
$f^*(x)$,
that gives us Baye's optimal risk $\rscr_\ell^*$,
which minimizes the total risk,
also minimizes each inidividual conditional risk. \\
First we know that Baye's optimal risk is:
\[ \rscr_\ell^* = \inf_f 
\ebb[\ebb[(\ell(f(X), Y) \mid X)]]
= \inf_f \int_{\xcal} r(f(x) \mid x) \cdot
\pbb_X(dx) \]
Here, the infimum means the largest lower bound
of the intgeral. \\
However, notice that if we took the sum of the
infimum of each individual conditional risk
(the optimal risk at each $x$)
then we would have to have a value smaller
or equal to the infimum of the sum at large:
\[ \inf_f \int_{\xcal} r(f(x) \mid x) \cdot
\pbb_X(dx)
\geq \int_{\xcal} \inf_{fx} r(f(x) \mid x) \cdot
\pbb_X(dx) = \rscr_\ell(f^*) \]
This is a general result;
the minimum (or infimum) of the sum
of a function is larger or equal to the
sum of the minimum values attained
by the function at each individual input. \\
So that means that:
\[ \rscr_\ell^* \geq \rscr_\ell(f^*) \]
But we defined Baye's optimal risk
to be the infimum (largest value
smaller than all others),
so it can't be larger, which means that:
\[ \rscr_\ell^* = \rscr_\ell(f^*) \]
So we conclude that Baye's optimal classifier,
which has the lowest overall risk,
must also be the most optimal learning rule
that minimizes each individual conditional
risk. \\

This makes machine learning seem trivial;
finding the optimal classifier boilds
down to finding the best output for each
indidividual input. \\
However, in practice, we usually don't have
the entire distribution $\pbb_{X, Y}$.
We only have some samples, such as the training
and testing data. \\

\newpage

\subsection*{Binary Classification}

We can calculate the risk of a binary
classification as follows:
\[ \rscr_\ell(f)
= \ebb[\ebb[\ell(f(X), Y) \mid X]] \]
\[ \rscr_\ell(f)
=  \sum_x \pbb(X = x)
\sum_y \ell(f(x), y) \cdot
\pbb(Y = y \mid X = x) \]
\[ \rscr_\ell(f)
= \sum_x \pbb(X = x)
\sum_y \indicator_{f(x) \neq y}\cdot
\pbb(Y = y \mid X = x) \]
Notice that we only have two possible
values for $Y$, so:
\[ \rscr_\ell(f)
= \ebb[ \indicator_{f(X) \neq y_1}\cdot
\pbb(Y = y_1 \mid X) +
\indicator_{f(X) \neq y_2}\cdot
\pbb(Y = y_2 \mid X)] \]
The conditional probability of the one
that equals our prediction $f(x)$
will be mutliplied by $0$ as per the loss,
and the other other will be mutliplied
by $1$,
so only one of them will contribute
to the risk. \\

Notice what this means.
In order to optimize $f(x)$
for each $x \in \xcal$,
we want to pick $f(x)$
such that the risk is as small as possible.
And since:
\[ \rscr_\ell(f)
= \ebb[ \pbb(Y = y_1 \mid X)] \]
when $f(x) = y_2$,
that means we want to pick $y_1$
over $y_2$ if:
\[ \pbb(Y = y_1 \mid X) >
\pbb(Y = y_2 \mid X)  \]
which makes intuitive sense, both in terms 
of minimizing the risk,
and choosing the more likely output.
So we will have:
\[ f^*(x) = \piecewise{y_1 \quad \text{ if }
\pbb(Y = y_1 \mid X) >
\pbb(Y = y_2 \mid X)}
{y_2 \quad \text{ otherwise }} \]
for the optimal classifier. \\

Note that we can actually bound the error
on Baye's optimal classifier. \\
We already know that:
\[ \rscr_\ell(f)
= \ebb[ \indicator_{f(X) \neq y_1}\cdot
\pbb(Y = y_1 \mid X) +
\indicator_{f(X) \neq y_2}\cdot
\pbb(Y = y_2 \mid X)] \]
where one of the two probabilities
will be multiplies by $0$,
and the other by $1$,
depending on the output of the classifier. \\
In the optimal classifier, we have:
\[ f^*(x) = \piecewise{y_1 \quad \text{ if }
\pbb(Y = y_1 \mid X = x) >
\pbb(Y = y_2 \mid X = x)}
{y_2 \quad \text{ otherwise }} \]
So that means that we can write the
optimal risk as:
\[ \rscr_\ell^*
= \ebb[ \min(
\pbb(Y = y_1 \mid X), \,
\pbb(Y = y_2 \mid X))] \]
Now, a general result that we can use
is the fact that the expected value of the
minimum is less than or equal to the minimum
of the expected value:
\[ \rscr_\ell^* \leq \min(
\ebb[\pbb(Y = y_1 \mid X)], \,
\ebb[\pbb(Y = y_2 \mid X)]) \]
Then, we can unfurl the expected value
by multiplying the contents by the
marginal probability of $X = x$
for each $x$,
which when summed over all values of $x$
just gives us the marginal probability of $Y$:
\[ \rscr_\ell^* \leq \min(
\pbb_Y(y_1), \, \pbb_Y(y_2)) \]
Which gives us a bound on the best-case error
(optimal risk). \\

This bound tells us that
if the probability that the base class even
take place is low,
then the risk will be (deceptively) low
as well,
even if they classifier is bad. \\

This can be resolved by using
cost sensitive cost. \\

\newpage

\subsection*{Cost Senstitive Classification}

Any classification or regression can be
made cost sensitive,
not just binary classification. \\

The idea of cost is to assign an extra penalty
to making certain mistakes in classification,
which are supposedly more costly than other
mistakes. \\
For example, in a binary classification,
we can create a $2 \times 2$ confusion matrix
that shows every possible outcome:
true positive, true negative, false positive,
and false negative. \\
In cancer detection software for example,
it is far worse for an AI to not detect cancer
that is there than to detect cancer that is not
there. \\

We can assign new costs
by modyfing our loss function:
\[ \ell_C(f(x), y) = 
C_{ij}\indicator_{f(y = i, f(x) = j)} \]
where $C_{ij}$
is the cost for predicting
the $\nth{j}$ outcome
when the real output is $\nth{i}$. \\
This value can be more or less than $1$,
unlike our old loss function which always
assigned $1$ to false predictions. \\
The cost for true predictions $C_{ii}$
should remain $0$. \\

This change cascades and affects
every other outcome we have derived. \\
For example,
the risk can be written as:
\[ \rscr_\ell(f)
= \sum_{x} \pbb(X = x)
\sum_{y} C_{ij}\indicator_{y = i, f(x) = j}\cdot
\pbb(Y = y \mid X = x) \]
And the bound on the optimal risk can be
written as:
\[ \rscr_\ell^* \leq \min(
C_{y_1y_2}\pbb_Y(y_1), \, 
C_{y_2y_1}\pbb_Y(y_2)) \] \\

\newpage

\subsection*{Multi-class Classification}

When we have more than two values that
our output $y$ can have,
then we are doing multi-class classification. \\

Since the output is still given by a discrete
random variable, most formulas
are hardly changed. \\

Baye's risk formula is
unchanged:
\[ \rscr_\ell(f)
= \ebb\brac{\sum_y \indicator_{f(X) \neq y}
\pbb(Y = y \mid X)} \]
However, note that only one
of the values in the sum will go to 0
(when $f(X) = y$),
the rest will all remain,
so when we simplify by removing
$\pbb(Y = y \mid X)$
where $y = f(X)$,
we end up with:
\[ \rscr_\ell^*
= \ebb\brac{
\olsi{\pbb(Y = y \mid X)}}
= \ebb\brac{
1 - \pbb(Y = y \mid X)} \]
where $y$ is the output
equal to the choice that was made
$y = f(X)$ (for each $x$). \\
This is the same formula as before,
though before, $\olsi{\pbb(Y = y \mid X)}$
was just one event's probability
(when $y \neq f(X)$),
but now it's several. \\

The formulas derived from this one,
such as the optimal risk and optimal
classifier, are all unchaged. \\

\newpage

\subsection*{Convex Functions}

Intuitively, we know that a convex function
is a real valued function that has a
global minimum. \\
In this section we will formalize this notion. \\

A function:
\[ f: \rbb^d \to \rbb \]
is called convex if and only if
for every pair
$x, y \in \rbb^d$,
and $\lambda \in [0, 1]$,
the following equation holds:
\[ f((1-\lambda)x + \lambda y)
\leq (1-\lambda)f(x) + \lambda f(x) \]
Here, $\lambda$ is a parameter,
which goes from $0$ to $1$. \\
This means that 
$(1-\lambda)f(x) + \lambda f(x)$
goes from $f(x)$ to $f(y)$
(defines the line that connects these
two points). \\
It also means that $(1-\lambda)x + \lambda y$
defines the line that goes from $x$ to $y$. \\
therefore, what the equation is saying
is that for any two points $x$ and $y$,
if we have a point $z \in [x, y]$,
then $f(z)$ is lower than the
line connecting $f(x)$ to $f(y)$. \\
In other words,
the curve or surface of $f$
is always below a line segments
that connects any two points
on said curve. \\

This property of always being
below a line segment connecting
two points on the curve is what ensures
that any local minimum on the function
is also a global minimum. \\
That is to say, there are no local
minima,
only one global minimum. \\

Some other useful properties of convex
functions:
\begin{enumerate}
    \item
    The sum of convex functions is
    convex.
    \item 
    The pointwise supremum of a collection
    of convex functions is a convex function.
    This means that if we take a collection
    of convex functions $\{f_1, f_2, \dots f_n\}$,
    and create a new function $f$ such that
    and for each point $x$,
    $f(x) = f_k(x)$ for the function $f_k$
    such that $f_k(x)$ is the largest,
    $f$ will be convex.
    \item 
    If we multiply a convex function by a 
    non-negative scalar, it remains convex. \\
\end{enumerate}

A function is convex if and only if it is
convex on every line segment. \\
We already saw that that was the case
when we averred that convex functions
should always be below a line
that connects two points on it. \\
In other words, if we have a convex function:
\[ f: \rbb^d \to \rbb \],
and we slice it with any plane,
the slice of $f$ on that plane,
which is a one dimensional function,
must also be convex. \\

If a function $f$ defined on $C$ is
differentiable,
then it is convex if:
\[ \ang{x_1 - x_0, 
\nabla f(x_1) - \nabla f(x_0)} \geq 0 \qquad
\forall x_0, x_1 \in C \]
What this means is that for any interval
between $x_0$ and $x_1$,
we expect the gradient (or derivative)
of $f$ to increase if the input $x$
is increasing,
and decrease if the input $x$ is decreasing
(they have the same sign). \\
It is very easy to see why this is so
in a function $\rbb \to \rbb$
(just a curve). \\
We expect a convex function to look like a cup,
which means that its rate of change must
always be increasing as $x$ moves in the
positive direction. \\
We can apply the same logic for higher
dimensional functions
by just ensuring the property holds
for all variables. \\
That is, that $\nabla f$
increases in the positive direction
of any variable. \\

Alternatively,
if $f$ defined on $C$ is twice differentiable,
we can just check that the Hessian
$\nabla^2 f(x)$ is positive semi-definite
for all $x \in C$. \\

The directional derivative of a convex
function $f$ is monotone (either increasing
or decreasing). \\
This is similar to the last theorem
we postulated with the derivative of $f(x)$
always being increasing in the positive
direction of any variable,
but gets extended to any direction,
even those that don't coincide
with the axis of a variables. \\

For a convex function:
\[ f:\rbb \to \rbb \]
we also have that:
\[ x \in \arg \min f \iff 
0 \in [f'(x; -1), f'(x; +1)] \]
Here $f'(x; -1)$ is the derivative from
the left (negative side),
and $f'(x; +1)$ is the derivative
from the right (positive side). \\
What this says is that if $x$ minimizes $f$,
the the derivative at $x$
must be $0$,
which is a well known result in calculus. \\

\newpage

\subsection*{Real Valued Predictions}

It can be incredibly difficult (NP-Hard)
to solve optimization problems with
discrete vairables. \\
On the other hand, continuous (real-valued)
convex functions are very easy to optimize;
all we have to do is find the critical
points of the function on an interval. \\

If we want to create a discerete
(binary for example)
classifier $f(x)$:
\[ f: \xcal \to \ycal \]
where $\ycal = \{-1, 1\}$,
we can instead create
a continuous,
surrogate prediction function $g(x)$:
\[ g: \xcal \to \rbb \]
Then, to map to $\ycal$ (our final output),
we can use threshholding 
(pick certain intervals to map to certain
discrete outputs). \\
The most natural mapping we can choose 
(and design $g(x)$ around)
is mapping all negatives values of $g(x)$
to $-1$,
and all positive values of $g(x)$ to $1$:
\[ \sign(g(x))
= \piecewiseThree{-1 \hspace{146pt} \IF g(x) < 0}
{1 \hspace{155pt} \IF g(x) > 1}
{\text{Sample uniformly from } \{-1, 1\}
\IF g(x) = 0} \]
The decision boundaries are the points
in the interval where the sign switches.
For example, if at $x = x_0$
the sign goes from positive to negative,
then that is a decision boundary.
They are just the $x$-axis intercepts
of $g(x)$ where the sign changes. \\

We now need to define a new, margin-based
loss function for our surrogate
classifier $g(x)$; this is what will allow
us to optimize $g(x)$. \\
Intuitively, if $g(x) = y$,
then the classifier did a very good job,
and the loss function should add little to
no cost at all. \\
If $g(x)$ and $y$ share a sign
(both are positive or negative),
then $g(x)$ did a decent job at least,
so only a little cost is added. \\
If on the other hand $g(x)$ and $y$
have opposite signs, a lot of cost must be
added, as $g(x)$ made a bas predicion. \\
We can do that using the margin $yg(x)$.
We can define a function $\Phi$:
\[ \Phi: \rbb \to \rbb \]
which gives us the loss function:
\[ \ell(g(x), y) = \Phi(yg(x)) \]
such that $\Phi$
assigns high cost if $yg(x) < 0$,
low cost if $yg(x) \geq 0$,
and little to no cost if $g(x) = y$. \\

Many such loss functions exist;
the slides have several examples:
\begin{enumerate}
    \item 
    0-1 loss: $\Phi(z) = \indicator_{z<0}$
    \item
    Exponential loss: $\Phi(z) = e^{-z}$
    \item
    Hinge loss
    (not differentiable at $z = 1$):
    $\Phi(z) = \max{0, 1-z}$
    \item
    Trunctated Squared loss:
    $\Phi(z) \piecewise{(1-z)^2 \IF z < 1}
    {0 \hspace{35pt} \IF z \geq 1}$
    \item 
    Logistic loss:
    $\Phi(z) = \ln(1 + e^{-z})$ \\
\end{enumerate}

From here we can define Baye's risk for a
surrogate classifier $g$,
which is the ecpected value of the loss:
\[ \rscr_\Phi(g) = \ebb[\ell(g(x), y)]
= \ebb[\Phi(yg(x))] \]
We can refer to it as the $\Phi$-risk. \\

Our goal again will be to minimize the
$\Phi$-risk.
That is why we expect that $\Phi$ be convex. \\
There will be an optimal surrogate classifier
$g^*$ that gives us this minimal risk. \\
The question is whether:
\[ \sign(g^*(x)) = f^*(x) \]
where $f^*(x)$ is Baye's optimal classifier.
Will the optimal surrogate classifier give
us the same predictions
(once mapped to $\ycal$) as the optimal
classifier? \\

This all depends on the margin-based 
loss function $\Phi$ that we choose. \\
We call $\Phi$ classification-calibrated
if $\sign(g^*(x)) = f^*(x)$ holds. \\

In fact, $\Phi$ is calibration calibrated
if and only if $\Phi$ is differentiable at $0$
and $\Phi'(0) < 0$. \\
Every single loss function we proposed
is classification calibrated,
and optimizing their risk gives us a surrogate
classifier that yields the same predictions
as Baye's classifier would have
if we had remained in the discrete space. \\

\end{document}
