
\documentclass[12pt]{article}

\usepackage[margin=1in]{geometry}


%===============================================================================
%================================== PACKAGES ===================================
%===============================================================================

% For using float option H that places figures 
% exatcly where we want them
\usepackage{float}
% makes figure font bold
\usepackage{caption}
\captionsetup[figure]{labelfont=bf}
% For text generation
\usepackage{lipsum}
% For drawing
\usepackage{tikz}
% For smaller or equal sign and not divide sign
\usepackage{amssymb}
% For the diagonal fraction
\usepackage{xfrac}
% For enumerating exercise parts with letters instead of numbers
\usepackage{enumitem}
% For dfrac, which forces the fraction to be in display mode (large) e
% even in math mode (small)
\usepackage{amsmath}
% For degree sign
\usepackage{gensymb}
% For "\mathbb" macro
\usepackage{amsfonts}
% For positioning 
\usepackage{indentfirst}
\usetikzlibrary{shapes,positioning,fit,calc}
% for adjustwidth environment
\usepackage{changepage}
% for arrow on top
\usepackage{esvect}
% for mathbb 1
\usepackage{bbm}
% for mathsrc
\usepackage[mathscr]{eucal}
% For degree sign
\usepackage{gensymb}
% For quotes
\usepackage{csquotes}
% For vertical lines
\usepackage{mathtools}
% For cols
\usepackage{multicol}

% for tikz
\usepackage{pgfplots}
\pgfplotsset{compat=1.18}
\usepackage{amsmath}
\usepgfplotslibrary{groupplots}


%===============================================================================
%==================================== FONTS ====================================
%===============================================================================


% Mathcal
\newcommand{\acal}{\mathcal{A}}
\newcommand{\bcal}{\mathcal{B}}
\newcommand{\ccal}{\mathcal{C}}
\newcommand{\dcal}{\mathcal{D}}
\newcommand{\ecal}{\mathcal{E}}
\newcommand{\fcal}{\mathcal{F}}
\newcommand{\gcal}{\mathcal{G}}
\newcommand{\hcal}{\mathcal{H}}
\newcommand{\ical}{\mathcal{I}}
\newcommand{\jcal}{\mathcal{J}}
\newcommand{\kcal}{\mathcal{K}}
\newcommand{\lcal}{\mathcal{L}}
\newcommand{\mcal}{\mathcal{M}}
\newcommand{\ncal}{\mathcal{N}}
\newcommand{\ocal}{\mathcal{O}}
\newcommand{\pcal}{\mathcal{P}}
\newcommand{\qcal}{\mathcal{Q}}
\newcommand{\rcal}{\mathcal{R}}
\newcommand{\scal}{\mathcal{S}}
\newcommand{\tcal}{\mathcal{T}}
\newcommand{\ucal}{\mathcal{U}}
\newcommand{\vcal}{\mathcal{V}}
\newcommand{\wcal}{\mathcal{W}}
\newcommand{\xcal}{\mathcal{X}}
\newcommand{\ycal}{\mathcal{Y}}
\newcommand{\zcal}{\mathcal{Z}}

% Mathfrak
\newcommand{\afrak}{\mathfrak{A}}
\newcommand{\bfrak}{\mathfrak{B}}
\newcommand{\cfrak}{\mathfrak{C}}
\newcommand{\dfrak}{\mathfrak{D}}
\newcommand{\efrak}{\mathfrak{E}}
\newcommand{\ffrak}{\mathfrak{F}}
\newcommand{\gfrak}{\mathfrak{G}}
\newcommand{\hfrak}{\mathfrak{H}}
\newcommand{\ifrak}{\mathfrak{I}}
\newcommand{\jfrak}{\mathfrak{J}}
\newcommand{\kfrak}{\mathfrak{K}}
\newcommand{\lfrak}{\mathfrak{L}}
\newcommand{\mfrak}{\mathfrak{M}}
\newcommand{\nfrak}{\mathfrak{N}}
\newcommand{\ofrak}{\mathfrak{O}}
\newcommand{\pfrak}{\mathfrak{P}}
\newcommand{\qfrak}{\mathfrak{Q}}
\newcommand{\rfrak}{\mathfrak{R}}
\newcommand{\sfrak}{\mathfrak{S}}
\newcommand{\tfrak}{\mathfrak{T}}
\newcommand{\ufrak}{\mathfrak{U}}
\newcommand{\vfrak}{\mathfrak{V}}
\newcommand{\wfrak}{\mathfrak{W}}
\newcommand{\xfrak}{\mathfrak{X}}
\newcommand{\yfrak}{\mathfrak{Y}}
\newcommand{\zfrak}{\mathfrak{Z}}

% Mathscr
\newcommand{\ascr}{\mathscr{A}}
\newcommand{\bscr}{\mathscr{B}}
\newcommand{\cscr}{\mathscr{C}}
\newcommand{\dscr}{\mathscr{D}}
\newcommand{\escr}{\mathscr{E}}
\newcommand{\fscr}{\mathscr{F}}
\newcommand{\gscr}{\mathscr{G}}
\newcommand{\hscr}{\mathscr{H}}
\newcommand{\iscr}{\mathscr{I}}
\newcommand{\jscr}{\mathscr{J}}
\newcommand{\kscr}{\mathscr{K}}
\newcommand{\lscr}{\mathscr{L}}
\newcommand{\mscr}{\mathscr{M}}
\newcommand{\nscr}{\mathscr{N}}
\newcommand{\oscr}{\mathscr{O}}
\newcommand{\pscr}{\mathscr{P}}
\newcommand{\qscr}{\mathscr{Q}}
\newcommand{\rscr}{\mathscr{R}}
\newcommand{\sscr}{\mathscr{S}}
\newcommand{\tscr}{\mathscr{T}}
\newcommand{\uscr}{\mathscr{U}}
\newcommand{\vscr}{\mathscr{V}}
\newcommand{\wscr}{\mathscr{W}}
\newcommand{\xscr}{\mathscr{X}}
\newcommand{\yscr}{\mathscr{Y}}
\newcommand{\zscr}{\mathscr{Z}}

% Mathbb
\newcommand{\abb}{\mathbb{A}}
\newcommand{\bbb}{\mathbb{B}}
\newcommand{\cbb}{\mathbb{C}}
\newcommand{\dbb}{\mathbb{D}}
\newcommand{\ebb}{\mathbb{E}}
\newcommand{\fbb}{\mathbb{F}}
\newcommand{\gbb}{\mathbb{G}}
\newcommand{\hbb}{\mathbb{H}}
\newcommand{\ibb}{\mathbb{I}}
\newcommand{\jbb}{\mathbb{J}}
\newcommand{\kbb}{\mathbb{K}}
\newcommand{\lbb}{\mathbb{L}}
\newcommand{\mbb}{\mathbb{M}}
\newcommand{\nbb}{\mathbb{N}}
\newcommand{\obb}{\mathbb{O}}
\newcommand{\pbb}{\mathbb{P}}
\newcommand{\qbb}{\mathbb{Q}}
\newcommand{\rbb}{\mathbb{R}}
\newcommand{\sbb}{\mathbb{S}}
\newcommand{\tbb}{\mathbb{T}}
\newcommand{\ubb}{\mathbb{U}}
\newcommand{\vbb}{\mathbb{V}}
\newcommand{\wbb}{\mathbb{W}}
\newcommand{\xbb}{\mathbb{X}}
\newcommand{\ybb}{\mathbb{Y}}
\newcommand{\zbb}{\mathbb{Z}}


%===============================================================================
%=============================== SPECIAL SYMBOLS ===============================
%===============================================================================


% Orbit (group theory)
\newcommand{\orbit}{\mathcal{O}}
% Normal group
\newcommand{\normal}{\mathcal{N}}
% Indicator function
\newcommand{\indicator}{\mathbbm{1}}
% Laplace transform
\newcommand{\laplace}[1]{\mathcal{L}}
% Epsilon shorthand
\newcommand{\eps}{\varepsilon}
% Omega
\newcommand{\om}{\omega}
\newcommand{\Om}{\Omega}

%===============================================================================
%================================== OPERATORS ==================================
%===============================================================================


% Inverse exponent
\newcommand{\inv}[0]{^{-1}}
% Overline bar
\newcommand{\olsi}[1]{\,\overline{\!{#1}}}
% Less than or equal slanted
\newcommand{\seqs}{\leqslant}
% Greater or equal slanted
\newcommand{\geqs}{\geqslant}
% Subset or equal
\newcommand{\sub}{\subseteq}
% Proper subset
\newcommand{\prosub}{\subset}
% from
\newcommand{\from}{\leftarrow}

% Parantheses
\newcommand{\para}[1]{\left( #1 \right)}
% Curly Braces
\newcommand{\curl}[1]{\left\{ #1 \right\}}
% Brackets
\newcommand{\brac}[1]{\left[ #1 \right]}
% Angled Brackets
\newcommand{\ang}[1]{\left\langle #1 \right\rangle}
% Norm
\newcommand{\norm}[1]{\left\| #1 \right\|}

% Piece wise (use \\ between cases)
\newcommand{\piecewise}[1]{\begin{cases} #1 \end{cases}}

% Bold symbol shorthand
\newcommand{\bl}{\boldsymbol}

% Vertical space
\newcommand{\vs}[1]{\vspace{#1 pt}}
% Horizontal ertical space
\newcommand{\hs}[1]{\hspace{#1 pt}}


%===============================================================================
%============================== TEXT BASED SYMBOLS =============================
%===============================================================================


% Radians
\newcommand{\rad}{\text{rad}}
% Least Common Multiple
\newcommand{\lcm}{\text{lcm}}
% Automorphism
\newcommand{\Aut}{\text{Aut}}
% Variance
\newcommand{\var}{\text{Var}}
% Covariance
\newcommand{\cov}{\text{Cov}}
% Cofactor (matrix)
\newcommand{\cof}{\text{Cof}}
% Adjugate (matrix)
\newcommand{\adj}{\text{Adj}}
% Trace (matrix)
\newcommand{\tr}{\text{tr}}
% Standard deviation
\newcommand{\std}{\text{Std}}
% Correlation coefficient
\newcommand{\corr}{\text{Corr}}
% Sign
\newcommand{\sign}{\text{sign}}

% And text
\newcommand{\AND}{\text{ and }}
% Or text 
\newcommand{\OR}{\text{ or }}
% For text 
\newcommand{\FOR}{\text{ for }}
% If text
\newcommand{\IF}{\text{ if }}
% When text
\newcommand{\WHEN}{\text{ when }}
% Where text
\newcommand{\WHERE}{\text{ where }}
% Then text
\newcommand{\THEN}{\text{ then }}
% Such that text
\newcommand{\SUCHTHAT}{\text{ such that }}

% 1st
\newcommand{\st}[1]{#1^{\text{st}}}
% 2nd
\newcommand{\nd}[1]{#1^{\text{nd}}}
% 3rd
\newcommand{\rd}[1]{#1^{\text{rd}}}
% nth
\newcommand{\nth}[1]{#1^{\text{th}}}


%===============================================================================
%========================= PROBABILITY AND STATISTICS ==========================
%===============================================================================


% Permutation
\newcommand{\perm}[2]{{}^{#1}\!P_{#2}}
% Combination
\newcommand{\comb}[2]{{}^{#1}C_{#2}}

% Baye's risk
\newcommand{\risk}[1]{\mathscr{R}_{#1}}
% Baye's optimal risk
\newcommand{\riskOptimal}[1]{\mathscr{R}_{#1}^*}
% Baye's empirical risk
\newcommand{\riskEmpirical}[2]{\hat{\mathscr{R}}_{#1}^{#2}}


%===============================================================================
%=================================== CALCULUS ==================================
%===============================================================================


% d over d derivative
\newcommand{\dd}[2]{\dfrac{d#1}{d#2}}
% partial d over d derivative
\newcommand{\partialdd}[2]{\dfrac{\partial #1}{\partial #2}}
% delta d over d derivative
\newcommand{\deltadd}[2]{\dfrac{\Delta #1}{\Delta #2}}

% Integration between a and b
\newcommand{\integral}[4]{\int_{#1}^{#2} #3 \, #4}
% Integration in some space 
\newcommand{\boundIntegral}[2]{\int_{#1} #2 \, d#1}

% Limit
\newcommand{\limit}[3]{\lim_{#1 \to #2} #3}


%===============================================================================
%================================  BIG SYMBOLS  ================================
%===============================================================================


% Sum
\newcommand{\sumof}[2]{\sum_{#1}^{#2}}
% Product
\newcommand{\productof}[2]{\prod_{#1}^{#2}}
% Union
\newcommand{\unionof}[2]{\bigcup_{#1}^{#2}}
% Intersection
\newcommand{\intersectionof}[2]{\bigcap_{#1}^{#2}}
% Or
\newcommand{\orof}[2]{\bigvee_{#1}^{#2}}
% And
\newcommand{\andof}[2]{\bigwedge_{#1}^{#2}}


%===============================================================================
%=============================== LINEAR ALGEBRA ================================
%===============================================================================


% Bold vector arrow
\newcommand{\bv}[1]{\vec{\mathbf{#1}}}

% Matrix or vector (use // for column, & for row) 
% with brackets
\newcommand{\bmat}[1]{\begin{bmatrix} #1 \end{bmatrix}}
% Matrix or vector (use // for column, & for row) 
% with curved brackets
\newcommand{\pmat}[1]{\begin{pmatrix} #1 \end{pmatrix}}
% Matrix or vector (use // for column, & for row) 
% with lines on either side 
\newcommand{\lmat}[1]{\begin{vmatrix} #1 \end{vmatrix}}
% Matrix or vector (use // for column, & for row) 
% with curly braces
\newcommand{\cmat}[1]{\begin{Bmatrix} #1 \end{Bmatrix}}
% Matrix or vector (use // for column, & for row) 
% with no braces
\newcommand{\mat}[1]{\begin{matrix} #1 \end{matrix}}


%===============================================================================
%================================ LARGE OBJECTS ================================
%===============================================================================

% Multiple lines
\newcommand{\multiline}[1]{
\begin{align*}
    #1
\end{align*}
}

% Multiple lines with equation numbers
\newcommand{\eqmultiline}[1]{
\begin{align*}
    #1
\end{align*}
}

% Color
\newcommand{\colorText}[2]{
\begingroup
\color{#1}
    #2
\endgroup
}

% Centered figure
\newcommand{\centerFigure}[2]{
    \begin{figure}[h]
        \centering
            #1
        \caption{#2}
    \end{figure}
}

% Tikz figure
\newcommand{\tikzGraphic}[1]{
    \begin{center}
    \begin{tikzpicture}
        #1
    \end{tikzpicture}
    \end{center}
}

% Enumerate numbers (seperate by \item)
\newcommand{\numbers}{\textbf{\number*)}}

% Enumerate letters (seperate by \item)
\newcommand{\letters}{\textbf{\alph*)}}

\title{%
    \Huge Machine Learning \\
    \Large Lecture III
}
\date{2025-05-05}
\author{Michael Saba}

\begin{document}
\pagenumbering{gobble}
\maketitle
\newpage
\setlength{\parindent}{0pt}
\pagenumbering{arabic}

\section*{Learning in Practice}

\subsection*{Datasets}

In theory, we have an underlying
probability space:
\[ (\Omega, \acal, \pbb) \]
And a data space:
\[(\xcal\times \ycal, \xscr \otimes \yscr, 
\pbb_{X, Y})\]
Where $\pbb_{X, Y}$ is the joint distribution 
and $\xcal, \ycal$
are the input and output spaces. \\

In practice though, we don't have access
to the entire distribution. \\

Instead, what we have is a random dataset
from the data space,
which is a random variable
\[ S: \Omega \to (\xcal, \ycal)^n \ \]
Which is composed of $n$ data points:
\[ S_i: \Omega \to \xcal, \ycal\ \]
Each data point is just a pair
of input and output from the random
variables $X$ and $Y$:
\[ S_i = (X_i, Y_i)\]
For $1 \leq i \leq n$. \\

Note that the random variable is
a random variable of datasets of size $n$,
meaning that each element of this
random variable (each output $S(\omega)$)
is a dataset with $n$ elements. \\

So $S_i$ can be thought of as the random
variable of one data point
(not the datapoint itself). \\

Note that this induces a probability
measure for sample points $\pbb_S$,
with the $\sigma$-algebra
\[ \sscr = (\xscr \otimes \yscr)^{\otimes n} \]
and sample space:
\[ \scal = (\xcal \times \ycal)^n \]
Which form a measure space:
\[ ((\xcal, \ycal)^n,
(\xscr \otimes \yscr)^{\otimes n}, \pbb_S) \]
for $S$, the random variable. \\

Note that one instance of the random variable $S$
is an actual dataset:
\[ B = (x_i, y_i)^n 
\in (\xcal\times \ycal)^n \]
Recall that $S$ is not a dataset,
but a random variable of datasets,
with elements that are datasets. \\
Then $B_i$ is just one data point $(x_i, y_i)$,
which comes from the random variable
$S_i$ (a random variable for one data point). \\

We will always assume that the random variables
$S_i$ of $S$
(the random variables of each data point)
are independently and identity distributed.
This works out surprisingly well. \\

For example,
if we have a dataset that includes images
of dogs with labels of their breed,
we epect that each image (data point)
is identically distributed to the others,
but is also completely independently
distributed. \\
The frames of a video for example would not
be considered independent,
as they have temporal dependence. \\

\newpage

\subsection*{Empirical Performance Measure}

We can't use Baye's risk to measure the performance
of a classifier in practice since
it requires full knowledge of the distribution. \\

Instead, given a loss function:
\[ \ell: \ycal\times \ycal\to \rbb^+ \]
And a classifier:
\[ f: \xcal\to \ycal\]
And a dataset:
\[ B = (x_i, y_i)^n \in (\xcal\times \ycal)^n \]
Then what we actually use to measure the performance
of the classifier is the empirical risk:
\[ \hat{\rscr}_\ell^n(f)
= \dfrac{1}{n}\sum_{i = 1}^{n}\ell(f(x_i), y_i) \]
Note that this is just the mean (average) of the
dataset,
which really is just the empirical analogue
of the expected value, which is also just
an average value of the distribution. \\

The empirical risk $\hat{\rscr}_\ell^n(f)$
is deterministic for a fixed dataset
$B \in S$,
outputting one number. \\

The empirical risk $\hat{\rscr}_\ell^n(f)$
is that of a fixed dataset $B \in S$. \\
If instead we learn $f$
from some random dataset in $S$
(the random variable of all datasets),
then $f_S$ depends not only on the dataset,
but also on the randomness of $S$.
So the empirical risk:
\[ \hat{\rscr}_\ell^S(f_S) \]
is not deterministic (one number),
but is itself a random variable
(a function $h(S)$ of the random 
variable $S$). \\
We mentioned this briefly in the first
lecture, saying that the risk
in practice also depends on the randomness
(choice) of the dataset. \\

When we are done training our classifier
on dataset $B$ from the random dataset variable $S$,
we have a function $f_S$. \\
We then have to evaulate the performance of $S$
on the testing data instead of the new training data,
which means we calculate the empirical risk,
but this time over the testing dataset,
which we get from another random dataset variable $S'$:
\[ \hat{\rscr}_\ell^{S'}(f_S)
= \dfrac{1}{n}\sum_{i = 1}^{n}\ell(f_S(X_i'), Y_i') \]
which depends on both the randomeness in $S$
and $S'$
(choice of training and testing data). \\

Given a fixed dataset $B \in S$,
a loss function $\ell: \ycal \times \ycal \to \rbb^+$,
and a hypothesis class $\ffrak$,
we can find a classifier $f: \xcal \to \ycal$
in $\ffrak$ that minimizes the empirical risk:
\[ \hat{f}_n = 
\min_{f \in \ffrak} \hat{\rscr}_\ell^n(f) \]
Which we call the empirical risk minimizer. \\

Note that like the empirical risk,
if we don't have a fixed dataset,
the empirical risk minimizer will also
depend on the randomness of the
random variable $S$ (the choice of dataset).
We can then use $\hat{f}_S$ to denote it,
and define it as:
\[ \hat{f}_S = 
\min_{f \in \ffrak} \hat{\rscr}_\ell^{S}(f) \] \\

We mentioned earlier that given a full
distribution, a machine learning algorithm
takes in a dataset and spits out
a classifier from a hypothesis class:
\[ \ascr: (\xcal \times \ycal)^n \to \ffrak \]
We can also write as:
\[ \ascr: \scal \to \ffrak \]
where $\scal$ is the sample space of
the random dataset $S$
($S$ is the random variable
of all datasets,
it is not a dataset itself).
The machine learning algorithm maps 
an element in the random dataset space $\scal$,
which is a dataset,
to a classifier $f \in \ffrak$. \\

The machine learning algorithm should,
in theory, spit out the empirical
risk minimizer, which is the best
classifier $\hat{f}_S$ (depending on the
choice of dataset). \\
So we can write:
\[ \ascr(\scal) = \hat{f}_S \]
That is to say,
given some dataset $B$ from the
space of datasets $\scal = (\xcal \times \ycal)^n$
($B$ a random dataset from $S$),
the machine learning algorithm 
returns the empirical risk minimizer,
depending of course on the choice
of dataset from $S$. \\

Recall that $\ffrak$ is a hypothesis class;
a family of functions with a similar structure,
and a set of parameters that themselves
get optimized for a specific task. \\

But why do we restrict $f$
to a hypothesis class when attempting
to minimize it? \\

In the last chapter, when we had a full distribution,
we just found the optimal classifier $f^*$
without focusing on one family or class
of functions $\ffrak$. \\
The reason why we did that is that we had the
full distribution and data space.
So what we ended up doing was finding the
optimal prediction $f^*(x)$
for each individual input $x$,
by using the marginal distribution 
$\pbb_{Y \mid X = x}$ and attempting to maximize
it for each point $x$. \\

However, if we do that in practice,
given one dataset of size $n$,
we would only have a function defined
for those $n$ data points.
It would not actually generalize to new
data, including the testing data. \\

So instead, we restrict $f$
to be part of some hypothesis class
with some structure imposed on the function,
and a set of parameters that tune the
function (for example, a class of line functions
with parameters including the slope 
and $y$ intercept). \\
So finding the optimal parameters that make
the function fit the training dataset best
is very likely to generalize well
to data outside the training dataset. \\

Note that as the number of data points $n$
in the dataset tends to infinity $n \to \infty$,
the dataset approximates the distribution
better and better,
and the empirical risk approaches Baye's risk
\[ \hat{\rscr}_\ell^n \to \rscr_\ell  
\qquad \text{ as } \qquad n \to \infty \]
This is called the law of large numbers. \\

\newpage

\subsection*{Hypothesis Classes and Error Types}

Now we can formalize what we mean
by a hypothesis class. \\

A hypothesis class $\ffrak_\Theta$
is a family of functions
with a parameter space $\Theta \subset \rbb^d$:
\[ \ffrak_\Theta = 
\{ f_\theta \mid \theta \in \Theta \} \]
The set of functions $f_\theta$ in $\ffrak_\Theta$
includes all functions that have these parameters,
for any combination of values. \\

For example, one family of functions is the
family of lines:
\[ \ffrak(x) = \{ wx + b \mid w, b \in \rbb \} \]
Which includes every linear function
$f: \rbb \to \rbb$. \\

The hypothesis classes are essentially
the models we use while doing machine learning,
such as SVM, Linear Regression... \\

However, not all hypothesis
classes are a good fit for a given dataset. \\
For example, if we have data that is curved
in space,
a linear hypothesis class is likely not
going to learn the data very well. \\
We can minimize the empirical risk 
$\hat{\rscr}_\ell^n(f)$ and find the
empirical risk minimizer $\hat{f}_n$,
however, while the result would be the line
of best fit (best classifier
from this hypothesis class for the given dataset),
it might not be good in general,
compared to the best classifiers from more
suitable hypothesis classes. \\

This motivates the next idea:
decomposing the excess risk
of a classifier into two parts. \\

As we know,
if we had a full distribution,
the excess risk for a classifier $f$
given some loss function $\ell$
would be the difference between
Baye's risk of $f$ and Baye's optimal
risk:
\[ \rscr_\ell(f) - \rscr_\ell^* \]
It is a measure of how far off
$f$ is from being an optimal classifier. \\

Now we can calculate this excess risk
for the empirical risk minimizer
$\hat{f}_S \in \ffrak$
which is the best classifier
from some hypothesis class that we can get
given some random dataset: 
\[ \rscr_\ell(\hat{f}_S) - \rscr_\ell^* \]
This value measures how far off
empirical risk minimizer is from
the optimal classifier for the whole distribution
(had it been known). \\
This isn't a very useful measure on its
own however; for one,
the optimal classifier $f^*$
that gives us Baye's optimal risk
is likely not a function from the
same hypothesis class $\ffrak$ as $\hat{f}_S$,
so it's not attainable. \\

Instead, what we can do is decompose this excess
risk into two parts:
\[ \rscr_\ell(\hat{f}_S) - \rscr_\ell^*
= \rscr_\ell(\hat{f}_S)
+ (\inf_{f' \in \ffrak} \rscr_\ell(f')
- \inf_{f' \in \ffrak} \rscr_\ell(f')) 
- \rscr_\ell^* \]
\[ \rscr_\ell(\hat{f}_S) - \rscr_\ell^*
= \underbrace{\para{\rscr_\ell(\hat{f}_S)
- \inf_{f' \in \ffrak} \rscr_\ell(f')}}_
{\text{Estimation error}}
+ \underbrace{\para{\inf_{f' \in \ffrak} \rscr_\ell(f')
- \rscr_\ell^*}}_{\text{Approximation error}} \]
The left term is the estimation error,
and the right term is the approximation error. \\
These are far better measures
for how far off we are,
and we will see why. \\

First note that the term 
$\inf_{f' \in \ffrak} \rscr_\ell(f')$
represents the optimal classifier 
in the hypothesis class $\ffrak$
that we can get had we the full distribution. \\
It isn't necessarily the optimal classifier,
but the optimal classifier restricted
to the hypothesis class $\ffrak$. \\

The estimation error is the difference
between the risk of the empirical risk
minimizer
and the theoretical optimal classifier
we could have found if we had the full
distribution
(restricted to the same hypothesis class). \\
It tells us how far off we are
from the best classifier whithin the hypothesis
class of $\hat{f}_S$.
So this is a good measure of how well
our machine learning algorithm learned
the data and generated a classifier $\hat{f}_S$.
This is because the optimal classifier
we compare it to is part of the same
family of function, so the distance
between it and the classifier $\hat{f}_S$
is more meaningful,
knowing that it is technically attainable
given enough data
(since it's the same function structure
and family). \\

On the other hand, the approximation error
is the difference between Baye's optimal risk,
and the risk of the theoretical
optimal classifier
restricted to the hypothesis class $\ffrak$. \\
In other words, both of these
risks are only attainable given the full
distribution,
but one of them is the best risk of all,
for any function, and the other is best
risk attained for a function in the family $\ffrak$. \\
We can thus think of this as a measure
of how much of a good fit the hypothesis
class $\ffrak$ for this distribution. \\
This measure,
unlike the estimation error,
 does not depend on the dataset at all,
it is just a way for us to evaluate the fit
of the hypothesis class for the given 
distribution.
For example, if we have a distribution
that is curved, and a hypothesis class
of linear functions, 
we expected the approximation error to be
quite high. \\

Note that both of these measure require
full access to the distribution,
which we don't have in practice,
so we can't calculate them. \\
So in practice, if a classifier is performing
poorly, we have no way of knowing 
if it's because we trained it badly
(estimation error)
or if it's because the class of functions 
we chose is a bad fit (approximation error). \\

The tradeoff between these two types of
error, which will be explored in more detail
at a later point,
gives rise to the tension between under and
over-fitting. \\

Under-fitting happens when the hypothesis class
chosen isn't complex enough to capture the
nuances of the data. \\

Over-fitting happens when the hypothesis
class is too complex, when it has too many 
degrees of freedom,
it would not only learn the data,
but would also learn some of the random noise
you'd get in a real dataset,
causing it to become very good at classifying
training data, but generalizing badly to 
unseen data. \\

There are two remedies to overfitting:
\begin{enumerate}
    \item 
    Constraining: Restricting $\ffrak$
    to a less complex class (that is nonetheless
    not so simple it can't fit the data properly). 
    \item 
    Regularization: Adding a cost penalty
    for predictions that go against our 
    prior assumptions about the data. \\
\end{enumerate}

\newpage

\subsection*{Regularization}

As mentioned in the last section,
regularization introduces a penalty
for going against prior assumptions about the
distribution. \\

Suppose we have a training dataset $B$ 
with $n$ points:
\[ B = \{(x_i, y_i) \mid 
x_i, y_i \in \rbb \}_{i = 1, \dots n} \]
a hypothesis class $\ffrak$,
and a oss function:
\[ \ell: \ycal \times \ycal \to \rbb^+ \]
Then we can define a regularization functional:
\[ \rcal: \ffrak \to \rbb^+ \]
Which is basically a function that takes
a function (classifier) as input,
and assigns a penalty for going against some
prior assumption. \\

Our goal now will be to minimize the regularized
empirical risk, that is:
\[ \min_{f \in \ffrak} \para{ 
\riskEmpirical{\ell}{n}(f) + \lambda\rcal(f)} \]
Where $\lambda \geq 0$ is the regularization
parameter. \\

We can also minimize regularized empirical
risk with a dependence on a random dataset $S$:
\[ f_S = \arg\min_{f \in \ffrak} \para{ 
\riskEmpirical{\ell}{S}(f) + \lambda\rcal(f)} \]
Note that the regularization term does not
depend on the randomness (as it is not a function
of the dataset). \\

Minimizing the regularized risk is another
example of $\ascr$, a machine learning
algorithm. \\

The regularization parameter $\lambda$
basically determines how much regularization
is taking place. \\
It determines the tradeoff between trustin
our assumptions (larger $\lambda$)
or trusting the data (smaller $\lambda$). \\

Basically, as $\lambda \to \infty$,
we no longer care about minimizing the risk;
all our resources have to go towards choosing
a minimizer $\hat{f}$ that turns the
regularization term $\rcal(f)$ to $0$. \\
As $\lambda \to \infty$,
we don't care about regularization,
as it has no effect, so we just minimize the risk
as usual. \\

Now we will look at some example of regularization
terms. \\
We will assume we have $\xcal = [0, 1]$ and
$\ycal = \rbb$. \\

We can assume that $\ffrak = C^2(\xcal, \ycal)$
(is twice continuously differentiable,
or smooth basically),
and that the values of the prediction function
change slowly. \\
\[ \rcal(f) = \dfrac{1}{2}\int_{[0, 1]} 
(f'(x))^2 dx \]
This assumption can be enforced through the term. \\

We assume that the penalty for large 
changes is proportional to the penalty for
small changes. \\
\[ \rcal(f) = \|f\|_{\text{TV}} 
= \dfrac{1}{2}\int_{[0, 1]} |f'(x)| dx \]
This assumption can be enforced through the term. \\

We assume that the values should not change
abruptly. \\
\[ \rcal(f) = \text{Lip}(f) \]
This assumption can be enforced through this term,
which uses Lipshitz continuity. \\

If we have:
\[ f_\theta(x) 
= \sumof{i=1}{d}\theta_i\varphi_i(x) \]
And assume that many of the coordinates
are $0$ (that is $\theta = 0$).
Note that $\theta$ is a vector of
parameters, not just one parameter.
\[ \rcal(f_\theta) = \|\theta\|_1 \]
Notice how the term that enforces the assumption
is equal to the 1-norm of the vector $\theta$,
the sum of the absolute value of its entries,
such that minimizing it reduces to minimizing
that sum (so $\theta \rightsquigarrow 0$). \\

\newpage

\end{document}
