
\documentclass[12pt]{article}

\usepackage[margin=1in]{geometry}


%===============================================================================
%================================== PACKAGES ===================================
%===============================================================================

% For using float option H that places figures 
% exatcly where we want them
\usepackage{float}
% makes figure font bold
\usepackage{caption}
\captionsetup[figure]{labelfont=bf}
% For text generation
\usepackage{lipsum}
% For drawing
\usepackage{tikz}
% For smaller or equal sign and not divide sign
\usepackage{amssymb}
% For the diagonal fraction
\usepackage{xfrac}
% For enumerating exercise parts with letters instead of numbers
\usepackage{enumitem}
% For dfrac, which forces the fraction to be in display mode (large) e
% even in math mode (small)
\usepackage{amsmath}
% For degree sign
\usepackage{gensymb}
% For "\mathbb" macro
\usepackage{amsfonts}
% For positioning 
\usepackage{indentfirst}
\usetikzlibrary{shapes,positioning,fit,calc}
% for adjustwidth environment
\usepackage{changepage}
% for arrow on top
\usepackage{esvect}
% for mathbb 1
\usepackage{bbm}
% for mathsrc
\usepackage[mathscr]{eucal}
% For degree sign
\usepackage{gensymb}
% For quotes
\usepackage{csquotes}
% For vertical lines
\usepackage{mathtools}
% For cols
\usepackage{multicol}

% for tikz
\usepackage{pgfplots}
\pgfplotsset{compat=1.18}
\usepackage{amsmath}
\usepgfplotslibrary{groupplots}


%===============================================================================
%==================================== FONTS ====================================
%===============================================================================


% Mathcal
\newcommand{\acal}{\mathcal{A}}
\newcommand{\bcal}{\mathcal{B}}
\newcommand{\ccal}{\mathcal{C}}
\newcommand{\dcal}{\mathcal{D}}
\newcommand{\ecal}{\mathcal{E}}
\newcommand{\fcal}{\mathcal{F}}
\newcommand{\gcal}{\mathcal{G}}
\newcommand{\hcal}{\mathcal{H}}
\newcommand{\ical}{\mathcal{I}}
\newcommand{\jcal}{\mathcal{J}}
\newcommand{\kcal}{\mathcal{K}}
\newcommand{\lcal}{\mathcal{L}}
\newcommand{\mcal}{\mathcal{M}}
\newcommand{\ncal}{\mathcal{N}}
\newcommand{\ocal}{\mathcal{O}}
\newcommand{\pcal}{\mathcal{P}}
\newcommand{\qcal}{\mathcal{Q}}
\newcommand{\rcal}{\mathcal{R}}
\newcommand{\scal}{\mathcal{S}}
\newcommand{\tcal}{\mathcal{T}}
\newcommand{\ucal}{\mathcal{U}}
\newcommand{\vcal}{\mathcal{V}}
\newcommand{\wcal}{\mathcal{W}}
\newcommand{\xcal}{\mathcal{X}}
\newcommand{\ycal}{\mathcal{Y}}
\newcommand{\zcal}{\mathcal{Z}}

% Mathfrak
\newcommand{\afrak}{\mathfrak{A}}
\newcommand{\bfrak}{\mathfrak{B}}
\newcommand{\cfrak}{\mathfrak{C}}
\newcommand{\dfrak}{\mathfrak{D}}
\newcommand{\efrak}{\mathfrak{E}}
\newcommand{\ffrak}{\mathfrak{F}}
\newcommand{\gfrak}{\mathfrak{G}}
\newcommand{\hfrak}{\mathfrak{H}}
\newcommand{\ifrak}{\mathfrak{I}}
\newcommand{\jfrak}{\mathfrak{J}}
\newcommand{\kfrak}{\mathfrak{K}}
\newcommand{\lfrak}{\mathfrak{L}}
\newcommand{\mfrak}{\mathfrak{M}}
\newcommand{\nfrak}{\mathfrak{N}}
\newcommand{\ofrak}{\mathfrak{O}}
\newcommand{\pfrak}{\mathfrak{P}}
\newcommand{\qfrak}{\mathfrak{Q}}
\newcommand{\rfrak}{\mathfrak{R}}
\newcommand{\sfrak}{\mathfrak{S}}
\newcommand{\tfrak}{\mathfrak{T}}
\newcommand{\ufrak}{\mathfrak{U}}
\newcommand{\vfrak}{\mathfrak{V}}
\newcommand{\wfrak}{\mathfrak{W}}
\newcommand{\xfrak}{\mathfrak{X}}
\newcommand{\yfrak}{\mathfrak{Y}}
\newcommand{\zfrak}{\mathfrak{Z}}

% Mathscr
\newcommand{\ascr}{\mathscr{A}}
\newcommand{\bscr}{\mathscr{B}}
\newcommand{\cscr}{\mathscr{C}}
\newcommand{\dscr}{\mathscr{D}}
\newcommand{\escr}{\mathscr{E}}
\newcommand{\fscr}{\mathscr{F}}
\newcommand{\gscr}{\mathscr{G}}
\newcommand{\hscr}{\mathscr{H}}
\newcommand{\iscr}{\mathscr{I}}
\newcommand{\jscr}{\mathscr{J}}
\newcommand{\kscr}{\mathscr{K}}
\newcommand{\lscr}{\mathscr{L}}
\newcommand{\mscr}{\mathscr{M}}
\newcommand{\nscr}{\mathscr{N}}
\newcommand{\oscr}{\mathscr{O}}
\newcommand{\pscr}{\mathscr{P}}
\newcommand{\qscr}{\mathscr{Q}}
\newcommand{\rscr}{\mathscr{R}}
\newcommand{\sscr}{\mathscr{S}}
\newcommand{\tscr}{\mathscr{T}}
\newcommand{\uscr}{\mathscr{U}}
\newcommand{\vscr}{\mathscr{V}}
\newcommand{\wscr}{\mathscr{W}}
\newcommand{\xscr}{\mathscr{X}}
\newcommand{\yscr}{\mathscr{Y}}
\newcommand{\zscr}{\mathscr{Z}}

% Mathbb
\newcommand{\abb}{\mathbb{A}}
\newcommand{\bbb}{\mathbb{B}}
\newcommand{\cbb}{\mathbb{C}}
\newcommand{\dbb}{\mathbb{D}}
\newcommand{\ebb}{\mathbb{E}}
\newcommand{\fbb}{\mathbb{F}}
\newcommand{\gbb}{\mathbb{G}}
\newcommand{\hbb}{\mathbb{H}}
\newcommand{\ibb}{\mathbb{I}}
\newcommand{\jbb}{\mathbb{J}}
\newcommand{\kbb}{\mathbb{K}}
\newcommand{\lbb}{\mathbb{L}}
\newcommand{\mbb}{\mathbb{M}}
\newcommand{\nbb}{\mathbb{N}}
\newcommand{\obb}{\mathbb{O}}
\newcommand{\pbb}{\mathbb{P}}
\newcommand{\qbb}{\mathbb{Q}}
\newcommand{\rbb}{\mathbb{R}}
\newcommand{\sbb}{\mathbb{S}}
\newcommand{\tbb}{\mathbb{T}}
\newcommand{\ubb}{\mathbb{U}}
\newcommand{\vbb}{\mathbb{V}}
\newcommand{\wbb}{\mathbb{W}}
\newcommand{\xbb}{\mathbb{X}}
\newcommand{\ybb}{\mathbb{Y}}
\newcommand{\zbb}{\mathbb{Z}}


%===============================================================================
%=============================== SPECIAL SYMBOLS ===============================
%===============================================================================


% Orbit (group theory)
\newcommand{\orbit}{\mathcal{O}}
% Normal group
\newcommand{\normal}{\mathcal{N}}
% Indicator function
\newcommand{\indicator}{\mathbbm{1}}
% Laplace transform
\newcommand{\laplace}[1]{\mathcal{L}}
% Epsilon shorthand
\newcommand{\eps}{\varepsilon}
% Omega
\newcommand{\om}{\omega}
\newcommand{\Om}{\Omega}

%===============================================================================
%================================== OPERATORS ==================================
%===============================================================================


% Inverse exponent
\newcommand{\inv}[0]{^{-1}}
% Overline bar
\newcommand{\olsi}[1]{\,\overline{\!{#1}}}
% Less than or equal slanted
\newcommand{\seqs}{\leqslant}
% Greater or equal slanted
\newcommand{\geqs}{\geqslant}
% Subset or equal
\newcommand{\sub}{\subseteq}
% Proper subset
\newcommand{\prosub}{\subset}
% from
\newcommand{\from}{\leftarrow}

% Parantheses
\newcommand{\para}[1]{\left( #1 \right)}
% Curly Braces
\newcommand{\curl}[1]{\left\{ #1 \right\}}
% Brackets
\newcommand{\brac}[1]{\left[ #1 \right]}
% Angled Brackets
\newcommand{\ang}[1]{\left\langle #1 \right\rangle}
% Norm
\newcommand{\norm}[1]{\left\| #1 \right\|}

% Piece wise (use \\ between cases)
\newcommand{\piecewise}[1]{\begin{cases} #1 \end{cases}}

% Bold symbol shorthand
\newcommand{\bl}{\boldsymbol}

% Vertical space
\newcommand{\vs}[1]{\vspace{#1 pt}}
% Horizontal ertical space
\newcommand{\hs}[1]{\hspace{#1 pt}}


%===============================================================================
%============================== TEXT BASED SYMBOLS =============================
%===============================================================================


% Radians
\newcommand{\rad}{\text{rad}}
% Least Common Multiple
\newcommand{\lcm}{\text{lcm}}
% Automorphism
\newcommand{\Aut}{\text{Aut}}
% Variance
\newcommand{\var}{\text{Var}}
% Covariance
\newcommand{\cov}{\text{Cov}}
% Cofactor (matrix)
\newcommand{\cof}{\text{Cof}}
% Adjugate (matrix)
\newcommand{\adj}{\text{Adj}}
% Trace (matrix)
\newcommand{\tr}{\text{tr}}
% Standard deviation
\newcommand{\std}{\text{Std}}
% Correlation coefficient
\newcommand{\corr}{\text{Corr}}
% Sign
\newcommand{\sign}{\text{sign}}

% And text
\newcommand{\AND}{\text{ and }}
% Or text 
\newcommand{\OR}{\text{ or }}
% For text 
\newcommand{\FOR}{\text{ for }}
% If text
\newcommand{\IF}{\text{ if }}
% When text
\newcommand{\WHEN}{\text{ when }}
% Where text
\newcommand{\WHERE}{\text{ where }}
% Then text
\newcommand{\THEN}{\text{ then }}
% Such that text
\newcommand{\SUCHTHAT}{\text{ such that }}

% 1st
\newcommand{\st}[1]{#1^{\text{st}}}
% 2nd
\newcommand{\nd}[1]{#1^{\text{nd}}}
% 3rd
\newcommand{\rd}[1]{#1^{\text{rd}}}
% nth
\newcommand{\nth}[1]{#1^{\text{th}}}


%===============================================================================
%========================= PROBABILITY AND STATISTICS ==========================
%===============================================================================


% Permutation
\newcommand{\perm}[2]{{}^{#1}\!P_{#2}}
% Combination
\newcommand{\comb}[2]{{}^{#1}C_{#2}}

% Baye's risk
\newcommand{\risk}[1]{\mathscr{R}_{#1}}
% Baye's optimal risk
\newcommand{\riskOptimal}[1]{\mathscr{R}_{#1}^*}
% Baye's empirical risk
\newcommand{\riskEmpirical}[2]{\hat{\mathscr{R}}_{#1}^{#2}}


%===============================================================================
%=================================== CALCULUS ==================================
%===============================================================================


% d over d derivative
\newcommand{\dd}[2]{\dfrac{d#1}{d#2}}
% partial d over d derivative
\newcommand{\partialdd}[2]{\dfrac{\partial #1}{\partial #2}}
% delta d over d derivative
\newcommand{\deltadd}[2]{\dfrac{\Delta #1}{\Delta #2}}

% Integration between a and b
\newcommand{\integral}[4]{\int_{#1}^{#2} #3 \, #4}
% Integration in some space 
\newcommand{\boundIntegral}[2]{\int_{#1} #2 \, d#1}

% Limit
\newcommand{\limit}[3]{\lim_{#1 \to #2} #3}


%===============================================================================
%================================  BIG SYMBOLS  ================================
%===============================================================================


% Sum
\newcommand{\sumof}[2]{\sum_{#1}^{#2}}
% Product
\newcommand{\productof}[2]{\prod_{#1}^{#2}}
% Union
\newcommand{\unionof}[2]{\bigcup_{#1}^{#2}}
% Intersection
\newcommand{\intersectionof}[2]{\bigcap_{#1}^{#2}}
% Or
\newcommand{\orof}[2]{\bigvee_{#1}^{#2}}
% And
\newcommand{\andof}[2]{\bigwedge_{#1}^{#2}}


%===============================================================================
%=============================== LINEAR ALGEBRA ================================
%===============================================================================


% Bold vector arrow
\newcommand{\bv}[1]{\vec{\mathbf{#1}}}

% Matrix or vector (use // for column, & for row) 
% with brackets
\newcommand{\bmat}[1]{\begin{bmatrix} #1 \end{bmatrix}}
% Matrix or vector (use // for column, & for row) 
% with curved brackets
\newcommand{\pmat}[1]{\begin{pmatrix} #1 \end{pmatrix}}
% Matrix or vector (use // for column, & for row) 
% with lines on either side 
\newcommand{\lmat}[1]{\begin{vmatrix} #1 \end{vmatrix}}
% Matrix or vector (use // for column, & for row) 
% with curly braces
\newcommand{\cmat}[1]{\begin{Bmatrix} #1 \end{Bmatrix}}
% Matrix or vector (use // for column, & for row) 
% with no braces
\newcommand{\mat}[1]{\begin{matrix} #1 \end{matrix}}


%===============================================================================
%================================ LARGE OBJECTS ================================
%===============================================================================

% Multiple lines
\newcommand{\multiline}[1]{
\begin{align*}
    #1
\end{align*}
}

% Multiple lines with equation numbers
\newcommand{\eqmultiline}[1]{
\begin{align*}
    #1
\end{align*}
}

% Color
\newcommand{\colorText}[2]{
\begingroup
\color{#1}
    #2
\endgroup
}

% Centered figure
\newcommand{\centerFigure}[2]{
    \begin{figure}[h]
        \centering
            #1
        \caption{#2}
    \end{figure}
}

% Tikz figure
\newcommand{\tikzGraphic}[1]{
    \begin{center}
    \begin{tikzpicture}
        #1
    \end{tikzpicture}
    \end{center}
}

% Enumerate numbers (seperate by \item)
\newcommand{\numbers}{\textbf{\number*)}}

% Enumerate letters (seperate by \item)
\newcommand{\letters}{\textbf{\alph*)}}

\title{%
    \Huge Machine Learning \\
    \Large Lecture III
}
\date{2025-05-05}
\author{Michael Saba}

\begin{document}
\pagenumbering{gobble}
\maketitle
\newpage
\setlength{\parindent}{0pt}
\pagenumbering{arabic}

\section*{Bayesian Learning Theory}

\subsection*{Proof of Classification Calibration}

This section is optional. \\
It focuses on a proof that $\Phi$
is classification calibrated
if and only if $\Phi'(0)$ exists
and $\Phi'(0) < 0$. \\

In the last lecture we
talked about classification calibrated
loss functions $\Phi$,
in the context of a binary classification
of $\ycal = \{-1, 1\}$,
and real valued surrogate prediction
functions. \\

First we define a regression:
\[ \eta: \xcal \to [0, 1] \]
Which is given by:
\[ \eta(x) = \Pb_{Y \mid X = x}(1)
= \Pb(Y = 1 \mid X = x) \]
Which is essentially the probability
that $Y$ is the label $1$
given $x$. \\

We can write the expected risk over $Y$ as:
\[ E[Y \mid X = x]
= \Pb(Y = 1\mid X = x)(1)
+ \Pb(Y = -1 \mid X = x)(-1) \]
\[ E[Y \mid X = x] 
= \Pb(Y = 1\mid X = x)
- \Pb(Y = -1 \mid X = x) \]
\[  E[Y \mid X = x] 
= \Pb(Y = 1\mid X = x)
- (1 - \Pb(Y = 1\mid X = x)) \]
\[ E[Y \mid X = x] 
= 2\eta(x) - 1  \]
As we know, we have outputs
$-1$ and $1$ which multiply the
probabilities that each label $-1$
and $1$ will take place. \\
So if the result of the expected
value is larger than $0$
then: 
\[ \Pb(Y = 1 \mid X = x) >
\Pb(Y = -1 \mid X = x) \]
Which means that best possible
classification $f^*(x)$
for $x$
that we can output
is going to be $1$ if $2\eta(x) - 1 > 1$,
and the best output is $-1$ 
if $2\eta(x) - 1 < 1$. \\

So $f^*(x) = \sign(2\eta(x) - 1)$,
which means that the optimal prediction
is just $\eta(x)$ threshholded
at $0.5$. \\
Note that these numbers would change
if the labels of $\ycal$
weren't $1$ and $-1$,
but the general idea is the same;
we could threshold the regression
to get the optimal prediction $f^*(x)$,
which just depends on the probability
$\Pb(Y \mid X = x)$. \\

We can use the regression function
to prove that a loss function $\Phi$
is classification-calibrarted
if it is differentiable at $0$
and $\Phi(0)' < 0$. \\

First note that:
\[ \rscr_\Phi(g)
= \E[\Phi(Yg(X))]
= \E[\E[\Phi(Yg(X))] \mid X ] \]
We can then unfurl the expected value
over $Y$
(recall that $\ycal = \{-1, 1\}$):
\[ \E[\Pb(Y = -1 \mid X)\Phi(-g(X))
+ \Pb(Y = 1 \mid X)\Phi(g(X))] \]
\[ \E[(1-\eta(X))\Phi(-g(X))
+ \eta(X)\Phi(g(X))] \]
If we take $\xi = \eta(x)$,
then we can write:
\[ r_\Phi(z \mid \xi)
= (1-\xi)\Phi(-z) + \xi\Phi(z) \]
Which is the conditional risk for
a single prediction $z$
on one fixed output $x$. \\

Similarly to before, the problem of optimizing
$\rscr_\Phi(g)$ reduces to optimizing
$r_\Phi(z \mid \xi)$ for each point $x$
(pointwise optimization). \\
This is because we can just sum over
every conditional risk weighted by the probability
that $x$ will take place,
and get the total risk,
which means that:
\[ \rscr_\Phi(g) = E[r_\Phi(z \mid \xi)]
= E[r_\Phi(g(X) \mid \eta(X))] \]
which is just one expected value over $X$. \\

Now, finally, to optimize $r_\Phi(z \mid \xi)$. \\
We want to prove that 
the minimal value of $r_\Phi(z \mid \xi)$
matches that of Baye's classifier $f^*$
if and only if $\Phi$
is differentiable at $0$
and $\Phi(0)' < 0$. \\

Recall that:
\[ f^*(x) = \piecewise{1 \quad \;\;\; \text{ if }
\Pb(Y = 1 \mid X) >
\Pb(Y = -1 \mid X)}
{-1 \quad \text{ otherwise }} \]
is the optimal classifier. \\

Here, $\Pb(Y = 1 \mid X) > \Pb(Y = -1 \mid X)$
means that $\Pb(Y = 1 \mid X) > 0.5$,
which means that $\xi > 0.5$,
which means that $2\xi - 1 > 0$. \\
So in short, 
the minimal value of $r_\Phi(z \mid \xi)$
matches that of Baye's classifier $f^*$ when
these hold:
\[ 2\xi - 1 > 0 \iff 
\arg \min_z r_\Phi(z \mid \xi) \in (0, \infty) \]
\[ 2\xi - 1 < 0 \iff 
\arg \min_z r_\Phi(z \mid \xi) \in (-\infty, 0) \]
Meaning that the values that minimize the
conditional risk
should be the same as the outputs of Baye's
optimal classifier $f^*$. \\

Since $\Phi$ was assumed to be convex,
then so is $r_\Phi(z \mid \xi)$. \\
We can define $c_\xi(z) = r_\Phi(z \mid \xi)$
for simplicty. \\

Now note that:
\[ \arg \min_z r_\Phi(z \mid \xi) \in (0, \infty)
\iff c_\xi'(0; +1) < 0 \]
Where the derivative is:
\[ c_\xi'(0; +1) 
= (1-\xi)\Phi'(0; -1) + \xi\Phi'(0; +1)\] 
This is true because $c$ is convex. \\
Since $c_\xi$ is convex,
if the global minimizer lies after the point $0$
(in $(0, \infty)$),
then the function must be descreasing
up to that point,
which means that the derivatie of the function
at $0$ is negative.
So this means that:
\[ c_\xi'(0; +1) < 0\iff 
\arg \min_z r_\Phi(z \mid \xi) \in (0, \infty) \]
As well as (for the same reason):
\[ c_\xi'(0; +1) > 0 \iff 
\arg \min_z r_\Phi(z \mid \xi) \in (-\infty, 0) \]
So we can now say that $\Phi$ is classification
calibrated:
\[ 2\xi - 1 > 0 \iff 
\arg \min_z r_\Phi(z \mid \xi) \in (0, \infty)
\iff c_\xi'(0; +1) < 0 \]
\[ 2\xi - 1 < 0 \iff 
\arg \min_z r_\Phi(z \mid \xi) \in (-\infty, 0)
\iff c_\xi'(0; +1) > 0 \]
when these statements hold. \\

So now, let's prove that:
\[ \Phi'(0) \text{ exists and } \Phi'(0) < 0
\iff z \text{ that minimizes } r_\Phi 
\text{ matches } f^* \]
The right hand side of this equation can be
shown to be true using the equations we defined previously:
\[ 2\xi - 1 > 0 \iff 
\arg \min_z r_\Phi(z \mid \xi) \in (0, \infty)
\iff c_\xi'(0; +1) < 0 \]
\[ 2\xi - 1 < 0 \iff 
\arg \min_z r_\Phi(z \mid \xi) \in (-\infty, 0)
\iff c_\xi'(0; +1) > 0 \]
So with that we can prove this statement. \\

For the $\impliedby$ direction of the statement,
we will assume the right hand side is true,
and we will try to show that this implies 
the left hand side. \\
First notice that
if we approach $\xi \to \sfrac{1}{2}$
from the positive side:
\[ \limit{\xi}{\sfrac{1}{2}}{c_\xi'(0; +1) = 
\dfrac{1}{2}(\Phi'(0; +1) - \Phi'(0; -1))} \leq 0 \]
It's smaller than $0$ because as mentioned
earlier, $c_\xi'(0; +1) \leq 0$ for $\xi \geq 0$,
and we are approaching from the positive side. \\
Moreover, this implies that:
\[\Phi'(0; +1) \leq \Phi'(0; -1)\]
However, also note that because $\Phi$
is convex, then it's derivative must be increasing
(cup shape), then by definition:
\[\Phi'(0; +1) \geq \Phi'(0; -1)\]
So that must mean that:
\[\Phi'(0; +1) = \Phi'(0; -1)\]
which means that $\Phi$ is differentiable at $0$. \\
Now that we know it exists, we have that:
\[ c_\xi'(0) = (2\xi - 1)\Phi'(0) \] 
If $\xi > 0.5$,
then $c_\xi'(0) < 0$
(which we proved earlier and now assume is true).
This means that the sign of $c_\xi'(0; +1) $
is negative, 
and the sign of $(2\xi - 1)$ is positive,
which means that $\Phi'(0)$ must have a 
negative sign. \\
Likewise, if $\xi < 0.5$,
then $c_\xi'(0) > 0$
(which we proved earlier and now assume is true).
This means that the sign of $c_\xi'(0; +1) $
is positive, 
and the sign of $(2\xi - 1)$ is negative,
which means that $\Phi'(0)$ must have a negative sign. \\
So either way, $\Phi'(0) < 0$. \\

Now to prove the converse direction $\implies$,
we assume that $\Phi'(0)$ exists
and that $\Phi'(0) < 0$. \\
Since we have that:
\[ c_\xi'(0) = (2\xi - 1)\Phi'(0) \] 
If $\xi > 0.5$,
then $(2\xi - 1)$ is positive,
which means that $c_\xi'(0)$ will be negative.
This implies that the solution is to to
the positive side of $0$ in a convex function,
since we are descreasing in the positive direction.
So:
\[ \arg \min_z r_\Phi(z \mid \xi) \in (0, \infty)\]
Moreover, if $\xi < 0.5$,
then $(2\xi - 1)$ is negative,
which means that $c_\xi'(0)$ will be positive.
This implies that the solution is to to
the negative side of $0$ in a convex function,
since we are increasing in the positive direction.
So:
\[ \arg \min_z r_\Phi(z \mid \xi) \in (-\infty, 0) \]
This shows that $\Phi$
will lead to the optimal output $z = f^*$. \\

\newpage

\subsection*{Multi-Class Classification}

We can often decompose multiclass classifications
into multiple binary class classifications;
this makes working with them easier,
especially when most theorems we have derived 
are for binary classifications. \\

A multiclass prediciton is one where we have:
\[ \ycal = \{1, 2, 3, \dots K\} \]
many classes, and where the classifier is
a function:
\[ f: \xcal \to \{1, 2, 3, \dots K\} \]
returning one of these classes. \\

If we decompose the classification into multiple
binary classifiactions,
we would have several classifiers $f$,
as we will see soon. \\

The first approach is called One vs All. \\
If we have $K$ classes in $\ycal$,
then we can create $K$ binary classifiers,
one for each class. \\
Then the binary classifier of class $l$
would take $x$ as input and return
not a label, but the probability of $x$
having label $l$: 
\[ f_l: \xcal \to \R^+ \]
We can then form our multiclass classifier
$f_{\text{OVA}}(x)$ as such:
\[ f_{\text{OVA}}(x) = 
\arg \max_l f_l(x)  \]
Which is just the classifier that 
has the largest confidence in its class. \\

The other strategy is called One vs One. \\
We create $\comb{n}{2}$
binary classifiers, 
one for each combination of labels,
where $f_{lm}(x)$
returns either label $l$ or $m$
(represented by $-1$ and $1$)
depending on which has a higher probability
given $x$:
\[ f_{lm}: \xcal \to \{-1, 1\} \]
We can then form our multiclass classifier
$f_{\text{OVO}}(x)$ as such:
\[ f_{\text{OVO}}(x) = 
\arg \max_l \sum_{m=1, \; m \neq l}^{K} 
\indicator_{f_{lm}(x) > 0}  \]
So basically the indicator function
is $1$ when the probability of $l$
is larger than that of $m$. \\
So the idea of the sum is that it counts
the number of classifiers from $m=1$
to $m=K$ that favour label $l$
over the label $m$. \\
That classifier that is the most favored
(has the largest number of binary classifiers
where it was chosen instead of the other class)
is then chosen. \\

Both of these strategies lead to Baye's
optimal classifier of the multiclass classification,
as long as the individual binary classifiers
are strictly monotonically increasing
functions of the conditinonal
ditribution (if the probability that $Y = l$
increases, so does $f_l$). \\

If we don't use of these methods with a binary
classification, we could go directly
to multiclass classification with a surrogate
function $g$. \\

If we have $K$ classes, then $g$
outputs a vector of $K$ values:
\[ g: \xcal \to \R^K \]
where each value gives the confidence
of $g$ that the class that corresponds to this
entry in the vector is the correct class. \\

For this classifier we can use the softmax loss:
\[ \ell: \R^K \ycal \to \R^+ \]
\[ \ell(g(x), y) = -\log\para{\dfrac{e^{g_y(x)}}
{\sum_{k = 1}^K e^{g_k(x)}}} \]
Which will have low cost if the $\nth(y)$
entry where $y$ is the correct label
has a high (confidence) value. \\

We intuitively expect that a good classifier
$g(x)$ must scale the outputs of each entry $k$
in the vector with the probability
that $Y = k$ given $x$. \\

In one of the exercises in assignment 3,
we differentiate the conditional risk for an input $x$,
and set it to $0$,
and then derive the optimal classifier,
which will have a format:
\[ g^*(x)
    = \begin{bmatrix}
    \log\para{\alpha
    \Pb(Y = 1 \mid X = x)} \\
    \log\para{\alpha
    \Pb(Y = 2 \mid X = x)} \\ 
    \vdots \\
    \log\para{\alpha
    \Pb(Y = K \mid X = x)} \\ 
\end{bmatrix} \]
where $\log$ is strictly increasing
and $\alpha$ is a positive value
that does not change between entries. \\
In other words,
this means that the maximal entry
is the one that corresponds to the
highest probability:
\[ \arg\max_k g^*_k(x) = \arg \max_k
\Pb(Y = k \mid X = x) \]
So since the entry with the highest value
in the output of $g^*(x)$
is the optimal output (the one with the
highest probability), $f^*(x)$:
\[ \arg\max_k g^*_k(x) = f^*(x) \]
Which makes the optimal surrogate
classifier $g^*$ the same as $f^*$
while using softmax loss. \\

\newpage

\subsection*{Regression}

In regression, the output is continuous,
instead of discrete. \\

So, we would have a data space:
\[ \xcal  \AND \ycal = \R^{d_y}\]
where $d_y$ is the dimension of
the output for that specific output point. \\

Every single formula regarding risk
and conditional risk and expected value
is the same. \\

The classifier:
\[ f: \xcal \to \ycal \]
is now called the regressor. \\

However, the loss used is not usually
the 0-1 loss, because it is too strict
for a regression problem;
if the expected output $y$
and the predicted value $f(x)$
are very close, but not exactly the same,
we still want the cost to be low. \\

In fact, there is no one best loss function;
it will depend on the problem at hand. \\

One example of a loss function used in regression
is the euclidian squared norm,
which calculates the distance squared
between the expected output $y$
and the predicted ouput $f(x)$:
\[ \ell(f(x), y) = \|f(x) - y\|^2 \]
Note that this loss yields the following
result:
\[ \E[\ell(f(x), Y) \mid X = x]
= \E[\|f(x) - Y\|^2  \mid X = x]
= \E[Y \mid X = x] \]
which was proven in the exerices. \\
Minimizing this value (which is
just the conditional risk at a single
input $X = x$)
gives us the optimal output for $x$:
\[ f^*(x) = \arg\min_{f(x)}
\E[\|f(x) - Y\|^2  \mid X = x]
= \E[Y \mid X = x] = \text{mean}_{X=x}(Y) \]
Which is just the mean since that is
what the expected value is. \\

In fact, the result is just the expected
value of $Y$ given $X = x$,
which is the average value of $Y$
for the fixed input $x$. \\
This makes sense, to get a prediction that minimizes
the distance between several other points,
you just find the average (center of mass)
of the points. \\

Another loss function we can use is the
$L_1$ norm,
which uses the $1$-norm
(sum of absolute value of entries):
\[ \ell(f(x), y) = \|f(x) - y\|_1 \]
Again, minimizing the expected value of the
loss given $X = x$
gives us the optimal prediction $f^*(x)$
for a fixed $x$:
\[ f^*(x) = \arg\min_{f(x)}
\E[\|f(x) - Y\|_1  \mid X = x]
= \text{median}_{X=x}(Y) \]
This is again a well known value, but it 
is the median. \\

Note that $L_1$ loss having the optimal classifier
return the median of the $Y$ distribution given $x$,
instead of the mean,
tells us that it is more robust towards outliers,
since by definition the median
is less sensitive to outliers. \\

The final part of the lecture discusses
noise-sensitive loss functions,
which train the classifiers to offset noise. \\

\newpage

\end{document}
