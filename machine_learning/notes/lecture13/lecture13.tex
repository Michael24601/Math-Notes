
\documentclass[12pt]{article}

\usepackage[margin=1in]{geometry}


%===============================================================================
%================================== PACKAGES ===================================
%===============================================================================

% For using float option H that places figures 
% exatcly where we want them
\usepackage{float}
% makes figure font bold
\usepackage{caption}
\captionsetup[figure]{labelfont=bf}
% For text generation
\usepackage{lipsum}
% For drawing
\usepackage{tikz}
% For smaller or equal sign and not divide sign
\usepackage{amssymb}
% For the diagonal fraction
\usepackage{xfrac}
% For enumerating exercise parts with letters instead of numbers
\usepackage{enumitem}
% For dfrac, which forces the fraction to be in display mode (large) e
% even in math mode (small)
\usepackage{amsmath}
% For degree sign
\usepackage{gensymb}
% For "\mathbb" macro
\usepackage{amsfonts}
% For positioning 
\usepackage{indentfirst}
\usetikzlibrary{shapes,positioning,fit,calc}
% for adjustwidth environment
\usepackage{changepage}
% for arrow on top
\usepackage{esvect}
% for mathbb 1
\usepackage{bbm}
% for mathsrc
\usepackage[mathscr]{eucal}
% For degree sign
\usepackage{gensymb}
% For quotes
\usepackage{csquotes}
% For vertical lines
\usepackage{mathtools}
% For cols
\usepackage{multicol}

% for tikz
\usepackage{pgfplots}
\pgfplotsset{compat=1.18}
\usepackage{amsmath}
\usepgfplotslibrary{groupplots}


%===============================================================================
%==================================== FONTS ====================================
%===============================================================================


% Mathcal
\newcommand{\acal}{\mathcal{A}}
\newcommand{\bcal}{\mathcal{B}}
\newcommand{\ccal}{\mathcal{C}}
\newcommand{\dcal}{\mathcal{D}}
\newcommand{\ecal}{\mathcal{E}}
\newcommand{\fcal}{\mathcal{F}}
\newcommand{\gcal}{\mathcal{G}}
\newcommand{\hcal}{\mathcal{H}}
\newcommand{\ical}{\mathcal{I}}
\newcommand{\jcal}{\mathcal{J}}
\newcommand{\kcal}{\mathcal{K}}
\newcommand{\lcal}{\mathcal{L}}
\newcommand{\mcal}{\mathcal{M}}
\newcommand{\ncal}{\mathcal{N}}
\newcommand{\ocal}{\mathcal{O}}
\newcommand{\pcal}{\mathcal{P}}
\newcommand{\qcal}{\mathcal{Q}}
\newcommand{\rcal}{\mathcal{R}}
\newcommand{\scal}{\mathcal{S}}
\newcommand{\tcal}{\mathcal{T}}
\newcommand{\ucal}{\mathcal{U}}
\newcommand{\vcal}{\mathcal{V}}
\newcommand{\wcal}{\mathcal{W}}
\newcommand{\xcal}{\mathcal{X}}
\newcommand{\ycal}{\mathcal{Y}}
\newcommand{\zcal}{\mathcal{Z}}

% Mathfrak
\newcommand{\afrak}{\mathfrak{A}}
\newcommand{\bfrak}{\mathfrak{B}}
\newcommand{\cfrak}{\mathfrak{C}}
\newcommand{\dfrak}{\mathfrak{D}}
\newcommand{\efrak}{\mathfrak{E}}
\newcommand{\ffrak}{\mathfrak{F}}
\newcommand{\gfrak}{\mathfrak{G}}
\newcommand{\hfrak}{\mathfrak{H}}
\newcommand{\ifrak}{\mathfrak{I}}
\newcommand{\jfrak}{\mathfrak{J}}
\newcommand{\kfrak}{\mathfrak{K}}
\newcommand{\lfrak}{\mathfrak{L}}
\newcommand{\mfrak}{\mathfrak{M}}
\newcommand{\nfrak}{\mathfrak{N}}
\newcommand{\ofrak}{\mathfrak{O}}
\newcommand{\pfrak}{\mathfrak{P}}
\newcommand{\qfrak}{\mathfrak{Q}}
\newcommand{\rfrak}{\mathfrak{R}}
\newcommand{\sfrak}{\mathfrak{S}}
\newcommand{\tfrak}{\mathfrak{T}}
\newcommand{\ufrak}{\mathfrak{U}}
\newcommand{\vfrak}{\mathfrak{V}}
\newcommand{\wfrak}{\mathfrak{W}}
\newcommand{\xfrak}{\mathfrak{X}}
\newcommand{\yfrak}{\mathfrak{Y}}
\newcommand{\zfrak}{\mathfrak{Z}}

% Mathscr
\newcommand{\ascr}{\mathscr{A}}
\newcommand{\bscr}{\mathscr{B}}
\newcommand{\cscr}{\mathscr{C}}
\newcommand{\dscr}{\mathscr{D}}
\newcommand{\escr}{\mathscr{E}}
\newcommand{\fscr}{\mathscr{F}}
\newcommand{\gscr}{\mathscr{G}}
\newcommand{\hscr}{\mathscr{H}}
\newcommand{\iscr}{\mathscr{I}}
\newcommand{\jscr}{\mathscr{J}}
\newcommand{\kscr}{\mathscr{K}}
\newcommand{\lscr}{\mathscr{L}}
\newcommand{\mscr}{\mathscr{M}}
\newcommand{\nscr}{\mathscr{N}}
\newcommand{\oscr}{\mathscr{O}}
\newcommand{\pscr}{\mathscr{P}}
\newcommand{\qscr}{\mathscr{Q}}
\newcommand{\rscr}{\mathscr{R}}
\newcommand{\sscr}{\mathscr{S}}
\newcommand{\tscr}{\mathscr{T}}
\newcommand{\uscr}{\mathscr{U}}
\newcommand{\vscr}{\mathscr{V}}
\newcommand{\wscr}{\mathscr{W}}
\newcommand{\xscr}{\mathscr{X}}
\newcommand{\yscr}{\mathscr{Y}}
\newcommand{\zscr}{\mathscr{Z}}

% Mathbb
\newcommand{\abb}{\mathbb{A}}
\newcommand{\bbb}{\mathbb{B}}
\newcommand{\cbb}{\mathbb{C}}
\newcommand{\dbb}{\mathbb{D}}
\newcommand{\ebb}{\mathbb{E}}
\newcommand{\fbb}{\mathbb{F}}
\newcommand{\gbb}{\mathbb{G}}
\newcommand{\hbb}{\mathbb{H}}
\newcommand{\ibb}{\mathbb{I}}
\newcommand{\jbb}{\mathbb{J}}
\newcommand{\kbb}{\mathbb{K}}
\newcommand{\lbb}{\mathbb{L}}
\newcommand{\mbb}{\mathbb{M}}
\newcommand{\nbb}{\mathbb{N}}
\newcommand{\obb}{\mathbb{O}}
\newcommand{\pbb}{\mathbb{P}}
\newcommand{\qbb}{\mathbb{Q}}
\newcommand{\rbb}{\mathbb{R}}
\newcommand{\sbb}{\mathbb{S}}
\newcommand{\tbb}{\mathbb{T}}
\newcommand{\ubb}{\mathbb{U}}
\newcommand{\vbb}{\mathbb{V}}
\newcommand{\wbb}{\mathbb{W}}
\newcommand{\xbb}{\mathbb{X}}
\newcommand{\ybb}{\mathbb{Y}}
\newcommand{\zbb}{\mathbb{Z}}


%===============================================================================
%=============================== SPECIAL SYMBOLS ===============================
%===============================================================================


% Orbit (group theory)
\newcommand{\orbit}{\mathcal{O}}
% Normal group
\newcommand{\normal}{\mathcal{N}}
% Indicator function
\newcommand{\indicator}{\mathbbm{1}}
% Laplace transform
\newcommand{\laplace}[1]{\mathcal{L}}
% Epsilon shorthand
\newcommand{\eps}{\varepsilon}
% Omega
\newcommand{\om}{\omega}
\newcommand{\Om}{\Omega}

%===============================================================================
%================================== OPERATORS ==================================
%===============================================================================


% Inverse exponent
\newcommand{\inv}[0]{^{-1}}
% Overline bar
\newcommand{\olsi}[1]{\,\overline{\!{#1}}}
% Less than or equal slanted
\newcommand{\seqs}{\leqslant}
% Greater or equal slanted
\newcommand{\geqs}{\geqslant}
% Subset or equal
\newcommand{\sub}{\subseteq}
% Proper subset
\newcommand{\prosub}{\subset}
% from
\newcommand{\from}{\leftarrow}

% Parantheses
\newcommand{\para}[1]{\left( #1 \right)}
% Curly Braces
\newcommand{\curl}[1]{\left\{ #1 \right\}}
% Brackets
\newcommand{\brac}[1]{\left[ #1 \right]}
% Angled Brackets
\newcommand{\ang}[1]{\left\langle #1 \right\rangle}
% Norm
\newcommand{\norm}[1]{\left\| #1 \right\|}

% Piece wise (use \\ between cases)
\newcommand{\piecewise}[1]{\begin{cases} #1 \end{cases}}

% Bold symbol shorthand
\newcommand{\bl}{\boldsymbol}

% Vertical space
\newcommand{\vs}[1]{\vspace{#1 pt}}
% Horizontal ertical space
\newcommand{\hs}[1]{\hspace{#1 pt}}


%===============================================================================
%============================== TEXT BASED SYMBOLS =============================
%===============================================================================


% Radians
\newcommand{\rad}{\text{rad}}
% Least Common Multiple
\newcommand{\lcm}{\text{lcm}}
% Automorphism
\newcommand{\Aut}{\text{Aut}}
% Variance
\newcommand{\var}{\text{Var}}
% Covariance
\newcommand{\cov}{\text{Cov}}
% Cofactor (matrix)
\newcommand{\cof}{\text{Cof}}
% Adjugate (matrix)
\newcommand{\adj}{\text{Adj}}
% Trace (matrix)
\newcommand{\tr}{\text{tr}}
% Standard deviation
\newcommand{\std}{\text{Std}}
% Correlation coefficient
\newcommand{\corr}{\text{Corr}}
% Sign
\newcommand{\sign}{\text{sign}}

% And text
\newcommand{\AND}{\text{ and }}
% Or text 
\newcommand{\OR}{\text{ or }}
% For text 
\newcommand{\FOR}{\text{ for }}
% If text
\newcommand{\IF}{\text{ if }}
% When text
\newcommand{\WHEN}{\text{ when }}
% Where text
\newcommand{\WHERE}{\text{ where }}
% Then text
\newcommand{\THEN}{\text{ then }}
% Such that text
\newcommand{\SUCHTHAT}{\text{ such that }}

% 1st
\newcommand{\st}[1]{#1^{\text{st}}}
% 2nd
\newcommand{\nd}[1]{#1^{\text{nd}}}
% 3rd
\newcommand{\rd}[1]{#1^{\text{rd}}}
% nth
\newcommand{\nth}[1]{#1^{\text{th}}}


%===============================================================================
%========================= PROBABILITY AND STATISTICS ==========================
%===============================================================================


% Permutation
\newcommand{\perm}[2]{{}^{#1}\!P_{#2}}
% Combination
\newcommand{\comb}[2]{{}^{#1}C_{#2}}

% Baye's risk
\newcommand{\risk}[1]{\mathscr{R}_{#1}}
% Baye's optimal risk
\newcommand{\riskOptimal}[1]{\mathscr{R}_{#1}^*}
% Baye's empirical risk
\newcommand{\riskEmpirical}[2]{\hat{\mathscr{R}}_{#1}^{#2}}


%===============================================================================
%=================================== CALCULUS ==================================
%===============================================================================


% d over d derivative
\newcommand{\dd}[2]{\dfrac{d#1}{d#2}}
% partial d over d derivative
\newcommand{\partialdd}[2]{\dfrac{\partial #1}{\partial #2}}
% delta d over d derivative
\newcommand{\deltadd}[2]{\dfrac{\Delta #1}{\Delta #2}}

% Integration between a and b
\newcommand{\integral}[4]{\int_{#1}^{#2} #3 \, #4}
% Integration in some space 
\newcommand{\boundIntegral}[2]{\int_{#1} #2 \, d#1}

% Limit
\newcommand{\limit}[3]{\lim_{#1 \to #2} #3}


%===============================================================================
%================================  BIG SYMBOLS  ================================
%===============================================================================


% Sum
\newcommand{\sumof}[2]{\sum_{#1}^{#2}}
% Product
\newcommand{\productof}[2]{\prod_{#1}^{#2}}
% Union
\newcommand{\unionof}[2]{\bigcup_{#1}^{#2}}
% Intersection
\newcommand{\intersectionof}[2]{\bigcap_{#1}^{#2}}
% Or
\newcommand{\orof}[2]{\bigvee_{#1}^{#2}}
% And
\newcommand{\andof}[2]{\bigwedge_{#1}^{#2}}


%===============================================================================
%=============================== LINEAR ALGEBRA ================================
%===============================================================================


% Bold vector arrow
\newcommand{\bv}[1]{\vec{\mathbf{#1}}}

% Matrix or vector (use // for column, & for row) 
% with brackets
\newcommand{\bmat}[1]{\begin{bmatrix} #1 \end{bmatrix}}
% Matrix or vector (use // for column, & for row) 
% with curved brackets
\newcommand{\pmat}[1]{\begin{pmatrix} #1 \end{pmatrix}}
% Matrix or vector (use // for column, & for row) 
% with lines on either side 
\newcommand{\lmat}[1]{\begin{vmatrix} #1 \end{vmatrix}}
% Matrix or vector (use // for column, & for row) 
% with curly braces
\newcommand{\cmat}[1]{\begin{Bmatrix} #1 \end{Bmatrix}}
% Matrix or vector (use // for column, & for row) 
% with no braces
\newcommand{\mat}[1]{\begin{matrix} #1 \end{matrix}}


%===============================================================================
%================================ LARGE OBJECTS ================================
%===============================================================================

% Multiple lines
\newcommand{\multiline}[1]{
\begin{align*}
    #1
\end{align*}
}

% Multiple lines with equation numbers
\newcommand{\eqmultiline}[1]{
\begin{align*}
    #1
\end{align*}
}

% Color
\newcommand{\colorText}[2]{
\begingroup
\color{#1}
    #2
\endgroup
}

% Centered figure
\newcommand{\centerFigure}[2]{
    \begin{figure}[h]
        \centering
            #1
        \caption{#2}
    \end{figure}
}

% Tikz figure
\newcommand{\tikzGraphic}[1]{
    \begin{center}
    \begin{tikzpicture}
        #1
    \end{tikzpicture}
    \end{center}
}

% Enumerate numbers (seperate by \item)
\newcommand{\numbers}{\textbf{\number*)}}

% Enumerate letters (seperate by \item)
\newcommand{\letters}{\textbf{\alph*)}}

\title{%
    \Huge Machine Learning \\
    \Large Lecture XIV
}
\date{2025-06-23}
\author{Michael Saba}

\begin{document}
\pagenumbering{gobble}
\maketitle
\newpage
\setlength{\parindent}{0pt}
\pagenumbering{arabic}

\section*{Unsupervised Learning}
\subsection*{Clustering}

When we have an unlabled sataset, we can use
unsupervised learning. \\
The dataset is fixed ans only has inputs $x$. \\

The goal is to disover the intrinsic structure of
the data. One way to do this is using clustering. \\
We assume there is some underlying probability space,
and our goal is to try and reverse engineer it using the
data. \\

We will use $K$-means clustering, which is actually not
very efficient, but can be imporved. \\

The idea behind clustering is that we will group
points together based on similarity measure.
This can be the distance for example; there is no
specific goal however. the objective is ill-defined. \\

We want to find $K$ prototypes $\mu_i$ that represent
similar data points in an optimal way
(for distance, this is the mean). \\
We call $C_i$ the $\nth{i}$ cluster. \\

So our goal is to optimize the sum over all
clusters, the sum over all points in each cluster,
the similarity measure between the points
and each proptotype $\mu_i$ for each cluster. \\
This similarity measure is the Euclidian norm
squared for the distance. \\

So we need a prototype, and a similarity measure;
for instance: mean and distance. \\

The sum is:
\[ \arg \min_{(C_k, \mu_k)} \sum_{i=1}^K \sum_{j=1}^K
\|x_j - \mu_i\|^2 \]
We can minimize this and get the clusters and prototype
pairs that are optimal. \\

This will lead us to find sphere like clusters; non
sphere like clusters require other similarity measures,
and are harder to fit. \\

Clustering is heavily influenced by outliers, 
since we are using means (the squared distance
is highly sensitive). \\

Lloyd's algorithms is used to solve the
sum minimization problem. \\
We first initialize the prototypes (not their true values). \\
Then we will compute the clusters such that the points
are assigend to nearest prototypes; we then use
these clusters to recompute the prototypes.
We then repeat the update step until the prototypes
stop changing (converged to solution). \\

This is the same as performing alternating minimization
on the double sum; that means fixing some variables
and minimize some free variables, then fix the optimized
variables and minimize using the others. \\

Clustering with $K$-means works, but only for simple
cluster shapes. \\

We only need to define a prototype (cluster centers)
and a similarity measure compatible with (notion of
a distance). \\

For example, we can also use medians (coordinate-wise,
that is, the median of an $N$ dimensional point
is the point of coordinate-wise medians), along with 
the 1-norm distance. This is less sensitive to
outliers. \\

Really for any metric space, with a distance function,
we can define $K$-means clustering. \\
Likewise there exists $K$-medoids. \\

It is quite a challenge to generalize the shape
of the clusters; each distance metric has a shape
that is naturally easier to fit. \\

Local consistency can however be used.
We will first look at semi-supervised learning. \\

\newpage

\subsection*{Semi-Supervised Learning}

Some of our data is labled, while a much larger part
of it is unlabled.
We have $n = n_L + n_U$. \\

There are two steps. \\
transductive learning; predicting the
lavels of the unlabled data using the labled data. \\
And the second step is SSL, where we construct the
classifier using our entire dataset. \\

For the transduction step:
first we take the labled data points, build a classifier,
and use it on the unlabled data, to give them labels;
this is called self training. \\
We apply this step only to the unlabled data
the classifier is most confident about,
making them labeled. \\
We then repeat this step untill all points are labeled
(each time we become more confident as we have
more labeled data and better classifiers). \\

The problems are that bad labels at the start ruin the
whole of the learning process. \\
We also need a way to measure confidence. \\

\newpage

\subsection*{Graph}

So how can we improve clustering shapes
in semi-supervised learning? \\

Clustering assumption; if we can find a path from
a point to another throigh a high density
region (density of points), then these points
must have the same label. \\

Another asumption is the Manifold Assumption:
Points that have the same label lie on the same manifold;
seperate classes belong on seperate manifolds. \\

We can combine both assumption: points that can be
connected by a path through a high density region
on the data manifold should have the same label. \\
We can achieve this using a regularizer that prefers
functions that vary very smoothly and do not vary in
high density regions much (the label must be the same). \\

In practice we don't have the data manifold. \\
So we generate an approximation of the manifold
and the density on it by building a graph of our
points. \\

A weighted graph is a triplet $(\vcal, \ecal, w)$,
which represent the points, edges, and the weights
on the edges. \\

The datapoints are going to be our graph points $\vscr$. \\

We will call kNN($x_i$) the $k$ least dissimilar
points to $x_i$ ($k$ nearest neighbors). \\

We connect point $x_i$ to $x_j$ if $x_j \in 
\text{kNN}(x_i)$.
Same for $x_j$ and $x_i$. \\
We can reuire one at least, or both, to create
an edge. These are called mutual and symmetric. \\

We can also use an epsilon-graph, where
we connect points if the dissimilarity measure
$d$ is smaller than some $\eps$. \\

So now that we have the points and edges, we need
the weights. What weights to use when two points
$x_j$ and $x_i$ connect? \\

One idea is Gaussian weights, with a variance
$\sigma^2$. \\
There is the single-scale case and an adaptive
scaling case. \\

Now that we have a graph, we need a learning
rule:
\[ f: \vcal \to \ycal \]
It takes in the datapoints, and outputs
the correct labels, using the whole graph. \\

The regularization we use is one that prefers smooth
functions, and penalizes variance in high density
regions. \\
The dissimilarity measure used in the
regularizer (like norm squared) is weighted by
the edge weight between the two points
$x_i$ and $x_j$ we are using the dissimilarity
measure against. \\

So our goal is to minimize the regularization
term $\rcal(f)$, such that $f(x_i) = y_i$
(if there is a label, that is,
if $(x_i, y_i) \in S_L$). \\
So we have a regularization and constraint. \\

Then we apply transductive learning to
the data, labeling unlabeled data, which in turn
gives us the clusters we want.
Recall that transduction is about labeling unlabeled
data using a learning rule. \\

There are two other strategies that use different
regularizations and constraints. \\

The regularization functions defines the graph
laplacian $\Delta$. \\
This will be useful when we move back to
fully unsupervised learning. \\

We can use that to define our regularization
sum term we tried to minimize as $\ang{f, \Delta f}$. \\

This is related to the laplacian of a function
(sum of second derivatives $\nabla^2$, the trace
of the Hessian matrix). \\

We can see in the epsilon graph that the regularizer
scales with teh gradient; meaning $f$ should not
change the label when the gradient is large. \\
the formula uses the density $p$ of the data. \\

There is a normalized graph Laplacian $\tilde{\Delta}$,
which is actually given by the third method we saw
earlier. \\

We then use the compact form of the regularization
with the graph laplacians for all three methods. \\
They all become quadratic minimizations. \\

\end{document}
