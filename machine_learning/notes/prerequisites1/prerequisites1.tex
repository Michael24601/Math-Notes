
\documentclass[12pt]{article}

\usepackage[margin=1in]{geometry}


%===============================================================================
%================================== PACKAGES ===================================
%===============================================================================

% For using float option H that places figures 
% exatcly where we want them
\usepackage{float}
% makes figure font bold
\usepackage{caption}
\captionsetup[figure]{labelfont=bf}
% For text generation
\usepackage{lipsum}
% For drawing
\usepackage{tikz}
% For smaller or equal sign and not divide sign
\usepackage{amssymb}
% For the diagonal fraction
\usepackage{xfrac}
% For enumerating exercise parts with letters instead of numbers
\usepackage{enumitem}
% For dfrac, which forces the fraction to be in display mode (large) e
% even in math mode (small)
\usepackage{amsmath}
% For degree sign
\usepackage{gensymb}
% For "\mathbb" macro
\usepackage{amsfonts}
% For positioning 
\usepackage{indentfirst}
\usetikzlibrary{shapes,positioning,fit,calc}
% for adjustwidth environment
\usepackage{changepage}
% for arrow on top
\usepackage{esvect}
% for mathbb 1
\usepackage{bbm}
% for mathsrc
\usepackage[mathscr]{eucal}
% For degree sign
\usepackage{gensymb}
% For quotes
\usepackage{csquotes}
% For vertical lines
\usepackage{mathtools}
% For cols
\usepackage{multicol}

% for tikz
\usepackage{pgfplots}
\pgfplotsset{compat=1.18}
\usepackage{amsmath}
\usepgfplotslibrary{groupplots}


%===============================================================================
%==================================== FONTS ====================================
%===============================================================================


% Mathcal
\newcommand{\acal}{\mathcal{A}}
\newcommand{\bcal}{\mathcal{B}}
\newcommand{\ccal}{\mathcal{C}}
\newcommand{\dcal}{\mathcal{D}}
\newcommand{\ecal}{\mathcal{E}}
\newcommand{\fcal}{\mathcal{F}}
\newcommand{\gcal}{\mathcal{G}}
\newcommand{\hcal}{\mathcal{H}}
\newcommand{\ical}{\mathcal{I}}
\newcommand{\jcal}{\mathcal{J}}
\newcommand{\kcal}{\mathcal{K}}
\newcommand{\lcal}{\mathcal{L}}
\newcommand{\mcal}{\mathcal{M}}
\newcommand{\ncal}{\mathcal{N}}
\newcommand{\ocal}{\mathcal{O}}
\newcommand{\pcal}{\mathcal{P}}
\newcommand{\qcal}{\mathcal{Q}}
\newcommand{\rcal}{\mathcal{R}}
\newcommand{\scal}{\mathcal{S}}
\newcommand{\tcal}{\mathcal{T}}
\newcommand{\ucal}{\mathcal{U}}
\newcommand{\vcal}{\mathcal{V}}
\newcommand{\wcal}{\mathcal{W}}
\newcommand{\xcal}{\mathcal{X}}
\newcommand{\ycal}{\mathcal{Y}}
\newcommand{\zcal}{\mathcal{Z}}

% Mathfrak
\newcommand{\afrak}{\mathfrak{A}}
\newcommand{\bfrak}{\mathfrak{B}}
\newcommand{\cfrak}{\mathfrak{C}}
\newcommand{\dfrak}{\mathfrak{D}}
\newcommand{\efrak}{\mathfrak{E}}
\newcommand{\ffrak}{\mathfrak{F}}
\newcommand{\gfrak}{\mathfrak{G}}
\newcommand{\hfrak}{\mathfrak{H}}
\newcommand{\ifrak}{\mathfrak{I}}
\newcommand{\jfrak}{\mathfrak{J}}
\newcommand{\kfrak}{\mathfrak{K}}
\newcommand{\lfrak}{\mathfrak{L}}
\newcommand{\mfrak}{\mathfrak{M}}
\newcommand{\nfrak}{\mathfrak{N}}
\newcommand{\ofrak}{\mathfrak{O}}
\newcommand{\pfrak}{\mathfrak{P}}
\newcommand{\qfrak}{\mathfrak{Q}}
\newcommand{\rfrak}{\mathfrak{R}}
\newcommand{\sfrak}{\mathfrak{S}}
\newcommand{\tfrak}{\mathfrak{T}}
\newcommand{\ufrak}{\mathfrak{U}}
\newcommand{\vfrak}{\mathfrak{V}}
\newcommand{\wfrak}{\mathfrak{W}}
\newcommand{\xfrak}{\mathfrak{X}}
\newcommand{\yfrak}{\mathfrak{Y}}
\newcommand{\zfrak}{\mathfrak{Z}}

% Mathscr
\newcommand{\ascr}{\mathscr{A}}
\newcommand{\bscr}{\mathscr{B}}
\newcommand{\cscr}{\mathscr{C}}
\newcommand{\dscr}{\mathscr{D}}
\newcommand{\escr}{\mathscr{E}}
\newcommand{\fscr}{\mathscr{F}}
\newcommand{\gscr}{\mathscr{G}}
\newcommand{\hscr}{\mathscr{H}}
\newcommand{\iscr}{\mathscr{I}}
\newcommand{\jscr}{\mathscr{J}}
\newcommand{\kscr}{\mathscr{K}}
\newcommand{\lscr}{\mathscr{L}}
\newcommand{\mscr}{\mathscr{M}}
\newcommand{\nscr}{\mathscr{N}}
\newcommand{\oscr}{\mathscr{O}}
\newcommand{\pscr}{\mathscr{P}}
\newcommand{\qscr}{\mathscr{Q}}
\newcommand{\rscr}{\mathscr{R}}
\newcommand{\sscr}{\mathscr{S}}
\newcommand{\tscr}{\mathscr{T}}
\newcommand{\uscr}{\mathscr{U}}
\newcommand{\vscr}{\mathscr{V}}
\newcommand{\wscr}{\mathscr{W}}
\newcommand{\xscr}{\mathscr{X}}
\newcommand{\yscr}{\mathscr{Y}}
\newcommand{\zscr}{\mathscr{Z}}

% Mathbb
\newcommand{\abb}{\mathbb{A}}
\newcommand{\bbb}{\mathbb{B}}
\newcommand{\cbb}{\mathbb{C}}
\newcommand{\dbb}{\mathbb{D}}
\newcommand{\ebb}{\mathbb{E}}
\newcommand{\fbb}{\mathbb{F}}
\newcommand{\gbb}{\mathbb{G}}
\newcommand{\hbb}{\mathbb{H}}
\newcommand{\ibb}{\mathbb{I}}
\newcommand{\jbb}{\mathbb{J}}
\newcommand{\kbb}{\mathbb{K}}
\newcommand{\lbb}{\mathbb{L}}
\newcommand{\mbb}{\mathbb{M}}
\newcommand{\nbb}{\mathbb{N}}
\newcommand{\obb}{\mathbb{O}}
\newcommand{\pbb}{\mathbb{P}}
\newcommand{\qbb}{\mathbb{Q}}
\newcommand{\rbb}{\mathbb{R}}
\newcommand{\sbb}{\mathbb{S}}
\newcommand{\tbb}{\mathbb{T}}
\newcommand{\ubb}{\mathbb{U}}
\newcommand{\vbb}{\mathbb{V}}
\newcommand{\wbb}{\mathbb{W}}
\newcommand{\xbb}{\mathbb{X}}
\newcommand{\ybb}{\mathbb{Y}}
\newcommand{\zbb}{\mathbb{Z}}


%===============================================================================
%=============================== SPECIAL SYMBOLS ===============================
%===============================================================================


% Orbit (group theory)
\newcommand{\orbit}{\mathcal{O}}
% Normal group
\newcommand{\normal}{\mathcal{N}}
% Indicator function
\newcommand{\indicator}{\mathbbm{1}}
% Laplace transform
\newcommand{\laplace}[1]{\mathcal{L}}
% Epsilon shorthand
\newcommand{\eps}{\varepsilon}
% Omega
\newcommand{\om}{\omega}
\newcommand{\Om}{\Omega}

%===============================================================================
%================================== OPERATORS ==================================
%===============================================================================


% Inverse exponent
\newcommand{\inv}[0]{^{-1}}
% Overline bar
\newcommand{\olsi}[1]{\,\overline{\!{#1}}}
% Less than or equal slanted
\newcommand{\seqs}{\leqslant}
% Greater or equal slanted
\newcommand{\geqs}{\geqslant}
% Subset or equal
\newcommand{\sub}{\subseteq}
% Proper subset
\newcommand{\prosub}{\subset}
% from
\newcommand{\from}{\leftarrow}

% Parantheses
\newcommand{\para}[1]{\left( #1 \right)}
% Curly Braces
\newcommand{\curl}[1]{\left\{ #1 \right\}}
% Brackets
\newcommand{\brac}[1]{\left[ #1 \right]}
% Angled Brackets
\newcommand{\ang}[1]{\left\langle #1 \right\rangle}
% Norm
\newcommand{\norm}[1]{\left\| #1 \right\|}

% Piece wise (use \\ between cases)
\newcommand{\piecewise}[1]{\begin{cases} #1 \end{cases}}

% Bold symbol shorthand
\newcommand{\bl}{\boldsymbol}

% Vertical space
\newcommand{\vs}[1]{\vspace{#1 pt}}
% Horizontal ertical space
\newcommand{\hs}[1]{\hspace{#1 pt}}


%===============================================================================
%============================== TEXT BASED SYMBOLS =============================
%===============================================================================


% Radians
\newcommand{\rad}{\text{rad}}
% Least Common Multiple
\newcommand{\lcm}{\text{lcm}}
% Automorphism
\newcommand{\Aut}{\text{Aut}}
% Variance
\newcommand{\var}{\text{Var}}
% Covariance
\newcommand{\cov}{\text{Cov}}
% Cofactor (matrix)
\newcommand{\cof}{\text{Cof}}
% Adjugate (matrix)
\newcommand{\adj}{\text{Adj}}
% Trace (matrix)
\newcommand{\tr}{\text{tr}}
% Standard deviation
\newcommand{\std}{\text{Std}}
% Correlation coefficient
\newcommand{\corr}{\text{Corr}}
% Sign
\newcommand{\sign}{\text{sign}}

% And text
\newcommand{\AND}{\text{ and }}
% Or text 
\newcommand{\OR}{\text{ or }}
% For text 
\newcommand{\FOR}{\text{ for }}
% If text
\newcommand{\IF}{\text{ if }}
% When text
\newcommand{\WHEN}{\text{ when }}
% Where text
\newcommand{\WHERE}{\text{ where }}
% Then text
\newcommand{\THEN}{\text{ then }}
% Such that text
\newcommand{\SUCHTHAT}{\text{ such that }}

% 1st
\newcommand{\st}[1]{#1^{\text{st}}}
% 2nd
\newcommand{\nd}[1]{#1^{\text{nd}}}
% 3rd
\newcommand{\rd}[1]{#1^{\text{rd}}}
% nth
\newcommand{\nth}[1]{#1^{\text{th}}}


%===============================================================================
%========================= PROBABILITY AND STATISTICS ==========================
%===============================================================================


% Permutation
\newcommand{\perm}[2]{{}^{#1}\!P_{#2}}
% Combination
\newcommand{\comb}[2]{{}^{#1}C_{#2}}

% Baye's risk
\newcommand{\risk}[1]{\mathscr{R}_{#1}}
% Baye's optimal risk
\newcommand{\riskOptimal}[1]{\mathscr{R}_{#1}^*}
% Baye's empirical risk
\newcommand{\riskEmpirical}[2]{\hat{\mathscr{R}}_{#1}^{#2}}


%===============================================================================
%=================================== CALCULUS ==================================
%===============================================================================


% d over d derivative
\newcommand{\dd}[2]{\dfrac{d#1}{d#2}}
% partial d over d derivative
\newcommand{\partialdd}[2]{\dfrac{\partial #1}{\partial #2}}
% delta d over d derivative
\newcommand{\deltadd}[2]{\dfrac{\Delta #1}{\Delta #2}}

% Integration between a and b
\newcommand{\integral}[4]{\int_{#1}^{#2} #3 \, #4}
% Integration in some space 
\newcommand{\boundIntegral}[2]{\int_{#1} #2 \, d#1}

% Limit
\newcommand{\limit}[3]{\lim_{#1 \to #2} #3}


%===============================================================================
%================================  BIG SYMBOLS  ================================
%===============================================================================


% Sum
\newcommand{\sumof}[2]{\sum_{#1}^{#2}}
% Product
\newcommand{\productof}[2]{\prod_{#1}^{#2}}
% Union
\newcommand{\unionof}[2]{\bigcup_{#1}^{#2}}
% Intersection
\newcommand{\intersectionof}[2]{\bigcap_{#1}^{#2}}
% Or
\newcommand{\orof}[2]{\bigvee_{#1}^{#2}}
% And
\newcommand{\andof}[2]{\bigwedge_{#1}^{#2}}


%===============================================================================
%=============================== LINEAR ALGEBRA ================================
%===============================================================================


% Bold vector arrow
\newcommand{\bv}[1]{\vec{\mathbf{#1}}}

% Matrix or vector (use // for column, & for row) 
% with brackets
\newcommand{\bmat}[1]{\begin{bmatrix} #1 \end{bmatrix}}
% Matrix or vector (use // for column, & for row) 
% with curved brackets
\newcommand{\pmat}[1]{\begin{pmatrix} #1 \end{pmatrix}}
% Matrix or vector (use // for column, & for row) 
% with lines on either side 
\newcommand{\lmat}[1]{\begin{vmatrix} #1 \end{vmatrix}}
% Matrix or vector (use // for column, & for row) 
% with curly braces
\newcommand{\cmat}[1]{\begin{Bmatrix} #1 \end{Bmatrix}}
% Matrix or vector (use // for column, & for row) 
% with no braces
\newcommand{\mat}[1]{\begin{matrix} #1 \end{matrix}}


%===============================================================================
%================================ LARGE OBJECTS ================================
%===============================================================================

% Multiple lines
\newcommand{\multiline}[1]{
\begin{align*}
    #1
\end{align*}
}

% Multiple lines with equation numbers
\newcommand{\eqmultiline}[1]{
\begin{align*}
    #1
\end{align*}
}

% Color
\newcommand{\colorText}[2]{
\begingroup
\color{#1}
    #2
\endgroup
}

% Centered figure
\newcommand{\centerFigure}[2]{
    \begin{figure}[h]
        \centering
            #1
        \caption{#2}
    \end{figure}
}

% Tikz figure
\newcommand{\tikzGraphic}[1]{
    \begin{center}
    \begin{tikzpicture}
        #1
    \end{tikzpicture}
    \end{center}
}

% Enumerate numbers (seperate by \item)
\newcommand{\numbers}{\textbf{\number*)}}

% Enumerate letters (seperate by \item)
\newcommand{\letters}{\textbf{\alph*)}}

\title{%
    \Huge Machine Learning \\
    \Large Math Prerequisites I
}
\date{2025-04-14}
\author{Michael Saba}

\begin{document}
\pagenumbering{gobble}
\maketitle
\newpage
\setlength{\parindent}{0pt}
\pagenumbering{arabic}

    \section*{Linear Algebra}

    \subsection*{Definitions}

    A vector space $V$
    over field $\F$ is the set of all vectors
    $v$ with elements in $\F$. \\
    It must satisfy certain axioms
    (beyond the scope of this  lecture). \\
    
    The elements $f \in \F$ are called scalars
    (motivation will be given later). \\
    
    A vector space is equipped with two operations:
    \begin{enumerate}
        \item 
        Vector addition:
        \[ +: V \times V \to V \qquad \text{ where }
        w = u + v \iff
        w_i = u_i + v_i \]
    \item 
        Scalar multiplication:
        \[ \cdot: \F \times V \to V \qquad \text{ where }
        w = \alpha \cdot u \iff
        w_i = \alpha u_i\]
    \end{enumerate}

    The inner product is a function:
    \[ \langle \cdot , \cdot \rangle : V \times V \to \F \]
    It must satisfy certain axioms
    (beyond the scope of this  lecture). \\

    The norm is a function:
    \[ \|\|  : V \to \F \]
    It must also satisfy certain axioms
    (beyond the scope of this  lecture). \\

    \newpage

    \subsection*{Euclidian Vector Spaces}

    What interests us for this course are
    Euclidian Vector Spaces.
    These are vector spaces defines over the real
    numbers $\R$, such as $\R^2$ (set of all 2D vectors). \\

    An example of a vector in $\R^3$:
    \[ \vecThree{1}{0}{2} \] \\

    The Euclidian inner product, called the dot product,
    is defined as:
    \[ x \cdot y = \sum_{i = 1}^nx_iy_i \]
    for vectors $x, y \in \R^n$. \\
    It is the length of one of the vectors when
    projected onto the other
    (the measure of how much one vector is in
    the direction of the other). \\
    If $x \cdot y = 0$,
    we can conclude that $x$ and $y$
    are othogonal (perpendicular). \\

    The Euclidian norm, also called the magnitude,
    or 2-norm,
    is defined as:
    \[ \|x\| = \sqrt{\sum_{i = 1}^nx_i^2} 
    = \sqrt{x \cdot x} \]
    for a vector $x \in \R^n$. \\
    It is the length of the vector. \\

    We can define a whole family of norms,
    where the p-norm is defined as
    \[ \| x \|_p = 
    \left( \sum_{i = 1}^n|x_i|^p  \right)^{\sfrac{1}{p}} \]
    for a vector $x \in R^n$. \\

    These include the 1-norm:
    \[ \| x \|_1 = \sum_{i = 1}^n|x_i| \]
    the sum of the absolute value of the entries in $x$. \\

    As well as the infinity norm:
    \[ \| x \|_\infty = \max_{1 \leq i \leq n} |x_i| \]
    the absolute value 
    of the entry with the largest absolute value. \\

    We can also define matrices over the real numbers.
    Such as this matrix in $\R^{2 \times 3}$:
    \[ \begin{bmatrix}
        1 & 0 & 3 \\
        2 & 1 & 0
    \end{bmatrix} \] \\

    The trace product
    (also called the Frobenius inner product)
    for two matrices
    $A, B \in \R^{m \times n}$ is defined as:
    \[ \langle A, B \rangle_F = 
    \sum_{i=1}^m\sum_{j=1}^n A_{ij}B_{ij} \]
    which is just the sum of the element-wise
    product of entries in $A$ and $B$. \\

    This also gives us the Frobenius norm:
    \[ \| A \|_F = = 
    \sqrt{\sum_{i=1}^m\sum_{j=1}^n A_{ij}^2}
    = \sqrt{\langle A, A \rangle_F } \]
    for a matrix $A \in \R^{m \times n}$. \\

    We can also define operator norms on matrices:
    \[ \|A\|_p = \max_{\|x\|_p = 1}\|Ax\|_p \]
    
    The 2-norm, or spectral norm,
    is defined as:
    \[ \|A\|_2 = \max_{\|x\|_2 = 1}\|Ax\|_2
    = \sigma_{\max}(A) \]
    the largest singular value of A
    (we're trying to maximize the amount by which
    $A$ scales $x$, a unit vector). \\

    The 1-norm is define as:
    \[ \|A\|_1 = \max_{1 \leq j \leq n} \sum_{i = 1}^m |a_{ij}| \]
    which is essentially the column
    with the largest absolute sum.
     
    The infinity norm is defined as:
    \[ \|A\|_\infty = \max_{1 \leq i \leq m} 
    \sum_{j = 1}^n |a_{ij}| \]
    which is essentially the row
    with the largest absolute sum. \\

    The Cauchy-Schwarz inequality states that 
    for any vectors $x, y \in \R^n$
    \[|x \cdot y| \leq \|x\|\|y\|\]

    \newpage

    \subsection*{Linear Transformations}

    Given a vector space $V$ over $\R$,
    a subset $B$ of $V$
    \[ B = \{v_1, v_2, \dots v_n \} \]
    is called a basis of $V$ as long as the
    vectors $v_1, v_2, \dots v_n$ are linearly independent,
    and if every vector in $V$ could be written
    as a linear combination of the vectors in $B$. \\

    We can think of the basis as the coordinate
    system of the vector space.
    For instance, when we have a vector
    \[ \vecThree{1}{5}{6} \]
    The entries refer to the scalar factors
    that have to be applied to the basis vectors
    such that their linear combinaton produces
    the vector. \\
    The basis vectors are essentially the reference
    point using which we define all other vectors. \\ 

    The first rule implies that if
    \[ a_1v_1 + a_2v_2 + \dots + a_nv_n = 0 \]
    then 
    \[a_1 = a_2 = \dots = a_n = 0\]
    This ensures no two vectors are colinear,
    which is important as colinear vectors
    cause redundant information,
    and allow us to get to the same point
    using different linear combinations
    (so two vectors with different entries could
    be equal). \\

    The second rule states that any vector $v \in V$,
    for some setting $a_1, a_2, \dots a_n$:
    \[ v = a_1v_1 + a_2v_2 + \dots + a_nv_n \]
    This ensures that any vector in $V$
    can be written as the linear combination
    of the basis vectors. \\

    For example, for $\R^2$,
    we have the standard basis:
    \[ \vecTwo{1}{0} \qquad
    \text{ and } \qquad \vecTwo{0}{1} \]
    This satisfies both properties. \\
    For instance, for any two scalars $a$ and $b$,
    \[ a\vecTwo{1}{0} + b\vecTwo{0}{1} = \vecTwo{0}{0} \]
    implies that 
    \[ \vecTwo{a}{b} = \vecTwo{0}{0}  \]
    which means that $a = b = 0$. \\
    Moroever, we can write any vector
    \[ \vecTwo{x}{y} = x\vecTwo{1}{0} + y\vecTwo{0}{1} \]

    We can now define the notion of a linear
    mapping.
    A linear mapping is a function
    \[ f: E \to F \]
    from vector space $E$ to vector space $F$
    if for any $x, y \in E$ and $a \in \R$:
    \[ f(x+y) = f(x) + f(y) \]
    \[ f(ax) = af(x) \]
    which are called additivity and homogeneity. \\
    Intuitively, a linear mapping or transformation
    is a function that transforms the vectors
    in a vector space $V$ while keeping
    every straight line between two vectors $u$ and $v$
    straight, and keeping the origin $(0, 0)$
    in place. \\
    I
    Examples of linear mappings include rotations,
    which map each vector to the point it would 
    be pointing at had the space been rotated. \\

    If
    \[ u_1, u_2, \dots u_m \]
    and
    \[ v_1, v_2, \dots v_n \]
    are the basis vectors of $E$ and $F$
    respectively,
    then we can define linear mappings using 
    a matrix $A$ of dimension $m \times n$,
    such that 
    \[ A (u_j) = \sum_{i=1}^{n} a_{ij}v_i \qquad
    1 \leq j \leq m \]
    which gives us how the transformed basis
    vectors can be described using
    the basis vectors of the new vector space $V$. \\
    This is useful, because, as we know,
    the vectors in $E$,
    which were expressed as linear combinations
    of its basis,
    once transformed by $f$,
    can be expressed as the same linear combination
    of the new (transformed) basis. \\

    The identity matrix $I$:
    \[ \matThree{1}{0}{0}{0}{1}{0}{0}{0}{1} \]
    is a linear transformation that keeps
    every vector in place
    (can be defined for any vector space $\R^n$). \\

    The inverse $A^{-1}$ of a matrix $A$
    does the opposite transformation. \\
    For example, the inverse of
    a matrix that rotates the space by $90^o$
    is a matrix that rotates the space by $-90^o$. \\

    The determinant of matrix is a function
    \[ \det: \R^{n \times n} \to \R \]
    that describes the scaling factor of the
    linear mapping the matrix represents
    (e.g., it is the area of a 1 by 1 square
    in the original vector space once
    transformed by the matrix). \\

    The determinant is only defined for square matrices. \\
    
    If $\det(A) = 0$,
    then the matrix is called singular
    (has no inverse). \\

    The determinant is multiplicative,
    meaning that
    \[ \det(AB) = \det(A)\det(B) \]
    This makes sense;
    the linear transformation $AB$
    cn be thought of as applying $A$
    first, then $B$.
    So we've essentially scaled the area/volume
    by $\det(A)$ first, then by $\det(B)$,
    which is the same as scaling it by $\det(A)\det(B)$. \\

    The determinant of a $2 \times 2$ matrix
    can be computed as follows:
    \[ \det\left( \matTwo{a}{b}{c}{d} \right) 
    = ad - bc \]
    As for $3 \times 3$ matrices:
     \[ \det\left( \matThree{a}{b}{c}{d}{e}{f}{g}{h}{i} \right) 
    = aei + bfg + cdh - gec - hfa - idb \]
    It can be calculated by summing
    the products main diagonals starting at each entry in
    the first row,
    and them subtracting the products
    of the secondary diagonals starting at each
    entry in the bottom row. \\

    \newpage

    \subsection*{Eigen Vectors}

    Given a matrix $A$, linear combination
    \[f: E \to E \]
    we call a vector $v \in E$ an eigen vector
    if the transformation keeps the vector
    on the same span
    (scales it but does not change its direction). \\
    In particular, if
    \[ f(v) = \lambda v \]
    or 
    \[ Av = \lambda v \]
    then we call $v$ an eigen vector,
    and $\lambda$ an eigen value. \\
    
    Now, to find the eigen vectors,
    we need to find the vectors $v$ such that 
    \[ Av = \lambda v \]
    So
    \[ Av - \lambda v = 0 \]
    \[ Av - \lambda I v = 0 \]
    \[ (A - \lambda I)v = 0 \]
    where $I$ is the identity matrix. \\

    The solution to a linear system $Ax = 0$
    is always trivial (the 0 vector)
    if the matrix $A$ is invertible (non-singular). \\
    So, to find non-trivial solutions,
    we want to find the values of $\lambda$
    such that 
    \[ \det(A - \lambda I) = 0\]
    This will lead to some polynomial that we
    can solve (in $\R^2$ it's a quadratic). \\
    
    Once we have the eigen values,
    we can plug them into 
    the original formula $(A - \lambda I)v = 0$
    each in turn, 
    which will give us our eigen vectors. \\
    
    Eigen vectors are very useful because
    they allow us to very quickly and efficiently
    compute the power of a matrix. \\
    Eigen vectors are only defined for square
    matrices however,
    and many of the matrices that we need to use them
    for are rectangular. \\
    We can use SVD to extend them to non-square matrices. \\

    Any matrix $A \in \R^{m \times n}$
    can be expressed as:
    \[ A = U\Sigma V^T \]
    where $U$ is an orthogonal $m \times m$ matrix,
    $\Sigma$ is an $m \times n$ diagonal matrix,
    and $V$ is an orthogonal $n \times n$ matrix. \\

    If a matrix $M$ is orthogonal, then
    \[ M^{-1} = M^T \]
    its inverse is just its transpose. \\
    So $M^TM = I$. \\

    The matrix $A^TA$ is square, 
    and can be written as:
    \[A^TA =  (U\Sigma V^T)^T(U\Sigma V^T)\]
    \[A^TA =  ((V^T)^T\Sigma^TU^T)(U\Sigma V^T)\]
    \[A^TA =  V\Sigma^TU^TU\Sigma V^T \]
    \[A^TA =  V\Sigma^T\Sigma V^T\]
    Now, the columns of $V$
    will be the eigen vectors of $A^TA$,
    and the diagonal entries of $\Sigma$
    will be the square root of the eigen values. \\
 
    \newpage

    \section*{Calculus}

    \subsection*{Limits and Continuity}

    For vector spaces $E$ and $F$,
    we can define a function
    \[ f: X \to F \]
    where $X \subseteq E$
    (basically vector input and output). \\

    We say $f$ is continuous at a point $x_0$
    if
    \[ \limit{x}{x_0}{f(x)} = f(x_0) \]
    We say $f$ is continuous on $X$
    if $f(x)$ is continuous on all $x_0 \in X$. \\

    We say $f$ is Lipschitz continuous on $X$
    with Lipschitz constant $L \geq 0$ if: 
    \[ \| f(x) - f(y) \| \leq L \| x - y \| \]
    for any $x, y \in X$. \\

    Now, if $X$ is an open interval on $R$,
    we say that the function
    \[ f: X \to \R \]
    is differentiable at $x_0 \in X$ if the limit
    \[ f'(x_0) = \limit{\epsilon}{0}{
        \dfrac{f(x_0+\epsilon) - f(x_0)}{\epsilon}
    }\]
    exists. \\
    If so, we call $f'(x_0)$
    the derivative of $f$ at $x_0$. \\
    We say $f$ is differentiable on $X$
    if it is differentiable on all $x_0 \in X$. \\

    If
    \[f': X \to \R \]
    is itself differentiable on $X$,
    then we say $f$ is twice differentiable,
    and call $f"(x_0)$ the second derivative
    of $f$ at $x_0$. \\

    We can also write
    \[ \dd{f}{x} \qquad \text{ and } \qquad \dfrac{d^2f}{dx^2} \]
    for the first and second derivatives. \\

    \newpage

    \subsection*{Differentiation}

    Instead of using the limit definition each time,
    we can use these rules to find the derivative
    of a function: \\
    Power rule: \[ (x^n)' = nx^{n-1} \]
    Constant rule: \[ (c)' = 0  \]
    Sum rule: \[ (f(x) + g(x))' = f'(x) + g'(x) \]
    Product rule: \[ (f(x)g(x))' = f'(x)g(x) + f(x)g'(x) \]
    Quotient rule: \[ \left(\dfrac{f(x)}{g(x)}\right)' =
    \dfrac{f'(x)g(x) - f(x)g'(x)}{g(x)^2} \]
    Chain rule: \[ (f(g(x)))' = f'(g(x))g'(x) \]

    If we have a function
    \[ f: E \to \R \]
    where $E$ is some vector space,
    then teh directional derivative of $f$
    at $x_0$ in the direction $d \in E$
    exists if the limit
    \[ f'(x_0; d) = \limit{\epsilon}{0}{
        \dfrac{f(x_0+\epsilon d) - f(x_0)}{\epsilon}
    }\]
    exists. \\
    If so, we say $f$ is directionally differentiable
    at $x_0$ in the direction of $d$. \\
    We say $f$ id directionally differentiable
    in the direction of $d$
    if $f'(x_0)$ exists for all $x_0 \in E$. \\

    The most well known and used directional
    derivatives are the partial derivatives of a
    multivariable function. \\
    Suppose that we have a function:
    \[ f: E \to \R \]
    then the directional derivative in the direction 
    of $e_i$,
    the unit vector in the direction of the $i$th
    coordinate axis,
    is known as the partial derivative:
    \[ \partialdd{f}{x_i}(x)\]
    For example, for a function:
    \[ z = f(x, y) \]
    the partial derivatives are
    \[ \partialdd{f}{x}(x, y) \qquad \text{ and } 
    \qquad \partialdd{f}{y}(x, y) \]

    The gradient of a function
    \[ f: E \to \R \]
    is a vector field
    that defines the direction in which
    $f$ varies the most.
    For any point $x \in E$:
    \[ \nabla f(x) =
    \left( \partialdd{f}{x_1}(x)
    \partialdd{f}{x_2}(x) 
    \dots \partialdd{f}{x_n}(x) \right) \]

    We can use the gradient to calculate any
    directional derivative (using the inner product):
    \[ f'(x, d) = \langle \nabla f(x), d \rangle \]

    The Jacobian matris of a vector valued function
    \[ f: \R^n \to R^m \]
    is defined as:
    \[   J_f =
        \matThree{\partialdd{f_1}{x_1}}
        {\dots}
        {\partialdd{f_1}{x_n}}
        {\vdots}
        {\ddots}
        {\vdots}
        {\partialdd{f_m}{x_1}}
        {\dots}
        {\partialdd{f_m}{x_n}}
    \] \\

    We can use the Jacobian to calculate
    the multivariable chain rule.
    Suppose that
    \[ g: \R^n \to \R^m \]
    Then
    \[ \nabla(f \circ g)(x) = \nabla f(g(x))J_g(x) \]
    where $x \in \R^n$ is a vector. \\ 
    For example, for 
    \[ f(u(x, y), v(x, y)) \]
    We have
    \[ \partialdd{f}{x} = \partialdd{f}{u}\partialdd{u}{x}
    + \partialdd{f}{v}\partialdd{v}{x} 
    \qquad \text{ and } \qquad   
    \partialdd{f}{y} = \partialdd{f}{u}\partialdd{u}{y}
    + \partialdd{f}{v}\partialdd{v}{y} 
    \]

\end{document}
