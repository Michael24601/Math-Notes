
\documentclass[12pt]{article}

\usepackage[margin=1in]{geometry}


%===============================================================================
%================================== PACKAGES ===================================
%===============================================================================

% For using float option H that places figures 
% exatcly where we want them
\usepackage{float}
% makes figure font bold
\usepackage{caption}
\captionsetup[figure]{labelfont=bf}
% For text generation
\usepackage{lipsum}
% For drawing
\usepackage{tikz}
% For smaller or equal sign and not divide sign
\usepackage{amssymb}
% For the diagonal fraction
\usepackage{xfrac}
% For enumerating exercise parts with letters instead of numbers
\usepackage{enumitem}
% For dfrac, which forces the fraction to be in display mode (large) e
% even in math mode (small)
\usepackage{amsmath}
% For degree sign
\usepackage{gensymb}
% For "\mathbb" macro
\usepackage{amsfonts}
% For positioning 
\usepackage{indentfirst}
\usetikzlibrary{shapes,positioning,fit,calc}
% for adjustwidth environment
\usepackage{changepage}
% for arrow on top
\usepackage{esvect}
% for mathbb 1
\usepackage{bbm}
% for mathsrc
\usepackage[mathscr]{eucal}
% For degree sign
\usepackage{gensymb}
% For quotes
\usepackage{csquotes}
% For vertical lines
\usepackage{mathtools}
% For cols
\usepackage{multicol}

% for tikz
\usepackage{pgfplots}
\pgfplotsset{compat=1.18}
\usepackage{amsmath}
\usepgfplotslibrary{groupplots}


%===============================================================================
%==================================== FONTS ====================================
%===============================================================================


% Mathcal
\newcommand{\acal}{\mathcal{A}}
\newcommand{\bcal}{\mathcal{B}}
\newcommand{\ccal}{\mathcal{C}}
\newcommand{\dcal}{\mathcal{D}}
\newcommand{\ecal}{\mathcal{E}}
\newcommand{\fcal}{\mathcal{F}}
\newcommand{\gcal}{\mathcal{G}}
\newcommand{\hcal}{\mathcal{H}}
\newcommand{\ical}{\mathcal{I}}
\newcommand{\jcal}{\mathcal{J}}
\newcommand{\kcal}{\mathcal{K}}
\newcommand{\lcal}{\mathcal{L}}
\newcommand{\mcal}{\mathcal{M}}
\newcommand{\ncal}{\mathcal{N}}
\newcommand{\ocal}{\mathcal{O}}
\newcommand{\pcal}{\mathcal{P}}
\newcommand{\qcal}{\mathcal{Q}}
\newcommand{\rcal}{\mathcal{R}}
\newcommand{\scal}{\mathcal{S}}
\newcommand{\tcal}{\mathcal{T}}
\newcommand{\ucal}{\mathcal{U}}
\newcommand{\vcal}{\mathcal{V}}
\newcommand{\wcal}{\mathcal{W}}
\newcommand{\xcal}{\mathcal{X}}
\newcommand{\ycal}{\mathcal{Y}}
\newcommand{\zcal}{\mathcal{Z}}

% Mathfrak
\newcommand{\afrak}{\mathfrak{A}}
\newcommand{\bfrak}{\mathfrak{B}}
\newcommand{\cfrak}{\mathfrak{C}}
\newcommand{\dfrak}{\mathfrak{D}}
\newcommand{\efrak}{\mathfrak{E}}
\newcommand{\ffrak}{\mathfrak{F}}
\newcommand{\gfrak}{\mathfrak{G}}
\newcommand{\hfrak}{\mathfrak{H}}
\newcommand{\ifrak}{\mathfrak{I}}
\newcommand{\jfrak}{\mathfrak{J}}
\newcommand{\kfrak}{\mathfrak{K}}
\newcommand{\lfrak}{\mathfrak{L}}
\newcommand{\mfrak}{\mathfrak{M}}
\newcommand{\nfrak}{\mathfrak{N}}
\newcommand{\ofrak}{\mathfrak{O}}
\newcommand{\pfrak}{\mathfrak{P}}
\newcommand{\qfrak}{\mathfrak{Q}}
\newcommand{\rfrak}{\mathfrak{R}}
\newcommand{\sfrak}{\mathfrak{S}}
\newcommand{\tfrak}{\mathfrak{T}}
\newcommand{\ufrak}{\mathfrak{U}}
\newcommand{\vfrak}{\mathfrak{V}}
\newcommand{\wfrak}{\mathfrak{W}}
\newcommand{\xfrak}{\mathfrak{X}}
\newcommand{\yfrak}{\mathfrak{Y}}
\newcommand{\zfrak}{\mathfrak{Z}}

% Mathscr
\newcommand{\ascr}{\mathscr{A}}
\newcommand{\bscr}{\mathscr{B}}
\newcommand{\cscr}{\mathscr{C}}
\newcommand{\dscr}{\mathscr{D}}
\newcommand{\escr}{\mathscr{E}}
\newcommand{\fscr}{\mathscr{F}}
\newcommand{\gscr}{\mathscr{G}}
\newcommand{\hscr}{\mathscr{H}}
\newcommand{\iscr}{\mathscr{I}}
\newcommand{\jscr}{\mathscr{J}}
\newcommand{\kscr}{\mathscr{K}}
\newcommand{\lscr}{\mathscr{L}}
\newcommand{\mscr}{\mathscr{M}}
\newcommand{\nscr}{\mathscr{N}}
\newcommand{\oscr}{\mathscr{O}}
\newcommand{\pscr}{\mathscr{P}}
\newcommand{\qscr}{\mathscr{Q}}
\newcommand{\rscr}{\mathscr{R}}
\newcommand{\sscr}{\mathscr{S}}
\newcommand{\tscr}{\mathscr{T}}
\newcommand{\uscr}{\mathscr{U}}
\newcommand{\vscr}{\mathscr{V}}
\newcommand{\wscr}{\mathscr{W}}
\newcommand{\xscr}{\mathscr{X}}
\newcommand{\yscr}{\mathscr{Y}}
\newcommand{\zscr}{\mathscr{Z}}

% Mathbb
\newcommand{\abb}{\mathbb{A}}
\newcommand{\bbb}{\mathbb{B}}
\newcommand{\cbb}{\mathbb{C}}
\newcommand{\dbb}{\mathbb{D}}
\newcommand{\ebb}{\mathbb{E}}
\newcommand{\fbb}{\mathbb{F}}
\newcommand{\gbb}{\mathbb{G}}
\newcommand{\hbb}{\mathbb{H}}
\newcommand{\ibb}{\mathbb{I}}
\newcommand{\jbb}{\mathbb{J}}
\newcommand{\kbb}{\mathbb{K}}
\newcommand{\lbb}{\mathbb{L}}
\newcommand{\mbb}{\mathbb{M}}
\newcommand{\nbb}{\mathbb{N}}
\newcommand{\obb}{\mathbb{O}}
\newcommand{\pbb}{\mathbb{P}}
\newcommand{\qbb}{\mathbb{Q}}
\newcommand{\rbb}{\mathbb{R}}
\newcommand{\sbb}{\mathbb{S}}
\newcommand{\tbb}{\mathbb{T}}
\newcommand{\ubb}{\mathbb{U}}
\newcommand{\vbb}{\mathbb{V}}
\newcommand{\wbb}{\mathbb{W}}
\newcommand{\xbb}{\mathbb{X}}
\newcommand{\ybb}{\mathbb{Y}}
\newcommand{\zbb}{\mathbb{Z}}


%===============================================================================
%=============================== SPECIAL SYMBOLS ===============================
%===============================================================================


% Orbit (group theory)
\newcommand{\orbit}{\mathcal{O}}
% Normal group
\newcommand{\normal}{\mathcal{N}}
% Indicator function
\newcommand{\indicator}{\mathbbm{1}}
% Laplace transform
\newcommand{\laplace}[1]{\mathcal{L}}
% Epsilon shorthand
\newcommand{\eps}{\varepsilon}
% Omega
\newcommand{\om}{\omega}
\newcommand{\Om}{\Omega}

%===============================================================================
%================================== OPERATORS ==================================
%===============================================================================


% Inverse exponent
\newcommand{\inv}[0]{^{-1}}
% Overline bar
\newcommand{\olsi}[1]{\,\overline{\!{#1}}}
% Less than or equal slanted
\newcommand{\seqs}{\leqslant}
% Greater or equal slanted
\newcommand{\geqs}{\geqslant}
% Subset or equal
\newcommand{\sub}{\subseteq}
% Proper subset
\newcommand{\prosub}{\subset}
% from
\newcommand{\from}{\leftarrow}

% Parantheses
\newcommand{\para}[1]{\left( #1 \right)}
% Curly Braces
\newcommand{\curl}[1]{\left\{ #1 \right\}}
% Brackets
\newcommand{\brac}[1]{\left[ #1 \right]}
% Angled Brackets
\newcommand{\ang}[1]{\left\langle #1 \right\rangle}
% Norm
\newcommand{\norm}[1]{\left\| #1 \right\|}

% Piece wise (use \\ between cases)
\newcommand{\piecewise}[1]{\begin{cases} #1 \end{cases}}

% Bold symbol shorthand
\newcommand{\bl}{\boldsymbol}

% Vertical space
\newcommand{\vs}[1]{\vspace{#1 pt}}
% Horizontal ertical space
\newcommand{\hs}[1]{\hspace{#1 pt}}


%===============================================================================
%============================== TEXT BASED SYMBOLS =============================
%===============================================================================


% Radians
\newcommand{\rad}{\text{rad}}
% Least Common Multiple
\newcommand{\lcm}{\text{lcm}}
% Automorphism
\newcommand{\Aut}{\text{Aut}}
% Variance
\newcommand{\var}{\text{Var}}
% Covariance
\newcommand{\cov}{\text{Cov}}
% Cofactor (matrix)
\newcommand{\cof}{\text{Cof}}
% Adjugate (matrix)
\newcommand{\adj}{\text{Adj}}
% Trace (matrix)
\newcommand{\tr}{\text{tr}}
% Standard deviation
\newcommand{\std}{\text{Std}}
% Correlation coefficient
\newcommand{\corr}{\text{Corr}}
% Sign
\newcommand{\sign}{\text{sign}}

% And text
\newcommand{\AND}{\text{ and }}
% Or text 
\newcommand{\OR}{\text{ or }}
% For text 
\newcommand{\FOR}{\text{ for }}
% If text
\newcommand{\IF}{\text{ if }}
% When text
\newcommand{\WHEN}{\text{ when }}
% Where text
\newcommand{\WHERE}{\text{ where }}
% Then text
\newcommand{\THEN}{\text{ then }}
% Such that text
\newcommand{\SUCHTHAT}{\text{ such that }}

% 1st
\newcommand{\st}[1]{#1^{\text{st}}}
% 2nd
\newcommand{\nd}[1]{#1^{\text{nd}}}
% 3rd
\newcommand{\rd}[1]{#1^{\text{rd}}}
% nth
\newcommand{\nth}[1]{#1^{\text{th}}}


%===============================================================================
%========================= PROBABILITY AND STATISTICS ==========================
%===============================================================================


% Permutation
\newcommand{\perm}[2]{{}^{#1}\!P_{#2}}
% Combination
\newcommand{\comb}[2]{{}^{#1}C_{#2}}

% Baye's risk
\newcommand{\risk}[1]{\mathscr{R}_{#1}}
% Baye's optimal risk
\newcommand{\riskOptimal}[1]{\mathscr{R}_{#1}^*}
% Baye's empirical risk
\newcommand{\riskEmpirical}[2]{\hat{\mathscr{R}}_{#1}^{#2}}


%===============================================================================
%=================================== CALCULUS ==================================
%===============================================================================


% d over d derivative
\newcommand{\dd}[2]{\dfrac{d#1}{d#2}}
% partial d over d derivative
\newcommand{\partialdd}[2]{\dfrac{\partial #1}{\partial #2}}
% delta d over d derivative
\newcommand{\deltadd}[2]{\dfrac{\Delta #1}{\Delta #2}}

% Integration between a and b
\newcommand{\integral}[4]{\int_{#1}^{#2} #3 \, #4}
% Integration in some space 
\newcommand{\boundIntegral}[2]{\int_{#1} #2 \, d#1}

% Limit
\newcommand{\limit}[3]{\lim_{#1 \to #2} #3}


%===============================================================================
%================================  BIG SYMBOLS  ================================
%===============================================================================


% Sum
\newcommand{\sumof}[2]{\sum_{#1}^{#2}}
% Product
\newcommand{\productof}[2]{\prod_{#1}^{#2}}
% Union
\newcommand{\unionof}[2]{\bigcup_{#1}^{#2}}
% Intersection
\newcommand{\intersectionof}[2]{\bigcap_{#1}^{#2}}
% Or
\newcommand{\orof}[2]{\bigvee_{#1}^{#2}}
% And
\newcommand{\andof}[2]{\bigwedge_{#1}^{#2}}


%===============================================================================
%=============================== LINEAR ALGEBRA ================================
%===============================================================================


% Bold vector arrow
\newcommand{\bv}[1]{\vec{\mathbf{#1}}}

% Matrix or vector (use // for column, & for row) 
% with brackets
\newcommand{\bmat}[1]{\begin{bmatrix} #1 \end{bmatrix}}
% Matrix or vector (use // for column, & for row) 
% with curved brackets
\newcommand{\pmat}[1]{\begin{pmatrix} #1 \end{pmatrix}}
% Matrix or vector (use // for column, & for row) 
% with lines on either side 
\newcommand{\lmat}[1]{\begin{vmatrix} #1 \end{vmatrix}}
% Matrix or vector (use // for column, & for row) 
% with curly braces
\newcommand{\cmat}[1]{\begin{Bmatrix} #1 \end{Bmatrix}}
% Matrix or vector (use // for column, & for row) 
% with no braces
\newcommand{\mat}[1]{\begin{matrix} #1 \end{matrix}}


%===============================================================================
%================================ LARGE OBJECTS ================================
%===============================================================================

% Multiple lines
\newcommand{\multiline}[1]{
\begin{align*}
    #1
\end{align*}
}

% Multiple lines with equation numbers
\newcommand{\eqmultiline}[1]{
\begin{align*}
    #1
\end{align*}
}

% Color
\newcommand{\colorText}[2]{
\begingroup
\color{#1}
    #2
\endgroup
}

% Centered figure
\newcommand{\centerFigure}[2]{
    \begin{figure}[h]
        \centering
            #1
        \caption{#2}
    \end{figure}
}

% Tikz figure
\newcommand{\tikzGraphic}[1]{
    \begin{center}
    \begin{tikzpicture}
        #1
    \end{tikzpicture}
    \end{center}
}

% Enumerate numbers (seperate by \item)
\newcommand{\numbers}{\textbf{\number*)}}

% Enumerate letters (seperate by \item)
\newcommand{\letters}{\textbf{\alph*)}}

\title{%
    \Huge Machine Learning \\
    \Large Lecture V
}
\date{2025-05-12}
\author{Michael Saba}

\begin{document}
\pagenumbering{gobble}
\maketitle
\newpage
\setlength{\parindent}{0pt}
\pagenumbering{arabic}

\section*{Learning in Practice}

\subsection*{Linear Regression}

For linear gression we assume that
we have $\xcal$ and $\ycal = \rbb$
(the output is continuous). \\

For now we assume that $\rcal(f) \equiv 0$. \\

We will use the squared loss $\ell = (z- y)^2$. \\

Roughly, what we mean by linear regression is that
the prediction function is linear in its parameters. \\
It can still be non-linear with repsect to
the input! \\

The first step of Linear Regression is 
to take the input data $\xcal$ and use a feature
map that maps the data to $\rbb^d$:
\[ \varphi: \xcal \to \rbb^d \]
This is representation of $\xcal$ 
that eases learning while carrying the same
information. \\

The hypothesis class $\ffrak$ is a family of
linear functions $f_\theta$ with
repsect to its parameters $\theta \in \Theta$:
\[ f_\theta(x) = \ang{\theta, \varphi(x)}
= \sum_{i = 1}^d \theta_i \varphi_i(x) \]
Where $f_\theta$ is linear in terms of the
$\theta$ parameters, but not $x$,
since we don't know what $\varphi$ does. \\

Note that instead of having $\varphi(x)$
have all $d$ dimensions, we can instead define
a lower dimension $\tilde{\varphi}(x)$
that is in $\rbb^{d-1}$,
and then set $\varphi(x) = (\tilde{\varphi}(x), 1)$. \\
Basically, we want $\varphi(x)$ to have
a single constant $1$ at the end. \\
Note that this allows that the last parameter,
$\theta_d$, is just a shift in this case;
a $b$ offset,
turning the function into an affine function
(since it now has a simple shift element constant). \\

One example of a $\varphi(x)$ is:
\[ \varphi(x) = (1, x, x^2, x^3, \dots, x^{d-1})^T \]
This gives us a polynomial prediction function:
\[ f\theta = \sumof{i=1}{d}\theta_i x^i \]
This is a monomial basis, where the function
is polynomial in terms of $x$,
but linear or affine in terms of $\theta$. \\

There is also a fourrier basis.

We should be careful not to use a basis with too
high a dimension as it would increase the number
of features by a lot. \\

We are using squared loss; we can think
of the result as finding the values of $\theta$
that averages out over all $y$ values.
This makes sense when you look at the
empirical risk and what it calculates. \\

Note that we can write $\riskEmpirical{\ell}{n}
(f_\theta)$ as $\riskEmpirical{\ell}{n}
(\theta)$. \\

We can rewrite risk in vector and matrix
notation. \\
We define $\Phi$ to be the matrix
where each row is $\varphi(x_i)^T$
of the $\nth{i}$ input in the dataset $x_i$. \\
So we get $\Phi \theta - \boldsymbol{y}$
where $\boldsymbol{y}$ is the vector of all
outputs $(y_1, y_2, \dots y_n)^T$. \\
Which means that the risk would look like:
\[ \dfrac{1}{n}\|\Phi \theta - \boldsymbol{y}\|^2 \]
Which makes it clear we are finding the $\theta$
that minimizes the distance expressed above. \\

Now how do we solve:
\[ \dfrac{1}{n}\|\Phi \theta - \boldsymbol{y}\|^2 = 0 \]
We know that we can just differentiate and finding
the gradient of the expression and setting
it to $0$. \\
We know that:
\[ \nabla \riskEmpirical{\ell}{n}(f_{\theta}) 
= \nabla \dfrac{1}{n}
\|\Phi \theta - \boldsymbol{y}\|^2 
= \dfrac{2}{n}\Phi^T(\Phi \theta - \boldsymbol{y}) \]
So setting it to $0$ we get:
\[ \theta = (\Phi^T\Phi)^{-1} \Phi^T 
\boldsymbol{y} \]
This $\theta$ is the set of arguments
that minimizes the empitical risk. \\
Note that this only works if $\Phi$
is full column rank to begin with;
that is, all its columns must linearly
independent. \\
We call $(\Phi^T\Phi)^{-1} \Phi^T$
the pseudo inverse. \\

Note that this is just the normal form
from the Numerical Algorithms course. \\

So if $\Phi$ has full column rank,
we have a unique closed form solution. \\
Note that $\Phi$ having full column rank
makes the $\riskEmpirical{\ell}{n}$
coercive, meaning it has a global minimizer
$\hat{\theta}$. \\
Pratically, $\Phi^T\Phi$ is invertible
if all the columns in $\Phi$
are linearly independent. \\
The full proof is on the slides. \\

Note that $\dfrac{1}{n}\Phi^T\Phi$ is the
empirical covariance matrix (of what?).

Instead of using matrix and vector gradients,
we can try finding the partial derivatives
of the original function and setting the gradient
to $0$ as an exercise. \\

Now that we have a unique closed form solution,
we can add regularization back. \\

We now have to minimize:
\[ \min_{\theta} = \riskEmpirical{\ell}{n}(f_\theta)
+ \lambda \rcal(f_\theta) \] \\
Note that we can also write $\rcal(f_\theta)$
as $\rcal(\theta)$. \\

One example of a regularizer is shrinkage,
or quadratic regularization:
\[ \rcal(\theta) = \|\theta\|^2 \]
What this means is that this regularizer
will favour small parameters $\theta_i$,
and it can be thought of as a measure of
simplicity to ensure the regressor doesn't
become too complex. \\

Note that in certain case, $\rcal(f_\theta)$
and $\rcal(\theta)$ could be different. \\
Sometimes imposing regularization on function
or parameters is not the same. \\

Putting everything together, we have Ridge Regression:
\[ \min_{\theta} = \nabla \dfrac{1}{n}
\|\Phi \theta - \boldsymbol{y}\|^2 
+ \lambda \|\theta\|^2 \]
We can also minimize this,
and it does have a unique closed form solution that
we can find analytically:
\[ \hat{\theta}_\lambda = 
(n\lambda I + \Phi^T\Phi)\inv \Phi^T\boldsymbol{y} \]
Note that $n\lambda I + \Phi^T\Phi$ is always
invertible, no need to impose any column rank
rules on $\Phi$. \\
The proof comes from the fact that:
\[ \left\| \pmat{x \\ y} \right\|^2 
= \|x^2\| + \|y^2\| \]
The full proof is on the slides. \\

Note that we can take our regularized
Ridge Regression risk, and rewrite as an 
unregularized regression problem with a modified
$\tilde{\Phi}$, which we know is solved by 
$\hat{\theta}_\lambda$. \\
We at some point arrive at a matrix:
\[ \Phi^T\Phi + n \lambda I \]
which we need to invert. \\
Note that this matrix is always invertible,
since it is positive definite (the first
part is positive semi-definite,
then we add the diagonal matrix to it,
making it positive-definite),
which is always invertible. \\
That is why Ridge Regression always has a
solution regardless of $\Phi$'s column rank. \\

\newpage

\subsection*{Statistical Analysis of Least Square
Regression}

This section is about analyzing how close we can
get using real data to the optimal classifiers
and risk (Baye's). \\

So we will basically measure how close we
can get to the theoretical best output. \\

We have two settings. \\
In the first one, we assume we have a random
data set. \\
In the second one we have a fixed dataset that
we want to analyze and try to best predict the 
outputs of these fixed inputs. \\
This is like the difference between
$\riskEmpirical{\ell}{S}$ 
and $\riskEmpirical{\ell}{n}$
where $S$ is the random dataset variable. \\

We will make our life a little easier and
focus on the fixed design setting. \\

We can calculate the true (non-empirical)
risk:
\[ \risk{\ell}{}(f_\theta) = 
\dfrac{1}{n}\ebb[\|\Phi \theta - \boldsymbol{Y}\|] \]
Where $\boldsymbol{Y}$
is the random variable of the vector of outputs:
\[ \boldsymbol{Y} = (Y_1,Y_2, \dots, Y_n)^T \]
where each $\boldsymbol{Y_i}$
is the random variable of one output in the dataset. \\

We assume that $\Phi^T\Phi$ is invertible. \\

We will also assume that the optimal classifier
belongs to our hypothesis class $\ffrak$,
and impose that. \\
In other words, we assume that
$\boldsymbol{Y} = \Phi \theta^* + \varepsilon$
where $\varepsilon$ is some random noise
with a known mean and variance (look at slides). \\

Note that $\varepsilon$ is also a vector. \\

We can use this fact to rewrite the expectation
in terms of the noise $\theta^*$ and the noise 
$\varepsilon$. \\

We can prove that if 
$\boldsymbol{Y} = \Phi \theta^* + \varepsilon$,
then this $\theta^*$ gives us Baye's optimal
risk and classifier. \\

The next part we compare Baye's risk with
any $\theta$ to the optimal $\theta^*$
that we defined above. \\
If we calculate some $\tilde{\theta}$
using the random output $\boldsymbol{Y}$,
we can calculate the expected risk
of this $\tilde{\theta}$,
and comapre that to Baye's risk. \\
We can decompose this risk into two parts,
the bias and the variance,
which is basically the average performance
of these parameters that we found. \\

The bias is teh distance between $\tilde{\theta}$
and $\theta^*$. \\
A low bias means that we have a too simple
learning rule, and we have an underfit.\\

The variance measures the variation between them. \\
If it's too high, it means we have overfitting. \\

The slides have a proof of this decomposition. \\

We now propose a linear least squares
estimator 
$\hat{\theta} = \dfrac{1}{n}\Sigma\inv \Phi^T 
\boldsymbol{Y}$. \\
This is the formula we saw before. \\
It is unbiased, and often gives us:
\[ \ebb[\hat{\theta}] = \theta^* \]
So it often gives us the best prameters. \\
Notice that unbiased means the bias
term we saw earlier would go to $0$
(so no bias). \\

So the only term left is the variance. \\
The difference between expected risk and
Baye's risk is given by the variance formula
that is basically our convergence rate
(how far off we are). \\

Note that the formula on the slides for the
convergence rate tells us that we get closer
to Baye's optimal risk with more data
(we divide by $n$). \\

Also the more dimensions we have $d$,
the harder it is to approach Baye's optimal 
risk (we multiply by $d$). \\

NOTE: we use expected risk for a value
if that value depends on a random variable,
which is why we do that for $\tilde{\theta}$
and $\hat{\theta}$. \\

Note that this is best estimator among all
unbiased estimators. \\



\newpage

\end{document}
