
\documentclass[12pt]{article}

\usepackage[margin=1in]{geometry}


%===============================================================================
%================================== PACKAGES ===================================
%===============================================================================

% For using float option H that places figures 
% exatcly where we want them
\usepackage{float}
% makes figure font bold
\usepackage{caption}
\captionsetup[figure]{labelfont=bf}
% For text generation
\usepackage{lipsum}
% For drawing
\usepackage{tikz}
% For smaller or equal sign and not divide sign
\usepackage{amssymb}
% For the diagonal fraction
\usepackage{xfrac}
% For enumerating exercise parts with letters instead of numbers
\usepackage{enumitem}
% For dfrac, which forces the fraction to be in display mode (large) e
% even in math mode (small)
\usepackage{amsmath}
% For degree sign
\usepackage{gensymb}
% For "\mathbb" macro
\usepackage{amsfonts}
% For positioning 
\usepackage{indentfirst}
\usetikzlibrary{shapes,positioning,fit,calc}
% for adjustwidth environment
\usepackage{changepage}
% for arrow on top
\usepackage{esvect}
% for mathbb 1
\usepackage{bbm}
% for mathsrc
\usepackage[mathscr]{eucal}
% For degree sign
\usepackage{gensymb}
% For quotes
\usepackage{csquotes}
% For vertical lines
\usepackage{mathtools}
% For cols
\usepackage{multicol}

% for tikz
\usepackage{pgfplots}
\pgfplotsset{compat=1.18}
\usepackage{amsmath}
\usepgfplotslibrary{groupplots}


%===============================================================================
%==================================== FONTS ====================================
%===============================================================================


% Mathcal
\newcommand{\acal}{\mathcal{A}}
\newcommand{\bcal}{\mathcal{B}}
\newcommand{\ccal}{\mathcal{C}}
\newcommand{\dcal}{\mathcal{D}}
\newcommand{\ecal}{\mathcal{E}}
\newcommand{\fcal}{\mathcal{F}}
\newcommand{\gcal}{\mathcal{G}}
\newcommand{\hcal}{\mathcal{H}}
\newcommand{\ical}{\mathcal{I}}
\newcommand{\jcal}{\mathcal{J}}
\newcommand{\kcal}{\mathcal{K}}
\newcommand{\lcal}{\mathcal{L}}
\newcommand{\mcal}{\mathcal{M}}
\newcommand{\ncal}{\mathcal{N}}
\newcommand{\ocal}{\mathcal{O}}
\newcommand{\pcal}{\mathcal{P}}
\newcommand{\qcal}{\mathcal{Q}}
\newcommand{\rcal}{\mathcal{R}}
\newcommand{\scal}{\mathcal{S}}
\newcommand{\tcal}{\mathcal{T}}
\newcommand{\ucal}{\mathcal{U}}
\newcommand{\vcal}{\mathcal{V}}
\newcommand{\wcal}{\mathcal{W}}
\newcommand{\xcal}{\mathcal{X}}
\newcommand{\ycal}{\mathcal{Y}}
\newcommand{\zcal}{\mathcal{Z}}

% Mathfrak
\newcommand{\afrak}{\mathfrak{A}}
\newcommand{\bfrak}{\mathfrak{B}}
\newcommand{\cfrak}{\mathfrak{C}}
\newcommand{\dfrak}{\mathfrak{D}}
\newcommand{\efrak}{\mathfrak{E}}
\newcommand{\ffrak}{\mathfrak{F}}
\newcommand{\gfrak}{\mathfrak{G}}
\newcommand{\hfrak}{\mathfrak{H}}
\newcommand{\ifrak}{\mathfrak{I}}
\newcommand{\jfrak}{\mathfrak{J}}
\newcommand{\kfrak}{\mathfrak{K}}
\newcommand{\lfrak}{\mathfrak{L}}
\newcommand{\mfrak}{\mathfrak{M}}
\newcommand{\nfrak}{\mathfrak{N}}
\newcommand{\ofrak}{\mathfrak{O}}
\newcommand{\pfrak}{\mathfrak{P}}
\newcommand{\qfrak}{\mathfrak{Q}}
\newcommand{\rfrak}{\mathfrak{R}}
\newcommand{\sfrak}{\mathfrak{S}}
\newcommand{\tfrak}{\mathfrak{T}}
\newcommand{\ufrak}{\mathfrak{U}}
\newcommand{\vfrak}{\mathfrak{V}}
\newcommand{\wfrak}{\mathfrak{W}}
\newcommand{\xfrak}{\mathfrak{X}}
\newcommand{\yfrak}{\mathfrak{Y}}
\newcommand{\zfrak}{\mathfrak{Z}}

% Mathscr
\newcommand{\ascr}{\mathscr{A}}
\newcommand{\bscr}{\mathscr{B}}
\newcommand{\cscr}{\mathscr{C}}
\newcommand{\dscr}{\mathscr{D}}
\newcommand{\escr}{\mathscr{E}}
\newcommand{\fscr}{\mathscr{F}}
\newcommand{\gscr}{\mathscr{G}}
\newcommand{\hscr}{\mathscr{H}}
\newcommand{\iscr}{\mathscr{I}}
\newcommand{\jscr}{\mathscr{J}}
\newcommand{\kscr}{\mathscr{K}}
\newcommand{\lscr}{\mathscr{L}}
\newcommand{\mscr}{\mathscr{M}}
\newcommand{\nscr}{\mathscr{N}}
\newcommand{\oscr}{\mathscr{O}}
\newcommand{\pscr}{\mathscr{P}}
\newcommand{\qscr}{\mathscr{Q}}
\newcommand{\rscr}{\mathscr{R}}
\newcommand{\sscr}{\mathscr{S}}
\newcommand{\tscr}{\mathscr{T}}
\newcommand{\uscr}{\mathscr{U}}
\newcommand{\vscr}{\mathscr{V}}
\newcommand{\wscr}{\mathscr{W}}
\newcommand{\xscr}{\mathscr{X}}
\newcommand{\yscr}{\mathscr{Y}}
\newcommand{\zscr}{\mathscr{Z}}

% Mathbb
\newcommand{\abb}{\mathbb{A}}
\newcommand{\bbb}{\mathbb{B}}
\newcommand{\cbb}{\mathbb{C}}
\newcommand{\dbb}{\mathbb{D}}
\newcommand{\ebb}{\mathbb{E}}
\newcommand{\fbb}{\mathbb{F}}
\newcommand{\gbb}{\mathbb{G}}
\newcommand{\hbb}{\mathbb{H}}
\newcommand{\ibb}{\mathbb{I}}
\newcommand{\jbb}{\mathbb{J}}
\newcommand{\kbb}{\mathbb{K}}
\newcommand{\lbb}{\mathbb{L}}
\newcommand{\mbb}{\mathbb{M}}
\newcommand{\nbb}{\mathbb{N}}
\newcommand{\obb}{\mathbb{O}}
\newcommand{\pbb}{\mathbb{P}}
\newcommand{\qbb}{\mathbb{Q}}
\newcommand{\rbb}{\mathbb{R}}
\newcommand{\sbb}{\mathbb{S}}
\newcommand{\tbb}{\mathbb{T}}
\newcommand{\ubb}{\mathbb{U}}
\newcommand{\vbb}{\mathbb{V}}
\newcommand{\wbb}{\mathbb{W}}
\newcommand{\xbb}{\mathbb{X}}
\newcommand{\ybb}{\mathbb{Y}}
\newcommand{\zbb}{\mathbb{Z}}


%===============================================================================
%=============================== SPECIAL SYMBOLS ===============================
%===============================================================================


% Orbit (group theory)
\newcommand{\orbit}{\mathcal{O}}
% Normal group
\newcommand{\normal}{\mathcal{N}}
% Indicator function
\newcommand{\indicator}{\mathbbm{1}}
% Laplace transform
\newcommand{\laplace}[1]{\mathcal{L}}
% Epsilon shorthand
\newcommand{\eps}{\varepsilon}
% Omega
\newcommand{\om}{\omega}
\newcommand{\Om}{\Omega}

%===============================================================================
%================================== OPERATORS ==================================
%===============================================================================


% Inverse exponent
\newcommand{\inv}[0]{^{-1}}
% Overline bar
\newcommand{\olsi}[1]{\,\overline{\!{#1}}}
% Less than or equal slanted
\newcommand{\seqs}{\leqslant}
% Greater or equal slanted
\newcommand{\geqs}{\geqslant}
% Subset or equal
\newcommand{\sub}{\subseteq}
% Proper subset
\newcommand{\prosub}{\subset}
% from
\newcommand{\from}{\leftarrow}

% Parantheses
\newcommand{\para}[1]{\left( #1 \right)}
% Curly Braces
\newcommand{\curl}[1]{\left\{ #1 \right\}}
% Brackets
\newcommand{\brac}[1]{\left[ #1 \right]}
% Angled Brackets
\newcommand{\ang}[1]{\left\langle #1 \right\rangle}
% Norm
\newcommand{\norm}[1]{\left\| #1 \right\|}

% Piece wise (use \\ between cases)
\newcommand{\piecewise}[1]{\begin{cases} #1 \end{cases}}

% Bold symbol shorthand
\newcommand{\bl}{\boldsymbol}

% Vertical space
\newcommand{\vs}[1]{\vspace{#1 pt}}
% Horizontal ertical space
\newcommand{\hs}[1]{\hspace{#1 pt}}


%===============================================================================
%============================== TEXT BASED SYMBOLS =============================
%===============================================================================


% Radians
\newcommand{\rad}{\text{rad}}
% Least Common Multiple
\newcommand{\lcm}{\text{lcm}}
% Automorphism
\newcommand{\Aut}{\text{Aut}}
% Variance
\newcommand{\var}{\text{Var}}
% Covariance
\newcommand{\cov}{\text{Cov}}
% Cofactor (matrix)
\newcommand{\cof}{\text{Cof}}
% Adjugate (matrix)
\newcommand{\adj}{\text{Adj}}
% Trace (matrix)
\newcommand{\tr}{\text{tr}}
% Standard deviation
\newcommand{\std}{\text{Std}}
% Correlation coefficient
\newcommand{\corr}{\text{Corr}}
% Sign
\newcommand{\sign}{\text{sign}}

% And text
\newcommand{\AND}{\text{ and }}
% Or text 
\newcommand{\OR}{\text{ or }}
% If text
\newcommand{\IF}{\text{ if }}
% When text
\newcommand{\WHEN}{\text{ when }}
% Then text
\newcommand{\THEN}{\text{ then }}

% 1st
\newcommand{\st}[1]{#1^{\text{st}}}
% 2nd
\newcommand{\nd}[1]{#1^{\text{nd}}}
% 3rd
\newcommand{\rd}[1]{#1^{\text{rd}}}
% nth
\newcommand{\nth}[1]{#1^{\text{th}}}


%===============================================================================
%========================= PROBABILITY AND STATISTICS ==========================
%===============================================================================


% Permutation
\newcommand{\perm}[2]{{}^{#1}\!P_{#2}}
% Combination
\newcommand{\comb}[2]{{}^{#1}C_{#2}}

% Baye's risk
\newcommand{\risk}[1]{\mathscr{R}_{#1}}
% Baye's optimal risk
\newcommand{\riskOptimal}[1]{\mathscr{R}_{#1}^*}
% Baye's empirical risk
\newcommand{\riskEmpirical}[2]{\hat{\mathscr{R}}_{#1}^{#2}}


%===============================================================================
%=================================== CALCULUS ==================================
%===============================================================================


% d over d derivative
\newcommand{\dd}[2]{\dfrac{d#1}{d#2}}
% partial d over d derivative
\newcommand{\partialdd}[2]{\dfrac{\partial #1}{\partial #2}}
% delta d over d derivative
\newcommand{\deltadd}[2]{\dfrac{\Delta #1}{\Delta #2}}

% Integration between a and b
\newcommand{\integral}[4]{\int_{#1}^{#2} #3 \, #4}
% Integration in some space 
\newcommand{\boundIntegral}[2]{\int_{#1} #2 \, d#1}

% Limit
\newcommand{\limit}[3]{\lim_{#1 \to #2} #3}


%===============================================================================
%================================  BIG SYMBOLS  ================================
%===============================================================================


% Sum
\newcommand{\sumof}[2]{\sum_{#1}^{#2}}
% Product
\newcommand{\productof}[2]{\prod_{#1}^{#2}}
% Union
\newcommand{\unionof}[2]{\bigcup_{#1}^{#2}}
% Intersection
\newcommand{\intersectionof}[2]{\bigcap_{#1}^{#2}}
% Or
\newcommand{\orof}[2]{\bigvee_{#1}^{#2}}
% And
\newcommand{\andof}[2]{\bigwedge_{#1}^{#2}}


%===============================================================================
%=============================== LINEAR ALGEBRA ================================
%===============================================================================


% Bold vector arrow
\newcommand{\bv}[1]{\vec{\mathbf{#1}}}

% Matrix or vector (use // for column, & for row) 
% with brackets
\newcommand{\bmat}[1]{\begin{bmatrix} #1 \end{bmatrix}}
% Matrix or vector (use // for column, & for row) 
% with curved brackets
\newcommand{\pmat}[1]{\begin{pmatrix} #1 \end{pmatrix}}
% Matrix or vector (use // for column, & for row) 
% with lines on either side 
\newcommand{\lmat}[1]{\begin{vmatrix} #1 \end{vmatrix}}
% Matrix or vector (use // for column, & for row) 
% with curly braces
\newcommand{\cmat}[1]{\begin{Bmatrix} #1 \end{Bmatrix}}
% Matrix or vector (use // for column, & for row) 
% with no braces
\newcommand{\mat}[1]{\begin{matrix} #1 \end{matrix}}


%===============================================================================
%================================ LARGE OBJECTS ================================
%===============================================================================

% Multiple lines
\newcommand{\multiline}[1]{
\begin{align*}
    #1
\end{align*}
}

% Multiple lines with equation numbers
\newcommand{\eqmultiline}[1]{
\begin{align*}
    #1
\end{align*}
}

% Color
\newcommand{\colorText}[2]{
\begingroup
\color{#1}
    #2
\endgroup
}

% Centered figure
\newcommand{\centerFigure}[2]{
    \begin{figure}[h]
        \centering
            #1
        \caption{#2}
    \end{figure}
}

% Tikz figure
\newcommand{\tikzGraphic}[1]{
    \begin{center}
    \begin{tikzpicture}
        #1
    \end{tikzpicture}
    \end{center}
}

% Enumerate numbers (seperate by \item)
\newcommand{\numbers}{\textbf{\number*)}}

% Enumerate letters (seperate by \item)
\newcommand{\letters}{\textbf{\alph*)}}

\title{%
    \Huge Machine Learning \\
    \Large Lecture XI
}
\date{2025-06-09}
\author{Michael Saba}

\begin{document}
\pagenumbering{gobble}
\maketitle
\newpage
\setlength{\parindent}{0pt}
\pagenumbering{arabic}

\section*{Convex Optimization}
\subsection*{Lower Bound for the Convergence Rate}

We can show, non-constructively (only in theory)
what the optimal first order iterative
numerical method is for solving
optimization problems. \\

This gives us the lower bounds on the
convergence rate given some prior assumptions. \\

Optimal methods are those that have a convergence
rates proportional to the lower bound. \\

If we assume that the function $f$
is $m$-strongly convex and that $\nabla f$
is $L$-Lipschitz continuous,
then the lower bound on the convergence 
rate is 
\[ O(\omega^k)\] 
where $\omega$ depends on 
\[ \dfrac{\sqrt{L/m} - 1}{\sqrt{L/m} + 1} \]
The conjugate gradient and heavy-ball method
are both optimal in that case. \\

The heavy ball method is gradient descent 
with momentum. \\

On the other hand, if we only assume that $f$
is convex, and that $\nabla f$
is $L$-Lipschitz continuous,
then the convergence rate is 
\[ O\para{\dfrac{1}{k^2}} \]
Gradient descent is not optimal in this case,
but the Nesterov Accelerated Gradient method
is optimal. \\

\newpage

\subsection*{Nesterov's Accelerated Gradient
Method}

It is an improved Gradient Descent method.
Each iteration is at least as costly 
as gradient descent
since it also uses the gradient. \\

In Nesterov's Accelerated Gradient
Method, we want to minimize $f$:
\[ \min_{\theta \in \mathbb{R}^d} f(\theta) \]
And we assume that $f$
is convex and that $\nabla f$
is $L$-Lipschitz continuous. \\
We then choose two starting values:
\[ \theta^{(0)} = \theta^{(-1)} \in \rbb^d \]
A starting time $t_0 = 0$,
and a step size $\tau > 0$. \\
This yields the update steps:
\[ t_{k+1}^2 - t_{k+1} \leq t_k^2 \]
\[ \eta^{(k)} = \theta^{(k)} 
+ \frac{t_k - 1}{t_{k+1}} \left( \theta^{(k)} 
- \theta^{(k-1)} \right) \quad 
\text{(extrapolation step)} \]
\[ \theta^{(k+1)} = \eta^{(k)} - \tau \nabla 
f(\eta^{(k)}) \quad \text{(gradient 
descent step)} \]
Which is similar to gradient descent,
but with an extra $\eta(x)$ term. \\

This term is calculated in the extrapolation 
step. It is like a small extra step
we take in a specific direction
we take each iteration,
which gets us closer to the answer. \\

This yields the convergence rate:
\[ f(\theta^{(k)}) - f(\theta^*) \leq 
\frac{1}{2\tau} \frac{\|\theta^* - 
\theta^{(0)}\|^2}{t_k^2} \]
If we further assume that $t_k = \frac{k+1}{2}$,
we can bound the convergence rate further:
\[ \frac{2}{\tau} \frac{\|\theta^* - 
\theta^{(0)}\|^2}{(k+1)^2} = 
O\left(\frac{1}{k^2}\right) \]
Which is optimal as mentioned earlier. \\

The gradient calculation is the only costly 
part of this algorithm. \\

Note that it works for even non quadratic-functions,
so long as the smoothness and convexity
conditions are met. \\

\newpage

\subsection*{Proximal mapping}

If we want to generalize our optimization
methods to non-smooth functions, we need
to first calculate their proximal mappings. \\

The formula for the proximal mapping of $f$
with step size $\tau$ of the point $\theta^{(k)}$
is:
\[ \text{prox}_{\tau f}(\theta^{(k)}) 
= \arg\min_{\theta} \left[ f(\theta) 
+ \frac{1}{2\tau} \|\theta - \theta^{(k)}\|^2 
\right] \]
Notice that this will need to be done
each iteration. \\

If the proximal mapping of $f$ is easy to find,
we say that $f$ is prox-friendly. \\

Note that the function $\delta_C(\theta)$
is defined as:
\[ \delta_C(\theta)
= \piecewise{
    0, \qquad \;\; \IF  \theta \in C \\
    \infty, \qquad \IF \theta \notin C
} \]
It is the inicator function of a convex set $C$,
and it has infinite value if the point
is not in the convex set $C$. \\

So, for example, if we have
to find the proximal mapping of an indicator
function for some convex set:
\[ \text{prox}_{\tau \delta_C}(\theta^{(k)})
= \arg\min_{\theta} \left[ \delta_C(\theta) 
+ \frac{1}{2\tau} \|\theta - \theta^{(k)}\|^2 
\right] \]
Notice that the indictaor function is infinite
if we allow that $\theta \notin C$,
which dominates all other terms, so in order
to minimize we sort of have ot have $\theta$
be in $C$, which means that:
\[ \text{prox}_{\tau \delta_C}(\theta^{(k)})
= \arg\min_{\theta \in C} \left[ \frac{1}{2\tau} 
\|\theta - \theta^{(k)}\|^2 \right] \]
Where $\delta_C(\theta)$ became $0$.
Notice what we are saying here;
we want to find the $\theta$
in $C$ such that the distance between 
$\theta^{(k)}$ and $\theta$ is minimal. \\
So if $\theta^{(k)}$ is in $C$,
we just set $\theta = \theta^{(k)}$. \\
Otherwise, if $\theta^{(k)}$ is not in $C$,
then the closest point to $\theta^{(k)}$
on $C$ is the projection of $\theta^{(k)}$
on the convex set $C$. \\

Some examples of proximal mappings:
\begin{enumerate}
    \item For $f(\theta) = \|\theta\|^2_2$:
    \[ \left[ \text{prox}_{\tau \|\cdot\|_2^2}
    (\bar{\theta}) \right]_i = 
    \frac{\bar{\theta}_i}{1 + 2\tau}, 
    \quad \forall i = 1, \ldots, d \]
    \item For $f(\theta) = 
    \delta_{\|\theta \|_2 \leq 1}(\theta)$,
    where the convex set is the unit disk,
    the proximal mapping is:
    \[ \text{prox}_{\tau \delta_{\|\cdot\|_2 \leq 1}}
    (\bar{\theta}) = \text{proj}_{\{\theta \mid 
    \|\theta\|_2 \leq 1\}}
    (\bar{\theta}) = \frac{\bar{\theta}}
    {\max(1, \|\bar{\theta}\|)}\]
    Which just means that we either set
    $\olsi{theta}$ to $\theta$ or the projection
    onto the unit circle $\|\theta\|_2 \leq 1$.
    \item For $f(\theta) = \|\theta\|_1$,
    also called the soft shrinkage-thresholding 
    operator:
    \[ \left[ \text{prox}_{\tau \|\cdot\|_1}
    (\bar{\theta}) \right]_i = 
    \max(0, |\bar{\theta}_i| - \tau) \cdot 
    \text{sign}(\bar{\theta}_i) 
    \quad \forall i = 1, \ldots, d \]
\end{enumerate}
We prove these are true in an assignment. \\

Note that while $\|\theta\|_1$
is prox-friendly, and can have proximal mapping
done to it,
this isn't true for $\|A \theta\|_1$
for some arbitrary matrix $A$. \\

\newpage

\subsection*{Proximal Gradient Descent}

Structured/Additive Composite optimization
is used in order to split an optimization problem. \\
If we have:
\[ \min_\theta h(\theta) \]
And we know that we can split $h$ such that:
\[ h(\theta) = f(\theta) + g(\theta)\]
where $f$ is $L$-smooth and convex,
and $g$ is convex,
non-smooth, and prox-friendly,
then we just need to minimize each
function individually. \\

Notice that this form
resembles the formula we get
when we want to minimize the empirical risk
plus the regularization term. \\
The empirical risk is smooth, and the regularizer
may be non-smooth, such as in LASSO. \\

Each step, we can update $\theta$
with respect to $f$ using the gradient
descent method (forward step):
\[ \theta^{(k)} = \theta^{(k)} - 
\tau \nabla f(\theta^{(k)}) \]
And then follow it up by an update
with respect to $g$ using proximal mapping
(backwards step):
\[ \theta^{(k+1)} = \text{prox}_{\tau g}
(\bar{\theta}^{(k)}) \]
Putting both of them together,
we get a single proximal-gradient step:
\[ \theta^{(k+1)} = \text{prox}_{\tau g} 
\left( \theta^{(k)} - \tau \nabla 
f(\theta^{(k)}) \right) \]
This algorithm converges towards
the minimum of $h(\theta)$,
and is called proximal-gradient descent. \\

So we just have to assume that $f$
is $L$-smooth (meaning that $\nabla f$
is $L$-Lipschitz continuous)
and convex, and that $g$
is convex and prox friendly. \\
We can pick any starting variable 
$\theta^{(0)} \in \rbb^d$,
and a step size:
\[ 0 < \tau < \dfrac{2}{L} \]
Then, we can apply the proximal gradient step
repeatedly. \\

This yields a sub-linear convergence rate
$O\para(\dfrac{1}{k})$. \\

If we further assume that $f$
is $m$-strongly convex, then we
get a linear convergence rate
$O(\omega^k)$ where:
\[ \omega = \frac{L/m - 1}{L/m + 1} \quad 
\FOR \quad \tau = \frac{2}{L + m}\]
Recall that a linear convergence rate
is faster than sublinear. \\

We can also use accelerated gradient descent 
here, where we use Nestorov's algorithm
instead of vanilla gradient descent. \\

We again make the same assumptions
about $f$ and $g$,
and pick any starting variables:
\[ \theta^{(0)} = \theta^{(-1)} \in \rbb^d \]
And some step size $\tau > 0$,
which gives us the update steps:
\[ t_{k+1}^2 - t_{k+1} \leq t_k^2 \]
\[ \eta^{(k)} = \theta^{(k)} + 
\frac{t_k - 1}{t_{k+1}} \left( \theta^{(k)} 
- \theta^{(k-1)} \right) \]
\[ \theta^{(k+1)} = \text{prox}_{\tau g} 
\left( \eta^{(k)} - \tau \nabla f(\eta^{(k)}) 
\right)\]
Which converges towards the minimizer. \\

This yields a convergence rate:
\[ h(\theta^{(k)}) - h(\theta^*) 
\leq \frac{1}{2\tau} \frac{\|\theta^* - 
\theta^{(0)}\|^2}{t_k^2} \]
And if we take $t_k = \dfrac{k+1}{2}$,
then the convergence rate becomes:
\[ \frac{2}{\tau} \frac{\|\theta^* - 
\theta^{(0)}\|^2}{(k+1)^2} = 
O\left(\frac{1}{k^2}\right) \]
Note that we don't call this a gradient
descent method because there is no descent;
there is convergence, but not every step
gets us closer. \\

So it is simpy called Accelerated Proximal 
Gradient Method.
It is also called FISTA. \\

\newpage

\subsection*{Duality}

Now we still have to find a way to solve
Hard and soft margin SVM, as well as regularized
least 1-norm minimization,
since they are not prox-friendly or are
non-smooth. \\

A disk is just a collection of points in space. \\

Or we can think of it as the intersection of
several half-spaces (all tangent to the circle). \\

We can call a closed set $C$
convex if it is the intersection of
closed half-spaces that contain it. \\

The same applies to convex functions:
a convex function $f$ is always the
pointwise supremum (maximum or largest
point not in the set) of the
collection of affine functions $h$
such that $h \leq f$. \\
Think of it as lines of tangents below a graph
of $f$. \\

The conjugate of $f$ is $f^*$,
and it can be thought of as a function that 
operates on slopes;
it takes in a slope as input, and returns
the part where the affine function
intersects the $y$-axis.
It does so by sort of sliding the affine
function with the slope $\xi$
until it becomes tangent to $f(\theta)$. \\
We do that by maximizing the distance
between $f(\theta)$
and $\ang{\xi, \theta}$, 
which is the affine function
(tangent). \\

The dual of $f(\theta)$,
which is a function:
\[ f^* : \rbb^d \to \olsi{\rbb} \]
Is given by:
\[ f^*(\xi) = \sup_{\theta \in \rbb^d} 
\brac{\ang{\xi, \theta} - f(\theta)} \]
Which is again our way of maximizing
the distance. \\

One small note on notation:
\[ \delta_{\|\theta\|_2 \leq 1} = \delta_{B_1^2(0)} \]
We use $B_1^2$ to refer to the convex set of 
the unit ball centered at $0$ (in any dimension).\\
Note that $B_1^n$ refers to the unit ball with
respect to the $n$-norm, and $B_1^\infty$
with respect to the infinity norm. \\ 
If we write $B_1$, we mean $B_1^2$,
with respect to distance. \\

In other words:
\[ B_n^r(c) = \{ \theta \mid \| 
\theta - c \|_r \leq n \} \]
Is a shorthand that will be used. \\

For example, we have:
\begin{enumerate}
    \item $f(\theta) = \dfrac{1}{2}\theta^2
    \rightsquigarrow f^*(\xi) = \dfrac{1}{2}\xi^2$.
    \item $f(\theta) = \dfrac{1}{2}\|\theta\|_2^2
    \rightsquigarrow f^*(\xi) = \dfrac{1}{2}\|\xi\|_2^2$.
    \item $f(\theta) = \|\theta\|_2
    \rightsquigarrow f^*(\xi) = \delta_{B_1^2(0)}(\xi)$.
    \item $f(\theta) = \|\theta\|_1
    \rightsquigarrow f^*(\xi) = \delta_{B_1^\infty(0)}(\xi)$.
    \item $f(\theta) = \delta_{\rbb_-^d}(\theta)
    \rightsquigarrow f^*(\xi) = \delta_{\rbb_+^d}(\xi)$.
\end{enumerate}
The goal will be to try and reduce other
functions' dual to a function one of these,
as we can see in the slides. \\

The bi-conjugate is the conjugate of the
conjugate $f^{**}$. \\

The conjugate $f^*$ is always convex. \\

If $f$ is convex, then $f = f^{**}$. \\

The Fenchel-Rockafellar duality tells
us that if we have some optimization problem
with some structure:
\[ \min_\theta f(A\theta) + g(\theta) \]
then solving the primal problem corresponds
to solving the dual problem:
\[ \max_\xi - g^*(-A^T\xi) - f^*(\xi) \]
We do so by computing the conjugates of $f$
and $g$ seperately. \\
Notice that we ignore the $A$
in the primal problem. \\
Note that $f$ and $g$ must be convex
and $A$ must be a linear mapping. \\

To prove that solving a minimization problem
on the primal is the same as solving a maximization
on the dual:
\[ \min_{\theta} f(A\theta) + g(\theta) \]
Now recall that the dual of a dual is the
the original function, so:
\[\min_{\theta} \sup_{\xi} \brac{ \langle \xi, 
A\theta \rangle - f^*(\xi) + g(\theta)} \]
\[\max_{\xi}  \inf_{\theta}\brac{ \langle \xi, 
A\theta \rangle - f^*(\xi) + g(\theta)} \]
\[\max_{\xi} \brac{ \inf_{\theta} \brac{ \langle \xi, 
A\theta \rangle + g(\theta)} - f^*(\xi)} \]
\[ \max_{\xi} - g^*(-A^\top \xi) - f^*(\xi) \]
This is basically the
Fenchel-Rockafellar duality. \\

So we may be able to use proximal gradient
methods on the dual problem.
First we get the dual, which 
gets us a form solvable by proximal gradient
methods. \\

We can use Moreau's identity:
\[ \theta = \operatorname{prox}_{\tau f}(\theta) 
+ \tau \, \operatorname{prox}_{\frac{1}{\tau} f^*}
\left(\frac{\theta}{\tau}\right) \]
when solving the optimization. \\

Note that we can write $\delta_{\rbb_-}(1-t)$
as a loss function for the hard-margin SVM. \\

\end{document}
