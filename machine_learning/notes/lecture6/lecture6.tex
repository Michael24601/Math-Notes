
\documentclass[12pt]{article}

\usepackage[margin=1in]{geometry}


%===============================================================================
%================================== PACKAGES ===================================
%===============================================================================

% For using float option H that places figures 
% exatcly where we want them
\usepackage{float}
% makes figure font bold
\usepackage{caption}
\captionsetup[figure]{labelfont=bf}
% For text generation
\usepackage{lipsum}
% For drawing
\usepackage{tikz}
% For smaller or equal sign and not divide sign
\usepackage{amssymb}
% For the diagonal fraction
\usepackage{xfrac}
% For enumerating exercise parts with letters instead of numbers
\usepackage{enumitem}
% For dfrac, which forces the fraction to be in display mode (large) e
% even in math mode (small)
\usepackage{amsmath}
% For degree sign
\usepackage{gensymb}
% For "\mathbb" macro
\usepackage{amsfonts}
% For positioning 
\usepackage{indentfirst}
\usetikzlibrary{shapes,positioning,fit,calc}
% for adjustwidth environment
\usepackage{changepage}
% for arrow on top
\usepackage{esvect}
% for mathbb 1
\usepackage{bbm}
% for mathsrc
\usepackage[mathscr]{eucal}
% For degree sign
\usepackage{gensymb}
% For quotes
\usepackage{csquotes}
% For vertical lines
\usepackage{mathtools}
% For cols
\usepackage{multicol}

% for tikz
\usepackage{pgfplots}
\pgfplotsset{compat=1.18}
\usepackage{amsmath}
\usepgfplotslibrary{groupplots}


%===============================================================================
%==================================== FONTS ====================================
%===============================================================================


% Mathcal
\newcommand{\acal}{\mathcal{A}}
\newcommand{\bcal}{\mathcal{B}}
\newcommand{\ccal}{\mathcal{C}}
\newcommand{\dcal}{\mathcal{D}}
\newcommand{\ecal}{\mathcal{E}}
\newcommand{\fcal}{\mathcal{F}}
\newcommand{\gcal}{\mathcal{G}}
\newcommand{\hcal}{\mathcal{H}}
\newcommand{\ical}{\mathcal{I}}
\newcommand{\jcal}{\mathcal{J}}
\newcommand{\kcal}{\mathcal{K}}
\newcommand{\lcal}{\mathcal{L}}
\newcommand{\mcal}{\mathcal{M}}
\newcommand{\ncal}{\mathcal{N}}
\newcommand{\ocal}{\mathcal{O}}
\newcommand{\pcal}{\mathcal{P}}
\newcommand{\qcal}{\mathcal{Q}}
\newcommand{\rcal}{\mathcal{R}}
\newcommand{\scal}{\mathcal{S}}
\newcommand{\tcal}{\mathcal{T}}
\newcommand{\ucal}{\mathcal{U}}
\newcommand{\vcal}{\mathcal{V}}
\newcommand{\wcal}{\mathcal{W}}
\newcommand{\xcal}{\mathcal{X}}
\newcommand{\ycal}{\mathcal{Y}}
\newcommand{\zcal}{\mathcal{Z}}

% Mathfrak
\newcommand{\afrak}{\mathfrak{A}}
\newcommand{\bfrak}{\mathfrak{B}}
\newcommand{\cfrak}{\mathfrak{C}}
\newcommand{\dfrak}{\mathfrak{D}}
\newcommand{\efrak}{\mathfrak{E}}
\newcommand{\ffrak}{\mathfrak{F}}
\newcommand{\gfrak}{\mathfrak{G}}
\newcommand{\hfrak}{\mathfrak{H}}
\newcommand{\ifrak}{\mathfrak{I}}
\newcommand{\jfrak}{\mathfrak{J}}
\newcommand{\kfrak}{\mathfrak{K}}
\newcommand{\lfrak}{\mathfrak{L}}
\newcommand{\mfrak}{\mathfrak{M}}
\newcommand{\nfrak}{\mathfrak{N}}
\newcommand{\ofrak}{\mathfrak{O}}
\newcommand{\pfrak}{\mathfrak{P}}
\newcommand{\qfrak}{\mathfrak{Q}}
\newcommand{\rfrak}{\mathfrak{R}}
\newcommand{\sfrak}{\mathfrak{S}}
\newcommand{\tfrak}{\mathfrak{T}}
\newcommand{\ufrak}{\mathfrak{U}}
\newcommand{\vfrak}{\mathfrak{V}}
\newcommand{\wfrak}{\mathfrak{W}}
\newcommand{\xfrak}{\mathfrak{X}}
\newcommand{\yfrak}{\mathfrak{Y}}
\newcommand{\zfrak}{\mathfrak{Z}}

% Mathscr
\newcommand{\ascr}{\mathscr{A}}
\newcommand{\bscr}{\mathscr{B}}
\newcommand{\cscr}{\mathscr{C}}
\newcommand{\dscr}{\mathscr{D}}
\newcommand{\escr}{\mathscr{E}}
\newcommand{\fscr}{\mathscr{F}}
\newcommand{\gscr}{\mathscr{G}}
\newcommand{\hscr}{\mathscr{H}}
\newcommand{\iscr}{\mathscr{I}}
\newcommand{\jscr}{\mathscr{J}}
\newcommand{\kscr}{\mathscr{K}}
\newcommand{\lscr}{\mathscr{L}}
\newcommand{\mscr}{\mathscr{M}}
\newcommand{\nscr}{\mathscr{N}}
\newcommand{\oscr}{\mathscr{O}}
\newcommand{\pscr}{\mathscr{P}}
\newcommand{\qscr}{\mathscr{Q}}
\newcommand{\rscr}{\mathscr{R}}
\newcommand{\sscr}{\mathscr{S}}
\newcommand{\tscr}{\mathscr{T}}
\newcommand{\uscr}{\mathscr{U}}
\newcommand{\vscr}{\mathscr{V}}
\newcommand{\wscr}{\mathscr{W}}
\newcommand{\xscr}{\mathscr{X}}
\newcommand{\yscr}{\mathscr{Y}}
\newcommand{\zscr}{\mathscr{Z}}

% Mathbb
\newcommand{\abb}{\mathbb{A}}
\newcommand{\bbb}{\mathbb{B}}
\newcommand{\cbb}{\mathbb{C}}
\newcommand{\dbb}{\mathbb{D}}
\newcommand{\ebb}{\mathbb{E}}
\newcommand{\fbb}{\mathbb{F}}
\newcommand{\gbb}{\mathbb{G}}
\newcommand{\hbb}{\mathbb{H}}
\newcommand{\ibb}{\mathbb{I}}
\newcommand{\jbb}{\mathbb{J}}
\newcommand{\kbb}{\mathbb{K}}
\newcommand{\lbb}{\mathbb{L}}
\newcommand{\mbb}{\mathbb{M}}
\newcommand{\nbb}{\mathbb{N}}
\newcommand{\obb}{\mathbb{O}}
\newcommand{\pbb}{\mathbb{P}}
\newcommand{\qbb}{\mathbb{Q}}
\newcommand{\rbb}{\mathbb{R}}
\newcommand{\sbb}{\mathbb{S}}
\newcommand{\tbb}{\mathbb{T}}
\newcommand{\ubb}{\mathbb{U}}
\newcommand{\vbb}{\mathbb{V}}
\newcommand{\wbb}{\mathbb{W}}
\newcommand{\xbb}{\mathbb{X}}
\newcommand{\ybb}{\mathbb{Y}}
\newcommand{\zbb}{\mathbb{Z}}


%===============================================================================
%=============================== SPECIAL SYMBOLS ===============================
%===============================================================================


% Orbit (group theory)
\newcommand{\orbit}{\mathcal{O}}
% Normal group
\newcommand{\normal}{\mathcal{N}}
% Indicator function
\newcommand{\indicator}{\mathbbm{1}}
% Laplace transform
\newcommand{\laplace}[1]{\mathcal{L}}
% Epsilon shorthand
\newcommand{\eps}{\varepsilon}
% Omega
\newcommand{\om}{\omega}
\newcommand{\Om}{\Omega}

%===============================================================================
%================================== OPERATORS ==================================
%===============================================================================


% Inverse exponent
\newcommand{\inv}[0]{^{-1}}
% Overline bar
\newcommand{\olsi}[1]{\,\overline{\!{#1}}}
% Less than or equal slanted
\newcommand{\seqs}{\leqslant}
% Greater or equal slanted
\newcommand{\geqs}{\geqslant}
% Subset or equal
\newcommand{\sub}{\subseteq}
% Proper subset
\newcommand{\prosub}{\subset}
% from
\newcommand{\from}{\leftarrow}

% Parantheses
\newcommand{\para}[1]{\left( #1 \right)}
% Curly Braces
\newcommand{\curl}[1]{\left\{ #1 \right\}}
% Brackets
\newcommand{\brac}[1]{\left[ #1 \right]}
% Angled Brackets
\newcommand{\ang}[1]{\left\langle #1 \right\rangle}
% Norm
\newcommand{\norm}[1]{\left\| #1 \right\|}

% Piece wise (use \\ between cases)
\newcommand{\piecewise}[1]{\begin{cases} #1 \end{cases}}

% Bold symbol shorthand
\newcommand{\bl}{\boldsymbol}

% Vertical space
\newcommand{\vs}[1]{\vspace{#1 pt}}
% Horizontal ertical space
\newcommand{\hs}[1]{\hspace{#1 pt}}


%===============================================================================
%============================== TEXT BASED SYMBOLS =============================
%===============================================================================


% Radians
\newcommand{\rad}{\text{rad}}
% Least Common Multiple
\newcommand{\lcm}{\text{lcm}}
% Automorphism
\newcommand{\Aut}{\text{Aut}}
% Variance
\newcommand{\var}{\text{Var}}
% Covariance
\newcommand{\cov}{\text{Cov}}
% Cofactor (matrix)
\newcommand{\cof}{\text{Cof}}
% Adjugate (matrix)
\newcommand{\adj}{\text{Adj}}
% Trace (matrix)
\newcommand{\tr}{\text{tr}}
% Standard deviation
\newcommand{\std}{\text{Std}}
% Correlation coefficient
\newcommand{\corr}{\text{Corr}}
% Sign
\newcommand{\sign}{\text{sign}}

% And text
\newcommand{\AND}{\text{ and }}
% Or text 
\newcommand{\OR}{\text{ or }}
% If text
\newcommand{\IF}{\text{ if }}
% When text
\newcommand{\WHEN}{\text{ when }}
% Then text
\newcommand{\THEN}{\text{ then }}

% 1st
\newcommand{\st}[1]{#1^{\text{st}}}
% 2nd
\newcommand{\nd}[1]{#1^{\text{nd}}}
% 3rd
\newcommand{\rd}[1]{#1^{\text{rd}}}
% nth
\newcommand{\nth}[1]{#1^{\text{th}}}


%===============================================================================
%========================= PROBABILITY AND STATISTICS ==========================
%===============================================================================


% Permutation
\newcommand{\perm}[2]{{}^{#1}\!P_{#2}}
% Combination
\newcommand{\comb}[2]{{}^{#1}C_{#2}}

% Baye's risk
\newcommand{\risk}[1]{\mathscr{R}_{#1}}
% Baye's optimal risk
\newcommand{\riskOptimal}[1]{\mathscr{R}_{#1}^*}
% Baye's empirical risk
\newcommand{\riskEmpirical}[2]{\hat{\mathscr{R}}_{#1}^{#2}}


%===============================================================================
%=================================== CALCULUS ==================================
%===============================================================================


% d over d derivative
\newcommand{\dd}[2]{\dfrac{d#1}{d#2}}
% partial d over d derivative
\newcommand{\partialdd}[2]{\dfrac{\partial #1}{\partial #2}}
% delta d over d derivative
\newcommand{\deltadd}[2]{\dfrac{\Delta #1}{\Delta #2}}

% Integration between a and b
\newcommand{\integral}[4]{\int_{#1}^{#2} #3 \, #4}
% Integration in some space 
\newcommand{\boundIntegral}[2]{\int_{#1} #2 \, d#1}

% Limit
\newcommand{\limit}[3]{\lim_{#1 \to #2} #3}


%===============================================================================
%================================  BIG SYMBOLS  ================================
%===============================================================================


% Sum
\newcommand{\sumof}[2]{\sum_{#1}^{#2}}
% Product
\newcommand{\productof}[2]{\prod_{#1}^{#2}}
% Union
\newcommand{\unionof}[2]{\bigcup_{#1}^{#2}}
% Intersection
\newcommand{\intersectionof}[2]{\bigcap_{#1}^{#2}}
% Or
\newcommand{\orof}[2]{\bigvee_{#1}^{#2}}
% And
\newcommand{\andof}[2]{\bigwedge_{#1}^{#2}}


%===============================================================================
%=============================== LINEAR ALGEBRA ================================
%===============================================================================


% Bold vector arrow
\newcommand{\bv}[1]{\vec{\mathbf{#1}}}

% Matrix or vector (use // for column, & for row) 
% with brackets
\newcommand{\bmat}[1]{\begin{bmatrix} #1 \end{bmatrix}}
% Matrix or vector (use // for column, & for row) 
% with curved brackets
\newcommand{\pmat}[1]{\begin{pmatrix} #1 \end{pmatrix}}
% Matrix or vector (use // for column, & for row) 
% with lines on either side 
\newcommand{\lmat}[1]{\begin{vmatrix} #1 \end{vmatrix}}
% Matrix or vector (use // for column, & for row) 
% with curly braces
\newcommand{\cmat}[1]{\begin{Bmatrix} #1 \end{Bmatrix}}
% Matrix or vector (use // for column, & for row) 
% with no braces
\newcommand{\mat}[1]{\begin{matrix} #1 \end{matrix}}


%===============================================================================
%================================ LARGE OBJECTS ================================
%===============================================================================

% Multiple lines
\newcommand{\multiline}[1]{
\begin{align*}
    #1
\end{align*}
}

% Multiple lines with equation numbers
\newcommand{\eqmultiline}[1]{
\begin{align*}
    #1
\end{align*}
}

% Color
\newcommand{\colorText}[2]{
\begingroup
\color{#1}
    #2
\endgroup
}

% Centered figure
\newcommand{\centerFigure}[2]{
    \begin{figure}[h]
        \centering
            #1
        \caption{#2}
    \end{figure}
}

% Tikz figure
\newcommand{\tikzGraphic}[1]{
    \begin{center}
    \begin{tikzpicture}
        #1
    \end{tikzpicture}
    \end{center}
}

% Enumerate numbers (seperate by \item)
\newcommand{\numbers}{\textbf{\number*)}}

% Enumerate letters (seperate by \item)
\newcommand{\letters}{\textbf{\alph*)}}

\title{%
    \Huge Machine Learning \\
    \Large Lecture VI
}
\date{2025-05-15}
\author{Michael Saba}

\begin{document}
\pagenumbering{gobble}
\maketitle
\newpage
\setlength{\parindent}{0pt}
\pagenumbering{arabic}

\section*{Learning in Practice}

\subsection*{Linear Algebra}

This section contains some useful Linear
Algebra rules. \\

The trace of a matrix is the sum of its eigenvalues:
\[ \tr(M) = \tr(P\Lambda P^T) = \tr(\Lambda) \]
This is true because of a more general rule,
which states that for any matrix $A$
and orthogonal matrix $P$:
\[ \tr(PAP^T) = \tr(A) \]
Which we can prove as such:
\[ \tr(PAP^T) = \tr(P^TPA) = \tr(IA) = \tr(A) \]
Using the cyclical property. \\

If the eigenvalues of a matrix $M$
are all smaller or equal to $\dfrac{1}{2}$,
then:
\[ w^TMw \leq \dfrac{1}{2} \|w\|^2 \]
for any vector $w \in \rbb^n$.
This is because:
\[ w^TMw = w^T(P\Lambda P^T)w
= (w^TP)\Lambda(P^Tw)
= (P^Tw)^T\Lambda(P^Tw) \]
Now, we can take $z = P^Tw$, and write:
\[ (P^Tw)^T\Lambda(P^Tw) 
= z^T \Lambda z = \sum_i \lambda_i z_i^2 \]
where $\lambda_i$ is the $\nth{i}$ eigenvalue. \\
Notice that because $\lambda_i \leq \dfrac{1}{2}$,
this means that:
\[\sum_i \lambda_i z_i^2 
\leq \sum_i \dfrac{1}{2} z_i^2 = \dfrac{1}{2} \|z\|^2 \]
Note that because $P$ is orthogonal,
multiplying $w$ by it won't change its length
(norm), so:
\[ \|w\|^2 = \|z\|^2 \]
Which means that:
\[w^TMw = \sum_i \lambda_i z_i^2 
\leq \dfrac{1}{2} \|w\|^2\]
Proving the theorem. \\

In one of the assignments we proved the
more general theorem of:
\[ \dfrac{x^TAx}{x^Tx} \leq \lambda_{\text{max}} \]
Where $\text{max}$ is the largest
eigenvalue of $A$. \\

We can prove a similar theorem applies for the trace. \\ 
If the eigenvalues of a matrix $M$
are all smaller or equal to $\dfrac{1}{2}$, then:
\[ \tr(BM) \leq \dfrac{\tr(B)}{2} \]
for any matrix $B$. \\
First note that
by the circular property of the trace:
\[ \tr(BM) = \tr(B P\Lambda P^T) = \tr(\Lambda P^T B P) \]
Now $\Lambda$ is just a diagonal matrix,
which means that its eigenvalues
$\lambda_i$ each multiply the diagonal
elements of $P^T B P$,
which then get summed by the trace.
Moreover, $\lambda_i \leq \dfrac{1}{2}$.
So we can write:
\[ \tr(\Lambda P^T B P) 
= \sum_i{\lambda_i (P^T B P)_{ii}}
\leq \sum_i{\dfrac{1}{2} (P^T B P)_{ii}} = 
\dfrac{\tr(P^T B P)}{2} \]
However, recall that since $P$ is orthogonal:
\[ \tr(P^T B P) = \tr(B) \]
Which means that:
\[ \tr(BM) \leq \dfrac{\tr(B)}{2} \]
Which completes the proof. \\

The matrix inversion lemma states that:
\[ {\displaystyle \left(A+UCV\right)^{-1}=
A^{-1}-A^{-1}U\left(C^{-1}+VA^{-1}U\right)^{-1}
VA^{-1}} \]
Which applies for any matrices. \\\

\newpage

\subsection*{Statistical Analysis of the 
Ridge Regression Estimator}

In this section,
we focus on analyzing the estimator 
$\hat{\theta}_\lambda$
that minimizes the ridge regression:
\[ \|\Phi \theta - \bl{y} \|^2 + \lambda\|\theta\|^2 \]
Which itself reduces to a linear least square
regression. \\

Recall that in our statistical analysis
of the linear least square regression,
we found that the excess risk
of an estimator $\tilde{\theta}$
that depends on the output data randomness 
$\bl{Y}$ decomposes into:
\[ \ebb[\risk{\ell}(\tilde{\theta})] 
- \riskOptimal{\ell} = \underbrace{
\| \ebb[\tilde{\theta}] - \theta^*
\|^2_{\hat{\Sigma}}}_{\text{Bias}}
+ \underbrace{
\ebb[\|\tilde{\theta} - \ebb[\tilde{\theta}] 
\|^2_{\hat{\Sigma}}]}_{\text{Variance}}\]
Where $\riskOptimal{\ell} = \sigma^2$. \\

Now, we will try to find the excess risk,
or convergence rate, of the ridge regression
estimator, which is given by:
\[ \hat{\theta}_\lambda
= \dfrac{1}{n}(\hat{\Sigma} + \lambda I)\inv
\Phi^T \bl{Y} \]
Where $\bl{Y}$ is the random dataset output
we used to compute the estimator. \\

The ridge regression estimator is not
unbiased; both the bias and the variance
will contribute to the total excess risk. \\

First note that:
\[ \ebb[\hat{\theta}_\lambda]
= \ebb\brac{\dfrac{1}{n}(\hat{\Sigma} 
+ \lambda I)\inv \Phi^T \bl{Y}}
= \dfrac{1}{n}(\hat{\Sigma} 
+ \lambda I)\inv \Phi^T \ebb\brac{\bl{Y}} \]
Where recall that 
$\bl{Y} = \Phi \theta^* + \bl{\eps}$.
So we get:
\[ \dfrac{1}{n}(\hat{\Sigma} 
+ \lambda I)\inv \Phi^T 
\ebb\brac{\Phi \theta^* + \bl{\eps}} = 
\dfrac{1}{n}(\hat{\Sigma} 
+ \lambda I)\inv \Phi^T 
(\Phi \theta^* + \ebb[\bl{\eps}]) \]
Where $\ebb[\bl{\eps}] = 0$, so:
\[  \ebb[\hat{\theta}_\lambda] 
= \dfrac{1}{n}(\hat{\Sigma} 
+ \lambda I)\inv \Phi^T \Phi \theta^*
= (\hat{\Sigma} + \lambda I)\inv 
\hat{\Sigma} \theta^* \]
Using the matrix inversion lemma,
we can rewrite this as:
\[ \theta^* - 
\lambda(\hat{\Sigma} + \lambda I)\inv \theta^* \]
Which is our final answer. \\

Now, to calculate the bias $B$:
\[ B = \| \ebb[\hat{\theta}_\lambda] 
- \theta^* \|_{\hat{\Sigma}}^2
= \norm{ \theta^* - \lambda(\hat{\Sigma} 
+ \lambda I)\inv \theta^* - \theta^* }^2
= \norm{\lambda(\hat{\Sigma} 
+ \lambda I)\inv \theta^*}_{\hat{\Sigma}}^2 \]
\[ B = (\lambda(\hat{\Sigma} 
+ \lambda I)\inv \theta^*)^T\hat{\Sigma}
\lambda(\hat{\Sigma} 
+ \lambda I)\inv \theta^*
= \lambda^2(\theta^*)^T
((\hat{\Sigma} + \lambda I)\inv)^T\hat{\Sigma}
(\hat{\Sigma} + \lambda I)\inv \theta^*\]
Note that because $\hat{\Sigma}$
is symmetric, so is $\hat{\Sigma} + \lambda I$,
and in turn, so is $(\hat{\Sigma} + \lambda I)\inv$,
so we can remove the transpose from on it
since it won't affect it. \\
So we can now write:
\[ \lambda^2(\theta^*)^T
(\hat{\Sigma} + \lambda I)\inv\hat{\Sigma}
(\hat{\Sigma} + \lambda I)\inv \theta^* \]
Now notice that $\hat{\Sigma}$
and $(\hat{\Sigma} + \lambda I)\inv$ commute:
\[ \hat{\Sigma}(\hat{\Sigma} + \lambda I)\inv
= (\hat{\Sigma}\inv)\inv
(\hat{\Sigma} + \lambda I)\inv
= (\hat{\Sigma}\hat{\Sigma}\inv 
+ \lambda \hat{\Sigma}\inv)\inv
= (I + \lambda \hat{\Sigma}\inv)\inv \]
\[ (\hat{\Sigma} + \lambda I)\inv \hat{\Sigma}
= (\hat{\Sigma} + \lambda I)\inv 
(\hat{\Sigma}\inv)\inv
= (\hat{\Sigma}\inv\hat{\Sigma}
+ \lambda \hat{\Sigma}\inv)\inv
= (I + \lambda \hat{\Sigma}\inv)\inv \]
So we can now rewrite the bias as:
\[ B = \lambda^2(\theta^*)^T
(\hat{\Sigma} + \lambda I)\inv
(\hat{\Sigma} + \lambda I)\inv \hat{\Sigma} \theta^*
= \lambda^2(\theta^*)^T
(\hat{\Sigma} + \lambda I)^{-2}
\hat{\Sigma} \theta^* \]
Which is our bias. \\

Now, for the variance $V$,
we can just replace the values
of $\hat{\theta}_\lambda$
and $\ebb[\hat{\theta}_\lambda]$ directly,
using the values we derived earlier:
\[ \ebb[\hat{\theta}_\lambda]
= \dfrac{1}{n}(\hat{\Sigma} 
+ \lambda I)\inv \Phi^T \Phi \theta^* \]
Now, we have:
\[ V = \ebb[\|\hat{\theta}_\lambda 
- \ebb[\hat{\theta}_\lambda] \|^2_{\hat{\Sigma}}]
= \ebb\brac{\norm{\dfrac{1}{n}(\hat{\Sigma} + \lambda I)\inv
\Phi^T \bl{Y}
- \dfrac{1}{n}(\hat{\Sigma} + \lambda I)\inv 
\Phi^T \Phi \theta^* }^2_{\hat{\Sigma}}} \]
We can then replace $\bl{Y}$ by 
$\Phi \theta^* + \bl{\eps}$:
\[ V = \ebb\brac{\norm{\dfrac{1}{n}
(\hat{\Sigma} + \lambda I)\inv \Phi^T \Phi \theta^*
+ \dfrac{1}{n}
(\hat{\Sigma} + \lambda I)\inv \Phi^T \bl{\eps}
- \dfrac{1}{n}(\hat{\Sigma} + \lambda I)\inv 
\Phi^T \Phi \theta^* }^2_{\hat{\Sigma}}} \]
Which simplifies to:
\[ V = \ebb\brac{\norm{ \dfrac{1}{n}
(\hat{\Sigma} + \lambda I)\inv \Phi^T \bl{\eps} 
}^2_{\hat{\Sigma}}}
= \ebb\brac{\para{\dfrac{1}{n}
(\hat{\Sigma} + \lambda I)\inv \Phi^T \bl{\eps}}^T
\hat{\Sigma} \para{\dfrac{1}{n}
(\hat{\Sigma} + \lambda I)\inv \Phi^T \bl{\eps}}} \]
\[ \ebb\brac{ \dfrac{1}{n^2} \bl{\eps}^T \Phi 
((\hat{\Sigma} + \lambda I)^{-1})^T
\hat{\Sigma}(\hat{\Sigma} + \lambda I)\inv 
\Phi^T \bl{\eps}} \]
Note that since $(\hat{\Sigma} + \lambda I)\inv$
is symmetric, we can remove the transpose since
it won't affect it:
\[ V = \ebb\brac{ \dfrac{1}{n^2} \bl{\eps}^T \Phi 
(\hat{\Sigma} + \lambda I)\inv
\hat{\Sigma}(\hat{\Sigma} + \lambda I)\inv 
\Phi^T \bl{\eps}} \]
Now, for the next step,
note that what is inside the expected value is
just a scalar, so we can use the trace on it:
\[ V = \ebb\brac{ \dfrac{1}{n^2} \tr\para{\bl{\eps}^T 
\Phi (\hat{\Sigma} + \lambda I)\inv
\hat{\Sigma}(\hat{\Sigma} + \lambda I)\inv 
\Phi^T \bl{\eps}}} \]
This allows us to use the cyclical property
to shift the elements:
\[ V = \ebb\brac{ \dfrac{1}{n^2} \tr\para{ \Phi^T \bl{\eps} 
\bl{\eps}^T \Phi (\hat{\Sigma} + \lambda I)\inv
\hat{\Sigma}(\hat{\Sigma} + \lambda I)\inv }} \]
Now, the trace of the expected value is the expected
value of the trace, so we have:
\[ V = \dfrac{1}{n^2} \tr\para{ \ebb\brac{ \Phi^T \bl{\eps} 
\bl{\eps}^T \Phi (\hat{\Sigma} + \lambda I)\inv
\hat{\Sigma}(\hat{\Sigma} + \lambda I)\inv }} \]
Of course, with the exception of the noise,
all of these terms are just constants as far 
as the expected value is concerned:
\[ V = \dfrac{1}{n^2} \tr\para{ \Phi^T \ebb\brac{ \bl{\eps} 
\bl{\eps}^T } \Phi (\hat{\Sigma} + \lambda I)\inv
\hat{\Sigma}(\hat{\Sigma} + \lambda I)\inv } \]
Now recall that $\ebb[\bl{\eps} \bl{\eps}^T] = \sigma^2 I$
and that $\dfrac{1}{n}\Phi^T\Phi = \hat{\Sigma}$, so:
\[ V = \dfrac{1}{n^2} \tr\para{ \Phi^T  \sigma^2 I \Phi (\hat{\Sigma} + \lambda I)\inv
\hat{\Sigma}(\hat{\Sigma} + \lambda I)\inv } \]
\[ V = \dfrac{\sigma^2}{n}
\tr\para{ \hat{\Sigma} (\hat{\Sigma} + \lambda I)\inv
\hat{\Sigma}(\hat{\Sigma} + \lambda I)\inv } \]
Which is our variance $V$. \\

So to recap, the excess risk is:
\[ \risk{\ell}(\hat{\theta}_\lambda) - \riskOptimal{\ell}
= B + V \]
Which is:
\[ \para{\lambda^2(\theta^*)^T
(\hat{\Sigma} + \lambda I)^{-2}
\hat{\Sigma} \theta^*}
+ \para{\dfrac{\sigma^2}{n}
\tr\para{ \hat{\Sigma} (\hat{\Sigma} + \lambda I)\inv
\hat{\Sigma}(\hat{\Sigma} + \lambda I)\inv }} \]
This is the exact value, but it is
also very difficult to analyze,
so instead we will find a simpler upper bound
for the bias and variance. \\

First, recall that if the eigenvalues of a matrix $M$
are all smaller or equal to $\dfrac{1}{2}$, then:
\[ w^TMw \leq \dfrac{1}{2} \|w\|^2 \]
for any vector $w \in \rbb^n$. \\
We proved this in the last section. \\

For an upper bound on the bias $B$, 
we can show that the matrix
$(\hat{\Sigma} + \lambda I)^{-2}\lambda \hat{\Sigma}$
has eigen values that are all smaller than
$\dfrac{1}{2}$. \\
First note that since $\hat{\Sigma}$
is positive semi-definite and symmetric,
it will definitely have an eigen decomposition:
\[ \hat{\Sigma} = P\Lambda P^T = P\Lambda P\inv \]
where $\Lambda$ is diagonal and $P$ is orthogonal. \\
This means that we can write 
$(\hat{\Sigma} + \lambda I)^{-2}\lambda \hat{\Sigma}$ 
as:
\[ (\hat{\Sigma} + \lambda I)^{-2}\lambda \hat{\Sigma}
= (P\Lambda P\inv + \lambda I)^{-2} \lambda( P\Lambda P\inv)\]
We can then do some matrix arithmetic to get:
\[ (P\Lambda P\inv + \lambda P I P\inv)^{-2} 
\lambda( P\Lambda P\inv)
= (P(\Lambda + \lambda I) P\inv)^{-2} 
\lambda( P\Lambda P\inv) \]
We can now distribute the inversion:
\[ ((P(\Lambda + \lambda I) P\inv)\inv)^2
\lambda( P\Lambda P\inv)
= (P(\Lambda + \lambda I)\inv P\inv)^2 
\lambda( P\Lambda P\inv) \]
And then multiply the total:
\[ (P(\Lambda + \lambda I)\inv P\inv)
(P(\Lambda + \lambda I)\inv P\inv)\lambda( P\Lambda P\inv)
=  P (\Lambda + \lambda I)^{-2} \lambda \Lambda P\inv\]
So:
\[ (\hat{\Sigma} + \lambda I)^{-2}\lambda \hat{\Sigma}
= P (\Lambda + \lambda I)^{-2} \lambda \Lambda P\inv \]
This is the eigen decomposition
of $(\hat{\Sigma} + \lambda I)^{-2}\lambda \hat{\Sigma}$,
since the matrix $(\Lambda + \lambda I)^{-2} \lambda \Lambda$
is diagonal. \\
This means that each eigenvalue $\mu_i$
of $(\hat{\Sigma} + \lambda I)^{-2}\lambda \hat{\Sigma}$
will have the form:
\[ \mu_i = 
\dfrac{\lambda \sigma_i}{(\lambda + \sigma_i)^2} \]
Where $\sigma_i$ is an eigenvalue in $\Lambda$. \\
Notice that this is always smaller than $\dfrac{1}{2}$
because:
\[ \dfrac{\lambda \sigma_i}{(\lambda + \sigma_i)^2}
\leq \dfrac{1}{2} \]
\[ 2\lambda \sigma_i
\leq (\lambda + \sigma_i)^2 \]
\[ 2\lambda \sigma_i
\leq \lambda^2 + \sigma_i^2 + 2\lambda \sigma_i \]
\[ 0 \leq \lambda^2 + \sigma_i^2 \]
Which is always true since $\lambda^2 + \sigma_i^2$
is always positive
(sum of two square numbers). \\

So now, we have an upperbound on our bias:
\[ B = \lambda^2(\theta^*)^T
(\hat{\Sigma} + \lambda I)^{-2}
\hat{\Sigma} \theta^*
= \lambda(\theta^*)^T (
(\hat{\Sigma} + \lambda I)^{-2} 
\lambda\hat{\Sigma}) \theta^* 
\leq \dfrac{\lambda}{2}\|\theta^*\|^2 \]
Which is much simpler to analyze. \\

Now, recall that our variance was:
\[ \dfrac{\sigma^2}{n}
\tr\para{ \hat{\Sigma} (\hat{\Sigma} + \lambda I)\inv
\hat{\Sigma}(\hat{\Sigma} + \lambda I)\inv } \]
And because $\hat{\Sigma}$ and 
$(\hat{\Sigma} + \lambda I)\inv$
commute, we can rewrite this as:
\[ \dfrac{\sigma^2}{n}
\tr\para{ \hat{\Sigma}^2 (\hat{\Sigma} + \lambda I)^{-2}}
= \dfrac{\sigma^2}{\lambda n}
\tr\para{ \hat{\Sigma}\lambda\hat{\Sigma} 
(\hat{\Sigma} + \lambda I)^{-2}}\]
Which we can now find an upperbound for. \\

Similar to the norm, 
if the eigenvalues of a matrix $M$
are all smaller or equal to $\dfrac{1}{2}$, then:
\[ \tr(BM) \leq \dfrac{\tr(B)}{2} \]
for any matrix $B$. \\
This was shown to be true in the last section. \\

Like before, using the exact same
argument, we can show that $\hat{\Sigma} 
(\hat{\Sigma} + \lambda I)^{-2}$
has eigenvalues that are smaller or equal 
to $\dfrac{1}{2}$. 
This in turn will give us the upperbound:
\[ B = \dfrac{\sigma^2}{\lambda n}
\tr\para{ \hat{\Sigma} \para{ \lambda\hat{\Sigma} 
(\hat{\Sigma} + \lambda I)^{-2}}}
\leq \dfrac{\sigma^2 \tr(\hat{\Sigma})}{2\lambda n}\]
Which is similar to the argument we made before,
but the with trace instead of the norm squared. \\

Now, we have an upperbound:
\[ \risk{\ell}(\hat{\theta}_\lambda) - \riskOptimal{\ell}
\leq \dfrac{\lambda}{2}\|\theta^*\|^2 +
\dfrac{\sigma^2 \tr(\hat{\Sigma})}{2\lambda n} \]
For the excess risk. \\

Notice that a larger $\lambda$ increases
the bias, but a smaller $\lambda$
increases the variance. \\
This is because, as we mentioned earlier,
the bias is higher if we have a simple 
underfitting model,
and the variance is higher if we have a
complex overfitting model.
The regularization, which is modulated by 
the parameter $\lambda$,
is what controls the simplicty
and complexity of the model,
with more regularization equalling
more simplicity. \\

There may be the a $\lambda$ sweet spot
for the optimal tradeoff between bias
and variance. \\
We can get it by trying to minimize
excess risk upperbound by setting its derivative
with respect to $\lambda$ equal to $0$:
\[ \dd{}{\lambda}\para{ \dfrac{\lambda}{2}\|\theta^*\|^2 +
\dfrac{\sigma^2 \tr(\hat{\Sigma})}{2\lambda n}} = 0 \]
\[ \dfrac{1}{2}\|\theta^*\|^2 -
\dfrac{\sigma^2 \tr(\hat{\Sigma})}{2\lambda^2 n} = 0 \]
\[ \|\theta^*\|^2 =
\dfrac{\sigma^2 \tr(\hat{\Sigma})}{\lambda^2 n} \]
\[ \lambda^2 =
\dfrac{\sigma^2 \tr(\hat{\Sigma})}{\|\theta^*\|^2 n} \]
\[ \lambda_* =
\dfrac{\sigma \sqrt{\tr(\hat{\Sigma})}}
{\|\theta^*\| \sqrt{n}} \]
Where $\lambda_*$ is the optimal 
regularization parameter. \\

Placing $\lambda_*$
back inside the upperbound, we get:
\[ \risk{\ell}(\hat{\theta}_\lambda) - \riskOptimal{\ell}
\leq \dfrac{\sigma \sqrt{\tr(\hat{\Sigma})} \|\theta^*\| }
{\sqrt{n}} \]
Which is the convergence rate given the optimal
regularization parameter $\lambda_*$ was chosen. \\

Note that $\lambda_*$ is not exactly
optimal since we've used
the upper bound of the excess risk,
and not the actual excess risk,
to calculate it. \\

Note that $d$, the number of dimensions of the feature
vector, does not affect the convergence rate
for ridge regression. \\

Because we have $\sigma$ instead of $\sigma^2$
in the convergence rate, the bound depends less
on the noise in the data than in non-regularized 
risk. \\

Note that because we have $\sqrt(n)$
in the denominator instead of $n$,
with a higher number of data points, the
the function converges slower towards optimality. \\

\newpage

\subsection*{Statistical Analysis in the Random 
Design Setting}

This is an optional section. \\

We can repeat the same steps we did in the last
two sections, but in the random design setting
instead of the fixed design setting. \\
where we have random datasets $\bl{X}$ and $\bl{Y}$. \\

I will not do this section. \\

\newpage

\subsection*{LASSO}

This section introduces a new regularizer. \\

It is the 1-norm regularizer, given by:
\[ \rcal{\theta} = \|\theta\|_1
= \sum_i |\theta_i| \]
Which is just the sum of the asbolute value
of the parameters. \\

The 1-norm regularization penalizes small and large
weights equally. \\
It leads to a non-smooth optimization problem,
and does not have a closed form solution. \\
We can still solve it, using iterative
optimization algorithms. \\

It induces sparcity in the $\theta$
parameter vector, meaning that it does
feature selection, such that $\Phi \theta$
will have many missing terms (less overhead). \\

Many other (non-smooth) regularizers also
induce sparsity. \\
\[ \|\theta\|_p \]
Where $p$ can even be smaller or equal to $1$. \\
They belong to a whole family
depending on norms used. \\

Infinity norm does not induce sparsity
(looks like a square, but straight). \\

Note that many of these "norms"
when $p < 1$, are not actually norms
mathematically (triangle inequality not satisfied). \\
We can still use them however. \\

We now just give the convergence rate of
several different norm regularizers,
no proofs, just the value. \\
For example:
\[ k \sqrt{\dfrac{\log(d)}{n}} \]
is the convergence rate of the $1$-norm
regularizer, where $k$ is the number of non-zero
coefficients in the true $\theta$ vector
(the optimal one). Moreover,
\[ \dfrac{\sigma^2 k \log(d)}{n} \]
is the convergence rate for the $0$-norm. \\

Another type of sparsity besides
is group sparsity. \\

\newpage

\end{document}
