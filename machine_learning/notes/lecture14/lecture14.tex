
\documentclass[12pt]{article}

\usepackage[margin=1in]{geometry}


%===============================================================================
%================================== PACKAGES ===================================
%===============================================================================

% For using float option H that places figures 
% exatcly where we want them
\usepackage{float}
% makes figure font bold
\usepackage{caption}
\captionsetup[figure]{labelfont=bf}
% For text generation
\usepackage{lipsum}
% For drawing
\usepackage{tikz}
% For smaller or equal sign and not divide sign
\usepackage{amssymb}
% For the diagonal fraction
\usepackage{xfrac}
% For enumerating exercise parts with letters instead of numbers
\usepackage{enumitem}
% For dfrac, which forces the fraction to be in display mode (large) e
% even in math mode (small)
\usepackage{amsmath}
% For degree sign
\usepackage{gensymb}
% For "\mathbb" macro
\usepackage{amsfonts}
% For positioning 
\usepackage{indentfirst}
\usetikzlibrary{shapes,positioning,fit,calc}
% for adjustwidth environment
\usepackage{changepage}
% for arrow on top
\usepackage{esvect}
% for mathbb 1
\usepackage{bbm}
% for mathsrc
\usepackage[mathscr]{eucal}
% For degree sign
\usepackage{gensymb}
% For quotes
\usepackage{csquotes}
% For vertical lines
\usepackage{mathtools}
% For cols
\usepackage{multicol}

% for tikz
\usepackage{pgfplots}
\pgfplotsset{compat=1.18}
\usepackage{amsmath}
\usepgfplotslibrary{groupplots}


%===============================================================================
%==================================== FONTS ====================================
%===============================================================================


% Mathcal
\newcommand{\acal}{\mathcal{A}}
\newcommand{\bcal}{\mathcal{B}}
\newcommand{\ccal}{\mathcal{C}}
\newcommand{\dcal}{\mathcal{D}}
\newcommand{\ecal}{\mathcal{E}}
\newcommand{\fcal}{\mathcal{F}}
\newcommand{\gcal}{\mathcal{G}}
\newcommand{\hcal}{\mathcal{H}}
\newcommand{\ical}{\mathcal{I}}
\newcommand{\jcal}{\mathcal{J}}
\newcommand{\kcal}{\mathcal{K}}
\newcommand{\lcal}{\mathcal{L}}
\newcommand{\mcal}{\mathcal{M}}
\newcommand{\ncal}{\mathcal{N}}
\newcommand{\ocal}{\mathcal{O}}
\newcommand{\pcal}{\mathcal{P}}
\newcommand{\qcal}{\mathcal{Q}}
\newcommand{\rcal}{\mathcal{R}}
\newcommand{\scal}{\mathcal{S}}
\newcommand{\tcal}{\mathcal{T}}
\newcommand{\ucal}{\mathcal{U}}
\newcommand{\vcal}{\mathcal{V}}
\newcommand{\wcal}{\mathcal{W}}
\newcommand{\xcal}{\mathcal{X}}
\newcommand{\ycal}{\mathcal{Y}}
\newcommand{\zcal}{\mathcal{Z}}

% Mathfrak
\newcommand{\afrak}{\mathfrak{A}}
\newcommand{\bfrak}{\mathfrak{B}}
\newcommand{\cfrak}{\mathfrak{C}}
\newcommand{\dfrak}{\mathfrak{D}}
\newcommand{\efrak}{\mathfrak{E}}
\newcommand{\ffrak}{\mathfrak{F}}
\newcommand{\gfrak}{\mathfrak{G}}
\newcommand{\hfrak}{\mathfrak{H}}
\newcommand{\ifrak}{\mathfrak{I}}
\newcommand{\jfrak}{\mathfrak{J}}
\newcommand{\kfrak}{\mathfrak{K}}
\newcommand{\lfrak}{\mathfrak{L}}
\newcommand{\mfrak}{\mathfrak{M}}
\newcommand{\nfrak}{\mathfrak{N}}
\newcommand{\ofrak}{\mathfrak{O}}
\newcommand{\pfrak}{\mathfrak{P}}
\newcommand{\qfrak}{\mathfrak{Q}}
\newcommand{\rfrak}{\mathfrak{R}}
\newcommand{\sfrak}{\mathfrak{S}}
\newcommand{\tfrak}{\mathfrak{T}}
\newcommand{\ufrak}{\mathfrak{U}}
\newcommand{\vfrak}{\mathfrak{V}}
\newcommand{\wfrak}{\mathfrak{W}}
\newcommand{\xfrak}{\mathfrak{X}}
\newcommand{\yfrak}{\mathfrak{Y}}
\newcommand{\zfrak}{\mathfrak{Z}}

% Mathscr
\newcommand{\ascr}{\mathscr{A}}
\newcommand{\bscr}{\mathscr{B}}
\newcommand{\cscr}{\mathscr{C}}
\newcommand{\dscr}{\mathscr{D}}
\newcommand{\escr}{\mathscr{E}}
\newcommand{\fscr}{\mathscr{F}}
\newcommand{\gscr}{\mathscr{G}}
\newcommand{\hscr}{\mathscr{H}}
\newcommand{\iscr}{\mathscr{I}}
\newcommand{\jscr}{\mathscr{J}}
\newcommand{\kscr}{\mathscr{K}}
\newcommand{\lscr}{\mathscr{L}}
\newcommand{\mscr}{\mathscr{M}}
\newcommand{\nscr}{\mathscr{N}}
\newcommand{\oscr}{\mathscr{O}}
\newcommand{\pscr}{\mathscr{P}}
\newcommand{\qscr}{\mathscr{Q}}
\newcommand{\rscr}{\mathscr{R}}
\newcommand{\sscr}{\mathscr{S}}
\newcommand{\tscr}{\mathscr{T}}
\newcommand{\uscr}{\mathscr{U}}
\newcommand{\vscr}{\mathscr{V}}
\newcommand{\wscr}{\mathscr{W}}
\newcommand{\xscr}{\mathscr{X}}
\newcommand{\yscr}{\mathscr{Y}}
\newcommand{\zscr}{\mathscr{Z}}

% Mathbb
\newcommand{\abb}{\mathbb{A}}
\newcommand{\bbb}{\mathbb{B}}
\newcommand{\cbb}{\mathbb{C}}
\newcommand{\dbb}{\mathbb{D}}
\newcommand{\ebb}{\mathbb{E}}
\newcommand{\fbb}{\mathbb{F}}
\newcommand{\gbb}{\mathbb{G}}
\newcommand{\hbb}{\mathbb{H}}
\newcommand{\ibb}{\mathbb{I}}
\newcommand{\jbb}{\mathbb{J}}
\newcommand{\kbb}{\mathbb{K}}
\newcommand{\lbb}{\mathbb{L}}
\newcommand{\mbb}{\mathbb{M}}
\newcommand{\nbb}{\mathbb{N}}
\newcommand{\obb}{\mathbb{O}}
\newcommand{\pbb}{\mathbb{P}}
\newcommand{\qbb}{\mathbb{Q}}
\newcommand{\rbb}{\mathbb{R}}
\newcommand{\sbb}{\mathbb{S}}
\newcommand{\tbb}{\mathbb{T}}
\newcommand{\ubb}{\mathbb{U}}
\newcommand{\vbb}{\mathbb{V}}
\newcommand{\wbb}{\mathbb{W}}
\newcommand{\xbb}{\mathbb{X}}
\newcommand{\ybb}{\mathbb{Y}}
\newcommand{\zbb}{\mathbb{Z}}


%===============================================================================
%=============================== SPECIAL SYMBOLS ===============================
%===============================================================================


% Orbit (group theory)
\newcommand{\orbit}{\mathcal{O}}
% Normal group
\newcommand{\normal}{\mathcal{N}}
% Indicator function
\newcommand{\indicator}{\mathbbm{1}}
% Laplace transform
\newcommand{\laplace}[1]{\mathcal{L}}
% Epsilon shorthand
\newcommand{\eps}{\varepsilon}
% Omega
\newcommand{\om}{\omega}
\newcommand{\Om}{\Omega}

%===============================================================================
%================================== OPERATORS ==================================
%===============================================================================


% Inverse exponent
\newcommand{\inv}[0]{^{-1}}
% Overline bar
\newcommand{\olsi}[1]{\,\overline{\!{#1}}}
% Less than or equal slanted
\newcommand{\seqs}{\leqslant}
% Greater or equal slanted
\newcommand{\geqs}{\geqslant}
% Subset or equal
\newcommand{\sub}{\subseteq}
% Proper subset
\newcommand{\prosub}{\subset}
% from
\newcommand{\from}{\leftarrow}

% Parantheses
\newcommand{\para}[1]{\left( #1 \right)}
% Curly Braces
\newcommand{\curl}[1]{\left\{ #1 \right\}}
% Brackets
\newcommand{\brac}[1]{\left[ #1 \right]}
% Angled Brackets
\newcommand{\ang}[1]{\left\langle #1 \right\rangle}
% Norm
\newcommand{\norm}[1]{\left\| #1 \right\|}

% Piece wise (use \\ between cases)
\newcommand{\piecewise}[1]{\begin{cases} #1 \end{cases}}

% Bold symbol shorthand
\newcommand{\bl}{\boldsymbol}

% Vertical space
\newcommand{\vs}[1]{\vspace{#1 pt}}
% Horizontal ertical space
\newcommand{\hs}[1]{\hspace{#1 pt}}


%===============================================================================
%============================== TEXT BASED SYMBOLS =============================
%===============================================================================


% Radians
\newcommand{\rad}{\text{rad}}
% Least Common Multiple
\newcommand{\lcm}{\text{lcm}}
% Automorphism
\newcommand{\Aut}{\text{Aut}}
% Variance
\newcommand{\var}{\text{Var}}
% Covariance
\newcommand{\cov}{\text{Cov}}
% Cofactor (matrix)
\newcommand{\cof}{\text{Cof}}
% Adjugate (matrix)
\newcommand{\adj}{\text{Adj}}
% Trace (matrix)
\newcommand{\tr}{\text{tr}}
% Standard deviation
\newcommand{\std}{\text{Std}}
% Correlation coefficient
\newcommand{\corr}{\text{Corr}}
% Sign
\newcommand{\sign}{\text{sign}}

% And text
\newcommand{\AND}{\text{ and }}
% Or text 
\newcommand{\OR}{\text{ or }}
% For text 
\newcommand{\FOR}{\text{ for }}
% If text
\newcommand{\IF}{\text{ if }}
% When text
\newcommand{\WHEN}{\text{ when }}
% Where text
\newcommand{\WHERE}{\text{ where }}
% Then text
\newcommand{\THEN}{\text{ then }}
% Such that text
\newcommand{\SUCHTHAT}{\text{ such that }}

% 1st
\newcommand{\st}[1]{#1^{\text{st}}}
% 2nd
\newcommand{\nd}[1]{#1^{\text{nd}}}
% 3rd
\newcommand{\rd}[1]{#1^{\text{rd}}}
% nth
\newcommand{\nth}[1]{#1^{\text{th}}}


%===============================================================================
%========================= PROBABILITY AND STATISTICS ==========================
%===============================================================================


% Permutation
\newcommand{\perm}[2]{{}^{#1}\!P_{#2}}
% Combination
\newcommand{\comb}[2]{{}^{#1}C_{#2}}

% Baye's risk
\newcommand{\risk}[1]{\mathscr{R}_{#1}}
% Baye's optimal risk
\newcommand{\riskOptimal}[1]{\mathscr{R}_{#1}^*}
% Baye's empirical risk
\newcommand{\riskEmpirical}[2]{\hat{\mathscr{R}}_{#1}^{#2}}


%===============================================================================
%=================================== CALCULUS ==================================
%===============================================================================


% d over d derivative
\newcommand{\dd}[2]{\dfrac{d#1}{d#2}}
% partial d over d derivative
\newcommand{\partialdd}[2]{\dfrac{\partial #1}{\partial #2}}
% delta d over d derivative
\newcommand{\deltadd}[2]{\dfrac{\Delta #1}{\Delta #2}}

% Integration between a and b
\newcommand{\integral}[4]{\int_{#1}^{#2} #3 \, #4}
% Integration in some space 
\newcommand{\boundIntegral}[2]{\int_{#1} #2 \, d#1}

% Limit
\newcommand{\limit}[3]{\lim_{#1 \to #2} #3}


%===============================================================================
%================================  BIG SYMBOLS  ================================
%===============================================================================


% Sum
\newcommand{\sumof}[2]{\sum_{#1}^{#2}}
% Product
\newcommand{\productof}[2]{\prod_{#1}^{#2}}
% Union
\newcommand{\unionof}[2]{\bigcup_{#1}^{#2}}
% Intersection
\newcommand{\intersectionof}[2]{\bigcap_{#1}^{#2}}
% Or
\newcommand{\orof}[2]{\bigvee_{#1}^{#2}}
% And
\newcommand{\andof}[2]{\bigwedge_{#1}^{#2}}


%===============================================================================
%=============================== LINEAR ALGEBRA ================================
%===============================================================================


% Bold vector arrow
\newcommand{\bv}[1]{\vec{\mathbf{#1}}}

% Matrix or vector (use // for column, & for row) 
% with brackets
\newcommand{\bmat}[1]{\begin{bmatrix} #1 \end{bmatrix}}
% Matrix or vector (use // for column, & for row) 
% with curved brackets
\newcommand{\pmat}[1]{\begin{pmatrix} #1 \end{pmatrix}}
% Matrix or vector (use // for column, & for row) 
% with lines on either side 
\newcommand{\lmat}[1]{\begin{vmatrix} #1 \end{vmatrix}}
% Matrix or vector (use // for column, & for row) 
% with curly braces
\newcommand{\cmat}[1]{\begin{Bmatrix} #1 \end{Bmatrix}}
% Matrix or vector (use // for column, & for row) 
% with no braces
\newcommand{\mat}[1]{\begin{matrix} #1 \end{matrix}}


%===============================================================================
%================================ LARGE OBJECTS ================================
%===============================================================================

% Multiple lines
\newcommand{\multiline}[1]{
\begin{align*}
    #1
\end{align*}
}

% Multiple lines with equation numbers
\newcommand{\eqmultiline}[1]{
\begin{align*}
    #1
\end{align*}
}

% Color
\newcommand{\colorText}[2]{
\begingroup
\color{#1}
    #2
\endgroup
}

% Centered figure
\newcommand{\centerFigure}[2]{
    \begin{figure}[h]
        \centering
            #1
        \caption{#2}
    \end{figure}
}

% Tikz figure
\newcommand{\tikzGraphic}[1]{
    \begin{center}
    \begin{tikzpicture}
        #1
    \end{tikzpicture}
    \end{center}
}

% Enumerate numbers (seperate by \item)
\newcommand{\numbers}{\textbf{\number*)}}

% Enumerate letters (seperate by \item)
\newcommand{\letters}{\textbf{\alph*)}}

\title{%
    \Huge Machine Learning \\
    \Large Lecture XVI
}
\date{2025-06-30}
\author{Michael Saba}

\begin{document}
\pagenumbering{gobble}
\maketitle
\newpage
\setlength{\parindent}{0pt}
\pagenumbering{arabic}

\section*{Principal Component Analysis}
\subsection*{Principal Component Analysis}

Dimensionality reduction, like clustering, is
a learning technqiue used in order to
discover the intrinsic structure of the data. \\

Dimensionality reduction is abotu constructing a map:
\[ \psi: \xcal \to \rbb^m \]
where $m$ is a much smaller dimension than that of $\xcal$,
the inputs. \\
The mapping should preserve the structure of the data,
that is, distances and relationships. \\

The point behind doing that is data compression;
it makes the machine learning process converge faster,
when the dimension is lower. \\

One method we've already seen that does dimensionality
reduction, in the supervised learning paradigm,
is Linear Discriminant Analysis, or LDA. \\
It projects the data onto a low-dimensional subspace
where the data is seperated. \\

However, in unsupervised learning, we use different
techniques to reduce the dimensionality of the data.
One such method is Principal Component Anlysis. \\

Suppose we have a dataset:
\[ S = \{ x_1, x_2, \dots, x_n \} \]
Which have a dimension of $d_x$. \\
Our goal is to approximate $S$ in an affine subspace
$\ucal$ of dimension $m$ with a basis:
\[ \{ u_1, u_2, \dots, u_m \} \]
Where each vector $u_i \in \rbb^{d_x}$. \\
We then define the matrix:
\[ X = (x_1, x_2, \dots x_n) \in \rbb^{d_x \times n} \]
And the matrix:
\[ U = (u_1, u_2, \dots u_m) \in \rbb^{d_x \times m} \]
Which is orthogonal as $U^TU = I$. \\
Points on $\ucal$ are expressed as linear combinations
of the basis vectors $u_i$: $U\alpha$,
where $\alpha$ is a coefficient vector
(the point expressed in the basis $U$). \\
We want to express all the points $x_i$ in the
linear subspace $\ucal$; so we will find the closest
point $\alpha$ in $\ucal$ to $x_i$ (project),
where $\alpha$ is an $m$-dimensional point. \\
We then collect all these points in a matrix
$A \in \rbb^{m \times n}$. \\

This yields an optimization problem
where we want to minimize the distance between our
original points and each point $\alpha$ in $\ucal$,
in order to get the projections we want:
\[ \min_{U, A} \frac{1}{n} \| X - UA \|_F^2 \quad 
\text{ such that } \quad U^\top U = I \]
Where $\|.\|^2_F$ is the Euclidian norm extended to
matrices:
\[ \|X\|^2_F = \tr(X^TX) = \sum_{i=1}^n \|x_i\|^2 \]
It's just the sum of the column norms squared. \\

Now, this optimization problem is not convex;
however, if we fix one of the variables, $U$ 
it does become convex in the other problem $A$. \\
This is easy to see; when $U$ is fixed, we
just have the minimum of a quadratic. \\
The solution is $A = U^TX$. \\

So we have:
\[ UA = UU^TX \]
Where $UA$ is the projection of the data points onto
$\ucal$, as expected. \\

We can now substiute the optimal $UA$ into the
problem:
\[ \min_{U} \frac{1}{n} \| X - UU^TX \|_F^2 \quad 
\text{ such that } \quad U^\top U = I \]
We now just have a minimization problem in terms of $U$. \\
We can rewrite:
\multiline{
|X - UU^\top X\|_F^2 &= \|(I - UU^\top)X\|_F^2 = 
\tr\left(X^\top (I - UU^\top)^\top (I - UU^\top) X\right) \\
&= \|X\|_F^2 - \operatorname{tr}
\left(X^\top UU^\top X\right)
}
Which comes from the fact that:
\[ (I - UU^T)^2 = I - UU^T \]
So now we have:
\[ \min_{U} \|X\|_F^2 - \operatorname{tr}
\left(X^\top UU^\top X\right) \quad 
\text{ such that } \quad U^\top U = I \]
Which is equivalent to:
\[ \max_{U} \dfrac{1}{n}\tr\para{U^TXX^TU} \quad 
\text{ such that } \quad U^\top U = I \]
Which is solved by finding $U$ to be th eigenvectors
of the largest $m$ eigenvalues of:
\[ \hat{\Sigma} = \frac{1}{n} XX^T \]
These eigenvectors are the principal components. \\

\newpage


\subsection*{Kernel PCA}

Given a positive definite kernel:
\[ k: \xcal \times \xcal \to \rbb \]
We know kernels induce a feature map:
\[ \varphi: \xcal \to \hcal_k \]
So to perform PCA when using kernels,
we perform it in the Hilbert space $\hcal_k$. \\

This can be tricky, since $\hcal_k$
is a space of functions. \\
How can we define eigenvectors? How many
components are there (what are dimensions in the
Hilbert space)? \\

We know in regular PCA, the eigenvectors $u$
are in the linear subsbspace $\ucal$, and solve:
\[  \hat{\Sigma} u = \lambda u \iff
XX^T u = n \lambda u \iff
\sum_{i = 1}^n (x_i^Tu)x_i = n \lambda u \]
Here we know that the eigenvectors $u$ lie in the
span of the points (since $\ucal$ is a subspace 
of $\xcal$). \\ 
We want to adapt that to kernels. \\

Using the feature map:
\[ \varphi: \xcal \to \hcal_k \]
We now have:
\[ \hat{\Sigma}_\varphi =  \sum_{i = 1}^n \varphi(x_i)
\varphi(x_i)^T \]
Where all we did was replace $x$ by its feature map. \\
This yields the eigenvector problem:
\[ \sum_{i=1}^n \langle \varphi(x_i), 
u \rangle_{\mathcal{H}_k} \, \varphi(x_i) = n\lambda u \]
Since the eigenvectors $u$ lie in the
span of the points, which are mapped with
$\varphi$ here, we know that $u$ is in the space of 
$\varphi(x_i)$ for all points $x_i$. \\
So:
\[ u = \sum_{i = 1}^n \beta_i\varphi(x_i) \]
For some vector of coefficients $\beta \in \rbb^n$. \\

\colorText{red}{SEE SLIDES 13 and 14 WHERE WE SOLVE THIS.\\}

We end up with:
\[ K^TK \beta = n \lambda K^T \beta \]
Where $K$ is the gram matrix of the kernel $k$.
Here $\beta$ gives us $u$
(the coefficients of the sum of points). \\

So we have a generalized eigenvectors problem
with a matrix $K^TK$.
Notice that the format is very similar to 
regular PCA, which had $X$ instead. \\
It is solvable with numeric solvers. \\

If $K$ is positive-definite (not just positive 
semi-definite), $K$ is invertible. \\
If we assume that is the case, this reduces to:
\[ K \beta = n\lambda \beta \]
Which is solvable analytically. \\
Note that we only need $m$ eigenvectors, the
first $m$ terms

The $u$ vectors we get are our principal components. \\

One main advantage here is that it fits the points
$\varphi(x_i)$ to a linear subspace $\ucal$ in the
feature space. But since $\varphi$
may not be linear, $\ucal$
will not linear in the input space. \\
So we are projecting the points $x_i$ to a non-linear
subspace of a lower dimension. \\

\newpage

\subsection*{Laplacian Eigenmap}

Recall that Variant-II spectral clustering does
dimensionality reduction on the vertices:
\[ \varphi: \vcal \to \rbb^k \]
This is just a type of dimensionality reduction. \\

In fact, we can show that this is Kernel-PCA with
a special kernel. \\

\newpage

\subsection*{Generalizing PCA}

Problems of PCA include the fact that it can only
fit linear structures, is sensitive to outliers,
depends on the scaling of variables. \\
These can be alleviated by normalizing the features,
centering the data at the origin by shifting it
so that the mean is at $0$. \\

So we introduce Robust-PCA. \\

In standard PCA, we have an objective:
\[ \min_{U, A} \| X - UA \|_F^2 \quad 
\text{ such that } \quad U^\top U = I \]
We instead see k some $Z \in \rbb^{d_x \times n}$ 
such that:
\[ \min_Z \|X - Z\|^2_F \quad \text{ such that }
\quad \text{rank}(Z) \leq m \]
The solution of this objective is given by:
\[Z = \sum_{i=1}^m \sigma_i u_iv_i^T \]
Where $u_i$ and $v_i$ are left and right singular
vectors. \\

Now, in Robust-PCA, we have $Z$ and $S$,
where:
\[ X = Z + S \]
We assume that $Z$ has low rank, and $S$
can have a large magnitude, but has a sparse support
($S$ is the error of $Z$ in approximating $UA$). \\
This gives us the objective:
\[ \min_{Z, S} \|Z\|_* + \lambda \|S\|_1 \]
The one norm $\|S\|_1$ ensures sparsity
(sum of absolute values),
and the nuclear norm $\|Z\|_*$
is just the sum of singular vectors,
which is the result we derived earlier. \\

We can solve $\ell_2$ squared error penalized
Robust-PCA using a proximal-gradient method, 
with a smooth part and non-smooth, but prox-friendly
and convex part:
\[ \min_{Z, S} 
\underbrace{\| X - Z - S \|^2_F}_{L-\text{smooth}} \; + \; 
\underbrace{\rho \|Z\|_* 
+ \lambda \|S\|_1}_{\text{prox-friendly}} \]
Where $\rho, \lambda > 0$. \\ 

Another way to generalize this is using matrix 
factorization. We can write:
\[ \min_{U, A} \frac{1}{n} \| X - UA \|_F^2 
+ \lambda \rcal(U, A) \]
Where the orthogonality condition is handled by 
regularization. \\

We can interpret this as dictionary learning. \\

\newpage


\end{document}
