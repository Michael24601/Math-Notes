
\documentclass[12pt]{article}

\usepackage[margin=1in]{geometry}


%===============================================================================
%================================== PACKAGES ===================================
%===============================================================================

% For using float option H that places figures 
% exatcly where we want them
\usepackage{float}
% makes figure font bold
\usepackage{caption}
\captionsetup[figure]{labelfont=bf}
% For text generation
\usepackage{lipsum}
% For drawing
\usepackage{tikz}
% For smaller or equal sign and not divide sign
\usepackage{amssymb}
% For the diagonal fraction
\usepackage{xfrac}
% For enumerating exercise parts with letters instead of numbers
\usepackage{enumitem}
% For dfrac, which forces the fraction to be in display mode (large) e
% even in math mode (small)
\usepackage{amsmath}
% For degree sign
\usepackage{gensymb}
% For "\mathbb" macro
\usepackage{amsfonts}
% For positioning 
\usepackage{indentfirst}
\usetikzlibrary{shapes,positioning,fit,calc}
% for adjustwidth environment
\usepackage{changepage}
% for arrow on top
\usepackage{esvect}
% for mathbb 1
\usepackage{bbm}
% for mathsrc
\usepackage[mathscr]{eucal}
% For degree sign
\usepackage{gensymb}
% For quotes
\usepackage{csquotes}
% For vertical lines
\usepackage{mathtools}
% For cols
\usepackage{multicol}

% for tikz
\usepackage{pgfplots}
\pgfplotsset{compat=1.18}
\usepackage{amsmath}
\usepgfplotslibrary{groupplots}


%===============================================================================
%==================================== FONTS ====================================
%===============================================================================


% Mathcal
\newcommand{\acal}{\mathcal{A}}
\newcommand{\bcal}{\mathcal{B}}
\newcommand{\ccal}{\mathcal{C}}
\newcommand{\dcal}{\mathcal{D}}
\newcommand{\ecal}{\mathcal{E}}
\newcommand{\fcal}{\mathcal{F}}
\newcommand{\gcal}{\mathcal{G}}
\newcommand{\hcal}{\mathcal{H}}
\newcommand{\ical}{\mathcal{I}}
\newcommand{\jcal}{\mathcal{J}}
\newcommand{\kcal}{\mathcal{K}}
\newcommand{\lcal}{\mathcal{L}}
\newcommand{\mcal}{\mathcal{M}}
\newcommand{\ncal}{\mathcal{N}}
\newcommand{\ocal}{\mathcal{O}}
\newcommand{\pcal}{\mathcal{P}}
\newcommand{\qcal}{\mathcal{Q}}
\newcommand{\rcal}{\mathcal{R}}
\newcommand{\scal}{\mathcal{S}}
\newcommand{\tcal}{\mathcal{T}}
\newcommand{\ucal}{\mathcal{U}}
\newcommand{\vcal}{\mathcal{V}}
\newcommand{\wcal}{\mathcal{W}}
\newcommand{\xcal}{\mathcal{X}}
\newcommand{\ycal}{\mathcal{Y}}
\newcommand{\zcal}{\mathcal{Z}}

% Mathfrak
\newcommand{\afrak}{\mathfrak{A}}
\newcommand{\bfrak}{\mathfrak{B}}
\newcommand{\cfrak}{\mathfrak{C}}
\newcommand{\dfrak}{\mathfrak{D}}
\newcommand{\efrak}{\mathfrak{E}}
\newcommand{\ffrak}{\mathfrak{F}}
\newcommand{\gfrak}{\mathfrak{G}}
\newcommand{\hfrak}{\mathfrak{H}}
\newcommand{\ifrak}{\mathfrak{I}}
\newcommand{\jfrak}{\mathfrak{J}}
\newcommand{\kfrak}{\mathfrak{K}}
\newcommand{\lfrak}{\mathfrak{L}}
\newcommand{\mfrak}{\mathfrak{M}}
\newcommand{\nfrak}{\mathfrak{N}}
\newcommand{\ofrak}{\mathfrak{O}}
\newcommand{\pfrak}{\mathfrak{P}}
\newcommand{\qfrak}{\mathfrak{Q}}
\newcommand{\rfrak}{\mathfrak{R}}
\newcommand{\sfrak}{\mathfrak{S}}
\newcommand{\tfrak}{\mathfrak{T}}
\newcommand{\ufrak}{\mathfrak{U}}
\newcommand{\vfrak}{\mathfrak{V}}
\newcommand{\wfrak}{\mathfrak{W}}
\newcommand{\xfrak}{\mathfrak{X}}
\newcommand{\yfrak}{\mathfrak{Y}}
\newcommand{\zfrak}{\mathfrak{Z}}

% Mathscr
\newcommand{\ascr}{\mathscr{A}}
\newcommand{\bscr}{\mathscr{B}}
\newcommand{\cscr}{\mathscr{C}}
\newcommand{\dscr}{\mathscr{D}}
\newcommand{\escr}{\mathscr{E}}
\newcommand{\fscr}{\mathscr{F}}
\newcommand{\gscr}{\mathscr{G}}
\newcommand{\hscr}{\mathscr{H}}
\newcommand{\iscr}{\mathscr{I}}
\newcommand{\jscr}{\mathscr{J}}
\newcommand{\kscr}{\mathscr{K}}
\newcommand{\lscr}{\mathscr{L}}
\newcommand{\mscr}{\mathscr{M}}
\newcommand{\nscr}{\mathscr{N}}
\newcommand{\oscr}{\mathscr{O}}
\newcommand{\pscr}{\mathscr{P}}
\newcommand{\qscr}{\mathscr{Q}}
\newcommand{\rscr}{\mathscr{R}}
\newcommand{\sscr}{\mathscr{S}}
\newcommand{\tscr}{\mathscr{T}}
\newcommand{\uscr}{\mathscr{U}}
\newcommand{\vscr}{\mathscr{V}}
\newcommand{\wscr}{\mathscr{W}}
\newcommand{\xscr}{\mathscr{X}}
\newcommand{\yscr}{\mathscr{Y}}
\newcommand{\zscr}{\mathscr{Z}}

% Mathbb
\newcommand{\abb}{\mathbb{A}}
\newcommand{\bbb}{\mathbb{B}}
\newcommand{\cbb}{\mathbb{C}}
\newcommand{\dbb}{\mathbb{D}}
\newcommand{\ebb}{\mathbb{E}}
\newcommand{\fbb}{\mathbb{F}}
\newcommand{\gbb}{\mathbb{G}}
\newcommand{\hbb}{\mathbb{H}}
\newcommand{\ibb}{\mathbb{I}}
\newcommand{\jbb}{\mathbb{J}}
\newcommand{\kbb}{\mathbb{K}}
\newcommand{\lbb}{\mathbb{L}}
\newcommand{\mbb}{\mathbb{M}}
\newcommand{\nbb}{\mathbb{N}}
\newcommand{\obb}{\mathbb{O}}
\newcommand{\pbb}{\mathbb{P}}
\newcommand{\qbb}{\mathbb{Q}}
\newcommand{\rbb}{\mathbb{R}}
\newcommand{\sbb}{\mathbb{S}}
\newcommand{\tbb}{\mathbb{T}}
\newcommand{\ubb}{\mathbb{U}}
\newcommand{\vbb}{\mathbb{V}}
\newcommand{\wbb}{\mathbb{W}}
\newcommand{\xbb}{\mathbb{X}}
\newcommand{\ybb}{\mathbb{Y}}
\newcommand{\zbb}{\mathbb{Z}}


%===============================================================================
%=============================== SPECIAL SYMBOLS ===============================
%===============================================================================


% Orbit (group theory)
\newcommand{\orbit}{\mathcal{O}}
% Normal group
\newcommand{\normal}{\mathcal{N}}
% Indicator function
\newcommand{\indicator}{\mathbbm{1}}
% Laplace transform
\newcommand{\laplace}[1]{\mathcal{L}}
% Epsilon shorthand
\newcommand{\eps}{\varepsilon}
% Omega
\newcommand{\om}{\omega}
\newcommand{\Om}{\Omega}

%===============================================================================
%================================== OPERATORS ==================================
%===============================================================================


% Inverse exponent
\newcommand{\inv}[0]{^{-1}}
% Overline bar
\newcommand{\olsi}[1]{\,\overline{\!{#1}}}
% Less than or equal slanted
\newcommand{\seqs}{\leqslant}
% Greater or equal slanted
\newcommand{\geqs}{\geqslant}
% Subset or equal
\newcommand{\sub}{\subseteq}
% Proper subset
\newcommand{\prosub}{\subset}
% from
\newcommand{\from}{\leftarrow}

% Parantheses
\newcommand{\para}[1]{\left( #1 \right)}
% Curly Braces
\newcommand{\curl}[1]{\left\{ #1 \right\}}
% Brackets
\newcommand{\brac}[1]{\left[ #1 \right]}
% Angled Brackets
\newcommand{\ang}[1]{\left\langle #1 \right\rangle}
% Norm
\newcommand{\norm}[1]{\left\| #1 \right\|}

% Piece wise (use \\ between cases)
\newcommand{\piecewise}[1]{\begin{cases} #1 \end{cases}}

% Bold symbol shorthand
\newcommand{\bl}{\boldsymbol}

% Vertical space
\newcommand{\vs}[1]{\vspace{#1 pt}}
% Horizontal ertical space
\newcommand{\hs}[1]{\hspace{#1 pt}}


%===============================================================================
%============================== TEXT BASED SYMBOLS =============================
%===============================================================================


% Radians
\newcommand{\rad}{\text{rad}}
% Least Common Multiple
\newcommand{\lcm}{\text{lcm}}
% Automorphism
\newcommand{\Aut}{\text{Aut}}
% Variance
\newcommand{\var}{\text{Var}}
% Covariance
\newcommand{\cov}{\text{Cov}}
% Cofactor (matrix)
\newcommand{\cof}{\text{Cof}}
% Adjugate (matrix)
\newcommand{\adj}{\text{Adj}}
% Trace (matrix)
\newcommand{\tr}{\text{tr}}
% Standard deviation
\newcommand{\std}{\text{Std}}
% Correlation coefficient
\newcommand{\corr}{\text{Corr}}
% Sign
\newcommand{\sign}{\text{sign}}

% And text
\newcommand{\AND}{\text{ and }}
% Or text 
\newcommand{\OR}{\text{ or }}
% If text
\newcommand{\IF}{\text{ if }}
% When text
\newcommand{\WHEN}{\text{ when }}
% Then text
\newcommand{\THEN}{\text{ then }}

% 1st
\newcommand{\st}[1]{#1^{\text{st}}}
% 2nd
\newcommand{\nd}[1]{#1^{\text{nd}}}
% 3rd
\newcommand{\rd}[1]{#1^{\text{rd}}}
% nth
\newcommand{\nth}[1]{#1^{\text{th}}}


%===============================================================================
%========================= PROBABILITY AND STATISTICS ==========================
%===============================================================================


% Permutation
\newcommand{\perm}[2]{{}^{#1}\!P_{#2}}
% Combination
\newcommand{\comb}[2]{{}^{#1}C_{#2}}

% Baye's risk
\newcommand{\risk}[1]{\mathscr{R}_{#1}}
% Baye's optimal risk
\newcommand{\riskOptimal}[1]{\mathscr{R}_{#1}^*}
% Baye's empirical risk
\newcommand{\riskEmpirical}[2]{\hat{\mathscr{R}}_{#1}^{#2}}


%===============================================================================
%=================================== CALCULUS ==================================
%===============================================================================


% d over d derivative
\newcommand{\dd}[2]{\dfrac{d#1}{d#2}}
% partial d over d derivative
\newcommand{\partialdd}[2]{\dfrac{\partial #1}{\partial #2}}
% delta d over d derivative
\newcommand{\deltadd}[2]{\dfrac{\Delta #1}{\Delta #2}}

% Integration between a and b
\newcommand{\integral}[4]{\int_{#1}^{#2} #3 \, #4}
% Integration in some space 
\newcommand{\boundIntegral}[2]{\int_{#1} #2 \, d#1}

% Limit
\newcommand{\limit}[3]{\lim_{#1 \to #2} #3}


%===============================================================================
%================================  BIG SYMBOLS  ================================
%===============================================================================


% Sum
\newcommand{\sumof}[2]{\sum_{#1}^{#2}}
% Product
\newcommand{\productof}[2]{\prod_{#1}^{#2}}
% Union
\newcommand{\unionof}[2]{\bigcup_{#1}^{#2}}
% Intersection
\newcommand{\intersectionof}[2]{\bigcap_{#1}^{#2}}
% Or
\newcommand{\orof}[2]{\bigvee_{#1}^{#2}}
% And
\newcommand{\andof}[2]{\bigwedge_{#1}^{#2}}


%===============================================================================
%=============================== LINEAR ALGEBRA ================================
%===============================================================================


% Bold vector arrow
\newcommand{\bv}[1]{\vec{\mathbf{#1}}}

% Matrix or vector (use // for column, & for row) 
% with brackets
\newcommand{\bmat}[1]{\begin{bmatrix} #1 \end{bmatrix}}
% Matrix or vector (use // for column, & for row) 
% with curved brackets
\newcommand{\pmat}[1]{\begin{pmatrix} #1 \end{pmatrix}}
% Matrix or vector (use // for column, & for row) 
% with lines on either side 
\newcommand{\lmat}[1]{\begin{vmatrix} #1 \end{vmatrix}}
% Matrix or vector (use // for column, & for row) 
% with curly braces
\newcommand{\cmat}[1]{\begin{Bmatrix} #1 \end{Bmatrix}}
% Matrix or vector (use // for column, & for row) 
% with no braces
\newcommand{\mat}[1]{\begin{matrix} #1 \end{matrix}}


%===============================================================================
%================================ LARGE OBJECTS ================================
%===============================================================================

% Multiple lines
\newcommand{\multiline}[1]{
\begin{align*}
    #1
\end{align*}
}

% Multiple lines with equation numbers
\newcommand{\eqmultiline}[1]{
\begin{align*}
    #1
\end{align*}
}

% Color
\newcommand{\colorText}[2]{
\begingroup
\color{#1}
    #2
\endgroup
}

% Centered figure
\newcommand{\centerFigure}[2]{
    \begin{figure}[h]
        \centering
            #1
        \caption{#2}
    \end{figure}
}

% Tikz figure
\newcommand{\tikzGraphic}[1]{
    \begin{center}
    \begin{tikzpicture}
        #1
    \end{tikzpicture}
    \end{center}
}

% Enumerate numbers (seperate by \item)
\newcommand{\numbers}{\textbf{\number*)}}

% Enumerate letters (seperate by \item)
\newcommand{\letters}{\textbf{\alph*)}}

\title{%
    \Huge Machine Learning \\
    \Large Lecture XVIII
}
\date{2025-07-07}
\author{Michael Saba}

\begin{document}
\pagenumbering{gobble}
\maketitle
\newpage
\setlength{\parindent}{0pt}
\pagenumbering{arabic}

\section*{Deep Learning}
\subsection*{Activation Functions}

There are multiple activation functions we can
use to output a number at each layer's neurons:
\[ a_{l} = \sigma_l(z_{l}) \]
Here $\sigma_l$ is the activation function 
on layer $l$. \\

The first one is the Sigmoid function:
\[ \sigma(x) = \dfrac{1}{1 + e^{-x}} \]
Which squashes the output into the $[0, 1]$
range. \\
The issue with it is that saturated neurons kill
the gradient, the output is not $0$ centered,
and the exponential is expensive to compute. \\

Then there is the tangent hyperbolic function:
\[ \sigma(x) = \tanh(x) \]
Which squashes the output into the $[-1, 1]$
range. \\
This still has the issue that saturated neurons kill
the gradient, but the output is $0$ centered. \\

The Rectified Linear Unit (ReLu) function:
\[ \sigma(x) = \max(0, x) \]
This function does not saturate for $x \in \rbb^+$. \\
It is very efficient to compute and converges quite
fast, but it is also no zero centered, 
and not differentiable at $x = 0$,
since $\sigma'(0; -1) = 0$ and $\sigma'(0; +1) = 1$. \\

The Leaky ReLU function:
\[ \sigma(x) = \max(0.01x, x) \]
This function does not saturate for any $x$. \\
It is vert efficient to compute and converges quite
fast, but it is also no zero centered, 
and not differentiable at $x = 0$,
since $\sigma'(0; -1) \neq \sigma'(0; +1)$. \\
Note that we can have any coefficient $\alpha x$;
we then call the activation function
Parametric Rectifier Liner Unit (PReLU).
Here $\alpha$ becomes another parameter that we can
optimize. \\

The Exponential Linear Unit (ELU) function:
\[ \sigma(x) = \piecewise{
    x \qquad & \IF x > 0 \\
    \alpha (e^{x}-1) \qquad & \IF x \leq 0 
} \]
It has all the benefis of ReLU,
but adds robustness to noise. \\
The main issue is that again, the computation
of the exponential is expensive. \\

There is also the MaxOut activation function which
we will see in the section about Convolutional
Neural Networks. \\


\newpage

\subsection*{Convolutional Neural Networks}

Convolutional Neural Networks, or CNNs,
are similar to Neural Networks, but are specialized
for image data. \\

Images are highly dimensional, which makes
the fully connected layers of the Neural Networks
(recall that the outputs of each neuron are
the onputs of each neuron in the next layer)
very inefficient. \\

CNNs solve this issue not being fully connected;
some layers of the CNNs has small receptive fields;
meaning that the neurons take as input the outputs
of only a few neurons from the previous layer,
corresponding to pixels in some region in tghe image. \\
Receptive field just means that neurons are affected
by the pixels of a local region. \\
Moreover there is weight sharing; meaning all neurons
in the layer share weights and biases. \\
These layers, called convolutions, are far
more efficient, and specialized, for learning
with images. \\

A kernel is filter, used to choose
which neuron outputs (pixels in the image)
are used as inputs for some neuron in the next layer. \\
The kernel is like a mask with weights;
image for example this $3\times3$ kernel:
\[ \begin{array}{|c|c|c|}
\hline
w_{-1,-1} & w_{-1,0} & w_{-1,1} \\
\hline
w_{0,-1} & w_{0,0} & w_{0,1} \\
\hline
w_{1,-1} & w_{1,0} & w_{1,1} \\
\hline
\end{array} \]
It only takes $9$ pixels as inputs,
and applies the weights specified by the kernel.
The same kernel is used by other neurons in the
layer. \\

For the $3 \times 3$ kernel, the computed 
value $z_{ij}$ in the layer $l$
will only depends on the $9$ values of pixels
in some region. \\
This is done by shifting the kernel along pixels
by the output indexes $i, j$:
\[ z_{ij} = \sum_{p=-1}^1\sum_{q=-1}^1 w_{p,q}
x_{i+1+p, j+1+q} = \ang{w, x_{i+1, j+1}} \]
Here the output layer is also a grid of pixels,
each corresponding to a neuron,
but smaller. \\
We are just sliding the kernel on the image;
note that because the radius of the kernel
is $1$ ($1$ pixel off the center),
so if we have a $10 \times 10$ input image,
the next layer will be $8 \times 8$
since we can't do the convolution step
to the edge pixels. \\

There are multiple types of filters we can
apply, identified by their size and weigths. \\
\begin{enumerate}[label = \numbers]
    \item 
    The identity kernel:
    \[ \begin{array}{|c|c|c|} \hline
    0 & 0 & 0 \\ \hline
    0 & 1 & 0 \\ \hline
    0 & 0 & 0 \\ \hline
    \end{array} \]
    This keeps the image intact (but removed
    the edge pixels as discussed before).
    \item 
    The $5\times 5$ Blur kernel:
    \[ \begin{array}{|c|c|c|c|c|} \hline
    0 & 0 & 0 & 0 & 0 \\ \hline
    0 & 1 & 1 & 1 & 0 \\ \hline
    0 & 1 & 1 & 1 & 0 \\ \hline
    0 & 1 & 1 & 1 & 0 \\ \hline
    0 & 0 & 0 & 0 & 0 \\ \hline
    \end{array} \]
    This just blurs the image by averaging values;
    note that we need to normalize the weights
    (divide by $9$),
    otherwise it becomes brighter.
    \item 
    The $5\times 5$ Sharpening kernel:
    \[ \begin{array}{|c|c|c|c|c|} \hline
    0 & 0 & 0 & 0 & 0 \\ \hline
    0 & 0 & -1 & 0 & 0 \\ \hline
    0 & -1 & 5 & -1 & 0 \\ \hline
    0 & 0 & -1 & 0 & 0 \\ \hline
    0 & 0 & 0 & 0 & 0 \\ \hline
    \end{array} \]
    Which does the opposite of a blur.
    \item 
    The vertical edge-detecting kernel:
    \[ \begin{array}{|c|c|c|} \hline
    0 & 0 & 0 \\ \hline
    -1 & 1 & 0 \\ \hline
    0 & 0 & 0 \\ \hline
    \end{array} \]
    \item 
    The all edge-detecting kernel:
    \[ \begin{array}{|c|c|c|} \hline
    0 & 1 & 0 \\ \hline
    1 & -4 & 1 \\ \hline
    0 & 1 & 0 \\ \hline
    \end{array} \]
\end{enumerate}
There are more. \\

Note that a $32 \times 32$ image is actually
$32 \times 32 \times 3$ dimensional since we
actually have $3$ color channels. \\
Same for kernels. \\ 

Sliding the kernel over the image and computing
the dot products is convolution. \\

In a CNN, we perform convolution on some layers,
and combine them with an activation function
layers (where we just apply the actiavtion function
to the output)... \\
We will also later see Max-Pooling layers. \\

To prevent the issue of kernels removing the edge
pixles, we can pad the image with $0$s around it. \\
This prevents the loss of edge pixels;
There are other boundary conditions we can use,
such as reflecting boundaries. \\
We can pad with as many pixels as needed
(the larger the filter the more pixels are needed). \\

Sometimes we actually want to reduce the dimension
of the image, so we don't have to pad the edges. \\
Likewise, if we want to reduce the dimension,
we can increase the stride; that means that
instead of shifting the kernel filter by $1$
pixel, we can shift it by more pixels. \\

A $2\times 2$ Max-Pooling filter with a $2$ stride
will essentially take $4$ squares of $4$ pixels,
that is, an $4 \times 4$ square of pixels,
and then turn it into a $2 \times 2$
square where we choose the largest value
from each $2 \times 2$ square. \\

Pooling layers are used, similarly to
convolution layers, but don't require an activation
layer afterwards. \\

We can imagine CNNs taking as input an array of pixels,
lowering the dimensionality such that the neurons
represent mid-level features (shapes, edges, ...),
then lowering the dinension again such that
we now have high level features (faces,
objects, ...), and then finally outputting
a prediction in the final layer. \\
Each layer aggregates the previous layer's data
into higher abstraction feature levels,
which are more complex, but also fewer;
allowing each layer to need to do less work. \\

This is what the kernel filters are doing;
taking pixels in a region and aggregating the
information, so that we now have higher-level
features;
a combination of pixels gives us edges and
simple shapes, a combination of shapes gives
us objects, organs, faces... \\

\newpage

\subsection*{Additional Discussions}

Unlike parameters, which are optimized in the
process of applying a machine learning algorithm,
hyperparameters have to be fine tuned
empiricially. \\
This includes things like the learning rate,
choice of layers... \\

If the dataset is too small, finetuning
is likely to lead to overfitting,
causing the neural network to generalize
poorly to new unseen data. \\
One way to fix this is to stop finetuning early,
and leaving some hyperparameters fixed. \\

Cross-dataset generalization can also be used
to check the overfitting of neural networks. \\
If the accuracy of tests on the training data
gets better and better, but that improvement does
not translate to testing data, it is a good sign
overfitting has taken place. \\

If the network overfits, we can decrease the number
of parameters, increase the number of training samples,
synthetically create new samples, and perfom 
data augmentation, which can be used to add variance
into the data. \\
For example, we can rotate some of our image,
which allows the neural network to learn rotation
invariance, which allows it to generalize better. \\
Other transformation include scaling, brightening,
blurring, and deforming the data. \\

Note that the transformations that can be used
depend on the task; if we want our network to
differentiate between red and green apples,
we can't shift colors in the hopes the network
becomes color invariant. \\

We can also try regularization,
such as Weight decay, for each layer, which
penalizes the value of the weights. \\
Weight decay is the sum of the norm squared
of the weights $\theta_i$. \\

We can also use dropout, where we probabilistically
remove some nodes from the network. 
It is a type of regularization. \\


\end{document}
