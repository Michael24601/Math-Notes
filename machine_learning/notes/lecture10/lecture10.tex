
\documentclass[12pt]{article}

\usepackage[margin=1in]{geometry}


%===============================================================================
%================================== PACKAGES ===================================
%===============================================================================

% For using float option H that places figures 
% exatcly where we want them
\usepackage{float}
% makes figure font bold
\usepackage{caption}
\captionsetup[figure]{labelfont=bf}
% For text generation
\usepackage{lipsum}
% For drawing
\usepackage{tikz}
% For smaller or equal sign and not divide sign
\usepackage{amssymb}
% For the diagonal fraction
\usepackage{xfrac}
% For enumerating exercise parts with letters instead of numbers
\usepackage{enumitem}
% For dfrac, which forces the fraction to be in display mode (large) e
% even in math mode (small)
\usepackage{amsmath}
% For degree sign
\usepackage{gensymb}
% For "\mathbb" macro
\usepackage{amsfonts}
% For positioning 
\usepackage{indentfirst}
\usetikzlibrary{shapes,positioning,fit,calc}
% for adjustwidth environment
\usepackage{changepage}
% for arrow on top
\usepackage{esvect}
% for mathbb 1
\usepackage{bbm}
% for mathsrc
\usepackage[mathscr]{eucal}
% For degree sign
\usepackage{gensymb}
% For quotes
\usepackage{csquotes}
% For vertical lines
\usepackage{mathtools}
% For cols
\usepackage{multicol}

% for tikz
\usepackage{pgfplots}
\pgfplotsset{compat=1.18}
\usepackage{amsmath}
\usepgfplotslibrary{groupplots}


%===============================================================================
%==================================== FONTS ====================================
%===============================================================================


% Mathcal
\newcommand{\acal}{\mathcal{A}}
\newcommand{\bcal}{\mathcal{B}}
\newcommand{\ccal}{\mathcal{C}}
\newcommand{\dcal}{\mathcal{D}}
\newcommand{\ecal}{\mathcal{E}}
\newcommand{\fcal}{\mathcal{F}}
\newcommand{\gcal}{\mathcal{G}}
\newcommand{\hcal}{\mathcal{H}}
\newcommand{\ical}{\mathcal{I}}
\newcommand{\jcal}{\mathcal{J}}
\newcommand{\kcal}{\mathcal{K}}
\newcommand{\lcal}{\mathcal{L}}
\newcommand{\mcal}{\mathcal{M}}
\newcommand{\ncal}{\mathcal{N}}
\newcommand{\ocal}{\mathcal{O}}
\newcommand{\pcal}{\mathcal{P}}
\newcommand{\qcal}{\mathcal{Q}}
\newcommand{\rcal}{\mathcal{R}}
\newcommand{\scal}{\mathcal{S}}
\newcommand{\tcal}{\mathcal{T}}
\newcommand{\ucal}{\mathcal{U}}
\newcommand{\vcal}{\mathcal{V}}
\newcommand{\wcal}{\mathcal{W}}
\newcommand{\xcal}{\mathcal{X}}
\newcommand{\ycal}{\mathcal{Y}}
\newcommand{\zcal}{\mathcal{Z}}

% Mathfrak
\newcommand{\afrak}{\mathfrak{A}}
\newcommand{\bfrak}{\mathfrak{B}}
\newcommand{\cfrak}{\mathfrak{C}}
\newcommand{\dfrak}{\mathfrak{D}}
\newcommand{\efrak}{\mathfrak{E}}
\newcommand{\ffrak}{\mathfrak{F}}
\newcommand{\gfrak}{\mathfrak{G}}
\newcommand{\hfrak}{\mathfrak{H}}
\newcommand{\ifrak}{\mathfrak{I}}
\newcommand{\jfrak}{\mathfrak{J}}
\newcommand{\kfrak}{\mathfrak{K}}
\newcommand{\lfrak}{\mathfrak{L}}
\newcommand{\mfrak}{\mathfrak{M}}
\newcommand{\nfrak}{\mathfrak{N}}
\newcommand{\ofrak}{\mathfrak{O}}
\newcommand{\pfrak}{\mathfrak{P}}
\newcommand{\qfrak}{\mathfrak{Q}}
\newcommand{\rfrak}{\mathfrak{R}}
\newcommand{\sfrak}{\mathfrak{S}}
\newcommand{\tfrak}{\mathfrak{T}}
\newcommand{\ufrak}{\mathfrak{U}}
\newcommand{\vfrak}{\mathfrak{V}}
\newcommand{\wfrak}{\mathfrak{W}}
\newcommand{\xfrak}{\mathfrak{X}}
\newcommand{\yfrak}{\mathfrak{Y}}
\newcommand{\zfrak}{\mathfrak{Z}}

% Mathscr
\newcommand{\ascr}{\mathscr{A}}
\newcommand{\bscr}{\mathscr{B}}
\newcommand{\cscr}{\mathscr{C}}
\newcommand{\dscr}{\mathscr{D}}
\newcommand{\escr}{\mathscr{E}}
\newcommand{\fscr}{\mathscr{F}}
\newcommand{\gscr}{\mathscr{G}}
\newcommand{\hscr}{\mathscr{H}}
\newcommand{\iscr}{\mathscr{I}}
\newcommand{\jscr}{\mathscr{J}}
\newcommand{\kscr}{\mathscr{K}}
\newcommand{\lscr}{\mathscr{L}}
\newcommand{\mscr}{\mathscr{M}}
\newcommand{\nscr}{\mathscr{N}}
\newcommand{\oscr}{\mathscr{O}}
\newcommand{\pscr}{\mathscr{P}}
\newcommand{\qscr}{\mathscr{Q}}
\newcommand{\rscr}{\mathscr{R}}
\newcommand{\sscr}{\mathscr{S}}
\newcommand{\tscr}{\mathscr{T}}
\newcommand{\uscr}{\mathscr{U}}
\newcommand{\vscr}{\mathscr{V}}
\newcommand{\wscr}{\mathscr{W}}
\newcommand{\xscr}{\mathscr{X}}
\newcommand{\yscr}{\mathscr{Y}}
\newcommand{\zscr}{\mathscr{Z}}

% Mathbb
\newcommand{\abb}{\mathbb{A}}
\newcommand{\bbb}{\mathbb{B}}
\newcommand{\cbb}{\mathbb{C}}
\newcommand{\dbb}{\mathbb{D}}
\newcommand{\ebb}{\mathbb{E}}
\newcommand{\fbb}{\mathbb{F}}
\newcommand{\gbb}{\mathbb{G}}
\newcommand{\hbb}{\mathbb{H}}
\newcommand{\ibb}{\mathbb{I}}
\newcommand{\jbb}{\mathbb{J}}
\newcommand{\kbb}{\mathbb{K}}
\newcommand{\lbb}{\mathbb{L}}
\newcommand{\mbb}{\mathbb{M}}
\newcommand{\nbb}{\mathbb{N}}
\newcommand{\obb}{\mathbb{O}}
\newcommand{\pbb}{\mathbb{P}}
\newcommand{\qbb}{\mathbb{Q}}
\newcommand{\rbb}{\mathbb{R}}
\newcommand{\sbb}{\mathbb{S}}
\newcommand{\tbb}{\mathbb{T}}
\newcommand{\ubb}{\mathbb{U}}
\newcommand{\vbb}{\mathbb{V}}
\newcommand{\wbb}{\mathbb{W}}
\newcommand{\xbb}{\mathbb{X}}
\newcommand{\ybb}{\mathbb{Y}}
\newcommand{\zbb}{\mathbb{Z}}


%===============================================================================
%=============================== SPECIAL SYMBOLS ===============================
%===============================================================================


% Orbit (group theory)
\newcommand{\orbit}{\mathcal{O}}
% Normal group
\newcommand{\normal}{\mathcal{N}}
% Indicator function
\newcommand{\indicator}{\mathbbm{1}}
% Laplace transform
\newcommand{\laplace}[1]{\mathcal{L}}
% Epsilon shorthand
\newcommand{\eps}{\varepsilon}
% Omega
\newcommand{\om}{\omega}
\newcommand{\Om}{\Omega}

%===============================================================================
%================================== OPERATORS ==================================
%===============================================================================


% Inverse exponent
\newcommand{\inv}[0]{^{-1}}
% Overline bar
\newcommand{\olsi}[1]{\,\overline{\!{#1}}}
% Less than or equal slanted
\newcommand{\seqs}{\leqslant}
% Greater or equal slanted
\newcommand{\geqs}{\geqslant}
% Subset or equal
\newcommand{\sub}{\subseteq}
% Proper subset
\newcommand{\prosub}{\subset}
% from
\newcommand{\from}{\leftarrow}

% Parantheses
\newcommand{\para}[1]{\left( #1 \right)}
% Curly Braces
\newcommand{\curl}[1]{\left\{ #1 \right\}}
% Brackets
\newcommand{\brac}[1]{\left[ #1 \right]}
% Angled Brackets
\newcommand{\ang}[1]{\left\langle #1 \right\rangle}
% Norm
\newcommand{\norm}[1]{\left\| #1 \right\|}

% Piece wise (use \\ between cases)
\newcommand{\piecewise}[1]{\begin{cases} #1 \end{cases}}

% Bold symbol shorthand
\newcommand{\bl}{\boldsymbol}

% Vertical space
\newcommand{\vs}[1]{\vspace{#1 pt}}
% Horizontal ertical space
\newcommand{\hs}[1]{\hspace{#1 pt}}


%===============================================================================
%============================== TEXT BASED SYMBOLS =============================
%===============================================================================


% Radians
\newcommand{\rad}{\text{rad}}
% Least Common Multiple
\newcommand{\lcm}{\text{lcm}}
% Automorphism
\newcommand{\Aut}{\text{Aut}}
% Variance
\newcommand{\var}{\text{Var}}
% Covariance
\newcommand{\cov}{\text{Cov}}
% Cofactor (matrix)
\newcommand{\cof}{\text{Cof}}
% Adjugate (matrix)
\newcommand{\adj}{\text{Adj}}
% Trace (matrix)
\newcommand{\tr}{\text{tr}}
% Standard deviation
\newcommand{\std}{\text{Std}}
% Correlation coefficient
\newcommand{\corr}{\text{Corr}}
% Sign
\newcommand{\sign}{\text{sign}}

% And text
\newcommand{\AND}{\text{ and }}
% Or text 
\newcommand{\OR}{\text{ or }}
% If text
\newcommand{\IF}{\text{ if }}
% When text
\newcommand{\WHEN}{\text{ when }}
% Then text
\newcommand{\THEN}{\text{ then }}

% 1st
\newcommand{\st}[1]{#1^{\text{st}}}
% 2nd
\newcommand{\nd}[1]{#1^{\text{nd}}}
% 3rd
\newcommand{\rd}[1]{#1^{\text{rd}}}
% nth
\newcommand{\nth}[1]{#1^{\text{th}}}


%===============================================================================
%========================= PROBABILITY AND STATISTICS ==========================
%===============================================================================


% Permutation
\newcommand{\perm}[2]{{}^{#1}\!P_{#2}}
% Combination
\newcommand{\comb}[2]{{}^{#1}C_{#2}}

% Baye's risk
\newcommand{\risk}[1]{\mathscr{R}_{#1}}
% Baye's optimal risk
\newcommand{\riskOptimal}[1]{\mathscr{R}_{#1}^*}
% Baye's empirical risk
\newcommand{\riskEmpirical}[2]{\hat{\mathscr{R}}_{#1}^{#2}}


%===============================================================================
%=================================== CALCULUS ==================================
%===============================================================================


% d over d derivative
\newcommand{\dd}[2]{\dfrac{d#1}{d#2}}
% partial d over d derivative
\newcommand{\partialdd}[2]{\dfrac{\partial #1}{\partial #2}}
% delta d over d derivative
\newcommand{\deltadd}[2]{\dfrac{\Delta #1}{\Delta #2}}

% Integration between a and b
\newcommand{\integral}[4]{\int_{#1}^{#2} #3 \, #4}
% Integration in some space 
\newcommand{\boundIntegral}[2]{\int_{#1} #2 \, d#1}

% Limit
\newcommand{\limit}[3]{\lim_{#1 \to #2} #3}


%===============================================================================
%================================  BIG SYMBOLS  ================================
%===============================================================================


% Sum
\newcommand{\sumof}[2]{\sum_{#1}^{#2}}
% Product
\newcommand{\productof}[2]{\prod_{#1}^{#2}}
% Union
\newcommand{\unionof}[2]{\bigcup_{#1}^{#2}}
% Intersection
\newcommand{\intersectionof}[2]{\bigcap_{#1}^{#2}}
% Or
\newcommand{\orof}[2]{\bigvee_{#1}^{#2}}
% And
\newcommand{\andof}[2]{\bigwedge_{#1}^{#2}}


%===============================================================================
%=============================== LINEAR ALGEBRA ================================
%===============================================================================


% Bold vector arrow
\newcommand{\bv}[1]{\vec{\mathbf{#1}}}

% Matrix or vector (use // for column, & for row) 
% with brackets
\newcommand{\bmat}[1]{\begin{bmatrix} #1 \end{bmatrix}}
% Matrix or vector (use // for column, & for row) 
% with curved brackets
\newcommand{\pmat}[1]{\begin{pmatrix} #1 \end{pmatrix}}
% Matrix or vector (use // for column, & for row) 
% with lines on either side 
\newcommand{\lmat}[1]{\begin{vmatrix} #1 \end{vmatrix}}
% Matrix or vector (use // for column, & for row) 
% with curly braces
\newcommand{\cmat}[1]{\begin{Bmatrix} #1 \end{Bmatrix}}
% Matrix or vector (use // for column, & for row) 
% with no braces
\newcommand{\mat}[1]{\begin{matrix} #1 \end{matrix}}


%===============================================================================
%================================ LARGE OBJECTS ================================
%===============================================================================

% Multiple lines
\newcommand{\multiline}[1]{
\begin{align*}
    #1
\end{align*}
}

% Multiple lines with equation numbers
\newcommand{\eqmultiline}[1]{
\begin{align*}
    #1
\end{align*}
}

% Color
\newcommand{\colorText}[2]{
\begingroup
\color{#1}
    #2
\endgroup
}

% Centered figure
\newcommand{\centerFigure}[2]{
    \begin{figure}[h]
        \centering
            #1
        \caption{#2}
    \end{figure}
}

% Tikz figure
\newcommand{\tikzGraphic}[1]{
    \begin{center}
    \begin{tikzpicture}
        #1
    \end{tikzpicture}
    \end{center}
}

% Enumerate numbers (seperate by \item)
\newcommand{\numbers}{\textbf{\number*)}}

% Enumerate letters (seperate by \item)
\newcommand{\letters}{\textbf{\alph*)}}

\title{%
    \Huge Machine Learning \\
    \Large Lecture X, XI
}
\date{2025-06-02}
\author{Michael Saba}

\begin{document}
\pagenumbering{gobble}
\maketitle
\newpage
\setlength{\parindent}{0pt}
\pagenumbering{arabic}

\section*{Convex Optimization}
\subsection*{Convexity}

We've thus far seen multiple classification and regression
problems, including linear classification,
linear classification with ridge or LASSO regression,
Linear Discriminant Analysis, Logistic Regression,
and Support Vector Machine with hard and soft margins. \\

Note that we can also use least 1-norm minimization
where we use the 1-norm instead of the 2-norm 
for the empirical risk, which penalizes large
errors and small errors proportionally. \\

Sometimes there exists a closed form expression for 
the optimal solution, but other times we ar forced to use
numerical algorithms. \\

Recall that a convex function is a function:
\[ f: \rbb^d \to \rbb \]
such that for all $x, y \in \rbb^d$ and $\lambda \in [0, 1]$:
\[ f((1-\lambda)x + \lambda y) 
\leq (1-\lambda)f(x) + \lambda f(y)  \]
A function being convex implies that any local minimum
is a global minimum. \\

The pointwise supremum (minimum or maximum) of two 
convex functions, their sum, and non-negative
scaling of a convex function all yield a convex function. \\

One way to check that a function is convex
is to look at the second derivative.
In $\rbb$, that means that
$f''(x) > 0$ implies the function is convex. \\
More generally, in $\rbb^d$, 
a function is convex if the hessian $\nabla^2 f$
is twice differentiable. \\
This check only works if the function 
is twice-differentiable. \\

If it's only once-differentiable, we can also
check that:
\[ \ang{x_1 - x_0, \nabla f(x_1) \nabla f(x_0)} \geq 0 \]
which means the first derivative is increasing
(means the second derivative is positive). \\

A function is convex if it is convex on all line segments
(each one dimensional function). \\

A line segment between $x$ and $y$ is defined by:
\[ [x, y] = \{(1-\lambda)x + \lambda y 
\mid \lambda \in [0, 1] \} \]
We can also define half-open and open line segments
$[x, y)$, $(x, y]$, and $(x, y)$. \\

We say a set $C$ is a Convex Set if for any two
points $x$ and $y$ in the set, the line segment
$[x, y]$ is on the set:
\[ \forall x, y \in C \quad [x, y] \subset C \]
For example, the area defined by $x^2 - y \leq 0$
is a convex set. \\

\newpage

\subsection*{Optimization}

A convex optimization problem (including all problems
listed in the last section) looks like:
\[ \min_{\theta \in \Theta} f(\theta) \]
Where we try to find the optimal parameters $\theta$.
We call $\theta$ the optimization variable. \\

The problem being a convex optimization problem
assume that the objective (function) $f$:
\[ f: \rbb^d \to \rbb \]
is convex.
It also assumes that the constraint (parameter space)
$\Theta \in \rbb^d$ is a convex set. \\

An optimal solution $\theta^*$ such that
\[ f(\theta^*) = \inf_{\theta \in \Theta} f(\theta) \]
may or may not exist. 
Note that the reason we use infimum is that the
minimum may not exist (for example, $e^x$
doesn't have a minimum, but the infimum is $0$). \\
The solution may or may not be unique. \\
We need to know how to check for optimality. \\
There may or may not be a 
closed form expression of the solution $\theta^*$. \\
We also need to know what randomness the solution depends
on (the noise, the data...). \\

There are several classes of optimization problems:
\begin{enumerate}
    \item Linear Programming: The function $f$
    is linear and the constraint $\Theta$
    is given by a linear system $A\theta = b$.
    \item Non-Linear Programming: One of the conditions
    of linear programming fails.
    \item Unconstrained optimization, where the parameters
    can be any numbers $\Theta = \rbb^d$. 
    \item Smooth optimization: We can characterize classes
    by whether they are differentiable (smooth),
    including higher order of differentiability.
    \item Non-smooth optimization: The function is not
    smooth on its entire domain.
    \item Structure optimization.
\end{enumerate}
It is often easier to study them and try to solve them
individually. \\

If there exists a solution to a convex optimization
problem, but no closed-form expression,
then we will have to use an iterative numerical algorithm
that converges towards the solution as the number of
iterations increases. \\

We use an interative algorithm $\acal$
to solve problems from some 
(family) of optimization problems $\cscr$. \\
We expect that $\acal$ solves all problems $\pcal \in \cscr$. \\
The algorithm constructs an improving sequence
$(\theta^{(k)})_{k \in \nbb}$ that converges towards
the solution $\theta^*$ of $\pcal$. \\
The class of problems $\cscr$ is the set of rules
that the problems $\pcal$ follow. \\
The algorithm as access to some information about $\pcal$
at any point $\olsi{\theta}$;
this is called an orcal. For example,
$\acal$ can have access to $f(\olsi{\theta})$,
the derivative $\nabla f(\olsi{\theta})$,
and the second derivative $\nabla^2 f(\olsi{\theta})$.
A zero order algorithm only uses the function value, 
a first order algorithm only uses the first order
derivative... \\

We say that $\acal$ solves $\pcal$ 
if $(\theta^{(k)})_{k \in \nbb}$
converges towards the solution as $k \to \infty$. \\
We are often interested in the convergence rate
of a given algortithm. \\

An $\eps$-solution is a solution that is at
most $\eps$ off from the true global minimizer:
\[ \olsi{\theta} \leq \theta^* + \eps \]
This can be used to induce the convergence rate. \\

We can only really compare the performance of algorithms
that follow the same rules (solve problems from the
same class). \\

A zero-order algorithm is not expensive at all,
but is also inaccurate and slow to converge. Second-order
algorithms are faster and better but also more expensive
to compure. First-order algorithms are a good
tradeoff between the two. \\

\newpage

\subsection*{Convex Quadratic Minimization}

The class of Convex Quadratic Minimization problems
all look like:
\[ \min_{\theta \in \rbb^d} f(\theta)
\quad f(\theta) = \|\Phi \theta - y\|^2 \]
This class includes linear regression, regularized linear 
regression with ridge regression, and linear
discriminant analysis. \\

Notice that there is no constraint on the 
optimization variable $\theta$,
so this is an unconstrained optimization problem. \\

We will call:
\[ L = \|\Phi^T \Phi\| \quad \AND
\quad m = \lambda_{\min}(\Phi^T\Phi) \geq 0 \]
Which are respecively the largest and smallest
eigenvalues of $\Phi^T\Phi$. \\

There are several algorithms that we can use to solve
problems of this class, including
a direct solver using a closed form epxression (as we
have seen before), gradient descent, and conjugate
gradient method. \\

The direct solver has a closed form solution
as we saw in the last two chapters. \\

\newpage

\subsection*{Gradient Descent}

It is also called the steepest descent algorithm,
because at each iteration, we walk in the direction
of steepest descent. \\

This is a first-order algorithm, as we need the gradient
in order to find the direction $d$ of steepest descent
(we minimize the inner product). \\

Recall that the speed of descent at $\theta^{(k)}$
in some direction $d$ is the directional
derivative:
\[ \frac{f(\theta^{(k)} + t d) - 
f(\theta^{(k)})}{|t|} = 
\langle \nabla f(\theta^{(k)}), 
d \rangle + \frac{o(|t|)}{|t|} \]
We can then solve for the direction $d^{(k)}$
that maximizes the descent speed:
\[ \min_{d \in \mathbb{R}^d} 
\langle \nabla f(\theta^{(k)}), 
d \rangle \quad \SUCHTHAT \quad \|d\| = 1\]
\[ d^{(k)} = -\nabla f(\theta^{(k)}) \]
So this is the direction of steepest descent,
and it is given by the gradient. \\

We choose some step length $\tau$ to move in the direction
of steepest descent:
\[ \theta^{(k+1)} = \theta^{(k)} -\tau \nabla f(\theta^{(k)}) \]
We repeat these steps until the error is acceptable. \\

We choose some starting point $\theta^{(0)}$. \\

A heuristic for the step size is $\tau = \dfrac{2}{L + m}$. \\

The update step is thus:
\[ \theta^{(k+1)} = \theta^{(k)} - \tau 
\nabla f(\theta^{(k)})
\quad \WHERE \quad \nabla 
f(\theta^{(k)}) = \Phi^\top 
(\Phi \theta^{(k)} - y)\]
For $k \geq 1$. \\

The convergence rate for the $\tau$
we heuristically set is:
\[ \|\theta^{(k)} - \theta^*\| 
\leq \omega^k \|\theta^{(0)} - \theta^*\| 
= O(\omega^k) \quad \WHERE
\quad \omega = \frac{L/m - 1}{L/m + 1} \]
We have a linear rate of convergence, 
where $\omega^k$
decreases since $\omega < 1$. \\
Note that the convergence rate we have
here does depend on the step size, but here the formula
assumes that we use $\dfrac{2}{L + m}$. \\

The condition number $\kappa = \dfrac{L}{m}$
impacts convergence. \\

The idea behind gradient-descent is that we are making
locally optimal decisions that lead us towards the 
optimal solution. If we use a bad step size, we get a
zig-zag behavior that slows down convergence. \\

\newpage

\subsection*{Conjugate Gradient Method}

This method was first created to solve linear systems
\[ Q\theta = b \]
where $Q$ is positive semi-definite and $b \in \rbb^d$. \\

This can be shown to be equivalent to solving
the quadratic minimization problem:
\[ \theta \in \mathbb{R}^d 
\quad \frac{1}{2} \langle \theta 
Q \theta \rangle - \langle b, \theta \rangle \]
This will help us accelerate gradient descent
but only using first order information. \\

Our problem is equivalent to the form
we showed above, with $Q = \Phi^\top \Phi$ and 
$b = \Phi^\top y$:
\[ \frac{1}{2} \|\Phi \theta - y\|^2 
= \frac{1}{2} \langle \theta 
\Phi^\top \Phi \theta \rangle 
- \langle \theta, \Phi^\top y 
\rangle + \frac{1}{2} \|y\|^2 \]
Which is the same form. \\

The idea behind the algorithm that minimizes
this is to move in directions that are conjugates,
which ensures faster convergence. \\

For the first point $\theta^{(0)}$,
we start with a direction:
\[ d^{(0)} = r^{(0)} = b - Q \theta^{(0)}\]
Then the update step
updates the direction $d^{(k)}$ 
and $\theta^{(k)}$:
\[ \alpha_k = \frac{\langle r^{(k)}, r^{(k)} 
\rangle}{\langle d^{(k)}, Q d^{(k)} \rangle} 
\quad \text{(optimal step size)}\]
\[ \theta^{(k+1)} = \theta^{(k)} + \alpha_k 
d^{(k)} \quad \text{(main update step)} \]
\[ r^{(k+1)} = r^{(k)} - \alpha_k Q d^{(k)} 
\quad \text{(compute gradient)} \]
\[ \beta_{k+1} = \frac{\langle r^{(k+1)}, 
r^{(k+1)} \rangle}{\langle r^{(k)}, r^{(k)} 
\rangle} \quad \text{(projection coefficient)} \]
\[ d^{(k+1)} = r^{(k+1)} + \beta_{k+1} 
d^{(k)} \quad \text{(project onto Q-conjugate 
direction)} \] \\
The idea is that we generate directions for our
descent that are optimal in order to get to the solution
in an optimal number of iterations. \\

The generated directions span the whole space.
We can show the chosen directions are optimal. \\

The convergence rate is:
\[ \|\theta^{(k)} - \theta^*\| \leq 2 
\sqrt{\frac{L}{m}} \, \omega^k \|\theta^{(0)} 
- \theta^*\| = O(\omega^k) 
\quad \WHERE \quad \omega = 
\frac{\sqrt{L/m} - 1}{\sqrt{L/m} + 1} \]
The convergence rate is also linear, but it is much faster
since $\sqrt{\dfrac{L}{m}}$ is much smaller 
than $\dfrac{L}{m}$. \\
\newpage

\subsection*{Gradient Descent Generalization}

We can't generalize the conjugate graident method
for non-quadratic methods. \\
There are non-linear generalizations, 
but they won't be nearly as powerful. \\

We can however, generalize Gradient Descent
for any problem, since it only needs
to compute the gradient. \\

The goal will be to minimize any function $f$:
\[ \min_{\theta \in \mathbb{R}^d} f(\theta) \]
where $f$ is convex and continuously 
differentiable. \\

The first algorithm that can do that is called
gradient descent with line search. \\
The goal is to find a step size $\tau_k > 0$
each step such that the update step:
\[ \theta^{(k+1)} = \theta^{(k)} 
- \tau_k \nabla f(\theta^{(k)})\]
satisfies the Armijo condition for $\gamma$:
\[ f(\theta^{(k+1)}) \leq f(\theta^{(k)}) 
- \gamma \tau_k \|\nabla f(\theta^{(k)})\|^2 \]
however, we avoid it because it is too
costly; it has no convergence rate because it
does not have a constant step size. \\

The other methos is to 
bound the function at each point
by two quadratics, one below and one above. \\

We will make use of two properties
a function can have. \\

The first is $L$-smoothness,
which is the same as saying the gradient
of the function $\nabla f$ is $L$-Lipschitz
continuous:
\[ \|\nabla f(\theta) - \nabla f(\bar{\theta})\| 
\leq L \|\theta - \bar{\theta}\| 
\quad \forall \theta, \bar{\theta}\]
This will be a necessary condition. \\

The second condition is that $f$
is $m$-strongly convex, which means that:
\[ f(m) - \dfrac{m}{2} \|\theta\|^2 \]
is convex (we assume that $m \geq 0$). \\
This condition is not necessary, but gives
desirable results. \\

If we assume the first condition
is true for all $\olsi{\theta} \in \rbb^d$,
then by the descent lemma:
\[ \left| f(\theta) - \left( f(\bar{\theta}) 
+ \langle \nabla f(\bar{\theta}), \theta - 
\bar{\theta} \rangle \right) \right| 
\leq \frac{L}{2} \|\theta - \bar{\theta}\|^2 
\quad \forall \theta \in \mathbb{R}^d \]
Which we can reformulate as:
\[ f(\bar{\theta}) + \langle \nabla 
f(\bar{\theta}), \theta - \bar{\theta} 
\rangle - \frac{L}{2} \|\theta - 
\bar{\theta}\|^2 \leq f(\theta) \leq 
\underbrace{
f(\bar{\theta}) + \langle \nabla f(\bar{\theta}), 
\theta - \bar{\theta} \rangle + \frac{L}{2} 
\|\theta - \bar{\theta}\|^2}
_{=: f_L(\theta; \bar{\theta})}\]
This gives us an upper and lower bound at
each point $\olsi{\theta}$ that are both
quadratic functions. \\

Now what this means is that we can use
gradient descent to minimize any function
$f$, so long as $\nabla f$
is $L$-Lipschitz continuous,
and $f$ is at least convex. \\
We again choose any $\theta^{(0)}$
to start. \\
We don't need to do line search for the step
size, and can instead just use a fixed
size, so long as
\[ 0 < \tau < \dfrac{2}{L}\]
which yields the same update step as before:
\[ \theta^{(k+1)} = \theta^{(k)} 
- \tau \nabla f(\theta^{(k)}) \]
Which now works for any function $f$
so long as it is convex and $\nabla f$
is $L$-Lipschitz continuous. \\

This will give us a sub linear convergence rate:
\[ f(\theta^{(k)}) - \inf f \leq 
\frac{\|\theta^{(0)} - \theta^*\|^2}{2 \tau k} 
= O\left(\frac{1}{k}\right) \]
When we assume that $f$ is convex. \\

On the other hand, if we further assume
that $f$ is $m$-strongly convex, then
we get a linear rate of convergence
to a global minimizer $\theta^*$:
\[ \|x^{(k)} - x^*\|^2 \leq \omega^k 
\|x^{(0)} - x^*\|^2 = O(\omega^k) 
\quad \WHERE \quad \omega = 1 - \tau m \]
Which can be improved to:
\[ \omega = \frac{L/m - 1}{L/m + 1}\]
If we further assume that
$\tau = \dfrac{2}{L + m}$. \\

Note that if the original function $f$ was
not convex, then it is bounded from 
above by a convex quadratic (cup),
and from below by a concave quadratic (cap). \\

If the original function $f$ was
convex, which we assume it is by default,
then it is bounded from above by a quadratic,
and from below by a linear function. \\

And if $f$ is strongly convex, then it is
bounded from above and below by quadratics
that are both convex (cups)
instead of having the lower bound turned down. \\

\newpage

\subsection*{Lower Bound for the Convergence Rate}

We can show, non-constructively (only in theory)
what the optimal first order iterative
numerical method is for solving
optimization problems. \\

This gives us the lower bounds on the
convergence rate given some prior assumptions. \\

Optimal methods are those that have a convergence
rates proportional to the lower bound. \\

If we assume that the function $f$
is $m$-strongly convex and that $\nabla f$
is $L$-Lipschitz continuous,
then the lower bound on the convergence 
rate is 
\[ O(\omega^k)\] 
where $\omega$ depends on 
\[ \dfrac{\sqrt{L/m} - 1}{\sqrt{L/m} + 1} \]
The conjugate gradient and heavy-ball method
are both optimal in that case. \\

The heavy ball method is gradient descent 
with momentum. \\

On the other hand, if we only assume that $f$
is convex, and that $\nabla f$
is $L$-Lipschitz continuous,
then the convergence rate is 
\[ O\para{\dfrac{1}{k^2}} \]
Gradient descent is not optimal in this case,
but the Nesterov Accelerated Gradient method
is optimal. \\

\newpage

\subsection*{Nesterov's Accelerated Gradient
Method}

It is an improved Gradient Descent method.
Each iteration is at least as costly 
as gradient descent
since it also uses the gradient. \\

In Nesterov's Accelerated Gradient
Method, we want to minimize $f$:
\[ \min_{\theta \in \mathbb{R}^d} f(\theta) \]
And we assume that $f$
is convex and that $\nabla f$
is $L$-Lipschitz continuous. \\
We then choose two starting values:
\[ \theta^{(0)} = \theta^{(-1)} \in \rbb^d \]
A starting time $t_0 = 0$,
and a step size $\tau > 0$. \\
This yields the update steps:
\[ t_{k+1}^2 - t_{k+1} \leq t_k^2 \]
\[ \eta^{(k)} = \theta^{(k)} 
+ \frac{t_k - 1}{t_{k+1}} \left( \theta^{(k)} 
- \theta^{(k-1)} \right) \quad 
\text{(extrapolation step)} \]
\[ \theta^{(k+1)} = \eta^{(k)} - \tau \nabla 
f(\eta^{(k)}) \quad \text{(gradient 
descent step)} \]
Which is similar to gradient descent,
but with an extra $\eta(x)$ term. \\

This term is calculated in the extrapolation 
step. It is like a small extra step
we take in a specific direction
we take each iteration,
which gets us closer to the answer. \\

This yields the convergence rate:
\[ f(\theta^{(k)}) - f(\theta^*) \leq 
\frac{1}{2\tau} \frac{\|\theta^* - 
\theta^{(0)}\|^2}{t_k^2} \]
If we further assume that $t_k = \frac{k+1}{2}$,
we can bound the convergence rate further:
\[ \frac{2}{\tau} \frac{\|\theta^* - 
\theta^{(0)}\|^2}{(k+1)^2} = 
O\left(\frac{1}{k^2}\right) \]
Which is optimal as mentioned earlier. \\

The gradient calculation is the only costly 
part of this algorithm. \\

Note that it works for even non quadratic-functions,
so long as the smoothness and convexity
conditions are met. \\

\newpage

\subsection*{Proximal mapping}

If we want to generalize our optimization
methods to non-smooth functions, we need
to first calculate their proximal mappings. \\

The formula for the proximal mapping of $f$
with step size $\tau$ of the point $\theta^{(k)}$
is:
\[ \text{prox}_{\tau f}(\theta^{(k)}) 
= \arg\min_{\theta} \left[ f(\theta) 
+ \frac{1}{2\tau} \|\theta - \theta^{(k)}\|^2 
\right] \]
Notice that this will need to be done
each iteration. \\

If the proximal mapping of $f$ is easy to find,
we say that $f$ is prox-friendly. \\

Note that the function $\delta_C(\theta)$
is defined as:
\[ \delta_C(\theta)
= \piecewise{
    0, \qquad \;\; \IF  \theta \in C \\
    \infty, \qquad \IF \theta \notin C
} \]
It is the inicator function of a convex set $C$,
and it has infinite value if the point
is not in the convex set $C$. \\

So, for example, if we have
to find the proximal mapping of an indicator
function for some convex set:
\[ \text{prox}_{\tau \delta_C}(\theta^{(k)})
= \arg\min_{\theta} \left[ \delta_C(\theta) 
+ \frac{1}{2\tau} \|\theta - \theta^{(k)}\|^2 
\right] \]
Notice that the indictaor function is infinite
if we allow that $\theta \notin C$,
which dominates all other terms, so in order
to minimize we sort of have ot have $\theta$
be in $C$, which means that:
\[ \text{prox}_{\tau \delta_C}(\theta^{(k)})
= \arg\min_{\theta \in C} \left[ \frac{1}{2\tau} 
\|\theta - \theta^{(k)}\|^2 \right] \]
Where $\delta_C(\theta)$ became $0$.
Notice what we are saying here;
we want to find the $\theta$
in $C$ such that the distance between 
$\theta^{(k)}$ and $\theta$ is minimal. \\
So if $\theta^{(k)}$ is in $C$,
we just set $\theta = \theta^{(k)}$. \\
Otherwise, if $\theta^{(k)}$ is not in $C$,
then the closest point to $\theta^{(k)}$
on $C$ is the projection of $\theta^{(k)}$
on the convex set $C$. \\

Some examples of proximal mappings:
\begin{enumerate}
    \item For $f(\theta) = \|\theta\|^2_2$:
    \[ \left[ \text{prox}_{\tau \|\cdot\|_2^2}
    (\bar{\theta}) \right]_i = 
    \frac{\bar{\theta}_i}{1 + 2\tau}, 
    \quad \forall i = 1, \ldots, d \]
    \item For $f(\theta) = 
    \delta_{\|\theta \|_2 \leq 1}(\theta)$,
    where the convex set is the unit disk,
    the proximal mapping is:
    \[ \text{proj}_{\{\theta \mid 
    \|\theta\|_2 \leq 1\}}
    (\bar{\theta}) = \frac{\bar{\theta}}
    {\max(1, \|\bar{\theta}\|)}\]
    Which just means that we either set
    $\olsi{theta}$ to $\theta$ or the projection
    onto the unit circle $\|\theta\|_2 \leq 1$.
    \item For $f(\theta) = \|\theta\|_1$,
    also called the soft shrinkage-thresholding 
    operator:
    \[ \left[ \text{prox}_{\tau \|\cdot\|_1}
    (\bar{\theta}) \right]_i = 
    \max(0, |\bar{\theta}_i| - \tau) \cdot 
    \text{sign}(\bar{\theta}_i) 
    \quad \forall i = 1, \ldots, d \]
\end{enumerate}
We prove these are true in an assignment. \\

Note that while $\|\theta\|_1$
is prox-friendly, and can have proximal mapping
done to it,
this isn't true for $\|A \theta\|_1$
for some arbitrary matrix $A$. \\

\newpage

\subsection*{Proximal Gradient Descent}

Structured/Additive Composite optimization
is used in order to split an optimization problem. \\
If we have:
\[ \min_\theta h(\theta) \]
And we know that we can split $h$ such that:
\[ h(\theta) = f(\theta) + g(\theta)\]
where $f$ is $L$-smooth and convex,
and $g$ is convex,
non-smooth, and prox-friendly,
then we just need to minimize each
function individually. \\

Notice that this form
resembles the formula we get
when we want to minimize the empirical risk
plus the regularization term. \\
The empirical risk is smooth, and the regularizer
may be non-smooth, such as in LASSO. \\

Each step, we can update $\theta$
with respect to $f$ using the gradient
descent method (forward step):
\[ \theta^{(k)} = \theta^{(k)} - 
\tau \nabla f(\theta^{(k)}) \]
And then follow it up by an update
with respect to $g$ using proximal mapping
(backwards step):
\[ \theta^{(k+1)} = \text{prox}_{\tau g}
(\bar{\theta}^{(k)}) \]
Putting both of them together,
we get a single proximal-gradient step:
\[ \theta^{(k+1)} = \text{prox}_{\tau g} 
\left( \theta^{(k)} - \tau \nabla 
f(\theta^{(k)}) \right) \]
This algorithm converges towards
the minimum of $h(\theta)$,
and is called proximal-gradient descent. \\

So we just have to assume that $f$
is $L$-smooth (meaning that $\nabla f$
is $L$-Lipschitz continuous)
and convex, and that $g$
is convex and prox friendly. \\
We can pick any starting variable 
$\theta^{(0)} \in \rbb^d$,
and a step size:
\[ 0 < \tau < \dfrac{2}{L} \]
Then, we can apply the proximal gradient step
repeatedly. \\

This yields a sub-linear convergence rate
$O\para(\dfrac{1}{k})$. \\

If we further assume that $f$
is $m$-strongly convex, then we
get a linear convergence rate
$O(\omega^k)$ where:
\[ \omega = \frac{L/m - 1}{L/m + 1} \quad 
\FOR \quad \tau = \frac{2}{L + m}\]
Recall that a linear convergence rate
is faster than sublinear. \\

We can also use accelerated gradient descent 
here, where we use Nestorov's algorithm
instead of vanilla gradient descent. \\

We again make the same assumptions
about $f$ and $g$,
and pick any starting variables:
\[ \theta^{(0)} = \theta^{(-1)} \in \rbb^d \]
And some step size $\tau > 0$,
which gives us the update steps:
\[ t_{k+1}^2 - t_{k+1} \leq t_k^2 \]
\[ \eta^{(k)} = \theta^{(k)} + 
\frac{t_k - 1}{t_{k+1}} \left( \theta^{(k)} 
- \theta^{(k-1)} \right) \]
\[ \theta^{(k+1)} = \text{prox}_{\tau g} 
\left( \eta^{(k)} - \tau \nabla f(\eta^{(k)}) 
\right)\]
Which converges towards the minimizer. \\

This yields a convergence rate:
\[ h(\theta^{(k)}) - h(\theta^*) 
\leq \frac{1}{2\tau} \frac{\|\theta^* - 
\theta^{(0)}\|^2}{t_k^2} \]
And if we take $t_k = \dfrac{k+1}{2}$,
then the convergence rate becomes:
\[ \frac{2}{\tau} \frac{\|\theta^* - 
\theta^{(0)}\|^2}{(k+1)^2} = 
O\left(\frac{1}{k^2}\right) \]
Note that we don't call this a gradient
descent method because there is no descent;
there is convergence, but not every step
gets us closer. \\

So it is simpy called Accelerated Proximal 
Gradient Method.
It is also called FISTA. \\

\newpage

\subsection*{Duality}

Now we still have to find a way to solve
Hard and soft margin SVM, as well as regularized
least 1-norm minimization,
since they are not prox-friendly or are
non-smooth. \\

A disk is just a collection of points in space. \\

Or we can think of it as the intersection of
several half-spaces (all tangent to the circle). \\

We can call a closed set $C$
convex if it is the intersection of
closed half-spaces that contain it. \\

The same applies to convex functions:
a convex function $f$ is always the
pointwise supremum (max or min) of the
collection of affine functions $h$
such that $h \leq f$. \\
Think of it as lines of tangents below a graph
of $f$. \\

The conjugate of $f$ is $f^*$,
and it can be thought of as a function that 
operates on slopes;
it takes in a slope as input, and returns
the part where the affine function
intersects the $y$-axis.
It does so by sort of sliding the affine
function with the slope $\xi$
until it becomes tangent to $f(\theta)$. \\
We do that by maximizing the distance
between $f(\theta)$
and $\ang{\xi, \theta}$, 
which is the affine function. \\

The bi-conjugate is the conjugate of the
conjugate $f^{**}$. \\

The conjugate $f^*$ is alwasy convex. \\

If $f$ is convex, then $f = f^{**}$. \\

We then have some examples of conjugates. \\

The Fenchel-Rockafellar duality tells
us that if we have some optimization problem
with some structure:
\[ \min_\theta f(A\theta) + g(\theta) \]
then solving the primal problem corresponds
to solving the dual problem:
\[ \max_\xi - g^*(-A^T\xi) - f^*(\xi) \]
We do so by computing the conjugates of $f$
and $g$ seperately. \\
Notice that we ignore the $A$
in the primal problem. \\
Note that $f$ and $g$ must be convex
and $A$ must be a linear mapping. \\

The slides tell us how to derive
Fenchel-Rockafellar duality. \\

So we may be able to use proximal gradient
methods on the dual problem.
First we get the dual, which 
gets us a form solvable by proximal gradient
methods. \\

\colorText{red}{Moreau's identity.} \\

Then we are given an example using the
$\ell$-1 loss minimization with $\ell$-2
regularization, which is not solvable
in primal form since $\ell$-1 is non-smooth. \\

We calculate $f^*$ and $g^*$,
then we get a $\max$ dual problem, 
then a $\min$ of the negation. \\

Then we have a conclusion slide. \\

Then we are given how to solve soft,
hard margin SVM, and logistic regression
using the dual. \\

Note that we can write $\delta_{\rbb_-}(1-t)$
as a loss function for the hard-margin SVM. \\

So we just compute the conjugates
using the $\sup$ of inner product - $f\theta$,
then use Fenchel-Rockafellar duality. \\

\end{document}
