
\documentclass[12pt]{article}

\usepackage[margin=1in]{geometry}


%===============================================================================
%================================== PACKAGES ===================================
%===============================================================================

% For using float option H that places figures 
% exatcly where we want them
\usepackage{float}
% makes figure font bold
\usepackage{caption}
\captionsetup[figure]{labelfont=bf}
% For text generation
\usepackage{lipsum}
% For drawing
\usepackage{tikz}
% For smaller or equal sign and not divide sign
\usepackage{amssymb}
% For the diagonal fraction
\usepackage{xfrac}
% For enumerating exercise parts with letters instead of numbers
\usepackage{enumitem}
% For dfrac, which forces the fraction to be in display mode (large) e
% even in math mode (small)
\usepackage{amsmath}
% For degree sign
\usepackage{gensymb}
% For "\mathbb" macro
\usepackage{amsfonts}
% For positioning 
\usepackage{indentfirst}
\usetikzlibrary{shapes,positioning,fit,calc}
% for adjustwidth environment
\usepackage{changepage}
% for arrow on top
\usepackage{esvect}
% for mathbb 1
\usepackage{bbm}
% for mathsrc
\usepackage[mathscr]{eucal}
% For degree sign
\usepackage{gensymb}
% For quotes
\usepackage{csquotes}
% For vertical lines
\usepackage{mathtools}
% For cols
\usepackage{multicol}

% for tikz
\usepackage{pgfplots}
\pgfplotsset{compat=1.18}
\usepackage{amsmath}
\usepgfplotslibrary{groupplots}


%===============================================================================
%==================================== FONTS ====================================
%===============================================================================


% Mathcal
\newcommand{\acal}{\mathcal{A}}
\newcommand{\bcal}{\mathcal{B}}
\newcommand{\ccal}{\mathcal{C}}
\newcommand{\dcal}{\mathcal{D}}
\newcommand{\ecal}{\mathcal{E}}
\newcommand{\fcal}{\mathcal{F}}
\newcommand{\gcal}{\mathcal{G}}
\newcommand{\hcal}{\mathcal{H}}
\newcommand{\ical}{\mathcal{I}}
\newcommand{\jcal}{\mathcal{J}}
\newcommand{\kcal}{\mathcal{K}}
\newcommand{\lcal}{\mathcal{L}}
\newcommand{\mcal}{\mathcal{M}}
\newcommand{\ncal}{\mathcal{N}}
\newcommand{\ocal}{\mathcal{O}}
\newcommand{\pcal}{\mathcal{P}}
\newcommand{\qcal}{\mathcal{Q}}
\newcommand{\rcal}{\mathcal{R}}
\newcommand{\scal}{\mathcal{S}}
\newcommand{\tcal}{\mathcal{T}}
\newcommand{\ucal}{\mathcal{U}}
\newcommand{\vcal}{\mathcal{V}}
\newcommand{\wcal}{\mathcal{W}}
\newcommand{\xcal}{\mathcal{X}}
\newcommand{\ycal}{\mathcal{Y}}
\newcommand{\zcal}{\mathcal{Z}}

% Mathfrak
\newcommand{\afrak}{\mathfrak{A}}
\newcommand{\bfrak}{\mathfrak{B}}
\newcommand{\cfrak}{\mathfrak{C}}
\newcommand{\dfrak}{\mathfrak{D}}
\newcommand{\efrak}{\mathfrak{E}}
\newcommand{\ffrak}{\mathfrak{F}}
\newcommand{\gfrak}{\mathfrak{G}}
\newcommand{\hfrak}{\mathfrak{H}}
\newcommand{\ifrak}{\mathfrak{I}}
\newcommand{\jfrak}{\mathfrak{J}}
\newcommand{\kfrak}{\mathfrak{K}}
\newcommand{\lfrak}{\mathfrak{L}}
\newcommand{\mfrak}{\mathfrak{M}}
\newcommand{\nfrak}{\mathfrak{N}}
\newcommand{\ofrak}{\mathfrak{O}}
\newcommand{\pfrak}{\mathfrak{P}}
\newcommand{\qfrak}{\mathfrak{Q}}
\newcommand{\rfrak}{\mathfrak{R}}
\newcommand{\sfrak}{\mathfrak{S}}
\newcommand{\tfrak}{\mathfrak{T}}
\newcommand{\ufrak}{\mathfrak{U}}
\newcommand{\vfrak}{\mathfrak{V}}
\newcommand{\wfrak}{\mathfrak{W}}
\newcommand{\xfrak}{\mathfrak{X}}
\newcommand{\yfrak}{\mathfrak{Y}}
\newcommand{\zfrak}{\mathfrak{Z}}

% Mathscr
\newcommand{\ascr}{\mathscr{A}}
\newcommand{\bscr}{\mathscr{B}}
\newcommand{\cscr}{\mathscr{C}}
\newcommand{\dscr}{\mathscr{D}}
\newcommand{\escr}{\mathscr{E}}
\newcommand{\fscr}{\mathscr{F}}
\newcommand{\gscr}{\mathscr{G}}
\newcommand{\hscr}{\mathscr{H}}
\newcommand{\iscr}{\mathscr{I}}
\newcommand{\jscr}{\mathscr{J}}
\newcommand{\kscr}{\mathscr{K}}
\newcommand{\lscr}{\mathscr{L}}
\newcommand{\mscr}{\mathscr{M}}
\newcommand{\nscr}{\mathscr{N}}
\newcommand{\oscr}{\mathscr{O}}
\newcommand{\pscr}{\mathscr{P}}
\newcommand{\qscr}{\mathscr{Q}}
\newcommand{\rscr}{\mathscr{R}}
\newcommand{\sscr}{\mathscr{S}}
\newcommand{\tscr}{\mathscr{T}}
\newcommand{\uscr}{\mathscr{U}}
\newcommand{\vscr}{\mathscr{V}}
\newcommand{\wscr}{\mathscr{W}}
\newcommand{\xscr}{\mathscr{X}}
\newcommand{\yscr}{\mathscr{Y}}
\newcommand{\zscr}{\mathscr{Z}}

% Mathbb
\newcommand{\abb}{\mathbb{A}}
\newcommand{\bbb}{\mathbb{B}}
\newcommand{\cbb}{\mathbb{C}}
\newcommand{\dbb}{\mathbb{D}}
\newcommand{\ebb}{\mathbb{E}}
\newcommand{\fbb}{\mathbb{F}}
\newcommand{\gbb}{\mathbb{G}}
\newcommand{\hbb}{\mathbb{H}}
\newcommand{\ibb}{\mathbb{I}}
\newcommand{\jbb}{\mathbb{J}}
\newcommand{\kbb}{\mathbb{K}}
\newcommand{\lbb}{\mathbb{L}}
\newcommand{\mbb}{\mathbb{M}}
\newcommand{\nbb}{\mathbb{N}}
\newcommand{\obb}{\mathbb{O}}
\newcommand{\pbb}{\mathbb{P}}
\newcommand{\qbb}{\mathbb{Q}}
\newcommand{\rbb}{\mathbb{R}}
\newcommand{\sbb}{\mathbb{S}}
\newcommand{\tbb}{\mathbb{T}}
\newcommand{\ubb}{\mathbb{U}}
\newcommand{\vbb}{\mathbb{V}}
\newcommand{\wbb}{\mathbb{W}}
\newcommand{\xbb}{\mathbb{X}}
\newcommand{\ybb}{\mathbb{Y}}
\newcommand{\zbb}{\mathbb{Z}}


%===============================================================================
%=============================== SPECIAL SYMBOLS ===============================
%===============================================================================


% Orbit (group theory)
\newcommand{\orbit}{\mathcal{O}}
% Normal group
\newcommand{\normal}{\mathcal{N}}
% Indicator function
\newcommand{\indicator}{\mathbbm{1}}
% Laplace transform
\newcommand{\laplace}[1]{\mathcal{L}}
% Epsilon shorthand
\newcommand{\eps}{\varepsilon}
% Omega
\newcommand{\om}{\omega}
\newcommand{\Om}{\Omega}

%===============================================================================
%================================== OPERATORS ==================================
%===============================================================================


% Inverse exponent
\newcommand{\inv}[0]{^{-1}}
% Overline bar
\newcommand{\olsi}[1]{\,\overline{\!{#1}}}
% Less than or equal slanted
\newcommand{\seqs}{\leqslant}
% Greater or equal slanted
\newcommand{\geqs}{\geqslant}
% Subset or equal
\newcommand{\sub}{\subseteq}
% Proper subset
\newcommand{\prosub}{\subset}
% from
\newcommand{\from}{\leftarrow}

% Parantheses
\newcommand{\para}[1]{\left( #1 \right)}
% Curly Braces
\newcommand{\curl}[1]{\left\{ #1 \right\}}
% Brackets
\newcommand{\brac}[1]{\left[ #1 \right]}
% Angled Brackets
\newcommand{\ang}[1]{\left\langle #1 \right\rangle}
% Norm
\newcommand{\norm}[1]{\left\| #1 \right\|}

% Piece wise (use \\ between cases)
\newcommand{\piecewise}[1]{\begin{cases} #1 \end{cases}}

% Bold symbol shorthand
\newcommand{\bl}{\boldsymbol}

% Vertical space
\newcommand{\vs}[1]{\vspace{#1 pt}}
% Horizontal ertical space
\newcommand{\hs}[1]{\hspace{#1 pt}}


%===============================================================================
%============================== TEXT BASED SYMBOLS =============================
%===============================================================================


% Radians
\newcommand{\rad}{\text{rad}}
% Least Common Multiple
\newcommand{\lcm}{\text{lcm}}
% Automorphism
\newcommand{\Aut}{\text{Aut}}
% Variance
\newcommand{\var}{\text{Var}}
% Covariance
\newcommand{\cov}{\text{Cov}}
% Cofactor (matrix)
\newcommand{\cof}{\text{Cof}}
% Adjugate (matrix)
\newcommand{\adj}{\text{Adj}}
% Trace (matrix)
\newcommand{\tr}{\text{tr}}
% Standard deviation
\newcommand{\std}{\text{Std}}
% Correlation coefficient
\newcommand{\corr}{\text{Corr}}
% Sign
\newcommand{\sign}{\text{sign}}

% And text
\newcommand{\AND}{\text{ and }}
% Or text 
\newcommand{\OR}{\text{ or }}
% For text 
\newcommand{\FOR}{\text{ for }}
% If text
\newcommand{\IF}{\text{ if }}
% When text
\newcommand{\WHEN}{\text{ when }}
% Where text
\newcommand{\WHERE}{\text{ where }}
% Then text
\newcommand{\THEN}{\text{ then }}
% Such that text
\newcommand{\SUCHTHAT}{\text{ such that }}

% 1st
\newcommand{\st}[1]{#1^{\text{st}}}
% 2nd
\newcommand{\nd}[1]{#1^{\text{nd}}}
% 3rd
\newcommand{\rd}[1]{#1^{\text{rd}}}
% nth
\newcommand{\nth}[1]{#1^{\text{th}}}


%===============================================================================
%========================= PROBABILITY AND STATISTICS ==========================
%===============================================================================


% Permutation
\newcommand{\perm}[2]{{}^{#1}\!P_{#2}}
% Combination
\newcommand{\comb}[2]{{}^{#1}C_{#2}}

% Baye's risk
\newcommand{\risk}[1]{\mathscr{R}_{#1}}
% Baye's optimal risk
\newcommand{\riskOptimal}[1]{\mathscr{R}_{#1}^*}
% Baye's empirical risk
\newcommand{\riskEmpirical}[2]{\hat{\mathscr{R}}_{#1}^{#2}}


%===============================================================================
%=================================== CALCULUS ==================================
%===============================================================================


% d over d derivative
\newcommand{\dd}[2]{\dfrac{d#1}{d#2}}
% partial d over d derivative
\newcommand{\partialdd}[2]{\dfrac{\partial #1}{\partial #2}}
% delta d over d derivative
\newcommand{\deltadd}[2]{\dfrac{\Delta #1}{\Delta #2}}

% Integration between a and b
\newcommand{\integral}[4]{\int_{#1}^{#2} #3 \, #4}
% Integration in some space 
\newcommand{\boundIntegral}[2]{\int_{#1} #2 \, d#1}

% Limit
\newcommand{\limit}[3]{\lim_{#1 \to #2} #3}


%===============================================================================
%================================  BIG SYMBOLS  ================================
%===============================================================================


% Sum
\newcommand{\sumof}[2]{\sum_{#1}^{#2}}
% Product
\newcommand{\productof}[2]{\prod_{#1}^{#2}}
% Union
\newcommand{\unionof}[2]{\bigcup_{#1}^{#2}}
% Intersection
\newcommand{\intersectionof}[2]{\bigcap_{#1}^{#2}}
% Or
\newcommand{\orof}[2]{\bigvee_{#1}^{#2}}
% And
\newcommand{\andof}[2]{\bigwedge_{#1}^{#2}}


%===============================================================================
%=============================== LINEAR ALGEBRA ================================
%===============================================================================


% Bold vector arrow
\newcommand{\bv}[1]{\vec{\mathbf{#1}}}

% Matrix or vector (use // for column, & for row) 
% with brackets
\newcommand{\bmat}[1]{\begin{bmatrix} #1 \end{bmatrix}}
% Matrix or vector (use // for column, & for row) 
% with curved brackets
\newcommand{\pmat}[1]{\begin{pmatrix} #1 \end{pmatrix}}
% Matrix or vector (use // for column, & for row) 
% with lines on either side 
\newcommand{\lmat}[1]{\begin{vmatrix} #1 \end{vmatrix}}
% Matrix or vector (use // for column, & for row) 
% with curly braces
\newcommand{\cmat}[1]{\begin{Bmatrix} #1 \end{Bmatrix}}
% Matrix or vector (use // for column, & for row) 
% with no braces
\newcommand{\mat}[1]{\begin{matrix} #1 \end{matrix}}


%===============================================================================
%================================ LARGE OBJECTS ================================
%===============================================================================

% Multiple lines
\newcommand{\multiline}[1]{
\begin{align*}
    #1
\end{align*}
}

% Multiple lines with equation numbers
\newcommand{\eqmultiline}[1]{
\begin{align*}
    #1
\end{align*}
}

% Color
\newcommand{\colorText}[2]{
\begingroup
\color{#1}
    #2
\endgroup
}

% Centered figure
\newcommand{\centerFigure}[2]{
    \begin{figure}[h]
        \centering
            #1
        \caption{#2}
    \end{figure}
}

% Tikz figure
\newcommand{\tikzGraphic}[1]{
    \begin{center}
    \begin{tikzpicture}
        #1
    \end{tikzpicture}
    \end{center}
}

% Enumerate numbers (seperate by \item)
\newcommand{\numbers}{\textbf{\number*)}}

% Enumerate letters (seperate by \item)
\newcommand{\letters}{\textbf{\alph*)}}

\title{%
    \Huge Machine Learning \\
    \Large Lecture XII
}
\date{2025-06-12}
\author{Michael Saba}

\begin{document}
\pagenumbering{gobble}
\maketitle
\newpage
\setlength{\parindent}{0pt}
\pagenumbering{arabic}

\section*{Kernel Methods}
\subsection*{Motivation}


We start with quadratic regularized empirical risk
risk minimization with a margin-based loss 
function $\tilde{\ell}$:
\[ \min_{\theta \in \mathbb{R}^d} \frac{1}{n} 
\sum_{i=1}^n \tilde{\ell}((\Phi \theta)_i y_i) + 
\frac{\lambda}{2} \|\theta\|_2^2 \] 
Where $\Phi \in 
\mathbb{R}^{n \times d}$ is the feature matrix,
$y_i \in \mathbb{R}^n$ is an output label in the
training dataset, $\theta \in \mathbb{R}^d$
is the weights vector, and  $\lambda > 0$
is the regularization parameter. \\
Since $\tilde{\ell}$ is margin based, it could for 
example, be the loss for the hard-margin SVM,
$\delta_{\rbb_-}(1-t)$. \\

We can place all $y_i$ labels in the diagonals of
a matrix, and call it $\Upsilon$,
which allows us to write the regularized risk
minimization as:
\[ \min_{\theta \in \mathbb{R}^d} \frac{1}{n} 
\sum_{i=1}^n \tilde{\ell}((\Upsilon \Phi \theta)_i) + 
\frac{\lambda}{2} \|\theta\|_2^2\]
This loss is not nice enough for proximal gradient
descent, so we use the Fenchel-Rockafellar Duality
to associate a dual problem to our problem with the
same objective function:
\[ \min_{\theta} f(A\theta) + g(\theta) = 
\max_{\xi} -g^*(-A^\top \xi) - f^*(\xi) \]
We then take:
\[ A = \Upsilon \Phi \qquad 
f(\eta) = \frac{1}{n} \sum_{i=1}^n \tilde{\ell}(\eta_i) 
\qquad g(\theta) = \frac{\lambda}{2} \|\theta\|^2_2\]
Which have conjugates:
\[ f^*(\xi) = \frac{1}{n} \sum_{i=1}^n 
\tilde{\ell}^*(n\xi_i) \qquad 
g^*(\xi) = \frac{1}{2\lambda} \|\xi\|^2\]
Which yields the dual:
\[ \min_{\xi} \frac{1}{n} \sum_{i=1}^n 
\tilde{\ell}^*(n\xi_i) + \frac{1}{2\lambda} 
\|\Phi^\top \Upsilon \xi\|^2\] 
\[ \min_{\xi} \frac{1}{n} \sum_{i=1}^n 
\tilde{\ell}^*(n\xi_i) + \frac{1}{2\lambda} \langle 
\Upsilon \xi, \Phi \Phi^\top \Upsilon \xi \rangle \]
Which we can minimize to get the optimal $\xi$. \\

Now we know that, by the primal-dual relation:
\[ \theta = \nabla g^{-1}(-A^\top \xi) \]
\[ \theta = -\frac{1}{\lambda} \Phi^\top \Upsilon \xi \]
We can substitute in here the optimal $\xi$
to get the optimal $\theta$. \\

Now notice that in the minimization
of the fual problem,
we have the $n$ by $n$ matrix $K = \Phi\Phi^T$.
Notice that it is dimension independent,
meaning that it does not depend on the dimension
of the feature map $\varphi(x)$. \\
It is called the Gram Matrix. \\

This matrix will have entries with inner products
of the feature vector of the input:
\[ K = \Phi \Phi^\top = 
\pmat{\langle \varphi(x_1), \varphi(x_1) \rangle 
& \cdots & \langle \varphi(x_1), \varphi(x_n) \rangle \\
\vdots & \ddots & \vdots \\
\langle \varphi(x_n), \varphi(x_1) \rangle & \cdots 
& \langle \varphi(x_n), \varphi(x_n) \rangle} \]
Notice that it only has the inner product
of the feature maps in it, not the features 
themselves. \\

Now recall that our predictor $g_\theta(x)$
looks like:
\[ g_\theta(x) = \sum_{i=1}^n \theta_i \phi_i(x) = 
\langle \varphi(x), \theta \rangle \]
Now, if we replace $\theta$ by the optimal one we found,
which we can rewrite as a sum:
\[ \theta = -\frac{1}{\lambda} \Phi^\top \Upsilon \xi
= -\frac{1}{\lambda} \sum_{i=1}^n y_i \xi_i x_i = 
-\frac{1}{\lambda} \sum_{i=1}^n y_i \xi_i \varphi(x_i) \]
We get:
\[ g_{\theta}(x)
= - \frac{1}{\lambda} \sum_{i=1}^n y_i \xi_i 
\langle \varphi(x_i), \varphi(x) \rangle \]
Notice that again, we only need to know the
inner product of the feature map, not the
feature map itself. \\

A kernel function $k(x_i, x_j)$
return these exact inner products,
with respect to some feature map.
It measures similarity between $\varphi(x_i)$ 
and $\varphi(x_j)$. \\

The advantage is that instead of constructing
a feature map, which can be difficult, it may be
easier to instead define a function that measures
similarity intsead. And since we only ever need
the inner product, and not the feature map itself,
we can avoid defining it if we have the inner product,
or kernel. \\

So we will define kernel functions without
knowing what the feature map is; it is a measures
of similarity, but no longer an inner product. \\

Unlike feature maps, kernels do not depend on the
dimensionality of the feature space
(it is just an inner product, a single scalar output
given vectors $x$ and $x_i$ with arbitrary dimensions). \\

\newpage

\subsection*{Kernels}

We can't just replace inner products by kernels,
we need to analyze what is happening under the hood. \\

Positive Definite Kernels have a one to one
association with a Hilbert space 
$\hcal_k$, which more specifically,
is a Reproducing Kernel Hilbert Space,
or RKHS. \\

For now it is enough to know that a Hilbert Space
contains functions, including the kernel. \\

The Hilbert class will be our hypothesis
class for learning. The norm of this class
yields a natural regularization term
$\rcal(g) = \|g\|^2_{\hcal_k}$. \\

One advantage is that regularization penalties
are not applied to weights, but to the function 
directly. \\

For example, for the Guassian kernel:
\[ k(x, z) = e^{-{(x - z)^2/2\sigma^2}} \]
If we were to find its associated Hilbert Space,
we could prove that its norm is of the form:
\[ \|g\|^2_{\hcal_k} = \int_{\mathbb{R}} \sum_{s=0}^{\infty} 
\frac{\sigma^{2s}}{s! \, 2^s} \big( (\partial_x^s g)(x) 
\big)^2 \, dx\]
This associated RKHS contains only smooth functions. \\

The kernel is nothing more than an inner product
in the Hilbert Space:
\[ k(x, z) = \ang{\varphi(x), \varphi(z)} \]
Where $\varphi: \xcal \to \hcal$. \\
It is a similarity measure. \\

We also have a dissimilarity measure between $x$
and $z$ in the Hilbert Space, given by the norm,
which makes sense since it measure the distance:
\[ d(x, z) = \| \phi(x) - \phi(z) \| 
= \sqrt{ k(x, x) + k(z, z) - 2k(x, z) } \]
Norm are defined in terms of the inner product,
which we can replace by the kernel here
since it is the inner product. \\

A kernel is a function:
\[ k: \xcal \times \xcal \to \rbb \]
It is called positive definite if for all
$m \geq 1$, and for any two sequences:
\[ x_1, x_2,\dots, x_m \in \xcal 
\qquad c_1, c_2, \dots, c_m \in \rbb \]
The following inequality holds:
\[ \sum_{i=1}^m \sum_{j=1}^m c_i c_j k(x_i, x_j) \geq 0 \]
If the sum is strictly larger than $0$
for all distinct sequences that aren't jsut $0$,
we say the kernel is strictly positive definite. \\

A kernel matrix is the Gram Matrix we saw at the start,
with the inner product entries, which in the kernel
matrix are replaced by the kernel of $x_i$
and $x_j$:
\[ K = \big( k(x_i, x_j) \big)_{i,j} \]
A kernel is positive-definite, if every
kernel matrix for all sequences $x_1, \dots, x_m$
are all positive 
semi-definite (any sequence from $\xcal$). \\
We can use this to check that a kernel is positive
definite:
\[ w^TKw \geq 0 \]
Notice that checking that this is $\geq 0$
is pretty much the same as checking that the
inequality above holds. \\

The same applies for strictly positive definite 
kernels; we just need to prove that the Kernel matrix 
is positive definite intsead of positive
semi-definite. \\

Now, if we have two kernels $k_1$ and $k_2$
on $\xcal \times \xcal$, both of which are
positive definite, then these kernels $k$
are also positive definite:
\begin{enumerate}
    \item \( k(x, z) = \alpha \, k_1(x, z) \), 
    for any \(\alpha > 0\).
    \item \( k(x, z) = k_1(x, z) + k_2(x, z) \) 
    (pointwise addition).
    \item \( k(x, z) = k_1(x, z) k_2(x, z) \) 
    (pointwise multiplication).
    \item The pointwise limit \( k \) 
    of a sequence of positive definite kernels 
    \( k_i \) as \( i \to \infty \).
    \item \( k(x, z) = f(x) f(z) k_1(x, z) \) 
    for any \( f : X \to \mathbb{R} \) 
    (especially \( k(x, z) = f(x) f(z) \)).
    \item \( k(x, z) = \langle \phi(x), \phi(z) 
    \rangle \) for any \( \phi : X \to H \) with 
    Hilbert space \( \hcal \).
\end{enumerate}
The first two are trivial to prove, the third
the lecturer proved in class. \\

\newpage

\subsection*{Hilbert Spaces}

A Hilbert Space is just a complete vector space with
an inner product. \\

One example of a Hilbert space is the 
space of square integrable functions $L_2(\xcal)$:
\[ L^2(\xcal) = \left\{ f : \xcal \to \mathbb{R} \ \bigg| 
\ \int_\xcal |f(x)|^2 \, dx < \infty \right\} \]
Together with its inner product:
\[ \langle f, g \rangle := \int_\xcal f(x) g(x) \, dx \]
The issue with this space for example,
is that it is not a space of pointwise defined 
functions; there is no precise value 
of points, just the integral, which is a deal breaker
for us, since we need to be able to make
pointwise predictions. \\

This is why we need the RKHS family for the
kernels. \\

A Reproducing Kernel Hilbert Space $\hcal$
on $\xcal$ is a Hilbert Space of functions
$f: \xcal \to \rbb$ along with a reproducing kernel
$k$ on $\xcal \times \xcal$ which satisfies:
\[ \forall x \in \xcal: \quad k(x, \cdot) \in \hcal \]
\[ \forall f \in \hcal: \quad 
\ang{f, k(x, \cdot)}_\hcal = f(x) \]
The first condition statest that
if we fix one of the inputs of $k$,
it becomes a function on $\xcal$, and must
be part of the Hilbert Space. \\
The second is the reproducing property. \\

This makes the association clear;
given a Hilbert Space with kernel $k$,
$k$ is the kernel of the Hilbert Space,
which is always going to be positive definite. 
This side is trivial. \\
On the other hand, if we have a positive definite
kernel, then we can always construct a Hilbert
Space using that kernel, that satisfies the
above conditions. We can prove this
Hilbert Space is unique. \\

So we have a one to one association between
positive definite kernels and RKHS. \\
Each Hilbert Space is assoicated with one positive
definite kernel, and given a positive definite
kernel, we may construct one unique Hilbert Space. \\

For the first side of the association,
we can prove that the kernel of a RKHS 
is always positive definite, which the 
lecturer now does. \\

For the other side, 
we can show how to contrusct a RKHS using
some kernel $k$, which induces an inner product,
which we then show is independent of
our specific representation we use for $g$
and $f$ (the choice of $\alpha$ and $\beta$ vectors). \\
We also check the reproducing property holds. \\
And finally we check that the norm induced from the
inner product is that whenever the norm is $0$,
the function is $0$: if $\|f(x)\|^2_{\hcal} = 0 \implies
|f(x)| = 0$. We use the Cauchy-Schwartz inequality. \\
The Hilbert Space is a complete Vector Space,
so we should show that each cauchy sequence
converges, but this isn't very insightful. 
So we need a standard completion argument. \\
Finally, we need to show the uniqueness of the space;
if there is another RKHS with the same kernel, we
can prove that it can't be smaller, since we
used the span, so the other hilbert space must be
larger, but we can show the gap between the two
is 0, so they are the same Hilbert Space. \\

This construction gives us a nice Hilbert space,
with functions that can be evaluated pointwise;
note that all the functions are just combinations
of the kernel at different points. \\

\end{document}
