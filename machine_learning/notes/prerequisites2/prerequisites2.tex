
\documentclass[12pt]{article}

\usepackage[margin=1in]{geometry}


%===============================================================================
%================================== PACKAGES ===================================
%===============================================================================

% For using float option H that places figures 
% exatcly where we want them
\usepackage{float}
% makes figure font bold
\usepackage{caption}
\captionsetup[figure]{labelfont=bf}
% For text generation
\usepackage{lipsum}
% For drawing
\usepackage{tikz}
% For smaller or equal sign and not divide sign
\usepackage{amssymb}
% For the diagonal fraction
\usepackage{xfrac}
% For enumerating exercise parts with letters instead of numbers
\usepackage{enumitem}
% For dfrac, which forces the fraction to be in display mode (large) e
% even in math mode (small)
\usepackage{amsmath}
% For degree sign
\usepackage{gensymb}
% For "\mathbb" macro
\usepackage{amsfonts}
% For positioning 
\usepackage{indentfirst}
\usetikzlibrary{shapes,positioning,fit,calc}
% for adjustwidth environment
\usepackage{changepage}
% for arrow on top
\usepackage{esvect}
% for mathbb 1
\usepackage{bbm}
% for mathsrc
\usepackage[mathscr]{eucal}
% For degree sign
\usepackage{gensymb}
% For quotes
\usepackage{csquotes}
% For vertical lines
\usepackage{mathtools}
% For cols
\usepackage{multicol}

% for tikz
\usepackage{pgfplots}
\pgfplotsset{compat=1.18}
\usepackage{amsmath}
\usepgfplotslibrary{groupplots}


%===============================================================================
%==================================== FONTS ====================================
%===============================================================================


% Mathcal
\newcommand{\acal}{\mathcal{A}}
\newcommand{\bcal}{\mathcal{B}}
\newcommand{\ccal}{\mathcal{C}}
\newcommand{\dcal}{\mathcal{D}}
\newcommand{\ecal}{\mathcal{E}}
\newcommand{\fcal}{\mathcal{F}}
\newcommand{\gcal}{\mathcal{G}}
\newcommand{\hcal}{\mathcal{H}}
\newcommand{\ical}{\mathcal{I}}
\newcommand{\jcal}{\mathcal{J}}
\newcommand{\kcal}{\mathcal{K}}
\newcommand{\lcal}{\mathcal{L}}
\newcommand{\mcal}{\mathcal{M}}
\newcommand{\ncal}{\mathcal{N}}
\newcommand{\ocal}{\mathcal{O}}
\newcommand{\pcal}{\mathcal{P}}
\newcommand{\qcal}{\mathcal{Q}}
\newcommand{\rcal}{\mathcal{R}}
\newcommand{\scal}{\mathcal{S}}
\newcommand{\tcal}{\mathcal{T}}
\newcommand{\ucal}{\mathcal{U}}
\newcommand{\vcal}{\mathcal{V}}
\newcommand{\wcal}{\mathcal{W}}
\newcommand{\xcal}{\mathcal{X}}
\newcommand{\ycal}{\mathcal{Y}}
\newcommand{\zcal}{\mathcal{Z}}

% Mathfrak
\newcommand{\afrak}{\mathfrak{A}}
\newcommand{\bfrak}{\mathfrak{B}}
\newcommand{\cfrak}{\mathfrak{C}}
\newcommand{\dfrak}{\mathfrak{D}}
\newcommand{\efrak}{\mathfrak{E}}
\newcommand{\ffrak}{\mathfrak{F}}
\newcommand{\gfrak}{\mathfrak{G}}
\newcommand{\hfrak}{\mathfrak{H}}
\newcommand{\ifrak}{\mathfrak{I}}
\newcommand{\jfrak}{\mathfrak{J}}
\newcommand{\kfrak}{\mathfrak{K}}
\newcommand{\lfrak}{\mathfrak{L}}
\newcommand{\mfrak}{\mathfrak{M}}
\newcommand{\nfrak}{\mathfrak{N}}
\newcommand{\ofrak}{\mathfrak{O}}
\newcommand{\pfrak}{\mathfrak{P}}
\newcommand{\qfrak}{\mathfrak{Q}}
\newcommand{\rfrak}{\mathfrak{R}}
\newcommand{\sfrak}{\mathfrak{S}}
\newcommand{\tfrak}{\mathfrak{T}}
\newcommand{\ufrak}{\mathfrak{U}}
\newcommand{\vfrak}{\mathfrak{V}}
\newcommand{\wfrak}{\mathfrak{W}}
\newcommand{\xfrak}{\mathfrak{X}}
\newcommand{\yfrak}{\mathfrak{Y}}
\newcommand{\zfrak}{\mathfrak{Z}}

% Mathscr
\newcommand{\ascr}{\mathscr{A}}
\newcommand{\bscr}{\mathscr{B}}
\newcommand{\cscr}{\mathscr{C}}
\newcommand{\dscr}{\mathscr{D}}
\newcommand{\escr}{\mathscr{E}}
\newcommand{\fscr}{\mathscr{F}}
\newcommand{\gscr}{\mathscr{G}}
\newcommand{\hscr}{\mathscr{H}}
\newcommand{\iscr}{\mathscr{I}}
\newcommand{\jscr}{\mathscr{J}}
\newcommand{\kscr}{\mathscr{K}}
\newcommand{\lscr}{\mathscr{L}}
\newcommand{\mscr}{\mathscr{M}}
\newcommand{\nscr}{\mathscr{N}}
\newcommand{\oscr}{\mathscr{O}}
\newcommand{\pscr}{\mathscr{P}}
\newcommand{\qscr}{\mathscr{Q}}
\newcommand{\rscr}{\mathscr{R}}
\newcommand{\sscr}{\mathscr{S}}
\newcommand{\tscr}{\mathscr{T}}
\newcommand{\uscr}{\mathscr{U}}
\newcommand{\vscr}{\mathscr{V}}
\newcommand{\wscr}{\mathscr{W}}
\newcommand{\xscr}{\mathscr{X}}
\newcommand{\yscr}{\mathscr{Y}}
\newcommand{\zscr}{\mathscr{Z}}

% Mathbb
\newcommand{\abb}{\mathbb{A}}
\newcommand{\bbb}{\mathbb{B}}
\newcommand{\cbb}{\mathbb{C}}
\newcommand{\dbb}{\mathbb{D}}
\newcommand{\ebb}{\mathbb{E}}
\newcommand{\fbb}{\mathbb{F}}
\newcommand{\gbb}{\mathbb{G}}
\newcommand{\hbb}{\mathbb{H}}
\newcommand{\ibb}{\mathbb{I}}
\newcommand{\jbb}{\mathbb{J}}
\newcommand{\kbb}{\mathbb{K}}
\newcommand{\lbb}{\mathbb{L}}
\newcommand{\mbb}{\mathbb{M}}
\newcommand{\nbb}{\mathbb{N}}
\newcommand{\obb}{\mathbb{O}}
\newcommand{\pbb}{\mathbb{P}}
\newcommand{\qbb}{\mathbb{Q}}
\newcommand{\rbb}{\mathbb{R}}
\newcommand{\sbb}{\mathbb{S}}
\newcommand{\tbb}{\mathbb{T}}
\newcommand{\ubb}{\mathbb{U}}
\newcommand{\vbb}{\mathbb{V}}
\newcommand{\wbb}{\mathbb{W}}
\newcommand{\xbb}{\mathbb{X}}
\newcommand{\ybb}{\mathbb{Y}}
\newcommand{\zbb}{\mathbb{Z}}


%===============================================================================
%=============================== SPECIAL SYMBOLS ===============================
%===============================================================================


% Orbit (group theory)
\newcommand{\orbit}{\mathcal{O}}
% Normal group
\newcommand{\normal}{\mathcal{N}}
% Indicator function
\newcommand{\indicator}{\mathbbm{1}}
% Laplace transform
\newcommand{\laplace}[1]{\mathcal{L}}
% Epsilon shorthand
\newcommand{\eps}{\varepsilon}
% Omega
\newcommand{\om}{\omega}
\newcommand{\Om}{\Omega}

%===============================================================================
%================================== OPERATORS ==================================
%===============================================================================


% Inverse exponent
\newcommand{\inv}[0]{^{-1}}
% Overline bar
\newcommand{\olsi}[1]{\,\overline{\!{#1}}}
% Less than or equal slanted
\newcommand{\seqs}{\leqslant}
% Greater or equal slanted
\newcommand{\geqs}{\geqslant}
% Subset or equal
\newcommand{\sub}{\subseteq}
% Proper subset
\newcommand{\prosub}{\subset}
% from
\newcommand{\from}{\leftarrow}

% Parantheses
\newcommand{\para}[1]{\left( #1 \right)}
% Curly Braces
\newcommand{\curl}[1]{\left\{ #1 \right\}}
% Brackets
\newcommand{\brac}[1]{\left[ #1 \right]}
% Angled Brackets
\newcommand{\ang}[1]{\left\langle #1 \right\rangle}
% Norm
\newcommand{\norm}[1]{\left\| #1 \right\|}

% Piece wise (use \\ between cases)
\newcommand{\piecewise}[1]{\begin{cases} #1 \end{cases}}

% Bold symbol shorthand
\newcommand{\bl}{\boldsymbol}

% Vertical space
\newcommand{\vs}[1]{\vspace{#1 pt}}
% Horizontal ertical space
\newcommand{\hs}[1]{\hspace{#1 pt}}


%===============================================================================
%============================== TEXT BASED SYMBOLS =============================
%===============================================================================


% Radians
\newcommand{\rad}{\text{rad}}
% Least Common Multiple
\newcommand{\lcm}{\text{lcm}}
% Automorphism
\newcommand{\Aut}{\text{Aut}}
% Variance
\newcommand{\var}{\text{Var}}
% Covariance
\newcommand{\cov}{\text{Cov}}
% Cofactor (matrix)
\newcommand{\cof}{\text{Cof}}
% Adjugate (matrix)
\newcommand{\adj}{\text{Adj}}
% Trace (matrix)
\newcommand{\tr}{\text{tr}}
% Standard deviation
\newcommand{\std}{\text{Std}}
% Correlation coefficient
\newcommand{\corr}{\text{Corr}}
% Sign
\newcommand{\sign}{\text{sign}}

% And text
\newcommand{\AND}{\text{ and }}
% Or text 
\newcommand{\OR}{\text{ or }}
% For text 
\newcommand{\FOR}{\text{ for }}
% If text
\newcommand{\IF}{\text{ if }}
% When text
\newcommand{\WHEN}{\text{ when }}
% Where text
\newcommand{\WHERE}{\text{ where }}
% Then text
\newcommand{\THEN}{\text{ then }}
% Such that text
\newcommand{\SUCHTHAT}{\text{ such that }}

% 1st
\newcommand{\st}[1]{#1^{\text{st}}}
% 2nd
\newcommand{\nd}[1]{#1^{\text{nd}}}
% 3rd
\newcommand{\rd}[1]{#1^{\text{rd}}}
% nth
\newcommand{\nth}[1]{#1^{\text{th}}}


%===============================================================================
%========================= PROBABILITY AND STATISTICS ==========================
%===============================================================================


% Permutation
\newcommand{\perm}[2]{{}^{#1}\!P_{#2}}
% Combination
\newcommand{\comb}[2]{{}^{#1}C_{#2}}

% Baye's risk
\newcommand{\risk}[1]{\mathscr{R}_{#1}}
% Baye's optimal risk
\newcommand{\riskOptimal}[1]{\mathscr{R}_{#1}^*}
% Baye's empirical risk
\newcommand{\riskEmpirical}[2]{\hat{\mathscr{R}}_{#1}^{#2}}


%===============================================================================
%=================================== CALCULUS ==================================
%===============================================================================


% d over d derivative
\newcommand{\dd}[2]{\dfrac{d#1}{d#2}}
% partial d over d derivative
\newcommand{\partialdd}[2]{\dfrac{\partial #1}{\partial #2}}
% delta d over d derivative
\newcommand{\deltadd}[2]{\dfrac{\Delta #1}{\Delta #2}}

% Integration between a and b
\newcommand{\integral}[4]{\int_{#1}^{#2} #3 \, #4}
% Integration in some space 
\newcommand{\boundIntegral}[2]{\int_{#1} #2 \, d#1}

% Limit
\newcommand{\limit}[3]{\lim_{#1 \to #2} #3}


%===============================================================================
%================================  BIG SYMBOLS  ================================
%===============================================================================


% Sum
\newcommand{\sumof}[2]{\sum_{#1}^{#2}}
% Product
\newcommand{\productof}[2]{\prod_{#1}^{#2}}
% Union
\newcommand{\unionof}[2]{\bigcup_{#1}^{#2}}
% Intersection
\newcommand{\intersectionof}[2]{\bigcap_{#1}^{#2}}
% Or
\newcommand{\orof}[2]{\bigvee_{#1}^{#2}}
% And
\newcommand{\andof}[2]{\bigwedge_{#1}^{#2}}


%===============================================================================
%=============================== LINEAR ALGEBRA ================================
%===============================================================================


% Bold vector arrow
\newcommand{\bv}[1]{\vec{\mathbf{#1}}}

% Matrix or vector (use // for column, & for row) 
% with brackets
\newcommand{\bmat}[1]{\begin{bmatrix} #1 \end{bmatrix}}
% Matrix or vector (use // for column, & for row) 
% with curved brackets
\newcommand{\pmat}[1]{\begin{pmatrix} #1 \end{pmatrix}}
% Matrix or vector (use // for column, & for row) 
% with lines on either side 
\newcommand{\lmat}[1]{\begin{vmatrix} #1 \end{vmatrix}}
% Matrix or vector (use // for column, & for row) 
% with curly braces
\newcommand{\cmat}[1]{\begin{Bmatrix} #1 \end{Bmatrix}}
% Matrix or vector (use // for column, & for row) 
% with no braces
\newcommand{\mat}[1]{\begin{matrix} #1 \end{matrix}}


%===============================================================================
%================================ LARGE OBJECTS ================================
%===============================================================================

% Multiple lines
\newcommand{\multiline}[1]{
\begin{align*}
    #1
\end{align*}
}

% Multiple lines with equation numbers
\newcommand{\eqmultiline}[1]{
\begin{align*}
    #1
\end{align*}
}

% Color
\newcommand{\colorText}[2]{
\begingroup
\color{#1}
    #2
\endgroup
}

% Centered figure
\newcommand{\centerFigure}[2]{
    \begin{figure}[h]
        \centering
            #1
        \caption{#2}
    \end{figure}
}

% Tikz figure
\newcommand{\tikzGraphic}[1]{
    \begin{center}
    \begin{tikzpicture}
        #1
    \end{tikzpicture}
    \end{center}
}

% Enumerate numbers (seperate by \item)
\newcommand{\numbers}{\textbf{\number*)}}

% Enumerate letters (seperate by \item)
\newcommand{\letters}{\textbf{\alph*)}}

\title{%
    \Huge Machine Learning \\
    \Large Math Prerequisites II
}
\date{2025-04-14}
\author{Michael Saba}

\begin{document}
\pagenumbering{gobble}
\maketitle
\newpage
\setlength{\parindent}{0pt}
\pagenumbering{arabic}

    \section*{Probability}

    \subsection*{Introduction}

    Several fields in math are concerned with
    formalizing subjects that up to one point
    had been taken for granted.
    For instance, Analysis can be thought of as
    the theory that allows Calculus to work,
    and a natural evolution of the field. \\
    
    Formalizing not only offers precise definitions
    that can be used to prove theorems and
    advance the field,
    but also carries the benefit of 
    allowing generalizations by abstracting
    lower level details and focusing on the core
    properties of a particular subject. \\
    A pertinent example of this is Abstract Algebra,
    which abstracts away the details of which specific
    set is being worked on,
    and instead focuses on the abstract idea of
    algebraic structures being worked on using binary
    operators. \\

    The field of Measure Theory,
    which will serve as a launchpad for the study
    of Probability Theory,
    if a formalization and generalization of the
    notion of measuring things. \\

    \newpage

    \subsection*{Measure Theory}

    Measure Theory is about formalizing,
    abstracting, and generalizing the idea of
    measurements, however, we will be using 
    applied and common examples of measurements
    in order to motivate axioms and definitions. \\

    A common measurement everyone should be
    familiar with is the measurement of length.
    Given some segment, 
    we can measure its length,
    which is represented by a positive real number. \\
    This notion can be extended to two dimensions,
    and three dimensions,
    measuring areas and volumes respectively. \\
    
    As mentioned earlier,
    formalization and generalization
    requires us to standardize what we mean by measure,
    and to focus on the core of measurement
    while abstracting away details of any specific
    measurement. \\

    For instance,
    we can formalize the notion of measurements
    by noting that the length,
    area, and volume are no more than functions
    \[ L:\pcal{\R} \to [0, \infty) \]    
    \[ A:\pcal{\R^2} \to [0, \infty) \]
    \[ V:\pcal{\R^3} \to [0, \infty) \]
    that map subsets of some global space
    to a non-negative real number. \\
    Here $\pcal{A}$ is the powerset 
    (set of all subsets) of $A$. \\
    So, for example,
    the area of a unit square
    with the bottom left corner at the origin
    is $A(S)$ where 
    \[ S = \{ (x, y) \in \R^2 \mid 
    0 \leq x, y \leq 1  \} \]
    We can think of this as taking a subset cotaining
    every point $(x, y)$ within the square,
    (which is just a section on the plane $\R^2$)
    and inputting it into the are function. \\

    We can now start stripping away details
    and focusing on the core of measurements:
    For instance, we can have any set act
    as our global space.
    We can call this set $\xcal$. \\
    Moreover, the set of elements that we can
    assign a measuremet to does not have to be
    the powerset of $\xcal$ (every subset)
    It can be a subset of the powerset
    (some subsets). \\

    However, we do expect some restrictions
    on which subsets can actually be measured.
    For instance, if we can measure some
    subset $A$, and some disjoint subset $B$,
    then it makes sense to be able to measure
    their union, as it is the joining of two
    measurable subsets.
    We should likewise be able to measure the whole,
    $\xcal$ (as a reference point for
    other measurements),
    as well as any complements of measurable subsets. \\

    Moreover, we expect some restrictions on the
    measurements we make. \\
    For instance, if we take the measurement $\mu$
    of some subset $A$, and the measurement
    of some disjoint subset $B$,
    the measurement of their union should
    be the sum of their measurement:
    \[ \mu(A \cup B) = \mu(A) + \mu(B) \]
    this rule also ensures are measurements
    have internal consistency. \\
    We also expect that for instance,
    the empty set has measure 0. \\

    Now, we can define measurable space. \\

    Suppose we have a set $\xcal$.
    A $\sigma$-algebra $X$ on that set $\xcal$
    is a collection of subsets of $\xcal$
    (in other words, $X \subseteq \pcal{\xcal}$)
    that satisfies three axioms:
    \begin{enumerate}
        \item 
        $\xcal \in X$
        \item 
        Closure under complements:
        \[A \in X \implies \olsi{A} \in X
        \text{ where } \olsi{A} = (\xcal - A) \]
        \item 
        Closure under countable unions:
        \[A, B \in X \text{ such that }
        A \cap B = \emptyset
        \implies A \cup B \in X \]
    \end{enumerate}
    We call the tuple $(\xcal, X)$ a measurable space.
    The set $\xcal$ contains the global sample space,
    and the set $X$ is the set of all measurable
    subsets (sections) of $\xcal$. \\
    Note that it matches our expectations regarding
    measuring sets. \\

    Notice that our axioms imply that
    for any measurable space $(\xcal, X)$
    \[ \emptyset \in X \]
    this is because
    \[ \xcal \in X \implies \olsi{\xcal} \in X 
    \implies \emptyset \in X\]
    Moreover, the $\sigma$-algebra is closed
    under countable intersections as well as unions:
    \[ A, B \in X \implies A \cap B \in X \]
    this is because
    \[ A, B \in X \implies \olsi{A}, \olsi{B} \in X 
    \implies \olsi{A} \cup \olsi{B} \in X \] 
    \[ \olsi{A} \cup \olsi{B} \in X
    \implies \olsi{(\olsi{A} \cup \olsi{B})} \in X
    \implies \olsi{(\olsi{A})} 
    \cap \olsi{(\olsi{B})} \in X 
    \implies A \cap B \in X \]
    With that, our axioms allow us to measure
    unions, intersections, and complements
    of measurable sets. \\

    Now, to rigorously define a measure.
    A measure on a measurable space $(\xcal, X)$
    is a function
    \[ \mu: \xcal \to [0, \infty) \]
    that satisfies the following axioms:     
    \begin{enumerate}
        \item 
        $\mu(\emptyset) = 0$
        \item 
        Countable additivity states that if
        $(A_n)_{n \in \N} \in X$ are disjoint subsets:
        \[ (A_n)_{n \in \N} \in X \implies 
        \mu\para{\bigcup_{n \in \N}A_n} 
        = \sum_{n \in N} \mu(A_n) \]
    \end{enumerate}
    These axioms basically ensure that the measure
    is internally consistent;
    if we measure two subsets (with no commonoalities),
    we expect the measure of their union
    (the set containing both subsets)
    to be the sum of their individual measures. \\

    Note that this implies that if a subset $A$
    contains another subset $B$,
    then $A$ must have a measure larger
    or equal to that of $B$. \\
    This is because $A$ can be expressed
    as the union of two disjoint sets:
    \[ A = B \cup (A-B) \]
    which implies that
    \[ \mu(A) = \mu(B) \cup \mu(A-B) \]
    And, since the measure is a non-negative
    real $x \in [0, \infty)$,
    this must mean that:
    \[ \mu(A) = \mu(B) + x 
    \implies \mu(A) \geq \mu(B) \]
    This is an important property of measures,
    and it shows how countable additivity
    ensures monotonicity. \\

    We call a triplet $(\xcal, X, \mu)$
    a measure space.
    It's a measurable space equipped with a measure. \\

    Another concept that can be introduced is that
    of measurable functios. \\
    Given two measurable spaces 
    $(\xcal, X)$ and $(\ycal, Y)$,
    we call
    \[ f: \xcal \to \ycal \]
    a measurable function if
    for any measurable subset $A \in Y$
    \[ f^{-1}(A) \in X \]
    we know already that $f^{-1}(A) \subseteq \xcal$,
    but we need to ensure that if $A$ is measurable
    in $\ycal$,
    its preimage is as measurable as well,
    but in $\xcal$. \\eq
    This operation corresponds to converting
    between measurable spaces,
    and will be useful for defining
    random variables rigorously. \\

    Another concept that needs introduction
    is that of a push-forward measure. \\
    Suppose we have a measure space
    $(\xcal, X, \mu)$
    and a measurable space $(\ycal, Y)$
    such that
    \[ f: \xcal \to \ycal \]
    is a measurable function. \\
    Then we can define the function
    \[ \mu \circ f\inv \{S\} = \mu(f\inv (S)) 
    = \mu(f \in S) \]
    where $S \in \ycal$ to be the push-forward
    measure of $\mu$ under $f$. \\
    Let's actually examine what it does.
    First, we take a measurable subset $S$
    in the space of $\ycal$.
    We then use the inverse $f\inv$ to convert
    this subset back to $\xcal$,
    where we can take its measurement using $\mu$
    (which is guaranteed to work by the fact
    that $f$ is measurable). \\
    In other words, if we have a measurable
    mapping from one measurable space to another,
    then we can use the measure of the first
    space to assign measurements to subsets
    in the second space. \\
    As we mentioned earlier,
    we can use measurable functions to define
    random variables.
    So a push-forward measure is a way to assign
    probabilities to random variable events
    as if they were part of the probability 
    space (which is what we call the distribution
    of the random variable). \\

    We call a subset of the real numbers $\R$
    open if we can move a little bit in the
    positive and negative directions and still
    be inside the subset. \\
    More formally, a set $S \subseteq \R$
    is open if there exists some $\epsilon > 0$
    such that for each element $s \in S$,
    the range of numbers
    $(s - \epsilon, s + \epsilon)$
    (not including the endpoints) is in $S$:
    \[
    S \text{ is open } \iff 
    \forall s \in S\; \exists \varepsilon > 0 
    \text{ such that } 
    (s - \varepsilon, s + \varepsilon) 
    \subseteq S \] \\

    The Borel $\sigma$-algebra $\bscr$
    is the smallest $\sigma$-algebra
    on $\R$ that contains
    all of the open subsets of $\R$. \\
    As we know, a $\sigma$-algebra has to be closed
    under complements and finite unions. 
    We can thus form $\bscr$ by taking all
    of the open subsets on $\R$,
    their complements,
    and finite unions. \\
    
    Suppose that $(\xcal, X)$ is a measurable
    space,
    and that $A \in X$ is a measurable subset. \\
    Then the indicator function of $A$ is a function:
    \[ \mathbbm{1}_A: \xcal \to \{0, 1\} \]
    where
    \[
        \mathbbm{1}_A(x) = 
        \piecewise
        {1, \quad \text{ if } x \in S }
        {0, \text{ if } x \notin S}
    \]

    As mentioned earlier,
    the length, area, and volume are commonly
    used measures. \\
    When formalized, they are called
    the Lebesgue Measures,
    a family of measures that for $\R$,
    $\R^2$, and $\R^3$,
    represent the length, are and volume,
    but which can be extended to higher
    dimensions. \\
    We denote the Lebesgue measure using
    $\lambda$. \\

    If we have two measures $\mu$ and $\gamma$
    defined on some measurable space
    $(\xcal, \X)$, then we say
    $\mu$ is absolutely continuous with respect
    to $\gamma$ if
    \[ \gamma(A) = 0 \implies \mu(A) = 0 \]
    for all $A \in \X$. \\
    This is the case if
    \[ \mu(A) = \int_{A}p(x)d\gamma(x) \]
    for some measurable function 
    \[ p: \xcal \to [0, \infty) \]
    Notice that all this is saying is that
    if the measurable set $A$
    measures 0 according to one measure,
    it must also do so according to
    the other measure. \\

    Given a measurable space
    $(\xcal, \X)$
    equipped with some measure $\mu$
    (measure space  $(\xcal, \X, \mu)$),
    we want to define a function
    \[f:\xcal \to [0, \infty) \]
    such that the measure of some subset
    $A \in \X$ is equal to the area
    under $f$ within the range of values $A$
    \[ \mu(A) = \int_{A}f(x)dx \]
    We can call this a density function. \\
    Of course, note here that
    we don't know what set $\xcal$ is,
    and integrals are usually tied closely
    to the notion of lengths and real numbers
    ($dx$ is a small length in $\R$). \\
    Instead, we define an integral in terms
    of some measure $v$ that is actually
    defined on $(\xcal, \X)$:
    \[ \mu(A) = \int_{A}f(x)dv(x) \]
    Note that in the original integral
    we showed, the measure that was (implicitely)
    used was the Lebesgue measure $\lambda$
    defined on vector spaces $\R^n$. \\
    Finally, we note that for $f$
    to be a density function,
    it needs to be measurable. \\
    Measurability ensures that the preimage of
    $f$ is measurable in its original space.
    If we want the area under $f$ to refer
    to the measure $\mu$ of any subset $A$,
    then its preimage needs to be measurable
    in $\xcal$,
    so it can be measured by $\mu$. \\

    Here, we can write the density function
    \[ f = \dd{\mu}{v} \]
    called the Radon-Nikodym derivative. \\

    Notice that the existence of a 
    density function implies that the measure
    $\mu$ is absolutely continuous
    with regards to $\gamma$. \\

    The final theorem we will look
    at is Fubini's Theorem. \\
    Suppose we have two $\sigma$-finite
    measure spaces:
    \[ (\xcal, \X, \mu)
    \AND (\ycal, \Y, v) \]
    then there exists a joint measure
    space:
    \[ (\xcal \times \ycal, 
    \X \otimes \Y) \]
    where $\xcal \times \ycal$
    is the cartesian product containing
    all ordered pairs with one element
    from $\xcal$ and one element from $\ycal$.
    This means that the list of outcomes
    of this measurable space contains
    every possibile combination of outcomes
    from $\xcal$ and $\ycal$. \\
    The term $\X \otimes \Y$,
    called the $\sigma$-algebra product,
    is the smallest $\sigma$-algebra
    containing the cartesian product
    $\X \times \Y$. \\
    When we take this cartesian product,
    there is no guarantee that the resulting
    set will fulfill the closure requirements
    of a $\sigma$-algebra.
    So if we were to take every element in
    $\X \times \Y$, their complements,
    and their finite unions,
    we would end up with the smallest $\sigma$-algebra
    that contains $\X \times \Y$. \\
    To conclude 
    \[ (\xcal \times \ycal, \X \otimes \Y) \]
    is a measurable space that allows us
    to measure every combination of outcomes
    in $\xcal$ and $\ycal$. \\
    In order to make this measure,
    we use a unique measure $\mu \otimes v$
    called the product measure,
    where for any $A \in \X$ and
    $B \in \Y$
    \[ \mu \otimes v(A, B) =
    \mu(A) \cdot v(B)  \]
    Similarly to any other measure space,
    we should be able to integrate
    over this space with respect to
    the product measure. \\
    So for any integrable function:
    \[ f:\xcal \times \ycal
    \to \R \]
    we can evaluate this integral
    over the joint measure space
    as a double integral
    over each measure space individually:
    \[ \int_{\xcal \times \ycal}
    f(x,y)(\mu \otimes v)(dx, dy)
    = \int_{\xcal}  \int_{\ycal}
    f(x,y)v(dy)\mu(dx) \]
    We can also flip the other:
    \[ \int_{\xcal \times \ycal}
    f(x,y)(\mu \otimes v)(dx, dy)
    = \int_{\ycal}  \int_{\xcal}
    f(x,y)\mu(dx)v(dy) \] \\

    \newpage

    \subsection*{Probability Theory}

    Our goal is to study the theory behind
    probability,
    but in order to motivate this theory,
    as well as the previously discussed
    Measure Theory,
    we need to anchor it to a tangible example. \\
    
    Suppose we roll a fair six-sided die. \\
    There are 6 possible outcomes:
    \[ \{ 1, 2, 3, 4, 5, 6 \} \]
    This is called our sample space. \\
    In that sample space,
    we can measure the probability of any
    outcome or event.
    Here an event is an outcome or collection of
    outcomes, and the probability of that event
    taking place is the probability that any 
    outcome in that event takes place.
    Each event can be thought of as a subset of
    this sample space. \\
    For example, the event that we roll
    an even number is represented by the subset
    \[ \{2, 4, 6\} \]
    In this case, every subset of the sample
    space can easily be assigned a probability.
    Since the probability of each outcome
    taking place is $\sfrac{1}{6}$,
    the probability that any event will transpire
    is just the that probability multiplied
    by the number of outcomes present in an event. \\

    As expected, we can model this using a measure
    space. \\
    For instance, the sample space will be our
    global set, which we denote $\Omega$. \\
    The set of all events whose probability
    we want to measure is the set of 
    measurable subsets of $\Omega$,
    the $\sigma$-algebra which we will
    denote $\fscr$. \\
    Finally, the probability is a measure
    $\Pb$ defined on the measurable
    space $(\Omega, \fscr)$,
    with $\Pb\{\Omega\} = 1$
    (It is certain that one outcome
    in the sample space will take place,
    so the probability that that will transpire
    has to be 1). \\
    
    This measure space:
    \[ (\Omega, \fscr, \Pb) \]
    is then called a probability space. \\

    In fact, any measure space
    $ (\xcal, X, \mu) $
    is a probability space,
    so long as $\mu\{\xcal\} = 1$. \\

    Now, as we know a random variable
    assigns a number to outcomes
    in our probability space. \\
    For example, take the random
    variable whose value is $0$
    if the roll is even, and $1$
    if the roll is odd. \\
    In probability, we usually
    want to be able to measure the probability
    that a random variable has
    some value or range of values. \\
    Notice that in order for that to happen,
    the space in which the random variable
    is defined needs to be measurable 
    as well. \\
    We also need to a way to convert from
    events in the space of the random variable
    back to the probability space in order to
    measure the probability of said events. \\

    A measurable function
    \[ X: \Omega \to \xcal \]
    from the probability space to the measurable
    space $(\xcal, \X)$
    is called a random element.
    If the set $\xcal$ contains numbers,
    then $X$ is more specifically a random variable. \\
    Notice that measurable functions are ideal
    because they can convert our probability
    space to another measurable space. \\
    Moreover, they guarantee that the inverse
    of a measurable event in this space
    is also measurable in the original probability
    space. \\

    Now, as we know, the distribution of a
    random element is no more than an assignment
    of a probability to each outcome $x$
    in the random elements's space. \\
    In order to get this probability,
    we need to convert back to the probability
    space using the inverse of the random element's
    function $X$.
    We then use the probability measure $\Pb$
    on the preimage to get the probability
    of the random variable's outcome:
    \[ \Pb \circ X\inv \{x\} 
    = \Pb(X\inv (x)) \]
    Notice that this is no more than the
    push-forward measure of $\Pb$ on $X$. \\
    In fact, $\Pb_X$ can be thought of
    as a measure defined on the measurable
    space $(\xcal, \X)$
    (since it assigns a measure to elements
    in $\xcal$). \\

    Recall that when we defined
    push-forward measures,
    we wrote
    \[ \mu \circ f\inv \{S\} = \mu(f\inv (S)) 
    = \mu(f \in S) \]
    and 
    $\mu(f \in S)$
    did not make much sense at the time.
    However, in the context of random variables,
    what we are saying is that:
    \[ \Pb_X(A) = \Pb(X \in A) \]
    which is notation we have seen before
    for random variables. \\

    In fact, we have a name for the
    function that describes the distribution
    of a random element $X$.
    It is denoted by 
    \[ \Pb_X = \Pb \circ X\inv \]
    which is called the probability
    distribution of $X$. \\

    If two random elements $X$ and $Y$
    on a probabilty space
    have the same distribution
    \[ \Pb_X = \Pb_Y \]
    then $X$ and $Y$ are equally
    distributed. \\

    Now, what if we have two random elements
    defined on the same probability space,
    describing different outcomes. \\
    This means we have two measurable
    space $(\xcal, \X)$
    and $(\ycal, \Y)$,
    and two random elements:
    \[ X: \Omega \to \xcal \qquad 
    Y: \Omega \to \ycal \]
    Then we can define a new measurable space:
    \[ (\xcal \times \ycal, \X \otimes \Y) \]
    Called the joint space of $X$
    and $Y$. \\
    We can map to this joint space using
    a random element:
    \[ (X, Y): \Omega \to \xcal \times \ycal  \]
    where for each $\omega \in \Omega$
    \[ (X, Y)(\omega) = (X(\omega), Y(\omega)) \]
    which can be shown to be a measurable
    function. \\
    If we were to then measure the probabilities
    of outcomes in this new joint random variable,
    we would get a distribution:
    \[ \Pb_{X, Y} = \Pb \circ (X, Y)\inv \]
    called the joint probability distribution
    of $X$ and $Y$,
    which describes the probability that 
    two events, one from $\xcal$
    and one from $\ycal$,
    would occur at the same time.
    It does so by taking the measure of all
    combinations of outcomes in
    $\xcal$ and $\ycal$ that make up
    said events. \\

    When we have a joint distribution
    of two random elements $X$ and $Y$,
    there is nothing stopping us from measuring
    the distributions $\Pb_X$ and $\Pb_Y$
    regardless. \\
    As expected, this measures the probability
    that events take place in each of
    $\xcal$ and $\ycal$,
    with no regard to events in the other. \\
    We call these distributions marginal. \\

    We know that there are continuous
    and discrete random variables. \\
    We also know that in the case of continuous
    random variables, the distribution follows
    what's called a probability density 
    function $f$,
    where the area under $f$ gives the
    distribution of the random variable. \\
    Now is the time to formalize that notion. \\

    For a random variable $X$
    defined on a measurable space $(\xcal, \X)$,
    equipped with a probability measure $\Pb_X$,
    we can create density functions using
    a measurable function:
    \[ f:\xcal \to [0, \infty) \]
    where
    \[ \Pb_X(A) = \int_{A}f(x)dv(x) \]
    for some measure $v$ on $\xcal$,
    commonly the Lebesgue measure
    if $\xcal = \R^n$. \\
    
    Notice that this implies that the 
    measure $\Pb_X$ is absolutely
    continuous with regards to the measure $v$.
    (the space we are integrating over). \\
    Notice that this implies that
    \[ v(A) = 0 \implies \Pb_X(A) = 0 \]
    which means that if a section $A$
    has a measure $v(A) = 0$,
    the probability that $A$ would happen in $0$
    as well. \\
    When we are dealing with real numbers,
    in $\R$ for example,
    this is easier to understand.
    The measure $v$ is the Lebesgue
    measure $\lambda$,
    which describes the length along $\xcal$.
    So what we are saying is that,
    if the length is so short that it equals
    to 0 (for example, a single point on $\R$),
    then the probability this outcome will take
    place is also 0,
    which makes sense since the area above a point
    is a line with area 0. \\

    We will be working with densities in $\R$
    for the most part,
    so if $\xcal = \R$,
    then we can define a density function $f$
    of a random variable $X$ as just:
    \[ \Pb(X \in A) = \Pb(A) = 
    \int_Af(x)dx \]
    which is to say, the probability
    that an event in $A$ occurs is the area
    under $f$ constrained to the $A$ interval
    (in this case measured with the Lebesgue measure,
    length). \\

    A famous example of a continuous
    distribution is the normal distribution,
    which has a density function:
    \[ f(x) = \dfrac{1}{2\sqrt{2\pi \sigma^2}}
    e^{\para{-\dfrac{(x-\mu)^2}{2\sigma^2}}} \]
    for any $x \in \R$. \\
    The normal, or Guassian distribution,
    is denoted $\mathcal{N}(\mu, \sigma)$,
    with mean $\mu$ and standard
    deviation $\sigma$. \\

    The joint distribution $\Pb_{(X, Y)}$
    is also accompanied by a joint density
    $p_{(X, Y)}$. \\ 

    The transformation law
    gives us a way to get the density function
    of a random variable $Y$ in $\R^n$
    given the density
    function of another $\R^n$ valued
    random variable $X$,
    and a measurable function $f: X \to Y$
    that maps $X$ to $Y$. \\
    So, given the density $p_X$ of $X$,
    we have
    \[ p_Y(y) = p_X(g\inv(y))\cdot|\det(J_{g\inv})| \]
    where $J_{g\inv}$ is the Jacobian
    of the inverse.

    \newpage

    \subsection*{Expected Value and Variance}
    
    If we define an integrable random varibale:
    \[ X: \Omega \to \R \]
    from the probability space
    $(\Omega, \fscr, \Pb)$,
    then the expected value
    of that random variable is:
    \[ \E[X] = \int_{\Omega}X(\omega)
    \Pb(d\omega) \]
    That is to say,
    it is the sum of all events
    in $X$ weighted by the probability
    of their happenning. \\
    We can think of the expected value
    as the average of the distribution
    of $X$. \\

    Now suppose we have a function
    \[ f: \xcal \to \R \]
    we want to try and find
    the expected value of $f(X)$. \\
    As such:
    \[ \E[f(X)]
    = \int_{\Omega}(f \circ X)(\omega)
    \Pb(d\omega) \]
    Here, note that $x = X(\omega)$,
    which means that $\omega = X\inv(x)$.
    So if we switch our variable to $x$:
    \[\int_{\xcal}f(x)(\Pb \circ X\inv)(dx) \]
    Then we notice that we have $\Pb \circ X\inv$,
    which is the push-forward measure of $\Pb$
    under $X$, also called $\Pb_X$,
    the probability measure on $\xcal$:
    So we have
    \[ \int_{\xcal}f(x)\Pb_X(dx) \]
    Notice here that in the result,
    the integral is basically just the expected
    value inetgral, but evaluated
    in $\xcal$ using its own probability
    measure $\Pb_X$,
    instead of $\Omega$ and the probability
    measure $\Pb$. \\
    So we can say:
    \[ \E[f(X)] 
    = \int_{\xcal}f(x)\Pb_X(dx) 
    = \E_X[f] \]
    where $\E_X$ is just the expected value,
    evaulated in $\xcal$,
    using $\Pb_X$. \\

    If we have a discrete (non integrable)
    random variable
    \[ X: \Omega \to \R \]
    then the expected value is a sum:
    \[ \sum_{i}x_i\Pb(X = x_i) \]
    This clearly has the same meaning
    of "average" of $X$. \\

    Likewise, we can generalize this idea
    to any measuable function of the random variable,
    such as $f(X)$, where
    \[ f: \R \to \R \]
    Here, we can see that
    \[ \E[f(X)] = \sum_{i}f(x_i)\Pb(X = x_i) \]
    which is the discrete analogue to
    integrating with respect to the law
    $\Pb_X$ instead of $\Pb$. \\

    If we have the probability
    density function $P_X$ of a continuous
    random variable
    \[ X: \Omega \to \R \]
    then the expected value of $X$
    can instead be given by:
    \[ \E[X] = \int_{\R} x P_X(x) dx \]
    which again clearly describes the average
    of the values of $X$. \\
    
    And, for any measurable function
    \[ f: \R \to \R \]
    we have
    \[ \E[f(X)] = \int_{\R} f(x)P_X(x) dx \]
    which corresponds to
    integrating with respect to the law
    $\Pb_X$ instead of $\Pb$. \\

    For integrable random variables $X$ and $Y$,
    and some numbers $a, b \in \R$,
    here are some properties of expected values:
    \begin{enumerate}
        \item 
        Linearity:
        \[ \E[aX + bY] = a\E[X] + b\E[Y] \]
        \item 
        Monotonicity: If $X \leq Y$, then 
        \[ \E[X] \leq \E[Y] \]
        \item
        Non-negativity: If $X \geq 0$, then
        \[ \E[X] \geq 0 \]
        \item
        Constant rule: For any constant $c \in R$,
        \[ \E[c] = c \]
        which makes sense since the average value
        of a constant is its own value.
        \item
        Indicator function:
        for any measurable subset $A \in \fscr$
        \[ \E[1_A] = \Pb(A) \]
        this makes sense, since
        we are taking every event $a \in A$,
        multiplying it by the probability
        it will take place, and summing the total,
        which is exactly the probability that
        some event in $A$ will take place. \\
    \end{enumerate}

    While the expected value gives us the average,
    the variance of a square-integrable
    random variable
    \[ X: \Omega \to \R \]
    gives us the spreadness:
    \[ \var(X) = \E[(X - \E[X])^2] \]
    it basically calculates the distance
    between each point in $X$
    and the average of $X$,
    giving us how spread out the
    distribution is. \\
    The standard deviation is defined as:
    \[ \std(X) = \sqrt{\var(X)} \]
    which calculates the average
    from any element to the mean of the
    distribution. \\
    Both the variance and standard devation
    are always positive. \\

    We can also relate the distribution
    of two random variables
    \[ X, Y: \Omega \to \R \]
    using the covariance:
    \[ \cov(X, Y) = 
    \E[(X - \E[X])(Y - \E[Y])] \]
    Note that $\cov(X, X)$
    is just the variance $\var(X)$. \\

    We can go further and define
    the correlation coefficient
    \[ \corr(X, Y) = \dfrac{\cov(X, Y)}
    {\sqrt{\var(X)\var(Y)}} \]
    which measure linear correlation
    between random variables. \\
    The correlation is a value in $[-1, 1]$,
    where $-1$ means there is an inverse
    correlation, $1$ means there is a
    strong correlation, and $0$
    means the events are independent,
    not correlated. \\

    If two random variables $X$ and $Y$
    are independent of each other,
    then their covariance is $\cov(X, Y)$
    is $0$. \\
    
    We can also extend the idea of variance
    to random variables in $\R^n$:
    \[ X: \Omega \to \R^n \]
    which we call the covariance matrix
    of $X$:
    \[ \cov(X) = 
    \E[(X - \E[X])(X - \E[X])^T] \]
    expanding this outer product,
    we get a matrix where each entry is
    \[ (\cov(X))_{ij} = \cov(X_i, X_j) \]
    Moreover, the covariance matrix
    is positive semi definite. \\

    We can also get the covariance
    matrix for a vector that has
    block elements. \\
    Suppose that we have the random vectors
    \[ X: \Omega \to \R^n
    \qquad Y: \Omega \to \R^d \]
    If we have a vector
    \[ \vecTwo{X}{Y} \]
    then its covariance matrix is
    \[ \cov\para{\vecTwo{X}{Y}}
    = \matTwo{\cov(X, X)}{\cov(X, Y)}
    {\cov(Y, X)}{\cov(Y, Y)}
    = \matTwo{\cov(X)}{\cov(X, Y)}
    {\cov(Y, X)}{\cov(Y)} \]
    Here, every entry of the covariance
    matrix is itself a matrix,
    since we treated the entries
    $X$ and $Y$ as block elements. \\
    We have $\cov(X)$ and $\cov(Y)$,
    which are just the covariance 
    matrices of $X$ and $Y$
    (basically the variance in
    multidimensional space). \\
    We have $\cov(X, Y)$
    and $\cov(Y, X)$,
    which can be calculated
    using:
    \[ \cov(X, Y) = 
    \E[(X - \E[X])(Y - \E[Y])^T] \]
    We note that
    \[ \cov(X, Y) = 
    \cov(Y, X)^T \]
    Also note that
    \[ \cov(X, Y) \neq 
    \cov\para{\vecTwo{X}{Y}} \]
    One is the covariance comparing
    $X$ to $Y$,
    the other is the covariance of the
    vector containing $X$ and $Y$,
    which is like teh variance but in
    $\R^{n + d}$ space. \\

    Some properties of variances include:
    \begin{enumerate}
        \item 
        \[ \cov(X, Y) = \E[XY] - \E[X]\E[Y] \]
        \item
        \[ \var(aX + bY)
        = a^2\var(X) + b^2\var(Y)
        + 2ab\cov(X, Y) \]
        \item
        The variance of the sum
        of random variables
        \[ X_1, X_2, \dots X_n \]
        is:
        \[ \var\para{\sum_{i=1}^{n}X_i}
        = \sum_{i=1}^{n}\sum_{j}^{n}
        \cov(X_i, X_j) \]
        \item
        The variance of the sum of
        independent random variables is
        just the sum of the variances:
        \[ \var\para{\sum_{i=1}^{n}X_i}
        = \sum_{i=1}^{n} \var(X_i) \] \\
    \end{enumerate}

    We can show why the last two
    properties are true. \\
    If we have a set of random 
    variables:
    \[ \{ X_1, X_2, \dots X_n \} \]
    then
    \[ \var\para{\sum_{i=1}^{n}X_i}
        = \E\brac{\para{\sum_{i=1}^{n}X_i 
    - \E\brac{\sum_{i=1}^{n}X_i}}^2}\]
    We then use the linearity of expected values:
    \[ \var\para{\sum_{i=1}^{n}X_i}
    = \E\brac{\para{\sum_{i=1}^{n}X_i 
    - \sum_{i=1}^{n}\E[X_i]}^2}\]
    \[ \var\para{\sum_{i=1}^{n}X_i}
    = \E\brac{\para{\sum_{i=1}^{n} (X_i 
    - \E[X_i])}^2}\]
    The square of a sum
    is the sum of the product 
    of every pair in the sum:
    \[ \var\para{\sum_{i=1}^{n}X_i}
    = \E\brac{\sum_{i=1}^{n}\sum_{j=1}^{n}
    (X_i - \E[X_i])(X_j - \E[X_j])}\]
    Again using the linearity of expected values:
    \[ \var\para{\sum_{i=1}^{n}X_i}
    = \sum_{i=1}^{n}\sum_{j=1}^{n}
    \E[X_i - \E[X_i])(X_j - \E[X_j])
    = \sum_{i=1}^{n}\sum_{j=1}^{n}\cov(X_i, X_j) \]
    However, if the variables are mutually
    independent, then
    \[ \cov(X_i, X_j) = 0 \text{ if } i \neq j\]
    \[ \cov(X_i, X_j) = \var(X_i) 
    \text{ if } i = j\]
    So in that case
    \[ \var\para{\sum_{i=1}^{n}X_i} = 
    \sum_{i=1}^{n} \var(X_i) \] \\

    \newpage

    \subsection*{Probability Fundamentals}

    Given a probability space
    $(\Omega, \fscr, \Pb)$,
    and two measurable subsets
    $A, B \in \fscr$,
    then $A$ and $B$
    are independent if
    \[ \Pb(A \cap B) 
    = \Pb(A) \cdot \Pb(B) \]
    where $ \Pb(A \cap B)$
    is the event where both $A$
    and $B$ take place. \\

    We can do the same thing
    with random elements. \\
    Suppose we have random elements
    \[ X: \Omega \to \xcal \qquad
    Y: \Omega \to \ycal \]
    then we say that $X$ and $Y$
    are independent if
    \[ \Pb_{(X, Y)}
    = \Pb_X \otimes \Pb_Y \]
    their joint distribution is the
    product of their marginal distributions. \\
    Note that the same applies
    for their densities.
    If the variables are independent,
    then 
    \[ p_{(X, Y)} = p_X \cdot p_Y \]
    holds. \\

    If we have the following random
    variables
    \[ X_1, X_2, \dots X_n \]
    such that they are all independent,
    then the expected value of their
    product is just the product
    of their expected value:
    \[ \E\brac{\prod_{i=1}^{n}
    X_i} = \prod_{i=1}^{n}\E[X_i] \]
    This can be proven by 
    using the integral or sum
    forumla for the expected
    value, coupled with the fact that
    $ \Pb_{(X, Y)}
    = \Pb_X \otimes \Pb_Y $. \\
    
    Recall that we know that
    \[ \cov(X, Y) = \E[XY] - \E[X]\E[Y] \]
    However, if $X, Y$
    are independent,
    then that implies that 
    \[ \E[XY] = \E[X]\E[Y] \]
    which means that $\cov(X, Y) = 0$. \\
    This is how we can derive the
    result that $\cov(X, Y) = 0$
    when $X$ and $Y$
    are independent. \\

    Given a probability space
    $(\Omega, \fscr, \Pb)$,
    and two events $A, B \in \fscr$,
    if $\Pb(A) \neq 0$,
    then we can define the conditional
    probability:
    \[ \Pb(B \mid A)
    = \dfrac{\Pb(A \cap B)}{\Pb(A)} \]
    where $\Pb(B \mid A)$
    means the probability of $B$
    happening given that we know $A$
    has happened. \\
    We can calculate it by restricting
    our sample space to just $A$,
    and then calculating the probability
    that the outcomes in $B$
    take place. \\

    Note that if we fix $A$,
    then the function
    \[ \Pb_{(\cdot \mid A)}: \fscr \to
    [0, 1] \]
    is a measure,
    specifically the measure of some
    event $B$ taking place given
    that $A$ has taken place. \\
    This measure is only well defined
    for $\Pb(A) \neq 0$
    (since the conditional probability
    divides by $\Pb(A)$). \\

    We know that
    \[ \Pb(B \mid A)
    = \dfrac{\Pb(A \cap B)}{\Pb(A)} 
    \quad \text{ and } \quad 
    \Pb(A \mid B)
    = \dfrac{\Pb(A \cap B)}{\Pb(B)} \]
    we can combine them to form:
    \[ \Pb(B \mid A)
    = \dfrac{\Pb(A \mid B) \cdot \Pb(B)}
    {\Pb(A)} \]
    which is called Baye's rule and allows
    the reversal of conditional
    probabilities. \\

    The law of total probability states that
    \[ \Pb(B)
    = \sum_{i \in I}\Pb(B \mid A_i)
    \cdot \Pb(A_i) \]
    if ${A_i}_{i \in I}$
    is a countable partition of $\fscr$
    (partition means the union of the subsets
    forms $\fscr$, and that they are all
    disjoint). \\
    This makes intuitive sense:
    If we sum the probabilities that $B$
    will take place knowing
    that every event possible in $\fscr$,
    weighted by the probability
    that each of these events takes place,
    then the result is just the probability
    that $B$ takes place (since we conditioned
    on the entirity of the 
    measurable space). \\

    On the other hand, if we also have 
    $\{A_i\}_{i \in I}$
    as a countable partition of $\fscr$,
    then
    \[ \sum_{i \in I}\Pb(A_i \mid B)
    = 1 \]
    This is because we are taking
    the probability that all events
    $A_i$ (which partition $\fscr$)
    will take place, knowing $B$
    has taken place.
    Surely one of them has to take place,
    so the probability is $1$. \\

\end{document}
