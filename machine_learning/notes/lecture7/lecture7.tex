
\documentclass[12pt]{article}

\usepackage[margin=1in]{geometry}


%===============================================================================
%================================== PACKAGES ===================================
%===============================================================================

% For using float option H that places figures 
% exatcly where we want them
\usepackage{float}
% makes figure font bold
\usepackage{caption}
\captionsetup[figure]{labelfont=bf}
% For text generation
\usepackage{lipsum}
% For drawing
\usepackage{tikz}
% For smaller or equal sign and not divide sign
\usepackage{amssymb}
% For the diagonal fraction
\usepackage{xfrac}
% For enumerating exercise parts with letters instead of numbers
\usepackage{enumitem}
% For dfrac, which forces the fraction to be in display mode (large) e
% even in math mode (small)
\usepackage{amsmath}
% For degree sign
\usepackage{gensymb}
% For "\mathbb" macro
\usepackage{amsfonts}
% For positioning 
\usepackage{indentfirst}
\usetikzlibrary{shapes,positioning,fit,calc}
% for adjustwidth environment
\usepackage{changepage}
% for arrow on top
\usepackage{esvect}
% for mathbb 1
\usepackage{bbm}
% for mathsrc
\usepackage[mathscr]{eucal}
% For degree sign
\usepackage{gensymb}
% For quotes
\usepackage{csquotes}
% For vertical lines
\usepackage{mathtools}
% For cols
\usepackage{multicol}

% for tikz
\usepackage{pgfplots}
\pgfplotsset{compat=1.18}
\usepackage{amsmath}
\usepgfplotslibrary{groupplots}


%===============================================================================
%==================================== FONTS ====================================
%===============================================================================


% Mathcal
\newcommand{\acal}{\mathcal{A}}
\newcommand{\bcal}{\mathcal{B}}
\newcommand{\ccal}{\mathcal{C}}
\newcommand{\dcal}{\mathcal{D}}
\newcommand{\ecal}{\mathcal{E}}
\newcommand{\fcal}{\mathcal{F}}
\newcommand{\gcal}{\mathcal{G}}
\newcommand{\hcal}{\mathcal{H}}
\newcommand{\ical}{\mathcal{I}}
\newcommand{\jcal}{\mathcal{J}}
\newcommand{\kcal}{\mathcal{K}}
\newcommand{\lcal}{\mathcal{L}}
\newcommand{\mcal}{\mathcal{M}}
\newcommand{\ncal}{\mathcal{N}}
\newcommand{\ocal}{\mathcal{O}}
\newcommand{\pcal}{\mathcal{P}}
\newcommand{\qcal}{\mathcal{Q}}
\newcommand{\rcal}{\mathcal{R}}
\newcommand{\scal}{\mathcal{S}}
\newcommand{\tcal}{\mathcal{T}}
\newcommand{\ucal}{\mathcal{U}}
\newcommand{\vcal}{\mathcal{V}}
\newcommand{\wcal}{\mathcal{W}}
\newcommand{\xcal}{\mathcal{X}}
\newcommand{\ycal}{\mathcal{Y}}
\newcommand{\zcal}{\mathcal{Z}}

% Mathfrak
\newcommand{\afrak}{\mathfrak{A}}
\newcommand{\bfrak}{\mathfrak{B}}
\newcommand{\cfrak}{\mathfrak{C}}
\newcommand{\dfrak}{\mathfrak{D}}
\newcommand{\efrak}{\mathfrak{E}}
\newcommand{\ffrak}{\mathfrak{F}}
\newcommand{\gfrak}{\mathfrak{G}}
\newcommand{\hfrak}{\mathfrak{H}}
\newcommand{\ifrak}{\mathfrak{I}}
\newcommand{\jfrak}{\mathfrak{J}}
\newcommand{\kfrak}{\mathfrak{K}}
\newcommand{\lfrak}{\mathfrak{L}}
\newcommand{\mfrak}{\mathfrak{M}}
\newcommand{\nfrak}{\mathfrak{N}}
\newcommand{\ofrak}{\mathfrak{O}}
\newcommand{\pfrak}{\mathfrak{P}}
\newcommand{\qfrak}{\mathfrak{Q}}
\newcommand{\rfrak}{\mathfrak{R}}
\newcommand{\sfrak}{\mathfrak{S}}
\newcommand{\tfrak}{\mathfrak{T}}
\newcommand{\ufrak}{\mathfrak{U}}
\newcommand{\vfrak}{\mathfrak{V}}
\newcommand{\wfrak}{\mathfrak{W}}
\newcommand{\xfrak}{\mathfrak{X}}
\newcommand{\yfrak}{\mathfrak{Y}}
\newcommand{\zfrak}{\mathfrak{Z}}

% Mathscr
\newcommand{\ascr}{\mathscr{A}}
\newcommand{\bscr}{\mathscr{B}}
\newcommand{\cscr}{\mathscr{C}}
\newcommand{\dscr}{\mathscr{D}}
\newcommand{\escr}{\mathscr{E}}
\newcommand{\fscr}{\mathscr{F}}
\newcommand{\gscr}{\mathscr{G}}
\newcommand{\hscr}{\mathscr{H}}
\newcommand{\iscr}{\mathscr{I}}
\newcommand{\jscr}{\mathscr{J}}
\newcommand{\kscr}{\mathscr{K}}
\newcommand{\lscr}{\mathscr{L}}
\newcommand{\mscr}{\mathscr{M}}
\newcommand{\nscr}{\mathscr{N}}
\newcommand{\oscr}{\mathscr{O}}
\newcommand{\pscr}{\mathscr{P}}
\newcommand{\qscr}{\mathscr{Q}}
\newcommand{\rscr}{\mathscr{R}}
\newcommand{\sscr}{\mathscr{S}}
\newcommand{\tscr}{\mathscr{T}}
\newcommand{\uscr}{\mathscr{U}}
\newcommand{\vscr}{\mathscr{V}}
\newcommand{\wscr}{\mathscr{W}}
\newcommand{\xscr}{\mathscr{X}}
\newcommand{\yscr}{\mathscr{Y}}
\newcommand{\zscr}{\mathscr{Z}}

% Mathbb
\newcommand{\abb}{\mathbb{A}}
\newcommand{\bbb}{\mathbb{B}}
\newcommand{\cbb}{\mathbb{C}}
\newcommand{\dbb}{\mathbb{D}}
\newcommand{\ebb}{\mathbb{E}}
\newcommand{\fbb}{\mathbb{F}}
\newcommand{\gbb}{\mathbb{G}}
\newcommand{\hbb}{\mathbb{H}}
\newcommand{\ibb}{\mathbb{I}}
\newcommand{\jbb}{\mathbb{J}}
\newcommand{\kbb}{\mathbb{K}}
\newcommand{\lbb}{\mathbb{L}}
\newcommand{\mbb}{\mathbb{M}}
\newcommand{\nbb}{\mathbb{N}}
\newcommand{\obb}{\mathbb{O}}
\newcommand{\pbb}{\mathbb{P}}
\newcommand{\qbb}{\mathbb{Q}}
\newcommand{\rbb}{\mathbb{R}}
\newcommand{\sbb}{\mathbb{S}}
\newcommand{\tbb}{\mathbb{T}}
\newcommand{\ubb}{\mathbb{U}}
\newcommand{\vbb}{\mathbb{V}}
\newcommand{\wbb}{\mathbb{W}}
\newcommand{\xbb}{\mathbb{X}}
\newcommand{\ybb}{\mathbb{Y}}
\newcommand{\zbb}{\mathbb{Z}}


%===============================================================================
%=============================== SPECIAL SYMBOLS ===============================
%===============================================================================


% Orbit (group theory)
\newcommand{\orbit}{\mathcal{O}}
% Normal group
\newcommand{\normal}{\mathcal{N}}
% Indicator function
\newcommand{\indicator}{\mathbbm{1}}
% Laplace transform
\newcommand{\laplace}[1]{\mathcal{L}}
% Epsilon shorthand
\newcommand{\eps}{\varepsilon}
% Omega
\newcommand{\om}{\omega}
\newcommand{\Om}{\Omega}

%===============================================================================
%================================== OPERATORS ==================================
%===============================================================================


% Inverse exponent
\newcommand{\inv}[0]{^{-1}}
% Overline bar
\newcommand{\olsi}[1]{\,\overline{\!{#1}}}
% Less than or equal slanted
\newcommand{\seqs}{\leqslant}
% Greater or equal slanted
\newcommand{\geqs}{\geqslant}
% Subset or equal
\newcommand{\sub}{\subseteq}
% Proper subset
\newcommand{\prosub}{\subset}
% from
\newcommand{\from}{\leftarrow}

% Parantheses
\newcommand{\para}[1]{\left( #1 \right)}
% Curly Braces
\newcommand{\curl}[1]{\left\{ #1 \right\}}
% Brackets
\newcommand{\brac}[1]{\left[ #1 \right]}
% Angled Brackets
\newcommand{\ang}[1]{\left\langle #1 \right\rangle}
% Norm
\newcommand{\norm}[1]{\left\| #1 \right\|}

% Piece wise (use \\ between cases)
\newcommand{\piecewise}[1]{\begin{cases} #1 \end{cases}}

% Bold symbol shorthand
\newcommand{\bl}{\boldsymbol}

% Vertical space
\newcommand{\vs}[1]{\vspace{#1 pt}}
% Horizontal ertical space
\newcommand{\hs}[1]{\hspace{#1 pt}}


%===============================================================================
%============================== TEXT BASED SYMBOLS =============================
%===============================================================================


% Radians
\newcommand{\rad}{\text{rad}}
% Least Common Multiple
\newcommand{\lcm}{\text{lcm}}
% Automorphism
\newcommand{\Aut}{\text{Aut}}
% Variance
\newcommand{\var}{\text{Var}}
% Covariance
\newcommand{\cov}{\text{Cov}}
% Cofactor (matrix)
\newcommand{\cof}{\text{Cof}}
% Adjugate (matrix)
\newcommand{\adj}{\text{Adj}}
% Trace (matrix)
\newcommand{\tr}{\text{tr}}
% Standard deviation
\newcommand{\std}{\text{Std}}
% Correlation coefficient
\newcommand{\corr}{\text{Corr}}
% Sign
\newcommand{\sign}{\text{sign}}

% And text
\newcommand{\AND}{\text{ and }}
% Or text 
\newcommand{\OR}{\text{ or }}
% If text
\newcommand{\IF}{\text{ if }}
% When text
\newcommand{\WHEN}{\text{ when }}
% Then text
\newcommand{\THEN}{\text{ then }}

% 1st
\newcommand{\st}[1]{#1^{\text{st}}}
% 2nd
\newcommand{\nd}[1]{#1^{\text{nd}}}
% 3rd
\newcommand{\rd}[1]{#1^{\text{rd}}}
% nth
\newcommand{\nth}[1]{#1^{\text{th}}}


%===============================================================================
%========================= PROBABILITY AND STATISTICS ==========================
%===============================================================================


% Permutation
\newcommand{\perm}[2]{{}^{#1}\!P_{#2}}
% Combination
\newcommand{\comb}[2]{{}^{#1}C_{#2}}

% Baye's risk
\newcommand{\risk}[1]{\mathscr{R}_{#1}}
% Baye's optimal risk
\newcommand{\riskOptimal}[1]{\mathscr{R}_{#1}^*}
% Baye's empirical risk
\newcommand{\riskEmpirical}[2]{\hat{\mathscr{R}}_{#1}^{#2}}


%===============================================================================
%=================================== CALCULUS ==================================
%===============================================================================


% d over d derivative
\newcommand{\dd}[2]{\dfrac{d#1}{d#2}}
% partial d over d derivative
\newcommand{\partialdd}[2]{\dfrac{\partial #1}{\partial #2}}
% delta d over d derivative
\newcommand{\deltadd}[2]{\dfrac{\Delta #1}{\Delta #2}}

% Integration between a and b
\newcommand{\integral}[4]{\int_{#1}^{#2} #3 \, #4}
% Integration in some space 
\newcommand{\boundIntegral}[2]{\int_{#1} #2 \, d#1}

% Limit
\newcommand{\limit}[3]{\lim_{#1 \to #2} #3}


%===============================================================================
%================================  BIG SYMBOLS  ================================
%===============================================================================


% Sum
\newcommand{\sumof}[2]{\sum_{#1}^{#2}}
% Product
\newcommand{\productof}[2]{\prod_{#1}^{#2}}
% Union
\newcommand{\unionof}[2]{\bigcup_{#1}^{#2}}
% Intersection
\newcommand{\intersectionof}[2]{\bigcap_{#1}^{#2}}
% Or
\newcommand{\orof}[2]{\bigvee_{#1}^{#2}}
% And
\newcommand{\andof}[2]{\bigwedge_{#1}^{#2}}


%===============================================================================
%=============================== LINEAR ALGEBRA ================================
%===============================================================================


% Bold vector arrow
\newcommand{\bv}[1]{\vec{\mathbf{#1}}}

% Matrix or vector (use // for column, & for row) 
% with brackets
\newcommand{\bmat}[1]{\begin{bmatrix} #1 \end{bmatrix}}
% Matrix or vector (use // for column, & for row) 
% with curved brackets
\newcommand{\pmat}[1]{\begin{pmatrix} #1 \end{pmatrix}}
% Matrix or vector (use // for column, & for row) 
% with lines on either side 
\newcommand{\lmat}[1]{\begin{vmatrix} #1 \end{vmatrix}}
% Matrix or vector (use // for column, & for row) 
% with curly braces
\newcommand{\cmat}[1]{\begin{Bmatrix} #1 \end{Bmatrix}}
% Matrix or vector (use // for column, & for row) 
% with no braces
\newcommand{\mat}[1]{\begin{matrix} #1 \end{matrix}}


%===============================================================================
%================================ LARGE OBJECTS ================================
%===============================================================================

% Multiple lines
\newcommand{\multiline}[1]{
\begin{align*}
    #1
\end{align*}
}

% Multiple lines with equation numbers
\newcommand{\eqmultiline}[1]{
\begin{align*}
    #1
\end{align*}
}

% Color
\newcommand{\colorText}[2]{
\begingroup
\color{#1}
    #2
\endgroup
}

% Centered figure
\newcommand{\centerFigure}[2]{
    \begin{figure}[h]
        \centering
            #1
        \caption{#2}
    \end{figure}
}

% Tikz figure
\newcommand{\tikzGraphic}[1]{
    \begin{center}
    \begin{tikzpicture}
        #1
    \end{tikzpicture}
    \end{center}
}

% Enumerate numbers (seperate by \item)
\newcommand{\numbers}{\textbf{\number*)}}

% Enumerate letters (seperate by \item)
\newcommand{\letters}{\textbf{\alph*)}}

\title{%
    \Huge Machine Learning \\
    \Large Lecture VII
}
\date{2025-05-19}
\author{Michael Saba}

\begin{document}
\pagenumbering{gobble}
\maketitle
\newpage
\setlength{\parindent}{0pt}
\pagenumbering{arabic}

\section*{Linear Classiciation}

\subsection*{Introduction}

In the last lecture, we saw Linear Least Squares
Regression,
which was a machine learning algorithm concerned
with fitting the data with a function $f_\theta$
that was linear in terms of $\theta$
(but not neccesarily the inputs $x$,
due to the use of a feature map $\varphi(x)$). \\

In this section, we will explore
linear classification, where we attempt
to classify data by seperating the input space
into two half-spaces using a linear 
hyperplane. \\

We assume the output space is binary
$\ycal = \{-1, 1\}$. \\

Why do we seperate the data instead of fitting it?
We can't really fit data using linear functions;
they are too simple. We can instead
find a linear function $\ang{w, x} + b$
such that $\sign(\ang{w, x} + b)$
matches that of the output.
Notice that what this does is seperate
the data points on each side of the line
depending on the label. \\

The idea is simple; if we can find a hyperplane
$H$, which is linear, which seperates the input
data with label $-1$ and label $1$
such that the margin between the classification
hyperplane and the data is maximal
(the hyperplane is right in the middle),
we assume that it will generalize well
to unseen data, and expect that data will
correctly fall into their half-space. \\

We assume that we have $\xcal = \rbb^d$ and that
$\ycal = \{-1, 1\}$,
and that the hyperplane $H$
is an $d-1$-dimensional hyperplane. \\
If $\xcal = \rbb^2$, then we use a
line to divide $\rbb^2$ into two halfspaces. \\

First, note that a hyperplane can be written
as an implicit function:
\[ H = \{ x \in \xcal \mid \ang{w, x} + b = 0  \} \]
Where $w$ is a vector of slopes,
and $b$ is the offset. The vector $x$
is the vector of variables in $\rbb^d$.
For example:
\[ ax_1 + bx_2 + c = 0 \]
is a line in $\rbb^2$. \\

Note that the vector $w$ of slopes
is the gradient of the hyperplane,
which makes it the normal $\nabla (\ang{w, x} + b)$
of the hyperplane (it is constant
since the hyperplane has the same normal
at all points). \\

Note that this implict function is not unique;
if we multiply $w$ and $b$ by a scalar $\alpha$
that is not $0$, we get the exact same
hyperplane. \\

We can also write a hyperplane
as a parametric function:
\[ H = \{ \olsi{x} + z \mid \olsi{x} \in H
\AND z \in w^\perp \} \]
Where $w$ is a vector of slopes,
or normal,
which makes $w^\perp$
orthogonal to $w$, which is to say,
any vector on the plane.
Moreover, $\olsi{x}$ any point on the plane. \\
Basically, we take some point $\olsi{x}$
as our origin, and then offset it by any
vector $w^\perp$ that belongs to the
plane to get any point on the plane. \\

Note that we can find $w^\perp$
by just generating all vectors $z$ such that
$\ang{z, w} = 0$. \\

The two halfspaces defined by the hyperplane are
the positive halfspace:
\[ H_+ = \{ x \in \xcal \mid \ang{w, x} + b > 0 \}
= \{ x \in \xcal \mid \sign(\ang{w, x} + b) = 1 \} \]
and the negative halfspace:
\[ H_- = \{ x \in \xcal \mid \ang{w, x} + b < 0 \}
= \{ x \in \xcal \mid \sign(\ang{w, x} + b) = -1 \} \]
Where $H_+$ is the halfspace in the same direction
as the normal $w$ is pointing to. \\

Can we always find a linear classifier for a dataset?
The answer is no; it has to be linearly seperable. \\

A dataset $S = (x_i, y_i)^n \in (\xcal \times \ycal)^n$
is called linearly seperable if there exists
a hyperplane that can seperate the inputs $x_i$
of the datapoints
with $y_i = 1$ and $y_i = -1$. \\
Essentially, if there exists a hyperplane $H$
such that:
\[ \{x_i \mid y_i = -1\} \subset H_- \]
\[ \{x_i \mid y_i = 1\} \subset H_+ \]
Not all datasets have this property. \\

Note that if a dataset is linearly seperable,
there are usually more than one seperating hyperplane;
it is not unique. \\

Another way of saying we want a hyperplane
to seperate the datapoints that have labels
$-1$ and $1$ is to say that we want to find
$w$ and $b$ such that:
\[ y_i(\ang{w, x} + b) > 0 \]
for all $i \in \{1, 2, \dots, n\}$. \\

If we define 
\[ g_{(w, x)}(x) = \ang{w, x} + b \]
then our classifier is $f(x) = \sign(g_{(w, x)}(x))$. \\
Note that the hyperplane where $g_{(w, x)}(x) = 0$
is the decision boundary. \\

Now, there is a small note to make.
We will use a feature map $\varphi(x)$,
so the hyperplane won't be linear in terms of
the input space $\xcal$,
but it will be linear in terms of the
variables $\varphi(x)$. \\

Increasing the dimensionality of $\varphi(x)$,
like for example,
a cubic $\varphi(x) = (1, x, x^2, x^3)^T$
is able to capture subtle patterns in the data. \\

We again have the same class of functions $\ffrak$
with parameters $\theta \in \Theta$. \\

Note here that $\theta = (w, b)$,
where $b$ will show up because on top of 
$\varphi(x)$, our feature also has a $1$ 
in its features. \\

We use regularized empirical risk minimization
to find the weights $\theta$ of the classifier $g_\theta$. \\

In principle, we use 0-1-loss, but it is hard to
optimize, so we prefer to use surrogate loss functions. \\

We will start with no regularization $\rcal(g) \equiv 0$. \\

There are three different methods to optimize
the empirical risk and find the learning rule;
each uses a loss function:
\begin{enumerate}
    \item 
    Linear Discriminant Analysis:
    where we use the squared loss $(z - y)^2$.
    \item 
    Logistic Regression:
    where we use the logistic loss $\log(1+e^{-yz})$.
    \item 
    Support Vector Machines:
    where we use the hinge loss $\max(0, 1-yz)$,
    typically with quadratic regularization.
\end{enumerate}
They get better as we go further down. \\

The last two loss functions are margin-based,
meaning that they have $yz$ in them:
\[ \ell(z, y) = \Phi(zy) \]
Where we have a product $zy$. \\

\newpage

\subsection*{Linear Discriminant Analysis}

In the LDA method, we have the squared 
loss $\ell(z, y) = (z-y)^2$,
which makes it similar to linear least square regression,
except that our output is binary. \\

So we again have to minimize:
\[ \dfrac{1}{2}\|\Phi \theta - \bl{y}\|^2 \]
We already know how to solve this,
but for our own understanding,
we will reformulate this problem. \\

First note that we have:
\[ \theta = \pmat{w \\ b} \]
where we have $d$ $w$ weights, one for each feature 
$\varphi(x_j)$, and a $b$ offset,
which multiplies the $1$ in the features vectors.
This means that $\theta \in \rbb^{d+1}$. \\ 

This also means that $\Phi$ is an 
$\rbb^{n \times (d+1)}$
dimensional matrix, where we have $d+1$
because we have $d$ features in $\varphi(x)$
and we have $1$ for the offset $b$. \\

We have:
\[ \bl{y} = \pmat{y_1 \\ \vdots \\ y_n} \]
which is the vector of binary outputs in the training
data set, and belongs to $\{-1, 1\}^n$. \\

We can also define:
\[ \bl{x_i} = \varphi(x_i) \]
Which is the feature vector of $x_i$ in $\rbb^d$. \\
We can also define:
\[ \bl{X} = \pmat{\bl{x_1} & \bl{x_2} \dots & \bl{x_n}} \]
where the columns are teh feature vectors of each $x_i$. \\

This means that we can write:
\[ \Phi = \pmat{\bl{X}^T & \bl{1}} \]
where $\bl{1}$ is a vector of only $1$s, of length $n$:
\[ \bl{1} = \pmat{1 \\ \vdots \\ 1} \]
This means that we can write:
\[ \Phi^T = \pmat{\bl{X} \\ \bl{1}^T} \]
Both of which we will be using later. \\

We know that to minimize the risk,
the solution is given by the system:
\[ \Phi^T\Phi \theta = \Phi^T \bl{y} \]
Which we can rewrite using our new notation
as a block matrix:
\[ \pmat{\bl{X} \\ \bl{1}^T}\pmat{\bl{X}^T & \bl{1}}
\theta = \pmat{\bl{X} \\ \bl{1}^T}\bl{y} \]
\[ \pmat{\bl{X}\bl{X}^T & \bl{X}\bl{1} \\ 
\bl{1}^T\bl{X}^T & \bl{1}^T\bl{1}}
\pmat{w \\ b}= \pmat{\bl{X}\bl{y} \\ \bl{1}^T \bl{y}} \]
Where $\bl{1}^T\bl{1}$ is just $n$
(the dot product of two vectors of $1$s with $n$
entries). \\
Note that we can also write $\bl{1}^T\bl{X}^T$
as $(\bl{X}\bl{1})^T$:
\[ \pmat{\bl{X}\bl{X}^T & \bl{X}\bl{1} \\ 
(\bl{X}\bl{1})^T & n}
\pmat{w \\ b}= \pmat{\bl{X}\bl{y} \\ \bl{1}^T \bl{y}} \]
This is just a linear system with a matrix and a vector
of $d+1$ unknowns, which we will treat as a block matrix
with a vector of $2$ block element unknowns. \\

We will use substitution (instead of elimination)
to solve this matrix. So, we start
with the second equation:
\[ (\bl{X}\bl{1})^T w + nb = \bl{1}^T\bl{y} \]
\[ b = \dfrac{1}{n}(\bl{1}^T\bl{y} - (\bl{X}\bl{1})^T w) \]
We can now plug this into the first equation:
\[ \bl{X}\bl{X}^T w + \bl{X}\bl{1}b = \bl{X}\bl{y} \]
\[ \bl{X}\bl{X}^T w + \bl{X}\bl{1} 
\dfrac{1}{n}(\bl{1}^T\bl{y} - (\bl{X}\bl{1})^T w)
= \bl{X}\bl{y} \]
\[ \bl{X}\bl{X}^T w + \dfrac{1}{n}
(\bl{X}\bl{1} \bl{1}^T\bl{y} - (\bl{X}\bl{1})^T w)
= \bl{X}\bl{y} \]
\[ \bl{X}\bl{X}^T w + \dfrac{1}{n}\bl{X}\bl{1} \bl{1}^T\bl{y} - 
\dfrac{1}{n}(\bl{X}\bl{1})^T w
= \bl{X}\bl{y} \]
\[ (\bl{X}\bl{X}^T - \dfrac{1}{n}(\bl{X}\bl{1})^T) w
= (\bl{X} - \dfrac{1}{n}\bl{X}\bl{1} \bl{1}^T) \bl{y} \]
We will stop here. \\

Now, recall that in our training dataset, some inputs
$x_i$ corresponded to the $-1$ label, and others to the
$1$ label.
We will now reflect this division in the data. \\

We can rearrange the data in these matrices and
vectors as such. \\
In $\bl{X}$, we will have the feature vector columns
of the $x_i$ points with labal $-1$ on the left:
\[ \bl{X} = \pmat{\bl{X}_- & \bl{X}_+} \]
We can do the same for the outputs $\bl{y}$:
\[ \bl{y} = \pmat{\bl{y}_- \\ \bl{y}_+}
= \pmat{-(\bl{1}_-) \\ \bl{1}_+} \]
Note here that the output is binary,
so $\bl{y}$ just contains $1$s and $-1$s,
which we arrange such that the $-1$s are first.
Note that $\bl{1}_-$ is the vector of $1$s
corresponding to the $-1$ labels, but it itself
contains $1$s; the point of it is that its length
matches that of $\bl{y}_-$, the negative labels. \\
Likewise, we can divide our vector of $1$s:
\[ \bl{1} = \pmat{\bl{1}_- \\ \bl{1}_+} \]
Which is still just a vector of $1$s,
but we write it like this to differentiate the $1$s
that correspond to the negative class outputs
in the $\bl{y}$ vector, so that our vectors are
matrices are consistently written. \\

Now using block matric arithmetic, we have:
\[ \bl{X}\bl{1} = \pmat{\bl{X}_- & \bl{X}_+}
\pmat{\bl{1}_- \\ \bl{1}_+} 
= \bl{X}_-\bl{1}_- + \bl{X}_+\bl{1}_+ \]
Note that what we are doing
when we multiply $\bl{X}_-$ by $\bl{1}_-$
here is just summing the feature vectors of each $x_i$,
so we can write these sums as the avergage times
the number of elements (which is what a sum is):
\[ \bl{X}_-\bl{1}_- + \bl{X}_+\bl{1}_+ 
= n_-\bl{m}_- + n_+\bl{m}_+ \]
Where $n_-$ is the number of negative data-points,
and $\bl{m}_-$ is the mean of the feature vectors
of the inputs $x_i$ of the negative datapoints,
or just the mean of the columns of $\bl{X}_-$. \\
For that same reason, we can write:
\[ \bl{X}\bl{y} = \pmat{\bl{X}_- & \bl{X}_+}
\pmat{-(\bl{1}_-) \\ \bl{1}_+} 
= \bl{X}_+\bl{1}_+ - \bl{X}_-\bl{1}_-
= n_+\bl{m}_+ - n_-\bl{m}_- \]
Finally we have:
\[ \bl{X}\bl{X}^T = \bl{X}_-\bl{X}^T_- + \bl{X}_-\bl{X}^T_- \]
Again we're just doing block matrix arithmetic here. \\

Now, we can define two new values.
The first is the within-class covariance, 
which is the covariance within each of the two
classes, the one corresponding to the negative
and the one corresponding to the positive data points. \\
The second one is the between-class covariance,
which is the variance between the two classes. \\

Note that the covariance is calculated
using the feature vectors of each $x_i$,
that is, the columns of $\bl{X}_-$ and $\bl{X}_+$,
and their respective means $\bl{m}_-$ and $\bl{m}_+$,
using the same formula we know from before,
but without the expected value since these quantities 
are deterministic. \\

Now, intuitively, the within-class covariance
is just the sum of the covariance of the positive
and negative class of datapoints:
\[ \hat{\Sigma}_w 
= (\bl{X}_- - \bl{m}_-\bl{1}^T_-)(\bl{X}_- 
- \bl{m}_-\bl{1}^T_-)^T
+ (\bl{X}_+ - \bl{m}_+\bl{1}^T_+)(\bl{X}_+ 
- \bl{m}_+\bl{1}^T_+)^T \]
We can then do some arithmetic and get:
\[ \hat{\Sigma}_w = \bl{X}_+\bl{X}^T_+
+ \bl{X}_-\bl{X}^T_- -n_+\bl{m}_+\bl{m}^T_+
-n_-\bl{m}_-\bl{m}^T_- \]
Here we use $\hat{\Sigma}_w$ to denote the within-class
covariance. \\

The between-class covariance is:
\[ \hat{\Sigma}_b = (\bl{m}_+ - \bl{m}_-)(\bl{m}_+ - \bl{m}_-)^T \]
which is denoted using $\hat{\Sigma}_b$. \\

Now, going back to our solution using substitution, we had:
\[ (\bl{X}\bl{X}^T - \dfrac{1}{n}(\bl{X}\bl{1})^T) w
= (\bl{X} - \dfrac{1}{n}\bl{X}\bl{1} \bl{1}^T) \bl{y} \]
We can rewrite the right hand side
using the division of the classes
that we created:
\[ (\bl{X} - \dfrac{1}{n}\bl{X}\bl{1} \bl{1}^T)\bl{y}
= \dfrac{2n_+n_-}{n_+ + n_-}(\bl{m}_+ - \bl{m}_-) \]
Which we can derive arithmetically. \\
We can do the same with the left hand side:
\[ (\bl{X}\bl{X}^T - \dfrac{1}{n}(\bl{X}\bl{1})(\bl{X}\bl{1})^T)
= \hat{\Sigma}_w + \dfrac{n_+n_-}{n_+ + n_-}\hat{\Sigma}_b \]
Which contains the covariances;
so solving the linear system for the weights $w$
depends on these covariances. \\

Now, we can derive the following from our equation
for the weight $w$:
\[ \para{\hat{\Sigma}_w + 
\dfrac{n_+n_-}{n_+ + n_-}\hat{\Sigma}_b}w
= \dfrac{2n_+n_-}{n_+ + n_-}(\bl{m}_+ - \bl{m}_-) \]
\[ \dfrac{n_+ + n_-}{2n_+n_-}\pmat{\hat{\Sigma}_w + 
\dfrac{n_+n_-}{n_+ + n_-}\hat{\Sigma}_b}w
= (\bl{m}_+ - \bl{m}_-) \]
\[ \para{\dfrac{n_+ + n_-}{2n_+n_-}\hat{\Sigma}_w + 
\dfrac{1}{2}\hat{\Sigma}_b}w
= (\bl{m}_+ - \bl{m}_-) \]
Now we can multiply both sides by the inverse
of the within class covariance:
\[ \hat{\Sigma}\inv_w
\para{\dfrac{n_+ + n_-}{2n_+n_-}\hat{\Sigma}_w + 
\dfrac{1}{2}\hat{\Sigma}_b}w
= \hat{\Sigma}\inv_w(\bl{m}_+ - \bl{m}_-) \]
\[\para{\dfrac{n_+ + n_-}{2n_+n_-}\hat{\Sigma}\inv_w\hat{\Sigma}_w + 
\dfrac{1}{2}\hat{\Sigma}\inv_w\hat{\Sigma}_b}w
= \hat{\Sigma}\inv_w(\bl{m}_+ - \bl{m}_-) \]
\[\para{\dfrac{n_+ + n_-}{2n_+n_-}I + 
\dfrac{1}{2}\hat{\Sigma}\inv_w\hat{\Sigma}_b}w
= \hat{\Sigma}\inv_w(\bl{m}_+ - \bl{m}_-) \]
This will lead to:
\[ w = \alpha \hat{\Sigma}\inv_w (\bl{m}_+ - \bl{m}_-) \]
Where:
\[ \alpha =  \]

Now note that the weight $w$
only depends on where the centroids (means)
of the two clusters/classes are, and the within-class
covariance. \\
This illuminates how LDA finds a seperator. \\

Alternatively, we can show that the optimal weight we found:
\[ w = \alpha \hat{\Sigma}\inv_w (\bl{m}_+ - \bl{m}_-) \]
is proportional to the Fischer criterion:
\[ \arg \max_w = \dfrac{\ang{w, \hat{\Sigma}_b w}}
{\ang{w, \hat{\Sigma}_w w}} \]
Where
\[ \ang{w, \hat{\Sigma}_b w} = (\ang{\bl{m}_+ - \bl{m}_-, w})^2 \]
and
\[ \ang{w, \hat{\Sigma}_w w} = 
\|(\bl{X}_- - \bl{m}_-\bl{1}^T_-)^Tw\|^2
+ \|(\bl{X}_+ - \bl{m}_+\bl{1}^T_+)^Tw\|^2 \]
Still unsure how we can show it is proportional. \\

We can interpret the result better now;
notice how $w$ is proportional to the Fischer
criterion, which means that it increases as the numerator 
icnreases, and decreases as the denominator increases. \\
This means that $w$ grows proportionally with
$\ang{w, \hat{\Sigma}_b w}$,
which means that $w$ wil align with direction of
the centroids (direction connecting the centroids). \\
On the other hand, $w$ is inverse proportional to 
$\ang{w, \hat{\Sigma}_w w}$,
which means that it will not align well with the 
within-class variation. \\

The intuition behind this is simple;
we want our projected centroids to be as far
away from each other as possible;
this leads to the two clusters optimally
being seperated. \\

One disadvantage of using LDA however is that it is
vert strongly affected by the within class
covariance, which it won't align with;
so if we have a seperator thats seperates two clusters
well, and one cluster had some mroe data points added to it,
this would change the seperator's direction, even if
the added data points were completely on the correct
side; it's because the covariance changed. \\

So LDA is just about maximizing the Fischer criterion. \\

It also does dimensionality reduction. \\

\newpage

\subsection*{Logistic Regression}

In logistic regression, we use logistic loss,
which is given by:
\[ \ell(z, y) = \log(1 + e^{-zy}) \]
It is classification calibrated, convex,
and is margin-based. \\
Note that solving for the optimal $\theta$
has no closed form solution. \\

Instead of maximizing $\log$,
we can minimize $-\log$,
which does the same thing. \\

Using logistic loss is equivalent to assuming
that the probability density follows sigmoid function
in space:
\[ p_{Y \mid X = x, \theta}(y) 
= \dfrac{1}{1 + e^{-y\ang{\varphi(x), \theta}}} \]
The sigmoid shape tells us that optimizing the logistic
loss will get us a seperator that prioritizes having 
data points with the correct label the furthest away
from the seperator. 
This makese sense; the sigmoid function
takes as input $y\and{\varphi(x), \theta}$.
If the input is negative, the sigmoid has
value $0$, and if the input is positive, 
it has value $1$, with a transitional phase
at $y\and{\varphi(x), \theta} = 0$
which can be as sharp or smooth as we want. \\
So if $y\and{\varphi(x), \theta}$ share the same sign,
the input is positive, and the density is nearly $1$,
and if they have opposite signs, the density is nearly $0$. \\
This means that our assumption is that we want a $\theta$
with a probability of assigning a correct
label being very high (density $1$),
and the probability of assigning a wrong label
being very low (density $0$). \\

This is why logistic regression is better than LDA;
the boundary is more centered between the
two clusters; this is because LDA cares too much
about the within-class covariances of the two
classes, and is easily skewed, and not often centered. \\

We can think of logistic regression as ensuring
that if a point is further away than another
point in terms of the seperator,
then it will have lower loss and will surely be
classified correctly. \\
So irrelevant data far away from the seperator
influences logistic regression's seperator
far less. \\

We often use quadratic regularization in logistic
regression:
\[ \min_\theta \sum_{i=1}^n 
\log(1 + e^{-y\ang{\varphi(x), \theta}}) 
+ \lambda \|\theta\|^2 \]
This is a formula showing the empirical risk
and regularization minimization. \\
Note that $|\theta\|^2 = \|w^2\| + b^2$. \\

Logistic regression generalizes very well
to multi-class classification. \\
For a $K$ multi-class classification, we assume the density
follows:
\[ p_{Y \mid X = x, \theta}(y)
= \dfrac{e^{\ang{\theta_y, \varphi(x)}}}
{\sum_{i = 1}^K e^{\ang{\theta_i, \varphi(x)}}} \]
This is accompanied by the multi-class loss function:
\[ \ell(z, y) = -\log\para{\dfrac{e^{\ang{\theta_y, z}}}
{\sum_{i = 1}^K e^{\ang{\theta_i, z}}}} \]
As described earlier; the output of the classifier
is a vector containing the confidence in each ouput.
The softmax evaluates the predictions; if the
correct label has high confidence and the rest have low
confidence, it gets lower. \\
It is called the softmax. \\

\newpage

\end{document}
