
\documentclass[12pt]{article}

\usepackage[margin=1in]{geometry}


%===============================================================================
%================================== PACKAGES ===================================
%===============================================================================

% For using float option H that places figures 
% exatcly where we want them
\usepackage{float}
% makes figure font bold
\usepackage{caption}
\captionsetup[figure]{labelfont=bf}
% For text generation
\usepackage{lipsum}
% For drawing
\usepackage{tikz}
% For smaller or equal sign and not divide sign
\usepackage{amssymb}
% For the diagonal fraction
\usepackage{xfrac}
% For enumerating exercise parts with letters instead of numbers
\usepackage{enumitem}
% For dfrac, which forces the fraction to be in display mode (large) e
% even in math mode (small)
\usepackage{amsmath}
% For degree sign
\usepackage{gensymb}
% For "\mathbb" macro
\usepackage{amsfonts}
% For positioning 
\usepackage{indentfirst}
\usetikzlibrary{shapes,positioning,fit,calc}
% for adjustwidth environment
\usepackage{changepage}
% for arrow on top
\usepackage{esvect}
% for mathbb 1
\usepackage{bbm}
% for mathsrc
\usepackage[mathscr]{eucal}
% For degree sign
\usepackage{gensymb}
% For quotes
\usepackage{csquotes}
% For vertical lines
\usepackage{mathtools}
% For cols
\usepackage{multicol}

% for tikz
\usepackage{pgfplots}
\pgfplotsset{compat=1.18}
\usepackage{amsmath}
\usepgfplotslibrary{groupplots}


%===============================================================================
%==================================== FONTS ====================================
%===============================================================================


% Mathcal
\newcommand{\acal}{\mathcal{A}}
\newcommand{\bcal}{\mathcal{B}}
\newcommand{\ccal}{\mathcal{C}}
\newcommand{\dcal}{\mathcal{D}}
\newcommand{\ecal}{\mathcal{E}}
\newcommand{\fcal}{\mathcal{F}}
\newcommand{\gcal}{\mathcal{G}}
\newcommand{\hcal}{\mathcal{H}}
\newcommand{\ical}{\mathcal{I}}
\newcommand{\jcal}{\mathcal{J}}
\newcommand{\kcal}{\mathcal{K}}
\newcommand{\lcal}{\mathcal{L}}
\newcommand{\mcal}{\mathcal{M}}
\newcommand{\ncal}{\mathcal{N}}
\newcommand{\ocal}{\mathcal{O}}
\newcommand{\pcal}{\mathcal{P}}
\newcommand{\qcal}{\mathcal{Q}}
\newcommand{\rcal}{\mathcal{R}}
\newcommand{\scal}{\mathcal{S}}
\newcommand{\tcal}{\mathcal{T}}
\newcommand{\ucal}{\mathcal{U}}
\newcommand{\vcal}{\mathcal{V}}
\newcommand{\wcal}{\mathcal{W}}
\newcommand{\xcal}{\mathcal{X}}
\newcommand{\ycal}{\mathcal{Y}}
\newcommand{\zcal}{\mathcal{Z}}

% Mathfrak
\newcommand{\afrak}{\mathfrak{A}}
\newcommand{\bfrak}{\mathfrak{B}}
\newcommand{\cfrak}{\mathfrak{C}}
\newcommand{\dfrak}{\mathfrak{D}}
\newcommand{\efrak}{\mathfrak{E}}
\newcommand{\ffrak}{\mathfrak{F}}
\newcommand{\gfrak}{\mathfrak{G}}
\newcommand{\hfrak}{\mathfrak{H}}
\newcommand{\ifrak}{\mathfrak{I}}
\newcommand{\jfrak}{\mathfrak{J}}
\newcommand{\kfrak}{\mathfrak{K}}
\newcommand{\lfrak}{\mathfrak{L}}
\newcommand{\mfrak}{\mathfrak{M}}
\newcommand{\nfrak}{\mathfrak{N}}
\newcommand{\ofrak}{\mathfrak{O}}
\newcommand{\pfrak}{\mathfrak{P}}
\newcommand{\qfrak}{\mathfrak{Q}}
\newcommand{\rfrak}{\mathfrak{R}}
\newcommand{\sfrak}{\mathfrak{S}}
\newcommand{\tfrak}{\mathfrak{T}}
\newcommand{\ufrak}{\mathfrak{U}}
\newcommand{\vfrak}{\mathfrak{V}}
\newcommand{\wfrak}{\mathfrak{W}}
\newcommand{\xfrak}{\mathfrak{X}}
\newcommand{\yfrak}{\mathfrak{Y}}
\newcommand{\zfrak}{\mathfrak{Z}}

% Mathscr
\newcommand{\ascr}{\mathscr{A}}
\newcommand{\bscr}{\mathscr{B}}
\newcommand{\cscr}{\mathscr{C}}
\newcommand{\dscr}{\mathscr{D}}
\newcommand{\escr}{\mathscr{E}}
\newcommand{\fscr}{\mathscr{F}}
\newcommand{\gscr}{\mathscr{G}}
\newcommand{\hscr}{\mathscr{H}}
\newcommand{\iscr}{\mathscr{I}}
\newcommand{\jscr}{\mathscr{J}}
\newcommand{\kscr}{\mathscr{K}}
\newcommand{\lscr}{\mathscr{L}}
\newcommand{\mscr}{\mathscr{M}}
\newcommand{\nscr}{\mathscr{N}}
\newcommand{\oscr}{\mathscr{O}}
\newcommand{\pscr}{\mathscr{P}}
\newcommand{\qscr}{\mathscr{Q}}
\newcommand{\rscr}{\mathscr{R}}
\newcommand{\sscr}{\mathscr{S}}
\newcommand{\tscr}{\mathscr{T}}
\newcommand{\uscr}{\mathscr{U}}
\newcommand{\vscr}{\mathscr{V}}
\newcommand{\wscr}{\mathscr{W}}
\newcommand{\xscr}{\mathscr{X}}
\newcommand{\yscr}{\mathscr{Y}}
\newcommand{\zscr}{\mathscr{Z}}

% Mathbb
\newcommand{\abb}{\mathbb{A}}
\newcommand{\bbb}{\mathbb{B}}
\newcommand{\cbb}{\mathbb{C}}
\newcommand{\dbb}{\mathbb{D}}
\newcommand{\ebb}{\mathbb{E}}
\newcommand{\fbb}{\mathbb{F}}
\newcommand{\gbb}{\mathbb{G}}
\newcommand{\hbb}{\mathbb{H}}
\newcommand{\ibb}{\mathbb{I}}
\newcommand{\jbb}{\mathbb{J}}
\newcommand{\kbb}{\mathbb{K}}
\newcommand{\lbb}{\mathbb{L}}
\newcommand{\mbb}{\mathbb{M}}
\newcommand{\nbb}{\mathbb{N}}
\newcommand{\obb}{\mathbb{O}}
\newcommand{\pbb}{\mathbb{P}}
\newcommand{\qbb}{\mathbb{Q}}
\newcommand{\rbb}{\mathbb{R}}
\newcommand{\sbb}{\mathbb{S}}
\newcommand{\tbb}{\mathbb{T}}
\newcommand{\ubb}{\mathbb{U}}
\newcommand{\vbb}{\mathbb{V}}
\newcommand{\wbb}{\mathbb{W}}
\newcommand{\xbb}{\mathbb{X}}
\newcommand{\ybb}{\mathbb{Y}}
\newcommand{\zbb}{\mathbb{Z}}


%===============================================================================
%=============================== SPECIAL SYMBOLS ===============================
%===============================================================================


% Orbit (group theory)
\newcommand{\orbit}{\mathcal{O}}
% Normal group
\newcommand{\normal}{\mathcal{N}}
% Indicator function
\newcommand{\indicator}{\mathbbm{1}}
% Laplace transform
\newcommand{\laplace}[1]{\mathcal{L}}
% Epsilon shorthand
\newcommand{\eps}{\varepsilon}
% Omega
\newcommand{\om}{\omega}
\newcommand{\Om}{\Omega}

%===============================================================================
%================================== OPERATORS ==================================
%===============================================================================


% Inverse exponent
\newcommand{\inv}[0]{^{-1}}
% Overline bar
\newcommand{\olsi}[1]{\,\overline{\!{#1}}}
% Less than or equal slanted
\newcommand{\seqs}{\leqslant}
% Greater or equal slanted
\newcommand{\geqs}{\geqslant}
% Subset or equal
\newcommand{\sub}{\subseteq}
% Proper subset
\newcommand{\prosub}{\subset}
% from
\newcommand{\from}{\leftarrow}

% Parantheses
\newcommand{\para}[1]{\left( #1 \right)}
% Curly Braces
\newcommand{\curl}[1]{\left\{ #1 \right\}}
% Brackets
\newcommand{\brac}[1]{\left[ #1 \right]}
% Angled Brackets
\newcommand{\ang}[1]{\left\langle #1 \right\rangle}
% Norm
\newcommand{\norm}[1]{\left\| #1 \right\|}

% Piece wise (use \\ between cases)
\newcommand{\piecewise}[1]{\begin{cases} #1 \end{cases}}

% Bold symbol shorthand
\newcommand{\bl}{\boldsymbol}

% Vertical space
\newcommand{\vs}[1]{\vspace{#1 pt}}
% Horizontal ertical space
\newcommand{\hs}[1]{\hspace{#1 pt}}


%===============================================================================
%============================== TEXT BASED SYMBOLS =============================
%===============================================================================


% Radians
\newcommand{\rad}{\text{rad}}
% Least Common Multiple
\newcommand{\lcm}{\text{lcm}}
% Automorphism
\newcommand{\Aut}{\text{Aut}}
% Variance
\newcommand{\var}{\text{Var}}
% Covariance
\newcommand{\cov}{\text{Cov}}
% Cofactor (matrix)
\newcommand{\cof}{\text{Cof}}
% Adjugate (matrix)
\newcommand{\adj}{\text{Adj}}
% Trace (matrix)
\newcommand{\tr}{\text{tr}}
% Standard deviation
\newcommand{\std}{\text{Std}}
% Correlation coefficient
\newcommand{\corr}{\text{Corr}}
% Sign
\newcommand{\sign}{\text{sign}}

% And text
\newcommand{\AND}{\text{ and }}
% Or text 
\newcommand{\OR}{\text{ or }}
% For text 
\newcommand{\FOR}{\text{ for }}
% If text
\newcommand{\IF}{\text{ if }}
% When text
\newcommand{\WHEN}{\text{ when }}
% Where text
\newcommand{\WHERE}{\text{ where }}
% Then text
\newcommand{\THEN}{\text{ then }}
% Such that text
\newcommand{\SUCHTHAT}{\text{ such that }}

% 1st
\newcommand{\st}[1]{#1^{\text{st}}}
% 2nd
\newcommand{\nd}[1]{#1^{\text{nd}}}
% 3rd
\newcommand{\rd}[1]{#1^{\text{rd}}}
% nth
\newcommand{\nth}[1]{#1^{\text{th}}}


%===============================================================================
%========================= PROBABILITY AND STATISTICS ==========================
%===============================================================================


% Permutation
\newcommand{\perm}[2]{{}^{#1}\!P_{#2}}
% Combination
\newcommand{\comb}[2]{{}^{#1}C_{#2}}

% Baye's risk
\newcommand{\risk}[1]{\mathscr{R}_{#1}}
% Baye's optimal risk
\newcommand{\riskOptimal}[1]{\mathscr{R}_{#1}^*}
% Baye's empirical risk
\newcommand{\riskEmpirical}[2]{\hat{\mathscr{R}}_{#1}^{#2}}


%===============================================================================
%=================================== CALCULUS ==================================
%===============================================================================


% d over d derivative
\newcommand{\dd}[2]{\dfrac{d#1}{d#2}}
% partial d over d derivative
\newcommand{\partialdd}[2]{\dfrac{\partial #1}{\partial #2}}
% delta d over d derivative
\newcommand{\deltadd}[2]{\dfrac{\Delta #1}{\Delta #2}}

% Integration between a and b
\newcommand{\integral}[4]{\int_{#1}^{#2} #3 \, #4}
% Integration in some space 
\newcommand{\boundIntegral}[2]{\int_{#1} #2 \, d#1}

% Limit
\newcommand{\limit}[3]{\lim_{#1 \to #2} #3}


%===============================================================================
%================================  BIG SYMBOLS  ================================
%===============================================================================


% Sum
\newcommand{\sumof}[2]{\sum_{#1}^{#2}}
% Product
\newcommand{\productof}[2]{\prod_{#1}^{#2}}
% Union
\newcommand{\unionof}[2]{\bigcup_{#1}^{#2}}
% Intersection
\newcommand{\intersectionof}[2]{\bigcap_{#1}^{#2}}
% Or
\newcommand{\orof}[2]{\bigvee_{#1}^{#2}}
% And
\newcommand{\andof}[2]{\bigwedge_{#1}^{#2}}


%===============================================================================
%=============================== LINEAR ALGEBRA ================================
%===============================================================================


% Bold vector arrow
\newcommand{\bv}[1]{\vec{\mathbf{#1}}}

% Matrix or vector (use // for column, & for row) 
% with brackets
\newcommand{\bmat}[1]{\begin{bmatrix} #1 \end{bmatrix}}
% Matrix or vector (use // for column, & for row) 
% with curved brackets
\newcommand{\pmat}[1]{\begin{pmatrix} #1 \end{pmatrix}}
% Matrix or vector (use // for column, & for row) 
% with lines on either side 
\newcommand{\lmat}[1]{\begin{vmatrix} #1 \end{vmatrix}}
% Matrix or vector (use // for column, & for row) 
% with curly braces
\newcommand{\cmat}[1]{\begin{Bmatrix} #1 \end{Bmatrix}}
% Matrix or vector (use // for column, & for row) 
% with no braces
\newcommand{\mat}[1]{\begin{matrix} #1 \end{matrix}}


%===============================================================================
%================================ LARGE OBJECTS ================================
%===============================================================================

% Multiple lines
\newcommand{\multiline}[1]{
\begin{align*}
    #1
\end{align*}
}

% Multiple lines with equation numbers
\newcommand{\eqmultiline}[1]{
\begin{align*}
    #1
\end{align*}
}

% Color
\newcommand{\colorText}[2]{
\begingroup
\color{#1}
    #2
\endgroup
}

% Centered figure
\newcommand{\centerFigure}[2]{
    \begin{figure}[h]
        \centering
            #1
        \caption{#2}
    \end{figure}
}

% Tikz figure
\newcommand{\tikzGraphic}[1]{
    \begin{center}
    \begin{tikzpicture}
        #1
    \end{tikzpicture}
    \end{center}
}

% Enumerate numbers (seperate by \item)
\newcommand{\numbers}{\textbf{\number*)}}

% Enumerate letters (seperate by \item)
\newcommand{\letters}{\textbf{\alph*)}}

\title{%
    \Huge Machine Learning \\
    \Large Assignment I
}
\date{2025-04-14}
\author{Michael Saba}

\begin{document}
\pagenumbering{gobble}
\maketitle
\newpage
\setlength{\parindent}{0pt}
\pagenumbering{arabic}

    \subsection*{Ex 1}

    \begin{enumerate}[label=\alph*)]
        \item 
        We know that if $v$ is an eigenvector of $A$,
        and $\lambda$ is its eigenvalue,
        then
        \[ Av = \lambda v \]
        Now, to find the eigenvectors of $A^k$
        with eigenvalues $\gamma$,
        we can solve
        \[ A^k v = \gamma v \]
        Note that by using the associativity
        of matrix multiplication to our advantage,
        we can write
        \[ A^kv = (A^{k-1}A)v = A^{k-1}(Av) 
        = A^{k-1}\lambda v \]
        Since scalar multiplication
        is commutative, we can then
        move the scalar $\lambda$:
        \[  A^kv = A^{k-1}\lambda v
        = \lambda(A^{k-1} v) \]
        We can repeat this process:
        \[ A^kv = \lambda(A^{k-1} v)
        = \lambda(\lambda(A^{k-2} v))
        = \dots \lambda^{n-1}(Av)
        = \lambda^n v  \]
        So now that we have
        \[ A^kv = \lambda^n v \]
        we conclude that for each eigenvector $v$
        and eigenvalue $\lambda$
        of $A$,
        the eigenvectors of $A^k$
        are the eigenvectors $v$ of $A$,
        and the eigenvalues of $A^k$
        are the eigenvalues of $A$ 
        raised to the $\nth{n}$ power $\lambda^n$.
        \item
        Given a vector $x$,
        we know that its 2-norm squared is
        $x^Tx = \|x\|^2$. \\
        An Orthonormal basis $U$ is a new basis
        in which we can express vectors such that
        their lengths (2-norm) remain unchanged. \\
        In other words,
        if $y$ expressed where the vector $x$ 
        would point to had $x$ been in the
        orthonormal basis $U$, then 
        \[ x = Uy \]
        And since the basis is orthonormal,
        the 2-norm of both vectors is the same:
        \[ x^Tx = y^Ty \]
        but we also know that
        \[ x^Tx = (Uy)^TUy = y^TU^TUy \]
        Since $y^TU^TUy = y^Ty$,
        we can conclude that $UU^T = I$. \\
        Now, we need to show that
        \[ \dfrac{x^TAx}{x^Tx} 
        \leq \lambda_{\text{max}} \]
        We can rewrite the numerator using the
        decomposition:
        \[ \dfrac{x^TU\Sigma U^Tx}{x^Tx} \]
        Since $UU^T = I$,
        we can rewrite the denominator as:
         \[ \dfrac{x^TU\Sigma U^Tx}{x^TIx} 
        = \dfrac{x^TU\Sigma U^Tx}{x^TUU^Tx}
        = \dfrac{y^T\Sigma y}{y^Ty} \]
        where $y = U^Tx$. \\
        Now, since $\Sigma$ is a diagonal matrix
        containing the eigenvalues $\lambda_i$,
        this means that
        \[ \Sigma y = \sum_{i = 1}^{n}\lambda_i y_i \]
        Which means that:
        \[ \dfrac{y^T\Sigma y}{y^Ty} 
        = \dfrac{y^T \sum_{i = 1}^{n}\lambda_i y_i}{y^Ty} 
        =  \dfrac{\sum_{i = 1}^{n}\lambda_i {y_i}^2}
        {\sum_{i=1}^{n}{y_i}^2} \]
        Now, there must be at least one value
        $\lambda_\text{max} \geq \lambda_i$
        for all $i \in \{1, 2, \dots n\}$. \\
        This means that:
        \[ \dfrac{\sum_{i = 1}^{n}\lambda_\text{max} {y_i}^2}
        {\sum_{i=1}^{n}{y_i}^2} \geq
        \dfrac{\sum_{i = 1}^{n}\lambda_i {y_i}^2}
        {\sum_{i=1}^{n}{y_i}^2} \]
        where
        \[ \dfrac{\sum_{i = 1}^{n}\lambda_\text{max} {y_i}^2}
        {\sum_{i=1}^{n}{y_i}^2} = 
        \dfrac{\lambda_\text{max}\sum_{i = 1}^{n}{y_i}^2}
        {\sum_{i=1}^{n}{y_i}^2} = 
        \lambda_\text{max} \]
        So $\lambda_\text{max} \geq 
        \dfrac{x^TAx}{x^Tx}$. \\
    \end{enumerate}

    \newpage

    \subsection*{Ex 2}

    \begin{enumerate}[label=\alph*)]
        \item 
        Given a $d \times n$ matrix
        \[ X = \begin{bmatrix}
            x_1 & x_2 & \dots & x_n
        \end{bmatrix} \]
        where each $x_i$ is a column vector,
        and a function
        \[ f:\R^d \to \R \]
        \[ f(c) = \sum_{i=1}^{n}\|x_i - c\|^2 \]
        that descibres the mean squared
        error (average difference between $c$
        and each column in $X$),
        we want to find the value $c^*$
        such that $f(c^*)$ is minimal. \\
        We can use calculus.
        We know that that gradient $\nabla f$
        describes the direction of most change
        in $f$,
        and that
        \[\nabla f = 0\]
        is an extremum (minimum or maximum).
        In this case it is a minimum since
        $f$ can grow arbitrarily large. \\
        We know that
        \[ \nabla f = \para{\partialdd{f}{c_1},
        \partialdd{f}{c_2}, \dots, \partialdd{f}{c_d}} \]
        To calculate the gradient,
        we can rewrite the function by expanding
        the 2-norm squared:
        \[ f(c) = \sum_{i=1}^{n}\|x_i - c\|^2
        = \sum_{i=1}^{n}\sum_{j=1}^{d}(x_{ij} - c_j)^2 \]
        Then we can differentiate.
        For each $\partialdd{f}{c_j}$,
        only the terms containing $c_j$
        have to be considered:
        \[ \partialdd{f}{c_j} = 
        \partialdd{}{c_j}\sum_{i=1}^{n}(x_{ij} - c_j)^2
        = 2\sum_{i=1}^{n}(x_{ij} - c_j)
        =  2\para{\sum_{i=1}^{n}x_{ij}} - 2n(c_j) \]
        Which means that
        \[ (\nabla f)_j
        = 2\para{\sum_{i=1}^{n}x_{ij}} - 2n(c_j) \]
        \[ \nabla f
        = 2\para{\sum_{i=1}^{n}x_{i}} - 2n(c) \]
        So we can now solve
        \[ \nabla f = 0 \]
        \[ 2\para{\sum_{i=1}^{n}x_{i}} - 2n(c) = 0 \]
        \[ 2\sum_{i=1}^{n}x_{i} = 2n(c) \]
        \[ \dfrac{1}{n}\sum_{i=1}^{n}x_{i} = c \]
        So the minimum of $f$ is reached at
        \[ c^* = \dfrac{1}{n}\sum_{i=1}^{n}x_{i} \]
        which is the mean of the column vector $x_i$
        (average). \\
        Intuitively this makes sense.
        Minimizing the mean squared error $f(c)$
        reduces to making $c$ as close as possible
        in value to each of $x_i$,
        which can be achieved by making it their average
        value. 
        \item
        The empirical covariance matrix for $X$
        is an $n \times n$ square matrix that
        encodes the empirical covariance of every pair
        of variables $x_i$ in $X$. \\
        As we know, the covariance of $X_i$ and $X_j$
        is:
        \[ \text{Cov}(x_i, x_j) =  
        \dfrac{1}{n}\E(x_i - \E(x_i))\E(x_j - \E(x_j)) \]
        where $\E(x_i)$ and $\E(x_j)$
        are the means of each distribution. \\
        The empirical covariance $\Sigma_X$
        is the same,
        but applies to some sample of points
        instead of the whole distribution:
        \[ \sum_{i = 1}^{n}(x_i - \mu)(x_i-\mu)^T \]
        Here $(x_i - \mu)(x_i-\mu)^T$
        is the outer product,
        which produces an $n \times n$ square matrix,
        and
        \[\mu = \dfrac{1}{n}\sum_{i=1}^{n}x_i \]
        is the vector encoding the empirical
        means of each vector $x_i$. \\
        We need to prove that $\Sigma_X$
        is positive semi-definite,
        meaning that for any vector $w \in \R^d$:
        \[ w^T\Sigma w \geq 0 \]
        Notice that:
        \[ w^T\Sigma w = w^T \para{
            \dfrac{1}{n}\sum_{i = 1}^{n}(x_i - \mu)
            (x_i-\mu)^T } w
        =  \dfrac{1}{n}\sum_{i = 1}^{n} \para{
        w^T (x_i - \mu)(x_i-\mu)^T w } \]
        Here, the two terms
        \[ w^T (x_i - \mu) \qquad (x_i-\mu)^T w \]
        Are the matrix multiplication
        of a transposed vector and a vector,
        which corresponds to the inner product
        (or more specifically, the dot product):
        \[ w^T (x_i - \mu) = w \cdot (x_i - \mu)
        \qquad (x_i-\mu)^T w = (x_i - \mu) \cdot w \]
        Since the dot product is commutative,
        the two terms are equal,
        which means that 
        \[ \dfrac{1}{n}\sum_{i = 1}^{n} \para{
            w^T (x_i - \mu)(x_i-\mu)^T w }
        =  \dfrac{1}{n}\sum_{i = 1}^{n}
            (w\cdot(x_i - \mu))^2 \]
        which is the sum of square scalar terms,
        meaning that it has to be non-negative. \\
        Therefore, 
        $\Sigma_X$ is positive semi-definite. \\

    \end{enumerate}

    \begingroup
    \color{red}
    \subsection*{Correction}
    \begin{enumerate}[label=\alph*)]
        \item
        The answer is correct,
        but an easier way we can do this is using
        vector and matrix differentials:
        \[ f(c) = \sum_{i=1}^{n}\|x_i - c\|^2 \]
        \[ f(c) = \sum_{i=1}^{n}(x_i - c)^T(x_i - c) \]
        \[ f(c) = \sum_{i=1}^{n}(x_i^Tx_i - 2cx_i + c^2) \]
        \[ \nabla f(c) 
        = \sum_{i=1}^{n}(- 2x_i + 2c) \]
        Which means that:
        \[ \nabla f(c) = 0 \]
        \[ \sum_{i=1}^{n}(2x_i) = \sum_{i=1}^{n}(2c) \]
        \[ \sum_{i=1}^{n}x_i = nc \]
        \[ \dfrac{1}{n}\sum_{i=1}^{n}x_i = c \]
        Here we can use the rules:
        \[ \nabla x = 1 \]
        \[ \nabla \| x \|^2 = 2x \]
        \[ \nabla \| Ax \|^2 = 2A^TAx \]
        \[ \nabla \| Ax -b \|^2 = 2A^T(Ax-b) \]
        \[ \nabla x^TAx = (A + A^T)x \]
        \[ (u + v)^T(u + v) 
        = u^Tu + u^Tv + v^Tu + v^Tv 
        = u^2 + 2u^Tv + v^2 \]
        where $x, u, v \in \R^n$. \\
        Another small remark is that $u^Tv = v^Tu$,
        both being the dot product,
        but neither is equal to $uv^T$,
        or $vu^T$,
        the outer product, which is a matrix.
    \end{enumerate}
    \endgroup

    \newpage

    \subsection*{Ex 3}

    \begin{enumerate}[label=\alph*)]
        \item 
        Now, we have the random variable
        \[ \sum_{i = 1}^{n} X_i \]
        where 
        \[ X_i \sim \normal(0, \Sigma_i) \]
        The sum will follow a Guassian Distribution
        with density function:
        \[ f(x) = (2\pi)^{-\sfrac{d}{2}}
        \det(\Sigma)^{\sfrac{1}{2}}
        \exp\para{-\dfrac{1}{2}(x-\mu)^T
        \Sigma\inv (x - \mu)} \]
        where $\Sigma$ is the variance
        and $\mu$ is the mean. \\
        As we know, the mean $\mu$
        is the expected value of the
        random variable:
        \[ \E\brac{\sum_{i = 1}^{n} X_i} \]
        By the linearity of the expected value:
        \[ \E\brac{\sum_{i = 1}^{n} X_i} =
        \sum_{i = 1}^{n} \E(X_i) = 0 \]
        since $\E(X_i) = 0$ for all $X_i$
        (since $X_i \sim \normal(0, \Sigma_i)$). \\
        As for the variance of the random
        variable (or covariance matrix),
        we know that since all $X_i$
        are independent,
        then 
        \[ \var\para{\sum_{i=1}^{n}X_i} = 
        \sum_{i=1}^{n} \var(X_i)
        = \sum_{i=1}^{n} \Sigma_i \]
        So the random variable follows
        the distribution 
        $\normal\para{0, \sum_{i=1}^{n} \Sigma_i}$,
        which means its desnity function is:
         \[ f(x) = (2\pi)^{-\sfrac{d}{2}}
        \det\para{\sum_{i=1}^{n} \Sigma_i}^{\sfrac{1}{2}}
        \exp\para{-\dfrac{1}{2}x^T
        \para{\sum_{i=1}^{n} \Sigma_i}\inv x} \]
        \item 
        We know that for a random vector
        in $\R^{2d}$
        \[ X = \vecTwo{X_1}{X_2} \]
        (where each of $X_1$ and $X_2$
        are themselves vectors, though we
        can treat them as block components),
        the variance is just
        the covariance matrix
        containing the covariances of each
        pair of entries in the vector
        (if the entry itself is a vector
        we can treat its covariance like 
        a block element):
        \[ \cov\para{\vecTwo{X_1}{X_2}}
        = \matTwo{\var(X_1)}{\cov(X_1X_2)}
        {\cov(X_2X_1)}{\var(X_2)} 
        = \matTwo{\Sigma_1}{C}
        {C^{T}}{\Sigma_2} = M \]
        (Here $C$ is not a scalar,
        but a matrix, since both $X_1$
        and $X_2$ are random vectors,
        so we need to transpose it when the order
        changes). \\
        As for the mean of this random variable,
        it is just 
        \[ \E\brac{ \vecTwo{X_1}{X_2}}
        = \vecTwo{\E(X_1)}{\E(X_2)}
        = \vecTwo{0}{0} \]
        So the random variable follows the
        distribution
        \[ X \sim \normal(0, M) \]
        which means that its density function is
        \[ f(x) = (2\pi)^{-\sfrac{d}{2}}
        \det(M)^{\sfrac{1}{2}}
        \exp\para{-\dfrac{1}{2}x^T
        M\inv x} \]
    \end{enumerate}

    \newpage

    \subsection*{Ex 4}

    \begin{enumerate}[label=\alph*)]
        \item 
        We have a function $f$:
        \[ f:\R^n \to \R \]
        \[ f(u) 
        = \dfrac{1}{2}\sum_{i=1}^{n}(u_i - c_i)^2
        + \dfrac{\mu}{2}\sum_{i=1}^{n-1}(u_{i+1} - u_i)^2
        \]
        defined for some $c \in \R^n$
        and scalar $\mu \in \R^+$. \\
        In order to write it using vector notation,
        we not ethat the first sum is just
        the 2-norm squared of $u - c$:
        \[ \dfrac{1}{2}\sum_{i=1}^{n}(u_i - c_i)^2
        = \dfrac{1}{2}\| u - c \|^2 \]
        Moreover, the second term:
        \[ \dfrac{\mu}{2}
        \sum_{i=1}^{n-1}(u_{i+1} - u_i)^2 \]
        is the sum of differences squared between
        consecutive entries in $u$. \\
        This can be thought of as the 2-norm squared
        of an $(n-1)$-dimensional vector $w$
        that encodes the difference of consecutive
        entries in $u$:
        \[ w = \begin{bmatrix}
            u_2 - u_1 \\
            u_3 - u_2 \\
            \vdots \\
            u_n - u_{n-1}
        \end{bmatrix} \]
        We know that given an $m \times n$
        matrix $A$,
        the result of matrix-vector
        multiplication $Av$ is
        an $m$-dimensional vector:
        \[ Av = \begin{bmatrix}
            A_{11}v_1 + A_{12}v_2 + \dots A_{1n}v_n \\
            A_{21}v_1 + A_{22}v_2 + \dots A_{2n}v_n \\
            \vdots \\
            A_{m1}v_1 + A_{m2}v_2 + \dots A_{mn}v_n
        \end{bmatrix} \]
        Where each entry of the resulting
        vector is the sum of all entries
        in $v$ multiplied by some entry
        in the matrix. \\
        The vector $w$ of differences in $u$
        can be rewritten as such:
        \[ w = \begin{bmatrix}
            (-1)u_1 + (1)u_2 + (0)u_3 + (0)u_4 
            + \dots + (0)u_n \\
            (0)u_1 + (-1)u_2 + (1)u_3 + (0)u_4 
            + \dots + (0)u_n \\
            \vdots \\
            (0)u_1 + \dots + (0)u_{n-3} + (0)u_{n-2} 
            + (-1)u_{n-1} + (1)u_n \\
        \end{bmatrix} \]
        where each entry in $w$ is the sum
        of entries in $u$ multiplied by some scalar,
        which can be encoded in a matrix $M$:
        \[ w = Mu =
        \begin{bmatrix}
            -1 & 1 & 0 & \dots & 0 \\
            0 & -1 & 1 & \dots & 0 \\
            \vdots & 0 & \ddots & \ddots & 0\\
            0 & 0 & \dots & -1 & 1 \\
        \end{bmatrix}
        \begin{bmatrix}
           u_1 \\ u_2 \\ \vdots \\ u_n
        \end{bmatrix} \]
        So, the second term can be written as
         \[ \dfrac{\mu}{2}
        \sum_{i=1}^{n-1}(u_{i+1} - u_i)^2
         = \dfrac{\mu}{2}\| Mu \|^2 \]
        So to conclude:
        \[ f(u) = \dfrac{1}{2}\| u - c \|^2
        + \dfrac{\mu}{2}\| Mu \|^2  \]
        is the vector notation of $f$. \\
        \item
        We know that the gradient of $f$ is 
        the vector field:
        \[ \nabla f = \para{\partialdd{f}{u_1},
        \partialdd{f}{u_2}, \dots, \partialdd{f}{u_n}} \]
        We can compute each term $\partial{f}{u_k}$
        of that vector by differentiating $f$.
        Only the terms containing $u_k$
        have to be considered however:
        \[ 
            \partialdd{f}{u_k}
            = \dfrac{1}{2}\partialdd{}{u_k}
            (u_k - c_k)^2
            - \dfrac{\mu}{2}\partialdd{}{u_k}
            (u_{k+1} - u_k)^2
            + \dfrac{\mu}{2}\partialdd{}{u_k}
            (u_{k} - u_{k-1})^2
        \]
        \[ \partialdd{f}{u_k} = (u_k - c_k)
        - \mu(u_{k+1} - u_k) + \mu(u_{k} - u_{k-1}) \]
        Where the term $\mu(u_{k+1} - u_k)$
        won't appear in the last entry of $\nabla f$,
        and the term  $\mu(u_{k} - u_{k-1})$
        won't appear in the first entry of $\nabla f$:
        \[ \nabla f = \begin{bmatrix}
        (u_1 - c_1) - \mu(u_{2} - u_1) \\
        \vdots \\
        (u_k - c_k) + \mu(2u_k - u_{k+1} - u_{k-1}) \\
        \vdots \\
        (u_n - c_n) + \mu(u_{n} - u_{n-1})
        \end{bmatrix} \]
        \item 
        We want to rewrite $\nabla f$ using vector
        notation. \\
        We can start by writing:
        \[ \nabla f = \begin{bmatrix}
        (u_1 - c_1) \\
        \vdots \\
        (u_k - c_k) \\
        \vdots \\
        (u_n - c_n)
        \end{bmatrix} + \begin{bmatrix}
        \mu(u_1-u_{2}) \\
        \vdots \\
        \mu(2u_k - u_{k+1} - u_{k-1}) \\
        \vdots \\
        \mu(u_{n} - u_{n-1})
        \end{bmatrix} \]
        We rewrite the first term as the
        difference of the two vectors $u$ and $c$:
        \[\begin{bmatrix}
            (u_1 - c_1) \\
            \vdots \\
            (u_k - c_k) \\
            \vdots \\
            (u_n - c_n)
        \end{bmatrix}  = u - c \]
        As for the second term,
        notice that all we are doing is multiplying
        terms in the vector $u$
        by scalars:
        \[ \begin{bmatrix}
            \mu(u_1-u_{2}) \\
            \vdots \\
            \mu(2u_k - u_{k+1} - u_{k-1}) \\
            \vdots \\
            \mu(u_{n} - u_{n-1})
        \end{bmatrix} 
        = \mu \begin{bmatrix}
            (1)u_1 + (-1)u_{2} + (0)u_3 + (0)u_4
            + \dots + (0)u_n \\
            \vdots \\
            (0)u_1 + \dots + (-1)u_{k-1} + (2)u_k
            + (-1)u_{k+1} + \dots + (0)u_n \\
            \vdots \\
            (0)u_1 + (0)u_{2} + \dots + 
            (-1)u_{n-1} + (1)u_{n} \\
        \end{bmatrix}  \]
        We can encode these scalars in an
        $n \times n$ matrix $N$ such that:
        \[ \mu Nu = 
        \mu \begin{bmatrix}
            1 & -1 & 0 & 0 & \dots & 0 \\ 
            -1 & 2 & -1 & 0 & \dots & 0 \\
            0 & -1 & 2 & -1 & \ddots & 0 \\
            \vdots & 0 & \ddots & \ddots & \ddots & 0 \\
            \vdots & \vdots & \ddots & -1 & 2 & -1 \\
            0 & 0 & \dots & 0 & -1 & 1
        \end{bmatrix}
        \begin{bmatrix}
            u_1 \\ u_2 \\ \vdots \\ u_n
        \end{bmatrix} \]
        To conclude,
        we can write:
        \[ \nabla f = u - c + \mu Nu \]
        in vector notation.
        \item 
        In order to solve
        \[ \nabla f = 0 \]
        We must solve
        \[  u - c + \mu Nu = 0 \]
        \[  Iu + \mu Nu = c \]
        \[  (\mu N + I)u = c \]
        which is a linear system that can solved
        for $u$:
        \[ u =  (\mu N + I)\inv c \]
        granted the inverse exists.
        \item 
        The matrix $\mu N + I$ must be invertible
        for there to be a unique solution:
        \[ \mu N + I = 
        \begin{bmatrix}
            \mu+1 & -\mu & 0 & 0 & \dots & 0 \\ 
            -\mu & 2\mu + 1 & -\mu & 0 & \dots & 0 \\
            0 & -\mu & 2\mu + 1 & -\mu & \ddots & 0 \\
            \vdots & 0 & \ddots & \ddots & \ddots & 0 \\
            \vdots & \vdots & \ddots & -\mu 
            & 2\mu + 1 & -\mu \\
            0 & 0 & \dots & 0 & -\mu & \mu + 1
        \end{bmatrix}
        \]
    \end{enumerate}
    
    \begingroup
    \color{red}
    \subsection*{Correction}
    \begin{enumerate}[label=\alph*)]
        \item
        
        \item 

        \item
        The answer is correct,
        but an easier way to find the gradient
        is to just use vector and matrix differentials.
        We already have $f$ in vector notation:
        \[ f(u) = \dfrac{1}{2}\| u - c \|^2
        + \dfrac{\mu}{2}\| Mu \|^2  \]
        \[ \nabla f = \dfrac{1}{2}2(u-c)
        + \dfrac{\mu}{2}2M^TMu  \]
        \[ \nabla f = u-c + \mu M^TMu  \]
        \item 

        \item
        We know that $M^TMu$
        is positive semi-definite,
        since clearly $x^T(M^TMu)x > 0$
        (for reasons highlighted above). \\
        Moreover, because $\mu > 0$,
        then $\mu M^TMu + I$
        is symmetric positive-definite,
        which is invertible. \\
        Positive-definite means
        positive semi-definite with all
        positive eigenvalues,
        and if you add symmetry then we
        guarantee it is invertible. \\
    \end{enumerate}
    \endgroup


\end{document}
