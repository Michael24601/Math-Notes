

\documentclass[12pt]{article}

\usepackage[margin=1in]{geometry}


%===============================================================================
%================================== PACKAGES ===================================
%===============================================================================

% For using float option H that places figures 
% exatcly where we want them
\usepackage{float}
% makes figure font bold
\usepackage{caption}
\captionsetup[figure]{labelfont=bf}
% For text generation
\usepackage{lipsum}
% For drawing
\usepackage{tikz}
% For smaller or equal sign and not divide sign
\usepackage{amssymb}
% For the diagonal fraction
\usepackage{xfrac}
% For enumerating exercise parts with letters instead of numbers
\usepackage{enumitem}
% For dfrac, which forces the fraction to be in display mode (large) e
% even in math mode (small)
\usepackage{amsmath}
% For degree sign
\usepackage{gensymb}
% For "\mathbb" macro
\usepackage{amsfonts}
% For positioning 
\usepackage{indentfirst}
\usetikzlibrary{shapes,positioning,fit,calc}
% for adjustwidth environment
\usepackage{changepage}
% for arrow on top
\usepackage{esvect}
% for mathbb 1
\usepackage{bbm}
% for mathsrc
\usepackage[mathscr]{eucal}
% For degree sign
\usepackage{gensymb}
% For quotes
\usepackage{csquotes}
% For vertical lines
\usepackage{mathtools}
% For cols
\usepackage{multicol}

% for tikz
\usepackage{pgfplots}
\pgfplotsset{compat=1.18}
\usepackage{amsmath}
\usepgfplotslibrary{groupplots}


%===============================================================================
%==================================== FONTS ====================================
%===============================================================================


% Mathcal
\newcommand{\acal}{\mathcal{A}}
\newcommand{\bcal}{\mathcal{B}}
\newcommand{\ccal}{\mathcal{C}}
\newcommand{\dcal}{\mathcal{D}}
\newcommand{\ecal}{\mathcal{E}}
\newcommand{\fcal}{\mathcal{F}}
\newcommand{\gcal}{\mathcal{G}}
\newcommand{\hcal}{\mathcal{H}}
\newcommand{\ical}{\mathcal{I}}
\newcommand{\jcal}{\mathcal{J}}
\newcommand{\kcal}{\mathcal{K}}
\newcommand{\lcal}{\mathcal{L}}
\newcommand{\mcal}{\mathcal{M}}
\newcommand{\ncal}{\mathcal{N}}
\newcommand{\ocal}{\mathcal{O}}
\newcommand{\pcal}{\mathcal{P}}
\newcommand{\qcal}{\mathcal{Q}}
\newcommand{\rcal}{\mathcal{R}}
\newcommand{\scal}{\mathcal{S}}
\newcommand{\tcal}{\mathcal{T}}
\newcommand{\ucal}{\mathcal{U}}
\newcommand{\vcal}{\mathcal{V}}
\newcommand{\wcal}{\mathcal{W}}
\newcommand{\xcal}{\mathcal{X}}
\newcommand{\ycal}{\mathcal{Y}}
\newcommand{\zcal}{\mathcal{Z}}

% Mathfrak
\newcommand{\afrak}{\mathfrak{A}}
\newcommand{\bfrak}{\mathfrak{B}}
\newcommand{\cfrak}{\mathfrak{C}}
\newcommand{\dfrak}{\mathfrak{D}}
\newcommand{\efrak}{\mathfrak{E}}
\newcommand{\ffrak}{\mathfrak{F}}
\newcommand{\gfrak}{\mathfrak{G}}
\newcommand{\hfrak}{\mathfrak{H}}
\newcommand{\ifrak}{\mathfrak{I}}
\newcommand{\jfrak}{\mathfrak{J}}
\newcommand{\kfrak}{\mathfrak{K}}
\newcommand{\lfrak}{\mathfrak{L}}
\newcommand{\mfrak}{\mathfrak{M}}
\newcommand{\nfrak}{\mathfrak{N}}
\newcommand{\ofrak}{\mathfrak{O}}
\newcommand{\pfrak}{\mathfrak{P}}
\newcommand{\qfrak}{\mathfrak{Q}}
\newcommand{\rfrak}{\mathfrak{R}}
\newcommand{\sfrak}{\mathfrak{S}}
\newcommand{\tfrak}{\mathfrak{T}}
\newcommand{\ufrak}{\mathfrak{U}}
\newcommand{\vfrak}{\mathfrak{V}}
\newcommand{\wfrak}{\mathfrak{W}}
\newcommand{\xfrak}{\mathfrak{X}}
\newcommand{\yfrak}{\mathfrak{Y}}
\newcommand{\zfrak}{\mathfrak{Z}}

% Mathscr
\newcommand{\ascr}{\mathscr{A}}
\newcommand{\bscr}{\mathscr{B}}
\newcommand{\cscr}{\mathscr{C}}
\newcommand{\dscr}{\mathscr{D}}
\newcommand{\escr}{\mathscr{E}}
\newcommand{\fscr}{\mathscr{F}}
\newcommand{\gscr}{\mathscr{G}}
\newcommand{\hscr}{\mathscr{H}}
\newcommand{\iscr}{\mathscr{I}}
\newcommand{\jscr}{\mathscr{J}}
\newcommand{\kscr}{\mathscr{K}}
\newcommand{\lscr}{\mathscr{L}}
\newcommand{\mscr}{\mathscr{M}}
\newcommand{\nscr}{\mathscr{N}}
\newcommand{\oscr}{\mathscr{O}}
\newcommand{\pscr}{\mathscr{P}}
\newcommand{\qscr}{\mathscr{Q}}
\newcommand{\rscr}{\mathscr{R}}
\newcommand{\sscr}{\mathscr{S}}
\newcommand{\tscr}{\mathscr{T}}
\newcommand{\uscr}{\mathscr{U}}
\newcommand{\vscr}{\mathscr{V}}
\newcommand{\wscr}{\mathscr{W}}
\newcommand{\xscr}{\mathscr{X}}
\newcommand{\yscr}{\mathscr{Y}}
\newcommand{\zscr}{\mathscr{Z}}

% Mathbb
\newcommand{\abb}{\mathbb{A}}
\newcommand{\bbb}{\mathbb{B}}
\newcommand{\cbb}{\mathbb{C}}
\newcommand{\dbb}{\mathbb{D}}
\newcommand{\ebb}{\mathbb{E}}
\newcommand{\fbb}{\mathbb{F}}
\newcommand{\gbb}{\mathbb{G}}
\newcommand{\hbb}{\mathbb{H}}
\newcommand{\ibb}{\mathbb{I}}
\newcommand{\jbb}{\mathbb{J}}
\newcommand{\kbb}{\mathbb{K}}
\newcommand{\lbb}{\mathbb{L}}
\newcommand{\mbb}{\mathbb{M}}
\newcommand{\nbb}{\mathbb{N}}
\newcommand{\obb}{\mathbb{O}}
\newcommand{\pbb}{\mathbb{P}}
\newcommand{\qbb}{\mathbb{Q}}
\newcommand{\rbb}{\mathbb{R}}
\newcommand{\sbb}{\mathbb{S}}
\newcommand{\tbb}{\mathbb{T}}
\newcommand{\ubb}{\mathbb{U}}
\newcommand{\vbb}{\mathbb{V}}
\newcommand{\wbb}{\mathbb{W}}
\newcommand{\xbb}{\mathbb{X}}
\newcommand{\ybb}{\mathbb{Y}}
\newcommand{\zbb}{\mathbb{Z}}


%===============================================================================
%=============================== SPECIAL SYMBOLS ===============================
%===============================================================================


% Orbit (group theory)
\newcommand{\orbit}{\mathcal{O}}
% Normal group
\newcommand{\normal}{\mathcal{N}}
% Indicator function
\newcommand{\indicator}{\mathbbm{1}}
% Laplace transform
\newcommand{\laplace}[1]{\mathcal{L}}
% Epsilon shorthand
\newcommand{\eps}{\varepsilon}
% Omega
\newcommand{\om}{\omega}
\newcommand{\Om}{\Omega}

%===============================================================================
%================================== OPERATORS ==================================
%===============================================================================


% Inverse exponent
\newcommand{\inv}[0]{^{-1}}
% Overline bar
\newcommand{\olsi}[1]{\,\overline{\!{#1}}}
% Less than or equal slanted
\newcommand{\seqs}{\leqslant}
% Greater or equal slanted
\newcommand{\geqs}{\geqslant}
% Subset or equal
\newcommand{\sub}{\subseteq}
% Proper subset
\newcommand{\prosub}{\subset}
% from
\newcommand{\from}{\leftarrow}

% Parantheses
\newcommand{\para}[1]{\left( #1 \right)}
% Curly Braces
\newcommand{\curl}[1]{\left\{ #1 \right\}}
% Brackets
\newcommand{\brac}[1]{\left[ #1 \right]}
% Angled Brackets
\newcommand{\ang}[1]{\left\langle #1 \right\rangle}
% Norm
\newcommand{\norm}[1]{\left\| #1 \right\|}

% Piece wise (use \\ between cases)
\newcommand{\piecewise}[1]{\begin{cases} #1 \end{cases}}

% Bold symbol shorthand
\newcommand{\bl}{\boldsymbol}

% Vertical space
\newcommand{\vs}[1]{\vspace{#1 pt}}
% Horizontal ertical space
\newcommand{\hs}[1]{\hspace{#1 pt}}


%===============================================================================
%============================== TEXT BASED SYMBOLS =============================
%===============================================================================


% Radians
\newcommand{\rad}{\text{rad}}
% Least Common Multiple
\newcommand{\lcm}{\text{lcm}}
% Automorphism
\newcommand{\Aut}{\text{Aut}}
% Variance
\newcommand{\var}{\text{Var}}
% Covariance
\newcommand{\cov}{\text{Cov}}
% Cofactor (matrix)
\newcommand{\cof}{\text{Cof}}
% Adjugate (matrix)
\newcommand{\adj}{\text{Adj}}
% Trace (matrix)
\newcommand{\tr}{\text{tr}}
% Standard deviation
\newcommand{\std}{\text{Std}}
% Correlation coefficient
\newcommand{\corr}{\text{Corr}}
% Sign
\newcommand{\sign}{\text{sign}}

% And text
\newcommand{\AND}{\text{ and }}
% Or text 
\newcommand{\OR}{\text{ or }}
% If text
\newcommand{\IF}{\text{ if }}
% When text
\newcommand{\WHEN}{\text{ when }}
% Then text
\newcommand{\THEN}{\text{ then }}

% 1st
\newcommand{\st}[1]{#1^{\text{st}}}
% 2nd
\newcommand{\nd}[1]{#1^{\text{nd}}}
% 3rd
\newcommand{\rd}[1]{#1^{\text{rd}}}
% nth
\newcommand{\nth}[1]{#1^{\text{th}}}


%===============================================================================
%========================= PROBABILITY AND STATISTICS ==========================
%===============================================================================


% Permutation
\newcommand{\perm}[2]{{}^{#1}\!P_{#2}}
% Combination
\newcommand{\comb}[2]{{}^{#1}C_{#2}}

% Baye's risk
\newcommand{\risk}[1]{\mathscr{R}_{#1}}
% Baye's optimal risk
\newcommand{\riskOptimal}[1]{\mathscr{R}_{#1}^*}
% Baye's empirical risk
\newcommand{\riskEmpirical}[2]{\hat{\mathscr{R}}_{#1}^{#2}}


%===============================================================================
%=================================== CALCULUS ==================================
%===============================================================================


% d over d derivative
\newcommand{\dd}[2]{\dfrac{d#1}{d#2}}
% partial d over d derivative
\newcommand{\partialdd}[2]{\dfrac{\partial #1}{\partial #2}}
% delta d over d derivative
\newcommand{\deltadd}[2]{\dfrac{\Delta #1}{\Delta #2}}

% Integration between a and b
\newcommand{\integral}[4]{\int_{#1}^{#2} #3 \, #4}
% Integration in some space 
\newcommand{\boundIntegral}[2]{\int_{#1} #2 \, d#1}

% Limit
\newcommand{\limit}[3]{\lim_{#1 \to #2} #3}


%===============================================================================
%================================  BIG SYMBOLS  ================================
%===============================================================================


% Sum
\newcommand{\sumof}[2]{\sum_{#1}^{#2}}
% Product
\newcommand{\productof}[2]{\prod_{#1}^{#2}}
% Union
\newcommand{\unionof}[2]{\bigcup_{#1}^{#2}}
% Intersection
\newcommand{\intersectionof}[2]{\bigcap_{#1}^{#2}}
% Or
\newcommand{\orof}[2]{\bigvee_{#1}^{#2}}
% And
\newcommand{\andof}[2]{\bigwedge_{#1}^{#2}}


%===============================================================================
%=============================== LINEAR ALGEBRA ================================
%===============================================================================


% Bold vector arrow
\newcommand{\bv}[1]{\vec{\mathbf{#1}}}

% Matrix or vector (use // for column, & for row) 
% with brackets
\newcommand{\bmat}[1]{\begin{bmatrix} #1 \end{bmatrix}}
% Matrix or vector (use // for column, & for row) 
% with curved brackets
\newcommand{\pmat}[1]{\begin{pmatrix} #1 \end{pmatrix}}
% Matrix or vector (use // for column, & for row) 
% with lines on either side 
\newcommand{\lmat}[1]{\begin{vmatrix} #1 \end{vmatrix}}
% Matrix or vector (use // for column, & for row) 
% with curly braces
\newcommand{\cmat}[1]{\begin{Bmatrix} #1 \end{Bmatrix}}
% Matrix or vector (use // for column, & for row) 
% with no braces
\newcommand{\mat}[1]{\begin{matrix} #1 \end{matrix}}


%===============================================================================
%================================ LARGE OBJECTS ================================
%===============================================================================

% Multiple lines
\newcommand{\multiline}[1]{
\begin{align*}
    #1
\end{align*}
}

% Multiple lines with equation numbers
\newcommand{\eqmultiline}[1]{
\begin{align*}
    #1
\end{align*}
}

% Color
\newcommand{\colorText}[2]{
\begingroup
\color{#1}
    #2
\endgroup
}

% Centered figure
\newcommand{\centerFigure}[2]{
    \begin{figure}[h]
        \centering
            #1
        \caption{#2}
    \end{figure}
}

% Tikz figure
\newcommand{\tikzGraphic}[1]{
    \begin{center}
    \begin{tikzpicture}
        #1
    \end{tikzpicture}
    \end{center}
}

% Enumerate numbers (seperate by \item)
\newcommand{\numbers}{\textbf{\number*)}}

% Enumerate letters (seperate by \item)
\newcommand{\letters}{\textbf{\alph*)}}

\title{
    \Huge Machine Learning \\
    \Large Assignment VII
}
\date{2025-05-29}
\author{Michael Saba}

\begin{document}
\pagenumbering{gobble}
\maketitle
\newpage
\setlength{\parindent}{0pt}
\pagenumbering{arabic}

\subsection*{Ex 1}

\begin{enumerate}[label = \letters]
\item 
    In order for a function to be classification
    calibrated, it needs to be convex,
    and it has to have a derivative at $x=0$,
    which is negative. \\
    We have:
    \[ \Phi(x) = \log(1 + e^{-x}) \]
    \[ \Phi'(x) = \dfrac{-e^{-x}}{1 + e^{-x}} \]
    \[ \Phi''(x) = \dfrac{e^{-x}(1 + e^{-x})
    - e^{-x}e^{-x}}{(1 + e^{-x})^2}
    = \dfrac{e^{-x}}{(1+e^{-x})^2} \]
    Both the numerator and denominator
    of the second derivative are positive,
    so the function is convex. \\
    Moreover, $\Phi'(x)$
    is defined at $x = 0$, and 
    $\Phi'(0) = -\dfrac{1}{2}$, which is negative.
\item 
    If we have the model:
    \[ p(y \mid x, \theta) = \dfrac{1}
    {1 + e^{-y\ang{\varphi(x), \theta}}} \]
    The negative logarithm of the model is:
    \[ -\log\para{\dfrac{1}
    {1 + e^{-y\ang{\varphi(x), \theta}}}}
    = \log\para{
    {1 + e^{-y\ang{\varphi(x), \theta}}}} \]
    Which is the logistic loss 
    $\Phi(y\ang{\varphi(x), \theta})$.
\item 
    We have:
    \[ p(y = 1 \mid x, \theta) = \dfrac{1}{1 + e} \]
    \[ \ell(y\ang{\varphi(x), \theta}) 
    = \log(1 + e) \]
    The model's prediction is the sign of the
    surrogate function $\ang{\varphi(x), \theta}$,
    which is $-1$, which has a negative sign.
\item 
    We have the softmax for $K$ classes:
    \[ p(y = k \mid x, \theta)
    = \dfrac{e^{\ang{\varphi(x), \theta_k}}}
    {\sum_i^K e^{\ang{\varphi(x), \theta_i}}}\]
    The gradient with respect to $\theta_k$ is:
    \[ \partialdd{}{\theta_k}
    = \dfrac{(e^{\ang{\varphi(x), \theta_k}})'
    \sum_i^K e^{\ang{\varphi(x), \theta_i}}
    - (\sum_i^K e^{\ang{\varphi(x), \theta_i}})'
    e^{\ang{\varphi(x), \theta_k}}}
    {(\sum_i^K e^{\ang{\varphi(x), \theta_i}})^2} \]
    \[ \partialdd{}{\theta_k}
    = \dfrac{\varphi(x)e^{\ang{\varphi(x), \theta_k}}
    \sum_i^K e^{\ang{\varphi(x), \theta_i}}
    - \varphi(x)e^{\ang{\varphi(x), \theta_k}}
    e^{\ang{\varphi(x), \theta_k}}}
    {(\sum_i^K e^{\ang{\varphi(x), \theta_i}})^2} \]
    \[ \partialdd{}{\theta_k}
    = \varphi(x) \dfrac{
    e^{\ang{\varphi(x), \theta_k}}}
    {\sum_i^K e^{\ang{\varphi(x), \theta_i}}}
    \dfrac{\sum_i^K e^{\ang{\varphi(x), \theta_i}}
    - e^{\ang{\varphi(x), \theta_k}}}
    {\sum_i^K e^{\ang{\varphi(x), \theta_i}}} \]
    \[ \partialdd{}{\theta_k}
    = \varphi(x) p(y = k \mid x, \theta)
    \para{1 - p(y = k \mid x, \theta)} \]
    We can now set $K = 2$,
    and see that we get the normal binary
    classification:
    \[  p(y = k \mid x, \theta) = 
    \dfrac{e^{\ang{\varphi(x), \theta_k}}}
    {\sum_i^K e^{\ang{\varphi(x), \theta_i}}}
    = \dfrac{e^{\ang{\varphi(x), \theta_k}}}
    {e^{\ang{\varphi(x), \theta_1}}
    + e^{\ang{\varphi(x), \theta_2}}}\]
    But we have two parameter vectors here,
    and we need one, so we can take
    $\theta = \theta_1 - \theta_2$,
    and set $k = 1$ for now:
    \multiline{p(y = 1 \mid x, \theta) & = 
    \dfrac{e^{\ang{\varphi(x), \theta_1}}}
    {e^{\ang{\varphi(x), \theta_1}}
    + e^{\ang{\varphi(x), \theta_2}}}
    = \dfrac{1}
    {1 + \dfrac{e^{\ang{\varphi(x), \theta_2}}}
    {e^{\ang{\varphi(x), \theta_1}}}} \\
    & = \dfrac{1}
    {1 + e^{\ang{\varphi(x), \theta_2 - \theta_1}}}
    = \dfrac{1}
    {1 + e^{-\ang{\varphi(x), \theta_1 - \theta_2}}}
    = \dfrac{1}
    {1 + e^{-y\ang{\varphi(x), \theta_1 - \theta_2}}}}
    This is just the binary case. \\
\end{enumerate}

\colorText{red}{
\subsection*{Correction}
\begin{enumerate}[label = \letters]
    \item Not a correction, but a note; the theorem
    that says a convex margin-based loss
    function is classification calibrated
    if it's derivative at $0$ is negative
    is sufficient but not necessary to show a function
    is classification clibrated. \\
    Also we should have noted that the function is
    margin-based (has $yg(x)$ as input). \\
    Another note is that applying a monotonically
    increasing function to a convex term results
    in a convex function, which we could have used
    instead of doing the second derivative. \\
    Also to show that the derivative exists,
    we should technically show that the limit
    from the left and right side is equal to $\Phi'(0)$.
    \item 
    \item 
    For the model predition, we should have just noted
    that $p(y = 1 \mid x, \theta) < 0.5$.
    \item 
    Should have calculated the gradient of the loss,
    not the density:
    \[ p(y = k \mid x, \theta)
    = \dfrac{e^{\ang{\varphi(x), \theta_k}}}
    {\sum_i^K e^{\ang{\varphi(x), \theta_i}}}\]
    We consider $z_k = \ang{\varphi(x), \theta_k}$"
    \[ \ell(y, z_k) =
    -\log(p(y = k \mid x, \theta))
    =\log\para{\dfrac{\sum_i^K e^{\ang{\varphi(x), \theta_i}}}
    { e^{\ang{\varphi(x), \theta_k}} }} \]
    \[ \ell(y, z_k) = \log\para{\sum_i^K e^{\ang{\varphi(x), \theta_i}}}
    - \log\para{{e^{\ang{\varphi(x), \theta_k}}}} \]
    \[ \ell(y, z_k) = \log\para{\sum_i^K e^{\ang{\varphi(x), \theta_i}}}
    - \ang{\varphi(x), \theta_k} \]
    Then we calculate the gradient
    in terms of $\theta_k$:
    \[ \nabla \ell(y, z_k)
    = \varphi(x) + 
    \dfrac{\varphi(x)e^{\ang{\varphi(x), \theta_k}}}
    {\sum_i^K e^{\ang{\varphi(x), \theta_i}}} \]
    Note that when he solved it, he solved it
    for a general case: the derivative with respect to
    $\theta^k$ of teh softmax where $y = j$,
    not necessarily $y = k$, so in that case
    the first term may or may not be $0$
    depending on whether the $k = j$,
    so we use an indicator function: 
    $\varphi(x) \indicator_{k = j}$. \\
    Also for th second part, I computed
    $p(y = 1 \mid x, \theta)$,
    but should have also computed
    $p(y = -1 \mid x, \theta)$,
    and shown that they lead to the same function
    with the sign flipped
    (in the last step, multiplying by $y$
    would flip the order for $\theta_2 - \theta_1$).
\end{enumerate}
}
\newpage

\subsection*{Ex 2}

\begin{enumerate}[label = \letters]
    \item 
    The issue with LDA assumes that the within-class
    covariances are equal, and they are not here,
    so we can exclude it. \\
    The only issue with SVM hard margin is if the
    data is not linearlly seperable. But this doesn't
    apply here. \\
    Logistic regression is far too sensitive to outliers.
    (If we have ont point from class A close to class B
    that is an outlier, logistic regression
    would be sensitive to the outlier and overfit). 
    The only advantage of logistic regression is that the
    output is a probability, which gives us
    a confidence score for the classifier. \\
    So we can go with a hard-margin SVM
    or maybe even logistic regression. \\ 
    \item 
    LDA and hard margin-SVM would not work since they
    are not linearlly seperable. But logistic regression
    and soft-margin SVM work.
    \item 
    Regularization works to mitigate overfitting.
    So it would help penalize overfitting,
    which would make the models reistant to outliers. \\
    In SVM, the soft-margin is the regularization. \\
    In Logistic regression, we can use L1 or L2 norms.
\end{enumerate}

\colorText{red}{
\subsection*{Correction}
\begin{enumerate}[label = \letters]
    \item We can choose SVM or logistic regression,
    so long as our reasoning is sound. But LDA is
    wrong.
    \item
    \item
\end{enumerate}
}

\newpage

\subsection*{Ex 3}

\begin{enumerate}[label = \letters]
    \item 
    The recall is the percentage of positive
    predictions among all data points that are actually
    posisve:
    \[ \text{Recall}_P = \dfrac{tp}{tp + fn} 
    = \dfrac{40}{40 + (5 + 5)}  = \dfrac{4}{5} \]
    \[ \text{Recall}_T = \dfrac{tp}{tp + fn} 
    = \dfrac{30}{30 + (6 + 4)}  = \dfrac{3}{4} \]
    \[ \text{Recall}_L = \dfrac{tp}{tp + fn} 
    = \dfrac{53}{53 + (3 + 4)}  = \dfrac{53}{60} \]
    The precision is the percentage of positive
    predictions among all data points that are
    predicted to be positive:
    \[ \text{Precision}_P = \dfrac{tp}{tp + fp} 
    = \dfrac{40}{40 + (4 + 3)}  = \dfrac{40}{47} \]
    \[ \text{Precision}_T = \dfrac{tp}{tp + fp} 
    = \dfrac{30}{30 + (4 + 5)}  = \dfrac{30}{39} \]
    \[ \text{Precision}_L = \dfrac{tp}{tp + fp} 
    = \dfrac{53}{53 + (5 + 6)}  = \dfrac{53}{64} \]
    The F-score is the harmonic mean:
    \[ \text{F-score}_P = 
    \dfrac{2 * \text{Recall}_P * \text{Precision}_P}
    {\text{Recall}_P + \text{Precision}_P} \approx
    0.824 \]
    \[ \text{F-score}_T = 
    \dfrac{2 * \text{Recall}_T * \text{Precision}_T}
    {\text{Recall}_T + \text{Precision}_T} \approx
    0.759 \]
    \[ \text{F-score}_L = 
    \dfrac{2 * \text{Recall}_L * \text{Precision}_L}
    {\text{Recall}_L + \text{Precision}_L} \approx
    0.855 \]
\item 
    The accuracy is the rate of correct
    predictions (one accuracy for all classes):
    \[ \text{Accuracy} = \dfrac{\sum (tp + tn)}
    {\sum (tp + tn + fp + fn)} = \dfrac{\sum (tp + tn)}{n}
    = \dfrac{40 + 30 + 53}{150} = \dfrac{123}{150} \]
    The error is just the rate of incorrect predictions:
    \[ \text{Error} = 1 - \text{Accuracy} = \dfrac{27}{150} \]
    The true positive rate is just the recall,
    so we need the average of the recalls:
    \[ \text{Average True Positive Rate}
    \approx 0.811 \]
    (the positive predicted value is the precision). \\

\end{enumerate}

\newpage

\end{document}