

\documentclass[12pt]{article}

\usepackage[margin=1in]{geometry}


%===============================================================================
%================================== PACKAGES ===================================
%===============================================================================

% For using float option H that places figures 
% exatcly where we want them
\usepackage{float}
% makes figure font bold
\usepackage{caption}
\captionsetup[figure]{labelfont=bf}
% For text generation
\usepackage{lipsum}
% For drawing
\usepackage{tikz}
% For smaller or equal sign and not divide sign
\usepackage{amssymb}
% For the diagonal fraction
\usepackage{xfrac}
% For enumerating exercise parts with letters instead of numbers
\usepackage{enumitem}
% For dfrac, which forces the fraction to be in display mode (large) e
% even in math mode (small)
\usepackage{amsmath}
% For degree sign
\usepackage{gensymb}
% For "\mathbb" macro
\usepackage{amsfonts}
% For positioning 
\usepackage{indentfirst}
\usetikzlibrary{shapes,positioning,fit,calc}
% for adjustwidth environment
\usepackage{changepage}
% for arrow on top
\usepackage{esvect}
% for mathbb 1
\usepackage{bbm}
% for mathsrc
\usepackage[mathscr]{eucal}
% For degree sign
\usepackage{gensymb}
% For quotes
\usepackage{csquotes}
% For vertical lines
\usepackage{mathtools}
% For cols
\usepackage{multicol}

% for tikz
\usepackage{pgfplots}
\pgfplotsset{compat=1.18}
\usepackage{amsmath}
\usepgfplotslibrary{groupplots}


%===============================================================================
%==================================== FONTS ====================================
%===============================================================================


% Mathcal
\newcommand{\acal}{\mathcal{A}}
\newcommand{\bcal}{\mathcal{B}}
\newcommand{\ccal}{\mathcal{C}}
\newcommand{\dcal}{\mathcal{D}}
\newcommand{\ecal}{\mathcal{E}}
\newcommand{\fcal}{\mathcal{F}}
\newcommand{\gcal}{\mathcal{G}}
\newcommand{\hcal}{\mathcal{H}}
\newcommand{\ical}{\mathcal{I}}
\newcommand{\jcal}{\mathcal{J}}
\newcommand{\kcal}{\mathcal{K}}
\newcommand{\lcal}{\mathcal{L}}
\newcommand{\mcal}{\mathcal{M}}
\newcommand{\ncal}{\mathcal{N}}
\newcommand{\ocal}{\mathcal{O}}
\newcommand{\pcal}{\mathcal{P}}
\newcommand{\qcal}{\mathcal{Q}}
\newcommand{\rcal}{\mathcal{R}}
\newcommand{\scal}{\mathcal{S}}
\newcommand{\tcal}{\mathcal{T}}
\newcommand{\ucal}{\mathcal{U}}
\newcommand{\vcal}{\mathcal{V}}
\newcommand{\wcal}{\mathcal{W}}
\newcommand{\xcal}{\mathcal{X}}
\newcommand{\ycal}{\mathcal{Y}}
\newcommand{\zcal}{\mathcal{Z}}

% Mathfrak
\newcommand{\afrak}{\mathfrak{A}}
\newcommand{\bfrak}{\mathfrak{B}}
\newcommand{\cfrak}{\mathfrak{C}}
\newcommand{\dfrak}{\mathfrak{D}}
\newcommand{\efrak}{\mathfrak{E}}
\newcommand{\ffrak}{\mathfrak{F}}
\newcommand{\gfrak}{\mathfrak{G}}
\newcommand{\hfrak}{\mathfrak{H}}
\newcommand{\ifrak}{\mathfrak{I}}
\newcommand{\jfrak}{\mathfrak{J}}
\newcommand{\kfrak}{\mathfrak{K}}
\newcommand{\lfrak}{\mathfrak{L}}
\newcommand{\mfrak}{\mathfrak{M}}
\newcommand{\nfrak}{\mathfrak{N}}
\newcommand{\ofrak}{\mathfrak{O}}
\newcommand{\pfrak}{\mathfrak{P}}
\newcommand{\qfrak}{\mathfrak{Q}}
\newcommand{\rfrak}{\mathfrak{R}}
\newcommand{\sfrak}{\mathfrak{S}}
\newcommand{\tfrak}{\mathfrak{T}}
\newcommand{\ufrak}{\mathfrak{U}}
\newcommand{\vfrak}{\mathfrak{V}}
\newcommand{\wfrak}{\mathfrak{W}}
\newcommand{\xfrak}{\mathfrak{X}}
\newcommand{\yfrak}{\mathfrak{Y}}
\newcommand{\zfrak}{\mathfrak{Z}}

% Mathscr
\newcommand{\ascr}{\mathscr{A}}
\newcommand{\bscr}{\mathscr{B}}
\newcommand{\cscr}{\mathscr{C}}
\newcommand{\dscr}{\mathscr{D}}
\newcommand{\escr}{\mathscr{E}}
\newcommand{\fscr}{\mathscr{F}}
\newcommand{\gscr}{\mathscr{G}}
\newcommand{\hscr}{\mathscr{H}}
\newcommand{\iscr}{\mathscr{I}}
\newcommand{\jscr}{\mathscr{J}}
\newcommand{\kscr}{\mathscr{K}}
\newcommand{\lscr}{\mathscr{L}}
\newcommand{\mscr}{\mathscr{M}}
\newcommand{\nscr}{\mathscr{N}}
\newcommand{\oscr}{\mathscr{O}}
\newcommand{\pscr}{\mathscr{P}}
\newcommand{\qscr}{\mathscr{Q}}
\newcommand{\rscr}{\mathscr{R}}
\newcommand{\sscr}{\mathscr{S}}
\newcommand{\tscr}{\mathscr{T}}
\newcommand{\uscr}{\mathscr{U}}
\newcommand{\vscr}{\mathscr{V}}
\newcommand{\wscr}{\mathscr{W}}
\newcommand{\xscr}{\mathscr{X}}
\newcommand{\yscr}{\mathscr{Y}}
\newcommand{\zscr}{\mathscr{Z}}

% Mathbb
\newcommand{\abb}{\mathbb{A}}
\newcommand{\bbb}{\mathbb{B}}
\newcommand{\cbb}{\mathbb{C}}
\newcommand{\dbb}{\mathbb{D}}
\newcommand{\ebb}{\mathbb{E}}
\newcommand{\fbb}{\mathbb{F}}
\newcommand{\gbb}{\mathbb{G}}
\newcommand{\hbb}{\mathbb{H}}
\newcommand{\ibb}{\mathbb{I}}
\newcommand{\jbb}{\mathbb{J}}
\newcommand{\kbb}{\mathbb{K}}
\newcommand{\lbb}{\mathbb{L}}
\newcommand{\mbb}{\mathbb{M}}
\newcommand{\nbb}{\mathbb{N}}
\newcommand{\obb}{\mathbb{O}}
\newcommand{\pbb}{\mathbb{P}}
\newcommand{\qbb}{\mathbb{Q}}
\newcommand{\rbb}{\mathbb{R}}
\newcommand{\sbb}{\mathbb{S}}
\newcommand{\tbb}{\mathbb{T}}
\newcommand{\ubb}{\mathbb{U}}
\newcommand{\vbb}{\mathbb{V}}
\newcommand{\wbb}{\mathbb{W}}
\newcommand{\xbb}{\mathbb{X}}
\newcommand{\ybb}{\mathbb{Y}}
\newcommand{\zbb}{\mathbb{Z}}


%===============================================================================
%=============================== SPECIAL SYMBOLS ===============================
%===============================================================================


% Orbit (group theory)
\newcommand{\orbit}{\mathcal{O}}
% Normal group
\newcommand{\normal}{\mathcal{N}}
% Indicator function
\newcommand{\indicator}{\mathbbm{1}}
% Laplace transform
\newcommand{\laplace}[1]{\mathcal{L}}
% Epsilon shorthand
\newcommand{\eps}{\varepsilon}
% Omega
\newcommand{\om}{\omega}
\newcommand{\Om}{\Omega}

%===============================================================================
%================================== OPERATORS ==================================
%===============================================================================


% Inverse exponent
\newcommand{\inv}[0]{^{-1}}
% Overline bar
\newcommand{\olsi}[1]{\,\overline{\!{#1}}}
% Less than or equal slanted
\newcommand{\seqs}{\leqslant}
% Greater or equal slanted
\newcommand{\geqs}{\geqslant}
% Subset or equal
\newcommand{\sub}{\subseteq}
% Proper subset
\newcommand{\prosub}{\subset}
% from
\newcommand{\from}{\leftarrow}

% Parantheses
\newcommand{\para}[1]{\left( #1 \right)}
% Curly Braces
\newcommand{\curl}[1]{\left\{ #1 \right\}}
% Brackets
\newcommand{\brac}[1]{\left[ #1 \right]}
% Angled Brackets
\newcommand{\ang}[1]{\left\langle #1 \right\rangle}
% Norm
\newcommand{\norm}[1]{\left\| #1 \right\|}

% Piece wise (use \\ between cases)
\newcommand{\piecewise}[1]{\begin{cases} #1 \end{cases}}

% Bold symbol shorthand
\newcommand{\bl}{\boldsymbol}

% Vertical space
\newcommand{\vs}[1]{\vspace{#1 pt}}
% Horizontal ertical space
\newcommand{\hs}[1]{\hspace{#1 pt}}


%===============================================================================
%============================== TEXT BASED SYMBOLS =============================
%===============================================================================


% Radians
\newcommand{\rad}{\text{rad}}
% Least Common Multiple
\newcommand{\lcm}{\text{lcm}}
% Automorphism
\newcommand{\Aut}{\text{Aut}}
% Variance
\newcommand{\var}{\text{Var}}
% Covariance
\newcommand{\cov}{\text{Cov}}
% Cofactor (matrix)
\newcommand{\cof}{\text{Cof}}
% Adjugate (matrix)
\newcommand{\adj}{\text{Adj}}
% Trace (matrix)
\newcommand{\tr}{\text{tr}}
% Standard deviation
\newcommand{\std}{\text{Std}}
% Correlation coefficient
\newcommand{\corr}{\text{Corr}}
% Sign
\newcommand{\sign}{\text{sign}}

% And text
\newcommand{\AND}{\text{ and }}
% Or text 
\newcommand{\OR}{\text{ or }}
% For text 
\newcommand{\FOR}{\text{ for }}
% If text
\newcommand{\IF}{\text{ if }}
% When text
\newcommand{\WHEN}{\text{ when }}
% Where text
\newcommand{\WHERE}{\text{ where }}
% Then text
\newcommand{\THEN}{\text{ then }}
% Such that text
\newcommand{\SUCHTHAT}{\text{ such that }}

% 1st
\newcommand{\st}[1]{#1^{\text{st}}}
% 2nd
\newcommand{\nd}[1]{#1^{\text{nd}}}
% 3rd
\newcommand{\rd}[1]{#1^{\text{rd}}}
% nth
\newcommand{\nth}[1]{#1^{\text{th}}}


%===============================================================================
%========================= PROBABILITY AND STATISTICS ==========================
%===============================================================================


% Permutation
\newcommand{\perm}[2]{{}^{#1}\!P_{#2}}
% Combination
\newcommand{\comb}[2]{{}^{#1}C_{#2}}

% Baye's risk
\newcommand{\risk}[1]{\mathscr{R}_{#1}}
% Baye's optimal risk
\newcommand{\riskOptimal}[1]{\mathscr{R}_{#1}^*}
% Baye's empirical risk
\newcommand{\riskEmpirical}[2]{\hat{\mathscr{R}}_{#1}^{#2}}


%===============================================================================
%=================================== CALCULUS ==================================
%===============================================================================


% d over d derivative
\newcommand{\dd}[2]{\dfrac{d#1}{d#2}}
% partial d over d derivative
\newcommand{\partialdd}[2]{\dfrac{\partial #1}{\partial #2}}
% delta d over d derivative
\newcommand{\deltadd}[2]{\dfrac{\Delta #1}{\Delta #2}}

% Integration between a and b
\newcommand{\integral}[4]{\int_{#1}^{#2} #3 \, #4}
% Integration in some space 
\newcommand{\boundIntegral}[2]{\int_{#1} #2 \, d#1}

% Limit
\newcommand{\limit}[3]{\lim_{#1 \to #2} #3}


%===============================================================================
%================================  BIG SYMBOLS  ================================
%===============================================================================


% Sum
\newcommand{\sumof}[2]{\sum_{#1}^{#2}}
% Product
\newcommand{\productof}[2]{\prod_{#1}^{#2}}
% Union
\newcommand{\unionof}[2]{\bigcup_{#1}^{#2}}
% Intersection
\newcommand{\intersectionof}[2]{\bigcap_{#1}^{#2}}
% Or
\newcommand{\orof}[2]{\bigvee_{#1}^{#2}}
% And
\newcommand{\andof}[2]{\bigwedge_{#1}^{#2}}


%===============================================================================
%=============================== LINEAR ALGEBRA ================================
%===============================================================================


% Bold vector arrow
\newcommand{\bv}[1]{\vec{\mathbf{#1}}}

% Matrix or vector (use // for column, & for row) 
% with brackets
\newcommand{\bmat}[1]{\begin{bmatrix} #1 \end{bmatrix}}
% Matrix or vector (use // for column, & for row) 
% with curved brackets
\newcommand{\pmat}[1]{\begin{pmatrix} #1 \end{pmatrix}}
% Matrix or vector (use // for column, & for row) 
% with lines on either side 
\newcommand{\lmat}[1]{\begin{vmatrix} #1 \end{vmatrix}}
% Matrix or vector (use // for column, & for row) 
% with curly braces
\newcommand{\cmat}[1]{\begin{Bmatrix} #1 \end{Bmatrix}}
% Matrix or vector (use // for column, & for row) 
% with no braces
\newcommand{\mat}[1]{\begin{matrix} #1 \end{matrix}}


%===============================================================================
%================================ LARGE OBJECTS ================================
%===============================================================================

% Multiple lines
\newcommand{\multiline}[1]{
\begin{align*}
    #1
\end{align*}
}

% Multiple lines with equation numbers
\newcommand{\eqmultiline}[1]{
\begin{align*}
    #1
\end{align*}
}

% Color
\newcommand{\colorText}[2]{
\begingroup
\color{#1}
    #2
\endgroup
}

% Centered figure
\newcommand{\centerFigure}[2]{
    \begin{figure}[h]
        \centering
            #1
        \caption{#2}
    \end{figure}
}

% Tikz figure
\newcommand{\tikzGraphic}[1]{
    \begin{center}
    \begin{tikzpicture}
        #1
    \end{tikzpicture}
    \end{center}
}

% Enumerate numbers (seperate by \item)
\newcommand{\numbers}{\textbf{\number*)}}

% Enumerate letters (seperate by \item)
\newcommand{\letters}{\textbf{\alph*)}}

\title{
    \Huge Machine Learning \\
    \Large Assignment VIII
}
\date{2025-06-07}
\author{Michael Saba}

\begin{document}
\pagenumbering{gobble}
\maketitle
\newpage
\setlength{\parindent}{0pt}
\pagenumbering{arabic}

\subsection*{Ex 1}

\begin{enumerate}[label = \letters]
\item
When plotting an ROC curve, we essentially
use different thresholds that determine when
we should choose the positive or negative class
(not just when the probability is larger or equal
to 0.5). \\
We will consider the thresholds:
\[ 0.3 \quad 0.4 \quad 0.5 \quad 0.6 \quad 0.7
\quad 0.8 \quad 0.9 \]
Where if the probability
of the class being positive is larger or equal to these
thresholds, we pick the positive class. \\
Now for the ROC curve, we plot the true positive
rate against the false positive rate:
\[ tpr = \dfrac{tp}{tp + fn} = \dfrac{tp}{P} \]
\[ fpr = \dfrac{fp}{fp + tn} = \dfrac{fp}{N} \]
Where $P$ are the actually positive patients,
and $N$ are the actually negative ones.
We have $P = 3$ and $N = 3$.
\begin{enumerate}[label = \numbers]
    \item 
    So for $0.9$, only patient A is predicted
    to have pneumonia, while the rest are predicted not to.
    So we have:
    \[ tp = 1 \quad fp = 0 \implies 
    tpr = \dfrac{1}{3} \quad 
    fpr = \dfrac{0}{3} = 0 \]
    \item For $0.8$:
    \[ tp = 1 \quad fp = 1 \implies 
    tpr = \dfrac{1}{3} \quad 
    fpr = \dfrac{1}{3} \]
    \item For $0.7$:
    \[ tp = 2 \quad fp = 1 \implies 
    tpr = \dfrac{2}{3} \quad 
    fpr = \dfrac{1}{3} \]
    \item For $0.6$:
    \[ tp = 2 \quad fp = 2 \implies 
    tpr = \dfrac{2}{3} \quad 
    fpr = \dfrac{2}{3} \]
    \item For $0.5$:
    \[ tp = 2 \quad fp = 2 \implies 
    tpr = \dfrac{2}{3} \quad 
    fpr = \dfrac{2}{3} \]
    \item For $0.4$:
    \[ tp = 3 \quad fp = 2 \implies 
    tpr = \dfrac{3}{3} = 1 \quad 
    fpr = \dfrac{2}{3} \]
    \item For $0.3$:
    \[ tp = 3 \quad fp = 3 \implies 
    tpr = \dfrac{3}{3} = 1 \quad 
    fpr = \dfrac{3}{3} = 1 \]
\end{enumerate}
And we are done. \\
\item 
Now we just sketch it with the $x$-axis
representing the $tpr$ and the $y$-axis
representing the $fpr$.
It will look like a stair case.
\item 
We can connect the points and calculate the area
under the curve. We end up with several
rectangles and trapezoids, with a total area 
of $\dfrac{15}{18}$.
\item 
Notice that in the $tpr$ and the $fpr$,
the denominator never changes when the thresholds
change, since $tp + fn$ and $fp + tn$
are just the number of actual positive and negative
cases. \\
So we just focus on the numerator:
when the number $tp$ increases,
it's only because we've made new predictions
that are positive; this either doesn't
change the $fp$, or also makes it increase,
since both are calculating the number of
"positives". \\  
\end{enumerate}

\colorText{red}{
    \subsection*{Correction}
    \begin{enumerate}[label = \letters]
        \item 
        \item 
        \item 
        the area should be $\dfrac{2}{3}$,
        draw the ROC again with TPR as the $y$-axis.
        \item
    \end{enumerate}
}

\newpage

\subsection*{Ex 2}
We have:
\[ f(x) = \int_{\omega \in \Omega} \|\omega - x\|_2^2 
\, p(\omega) \, d\omega \]
Where $\Omega \in \rbb^d$, 
and $p$ is a density function such that:
\[\int_{\omega \in \Omega} p(\omega) \, d\omega = 1\]
\begin{enumerate}[label = \letters]
    \item 
    Notice that:
    \[ f(x) = \int_{\omega \in \Omega} \|\omega - x\|_2^2 
    \, p(\omega) \, d\omega \]
    is just the expected value:
    \[ f(x) = \ebb[\|\Omega - x\|_2^2] \]
    To show that a solution exists for:
    \[ \text{inf}_{x\in \rbb^d} f(x) \]
    We need to show that a global minimizer exists,
    that is, that the function is coercive:
    \[ \limit{\|x\|}{\infty}{f(x)} = \infty \]
    This is easy to see. Since $f(x)$
    is the expected value of $ \ebb[\|\Omega - x\|_2^2]$,
    as $\|x\|$ tends towards towards infinity,
    does does $\|\Omega - x\|_2^2$,
    and since the expected value is just the weighted
    average of the function $\|\Omega - x\|_2^2$
    (where the weight $p(\omega)$ is positive),
    it too tends towards infinity. \\
    So the function is coercive.
    \item 
    We want to show that the solution is unique
    by proving that the function is convex. \\
    First we note that $\|\omega - x\|_2^2$
    is convex, since it is quadratic. \\
    Next we note that:
    \[ f(x) = \ebb[\|\Omega - x\|_2^2] \]
    The expected value is just the weighted average
    of the function, that is, each term 
    $\|\omega - x\|_2^2$ is scaled by a positive
    weight $p(\omega)$ and summed together with the other
    terms. \\
    Since positive scaling and summation
    of convex functions yields convex functions,
    the expected value of a convex function must
    be convex.
    \item 
    Now to find this unique solution,
    we can just differentiate
    with respect to $x$ and set the gradient
    equal to $0$, since the function is differentiable:
    \[ \nabla f = \nabla \para{
    \int_{\omega \in \Omega} \|\omega - x\|_2^2 
    \, p(\omega) \, d\omega} \]
    \[ \nabla f = \int_{\omega \in \Omega} 
    \nabla \|\omega - x\|_2^2 \, p(\omega) \, d\omega \]
    \[ \nabla f = \int_{\omega \in \Omega} 
    2(x - \omega) \, p(\omega) \, d\omega \]
    \[ \nabla f = \ebb[2(x - \Omega)] 
    = 2x - 2\ebb[\Omega] \]
    Setting the gradient equal to $0$, we get:
    \[ \nabla f = 0 \]
    \[ 2x - 2\ebb[\Omega] = 0 \]
    \[ x = \ebb[\Omega] \]
    And the mean of $\Omega$ is our minimizer. \\
\end{enumerate}

\colorText{red}{
    \subsection*{Correction}
    \begin{enumerate}[label = \letters]
        \item 
        We could also mention that we have a bounded
        set. But proving it's coercive
        is correct. \\
        It's also valid to just give the solution
        (which is also part c).
        \item 
        Instead of an expected value,
        we can also expand the term $\|\omega - x\|^2$,
        and show that each of the 3 resulting
        terms is convex:
        \[ \|x\|^2 - 2\ang{\int_\Omega 
        \omega p(\omega)d\omega, x}
        + \int_\Omega \|\omega\|^2p(\omega)d\omega\]
        The first term is a quadratic, the second
        is linear, so both are convex
        (all linear functions, going down or up,
        are convex), and the third
        term is just a constant. So the sum
        of the terms is a convex function.
        \item 
    \end{enumerate}
}


\newpage

\subsection*{Ex 3}
We have $x \in \rbb^d$ and:
\[ f_\infty(x) = \max(x_1, x_2, \dots, x_d) \]
\[ f_\lambda(x) = \dfrac{1}{\lambda}
\log\para{\sum_{i}^d e^{\lambda x_i}} \]
Where $\lambda \in \rbb$ and $\lambda > 0$. 
\begin{enumerate}[label = \letters]
    \item 
    Constant valued functions like:
    \[ x_1, \quad x_2, \quad \dots, \quad x_d \]
    Are all convex, 
    which means that their maximum is also
    convex, so $f_\infty$ must be convex.
    \item 
    We have:
    \[ f_\lambda = \dfrac{1}{\lambda}
    \log\para{\sum_{i}^d e^{\lambda x_i}} \]
    We want to show that:
    \[ f_\infty \leq f_\lambda \leq  f_\infty 
    + \dfrac{1}{\lambda}\log(d) \]
    First we can show that:
    \[ f_\infty \leq  f_\lambda \]
    We can do that by starting with the
    inequality:
    \[ \max(x_1, x_2, \dots, x_d)
    \leq \dfrac{1}{\lambda}
    \log\para{\sum_{i}^d e^{\lambda x_i}} \]
    And deriving a an inequality we know holds:
    \[ \lambda \max(x_1, x_2, \dots, x_d)
    \leq \log\para{\sum_{i}^d e^{\lambda x_i}} \]
    \[ e^{\lambda \max(x_1, x_2, \dots, x_d)}
    \leq \sum_{i}^d e^{\lambda x_i} \]
    Notice that the left hand side
    is the exponential of $\lambda$
    times the largest $x_i$,
    while the right hand side is the sum
    of the exponential of all terms
    of the form $\lambda x_i$,
    including the max, so the right hand
    side is at least larger or equal. \\
    On the other hand, we can show that
    this inequality holds using the same method:
    \[ f_\lambda \leq  f_\infty 
    + \dfrac{1}{\lambda}\log(d) \]
    \[ \dfrac{1}{\lambda}
    \log\para{\sum_{i}^d e^{\lambda x_i}} 
    \leq \max(x_1, x_2, \dots, x_d)
    + \dfrac{1}{\lambda}\log(d) \]
    \[ \log\para{\sum_{i}^d e^{\lambda x_i}} 
    \leq \lambda \max(x_1, x_2, \dots, x_d) + \log(d) \]
    \[ \sum_{i}^d e^{\lambda x_i}
    \leq e^{\lambda \max(x_1, x_2, \dots, x_d) + \log(d)} \]
    \[ \sum_{i}^d e^{\lambda x_i}
    \leq e^{\lambda \max(x_1, x_2, \dots, x_d)}
    e^{\log(d)} \]
    \[ \sum_{i}^d e^{\lambda x_i}
    \leq d \cdot e^{\lambda\max(x_1, x_2, \dots, x_d)} \]
    \[ \sum_{i}^d  e^{\lambda x_i}
    \leq \sum_{i}^d e^{\lambda\max(x_1, x_2, \dots, x_d)} \]
    The function $e^{\lambda x}$
    is monotonically increasing since $\lambda > 0$,
    which means that the right hand side
    is the sum of terms that are always larger or
    equal to the terms being summed on the left hand side.
    So the right hand side must be larger or equal.
    \item
\end{enumerate}

\colorText{red}{
    \subsection*{Correction}
    \begin{enumerate}[label = \letters]
        \item 
        We can't do what I did,
        since $x_1, x_2, \dots$
        are not constants.  \\
        We must use Jensen's inequality,
        and note that the maximum of a sum
        is smaller than the sum of the maximum. \\
        \item 
        \item 
        We just do the second derivative
        test, and prove the hessian $\nabla^2$
        is positive-semi definite:
        \[ \partial_{x_i} f_\lambda(x)
        = \dfrac{e^{\lambda x_i}}
        {\sum_k^d e^{\lambda x_k}}
        = \dfrac{e^{\lambda x_i}}{z} \]
        We can take $z = \sum_k^d e^{\lambda x_k}$. \\
        Each element of the gradient we will call 
        $p_i$. \\
        Then then second derivative is a matrix
        with diagonal terms:
        \[ \partial_{x_ix_i} f_\lambda(x)
        = \dfrac{z\lambda e^{\lambda x_i} - \lambda 
        e^{2\lambda x_i}}{z^2}
        = \lambda p_i(1-p_i) \]
        And off-diagonal terms:
        \[ \partial_{x_ix_j} f_\lambda(x)
        =  \dfrac{-\lambda e^{2\lambda x_i}
        e^{2\lambda x_j}}{z^2}
        = -\lambda p_i p_j \]
        Then the matrix can be written as:
        \[ (\nabla^2 f_\lambda)_{ij}
        =\piecewise{\lambda p_i(1-p_i) \quad &\IF
        i = j \\ -\lambda p_i p_j \quad 
        &\text{else}} \]
        Then we will find that:
        \[ \nabla^2 f_\lambda = \lambda (\text{diag}(p)
        -pp^T) \]
        Where:
        \[ w^t\nabla^2f_\lambda w
        = \lambda\brac{(p_1w_1^2+ p_2w_2^2 +
        \dots p_dw_d^2) - (p_1w_1 + p_2w_2 +
        \dots p_dw_d)^2} \]
        Which is clearly positive ($p$ is always positive),
        which makes $\nabla^2 f_\lambda$
        positive-semi definite.
    \end{enumerate}
}

\newpage

\subsection*{Ex 4}
The proximal map is given by:
\[ \text{prox}_{\tau f}(\bar{\theta})
= \arg\min_{\theta} \left[ f(\theta) 
+ \frac{1}{2\tau} \|\theta - \bar{\theta}\|^2 \right] \]
\begin{enumerate}[label = \letters]
    \item For $f(\theta) = \|\theta\|^2_2$:
    \[ \left[ \text{prox}_{\tau \|\cdot\|_2^2}
    (\bar{\theta}) \right]_i = 
    \frac{\bar{\theta}_i}{1 + 2\tau}, 
    \quad \forall i = 1, \ldots, d \]
    We can prove that by solving:
    \[ \arg\min_{\theta} \left[ \|\theta\|^2 + 
    \frac{1}{2\tau} \|\theta - \bar{\theta}\|^2 \right] \]
    Which we can do by differentiating and setting
    the gradient to $0$ since the function
    \[ \|\theta\|^2 + 
    \frac{1}{2\tau} \|\theta - \bar{\theta}\|^2 \]
    is convex (it is the sum of two quadratics).
    \[ \nabla \left[ \|\theta\|^2 + 
    \frac{1}{2\tau} \|\theta - \bar{\theta}\|^2 \right] \]
    \[ 2\theta + 
    \frac{1}{\tau}(\theta - \bar{\theta}) \]
    Setting it equal to $0$:
    \[ 2\theta + \frac{1}{\tau}(\theta - \bar{\theta})
    = 0 \]
    \[ (2 + \dfrac{1}{\tau})\theta
    = \frac{\bar{\theta}}{\tau} \]
    \[ \theta = \frac{\bar{\theta}}
    {\tau(2 + \dfrac{1}{\tau})} \]
    \[ \theta = \frac{\bar{\theta}}{2\tau + 1} \]
\item For $f(\theta) = 
    \delta_{\|\theta \|_2 \leq 1}(\theta)$,
    where the convex set is the unit disk,
    we want to show that the proximal mapping is:
    \[ \text{prox}_{\tau \delta_{\|\cdot\|_2 \leq 1}}
    (\bar{\theta}) = \frac{\bar{\theta}}
    {\max(1, \|\bar{\theta}\|)}\] 
    We want to solve:
    \[ \arg\min_{\theta} \left[ 
    \delta_{\|\cdot\|_2 \leq 0} (\theta) + 
    \frac{1}{2\tau} \|\theta - \bar{\theta}\|^2 \right] \]
    Recall that:
    \[ \delta_C(\theta)
    = \piecewise{0, \qquad \;\; \IF  \theta \in C \\
    \infty, \qquad \IF \theta \notin C } \]
    So in order to minimize the function,
    the indicator term can't be infinity,
    which means that $\|\theta\|_2 \leq 1$ must hold,
    meaning that $\theta$ is in the unit circle. \\
    So if we assume that $\|\theta\|_2 \leq 1$,
    the indicator term becomes $0$,
    and we now want to solve:
    \[ \arg\min_{\theta \in S} 
    \frac{1}{2\tau} \|\theta - \bar{\theta}\|^2\]
    Where $S$ is the unit disk. \\
    So we want to minimize the distance between
    $\theta$ and $\bar{\theta}$
    such that $\theta$ is in the unit circle. \\
    Now, if $\bar{\theta}$
    is also in the unit circle,
    we can just set $\theta = \bar{\theta}$. \\
    If it is not, the closest point in
    the unit circle to $\bar{\theta}$
    will be the projction of $\bar{\theta}$
    on the unit cirlce, that is:
    \[ \dfrac{\bar{\theta}}{\|\bar{\theta}\|_2} \]
    which is just $\bar{\theta}$ normalized,
    the point in the direction of $\bar{\theta}$
    on the unit circle. \\
    So:
    \[ \theta = \piecewise{
        \bar{\theta} \qquad \qquad 
        \IF \|\bar{\theta}\|_2 \leq 1
        \\  \dfrac{\bar{\theta}}{\|\bar{\theta}\|_2} 
        \qquad \; \IF \|\bar{\theta}\|_2 > 1
    } \]
    Which is precisely:
    \[ \theta = \frac{\bar{\theta}}
    {\max(1, \|\bar{\theta}\|_2)} \]
\item For $f(\theta) = \|\theta\|_1$,
    also called the soft shrinkage-thresholding 
    operator:
    \[ \left[ \text{prox}_{\tau \|\cdot\|_1}
    (\bar{\theta}) \right]_i = 
    \max(0, |\bar{\theta}_i| - \tau) \cdot 
    \text{sign}(\bar{\theta}_i) 
    \quad \forall i = 1, \ldots, d \]
    We can prove that by solving:
    \[ \arg\min_{\theta} \left[ \|\theta\|_1 + 
    \frac{1}{2\tau} \|\theta - \bar{\theta}\|^2 \right] \]
    Which we can do by differentiating and setting
    the gradient to $0$ since the function
    is convex:
    \[ \nabla \left[ \|\theta\|_1 + 
    \frac{1}{2\tau} \|\theta - \bar{\theta}\|^2 \right] \]
    \[ \nabla \left[ \sum_{i=1}^d |\theta_i| + 
    \frac{1}{2\tau} \sum_{i=1}^d (\theta_i
    - \bar{\theta}_i)^2 \right] \]
    We can differentiate each variable individually:
    \[ \partial_{\theta_i}
    \left[ \sum_{i=1}^d |\theta_i| + 
    \frac{1}{2\tau} \sum_{i=1}^d (\theta_i
    - \bar{\theta}_i)^2 \right] \]
    \[ \partial_{\theta_i} |\theta_i| + 
    \frac{1}{\tau}(\theta_i - \bar{\theta}_i) \]
    Setting it equal to $0$, we have:
    \[ \partial_{\theta_i} |\theta_i| + 
    \frac{1}{\tau}(\theta_i - \bar{\theta}_i) = 0 \]
    \[ \theta_i = \bar{\theta}_i 
    - \tau\partial_{\theta_i} |\theta_i| \]
    We now consider $3$ seperate cases: 
    \begin{enumerate}[label = \numbers]
        \item 
        If $\theta_i > 0$,
        we have $\partial_{\theta_i} |\theta_i| = 1$, so:
        \[ \theta_i = \bar{\theta}_i - \tau \]
        Which is only valid if $\theta_i < 0$,
        meaning that $\bar{\theta}_i > \tau$.
        \item
        And for $\theta_i < 0$,
        we have $\partial_{\theta_i} |\theta_i| = -1$, so:
        \[ \theta_i = \bar{\theta}_i + \tau \]
        Which is only valid if $\theta_i < 0$,
        meaning that $\bar{\theta}_i < -\tau$.
        \item
        And finally, for $\theta_i = 0$,
        we have $\partial_{\theta_i} |\theta_i|
        \in [-1, 1]$ (between the slopes of both sides).
        So instead of writing
        \[ \theta_i = \bar{\theta}_i 
        - \tau\partial_{\theta_i} |\theta_i| \]
        We write
        \[ \theta_i \in \bar{\theta}_i 
        - \tau\partial_{\theta_i} |\theta_i| \]
        \[ \theta_i \in [\bar{\theta}_i - \tau, 
        \bar{\theta}_i + \tau] \]
        And since $\theta_i = 0$,
        in order for $0$ to be in the range 
        $[\bar{\theta}_i - \tau, 
        \bar{\theta}_i + \tau]$, we need to have
        $|\bar{\theta_i}| \leq \tau$.
        \end{enumerate}
        We can summarize these 3 cases as:
        \[ \theta_i = \piecewise{
            \bar{\theta}_i - \tau \quad 
            \IF \bar{\theta}_i > \tau \\
            \bar{\theta}_i + \tau \quad 
            \IF \bar{\theta}_i < -\tau \\
            0 \quad \quad \quad \;\, \text{else
            (if $|\bar{\theta_i}| \leq \tau$)}
        } \]
        Which is just $\max(0, |\bar{\theta_i}| - \tau)
        \cdot \sign(\bar{\theta_i})$.
    \item
    For $f(\theta) = \|\theta\|_{J, 2, 1}$,
    the $\ell_{2, 1}$ norm:
    \[ \mathrm{prox}_{\tau \|\cdot\|_{J,2,1}}(\bar{\theta}) 
    = \piecewise{\para{1 - \frac{\tau}{\|\bar{\theta}_J\|_2}} 
    \bar{\theta}_J & \text{if } \|\bar{\theta}_J\|_2 > 
    \tau \\ 0 & \text{otherwise}} \]
    We can show that this is the case.
    First note that the function is differentiable:
    \[\arg\min_{\theta} \brac{ \frac{1}{2\tau} \|\theta - 
    \bar{\theta}\|_2^2 + \sum_{J \in \jcal} 
    \|\theta_J\|_2} \]
    So since we care about some subset $J$
    only, we will minimize it in terms of $\theta_J$:
    \[ \partialdd{}{\theta_J} 
    \brac{ \frac{1}{2\tau} \|\theta - 
    \bar{\theta}\|_2^2 + \sum_{J \in \jcal} 
    \|\theta_J\|_2} \]
    \[ \frac{1}{\tau}(\theta_J - 
    \bar{\theta}_J) + \dfrac{\theta_J}{\|\theta_J\|_2} \]
    Setting it equal to $0$, we get:
    \[ \frac{1}{\tau}(\theta_J - \bar{\theta}_J) 
    + \dfrac{\theta_J}{\|\theta_J\|_2} = 0 \]
    \[ \para{1+\frac{\tau}{\|\theta_J\|_2}}
    \theta_J = \bar{\theta}_J \]
    We need to find the norm of $\theta$,
    so we can take the norm of both sides:
    \[ \para{1+\frac{\tau}{\|\theta_J\|_2}}
    \|\theta_J\|_2 = \|\bar{\theta}_J\|_2 \]
    \[ \|\theta_J\|_2 +\tau = \|\bar{\theta}_J\|_2 \]
    \[ \|\theta_J\|_2 = \|\bar{\theta}_J\|_2 - \tau \]
    Notice that since this is a norm,
    we have to assume that 
    $\|\bar{\theta}_J\|_2 - \tau \geq 0$:
    Replacing it in the original, we get:
    \[ \para{1+\frac{\tau}{\|\bar{\theta}_J\|_2 - \tau}}
    \theta_J = \bar{\theta}_J \]
    \[ \para{\frac{\|\bar{\theta}_J\|_2 - \tau +
    \tau}{\|\bar{\theta}_J\|_2 - \tau}}
    \theta_J = \bar{\theta}_J \]
    \[ \para{\frac{\|\bar{\theta}_J\|_2}
    {\|\bar{\theta}_J\|_2 - \tau}}
    \theta_J = \bar{\theta}_J \]
    \[\theta_J = \para{\frac{\|\bar{\theta}_J\|_2 - \tau}
    {\|\bar{\theta}_J\|_2}}\bar{\theta}_J \]
    \[\theta_J = \para{1 -\frac{\tau}
    {\|\bar{\theta}_J\|_2}}\bar{\theta}_J \]
    On the other hand, if $\|\bar{\theta}_J\|_2 - \tau < 0$,
    then $\theta_J$ becomes $0$; if we look at the
    function we are minimizing, it is always non-negative,
    so it makes no sense to make it negative;
    we make it 0 instead. \\
\end{enumerate}


\colorText{red}{
    \subsection*{Correction}
    \begin{enumerate}[label = \letters]
        \item 
        \item 
        \item 
        She did not mention the third case,
        and only did the two cases $\theta > 0$
        and $\theta < 0$, and otherwise
        we just have $\theta = 0$,
        no need to worry about constraints,
        since we already have the constraints
        of the two other cases, and the third case
        takes place otherwise. \\
        So we can just ignore the case where $\theta_i = 0$,
        since it is just $0$ on whatever interval
        is left after the first two cases. \\
        \item 
        She used a different way,
        setting $\theta_J = \alpha \bar{\theta}_J$,
        assuming they go in the same direction. \\
    \end{enumerate}
}


\newpage

\end{document}