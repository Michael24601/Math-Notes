
\documentclass[12pt]{article}

\usepackage[margin=1in]{geometry}


%===============================================================================
%================================== PACKAGES ===================================
%===============================================================================

% For using float option H that places figures 
% exatcly where we want them
\usepackage{float}
% makes figure font bold
\usepackage{caption}
\captionsetup[figure]{labelfont=bf}
% For text generation
\usepackage{lipsum}
% For drawing
\usepackage{tikz}
% For smaller or equal sign and not divide sign
\usepackage{amssymb}
% For the diagonal fraction
\usepackage{xfrac}
% For enumerating exercise parts with letters instead of numbers
\usepackage{enumitem}
% For dfrac, which forces the fraction to be in display mode (large) e
% even in math mode (small)
\usepackage{amsmath}
% For degree sign
\usepackage{gensymb}
% For "\mathbb" macro
\usepackage{amsfonts}
% For positioning 
\usepackage{indentfirst}
\usetikzlibrary{shapes,positioning,fit,calc}
% for adjustwidth environment
\usepackage{changepage}
% for arrow on top
\usepackage{esvect}
% for mathbb 1
\usepackage{bbm}
% for mathsrc
\usepackage[mathscr]{eucal}
% For degree sign
\usepackage{gensymb}
% For quotes
\usepackage{csquotes}
% For vertical lines
\usepackage{mathtools}
% For cols
\usepackage{multicol}

% for tikz
\usepackage{pgfplots}
\pgfplotsset{compat=1.18}
\usepackage{amsmath}
\usepgfplotslibrary{groupplots}


%===============================================================================
%==================================== FONTS ====================================
%===============================================================================


% Mathcal
\newcommand{\acal}{\mathcal{A}}
\newcommand{\bcal}{\mathcal{B}}
\newcommand{\ccal}{\mathcal{C}}
\newcommand{\dcal}{\mathcal{D}}
\newcommand{\ecal}{\mathcal{E}}
\newcommand{\fcal}{\mathcal{F}}
\newcommand{\gcal}{\mathcal{G}}
\newcommand{\hcal}{\mathcal{H}}
\newcommand{\ical}{\mathcal{I}}
\newcommand{\jcal}{\mathcal{J}}
\newcommand{\kcal}{\mathcal{K}}
\newcommand{\lcal}{\mathcal{L}}
\newcommand{\mcal}{\mathcal{M}}
\newcommand{\ncal}{\mathcal{N}}
\newcommand{\ocal}{\mathcal{O}}
\newcommand{\pcal}{\mathcal{P}}
\newcommand{\qcal}{\mathcal{Q}}
\newcommand{\rcal}{\mathcal{R}}
\newcommand{\scal}{\mathcal{S}}
\newcommand{\tcal}{\mathcal{T}}
\newcommand{\ucal}{\mathcal{U}}
\newcommand{\vcal}{\mathcal{V}}
\newcommand{\wcal}{\mathcal{W}}
\newcommand{\xcal}{\mathcal{X}}
\newcommand{\ycal}{\mathcal{Y}}
\newcommand{\zcal}{\mathcal{Z}}

% Mathfrak
\newcommand{\afrak}{\mathfrak{A}}
\newcommand{\bfrak}{\mathfrak{B}}
\newcommand{\cfrak}{\mathfrak{C}}
\newcommand{\dfrak}{\mathfrak{D}}
\newcommand{\efrak}{\mathfrak{E}}
\newcommand{\ffrak}{\mathfrak{F}}
\newcommand{\gfrak}{\mathfrak{G}}
\newcommand{\hfrak}{\mathfrak{H}}
\newcommand{\ifrak}{\mathfrak{I}}
\newcommand{\jfrak}{\mathfrak{J}}
\newcommand{\kfrak}{\mathfrak{K}}
\newcommand{\lfrak}{\mathfrak{L}}
\newcommand{\mfrak}{\mathfrak{M}}
\newcommand{\nfrak}{\mathfrak{N}}
\newcommand{\ofrak}{\mathfrak{O}}
\newcommand{\pfrak}{\mathfrak{P}}
\newcommand{\qfrak}{\mathfrak{Q}}
\newcommand{\rfrak}{\mathfrak{R}}
\newcommand{\sfrak}{\mathfrak{S}}
\newcommand{\tfrak}{\mathfrak{T}}
\newcommand{\ufrak}{\mathfrak{U}}
\newcommand{\vfrak}{\mathfrak{V}}
\newcommand{\wfrak}{\mathfrak{W}}
\newcommand{\xfrak}{\mathfrak{X}}
\newcommand{\yfrak}{\mathfrak{Y}}
\newcommand{\zfrak}{\mathfrak{Z}}

% Mathscr
\newcommand{\ascr}{\mathscr{A}}
\newcommand{\bscr}{\mathscr{B}}
\newcommand{\cscr}{\mathscr{C}}
\newcommand{\dscr}{\mathscr{D}}
\newcommand{\escr}{\mathscr{E}}
\newcommand{\fscr}{\mathscr{F}}
\newcommand{\gscr}{\mathscr{G}}
\newcommand{\hscr}{\mathscr{H}}
\newcommand{\iscr}{\mathscr{I}}
\newcommand{\jscr}{\mathscr{J}}
\newcommand{\kscr}{\mathscr{K}}
\newcommand{\lscr}{\mathscr{L}}
\newcommand{\mscr}{\mathscr{M}}
\newcommand{\nscr}{\mathscr{N}}
\newcommand{\oscr}{\mathscr{O}}
\newcommand{\pscr}{\mathscr{P}}
\newcommand{\qscr}{\mathscr{Q}}
\newcommand{\rscr}{\mathscr{R}}
\newcommand{\sscr}{\mathscr{S}}
\newcommand{\tscr}{\mathscr{T}}
\newcommand{\uscr}{\mathscr{U}}
\newcommand{\vscr}{\mathscr{V}}
\newcommand{\wscr}{\mathscr{W}}
\newcommand{\xscr}{\mathscr{X}}
\newcommand{\yscr}{\mathscr{Y}}
\newcommand{\zscr}{\mathscr{Z}}

% Mathbb
\newcommand{\abb}{\mathbb{A}}
\newcommand{\bbb}{\mathbb{B}}
\newcommand{\cbb}{\mathbb{C}}
\newcommand{\dbb}{\mathbb{D}}
\newcommand{\ebb}{\mathbb{E}}
\newcommand{\fbb}{\mathbb{F}}
\newcommand{\gbb}{\mathbb{G}}
\newcommand{\hbb}{\mathbb{H}}
\newcommand{\ibb}{\mathbb{I}}
\newcommand{\jbb}{\mathbb{J}}
\newcommand{\kbb}{\mathbb{K}}
\newcommand{\lbb}{\mathbb{L}}
\newcommand{\mbb}{\mathbb{M}}
\newcommand{\nbb}{\mathbb{N}}
\newcommand{\obb}{\mathbb{O}}
\newcommand{\pbb}{\mathbb{P}}
\newcommand{\qbb}{\mathbb{Q}}
\newcommand{\rbb}{\mathbb{R}}
\newcommand{\sbb}{\mathbb{S}}
\newcommand{\tbb}{\mathbb{T}}
\newcommand{\ubb}{\mathbb{U}}
\newcommand{\vbb}{\mathbb{V}}
\newcommand{\wbb}{\mathbb{W}}
\newcommand{\xbb}{\mathbb{X}}
\newcommand{\ybb}{\mathbb{Y}}
\newcommand{\zbb}{\mathbb{Z}}


%===============================================================================
%=============================== SPECIAL SYMBOLS ===============================
%===============================================================================


% Orbit (group theory)
\newcommand{\orbit}{\mathcal{O}}
% Normal group
\newcommand{\normal}{\mathcal{N}}
% Indicator function
\newcommand{\indicator}{\mathbbm{1}}
% Laplace transform
\newcommand{\laplace}[1]{\mathcal{L}}
% Epsilon shorthand
\newcommand{\eps}{\varepsilon}
% Omega
\newcommand{\om}{\omega}
\newcommand{\Om}{\Omega}

%===============================================================================
%================================== OPERATORS ==================================
%===============================================================================


% Inverse exponent
\newcommand{\inv}[0]{^{-1}}
% Overline bar
\newcommand{\olsi}[1]{\,\overline{\!{#1}}}
% Less than or equal slanted
\newcommand{\seqs}{\leqslant}
% Greater or equal slanted
\newcommand{\geqs}{\geqslant}
% Subset or equal
\newcommand{\sub}{\subseteq}
% Proper subset
\newcommand{\prosub}{\subset}
% from
\newcommand{\from}{\leftarrow}

% Parantheses
\newcommand{\para}[1]{\left( #1 \right)}
% Curly Braces
\newcommand{\curl}[1]{\left\{ #1 \right\}}
% Brackets
\newcommand{\brac}[1]{\left[ #1 \right]}
% Angled Brackets
\newcommand{\ang}[1]{\left\langle #1 \right\rangle}
% Norm
\newcommand{\norm}[1]{\left\| #1 \right\|}

% Piece wise (use \\ between cases)
\newcommand{\piecewise}[1]{\begin{cases} #1 \end{cases}}

% Bold symbol shorthand
\newcommand{\bl}{\boldsymbol}

% Vertical space
\newcommand{\vs}[1]{\vspace{#1 pt}}
% Horizontal ertical space
\newcommand{\hs}[1]{\hspace{#1 pt}}


%===============================================================================
%============================== TEXT BASED SYMBOLS =============================
%===============================================================================


% Radians
\newcommand{\rad}{\text{rad}}
% Least Common Multiple
\newcommand{\lcm}{\text{lcm}}
% Automorphism
\newcommand{\Aut}{\text{Aut}}
% Variance
\newcommand{\var}{\text{Var}}
% Covariance
\newcommand{\cov}{\text{Cov}}
% Cofactor (matrix)
\newcommand{\cof}{\text{Cof}}
% Adjugate (matrix)
\newcommand{\adj}{\text{Adj}}
% Trace (matrix)
\newcommand{\tr}{\text{tr}}
% Standard deviation
\newcommand{\std}{\text{Std}}
% Correlation coefficient
\newcommand{\corr}{\text{Corr}}
% Sign
\newcommand{\sign}{\text{sign}}

% And text
\newcommand{\AND}{\text{ and }}
% Or text 
\newcommand{\OR}{\text{ or }}
% For text 
\newcommand{\FOR}{\text{ for }}
% If text
\newcommand{\IF}{\text{ if }}
% When text
\newcommand{\WHEN}{\text{ when }}
% Where text
\newcommand{\WHERE}{\text{ where }}
% Then text
\newcommand{\THEN}{\text{ then }}
% Such that text
\newcommand{\SUCHTHAT}{\text{ such that }}

% 1st
\newcommand{\st}[1]{#1^{\text{st}}}
% 2nd
\newcommand{\nd}[1]{#1^{\text{nd}}}
% 3rd
\newcommand{\rd}[1]{#1^{\text{rd}}}
% nth
\newcommand{\nth}[1]{#1^{\text{th}}}


%===============================================================================
%========================= PROBABILITY AND STATISTICS ==========================
%===============================================================================


% Permutation
\newcommand{\perm}[2]{{}^{#1}\!P_{#2}}
% Combination
\newcommand{\comb}[2]{{}^{#1}C_{#2}}

% Baye's risk
\newcommand{\risk}[1]{\mathscr{R}_{#1}}
% Baye's optimal risk
\newcommand{\riskOptimal}[1]{\mathscr{R}_{#1}^*}
% Baye's empirical risk
\newcommand{\riskEmpirical}[2]{\hat{\mathscr{R}}_{#1}^{#2}}


%===============================================================================
%=================================== CALCULUS ==================================
%===============================================================================


% d over d derivative
\newcommand{\dd}[2]{\dfrac{d#1}{d#2}}
% partial d over d derivative
\newcommand{\partialdd}[2]{\dfrac{\partial #1}{\partial #2}}
% delta d over d derivative
\newcommand{\deltadd}[2]{\dfrac{\Delta #1}{\Delta #2}}

% Integration between a and b
\newcommand{\integral}[4]{\int_{#1}^{#2} #3 \, #4}
% Integration in some space 
\newcommand{\boundIntegral}[2]{\int_{#1} #2 \, d#1}

% Limit
\newcommand{\limit}[3]{\lim_{#1 \to #2} #3}


%===============================================================================
%================================  BIG SYMBOLS  ================================
%===============================================================================


% Sum
\newcommand{\sumof}[2]{\sum_{#1}^{#2}}
% Product
\newcommand{\productof}[2]{\prod_{#1}^{#2}}
% Union
\newcommand{\unionof}[2]{\bigcup_{#1}^{#2}}
% Intersection
\newcommand{\intersectionof}[2]{\bigcap_{#1}^{#2}}
% Or
\newcommand{\orof}[2]{\bigvee_{#1}^{#2}}
% And
\newcommand{\andof}[2]{\bigwedge_{#1}^{#2}}


%===============================================================================
%=============================== LINEAR ALGEBRA ================================
%===============================================================================


% Bold vector arrow
\newcommand{\bv}[1]{\vec{\mathbf{#1}}}

% Matrix or vector (use // for column, & for row) 
% with brackets
\newcommand{\bmat}[1]{\begin{bmatrix} #1 \end{bmatrix}}
% Matrix or vector (use // for column, & for row) 
% with curved brackets
\newcommand{\pmat}[1]{\begin{pmatrix} #1 \end{pmatrix}}
% Matrix or vector (use // for column, & for row) 
% with lines on either side 
\newcommand{\lmat}[1]{\begin{vmatrix} #1 \end{vmatrix}}
% Matrix or vector (use // for column, & for row) 
% with curly braces
\newcommand{\cmat}[1]{\begin{Bmatrix} #1 \end{Bmatrix}}
% Matrix or vector (use // for column, & for row) 
% with no braces
\newcommand{\mat}[1]{\begin{matrix} #1 \end{matrix}}


%===============================================================================
%================================ LARGE OBJECTS ================================
%===============================================================================

% Multiple lines
\newcommand{\multiline}[1]{
\begin{align*}
    #1
\end{align*}
}

% Multiple lines with equation numbers
\newcommand{\eqmultiline}[1]{
\begin{align*}
    #1
\end{align*}
}

% Color
\newcommand{\colorText}[2]{
\begingroup
\color{#1}
    #2
\endgroup
}

% Centered figure
\newcommand{\centerFigure}[2]{
    \begin{figure}[h]
        \centering
            #1
        \caption{#2}
    \end{figure}
}

% Tikz figure
\newcommand{\tikzGraphic}[1]{
    \begin{center}
    \begin{tikzpicture}
        #1
    \end{tikzpicture}
    \end{center}
}

% Enumerate numbers (seperate by \item)
\newcommand{\numbers}{\textbf{\number*)}}

% Enumerate letters (seperate by \item)
\newcommand{\letters}{\textbf{\alph*)}}

\title{%
    \Huge Machine Learning \\
    \Large Assignment II
}
\date{2025-04-22}
\author{Michael Saba}

\begin{document}
\pagenumbering{gobble}
\maketitle
\newpage
\setlength{\parindent}{0pt}
\pagenumbering{arabic}

\subsection*{Ex 1}

Given random vector $X, Y \in \R^d$ where
\[ Y = g(X) \]
then the transformation law tells us that
the density function of $Y$ can be calculated
using
\[ p_Y(y) = p_X(g\inv (y)) \cdot 
| \det(J_{g\inv}) | \]
where $p_X$ is the density function of $X$. \\
We need to find the desnity function $p_{X_1, X_2}$,
where
\[ X = g(U) \]
\[ \vecTwo{X_1}{X_2} = g\para{\vecTwo{U_1}{U_2}}
= \vecTwo
{\sqrt{-2\ln(U_2)}\sin(2\pi U_1) \vspace{10pt}}
{\sqrt{-2\ln(U_2)}\cos(2\pi U_1)}\]
Note that the joint desnity $p_{X_1, X_2}$
is the same as the density of the vector $p_X$.
Likewise, $p_{U_1, U_2} = p_U$. \\ 
To start, we know that $U_1$ and $U_2$
are uniform random variables over $[0, 1]$. \\
Given a random variable $D$ uniform over $[a, b]$,
we know that its density is given by:
\[ p_A(d) = \piecewise{\dfrac{1}{b-a},
\qquad \text{ for } d \in [a, b]}
{ 0, \qquad \qquad \text{ for } d \notin [a, b]} \]
This means that 
\[ p_{U_1}(u) = p_{U_2}(u) = 
\piecewise{1, \qquad \text{ for } d \in [0, 1]}
{ 0, \qquad \text{ for } d \notin [0, 1]} \]
Since the two random variables are independent,
their joint density is just the product
of their densities:
\[ p_U(u) = p_{U_1, U_2}(u_1, u_2) =
p_{U_1}(u_1) \cdot p_{U_2}(u_2) = 
\piecewise{1, \qquad \text{ for } u \in [0, 1]^2}
{ 0, \qquad \text{ for } u \notin [0, 1]^2} \]
Now, we need to find $g\inv$. \\
First, note that:
\[ X_1 = \sqrt{-2\ln(U_2)}\sin(2\pi U_1)
\qquad X_2 = \sqrt{-2\ln(U_2)}\cos(2\pi U_1) \]
\[ \dfrac{X_1}{X_2} = 
\dfrac{\sin(2\pi U_1)}{\sin(2\pi U_1)}
= \tan(2\pi U_1) \]
\[ \arctan\para{\dfrac{X_1}{X_2}} = 2\pi U_1 \]
\[ \dfrac{1}{2\pi}\arctan\para{\dfrac{X_1}{X_2}} 
= U_1 \]
and
\[ X_1 = \sqrt{-2\ln(U_2)}\sin(2\pi U_1)
\qquad X_2 = \sqrt{-2\ln(U_2)}\cos(2\pi U_1) \]
\[ X_1^2 + X_2^2 = 
-2\ln(U_2)\sin^2(2\pi U_1) 
-2\ln(U_2)\cos^2(2\pi U_1)
= -2\ln(U_2) \]
\[ -\dfrac{1}{2}(X_1^2 + X_2^2) = \ln(U_2) \]
\[ \exp\para{-\dfrac{X_1^2 + X_2^2}{2}} = U_2 \]
which means that
\[ g\inv(x) = \vecTwo
{\dfrac{1}{2\pi}\arctan\para{\dfrac{X_1}{X_2}}}
{\exp\para{-\dfrac{X_1^2 + X_2^2}{2}} } \]
We know that:
\[ J_{g\inv} = \matTwo
{\partialdd{(g\inv)_1}{x_1} \hspace{10pt}}
{\partialdd{(g\inv)_1}{x_2} \vspace{10pt}}
{\partialdd{(g\inv)_2}{x_1} \hspace{10pt}}
{\partialdd{(g\inv)_2}{x_2}} \]
\[ J_{g\inv} = \matTwo
{-\dfrac{X_2}{2\pi(X_1^2 + X_2^2)} \hspace{30pt}
\vspace{10pt}}
{\dfrac{X_1}{2\pi(X_1^2 + X_2^2)}}
{-X_1\exp\para{-\dfrac{X_1^2 + X_2^2}{2}} 
\hspace{10pt}}
{-X_2\exp\para{-\dfrac{X_1^2 + X_2^2}{2}}} \]
Now we can find:
\[ \det(J_{g\inv}) = 
\dfrac{X_2^2 \exp\para{-\dfrac{X_1^2 + X_2^2}{2}}}
{2\pi(X_1^2 + X_2^2)}
+ \dfrac{X_1^2 \exp\para{-\dfrac{X_1^2 + X_2^2}{2}}}
{2\pi(X_1^2 + X_2^2)}\]
\[ \det(J_{g\inv}) = 
\dfrac{\para{X_1^2 + X_2^2 }
\exp\para{-\dfrac{X_1^2 + X_2^2}{2}}}
{2\pi(X_1^2 + X_2^2)} \]
\[ \det(J_{g\inv}) = 
\dfrac{\exp\para{-\dfrac{X_1^2 + X_2^2}{2}}}{2\pi} \]
\[ |\det(J_{g\inv})| = \det(J_{g\inv}) \]
since it's an exponential,
and thus always positive. \\
So to conclude
\[ p_X(x_1, x_2) =
\piecewise{
\dfrac{\exp\para{-\dfrac{x_1^2 + x_2^2}{2}}}{2\pi}, 
\qquad \text{ for } u \in [0, 1]^2}
{ 0, \qquad \qquad \qquad \qquad \quad 
\text{ for } u \notin [0, 1]^2} \]

\myColor{red}{
\subsection*{Correction}
The answer is correct, but there are two details
worth mentioning if we want to dot our is
and cross our ts. \\
First, we can only invert the function $g$
if it is bijective (injective and surjective). \\
Notice however that because $\sin$
and $\cos$ are periodic every $2\pi$: 
\[ g(0, u_2) = \vecTwo{0}{\sqrt{-2\ln(u_2)}} \]
\[ g(1, u_2) = \vecTwo{0}{\sqrt{-2\ln(u_2)}} \]
So we can't have $U_1=1$ and $U_1 = 0$
in the domain, we have to choose one only. \\
Also whenever $U_2 = 1$,
the entire output is always $0$ regardless
of $U_1$, which means it is also not
bijective (or injective in this case):
\[ g(u_1, 1) = \vecTwo{0}{0} \]
So we remove $U_2 = 1$. \\
We also remove $U_2 = 0$ since the logarithm
is not defined or differential there. \\
We thus get the domain:
\[ (0, 1] \times (0, 1) \to \R^2 - \{0\} \]
Because we only removed a finite subset of the
domain (only 1 or 2 points really),
the density function is not affected and remains $1$. \\
Another detail worth mentionining is that the function
$\arctan$ only maps values to outputs 
between $-\dfrac{\pi}{2}$ and $\dfrac{\pi}{2}$. \\
This is because $\tan$
is periodic every $\pi$ inputs,
so we restrict its domain to just 
$-\dfrac{\pi}{2}$ and $\dfrac{\pi}{2}$
to ensure bijectivity before inverting
with $\arctan$. \\
But the output we get from $\arctan$
needs to span all of $2\pi$
(since we have $2\pi U_1$
and $U_1 \in (0, 1]$). \\
This is actually a very common problem with
$\arctan$, which manifests when we have
\[ \arctan\para{\dfrac{X}{Y}} \]
which is a result we get when transforming from
cartesian to polar coordinates. \\
So to solve it, we notice that in our $[0, 2\pi]$
domain, we have 4 quadrants, 
and of those 4, the first $(0, \dfrac{\pi}{2})$
and fourth $(\dfrac{3\pi}{2}, 2\pi)$ which can also
be written $(-\dfrac{\pi}{2}, 0)$
will be correct (since they are between 
$-\dfrac{\pi}{2}$ and $\dfrac{\pi}{2}$,
where $\arctan$ is correct). \\
However for the second quadrant,
we need to add $\pi$ in order to end up back
in the correct range. \\
We can check for which quadrant we are in
using the sign of $X$ and $Y$:
\[ \arctan\para{\dfrac{X}{Y}}
= \piecewise{
\arctan\para{\dfrac{X}{Y}} \text{ if } X > 0
\vspace{10pt}}
{\arctan\para{\dfrac{X}{Y}} + \pi \text{ if } X < 0 } \]
In our case we have $X = X_2$
and $Y = X_1$ (because $X_1$ has $\sin$,
the y-coordinate, and $X_2$ has $\cos$,
the x-coordinate). \\
This also won't affect the result because the
offset by $\pi$ just vanishes when we differentiate. \\
}

\newpage

\subsection*{Ex 2}

\begin{enumerate}[label=\alph*)]
    \item 
    The probability that the prediction is correct
    includes the event where the prediction
    and flip are head, and the event where
    the prediction and flip are tails:
    \[(H, H) \qquad  (T, T) \]
    Each corresponds to the following propabilities:
    \[ pq \qquad (1-p)(1-q) \]
    since predicting and flipping are independent
    events. \\
    So the total probability of predicting
    the correct coin flip is:
    \[  pq + (1-p)(1-q) \]
    \item 
    The probability of getting the right flip
    for a fixed bias $p$ is given by the function:
    \[ f(p) = pq + (1-p)(1-q) = q(2p-1) + p - 1 \]
    This is a line with slope $2p-1$,
    constrained to the domain $[0, 1]$
    (since $q$ is a probability). \\
    A line in some domain $[a, b]$
    has a maximum and minimum, one at
    $a$ and the other at $b$, depending on
    whether it is ascending or descending. \\
    So
    \[ 2p-1 > 0 \implies p > 0.5 \implies
    \text{ the line is ascending } \implies
    \text{ the maximum is at } q = 1 \]
    \[ 2p-1 < 0 \implies p < 0.5 \implies
    \text{ the line is descending } \implies
    \text{ the maximum is at } q = 0 \]
    \[ 2p-1 = 0 \implies p = 0.5 \implies
    \text{ the line is horizontal } \implies
    \text{ the distribution is uniform } \]
    So, to conclude, the best strategy for predicting
    the correct coin flip result is to
    always choose heads if the bias is
    in favor of heads, and to always choose tails
    if the bias is in favor of tails. \\
    If the coin flip is fair,
    then our choice doesn't matter;
    we can predict heads or tails
    following any distribution we want,
    the probability we will succeed is uniform. \\
\end{enumerate}

\newpage

\subsection*{Ex 3}

\begin{enumerate}[label=\alph*)]
    \item 
    The probability that "Free" is in the text:
    \[\Pb(F) = \Pb(F \mid S) \cdot \Pb(S)
    +  \Pb(F \mid \olsi{S}) \cdot \Pb(\olsi{S})
    = 0.8 * 0.7 + 0.2 * 0.3 = 0.62 \]
    We can now calculate the posterior
    probabilities:
    \[ \Pb(S \mid F)
    = \dfrac{\Pb(F \mid S) \cdot \Pb(S)}{\Pb(F)}
    = \dfrac{0.8 \cdot 0.7}{0.62}
    \sim 0.90 \]
    \[ \Pb(\olsi{S} \mid F)
    = \dfrac{\Pb(F \mid \olsi{S}) \cdot \Pb(\olsi{S})}
    {\Pb(F)}
    = \dfrac{0.2 \cdot 0.3}{0.62}
    \sim 0.1 \]
    \item 
    We have a loss function:
    \[ \ell(a, b)
    = \piecewise{0, \quad \text{ if $a = b$}}
    {1, \quad \text{ if $a \neq b$}} \]
    which is just the indicator function
    $\indicator_{a \neq b}$. \\ 
    Let us assume that $f(X)$
    is our prediction.
    Where $X = F$ is the input (contains
    the word free).
    We know that the risk is the total
    expected value of the loss over
    all inputs $x$
    (which is a sum over all of $X$
    of a sum over all of $Y$):
    \[ \rscr_\ell(f) = \E[\ell(f(X), Y)]
    = \E[\E[\ell(f(X), Y) \mid X]] \]
    If we remove the outer sum (over $X$),
    we get the conditional risk for a 
    single input $x$, which is
    what we want to minimize:
    \[ \E[\ell(f(x), Y) \mid X = x] \]
    It is just a sum over all $Y$. \\
    Our goal is to minimize this value,
    meaning we need to find the value of $x$
    for which the expected value 
    of the conditional loss is minimal:
    \[ f^*(x) = \min_{f(x)} \\E[\ell(f(x), Y) \mid X = x]  \]
    We have:
    \[ \E[\ell(f(x), Y) \mid X = x]
    = \ell(f(x), S)\Pb(Y = S \mid X = x)
    + \ell(f(x), \olsi{S})\Pb(Y = \olsi{S} \mid X = x) \]
    \[ \E[\ell(f(F), Y) \mid F]
    = \indicator_{f(F) \neq \olsi{S}}
    \Pb(\olsi{S} \mid F) +
    \indicator_{f(F) \neq S}
    \Pb(S \mid X)  \]
    Now, suppose we predict that
    the email is spam $f(S)$:
    \[ \E[\ell(S, Y) \mid F]
    = \indicator_{S \neq \olsi{S}}
    \Pb(\olsi{S} \mid F) +
    \indicator_{S \neq S}
    \Pb(S \mid X) 
    \sim 1 \cdot 0.1 + 0 \cdot 0.9 = 0.1 \]
    If we predict that the email is not spam:
    \[ \E[\ell(\olsi{S}, Y) \mid F]
    = \indicator_{\olsi{S} \neq \olsi{S}}
    \Pb(\olsi{S} \mid F) +
    \indicator_{\olsi{S} \neq S}
    \Pb(S \mid F) 
    \sim 1 \cdot 0.9 + 0 \cdot 0.1 = 0.9 \]
    We need to minimize the expected value,
    which means
    choosing the inputs for which the output
    is the smallest.
    The output is smallest when we choose
    $a$ to be the case where the email is spam. \\
    So we should consider the email spam.
    \item 
    If we now have
    \[ \Pb(S) = 0.4 \qquad 
    \Pb(\olsi{S}) = 0.6 \]
    then we can recalculate:
    \[\Pb(F) = \Pb(F \mid S) \cdot \Pb(S)
    +  \Pb(F \mid \olsi{S}) \cdot \Pb(\olsi{S})
    = 0.8 * 0.4 + 0.2 * 0.6 = 0.44 \]
    \[ \Pb(S \mid F)
    = \dfrac{\Pb(F \mid S) \cdot \Pb(S)}{\Pb(F)}
    = \dfrac{0.8 \cdot 0.4}{0.44}
    \sim 0.73 \]
    \[ \Pb(\olsi{S} \mid F)
    = \dfrac{\Pb(F \mid \olsi{S}) \cdot \Pb(\olsi{S})}
    {\Pb(F)}
    = \dfrac{0.2 \cdot 0.6}{0.44}
    \sim 0.27 \]
    We can then find the expected value again. \\
    If we predict spam:
    \[ \sim 1 \cdot 0.27 + 0 \cdot 0.73 = 0.27  \]
    If we predict not spam:
    \[ \sim 1 \cdot 0.73 + 0 \cdot 0.27 = 0.73  \]
    So the expected value of the loss is minimized
    when we predict that the email is spam. \\
    Our decision should therefore not change. \\
\end{enumerate}

\end{document}

