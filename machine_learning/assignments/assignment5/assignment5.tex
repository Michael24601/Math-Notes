

\documentclass[12pt]{article}

\usepackage[margin=1in]{geometry}


%===============================================================================
%================================== PACKAGES ===================================
%===============================================================================

% For using float option H that places figures 
% exatcly where we want them
\usepackage{float}
% makes figure font bold
\usepackage{caption}
\captionsetup[figure]{labelfont=bf}
% For text generation
\usepackage{lipsum}
% For drawing
\usepackage{tikz}
% For smaller or equal sign and not divide sign
\usepackage{amssymb}
% For the diagonal fraction
\usepackage{xfrac}
% For enumerating exercise parts with letters instead of numbers
\usepackage{enumitem}
% For dfrac, which forces the fraction to be in display mode (large) e
% even in math mode (small)
\usepackage{amsmath}
% For degree sign
\usepackage{gensymb}
% For "\mathbb" macro
\usepackage{amsfonts}
% For positioning 
\usepackage{indentfirst}
\usetikzlibrary{shapes,positioning,fit,calc}
% for adjustwidth environment
\usepackage{changepage}
% for arrow on top
\usepackage{esvect}
% for mathbb 1
\usepackage{bbm}
% for mathsrc
\usepackage[mathscr]{eucal}
% For degree sign
\usepackage{gensymb}
% For quotes
\usepackage{csquotes}
% For vertical lines
\usepackage{mathtools}
% For cols
\usepackage{multicol}

% for tikz
\usepackage{pgfplots}
\pgfplotsset{compat=1.18}
\usepackage{amsmath}
\usepgfplotslibrary{groupplots}


%===============================================================================
%==================================== FONTS ====================================
%===============================================================================


% Mathcal
\newcommand{\acal}{\mathcal{A}}
\newcommand{\bcal}{\mathcal{B}}
\newcommand{\ccal}{\mathcal{C}}
\newcommand{\dcal}{\mathcal{D}}
\newcommand{\ecal}{\mathcal{E}}
\newcommand{\fcal}{\mathcal{F}}
\newcommand{\gcal}{\mathcal{G}}
\newcommand{\hcal}{\mathcal{H}}
\newcommand{\ical}{\mathcal{I}}
\newcommand{\jcal}{\mathcal{J}}
\newcommand{\kcal}{\mathcal{K}}
\newcommand{\lcal}{\mathcal{L}}
\newcommand{\mcal}{\mathcal{M}}
\newcommand{\ncal}{\mathcal{N}}
\newcommand{\ocal}{\mathcal{O}}
\newcommand{\pcal}{\mathcal{P}}
\newcommand{\qcal}{\mathcal{Q}}
\newcommand{\rcal}{\mathcal{R}}
\newcommand{\scal}{\mathcal{S}}
\newcommand{\tcal}{\mathcal{T}}
\newcommand{\ucal}{\mathcal{U}}
\newcommand{\vcal}{\mathcal{V}}
\newcommand{\wcal}{\mathcal{W}}
\newcommand{\xcal}{\mathcal{X}}
\newcommand{\ycal}{\mathcal{Y}}
\newcommand{\zcal}{\mathcal{Z}}

% Mathfrak
\newcommand{\afrak}{\mathfrak{A}}
\newcommand{\bfrak}{\mathfrak{B}}
\newcommand{\cfrak}{\mathfrak{C}}
\newcommand{\dfrak}{\mathfrak{D}}
\newcommand{\efrak}{\mathfrak{E}}
\newcommand{\ffrak}{\mathfrak{F}}
\newcommand{\gfrak}{\mathfrak{G}}
\newcommand{\hfrak}{\mathfrak{H}}
\newcommand{\ifrak}{\mathfrak{I}}
\newcommand{\jfrak}{\mathfrak{J}}
\newcommand{\kfrak}{\mathfrak{K}}
\newcommand{\lfrak}{\mathfrak{L}}
\newcommand{\mfrak}{\mathfrak{M}}
\newcommand{\nfrak}{\mathfrak{N}}
\newcommand{\ofrak}{\mathfrak{O}}
\newcommand{\pfrak}{\mathfrak{P}}
\newcommand{\qfrak}{\mathfrak{Q}}
\newcommand{\rfrak}{\mathfrak{R}}
\newcommand{\sfrak}{\mathfrak{S}}
\newcommand{\tfrak}{\mathfrak{T}}
\newcommand{\ufrak}{\mathfrak{U}}
\newcommand{\vfrak}{\mathfrak{V}}
\newcommand{\wfrak}{\mathfrak{W}}
\newcommand{\xfrak}{\mathfrak{X}}
\newcommand{\yfrak}{\mathfrak{Y}}
\newcommand{\zfrak}{\mathfrak{Z}}

% Mathscr
\newcommand{\ascr}{\mathscr{A}}
\newcommand{\bscr}{\mathscr{B}}
\newcommand{\cscr}{\mathscr{C}}
\newcommand{\dscr}{\mathscr{D}}
\newcommand{\escr}{\mathscr{E}}
\newcommand{\fscr}{\mathscr{F}}
\newcommand{\gscr}{\mathscr{G}}
\newcommand{\hscr}{\mathscr{H}}
\newcommand{\iscr}{\mathscr{I}}
\newcommand{\jscr}{\mathscr{J}}
\newcommand{\kscr}{\mathscr{K}}
\newcommand{\lscr}{\mathscr{L}}
\newcommand{\mscr}{\mathscr{M}}
\newcommand{\nscr}{\mathscr{N}}
\newcommand{\oscr}{\mathscr{O}}
\newcommand{\pscr}{\mathscr{P}}
\newcommand{\qscr}{\mathscr{Q}}
\newcommand{\rscr}{\mathscr{R}}
\newcommand{\sscr}{\mathscr{S}}
\newcommand{\tscr}{\mathscr{T}}
\newcommand{\uscr}{\mathscr{U}}
\newcommand{\vscr}{\mathscr{V}}
\newcommand{\wscr}{\mathscr{W}}
\newcommand{\xscr}{\mathscr{X}}
\newcommand{\yscr}{\mathscr{Y}}
\newcommand{\zscr}{\mathscr{Z}}

% Mathbb
\newcommand{\abb}{\mathbb{A}}
\newcommand{\bbb}{\mathbb{B}}
\newcommand{\cbb}{\mathbb{C}}
\newcommand{\dbb}{\mathbb{D}}
\newcommand{\ebb}{\mathbb{E}}
\newcommand{\fbb}{\mathbb{F}}
\newcommand{\gbb}{\mathbb{G}}
\newcommand{\hbb}{\mathbb{H}}
\newcommand{\ibb}{\mathbb{I}}
\newcommand{\jbb}{\mathbb{J}}
\newcommand{\kbb}{\mathbb{K}}
\newcommand{\lbb}{\mathbb{L}}
\newcommand{\mbb}{\mathbb{M}}
\newcommand{\nbb}{\mathbb{N}}
\newcommand{\obb}{\mathbb{O}}
\newcommand{\pbb}{\mathbb{P}}
\newcommand{\qbb}{\mathbb{Q}}
\newcommand{\rbb}{\mathbb{R}}
\newcommand{\sbb}{\mathbb{S}}
\newcommand{\tbb}{\mathbb{T}}
\newcommand{\ubb}{\mathbb{U}}
\newcommand{\vbb}{\mathbb{V}}
\newcommand{\wbb}{\mathbb{W}}
\newcommand{\xbb}{\mathbb{X}}
\newcommand{\ybb}{\mathbb{Y}}
\newcommand{\zbb}{\mathbb{Z}}


%===============================================================================
%=============================== SPECIAL SYMBOLS ===============================
%===============================================================================


% Orbit (group theory)
\newcommand{\orbit}{\mathcal{O}}
% Normal group
\newcommand{\normal}{\mathcal{N}}
% Indicator function
\newcommand{\indicator}{\mathbbm{1}}
% Laplace transform
\newcommand{\laplace}[1]{\mathcal{L}}
% Epsilon shorthand
\newcommand{\eps}{\varepsilon}
% Omega
\newcommand{\om}{\omega}
\newcommand{\Om}{\Omega}

%===============================================================================
%================================== OPERATORS ==================================
%===============================================================================


% Inverse exponent
\newcommand{\inv}[0]{^{-1}}
% Overline bar
\newcommand{\olsi}[1]{\,\overline{\!{#1}}}
% Less than or equal slanted
\newcommand{\seqs}{\leqslant}
% Greater or equal slanted
\newcommand{\geqs}{\geqslant}
% Subset or equal
\newcommand{\sub}{\subseteq}
% Proper subset
\newcommand{\prosub}{\subset}
% from
\newcommand{\from}{\leftarrow}

% Parantheses
\newcommand{\para}[1]{\left( #1 \right)}
% Curly Braces
\newcommand{\curl}[1]{\left\{ #1 \right\}}
% Brackets
\newcommand{\brac}[1]{\left[ #1 \right]}
% Angled Brackets
\newcommand{\ang}[1]{\left\langle #1 \right\rangle}
% Norm
\newcommand{\norm}[1]{\left\| #1 \right\|}

% Piece wise (use \\ between cases)
\newcommand{\piecewise}[1]{\begin{cases} #1 \end{cases}}

% Bold symbol shorthand
\newcommand{\bl}{\boldsymbol}

% Vertical space
\newcommand{\vs}[1]{\vspace{#1 pt}}
% Horizontal ertical space
\newcommand{\hs}[1]{\hspace{#1 pt}}


%===============================================================================
%============================== TEXT BASED SYMBOLS =============================
%===============================================================================


% Radians
\newcommand{\rad}{\text{rad}}
% Least Common Multiple
\newcommand{\lcm}{\text{lcm}}
% Automorphism
\newcommand{\Aut}{\text{Aut}}
% Variance
\newcommand{\var}{\text{Var}}
% Covariance
\newcommand{\cov}{\text{Cov}}
% Cofactor (matrix)
\newcommand{\cof}{\text{Cof}}
% Adjugate (matrix)
\newcommand{\adj}{\text{Adj}}
% Trace (matrix)
\newcommand{\tr}{\text{tr}}
% Standard deviation
\newcommand{\std}{\text{Std}}
% Correlation coefficient
\newcommand{\corr}{\text{Corr}}
% Sign
\newcommand{\sign}{\text{sign}}

% And text
\newcommand{\AND}{\text{ and }}
% Or text 
\newcommand{\OR}{\text{ or }}
% For text 
\newcommand{\FOR}{\text{ for }}
% If text
\newcommand{\IF}{\text{ if }}
% When text
\newcommand{\WHEN}{\text{ when }}
% Where text
\newcommand{\WHERE}{\text{ where }}
% Then text
\newcommand{\THEN}{\text{ then }}
% Such that text
\newcommand{\SUCHTHAT}{\text{ such that }}

% 1st
\newcommand{\st}[1]{#1^{\text{st}}}
% 2nd
\newcommand{\nd}[1]{#1^{\text{nd}}}
% 3rd
\newcommand{\rd}[1]{#1^{\text{rd}}}
% nth
\newcommand{\nth}[1]{#1^{\text{th}}}


%===============================================================================
%========================= PROBABILITY AND STATISTICS ==========================
%===============================================================================


% Permutation
\newcommand{\perm}[2]{{}^{#1}\!P_{#2}}
% Combination
\newcommand{\comb}[2]{{}^{#1}C_{#2}}

% Baye's risk
\newcommand{\risk}[1]{\mathscr{R}_{#1}}
% Baye's optimal risk
\newcommand{\riskOptimal}[1]{\mathscr{R}_{#1}^*}
% Baye's empirical risk
\newcommand{\riskEmpirical}[2]{\hat{\mathscr{R}}_{#1}^{#2}}


%===============================================================================
%=================================== CALCULUS ==================================
%===============================================================================


% d over d derivative
\newcommand{\dd}[2]{\dfrac{d#1}{d#2}}
% partial d over d derivative
\newcommand{\partialdd}[2]{\dfrac{\partial #1}{\partial #2}}
% delta d over d derivative
\newcommand{\deltadd}[2]{\dfrac{\Delta #1}{\Delta #2}}

% Integration between a and b
\newcommand{\integral}[4]{\int_{#1}^{#2} #3 \, #4}
% Integration in some space 
\newcommand{\boundIntegral}[2]{\int_{#1} #2 \, d#1}

% Limit
\newcommand{\limit}[3]{\lim_{#1 \to #2} #3}


%===============================================================================
%================================  BIG SYMBOLS  ================================
%===============================================================================


% Sum
\newcommand{\sumof}[2]{\sum_{#1}^{#2}}
% Product
\newcommand{\productof}[2]{\prod_{#1}^{#2}}
% Union
\newcommand{\unionof}[2]{\bigcup_{#1}^{#2}}
% Intersection
\newcommand{\intersectionof}[2]{\bigcap_{#1}^{#2}}
% Or
\newcommand{\orof}[2]{\bigvee_{#1}^{#2}}
% And
\newcommand{\andof}[2]{\bigwedge_{#1}^{#2}}


%===============================================================================
%=============================== LINEAR ALGEBRA ================================
%===============================================================================


% Bold vector arrow
\newcommand{\bv}[1]{\vec{\mathbf{#1}}}

% Matrix or vector (use // for column, & for row) 
% with brackets
\newcommand{\bmat}[1]{\begin{bmatrix} #1 \end{bmatrix}}
% Matrix or vector (use // for column, & for row) 
% with curved brackets
\newcommand{\pmat}[1]{\begin{pmatrix} #1 \end{pmatrix}}
% Matrix or vector (use // for column, & for row) 
% with lines on either side 
\newcommand{\lmat}[1]{\begin{vmatrix} #1 \end{vmatrix}}
% Matrix or vector (use // for column, & for row) 
% with curly braces
\newcommand{\cmat}[1]{\begin{Bmatrix} #1 \end{Bmatrix}}
% Matrix or vector (use // for column, & for row) 
% with no braces
\newcommand{\mat}[1]{\begin{matrix} #1 \end{matrix}}


%===============================================================================
%================================ LARGE OBJECTS ================================
%===============================================================================

% Multiple lines
\newcommand{\multiline}[1]{
\begin{align*}
    #1
\end{align*}
}

% Multiple lines with equation numbers
\newcommand{\eqmultiline}[1]{
\begin{align*}
    #1
\end{align*}
}

% Color
\newcommand{\colorText}[2]{
\begingroup
\color{#1}
    #2
\endgroup
}

% Centered figure
\newcommand{\centerFigure}[2]{
    \begin{figure}[h]
        \centering
            #1
        \caption{#2}
    \end{figure}
}

% Tikz figure
\newcommand{\tikzGraphic}[1]{
    \begin{center}
    \begin{tikzpicture}
        #1
    \end{tikzpicture}
    \end{center}
}

% Enumerate numbers (seperate by \item)
\newcommand{\numbers}{\textbf{\number*)}}

% Enumerate letters (seperate by \item)
\newcommand{\letters}{\textbf{\alph*)}}

\title{
    \Huge Machine Learning \\
    \Large Assignment V
}
\date{2025-05-16}
\author{Michael Saba}

\begin{document}
\pagenumbering{gobble}
\maketitle
\newpage
\setlength{\parindent}{0pt}
\pagenumbering{arabic}

\subsection*{Ex 1}

\begin{enumerate}[label = \letters]
\item 
We have:
\[ \riskEmpirical{\ell}{f} = \ebb[(f(X) - Y)^2] \]
We can add then substract the term $f^*(X)$,
and then expand the square term:
\[ \ebb[(f(X)- f^*(X) + f^*(X) - Y)^2] \]
\[ \ebb[((f(X)- f^*(X)) + (f^*(X) - Y))^2] \]
\[ \ebb[(f(X)- f^*(X))^2]
+ \ebb[(f^*(X) - Y)^2]
+ 2\ebb[(f(X)- f^*(X))(f^*(X) - Y)] \]
The result we want is the first two terms.
We can show the last term is just $0$
by decomposing the joint expected value into
a conditional over $Y$ and a marginal over $X$:
\[ 2\ebb[\ebb[(f(X)- f^*(X))(f^*(X) - Y) \mid X ]] \]
Notice that the term $(f(X)- f^*(X))$
does not depend on the randomness of $Y$,
so we can take it out of the inner expected value:
\[ 2\ebb[(f(X)- f^*(X))\ebb[f^*(X) - Y \mid X ]] \]
We can now also take out $f^*(X)$:
\[ 2\ebb[(f(X)- f^*(X)) (f^*(X) - \ebb[Y \mid X ])] \]
But recall that $f^*(X) = \ebb[Y \mid X ]$
by definition, so:
\[ \ebb[(f(X)- f^*(X)) ( \ebb[Y \mid X ] - \ebb[Y \mid X ])]
= \ebb[(f(X)- f^*(X)) \cdot 0 ] = 0 \]
So our final result is:
\[  \risk{\ell}(f) =  \ebb[(f^*(X) - f(X))^2]
+ \ebb[(Y - f^*(X))^2] \]
Which is what we needed to find.
\item
We need to caluclate this expected value,
for a risk that depends on the randomness of training
data $D$:
\[ \ebb[\risk{\ell}(f_D)] \]
We can mark the outer expected value over $D$
with a $D$ for clarity:
\[ \ebb_D[\risk{\ell}(f_D)] 
= \ebb_D[\ebb[(f^*(X) - f_D(X))^2] + \ebb[(Y - f^*(X))^2]] \]
The second term does not depend on $D$
so we can take it out. This will be the noise term:
\[ \ebb_D[\risk{\ell}(f_D)] 
= \ebb_D[\ebb[(f^*(X) - f_D(X))^2]] + \ebb[(Y - f^*(X))^2] \]
Now we will decompose the first term into
variance and bias. First we use Fubini's theorem
to flip the expected value order,
and focus on the inner one (the one over $D$):
\[\ebb_D[\ebb[(f_D(X) - f^*(X))^2]] 
= \ebb[\ebb_D[(f_D(X) - f^*(X))^2]] \]
Now, to do the bias-variance decomposition
(on the inner expected value)
we add and then subtract the expected value of $f_D(X)$:
\[ \ebb_D[(f_D(X) - \ebb_D[f_D(X)]
+ \ebb_D[f_D(X)] - f^*(X))^2] \]
\[ \ebb_D[((f_D(X) - \ebb_D[f_D(X)])
+ (\ebb_D[f_D(X)] - f^*(X)))^2] \]
\multiline{\ebb_D[((f_D(X) - \ebb_D[f_D(X)])^2]
& + \ebb_D[(\ebb[f_D(X)] - f^*(X)))^2] \\
& + 2\ebb_D[(f_D(X) - \ebb_D[f_D(X)]) 
(\ebb_D[f_D(X)] - f^*(X))] }
Notice that the third term is $0$.
We can factor out the second part
since it doesn't depend on $D$:
\[ 2\ebb_D[(f_D(X) - \ebb_D[f_D(X)])] 
\cdot (\ebb_D[f_D(X)] - f^*(X)) \]
We then also take out $\ebb_D[f_D(X)]$
from the expected valye since it's a constant:
\[ 2(\ebb_D[f_D(X)] - \ebb_D[f_D(X)]) 
\cdot (\ebb_D[f_D(X)] - f^*(X))
= 0 \cdot (\ebb_D[f_D(X)] - f^*(X)) = 0 \]
So we take our remaining two terms
and wrap them back in the outer expected
value over $X, Y$,
and then also add the noise:
\multiline{ \ebb[\ebb_D[(f_D(X) - \ebb_D[f_D(X)])^2]]
& + \ebb[\ebb_D[(\ebb_D[f_D(X)] - f^*(X)))^2] \\
& + \ebb[(Y - f^*(X))^2] }
Note that we can remove $\ebb_D$ from around the 
bias, as it does not depend on $D$:
\multiline{ \ebb[\ebb_D[(f_D(X) - \ebb_D[f_D(X)])^2]]
& + \ebb[(\ebb_D[f_D(X)] - f^*(X))^2] \\
& + \ebb[(Y - f^*(X))^2] }
This is the same result as the one in the
question sheet, but with the only difference
being that they also have all three
expected value over $D$ terms wfrrrerritten with
as a conditional over $X$:
\[ \ebb_D[ \cdot ] \to \ebb_D[\; \cdot \mid X] \]
We can do that in our answer as well since
$D$ and $X$, the testing and training data,
are assumed to be independent,
so adding the condition should change nothing. \\
\item
A simpler model that underfits the data will have
a higher bias, while a more complex model that overfits
the training data will have a higher variance. \\
It's usually a tradeoff between the two;
shrinking one usually increases the other. \\
So to get the minimal risk, we need to find the
minimal value of their sum (and the noise). \\
\item
Not sure if we're supposed to calculate anything,
but in theory the bias would increase with a smaller
$k$, and the variance would increase with a
larger $k$. \\
\end{enumerate}

\colorText{red}{
    \begin{enumerate}[label = \letters]
        \item 
        The noise is not optimizable, it is the part of
        the risk that we can't minimize, and will be
        the difference between the best achievable risk
        and the best theoretical risk
        that minimizes variance and bias.
        \item 
        Not a correction, just a note;
        what I did was correct, and it should always be done;
        noting with a subscript what the expected value
        is summing over is integral to correctly
        solving the problem. In a test we should do that,
        but the given may not have subscripts,
        we would need to figure them out ourselves. \\
        Also, for Fubini's theorem,
        which is about decomposing joint expected
        values and about flipping the order
        of the expected values, works only for
        independent variables; so we can switch the order
        of expected value $\ebb_D$ and $\ebb_{X, Y}$,
        but we need to note that they are independent. \\
        \item 
        \item 
    \end{enumerate}
}

\newpage

\subsection*{Ex 2}

\begin{enumerate}[label=\letters]
    \item 
    First note that:
    \[ f'(x) = (w_0 + w_1x)' = w_1 \]
    So:
    \[ \integral{0}{1}{|f'(x)|^2}{dx}
    = \integral{0}{1}{w_1^2}{dx} = w_1^2 \]
    Notice that what we are doing is penalizing
    using a quadratic term $w_1^2$,
    the slope of $x$ in the prediction function $f(x)$. \\
    This is essentially ridge regression,
    where $\|\theta\|^2$
    is the norm squared of the weights $\theta$
    representing the slope. \\
    \item 
    First note that:
    \[ f''(x) = (w_0 + w_1x + w_2x^2)'' = 2w_2 \]
    So:
    \[ \integral{0}{1}{|f'(x)|^2}{dx}
    = \integral{0}{1}{(2w_2)^2}{dx} = 4w_2^2 \]
    Notice that this regression only penalizes
    the term a large weight $w_2$
    for the quadratic term $w_2x^2$,
    so in pratice, what it does is ensure that
    the function has low curvature,
    and is more linear. \\
\end{enumerate}

\colorText{red}{
    \begin{enumerate}[label = \letters]
        \item 
        Not a correction, just a note;
        the ridge regression is also called
        the L2 norm.
        \item 
        Not a correction, just a note;
        when we minimize the risk and regularizer,
        we ned up minimizing the terms we found
        in parts a and b, which is why
        these terms become close to and remove the
        slope/curvature.
    \end{enumerate}
}

\newpage

\subsection*{Ex 3}

\begin{enumerate}[label=\letters]
    \item 
    First, we want to maximize:
    \[ \arg \max_\theta p(\theta \mid \dcal) \]
    Essentially, we want the parameters
    theta such that the probability that
    $f_\theta(x_i) = y_i$ generates the dataset
    $\dcal$ is highest. \\
    First note that by Baye's rule:
    \[ p(\theta \mid \dcal) =
    \dfrac{p(\dcal \mid \theta)p(\theta)}{p(\dcal)} \]
    Here i am assuming that the prior assumption
    $p(\theta)$ is the regularization and form
    of $\theta$. \\
    Notice that:
    \[ p(\dcal \mid \theta) = \prod_i p(y_i \mid x_i, \theta)\]
    Why? As mentioned in the given, the dataset $\dcal$
    is given by $p(y_i \mid x_i, \theta)$,
    so we expect that the probability that $\dcal$
    is generate using $\theta$ parameters,
    is the probability that each individual $y_i$
    is generated using $f_\theta(x_i)$. 
    Note that multiplying the posteriors like that
    only works because we assumed all data points
    are independent (independent and identically
    distributed). \\
    So we have:
    \[ p(\theta \mid \dcal) =
    \dfrac{\para{\prod_i p(y_i \mid x_i, \theta)}
    p(\theta)}{p(\dcal)} \]
    Now notice that the denominator is a constant,
    so it won't affect the maximization:
    \[ \arg \max_\theta \para{\prod_i p(y_i \mid x_i, \theta)}
    p(\theta) \]
    Moreover, notice that maximizing this
    is the same as minimizing the inverse:
    \[ \arg \min_\theta \dfrac{1}{\para{\prod_i p(y_i \mid x_i, \theta)}
    p(\theta)} \]
    Beccause $\log$ is a monotonic increasing function,
    we can apply it to the formula and it won't change
    the optimization:
    \[ \arg \min_\theta \log\para{\dfrac{1}{\para{\prod_i p(y_i \mid x_i, \theta)}
    p(\theta)}} \]
    Notice that $\log$ turns products to sums of logs,
    and inverses to negative logs:
    \[ \hat{\theta}_\text{MAP} = 
    \arg \min_\theta -\sum_i\log\para{p(y_i \mid x_i, \theta)}
    -\log(p(\theta)) \]
    Which completes the proof. \\
    \item 
    Our prior assumption about $\theta$ is that:
    \[ p(\theta) = \exp(-\lambda \rcal(\theta)) \]
    Where $\rcal(\theta)$ is some regularization term.
    This means that we now have:
    \[ \arg \min_\theta -\sum_i\log\para{p(y_i \mid x_i, \theta)}
    -\log(\exp(-\lambda \rcal(\theta))) \]
    Which reduces to:
    \[ \arg \min_\theta -\sum_i\log\para{p(y_i \mid x_i, \theta)}
    + \lambda \rcal(\theta) \]
    We are then given $\ell(y, f_\theta(x)) = 
    -\log\para{p(y_i \mid x_i, \theta)}$,
    So we can write:
    \[ \hat{\theta}_\text{MAP} = 
    \arg \min_\theta \sum_i \ell(y, f_\theta(x))
    + \lambda \rcal(\theta) \]
    Which is what we needed to show. \\
\end{enumerate}

\colorText{red}{
    \begin{enumerate}[label = \letters]
        \item 
        Note that in reality:
        \[ p(D \mid \theta) = 
        \prod_i p(y_i \mid x_i, \theta)p(x \mid \theta) \]
        We also need to consider the probability
        of $x$ in this situation,
        since we are conditioning over it;
        but in this case $p(x \mid \theta) $
        is constant, so we can ignore it.
        We are told so in the given.
        \item 
        Note that this exercise
        is about showing how Bayesian Learning
        is based on MAP estimation. \\
        In Maximum Likelihood Estimation MLE,
        what we do is try to maximize
        $\pbb(D \mid \theta)$;
        basically finding the parameters $\theta$
        of the classifier 
        most likely to generate $D$:
        \[ \arg \max_\theta \pbb(D \mid \theta) \]
        This differs from Maximum A Posteriori
        estimation,
        where we consider the posterior
        as well as the prior.
        Maximum A Posteriori estimation is what
        leads to Bayesian Learning as we know it. \\
        In MAP, we want the most likely $\theta$ parameters
        given the dataset:
        \[ \arg \max_\theta \pbb(\theta \mid D) \]
        We can decompose the probability
        using Baye's rule:
        \[ \pbb(\theta \mid D) \propto 
        \pbb(D \mid \theta)\pbb(\theta) \] 
        Which is when the prior shows up.
        Here we use proportional to $\propto$
        because we ignore $\pbb(D)$,
        which is a constant with respect to $\theta$. \\
        Here $\pbb(\theta)$ is a prior assumption 
        about $\theta$, which is essentially
        the regularization; this is the approach we
        use in this course. \\
        In the exercise we then
        derive the minimization of the risk
        and regularization,
        which is Bayesian Learning as we know it,
        and is based on MAP. \\

    \end{enumerate}
}

\newpage

\end{document}