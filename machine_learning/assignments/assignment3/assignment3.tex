
\documentclass[12pt]{article}

\usepackage[margin=1in]{geometry}


%===============================================================================
%================================== PACKAGES ===================================
%===============================================================================

% For using float option H that places figures 
% exatcly where we want them
\usepackage{float}
% makes figure font bold
\usepackage{caption}
\captionsetup[figure]{labelfont=bf}
% For text generation
\usepackage{lipsum}
% For drawing
\usepackage{tikz}
% For smaller or equal sign and not divide sign
\usepackage{amssymb}
% For the diagonal fraction
\usepackage{xfrac}
% For enumerating exercise parts with letters instead of numbers
\usepackage{enumitem}
% For dfrac, which forces the fraction to be in display mode (large) e
% even in math mode (small)
\usepackage{amsmath}
% For degree sign
\usepackage{gensymb}
% For "\mathbb" macro
\usepackage{amsfonts}
% For positioning 
\usepackage{indentfirst}
\usetikzlibrary{shapes,positioning,fit,calc}
% for adjustwidth environment
\usepackage{changepage}
% for arrow on top
\usepackage{esvect}
% for mathbb 1
\usepackage{bbm}
% for mathsrc
\usepackage[mathscr]{eucal}
% For degree sign
\usepackage{gensymb}
% For quotes
\usepackage{csquotes}
% For vertical lines
\usepackage{mathtools}
% For cols
\usepackage{multicol}

% for tikz
\usepackage{pgfplots}
\pgfplotsset{compat=1.18}
\usepackage{amsmath}
\usepgfplotslibrary{groupplots}


%===============================================================================
%==================================== FONTS ====================================
%===============================================================================


% Mathcal
\newcommand{\acal}{\mathcal{A}}
\newcommand{\bcal}{\mathcal{B}}
\newcommand{\ccal}{\mathcal{C}}
\newcommand{\dcal}{\mathcal{D}}
\newcommand{\ecal}{\mathcal{E}}
\newcommand{\fcal}{\mathcal{F}}
\newcommand{\gcal}{\mathcal{G}}
\newcommand{\hcal}{\mathcal{H}}
\newcommand{\ical}{\mathcal{I}}
\newcommand{\jcal}{\mathcal{J}}
\newcommand{\kcal}{\mathcal{K}}
\newcommand{\lcal}{\mathcal{L}}
\newcommand{\mcal}{\mathcal{M}}
\newcommand{\ncal}{\mathcal{N}}
\newcommand{\ocal}{\mathcal{O}}
\newcommand{\pcal}{\mathcal{P}}
\newcommand{\qcal}{\mathcal{Q}}
\newcommand{\rcal}{\mathcal{R}}
\newcommand{\scal}{\mathcal{S}}
\newcommand{\tcal}{\mathcal{T}}
\newcommand{\ucal}{\mathcal{U}}
\newcommand{\vcal}{\mathcal{V}}
\newcommand{\wcal}{\mathcal{W}}
\newcommand{\xcal}{\mathcal{X}}
\newcommand{\ycal}{\mathcal{Y}}
\newcommand{\zcal}{\mathcal{Z}}

% Mathfrak
\newcommand{\afrak}{\mathfrak{A}}
\newcommand{\bfrak}{\mathfrak{B}}
\newcommand{\cfrak}{\mathfrak{C}}
\newcommand{\dfrak}{\mathfrak{D}}
\newcommand{\efrak}{\mathfrak{E}}
\newcommand{\ffrak}{\mathfrak{F}}
\newcommand{\gfrak}{\mathfrak{G}}
\newcommand{\hfrak}{\mathfrak{H}}
\newcommand{\ifrak}{\mathfrak{I}}
\newcommand{\jfrak}{\mathfrak{J}}
\newcommand{\kfrak}{\mathfrak{K}}
\newcommand{\lfrak}{\mathfrak{L}}
\newcommand{\mfrak}{\mathfrak{M}}
\newcommand{\nfrak}{\mathfrak{N}}
\newcommand{\ofrak}{\mathfrak{O}}
\newcommand{\pfrak}{\mathfrak{P}}
\newcommand{\qfrak}{\mathfrak{Q}}
\newcommand{\rfrak}{\mathfrak{R}}
\newcommand{\sfrak}{\mathfrak{S}}
\newcommand{\tfrak}{\mathfrak{T}}
\newcommand{\ufrak}{\mathfrak{U}}
\newcommand{\vfrak}{\mathfrak{V}}
\newcommand{\wfrak}{\mathfrak{W}}
\newcommand{\xfrak}{\mathfrak{X}}
\newcommand{\yfrak}{\mathfrak{Y}}
\newcommand{\zfrak}{\mathfrak{Z}}

% Mathscr
\newcommand{\ascr}{\mathscr{A}}
\newcommand{\bscr}{\mathscr{B}}
\newcommand{\cscr}{\mathscr{C}}
\newcommand{\dscr}{\mathscr{D}}
\newcommand{\escr}{\mathscr{E}}
\newcommand{\fscr}{\mathscr{F}}
\newcommand{\gscr}{\mathscr{G}}
\newcommand{\hscr}{\mathscr{H}}
\newcommand{\iscr}{\mathscr{I}}
\newcommand{\jscr}{\mathscr{J}}
\newcommand{\kscr}{\mathscr{K}}
\newcommand{\lscr}{\mathscr{L}}
\newcommand{\mscr}{\mathscr{M}}
\newcommand{\nscr}{\mathscr{N}}
\newcommand{\oscr}{\mathscr{O}}
\newcommand{\pscr}{\mathscr{P}}
\newcommand{\qscr}{\mathscr{Q}}
\newcommand{\rscr}{\mathscr{R}}
\newcommand{\sscr}{\mathscr{S}}
\newcommand{\tscr}{\mathscr{T}}
\newcommand{\uscr}{\mathscr{U}}
\newcommand{\vscr}{\mathscr{V}}
\newcommand{\wscr}{\mathscr{W}}
\newcommand{\xscr}{\mathscr{X}}
\newcommand{\yscr}{\mathscr{Y}}
\newcommand{\zscr}{\mathscr{Z}}

% Mathbb
\newcommand{\abb}{\mathbb{A}}
\newcommand{\bbb}{\mathbb{B}}
\newcommand{\cbb}{\mathbb{C}}
\newcommand{\dbb}{\mathbb{D}}
\newcommand{\ebb}{\mathbb{E}}
\newcommand{\fbb}{\mathbb{F}}
\newcommand{\gbb}{\mathbb{G}}
\newcommand{\hbb}{\mathbb{H}}
\newcommand{\ibb}{\mathbb{I}}
\newcommand{\jbb}{\mathbb{J}}
\newcommand{\kbb}{\mathbb{K}}
\newcommand{\lbb}{\mathbb{L}}
\newcommand{\mbb}{\mathbb{M}}
\newcommand{\nbb}{\mathbb{N}}
\newcommand{\obb}{\mathbb{O}}
\newcommand{\pbb}{\mathbb{P}}
\newcommand{\qbb}{\mathbb{Q}}
\newcommand{\rbb}{\mathbb{R}}
\newcommand{\sbb}{\mathbb{S}}
\newcommand{\tbb}{\mathbb{T}}
\newcommand{\ubb}{\mathbb{U}}
\newcommand{\vbb}{\mathbb{V}}
\newcommand{\wbb}{\mathbb{W}}
\newcommand{\xbb}{\mathbb{X}}
\newcommand{\ybb}{\mathbb{Y}}
\newcommand{\zbb}{\mathbb{Z}}


%===============================================================================
%=============================== SPECIAL SYMBOLS ===============================
%===============================================================================


% Orbit (group theory)
\newcommand{\orbit}{\mathcal{O}}
% Normal group
\newcommand{\normal}{\mathcal{N}}
% Indicator function
\newcommand{\indicator}{\mathbbm{1}}
% Laplace transform
\newcommand{\laplace}[1]{\mathcal{L}}
% Epsilon shorthand
\newcommand{\eps}{\varepsilon}
% Omega
\newcommand{\om}{\omega}
\newcommand{\Om}{\Omega}

%===============================================================================
%================================== OPERATORS ==================================
%===============================================================================


% Inverse exponent
\newcommand{\inv}[0]{^{-1}}
% Overline bar
\newcommand{\olsi}[1]{\,\overline{\!{#1}}}
% Less than or equal slanted
\newcommand{\seqs}{\leqslant}
% Greater or equal slanted
\newcommand{\geqs}{\geqslant}
% Subset or equal
\newcommand{\sub}{\subseteq}
% Proper subset
\newcommand{\prosub}{\subset}
% from
\newcommand{\from}{\leftarrow}

% Parantheses
\newcommand{\para}[1]{\left( #1 \right)}
% Curly Braces
\newcommand{\curl}[1]{\left\{ #1 \right\}}
% Brackets
\newcommand{\brac}[1]{\left[ #1 \right]}
% Angled Brackets
\newcommand{\ang}[1]{\left\langle #1 \right\rangle}
% Norm
\newcommand{\norm}[1]{\left\| #1 \right\|}

% Piece wise (use \\ between cases)
\newcommand{\piecewise}[1]{\begin{cases} #1 \end{cases}}

% Bold symbol shorthand
\newcommand{\bl}{\boldsymbol}

% Vertical space
\newcommand{\vs}[1]{\vspace{#1 pt}}
% Horizontal ertical space
\newcommand{\hs}[1]{\hspace{#1 pt}}


%===============================================================================
%============================== TEXT BASED SYMBOLS =============================
%===============================================================================


% Radians
\newcommand{\rad}{\text{rad}}
% Least Common Multiple
\newcommand{\lcm}{\text{lcm}}
% Automorphism
\newcommand{\Aut}{\text{Aut}}
% Variance
\newcommand{\var}{\text{Var}}
% Covariance
\newcommand{\cov}{\text{Cov}}
% Cofactor (matrix)
\newcommand{\cof}{\text{Cof}}
% Adjugate (matrix)
\newcommand{\adj}{\text{Adj}}
% Trace (matrix)
\newcommand{\tr}{\text{tr}}
% Standard deviation
\newcommand{\std}{\text{Std}}
% Correlation coefficient
\newcommand{\corr}{\text{Corr}}
% Sign
\newcommand{\sign}{\text{sign}}

% And text
\newcommand{\AND}{\text{ and }}
% Or text 
\newcommand{\OR}{\text{ or }}
% For text 
\newcommand{\FOR}{\text{ for }}
% If text
\newcommand{\IF}{\text{ if }}
% When text
\newcommand{\WHEN}{\text{ when }}
% Where text
\newcommand{\WHERE}{\text{ where }}
% Then text
\newcommand{\THEN}{\text{ then }}
% Such that text
\newcommand{\SUCHTHAT}{\text{ such that }}

% 1st
\newcommand{\st}[1]{#1^{\text{st}}}
% 2nd
\newcommand{\nd}[1]{#1^{\text{nd}}}
% 3rd
\newcommand{\rd}[1]{#1^{\text{rd}}}
% nth
\newcommand{\nth}[1]{#1^{\text{th}}}


%===============================================================================
%========================= PROBABILITY AND STATISTICS ==========================
%===============================================================================


% Permutation
\newcommand{\perm}[2]{{}^{#1}\!P_{#2}}
% Combination
\newcommand{\comb}[2]{{}^{#1}C_{#2}}

% Baye's risk
\newcommand{\risk}[1]{\mathscr{R}_{#1}}
% Baye's optimal risk
\newcommand{\riskOptimal}[1]{\mathscr{R}_{#1}^*}
% Baye's empirical risk
\newcommand{\riskEmpirical}[2]{\hat{\mathscr{R}}_{#1}^{#2}}


%===============================================================================
%=================================== CALCULUS ==================================
%===============================================================================


% d over d derivative
\newcommand{\dd}[2]{\dfrac{d#1}{d#2}}
% partial d over d derivative
\newcommand{\partialdd}[2]{\dfrac{\partial #1}{\partial #2}}
% delta d over d derivative
\newcommand{\deltadd}[2]{\dfrac{\Delta #1}{\Delta #2}}

% Integration between a and b
\newcommand{\integral}[4]{\int_{#1}^{#2} #3 \, #4}
% Integration in some space 
\newcommand{\boundIntegral}[2]{\int_{#1} #2 \, d#1}

% Limit
\newcommand{\limit}[3]{\lim_{#1 \to #2} #3}


%===============================================================================
%================================  BIG SYMBOLS  ================================
%===============================================================================


% Sum
\newcommand{\sumof}[2]{\sum_{#1}^{#2}}
% Product
\newcommand{\productof}[2]{\prod_{#1}^{#2}}
% Union
\newcommand{\unionof}[2]{\bigcup_{#1}^{#2}}
% Intersection
\newcommand{\intersectionof}[2]{\bigcap_{#1}^{#2}}
% Or
\newcommand{\orof}[2]{\bigvee_{#1}^{#2}}
% And
\newcommand{\andof}[2]{\bigwedge_{#1}^{#2}}


%===============================================================================
%=============================== LINEAR ALGEBRA ================================
%===============================================================================


% Bold vector arrow
\newcommand{\bv}[1]{\vec{\mathbf{#1}}}

% Matrix or vector (use // for column, & for row) 
% with brackets
\newcommand{\bmat}[1]{\begin{bmatrix} #1 \end{bmatrix}}
% Matrix or vector (use // for column, & for row) 
% with curved brackets
\newcommand{\pmat}[1]{\begin{pmatrix} #1 \end{pmatrix}}
% Matrix or vector (use // for column, & for row) 
% with lines on either side 
\newcommand{\lmat}[1]{\begin{vmatrix} #1 \end{vmatrix}}
% Matrix or vector (use // for column, & for row) 
% with curly braces
\newcommand{\cmat}[1]{\begin{Bmatrix} #1 \end{Bmatrix}}
% Matrix or vector (use // for column, & for row) 
% with no braces
\newcommand{\mat}[1]{\begin{matrix} #1 \end{matrix}}


%===============================================================================
%================================ LARGE OBJECTS ================================
%===============================================================================

% Multiple lines
\newcommand{\multiline}[1]{
\begin{align*}
    #1
\end{align*}
}

% Multiple lines with equation numbers
\newcommand{\eqmultiline}[1]{
\begin{align*}
    #1
\end{align*}
}

% Color
\newcommand{\colorText}[2]{
\begingroup
\color{#1}
    #2
\endgroup
}

% Centered figure
\newcommand{\centerFigure}[2]{
    \begin{figure}[h]
        \centering
            #1
        \caption{#2}
    \end{figure}
}

% Tikz figure
\newcommand{\tikzGraphic}[1]{
    \begin{center}
    \begin{tikzpicture}
        #1
    \end{tikzpicture}
    \end{center}
}

% Enumerate numbers (seperate by \item)
\newcommand{\numbers}{\textbf{\number*)}}

% Enumerate letters (seperate by \item)
\newcommand{\letters}{\textbf{\alph*)}}

\title{%
    \Huge Machine Learning \\
    \Large Assignment III
}
\date{2025-04-22}
\author{Michael Saba}

\begin{document}
\pagenumbering{gobble}
\maketitle
\newpage
\setlength{\parindent}{0pt}
\pagenumbering{arabic}

\subsection*{Ex 1}
\begin{enumerate}[label = \alph*)]
\item 
We know that Baye's optimal risk is:
\[ \rscr_\ell^*
= \inf_{f(x)} \; \E[\ell(f(X), Y)] 
= \inf_{f(x)} \; 
\E[\E[\ell(f(X), Y) \mid X]] \]
which is the minimization of 
a sum over $X$ first (outer sum)
and then $Y$ (inner sum):
\[ \rscr_\ell(f)
= \sum_{x_i} \Pb(X = x_i)
\sum_{y_i} \Pb(Y = y_i \mid X = x_i)
\cdot \ell(f(x_i), y_i) \]
In the lecture, we showed that the
optimal risk (best possible learning rule)
can be achieved by optimizing the
learning rule for each individual
input $x \in \xcal$:
\[ \rscr_\ell^* = \rscr_\ell(f^*) \]
That is
the global optimum is achieved
by making a series of locally optimal
choices. \\
Even though $X$ is continuous,
the marginal and conditional probabilities
both have a constant value in each
of the three intervals,
so we focus on making one choice in each. \\
We can optimize $f(x)$
by choosing the output $y \in \ycal$
with the highest conditional
probability:
\[ f^*(x)
= \piecewiseThree
{-1 \; \text{ if } x \in [0, 0.25]}
{1 \quad \text{ if } x \in [0.25, 0.75]}
{-1 \; \text{ if } x \in [0.75, 1]} \]
As we know,
since we are using the 0-1 loss function,
in the risk of the learning rule,
only the marginal distribution
of the outcomes we didn't choose
will appear (which is only
one in this case, as $Y$ it is binary):
\[ \rscr_\ell(f^)
= \sum_{x_i} \Pb(X = x_i)
\Pb(Y = \olsi{f(x_i)} \mid X = x_i) \]
where $\olsi{f(x_i)}$
is the outcome we did not choose. \\
Since $X$ is uniform on $[0, 1]$,
the probability that we have
$x$ in an interval of length $a$
is just $a$. \\
So:
\[ \rscr_\ell(f^*) = 
0.25 \cdot 0.2 + 0.5 \cdot 0.1
+ 0.25 \cdot 0.2 = 0.15 \]
So $\rscr_\ell^* = 0.15$.
\item 
If we look at our optimal classifier 
$f^*(x)$, we would see that it is
a step function, that is $-1$
from $0$ to $0.25$,
$1$ from $0.25$ to $0.75$,
and then $-1$ from $0.75$ to $1$. \\
However, this means we have two
decision boundaries (where $f(x)$
changes),
one at $0.25$ and the other at $0.75$. \\
We want to use a continuous
surrogate classifier,
which is then wrapped in the sign function
(in order to output $-1$ or $1$):
\[ f_{(w, b)}(x) = \text{sign}(wx + b) \]
However notice that the function is
linear, meaning that it can at most
intersect the $x$-axis one time,
which means it can change its decision
once;
but since we have $2$ decision boundaries,
this means that $f_{(w, b)}(x)$
can never be the optimal classifier,
but we can find the parameters that
optimize it. \\
Notice that if we have 
an increasing function that intersects
the $x$-axis at $0.25$,
or a descreasing function
that intersects
the $x$-axis at $0.75$,
the function will match the optimal
learning rule from $0$ to $0.75$
and $0.25$ to $1$ respectively. \\
If we calculate the risk of these
two functions, we get that they are
both optimal (but with a higher risk
than $\rscr_\ell^*$). \\
Each of these functions will have the
following formulas
(we only care about the $x$
intercept, not the slope,
which can be anything so long
as the monotonicity is preserved):
\[ f_1(x) = \dfrac{x-0.25}{a} \qquad
f_2(x) = \dfrac{0.75 - x}{a} \]
for any $a \in \R$ where $a > 0$. \\
We can then extract the optimal
parameters:
\[ (w^*, b^*) \in
\curl{\para{\dfrac{1}{a}, \dfrac{-0.25}{a}}}
\cup
\curl{\para{\dfrac{-1}{a}, \dfrac{0.75}{a}}}
\qquad \forall a \in \R, a > 0 \]
Now we note that:
\[ f_1(x)
= \piecewiseThree
{-1 \; \text{ if } x \in [0, 0.25]}
{1 \quad \text{ if } x \in [0.25, 0.75]}
{1 \quad \text{ if } x \in [0.75, 1]} \]
And:
\[ f_2(x)
= \piecewiseThree
{1 \quad \text{ if } x \in [0, 0.25]}
{1 \quad \text{ if } x \in [0.25, 0.75]}
{-1 \; \text{ if } x \in [0.75, 1]} \]
We can then plug these into
the risk formula,
which gives us that:
\[ \rscr_\ell(f_1)
= 0.25 \cdot 0.2 + 0.5 \cdot 0.1
+ 0.25 \cdot 0.8 = 0.3 \]
And
\[ \rscr_\ell(f_2)
= 0.25 \cdot 0.8 + 0.5 \cdot 0.1
+ 0.25 \cdot 0.2 = 0.3 \]
Which are both optimal
for $f_{(w, b)}$,
but have a higher risk than
Baye's optimal classifier. \\
One small note to keep in mind is that
in this case, it coincided that the
optimal functions $f_1$
and $f_2$ have the most
in common with Baye's optimal
classifier $f^*$
on the interval (75\% of it each). \\
However, it might not always be the case
that the optimal function
$f_{(w, b)}$
is the one that has the most
outputs in common,
since the outputs are each
weighted by the conditional and marginal
probabilities;
so we need to calculate the risk
(which takes into account
the weight of these probabilities),
and not just look for the function
$f_{(w, b)}$ where the most outputs
matc h $f^*$. 
\end{enumerate}

\begingroup
\color{red}
\subsection*{Correction}
\begin{enumerate}[label=\alph*)]
    \item
    Since $X$ is continuous,
    we could have instead used:
    \[ \int_{[0, 1]} 
    \Pb(Y = \olsi{f^*(x)} \mid X = x) \Pb(dx) \]
    If $X$ is uniform on $[a, b]$, the probability that
    $x$ is in some interval $[c, d] \in [a, b]$ is
    \[ \dfrac{d-c}{b-a} \]
    Basically, the proportion of the interval itself
    (so if $X$ is uniform on $[0, 2]$,
    the probability that it is in $[1, 2]$
    is $0.5$). \\
    Note that since $X$ is uniform on $[0, 1]$,
    then $\Pb(dx) = dx$.
    The probability that $X$ is in some small interval $dx$
    is just $\dfrac{dx}{1} = dx$. \\
    So:
    \[ \int_{[0, 1]} 
    \Pb(Y = f^*(x) \mid X = x) dx =
    \int_0^{0.25} 0.2 dx
    + \int_{0.25}^{0.75} 0.1 dx
    +\int_{0.75}^{0.1} 0.2 dx \]
    Note how the integrals contains
    only the conditional probability,
    and not the marginal probability of $x$,
    since it is uniformally distributed
    on $[0, 1]$:
    \[ \int_0^{0.25} 0.2 dx = 0.2 \cdot 0.25 \]
    Which is the same as:
    \[ \Pb(Y = -1 \mid X \in [0, 0.25]) 
    \cdot \Pb(X \in [0, 0.25]) \]
    This is the same result we got,
    but we used the integral, which is more accurate
    (in this case, it doesn't matter,
    since $x$ is uniform on the interval,
    so it acts as if it's discrete,
    but we usually need the intgeral). \\
    We can do the same for each other interval
    and then sum the results. \\
    However my method is fine for this exercise. \\
    Note that if $X$ was not uniform,
    we would have needed to include the marginal
    probability function in the integral
    as well. \\
    This can be done using
    the density function. Recall that:
    \[ E[f(X)] = \int f(x)\Pb_X(dx) 
    = \int f(x) p(x)dx \]
    for any measurable function $f(x)$. \\
    Note that this is actually why $\Pb(dx) = dx$;
    because $p(x) = 1$ when the distribution
    is uniform (on $[0, 1]$). \\
    \item 
    The solution is correct. \\
    Just make sure you recall that the
    classifier could have been optimal if it
    was quadratic (two decision boundaries
    match two x-intercepts). \\
    An easier way to write the boundaries:
    \[ b_1 = 0.25w_1 \]
    \[ b_2 = -0.25w_2 \]
    Don't forget that if the probabilities were
    different,
    the optimal line would differ;
    we can't just look at the largest interval
    we can match $f^*$; we have the consider
    the probability weight.
\end{enumerate}
\endgroup

\newpage

\subsection*{Ex2}
\begin{enumerate}[label=\alph*)]
    \item 
    We have an output space
    $\ycal = \{ 1, 2, \dots K \}$
    and a loss function
    \[ \ell: \R^k \times \ycal \to \R \]
    where
    \[ \ell(z, y)
    = -z_y + \log\para{\sum_{j=1}^K e^{z_j} } \]
    meaning that the classifer will output
    a $K$ dimensional vector,
    which is then compared to the class $y$
    by the loss function. \\
    We want to find the value of
    the conditional risk for some 
    $x \in \xcal$: \\
    \[ \E[\ell(z, Y) \mid X = x] \]
    The expected value here is just a sum
    over $Y$ weighetd by the conditional
    probability:
    \[ \sum_{k = 1}^K \ell(z, Y \mid X = x)
    \cdot \Pb(Y = k \mid X = x) \]
    \[ \sum_{k = 1}^K
    \para{-z_y + \log\para{\sum_{j=1}^K e^{z_j}}}
    \cdot \Pb(Y = k \mid X = x) \]
    \[ \sum_{k = 1}^K \para{-z_y
    \cdot \Pb(Y = k \mid X = x)}
    + \sum_{k = 1}^K
    \para{\log\para{\sum_{j=1}^K e^{z_j}}
    \cdot \Pb(Y = k \mid X = x)} \]
    \[ \sum_{k = 1}^K \para{-z_y
    \cdot \Pb(Y = k \mid X = x)}
    + \log\para{\sum_{j=1}^K e^{z_j}}
    \cdot
    \sum_{k = 1}^K \Pb(Y = k \mid X = x) \]
    Now we know that:
    \[ \sum_{k = 1}^K
    \Pb(Y = k \mid X = x) = 1 \]
    because $Y = 1$, $Y = 2$,
    $\dots$ $Y = K$
    partition the $Y$ space $\ycal$. \\
    So we now have :
    \[ \sum_{k = 1}^K \para{-z_y
    \cdot \Pb(Y = k \mid X = x)}
    + \log\para{\sum_{j=1}^K e^{z_j}} \]
    \item 
    We want to minimize the conditional
    risk, which means finding the gradient
    with respect to $z$
    (since $z \in \R^K$ is continuous)
    and then setting it to 0
    \[ \nabla \E[\ell(z, Y \mid X = x)] = 0 \]
    This will give us the optimal
    output $z^*$ for the input $x$. \\
    We can focus on each partial
    derivative individually:
    \[\partialdd{}{z_k}
    \para{ \sum_{k = 1}^K \para{-z_y
    \cdot \Pb(Y = k \mid X = x)}
    + \log\para{\sum_{j=1}^K e^{z_j}}} \]
    \[\partialdd{}{z_k}
    \para{ \sum_{k = 1}^K \para{-z_y
    \cdot \Pb(Y = k \mid X = x)}} 
    + \partialdd{}{z_k}
    \para{\log\para{\sum_{j=1}^K e^{z_j}}} \]
    \[ \dfrac{e^{z_k}}
    {\log\para{\sum_{j=1}^K e^{z_j}}}
    - \Pb(Y = k \mid X = x) \]
    We then set the gradient to $0$,
    which gives us:
    \[ \dfrac{e^{z_k}}
    {\log\para{\sum_{j=1}^K e^{z_j}}}
    - \Pb(Y = k \mid X = x) = 0 \]
    \[ \dfrac{e^{z_k}}
    {\log\para{\sum_{j=1}^K e^{z_j}}}
    = \Pb(Y = k \mid X = x) \]
    \[ e^{z_k}
    = \log\para{\sum_{j=1}^K e^{z_j}}
    \Pb(Y = k \mid X = x) \]
    \[ z_k
    = \log\para{\log\para{\sum_{j=1}^K e^{z_j}}
    \Pb(Y = k \mid X = x)} \]
    So:
    \[ z^*_k = \log\para{
    \alpha \Pb(Y = k \mid X = x)} \]
    where we have:
    \[ \alpha = \log\para{\sum_{j=1}^K e^{z_j}} \]
    Note that $\alpha > 0$,
    and that $\alpha$ is independent
    of our index $k$;
    it is the constant for all entries
    in $z^*$
    (so it's a constant in terms of the index $k$,
    but not $z$ as a whole). \\
    \item 
    We saw that for a fixed input $x$,
    the optimal output is given by:
    \[ g^*_k(x) = z^*_k = \log\para{
    \alpha \Pb(Y = k \mid X = x)} \]
    which is just a $\log$
    and scaling of $z^*_k$. \\
    Now we want to show that:
    \[ \arg \max_k g^*_k(x)
    =  \arg \max_k \Pb(Y = k \mid X = x) \]
    That is to say,
    the choice of $k$
    that gives us the largest
    entry $z^*_k$ of the optimal prediction,
    is the same $k$ as the one that
    maximizes the probability
    $\Pb(Y = k \mid X = x)$. \\
    Now, we have:
    \[ g^*(x)
    = \begin{bmatrix}
        \log\para{\alpha
        \Pb(Y = 1 \mid X = x)} \\
        \log\para{\alpha
        \Pb(Y = 2 \mid X = x)} \\ 
        \vdots \\
        \log\para{\alpha
        \Pb(Y = k \mid X = x)} \\ 
        \vdots \\
        \log\para{\alpha
        \Pb(Y = K \mid X = x)} \\ 
    \end{bmatrix} \]
    Notice that as we mentioned earlier,
    $\alpha$ is independent of the index $k$,
    so its value has no bearing on
    which entry is the largest. \\
    And since $\log$ is a monotonic
    increasing function,
    and since $\alpha > 0$
    (multiplying by it won't 
    flip the monotonicity),
    this means that the largest entry
    is the one that has index $k$ such that
    $\Pb(Y = k \mid X = x)$
    is the largest. \\
    Note that this is just $f^*(x)$,
    since by definition $f^*(x)$
    chooses $Y = k$ such that $\Pb(Y = k \mid X = x)$
    is the largest (in order to minimize risk,
    which has $1 -  \Pb(Y = k \mid X = x)$
    in it). \\
    This completes our proof. \\
    
    Notice why this makes sense. \\
    Our classifier usually output a single value,
    the predicted class label,
    which is then compared to the real label $y$
    by a loss function such as the 0-1 loss. \\
    However, in this case,
    if we have $K$ possible classes for $y$,
    we output a vector with $K$
    entries.
    The idea behind it is that each
    entry at inded $k$ in the vector would be
    some number that represents the confidence
    of the classifier that the index $k$
    is the correct label. \\
    The loss function we have takes as input 
    this vector, and checks the performance
    of the classifier;
    if the entry that has the largest value
    is the correct label, then the classifier
    did well. \\
    Now, in part $c)$,
    we needed to show that the largest entry
    in the optimal prediction $g^*(x)$,
    has the same index $k$
    as the $k$ that maximizes
    $\Pb(Y = k \mid X = x)$. \\
    And that makes perfect sense!
    It just means that the best classifier,
    when given an input $x$,
    outputs a vector
    where the entry with the highest
    confidence of being true corresponds
    the class with the highest probability
    of being true given $X = x$. \\
\end{enumerate}


\begingroup
\color{red}
\subsection*{Correction}
\begin{enumerate}[label=\alph*)]
    \item
    \item 
    The result is correct, but you should have
    taken the second derivative as well,
    to ensure that at this point, we do have a
    global minimizer, and not a global maximizer!
    Since both happen at $\nabla f = 0$. \\
    Basically, we find $\nabla^2 f$,
    the second derivative,
    and check its value at the minimzer;
    if it is positive $\nabla^2 f > 0$,
    it means that $\nabla f$
    is increasing, so it is a cup shape,
    which means convex (minimzer). \\
    If it is negative, then the result
    we found is a maximizer, not minimizer,
    since $\nabla f$ is descreasing,
    meaning it is a cap shape. \\
    Or we could mention the function is
    convex (if we know it is). \\
    Also, I was right,
    $\alpha$ being constant just means it doesn'take
    vary depending on the entry $k$.
\end{enumerate}
\endgroup

\newpage

\subsection*{Ex3}

\begin{enumerate}[label=\alph*)]
    \item 
    We want to optimize
    Baye's optimal predictor for
    some $x \in \xcal$:
    \[ f^*(x) = \min_{z \in \R^n}
    \E[\ell(z, Y) \mid X = x] \]
    where $\ell(z, y) = \| z - y\|^2$. \\
    We can use the gradient
    since $z \in \R^n$ is continuous. \\
    \[ f^*(x)
    = \E[ \| z - Y\|^2 \mid X = x] \]
    \[ f^*(x)
    = \E[z^2 + Y^2 -2zY \mid X = x] \]
    \[ f^*(x)
    = z^2 + \E[Y^2 \mid X = x]
    + -2z\E[Y \mid X = x] \]
    We can factor out $z$ because the expected
    value is that of the $Y$ random variable. \\
    We can then find the gradient
    (with respect to $z$)
    \[ \nabla z = 2z - 2\E[Y \mid X = x] \]
    Then setting it to $0$:
    \[ 2z - 2\E[Y \mid X = x] = 0\]
    \[ z = \E[Y \mid X = x] \]
    So the value of $z$,
    (which is the output of the classifier
    given $x$)
    that optimizes the conditional risk
    is $\E[Y \mid X = x]$.
    \item 
    Now we want to optimize
    Baye's optimal predictor for
    some $x \in \xcal$:
    \[ f^*(x) = \min_{c \in \R^+}
    \E[\ell(z, Y) \mid X = x] \]
    where $\ell(c, y) = \log(f(x)) + \dfrac{Y}{f(x)}$. \\
    We can use the derivative
    since $c \in \R^+$ is continuous. \\
    \[ f^*(x)
    = \E[ \log(c) + \dfrac{Y}{c} \mid X = x] \]
    \[ f^*(x)
    = \log(c) + \dfrac{1}{c}\E[Y \mid X = x] \]
    We can factor out $c$ because the expected
    value is that of the $Y$ random variable. \\
    We can then find the derivative
    (with respect to $c$):
    \[ \dd{f}{c} = \dfrac{1}{c} - 
    \dfrac{1}{c^2}\E[Y \mid X = x] \]
    Then setting it to $0$:
    \[ \dfrac{1}{c} - 
    \dfrac{1}{c^2}\E[Y \mid X = x] = 0 \]
    \[ c = \E[Y \mid X = x] \]
    So the value of $c$,
    (which is the output of the classifier
    given $x$)
    that optimizes the conditional risk
    is $\E[Y \mid X = x]$.
\end{enumerate}

\begingroup
\color{red}
\subsection*{Correction}
\begin{enumerate}[label=\alph*)]
    \item
    Again, you should either mention that
    the function is convex
    (sum of convex functions),
    or do the second derivative and show it
    is positive. \\
    Also don't forget that $z^2$
    in this case just means dot product of $z$
    with itself,
    and $zY$ means dot product too $\ang{z, Y}$.
    \item
    This wasn't really in the lecture,
    but we have additive noise
    and multiplicative noise:
    \[ Y = f(x) + \varepsilon \AND Y = \varepsilon f(x) \]
    As $f(x)$ is optimized,
    it usually tends to try to offset this noise,
    by cancelling it out.
    So again, $z$,
    our prediction, will include the optimal
    classificatin as well as either multiplicative
    or additive noises:
    \[ z = f^*(x) + \varepsilon
    \AND z = \varepsilon f^*(x) \]
    (Or $c$ for part b.). \\
    Note how in the additive case,
    if we use mean squared loss, the two noises
    ($Y$'s noise and $z$'s)
    would cancel out (since both are being added,
    and we have the minus). \\
    On the other hand, if we have multiplicative
    noise, and we use the
    second loss function (part b.),
    then the two noises would cancel out 
    since we have $Y$ divided by $c$. \\

\end{enumerate}
\endgroup

\newpage

\end{document}

