

\documentclass[12pt]{article}

\usepackage[margin=1in]{geometry}


%===============================================================================
%================================== PACKAGES ===================================
%===============================================================================

% For using float option H that places figures 
% exatcly where we want them
\usepackage{float}
% makes figure font bold
\usepackage{caption}
\captionsetup[figure]{labelfont=bf}
% For text generation
\usepackage{lipsum}
% For drawing
\usepackage{tikz}
% For smaller or equal sign and not divide sign
\usepackage{amssymb}
% For the diagonal fraction
\usepackage{xfrac}
% For enumerating exercise parts with letters instead of numbers
\usepackage{enumitem}
% For dfrac, which forces the fraction to be in display mode (large) e
% even in math mode (small)
\usepackage{amsmath}
% For degree sign
\usepackage{gensymb}
% For "\mathbb" macro
\usepackage{amsfonts}
% For positioning 
\usepackage{indentfirst}
\usetikzlibrary{shapes,positioning,fit,calc}
% for adjustwidth environment
\usepackage{changepage}
% for arrow on top
\usepackage{esvect}
% for mathbb 1
\usepackage{bbm}
% for mathsrc
\usepackage[mathscr]{eucal}
% For degree sign
\usepackage{gensymb}
% For quotes
\usepackage{csquotes}
% For vertical lines
\usepackage{mathtools}
% For cols
\usepackage{multicol}

% for tikz
\usepackage{pgfplots}
\pgfplotsset{compat=1.18}
\usepackage{amsmath}
\usepgfplotslibrary{groupplots}


%===============================================================================
%==================================== FONTS ====================================
%===============================================================================


% Mathcal
\newcommand{\acal}{\mathcal{A}}
\newcommand{\bcal}{\mathcal{B}}
\newcommand{\ccal}{\mathcal{C}}
\newcommand{\dcal}{\mathcal{D}}
\newcommand{\ecal}{\mathcal{E}}
\newcommand{\fcal}{\mathcal{F}}
\newcommand{\gcal}{\mathcal{G}}
\newcommand{\hcal}{\mathcal{H}}
\newcommand{\ical}{\mathcal{I}}
\newcommand{\jcal}{\mathcal{J}}
\newcommand{\kcal}{\mathcal{K}}
\newcommand{\lcal}{\mathcal{L}}
\newcommand{\mcal}{\mathcal{M}}
\newcommand{\ncal}{\mathcal{N}}
\newcommand{\ocal}{\mathcal{O}}
\newcommand{\pcal}{\mathcal{P}}
\newcommand{\qcal}{\mathcal{Q}}
\newcommand{\rcal}{\mathcal{R}}
\newcommand{\scal}{\mathcal{S}}
\newcommand{\tcal}{\mathcal{T}}
\newcommand{\ucal}{\mathcal{U}}
\newcommand{\vcal}{\mathcal{V}}
\newcommand{\wcal}{\mathcal{W}}
\newcommand{\xcal}{\mathcal{X}}
\newcommand{\ycal}{\mathcal{Y}}
\newcommand{\zcal}{\mathcal{Z}}

% Mathfrak
\newcommand{\afrak}{\mathfrak{A}}
\newcommand{\bfrak}{\mathfrak{B}}
\newcommand{\cfrak}{\mathfrak{C}}
\newcommand{\dfrak}{\mathfrak{D}}
\newcommand{\efrak}{\mathfrak{E}}
\newcommand{\ffrak}{\mathfrak{F}}
\newcommand{\gfrak}{\mathfrak{G}}
\newcommand{\hfrak}{\mathfrak{H}}
\newcommand{\ifrak}{\mathfrak{I}}
\newcommand{\jfrak}{\mathfrak{J}}
\newcommand{\kfrak}{\mathfrak{K}}
\newcommand{\lfrak}{\mathfrak{L}}
\newcommand{\mfrak}{\mathfrak{M}}
\newcommand{\nfrak}{\mathfrak{N}}
\newcommand{\ofrak}{\mathfrak{O}}
\newcommand{\pfrak}{\mathfrak{P}}
\newcommand{\qfrak}{\mathfrak{Q}}
\newcommand{\rfrak}{\mathfrak{R}}
\newcommand{\sfrak}{\mathfrak{S}}
\newcommand{\tfrak}{\mathfrak{T}}
\newcommand{\ufrak}{\mathfrak{U}}
\newcommand{\vfrak}{\mathfrak{V}}
\newcommand{\wfrak}{\mathfrak{W}}
\newcommand{\xfrak}{\mathfrak{X}}
\newcommand{\yfrak}{\mathfrak{Y}}
\newcommand{\zfrak}{\mathfrak{Z}}

% Mathscr
\newcommand{\ascr}{\mathscr{A}}
\newcommand{\bscr}{\mathscr{B}}
\newcommand{\cscr}{\mathscr{C}}
\newcommand{\dscr}{\mathscr{D}}
\newcommand{\escr}{\mathscr{E}}
\newcommand{\fscr}{\mathscr{F}}
\newcommand{\gscr}{\mathscr{G}}
\newcommand{\hscr}{\mathscr{H}}
\newcommand{\iscr}{\mathscr{I}}
\newcommand{\jscr}{\mathscr{J}}
\newcommand{\kscr}{\mathscr{K}}
\newcommand{\lscr}{\mathscr{L}}
\newcommand{\mscr}{\mathscr{M}}
\newcommand{\nscr}{\mathscr{N}}
\newcommand{\oscr}{\mathscr{O}}
\newcommand{\pscr}{\mathscr{P}}
\newcommand{\qscr}{\mathscr{Q}}
\newcommand{\rscr}{\mathscr{R}}
\newcommand{\sscr}{\mathscr{S}}
\newcommand{\tscr}{\mathscr{T}}
\newcommand{\uscr}{\mathscr{U}}
\newcommand{\vscr}{\mathscr{V}}
\newcommand{\wscr}{\mathscr{W}}
\newcommand{\xscr}{\mathscr{X}}
\newcommand{\yscr}{\mathscr{Y}}
\newcommand{\zscr}{\mathscr{Z}}

% Mathbb
\newcommand{\abb}{\mathbb{A}}
\newcommand{\bbb}{\mathbb{B}}
\newcommand{\cbb}{\mathbb{C}}
\newcommand{\dbb}{\mathbb{D}}
\newcommand{\ebb}{\mathbb{E}}
\newcommand{\fbb}{\mathbb{F}}
\newcommand{\gbb}{\mathbb{G}}
\newcommand{\hbb}{\mathbb{H}}
\newcommand{\ibb}{\mathbb{I}}
\newcommand{\jbb}{\mathbb{J}}
\newcommand{\kbb}{\mathbb{K}}
\newcommand{\lbb}{\mathbb{L}}
\newcommand{\mbb}{\mathbb{M}}
\newcommand{\nbb}{\mathbb{N}}
\newcommand{\obb}{\mathbb{O}}
\newcommand{\pbb}{\mathbb{P}}
\newcommand{\qbb}{\mathbb{Q}}
\newcommand{\rbb}{\mathbb{R}}
\newcommand{\sbb}{\mathbb{S}}
\newcommand{\tbb}{\mathbb{T}}
\newcommand{\ubb}{\mathbb{U}}
\newcommand{\vbb}{\mathbb{V}}
\newcommand{\wbb}{\mathbb{W}}
\newcommand{\xbb}{\mathbb{X}}
\newcommand{\ybb}{\mathbb{Y}}
\newcommand{\zbb}{\mathbb{Z}}


%===============================================================================
%=============================== SPECIAL SYMBOLS ===============================
%===============================================================================


% Orbit (group theory)
\newcommand{\orbit}{\mathcal{O}}
% Normal group
\newcommand{\normal}{\mathcal{N}}
% Indicator function
\newcommand{\indicator}{\mathbbm{1}}
% Laplace transform
\newcommand{\laplace}[1]{\mathcal{L}}
% Epsilon shorthand
\newcommand{\eps}{\varepsilon}
% Omega
\newcommand{\om}{\omega}
\newcommand{\Om}{\Omega}

%===============================================================================
%================================== OPERATORS ==================================
%===============================================================================


% Inverse exponent
\newcommand{\inv}[0]{^{-1}}
% Overline bar
\newcommand{\olsi}[1]{\,\overline{\!{#1}}}
% Less than or equal slanted
\newcommand{\seqs}{\leqslant}
% Greater or equal slanted
\newcommand{\geqs}{\geqslant}
% Subset or equal
\newcommand{\sub}{\subseteq}
% Proper subset
\newcommand{\prosub}{\subset}
% from
\newcommand{\from}{\leftarrow}

% Parantheses
\newcommand{\para}[1]{\left( #1 \right)}
% Curly Braces
\newcommand{\curl}[1]{\left\{ #1 \right\}}
% Brackets
\newcommand{\brac}[1]{\left[ #1 \right]}
% Angled Brackets
\newcommand{\ang}[1]{\left\langle #1 \right\rangle}
% Norm
\newcommand{\norm}[1]{\left\| #1 \right\|}

% Piece wise (use \\ between cases)
\newcommand{\piecewise}[1]{\begin{cases} #1 \end{cases}}

% Bold symbol shorthand
\newcommand{\bl}{\boldsymbol}

% Vertical space
\newcommand{\vs}[1]{\vspace{#1 pt}}
% Horizontal ertical space
\newcommand{\hs}[1]{\hspace{#1 pt}}


%===============================================================================
%============================== TEXT BASED SYMBOLS =============================
%===============================================================================


% Radians
\newcommand{\rad}{\text{rad}}
% Least Common Multiple
\newcommand{\lcm}{\text{lcm}}
% Automorphism
\newcommand{\Aut}{\text{Aut}}
% Variance
\newcommand{\var}{\text{Var}}
% Covariance
\newcommand{\cov}{\text{Cov}}
% Cofactor (matrix)
\newcommand{\cof}{\text{Cof}}
% Adjugate (matrix)
\newcommand{\adj}{\text{Adj}}
% Trace (matrix)
\newcommand{\tr}{\text{tr}}
% Standard deviation
\newcommand{\std}{\text{Std}}
% Correlation coefficient
\newcommand{\corr}{\text{Corr}}
% Sign
\newcommand{\sign}{\text{sign}}

% And text
\newcommand{\AND}{\text{ and }}
% Or text 
\newcommand{\OR}{\text{ or }}
% If text
\newcommand{\IF}{\text{ if }}
% When text
\newcommand{\WHEN}{\text{ when }}
% Then text
\newcommand{\THEN}{\text{ then }}

% 1st
\newcommand{\st}[1]{#1^{\text{st}}}
% 2nd
\newcommand{\nd}[1]{#1^{\text{nd}}}
% 3rd
\newcommand{\rd}[1]{#1^{\text{rd}}}
% nth
\newcommand{\nth}[1]{#1^{\text{th}}}


%===============================================================================
%========================= PROBABILITY AND STATISTICS ==========================
%===============================================================================


% Permutation
\newcommand{\perm}[2]{{}^{#1}\!P_{#2}}
% Combination
\newcommand{\comb}[2]{{}^{#1}C_{#2}}

% Baye's risk
\newcommand{\risk}[1]{\mathscr{R}_{#1}}
% Baye's optimal risk
\newcommand{\riskOptimal}[1]{\mathscr{R}_{#1}^*}
% Baye's empirical risk
\newcommand{\riskEmpirical}[2]{\hat{\mathscr{R}}_{#1}^{#2}}


%===============================================================================
%=================================== CALCULUS ==================================
%===============================================================================


% d over d derivative
\newcommand{\dd}[2]{\dfrac{d#1}{d#2}}
% partial d over d derivative
\newcommand{\partialdd}[2]{\dfrac{\partial #1}{\partial #2}}
% delta d over d derivative
\newcommand{\deltadd}[2]{\dfrac{\Delta #1}{\Delta #2}}

% Integration between a and b
\newcommand{\integral}[4]{\int_{#1}^{#2} #3 \, #4}
% Integration in some space 
\newcommand{\boundIntegral}[2]{\int_{#1} #2 \, d#1}

% Limit
\newcommand{\limit}[3]{\lim_{#1 \to #2} #3}


%===============================================================================
%================================  BIG SYMBOLS  ================================
%===============================================================================


% Sum
\newcommand{\sumof}[2]{\sum_{#1}^{#2}}
% Product
\newcommand{\productof}[2]{\prod_{#1}^{#2}}
% Union
\newcommand{\unionof}[2]{\bigcup_{#1}^{#2}}
% Intersection
\newcommand{\intersectionof}[2]{\bigcap_{#1}^{#2}}
% Or
\newcommand{\orof}[2]{\bigvee_{#1}^{#2}}
% And
\newcommand{\andof}[2]{\bigwedge_{#1}^{#2}}


%===============================================================================
%=============================== LINEAR ALGEBRA ================================
%===============================================================================


% Bold vector arrow
\newcommand{\bv}[1]{\vec{\mathbf{#1}}}

% Matrix or vector (use // for column, & for row) 
% with brackets
\newcommand{\bmat}[1]{\begin{bmatrix} #1 \end{bmatrix}}
% Matrix or vector (use // for column, & for row) 
% with curved brackets
\newcommand{\pmat}[1]{\begin{pmatrix} #1 \end{pmatrix}}
% Matrix or vector (use // for column, & for row) 
% with lines on either side 
\newcommand{\lmat}[1]{\begin{vmatrix} #1 \end{vmatrix}}
% Matrix or vector (use // for column, & for row) 
% with curly braces
\newcommand{\cmat}[1]{\begin{Bmatrix} #1 \end{Bmatrix}}
% Matrix or vector (use // for column, & for row) 
% with no braces
\newcommand{\mat}[1]{\begin{matrix} #1 \end{matrix}}


%===============================================================================
%================================ LARGE OBJECTS ================================
%===============================================================================

% Multiple lines
\newcommand{\multiline}[1]{
\begin{align*}
    #1
\end{align*}
}

% Multiple lines with equation numbers
\newcommand{\eqmultiline}[1]{
\begin{align*}
    #1
\end{align*}
}

% Color
\newcommand{\colorText}[2]{
\begingroup
\color{#1}
    #2
\endgroup
}

% Centered figure
\newcommand{\centerFigure}[2]{
    \begin{figure}[h]
        \centering
            #1
        \caption{#2}
    \end{figure}
}

% Tikz figure
\newcommand{\tikzGraphic}[1]{
    \begin{center}
    \begin{tikzpicture}
        #1
    \end{tikzpicture}
    \end{center}
}

% Enumerate numbers (seperate by \item)
\newcommand{\numbers}{\textbf{\number*)}}

% Enumerate letters (seperate by \item)
\newcommand{\letters}{\textbf{\alph*)}}

\title{
    \Huge Machine Learning \\
    \Large Assignment VI
}
\date{2025-05-24}
\author{Michael Saba}

\begin{document}
\pagenumbering{gobble}
\maketitle
\newpage
\setlength{\parindent}{0pt}
\pagenumbering{arabic}

\subsection*{Ex 1}

\begin{enumerate}[label = \letters]
\item 
    \begin{enumerate}
    \item 
        We have an equation:
        \[ x - y +2 = 0 \]
        One way to draw it would be to
        find the $y$ when $x= 0$,
        which is $y = 2$,
        and the $x$ when $y = 0$,
        which is $x = -2$.
        We then connect the points.
    \item
        The point $(2, 1)$ is in $H_-$,
        the point $(2, 5)$ is in $H_+$,
        and the point $(-3, -1)$
        is actually on the hyperplane. \\
        In order to determine, we can just
        plot them and notice if they lie within
        the $w$ normal direction, or we can
        just plug them in and check the sign
        of the function.
    \end{enumerate}
\item 
     \begin{enumerate}
    \item 
        We again just plot them.
    \item
        We just draw a line,
        which looks like it passes through
        $(1, 1)$ and $(3, 2)$.
    \item 
        Since we have points $(1, 1)$ and $(3, 2)$,
        we have:
        \[ y = ax + b \]
        where:
        \[ a = \dfrac{2-1}{3-1} = 0.5 \]
        We now have $y = 0.5x + b$,
        and if we plug in $(1, 1)$,
        we would find that $b = 0.5$. \\
        So all in all, we have:
        \[ 0.5x - y + 0.5 = 0 \]
        where $w = (0.5, -1)^T$, and $b = 0.5$.
    \item 
        For $(1,2)$, we have $-1$. \\
        For $(2,3)$, we have $-1.5$. \\
        For $(3,3)$, we have $-1$. \\
        For $(2,1)$, we have $0.5$. \\
        For $(3,0)$, we have $2$. \\
        For $(4,1)$, we have $1.5$. \\
        For $(5,2)$, we have $1$.
    \item 
        No, this is because the
        point $(3, 1)$ is surrounded by the cluster
        of points with negative labels,
        so there is no linear classifier
        that can seperate them.  
    \end{enumerate}
\end{enumerate}

\newpage

\subsection*{Ex 2}

We have points in $(\xcal, \ycal)$:
\[ ((x_1, x_2), y) \]
where:
\[ y = \piecewise{ -1 \quad 
\IF x_1^2 + x_2^2 \leq 1 \\ 1 \qquad \text{else}} \]
The input space is thus $\xcal = \rbb^2$,
and the output space is binary $\ycal = \{-1, 1\}$. \\
\begin{enumerate}[label = \letters]
\item 
    This dataset has data points with labels
    $-1$ that are inside a circle of radius $1$,
    and data points with labels $1$
    that are outside the circle.
    This is because:
    \[ x_1^2 + x_2^2 < 1 \]
    is the boundary that seperates the two classes,
    which describes the unit circle in $\rbb^2$. \\
    Notice that the seperator:
    \[ x_1^2 + x_2^2 - 1 = 0 \]
    does seperate the two clusters,
    but is not linear, since it has a square. \\
\item 
    We alreadt know that:
    \[ y = \piecewise{ -1 \quad 
    \IF x_1^2 + x_2^2 \leq 1 \\ 1 \qquad \text{else}} \]
    Which means that if $x_1^2 + x_2^2 \leq 1$,
    we can label the point $-1$. \\
    In this new feature map, we have:
    \[ \varphi(x) = (x_1, x_2, x_1^2 + x_2^2) \]
    We can ignore the first two coordinates,
    and just look at the third: notice
    that if we just focus on seperating
    the halfspaces where $\varphi_3(x)$
    is smaller or larger than $1$,
    we will have the seperator we want. \\
    We need a seperator:
    \[ w_1x_1 + w_2x_2 + w_3(x_1^2 + x_2^2) + b = 0 \]
    Recall that in part a, we proposed a seperator:
    \[ x_1^2 + x_2^2 - 1 = 0 \]
    which was not linear in $\rbb^2$,
    but is linear now, in $\rbb^3$,
    since we have a feature $x_1^2 + x_2^2$.
    So if we just set $w_1$ and $w_2$ to $0$,
    and $w_3$ to $1$, and $b$ to $-1$, we get:
    \[ \varphi_3(x) - 1 = 0 \]
    Notice that the sign is negative when
    $\varphi_3(x) \leq 1$,
    which means when $x_1^2 + x_2^2 \leq 1$. \\
    So, our boundary has weights $(0, 0, 1)^T$,
    and bias $-1$. \\
\item 
    No, we can't, we need to increase the
    dimensionality.
\item 
    Yes, we just update our feature map
    to accomodate ellipses:
    \[ \varphi(x) = \para{x_1, x_2, 
    \dfrac{x_1^2}{a^2} + \dfrac{x_2^2}{b^2}} \]
    We can then use the same seperator as
    part b.
\end{enumerate}

\newpage

\subsection*{Ex 3}

In the lecture, when attempting to solve:
\[ \Phi^T\Phi \theta = \Phi^T \bl{y} \]
We rewrote it as:
\[ \pmat{\bl{X} \\ \bl{1}^T}\pmat{\bl{X}^T & \bl{1}}
\theta = \pmat{\bl{X} \\ \bl{1}^T}\bl{y} \]
And then used substitution to derive
the equation for the weight $w$:
\[ (\bl{X}\bl{X}^T - \dfrac{1}{n}(\bl{X}\bl{1})(\bl{X}\bl{1})^T) w
= (\bl{X} - \dfrac{1}{n}\bl{X}\bl{1} \bl{1}^T) \bl{y} \]
We also divided our input data
and output data based on whether
the label was $-1$ or $1$:
\[ \bl{X} = \pmat{\bl{X}_- & \bl{X}_+} \]
\[ \bl{y} = \pmat{\bl{y}_- \\ \bl{y}_+}
= \pmat{-(\bl{1}_-) \\ \bl{1}_+} \]
\[ \bl{1} = \pmat{\bl{1}_- \\ \bl{1}_+} \]
Which tells us that:
\[ \bl{X}\bl{1} = \pmat{\bl{X}_- & \bl{X}_+}
\pmat{\bl{1}_- \\ \bl{1}_+} 
= \bl{X}_-\bl{1}_- + \bl{X}_+\bl{1}_+
= n_-\bl{m}_- + n_+\bl{m}_+ \]
We also have:
\[ \bl{X}\bl{X}^T = \bl{X}_-\bl{X}^T_- + \bl{X}_-\bl{X}^T_- \]
And:
\[ \bl{X}\bl{y} = \pmat{\bl{X}_- & \bl{X}_+}
\pmat{-(\bl{1}_-) \\ \bl{1}_+} 
= \bl{X}_+\bl{1}_+ - \bl{X}_-\bl{1}_-
= n_+\bl{m}_+ - n_-\bl{m}_- \]
Where $\bl{m}_-$ is the mean of $\bl{X}_-$,
and $\bl{m}_+$ is the mean of $\bl{X}_+$. \\
We then defined the within class covariance:
\[ \hat{\Sigma}_w 
= (\bl{X}_- - \bl{m}_-\bl{1}^T_-)(\bl{X}_- 
- \bl{m}_-\bl{1}^T_-)^T
+ (\bl{X}_+ - \bl{m}_+\bl{1}^T_+)(\bl{X}_+ 
- \bl{m}_+\bl{1}^T_+)^T \]
Which we can expand and write as:
\[ \hat{\Sigma}_w = \bl{X}_+\bl{X}^T_+
+ \bl{X}_-\bl{X}^T_- -n_+\bl{m}_+\bl{m}^T_+
-n_-\bl{m}_-\bl{m}^T_- \]
And the between class covariance:
\[ \hat{\Sigma}_b = (\bl{m}_+ 
- \bl{m}_-)(\bl{m}_+ - \bl{m}_-)^T \]
All of this we did in the lecture. \\
\begin{enumerate}[label = \letters]
    \item 
    We can simplify our equation
    using the definitions we have:
    \[ (\bl{X}\bl{X}^T - \dfrac{1}{n}(\bl{X}\bl{1})(\bl{X}\bl{1})^T) w
    = (\bl{X} - \dfrac{1}{n}\bl{X}\bl{1} \bl{1}^T) \bl{y} \]
    First we will do the right hand side:
    \[ (\bl{X} - \dfrac{1}{n}\bl{X}\bl{1} \bl{1}^T) \bl{y} \]
    \[ \bl{X}\bl{y} - \dfrac{1}{n}(\bl{X}\bl{1}) \bl{1}^T \bl{y} \]
    \[ n_+\bl{m}_+ - n_-\bl{m}_- 
    - \dfrac{1}{n}(n_-\bl{m}_- + n_+\bl{m}_+) 
    \pmat{\bl{1}_- & \bl{1}_+}\pmat{-(\bl{1}_-) \\ \bl{1}_+} \]
    Note that:
    \[ \pmat{\bl{1}_- & \bl{1}_+}\pmat{-(\bl{1}_-) \\ \bl{1}_+}
    = -n_- + n_+ \]
    Since we are just summing $1$s and $-1$s. So:
    \[ n_+\bl{m}_+ - n_-\bl{m}_-
    - \dfrac{1}{n}(n_-\bl{m}_- + n_+\bl{m}_+)(-n_- + n_+) \]
    \[ n_+\bl{m}_+ - n_-\bl{m}_-
    - \dfrac{1}{n}({n_+}^2\bl{m}_+ - {n_-}^2\bl{m}_-
    + (n_+n_-)\bl{\bl{m_-}} - (n_+n_-)\bl{\bl{m_+}} ) \]
    We can then group the $\bl{\bl{m_-}}$ and $\bl{\bl{m_+}}$ elements:
    \[ \bl{\bl{m_+}} \para{n_+ - \dfrac{{n_+}^2}{n} + \dfrac{n_+n_-}{n}}
    + \bl{\bl{m_-}} \para{-n_- + \dfrac{{n_-}^2}{n} - \dfrac{n_+n_-}{n}} \]
    \[ \bl{\bl{m_+}} \para{\dfrac{nn_+ - {n_+}^2 + n_+n_-}{n}}
    + \bl{\bl{m_-}} \para{\dfrac{-nn_- + {n_-}^2 - n_+n_-}{n}} \]
    Now note that $n = n_+ + n_-$ for obvious reasons, so:
    \[ \bl{\bl{m_+}} \para{\dfrac{(n_+ + n_-)n_+ - {n_+}^2 + n_+n_-}{n}}
    + \bl{\bl{m_-}} \para{\dfrac{-(n_+ + n_-)n_- + {n_-}^2 - n_+n_-}{n}} \]
    \[ \bl{\bl{m_+}} \para{\dfrac{{n_+}^2 + n_+n_- - {n_+}^2 + n_+n_-}{n}}
    + \bl{\bl{m_-}} \para{\dfrac{-{n_-}^2 - n_+n_- + {n_-}^2 - n_+n_-}{n}} \]
    \[ \bl{\bl{m_+}} \para{\dfrac{2n_+n_-}{n}}
    + \bl{\bl{m_-}} \para{\dfrac{-2n_+n_-}{n}} \]
    \[ (\bl{\bl{m_+}} - \bl{\bl{m_-}})\dfrac{2n_+n_-}{n} \]
    Which is the right hand side we wanted. \\

    Now for the left hand side, we have:
    \[ \bl{X}\bl{X}^T - 
    \dfrac{1}{n}(\bl{X}\bl{1})(\bl{X}\bl{1})^T \]
    \[ \bl{X}_-\bl{X}^T_- + \bl{X}_-\bl{X}^T_-
    - \dfrac{1}{n}( n_-\bl{m}_- + n_+\bl{m}_+)
    ( n_-\bl{m}_- + n_+\bl{m}_+)^T \]
    \[ \bl{X}_-\bl{X}^T_- + \bl{X}_-\bl{X}^T_-
    - \dfrac{{n_-}^2\bl{m}_-\bl{m}_-^T}{n}
    - \dfrac{{n_+}^2\bl{m}_+\bl{m}_+^T}{n}
    - \dfrac{{n_+n_-}}{n}(\bl{m_-}\bl{m_+}^T + \bl{m_+}\bl{m_-}^T) \]
    Now recall that since $n = n_+ + n_-$, this means that:
    \[ \dfrac{n_+}{n} = \dfrac{n - n_-}{n} = 1 - \dfrac{n_-}{n}
    \quad \AND \quad 
    \dfrac{n_-}{n} = \dfrac{n - n_+}{n} = 1 - \dfrac{n_+}{n} \]
    So these terms can be rewritten as:
    \[-\dfrac{{n_-}^2\bl{m}_-\bl{m}_-^T}{n}
    - \dfrac{{n_+}^2\bl{m}_+\bl{m}_+^T}{n} \]
    \[-\dfrac{n_-}{n}(n_-\bl{m}_-\bl{m}_-^T)
    - \dfrac{n_+}{n}(n_+\bl{m}_+\bl{m}_+^T) \]
    \[(\dfrac{n_+}{n} - 1)(n_-\bl{m}_-\bl{m}_-^T)
    +(\dfrac{n_-}{n} - 1)(n_+\bl{m}_+\bl{m}_+^T) \]
    \[-(n_-\bl{m}_-\bl{m}_-^T) - (n_+\bl{m}_+\bl{m}_+^T)
    + \dfrac{n_-n_+}{n}\bl{m}_+\bl{m}_+^T 
    + \dfrac{n_-n_+}{n}\bl{m}_-\bl{m}_-^T \]
    Now placing this back in the equation:
    \multiline{& \bl{X}_-\bl{X}^T_- + \bl{X}_-\bl{X}^T_-
    -(n_-\bl{m}_-\bl{m}_-^T) - (n_+\bl{m}_+\bl{m}_+^T) \\
    & + \dfrac{n_-n_+}{n}\bl{m}_+\bl{m}_+^T 
    + \dfrac{n_-n_+}{n}\bl{m}_-\bl{m}_-^T
    - \dfrac{{n_+n_-}}{n}(\bl{m_-}\bl{m_+}^T + \bl{m_+}\bl{m_-}^T)}
    We notice
    that the within-class covariance $\hat{\Sigma}_w$ emerges:
    \[ \hat{\Sigma}_w + \dfrac{n_-n_+}{n}\bl{m}_+\bl{m}_+^T 
    + \dfrac{n_-n_+}{n}\bl{m}_-\bl{m}_-^T
    - \dfrac{{n_+n_-}}{n}(\bl{m_-}\bl{m_+}^T + \bl{m_+}\bl{m_-}^T)\]
    We can then factor out $\dfrac{n_-n_+}{n}$:
    \[ \hat{\Sigma}_w + \dfrac{n_-n_+}{n}(\bl{m}_+\bl{m}_+^T 
    + \bl{m}_-\bl{m}_-^T -\bl{m_-}\bl{m_+}^T - \bl{m_+}\bl{m_-}^T) \]
    \[ \hat{\Sigma}_w + \dfrac{n_-n_+}{n}
    (\bl{m_+} - \bl{m_-})(\bl{m_+} - \bl{m_-})^T \]
    Which we recognize as the between-class covariance:
    \[ \hat{\Sigma}_w + \dfrac{n_-n_+}{n} \hat{\Sigma}_b \]
    And this is the left hand side we wanted. 
    \item 
    We have:
    \[ \hat{\Sigma}_b w 
    = (\bl{m_+} - \bl{m_-})(\bl{m_+} - \bl{m_-})^T w 
    = (\bl{m_+} - \bl{m_-})((\bl{m_+} - \bl{m_-})^T w) \]
    Notice that $(\bl{m_+} - \bl{m_-})^T w$
    is a dot product, and just a scalar,
    so out result is:
    \[ \beta (\bl{m_+} - \bl{m_-}) \]
    where $\beta$ is a scalar. \\
    So when we add $\hat{\Sigma}_b w $
    to the right hand side, we are just rescaling it,
    since the right hand side itself is:
    \[ (\bl{\bl{m_+}} - \bl{\bl{m_-}})\dfrac{2n_+n_-}{n} \]
    Which is in the direction $(\bl{\bl{m_+}} - \bl{\bl{m_-}})$.
    \item 
    We have:
    \[ \para{\hat{\Sigma}_w + 
    \dfrac{n_-n_+}{n} \hat{\Sigma}_b} w
    = \dfrac{2n_+n_-}{n}(\bl{\bl{m_+}} - \bl{\bl{m_-}}) \]
    We can plug in:
    \[ w = \alpha 
    \hat{\Sigma}_w\inv(\bl{\bl{m_+}} - \bl{\bl{m_-}})\]
    And get:
    \[ \alpha\para{\hat{\Sigma}_w + 
    \dfrac{n_-n_+}{n} \hat{\Sigma}_b}
    \hat{\Sigma}_w\inv(\bl{\bl{m_+}} - \bl{\bl{m_-}})
    = \dfrac{2n_+n_-}{n}(\bl{\bl{m_+}} - \bl{\bl{m_-}}) \]
    We can simplify:
    \[ \alpha(\bl{\bl{m_+}} - \bl{\bl{m_-}}) 
    + \alpha\dfrac{n_-n_+}{n} 
    \hat{\Sigma}_b\hat{\Sigma}_w\inv
    (\bl{\bl{m_+}} - \bl{\bl{m_-}}) 
    = \dfrac{2n_+n_-}{n}(\bl{\bl{m_+}} - \bl{\bl{m_-}}) \]
    \[ \alpha(I + \dfrac{n_-n_+}{n} 
    \hat{\Sigma}_b\hat{\Sigma}_w\inv)
    (\bl{\bl{m_+}} - \bl{\bl{m_-}}) 
    = \dfrac{2n_+n_-}{n}(\bl{\bl{m_+}} - \bl{\bl{m_-}}) \]
    Since $(I + \dfrac{n_-n_+}{n} 
    \hat{\Sigma}_b\hat{\Sigma}_w\inv)
    (\bl{\bl{m_+}} - \bl{\bl{m_-}})$ is a vector,
    we can't move it to the other side,
    so we have to multiply both sides
    the transpose so we get a scalar:
    \[ \alpha (\bl{\bl{m_+}} - \bl{\bl{m_-}})^T
    (I + \dfrac{n_-n_+}{n}\hat{\Sigma}_b\hat{\Sigma}_w\inv)
    (\bl{\bl{m_+}} - \bl{\bl{m_-}}) 
    = \dfrac{2n_+n_-}{n}(\bl{\bl{m_+}} - \bl{\bl{m_-}})^T
    (\bl{\bl{m_+}} - \bl{\bl{m_-}}) \]
    \[ \alpha
    = \dfrac{2n_+n_- (\bl{\bl{m_+}} - \bl{\bl{m_-}})^T
    (\bl{\bl{m_+}} - \bl{\bl{m_-}})}{n(\bl{\bl{m_+}} - \bl{\bl{m_-}})^T
    (I + \dfrac{n_-n_+}{n}\hat{\Sigma}_b\hat{\Sigma}_w\inv)
    (\bl{\bl{m_+}} - \bl{\bl{m_-}})} \]
    \[ \alpha
    = \dfrac{2n_+n_- (\bl{\bl{m_+}} - \bl{\bl{m_-}})^T
    (\bl{\bl{m_+}} - \bl{\bl{m_-}}) }{(\bl{\bl{m_+}} - \bl{\bl{m_-}})^T
    (nI + n_-n_+\hat{\Sigma}_b\hat{\Sigma}_w\inv)
    (\bl{\bl{m_+}} - \bl{\bl{m_-}})} \]
    \item
    Not sure how to solve this.
\end{enumerate}

\newpage

\end{document}