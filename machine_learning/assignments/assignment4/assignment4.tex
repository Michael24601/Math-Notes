

\documentclass[12pt]{article}

\usepackage[margin=1in]{geometry}


%===============================================================================
%================================== PACKAGES ===================================
%===============================================================================

% For using float option H that places figures 
% exatcly where we want them
\usepackage{float}
% makes figure font bold
\usepackage{caption}
\captionsetup[figure]{labelfont=bf}
% For text generation
\usepackage{lipsum}
% For drawing
\usepackage{tikz}
% For smaller or equal sign and not divide sign
\usepackage{amssymb}
% For the diagonal fraction
\usepackage{xfrac}
% For enumerating exercise parts with letters instead of numbers
\usepackage{enumitem}
% For dfrac, which forces the fraction to be in display mode (large) e
% even in math mode (small)
\usepackage{amsmath}
% For degree sign
\usepackage{gensymb}
% For "\mathbb" macro
\usepackage{amsfonts}
% For positioning 
\usepackage{indentfirst}
\usetikzlibrary{shapes,positioning,fit,calc}
% for adjustwidth environment
\usepackage{changepage}
% for arrow on top
\usepackage{esvect}
% for mathbb 1
\usepackage{bbm}
% for mathsrc
\usepackage[mathscr]{eucal}
% For degree sign
\usepackage{gensymb}
% For quotes
\usepackage{csquotes}
% For vertical lines
\usepackage{mathtools}
% For cols
\usepackage{multicol}

% for tikz
\usepackage{pgfplots}
\pgfplotsset{compat=1.18}
\usepackage{amsmath}
\usepgfplotslibrary{groupplots}


%===============================================================================
%==================================== FONTS ====================================
%===============================================================================


% Mathcal
\newcommand{\acal}{\mathcal{A}}
\newcommand{\bcal}{\mathcal{B}}
\newcommand{\ccal}{\mathcal{C}}
\newcommand{\dcal}{\mathcal{D}}
\newcommand{\ecal}{\mathcal{E}}
\newcommand{\fcal}{\mathcal{F}}
\newcommand{\gcal}{\mathcal{G}}
\newcommand{\hcal}{\mathcal{H}}
\newcommand{\ical}{\mathcal{I}}
\newcommand{\jcal}{\mathcal{J}}
\newcommand{\kcal}{\mathcal{K}}
\newcommand{\lcal}{\mathcal{L}}
\newcommand{\mcal}{\mathcal{M}}
\newcommand{\ncal}{\mathcal{N}}
\newcommand{\ocal}{\mathcal{O}}
\newcommand{\pcal}{\mathcal{P}}
\newcommand{\qcal}{\mathcal{Q}}
\newcommand{\rcal}{\mathcal{R}}
\newcommand{\scal}{\mathcal{S}}
\newcommand{\tcal}{\mathcal{T}}
\newcommand{\ucal}{\mathcal{U}}
\newcommand{\vcal}{\mathcal{V}}
\newcommand{\wcal}{\mathcal{W}}
\newcommand{\xcal}{\mathcal{X}}
\newcommand{\ycal}{\mathcal{Y}}
\newcommand{\zcal}{\mathcal{Z}}

% Mathfrak
\newcommand{\afrak}{\mathfrak{A}}
\newcommand{\bfrak}{\mathfrak{B}}
\newcommand{\cfrak}{\mathfrak{C}}
\newcommand{\dfrak}{\mathfrak{D}}
\newcommand{\efrak}{\mathfrak{E}}
\newcommand{\ffrak}{\mathfrak{F}}
\newcommand{\gfrak}{\mathfrak{G}}
\newcommand{\hfrak}{\mathfrak{H}}
\newcommand{\ifrak}{\mathfrak{I}}
\newcommand{\jfrak}{\mathfrak{J}}
\newcommand{\kfrak}{\mathfrak{K}}
\newcommand{\lfrak}{\mathfrak{L}}
\newcommand{\mfrak}{\mathfrak{M}}
\newcommand{\nfrak}{\mathfrak{N}}
\newcommand{\ofrak}{\mathfrak{O}}
\newcommand{\pfrak}{\mathfrak{P}}
\newcommand{\qfrak}{\mathfrak{Q}}
\newcommand{\rfrak}{\mathfrak{R}}
\newcommand{\sfrak}{\mathfrak{S}}
\newcommand{\tfrak}{\mathfrak{T}}
\newcommand{\ufrak}{\mathfrak{U}}
\newcommand{\vfrak}{\mathfrak{V}}
\newcommand{\wfrak}{\mathfrak{W}}
\newcommand{\xfrak}{\mathfrak{X}}
\newcommand{\yfrak}{\mathfrak{Y}}
\newcommand{\zfrak}{\mathfrak{Z}}

% Mathscr
\newcommand{\ascr}{\mathscr{A}}
\newcommand{\bscr}{\mathscr{B}}
\newcommand{\cscr}{\mathscr{C}}
\newcommand{\dscr}{\mathscr{D}}
\newcommand{\escr}{\mathscr{E}}
\newcommand{\fscr}{\mathscr{F}}
\newcommand{\gscr}{\mathscr{G}}
\newcommand{\hscr}{\mathscr{H}}
\newcommand{\iscr}{\mathscr{I}}
\newcommand{\jscr}{\mathscr{J}}
\newcommand{\kscr}{\mathscr{K}}
\newcommand{\lscr}{\mathscr{L}}
\newcommand{\mscr}{\mathscr{M}}
\newcommand{\nscr}{\mathscr{N}}
\newcommand{\oscr}{\mathscr{O}}
\newcommand{\pscr}{\mathscr{P}}
\newcommand{\qscr}{\mathscr{Q}}
\newcommand{\rscr}{\mathscr{R}}
\newcommand{\sscr}{\mathscr{S}}
\newcommand{\tscr}{\mathscr{T}}
\newcommand{\uscr}{\mathscr{U}}
\newcommand{\vscr}{\mathscr{V}}
\newcommand{\wscr}{\mathscr{W}}
\newcommand{\xscr}{\mathscr{X}}
\newcommand{\yscr}{\mathscr{Y}}
\newcommand{\zscr}{\mathscr{Z}}

% Mathbb
\newcommand{\abb}{\mathbb{A}}
\newcommand{\bbb}{\mathbb{B}}
\newcommand{\cbb}{\mathbb{C}}
\newcommand{\dbb}{\mathbb{D}}
\newcommand{\ebb}{\mathbb{E}}
\newcommand{\fbb}{\mathbb{F}}
\newcommand{\gbb}{\mathbb{G}}
\newcommand{\hbb}{\mathbb{H}}
\newcommand{\ibb}{\mathbb{I}}
\newcommand{\jbb}{\mathbb{J}}
\newcommand{\kbb}{\mathbb{K}}
\newcommand{\lbb}{\mathbb{L}}
\newcommand{\mbb}{\mathbb{M}}
\newcommand{\nbb}{\mathbb{N}}
\newcommand{\obb}{\mathbb{O}}
\newcommand{\pbb}{\mathbb{P}}
\newcommand{\qbb}{\mathbb{Q}}
\newcommand{\rbb}{\mathbb{R}}
\newcommand{\sbb}{\mathbb{S}}
\newcommand{\tbb}{\mathbb{T}}
\newcommand{\ubb}{\mathbb{U}}
\newcommand{\vbb}{\mathbb{V}}
\newcommand{\wbb}{\mathbb{W}}
\newcommand{\xbb}{\mathbb{X}}
\newcommand{\ybb}{\mathbb{Y}}
\newcommand{\zbb}{\mathbb{Z}}


%===============================================================================
%=============================== SPECIAL SYMBOLS ===============================
%===============================================================================


% Orbit (group theory)
\newcommand{\orbit}{\mathcal{O}}
% Normal group
\newcommand{\normal}{\mathcal{N}}
% Indicator function
\newcommand{\indicator}{\mathbbm{1}}
% Laplace transform
\newcommand{\laplace}[1]{\mathcal{L}}
% Epsilon shorthand
\newcommand{\eps}{\varepsilon}
% Omega
\newcommand{\om}{\omega}
\newcommand{\Om}{\Omega}

%===============================================================================
%================================== OPERATORS ==================================
%===============================================================================


% Inverse exponent
\newcommand{\inv}[0]{^{-1}}
% Overline bar
\newcommand{\olsi}[1]{\,\overline{\!{#1}}}
% Less than or equal slanted
\newcommand{\seqs}{\leqslant}
% Greater or equal slanted
\newcommand{\geqs}{\geqslant}
% Subset or equal
\newcommand{\sub}{\subseteq}
% Proper subset
\newcommand{\prosub}{\subset}
% from
\newcommand{\from}{\leftarrow}

% Parantheses
\newcommand{\para}[1]{\left( #1 \right)}
% Curly Braces
\newcommand{\curl}[1]{\left\{ #1 \right\}}
% Brackets
\newcommand{\brac}[1]{\left[ #1 \right]}
% Angled Brackets
\newcommand{\ang}[1]{\left\langle #1 \right\rangle}
% Norm
\newcommand{\norm}[1]{\left\| #1 \right\|}

% Piece wise (use \\ between cases)
\newcommand{\piecewise}[1]{\begin{cases} #1 \end{cases}}

% Bold symbol shorthand
\newcommand{\bl}{\boldsymbol}

% Vertical space
\newcommand{\vs}[1]{\vspace{#1 pt}}
% Horizontal ertical space
\newcommand{\hs}[1]{\hspace{#1 pt}}


%===============================================================================
%============================== TEXT BASED SYMBOLS =============================
%===============================================================================


% Radians
\newcommand{\rad}{\text{rad}}
% Least Common Multiple
\newcommand{\lcm}{\text{lcm}}
% Automorphism
\newcommand{\Aut}{\text{Aut}}
% Variance
\newcommand{\var}{\text{Var}}
% Covariance
\newcommand{\cov}{\text{Cov}}
% Cofactor (matrix)
\newcommand{\cof}{\text{Cof}}
% Adjugate (matrix)
\newcommand{\adj}{\text{Adj}}
% Trace (matrix)
\newcommand{\tr}{\text{tr}}
% Standard deviation
\newcommand{\std}{\text{Std}}
% Correlation coefficient
\newcommand{\corr}{\text{Corr}}
% Sign
\newcommand{\sign}{\text{sign}}

% And text
\newcommand{\AND}{\text{ and }}
% Or text 
\newcommand{\OR}{\text{ or }}
% For text 
\newcommand{\FOR}{\text{ for }}
% If text
\newcommand{\IF}{\text{ if }}
% When text
\newcommand{\WHEN}{\text{ when }}
% Where text
\newcommand{\WHERE}{\text{ where }}
% Then text
\newcommand{\THEN}{\text{ then }}
% Such that text
\newcommand{\SUCHTHAT}{\text{ such that }}

% 1st
\newcommand{\st}[1]{#1^{\text{st}}}
% 2nd
\newcommand{\nd}[1]{#1^{\text{nd}}}
% 3rd
\newcommand{\rd}[1]{#1^{\text{rd}}}
% nth
\newcommand{\nth}[1]{#1^{\text{th}}}


%===============================================================================
%========================= PROBABILITY AND STATISTICS ==========================
%===============================================================================


% Permutation
\newcommand{\perm}[2]{{}^{#1}\!P_{#2}}
% Combination
\newcommand{\comb}[2]{{}^{#1}C_{#2}}

% Baye's risk
\newcommand{\risk}[1]{\mathscr{R}_{#1}}
% Baye's optimal risk
\newcommand{\riskOptimal}[1]{\mathscr{R}_{#1}^*}
% Baye's empirical risk
\newcommand{\riskEmpirical}[2]{\hat{\mathscr{R}}_{#1}^{#2}}


%===============================================================================
%=================================== CALCULUS ==================================
%===============================================================================


% d over d derivative
\newcommand{\dd}[2]{\dfrac{d#1}{d#2}}
% partial d over d derivative
\newcommand{\partialdd}[2]{\dfrac{\partial #1}{\partial #2}}
% delta d over d derivative
\newcommand{\deltadd}[2]{\dfrac{\Delta #1}{\Delta #2}}

% Integration between a and b
\newcommand{\integral}[4]{\int_{#1}^{#2} #3 \, #4}
% Integration in some space 
\newcommand{\boundIntegral}[2]{\int_{#1} #2 \, d#1}

% Limit
\newcommand{\limit}[3]{\lim_{#1 \to #2} #3}


%===============================================================================
%================================  BIG SYMBOLS  ================================
%===============================================================================


% Sum
\newcommand{\sumof}[2]{\sum_{#1}^{#2}}
% Product
\newcommand{\productof}[2]{\prod_{#1}^{#2}}
% Union
\newcommand{\unionof}[2]{\bigcup_{#1}^{#2}}
% Intersection
\newcommand{\intersectionof}[2]{\bigcap_{#1}^{#2}}
% Or
\newcommand{\orof}[2]{\bigvee_{#1}^{#2}}
% And
\newcommand{\andof}[2]{\bigwedge_{#1}^{#2}}


%===============================================================================
%=============================== LINEAR ALGEBRA ================================
%===============================================================================


% Bold vector arrow
\newcommand{\bv}[1]{\vec{\mathbf{#1}}}

% Matrix or vector (use // for column, & for row) 
% with brackets
\newcommand{\bmat}[1]{\begin{bmatrix} #1 \end{bmatrix}}
% Matrix or vector (use // for column, & for row) 
% with curved brackets
\newcommand{\pmat}[1]{\begin{pmatrix} #1 \end{pmatrix}}
% Matrix or vector (use // for column, & for row) 
% with lines on either side 
\newcommand{\lmat}[1]{\begin{vmatrix} #1 \end{vmatrix}}
% Matrix or vector (use // for column, & for row) 
% with curly braces
\newcommand{\cmat}[1]{\begin{Bmatrix} #1 \end{Bmatrix}}
% Matrix or vector (use // for column, & for row) 
% with no braces
\newcommand{\mat}[1]{\begin{matrix} #1 \end{matrix}}


%===============================================================================
%================================ LARGE OBJECTS ================================
%===============================================================================

% Multiple lines
\newcommand{\multiline}[1]{
\begin{align*}
    #1
\end{align*}
}

% Multiple lines with equation numbers
\newcommand{\eqmultiline}[1]{
\begin{align*}
    #1
\end{align*}
}

% Color
\newcommand{\colorText}[2]{
\begingroup
\color{#1}
    #2
\endgroup
}

% Centered figure
\newcommand{\centerFigure}[2]{
    \begin{figure}[h]
        \centering
            #1
        \caption{#2}
    \end{figure}
}

% Tikz figure
\newcommand{\tikzGraphic}[1]{
    \begin{center}
    \begin{tikzpicture}
        #1
    \end{tikzpicture}
    \end{center}
}

% Enumerate numbers (seperate by \item)
\newcommand{\numbers}{\textbf{\number*)}}

% Enumerate letters (seperate by \item)
\newcommand{\letters}{\textbf{\alph*)}}

\title{
    \Huge Machine Learning \\
    \Large Assignment IV
}
\date{2025-05-09}
\author{Michael Saba}

\begin{document}
\pagenumbering{gobble}
\maketitle
\newpage
\setlength{\parindent}{0pt}
\pagenumbering{arabic}

\subsection*{Ex 1}

The Leibniz Integral rule states that
the integral:
\[ \dd{}{z}
\para{\integral{a(z)}{b(z)}{f(z, y)}{dy}} \]
Evaluates to:
\[ f(z, b(z))\dd{}{z}b(z) 
- f(z, a(z))\dd{}{z}a(z)
+ \integral{a(z)}{b(z)}{\partialdd{}{z}
f(z, y)}{dy} \]

\begin{enumerate}[label = \letters]
\item 
We have the following expected absolute loss
which is just a sum over the random variabe
$Y$ (for a fixed $z$):
\[ \ebb[|z - Y|] 
= \integral{-\infty}{\infty}{|z - y|\pbb_Y}{(Y = dy)} \]
Which we can rewrite using the density:
\[ \integral{-\infty}{\infty}{|z - y|p_Y(y)}{(dy)} \]
We want to minimize this risk (expected loss)
by finding the minimizing value $\hat{z}$,
which we can do by finding the $z$
for which the derivative of the expected loss
is $0$:
\[ \dd{}{z}\para{\integral{-\infty}{\infty}
{|z - y|p_Y(y)}{(dy)} } = 0 \]
First note that:
\[ |z-y| = \piecewise{(z-y) \IF z \geq y \\
(y-z) \IF z \leq y} \]
So we can divide the integral into two sections
with $y \in (-\infty, z]$
and then $y \in [z, \infty)$:
\[ \integral{-\infty}{z}{(y - z)p_Y(y)}{(dy)}
+ \integral{z}{\infty}{(z - y)p_Y(y)}{(dy)} \]
We can then derivate both using the Leibniz rule
(note that $\pm\infty$ is a constant
and its derivative with respect to $z$ is just 0).
The first two terms in the Leibniz rule
are $0$ for both integrals
since they either have $|z-z|$
or are multiplied by the derivative of a constant:
\[ \dd{}{z}\para{
\integral{-\infty}{z}{(z - y)p_Y(y)}{(dy)}}
= \integral{-\infty}{z}{\partialdd{}{z}
((z - y)p_Y(y))}{dy}
= \integral{-\infty}{z}{p_Y(y)}{dy} \]
\[ \dd{}{z}\para{
\integral{z}{\infty}{(y - z)p_Y(y)}{(dy)}}
= \integral{z}{\infty}{\partialdd{}{z}
((y - z)p_Y(y))}{dy}
= -\integral{z}{\infty}{p_Y(y)}{dy} \]
So:
\[ \dd{}{z}\para{\integral{-\infty}{\infty}
{|z - y|p_Y(y)}{(dy)} } = 
\integral{-\infty}{z}{p_Y(y)}{dy}
-\integral{z}{\infty}{p_Y(y)}{dy}\]
Note that the integral of the density of $Y$
over some interval is just the probability
that $Y$ falls in that interval, so:
\[ \dd{}{z}\para{\integral{-\infty}{\infty}
{|z - y|p_Y(y)}{(dy)} } = 
\pbb(Y \leq z) - \pbb(Y \geq z) \]
Now, setting the derivative to $0$,
we get:
\[ \pbb(Y \leq z) - \pbb(Y \geq z) = 0 \]
\[ \pbb(Y \leq z) = \pbb(Y \geq z) \]
And since:
\[ \pbb(Y \leq z) + \pbb(Y \geq z) = 1 \]
That means that:
\[ \pbb(Y \leq z) = \pbb(Y \geq z) = \dfrac{1}{2} \]
So the value $\hat{z}$
at which the expected total loss is minimized
is just the point $z$
at which the probability that $Y$
is larger or smaller than $z$ is equal to half,
which makes $\hat{z}$ the median $m$.
\item
We want to find Baye's optimal predictor
for a fixed $X = x$,
which we can do by finding the value $z = m(x)$
such that the conditional risk 
(expected loss knowing $X = x$) is minimized:
\[ \ebb[|z - Y|] 
= \integral{-\infty}{\infty}
{|z - y|\pbb_{Y \mid X = x}}{(dy)} \]
Which we can rewrite using the density:
\[ \integral{-\infty}{\infty}
{|z - y|p_{Y | X = x}(y)}{(dy)} \]
Note that because $X = x$ is fixed,
this doesn't really change anything;
we still have an integral over $Y$,
and we are trying to find the value of $z$
that minimizes this expected absolute loss. \\
The only difference is that now it
is a function of whatever fixed $x$
we choose. \\
We can repeat the result but with the
density given $X = x$:
\[ \dd{}{z}\para{\integral{-\infty}{\infty}
{|z - y|p_{Y | X = x}(y)}{(dy)} } = 
\pbb(Y \leq z \mid X = x) - \pbb(Y \geq z \mid X = x) \]
Setting the derivative to $0$ tell us that:
\[ \pbb(Y \leq z \mid X = x) 
- \pbb(Y \geq z \mid X = x) = 0 \]
\[ \pbb(Y \leq z \mid X = x) 
= \pbb(Y \geq z \mid X = x) \]
And since:
\[ \pbb(Y \leq z \mid X = x) 
+ \pbb(Y \geq z \mid X = x) = 1 \]
This means that:
\[ \pbb(Y \leq z \mid X = x) 
= \pbb(Y \geq z \mid X = x) = \dfrac{1}{2} \]
Which again means that $\hat{z}$
is the median $m(x)$, but this times
depends on whatever $x$ we are given. \\
\end{enumerate}

\newpage

\colorText{red}{
    \subsection*{Correction}
    \begin{enumerate}[label = \letters]
        \item 
        Remember to either mention the function is
        known to be convex, or calculate the second
        derivative and show it's positive. \\
        This isn't necessary if we are told
        the function is convex, or if it's something
        like the risk, which is always convex if
        the loss is convex. \\
        \item
    \end{enumerate}
}


\newpage

\subsection*{Ex 2}

Proving that a loss function $\Phi$
is classification calibrated
just means we have to show that $\Phi'(0)$
exists and that $\Phi'(0) < 0$. \\

\begin{enumerate}[label = \letters]
    \item 
    First we have the hinge loss
    $\Phi_{\text{hinge}}(z) = \max(0, 1-z)$,
    which we can write as:
    \[ \Phi_{\text{hinge}}(z) = 
    \piecewise{1-z \IF z < 1 \\
    0 \IF \quad \; z \geq 1} \]
    The only point at which the function
    id not differentiable (smooth) is $z = 1$. \\
    At $z = 0$, we have 
    $\Phi_{\text{hinge}}(z) = 1-z$,
    which means that:
    \[ \Phi_{\text{hinge}}'(z) = (1-z)' = -1 \]
    So $\Phi_{\text{hinge}}'(0)$ 
    exists and is smaller than $0$.
    \item 
    We now have the exponential loss
    $\Phi_{\text{exp}}(z) = e^{-z}$,
    which is differentiable for all $z \in \rbb$. \\
    We can calculate the derivative:
    \[ \Phi_{\text{exp}}'(z) = (e^{-z})' = -e^{-z} \]
    Which means that:
    \[ \Phi_{\text{exp}}'(0) = -e^{0} = -1 \]
    So $\Phi_{\text{exp}}'(0)$
    exists and is smaller than $0$.
    \item 
    We now have the squared loss
    $\Phi_{\text{quad}}(z) = (1-z)^2$,
    which is differentiable for all $z \in \rbb$. \\
    We can calculate the derivative:
    \[ \Phi_{\text{quad}}'(z) = ((1-z)^2)' = 2z -2 \]
    Which means that:
    \[ \Phi_{\text{quad}}'(0) = 0 -2 = -2 \]
    So $\Phi_{\text{quad}}'(0)$
    exists and is smaller than $0$. \\
\end{enumerate}

\newpage

\colorText{red}{
    \subsection*{Correction}
    In order for the loss $\Phi$ to be classification
    calibrated, we need to show that $\Phi'(0)$
    exists and that $\Phi'(0) < 0$,
    but we also need it to be convex in the first
    place. \\
    I forgot to prove the convexity of the functions. \\
    \begin{enumerate}[label = \letters]
        \item 
        For the hinge loss, we have
        $\Phi = \max(0, 1-z)$.
        A function that is the max of two functions
        is convex if both functions are convex. \\
        So we can show that $0$ and $1-z$
        are convex by doing the second derivative. \\
        We also need to mention that it is
        differentiable at $0$,
        which we can do by just noting that $1-z$
        is differentiable.
        \item
        We can just find the second derivative 
        of $e^{-z}$, which will be $e^{-z}$,
        which is always positive, and thus
        it is convex. \\
        \item
        We can again find the second derivative
        of $(1-z)^2$, which is $2$,
        which is always positive. \\
    \end{enumerate}
    Note that the tutor mentioned that it
    might not be enough to just mention
    that the function is differentiable at $0$.
    We may need to find the limit as $z$
    approaches from both sides,
    and then show the derivative exists. \\

    Some other students also just directly
    proved that $\sign(g^*) = f^*$,
    which is correct but may be harder. \\
}

\newpage

\subsection*{Ex 3}

We have a training dataset $\dcal$ 
with $n$ points:
\[ \dcal = \{(x_i, y_i) \mid 
x_i, y_i \in \rbb \}_{i = 1, \dots n} \]
and a hypothesis class:
\[ \hcal = \{ h_{w}(x) = wx \mid w \in \rbb \} \]
Which is just the class of all line
functions passing through the origin
(the slope can be any $w \in \rbb$,
and the $y$-intercept is equal to $0$). \\
We also use the squared loss function:
\[ \ell(h_w(x_i), y_i) = (h_w(x_i) - y_i)^2 \]
We know that the empirical risk
is the mean of the dataset,
which we can write as:
\[ \hat{\rscr}_\ell^n(h_w)
= \dfrac{1}{n}\sum_{i = 1}^{n}(wx_i - y_i)^2 \]

\begin{enumerate}[label = \letters]
\item 
We want to find the value $w* \in \rbb$
that minimizes $\hat{\rscr}_\ell^n(h_w)$,
which would in turn give us the
empirical risk minimizer $h_w^* = h_{w^*}$
in the hypothesis class $\hcal$. \\
We can do it by just differentiating
with respect to $which$ and setting the derivative
equal to $w$,
then solving for $w$:
\[ \dd{}{w}\hat{\rscr}_\ell^n(h_w)
= \dd{}{w}\para{\dfrac{1}{n}\sum_{i = 1}^{n}
(wx_i - y_i)^2} 
= \dfrac{2}{n}\sum_{i = 1}^{n}x_i(wx_i - y_i) \]
Setting it equal to $0$, we get:
\[ \dfrac{2}{n}\sum_{i = 1}^{n}x_i(wx_i - y_i) = 0 \]
\[ \dfrac{2w}{n}\sum_{i = 1}^{n}x_i^2 
= \dfrac{2}{n}\sum_{i = 1}^{n} x_iy_i \]
\[ w = \dfrac{\sum_{i = 1}^{n}x_iy_i}
{\sum_{i = 1}^{n} x_i^2} \]
Which is the optimal slope $w^*$.
\item
We now have the dataset:
\[ \dcal = \{ (1, 2), (2, 3), (3, 5)\} \]
We can compute the optimal weight or slope $w^*$
for $\dcal$:
\[ w^* = \dfrac{\sum_{i = 1}^{n}x_iy_i}
{\sum_{i = 1}^{n} x_i^2}
= \dfrac{2 + 6 + 15}
{1 + 4 + 9} = \dfrac{23}{14} \sim 1.64 \]
We can then use this optimal weight $w^*$
in the classifier $h_{w^*}$
to get the empirical risk minimizer $h_w^*$. \\
We can test its performance on the training
data:
\[ \text{For } (x_i, y_i) = (1, 2) \qquad
h_{1.64}(1) = 1 \cdot 1.64 = 1.64 \sim 2 \]
\[ \text{For } (x_i, y_i) = (2, 3) \qquad
h_{1.64}(2) = 2 \cdot 1.64 = 3.28  \sim 3 \]
\[ \text{For } (x_i, y_i) = (3, 5) \qquad
h_{1.64}(3) = 3 \cdot 1.64 = 4.92  \sim 3 \]
It did an ok job.
\item
We now want to calculate the empirical
risk for $h_{2}$,
that is, for the function weight weight 
$\tilde{w} = 2$:
\[ \hat{\rscr}_\ell^n(h_2)
= \dfrac{1}{n}\sum_{i = 1}^{n}(2x_i - y_i)^2 \]
We will do this for the dataset $\dcal$:
\[ \hat{\rscr}_\ell^3(h_2)
= (2-2)^2 + (4-3)^2 + (6-5)^2 = 2 \]
The empirical risk is rather low,
though likely not as low as that of $w* = 1.64$,
the optimal weight. \\
\end{enumerate}


\end{document}