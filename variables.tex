
% For using float option H that places figures exatcly where we want them
\usepackage{float}
% makes figure font bold
\usepackage{caption}
\captionsetup[figure]{labelfont=bf}
% For text generation
\usepackage{lipsum}
% For drawing
\usepackage{tikz}
% For smaller or equal sign and not divide sign
\usepackage{amssymb}
% For the diagonal fraction
\usepackage{xfrac}
% For enumerating exercise parts with letters instead of numbers
\usepackage{enumitem}
% For dfrac, which forces the fraction to be in display mode (large) e
% even in math mode (small)
\usepackage{amsmath}
% For degree sign
\usepackage{gensymb}
% For "\mathbb" macro
\usepackage{amsfonts}
% For positioning 
\usepackage{indentfirst}
\usetikzlibrary{shapes,positioning,fit,calc}
% for adjustwidth environment
\usepackage{changepage}
% for arrow on top
\usepackage{esvect}
% for mathbb 1
\usepackage{bbm}
% for mathsrc
\usepackage[mathscr]{eucal}

% for tikz
\usepackage{pgfplots}
\pgfplotsset{compat=1.18}
\usepackage{amsmath}
\usepgfplotslibrary{groupplots}

% overline short italic
\newcommand{\olsi}[1]{\,\overline{\!{#1}}}

\newcommand*{\perm}[2]{{}^{#1}\!P_{#2}}
\newcommand*{\comb}[2]{{}^{#1}C_{#2}}

\newcommand{\binomial}[2]{%
  \ifnum#2=0
    1%
  \else
    \ifnum#1=0
      0%
    \else
      \the\numexpr%
        \binom{#1-1}{#2-1}+\binom{#1-1}{#2}%
    \fi
  \fi
}

% Commands for set symbols

% Natural numbers
\newcommand{\N}{\mathbb{N}}
% Integers
\newcommand{\Z}{\mathbb{Z}}
% Rational Numbers
\newcommand{\Q}{\mathbb{Q}}
% Real Numbers
\newcommand{\R}{\mathbb{R}}
% Complex Numbers
\newcommand{\C}{\mathbb{C}}
% Field
\newcommand{\F}{\mathbb{F}}
% Probability
\newcommand{\Pb}{\mathbb{P}}
% Others
\newcommand{\X}{\mathbb{X}}
\newcommand{\Y}{\mathbb{Y}}
% Expected value
\newcommand{\E}{\mathbb{E}}
% Others
\newcommand{\U}{\mathbb{U}}
\newcommand{\V}{\mathbb{V}}

% Powerset
\newcommand{\pcal}{\mathcal{P}}
\newcommand{\xcal}{\mathcal{X}}
\newcommand{\ycal}{\mathcal{Y}}
\newcommand{\acal}{\mathcal{A}}
\newcommand{\bcal}{\mathcal{B}}
\newcommand{\fcal}{\mathcal{F}}
\newcommand{\scal}{\mathcal{S}}
\newcommand{\ucal}{\mathcal{U}}
\newcommand{\vcal}{\mathcal{V}}
\newcommand{\rcal}{\mathcal{R}}
\newcommand{\dcal}{\mathcal{D}}
\newcommand{\hcal}{\mathcal{H}}

% Others
\newcommand{\zfrak}{\mathfrak{Z}}
\newcommand{\afrak}{\mathfrak{A}}
\newcommand{\ffrak}{\mathfrak{F}}
\newcommand{\rfrak}{\mathfrak{R}}

% Script (for sigma algebra and risk)
\newcommand{\rscr}{\mathscr{R}}
\newcommand{\xscr}{\mathscr{X}}
\newcommand{\yscr}{\mathscr{Y}}
\newcommand{\sscr}{\mathscr{S}}
\newcommand{\ascr}{\mathscr{A}}
\newcommand{\fscr}{\mathscr{F}}
\newcommand{\bscr}{\mathscr{B}}

% Normal
\newcommand{\normal}{\mathcal{N}}

% Indicator function
\newcommand{\indicator}{\mathbbm{1}}


% Bracket types

% Parantheses
\newcommand{\para}[1]{\left( #1 \right)}
% Curly Braces
\newcommand{\curl}[1]{\left\{ #1 \right\}}
% Brackets
\newcommand{\brac}[1]{\left[ #1 \right]}
% Angled Brackets
\newcommand{\ang}[1]{\left\langle #1 \right\rangle}

% Color
\newcommand{\myColor}[2]{
\begingroup
\color{#1}
    #2
\endgroup
}


\newcommand{\var}{\text{Var}}
\newcommand{\cov}{\text{Cov}}
\newcommand{\cof}{\text{Cof}}
\newcommand{\adj}{\text{Adj}}
\newcommand{\std}{\text{Std}}
\newcommand{\corr}{\text{Corr}}
\newcommand{\sign}{\text{sign}}


% And and Or in math mode
\newcommand{\AND}{\qquad \text{ and } \qquad}
\newcommand{\OR}{\qquad \text{ or } \qquad}
\newcommand{\IF}{\qquad \text{ if }}

% 1st
\newcommand{\st}[1]{#1^{\text{st}}}
% 2nd
\newcommand{\nd}[1]{#1^{\text{nd}}}
% 3rd
\newcommand{\rd}[1]{#1^{\text{rd}}}
% nth
\newcommand{\nth}[1]{#1^{\text{th}}}
% Derivation

% d over d
\newcommand{\dd}[2]{\dfrac{d#1}{d#2}}
% partial d over d
\newcommand{\partialdd}[2]{\dfrac{\partial #1}{\partial #2}}
% delta d over d
\newcommand{\deltadd}[2]{\dfrac{\Delta #1}{\Delta #2}}

% Integration

% One variable
\newcommand{\integral}[4]{\int_{#1}^{#2} #3 \, #4}
% Two variables
\newcommand{\multiIntegral}[6]{\int_{#1}^{#2}\int_{#3}^{#4} #5 \, #6}
% Bounded by some space 
\newcommand{\boundIntegral}[2]{\int_{#1} #2 \, d#1}

% Limit
\newcommand{\limit}[3]{\lim_{#1 \to #2} #3}

% Bold vector arrow
\newcommand{\bv}[1]{\vec{\mathbf{#1}}}

% Inverse exponent
\newcommand{\inv}[0]{^{-1}}


% 3D vector
\newcommand{\vecThree}[3]{\begin{bmatrix} 
    #1 \\ #2 \\ #3 
\end{bmatrix}}
% Transposed 3D vector
\newcommand{\vecThreeTranposed}[3]{\begin{bmatrix} 
    #1 & #2 & #3
\end{bmatrix}}

% 3x3 Matrix
\newcommand{\matThree}[9]{\begin{bmatrix} 
    #1 & #2 & #3 \\
    #4 & #5 & #6 \\
    #7 & #8 & #9
\end{bmatrix}}

% 2D vector
\newcommand{\vecTwo}[2]{\begin{bmatrix} 
    #1 \\ #2 
\end{bmatrix}}

% Transposed 2D vector
\newcommand{\vecTwoTransposed}[2]{\begin{bmatrix} 
    #1 & #2 
\end{bmatrix}}

% 2x2 Matrix
\newcommand{\matTwo}[4]{\begin{bmatrix} 
    #1 & #2 \\
    #3 & #4 
\end{bmatrix}}

% Piece wise
\newcommand{\piecewise}[2]{\begin{cases}
    #1 \\
    #2
\end{cases}}

% Piece wise
\newcommand{\piecewiseThree}[3]{\begin{cases}
    #1 \\
    #2 \\
    #3
\end{cases}}


% Three columns figure
\newcommand{\colThree}[4]{
    \begin{figure}[h]
        \centering
        \begin{minipage}{0.3\textwidth}
            \centering
            #1
        \end{minipage}
        \hspace{0.5cm}
        \begin{minipage}{0.3\textwidth}        
            \centering
            #2
        \end{minipage}
        \hspace{0.5cm}
        \begin{minipage}{0.3\textwidth}
            \centering
            #3
        \end{minipage}
        \caption{#4}
    \end{figure}
}

% Centered figure
\newcommand{\centerFigure}[2]{
    \begin{figure}[h]
        \centering
            #1
        \caption{#2}
    \end{figure}
}

% Tikz figure
\newcommand{\tikzFigure}[1]{
    \begin{center}
    \begin{tikzpicture}
        #1
    \end{tikzpicture}
    \end{center}
}