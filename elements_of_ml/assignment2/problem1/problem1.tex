

\documentclass[12pt]{article}

\usepackage[margin=1in]{geometry}


%===============================================================================
%================================== PACKAGES ===================================
%===============================================================================

% For using float option H that places figures 
% exatcly where we want them
\usepackage{float}
% makes figure font bold
\usepackage{caption}
\captionsetup[figure]{labelfont=bf}
% For text generation
\usepackage{lipsum}
% For drawing
\usepackage{tikz}
% For smaller or equal sign and not divide sign
\usepackage{amssymb}
% For the diagonal fraction
\usepackage{xfrac}
% For enumerating exercise parts with letters instead of numbers
\usepackage{enumitem}
% For dfrac, which forces the fraction to be in display mode (large) e
% even in math mode (small)
\usepackage{amsmath}
% For degree sign
\usepackage{gensymb}
% For "\mathbb" macro
\usepackage{amsfonts}
% For positioning 
\usepackage{indentfirst}
\usetikzlibrary{shapes,positioning,fit,calc}
% for adjustwidth environment
\usepackage{changepage}
% for arrow on top
\usepackage{esvect}
% for mathbb 1
\usepackage{bbm}
% for mathsrc
\usepackage[mathscr]{eucal}
% For degree sign
\usepackage{gensymb}
% For quotes
\usepackage{csquotes}
% For vertical lines
\usepackage{mathtools}
% For cols
\usepackage{multicol}

% for tikz
\usepackage{pgfplots}
\pgfplotsset{compat=1.18}
\usepackage{amsmath}
\usepgfplotslibrary{groupplots}


%===============================================================================
%==================================== FONTS ====================================
%===============================================================================


% Mathcal
\newcommand{\acal}{\mathcal{A}}
\newcommand{\bcal}{\mathcal{B}}
\newcommand{\ccal}{\mathcal{C}}
\newcommand{\dcal}{\mathcal{D}}
\newcommand{\ecal}{\mathcal{E}}
\newcommand{\fcal}{\mathcal{F}}
\newcommand{\gcal}{\mathcal{G}}
\newcommand{\hcal}{\mathcal{H}}
\newcommand{\ical}{\mathcal{I}}
\newcommand{\jcal}{\mathcal{J}}
\newcommand{\kcal}{\mathcal{K}}
\newcommand{\lcal}{\mathcal{L}}
\newcommand{\mcal}{\mathcal{M}}
\newcommand{\ncal}{\mathcal{N}}
\newcommand{\ocal}{\mathcal{O}}
\newcommand{\pcal}{\mathcal{P}}
\newcommand{\qcal}{\mathcal{Q}}
\newcommand{\rcal}{\mathcal{R}}
\newcommand{\scal}{\mathcal{S}}
\newcommand{\tcal}{\mathcal{T}}
\newcommand{\ucal}{\mathcal{U}}
\newcommand{\vcal}{\mathcal{V}}
\newcommand{\wcal}{\mathcal{W}}
\newcommand{\xcal}{\mathcal{X}}
\newcommand{\ycal}{\mathcal{Y}}
\newcommand{\zcal}{\mathcal{Z}}

% Mathfrak
\newcommand{\afrak}{\mathfrak{A}}
\newcommand{\bfrak}{\mathfrak{B}}
\newcommand{\cfrak}{\mathfrak{C}}
\newcommand{\dfrak}{\mathfrak{D}}
\newcommand{\efrak}{\mathfrak{E}}
\newcommand{\ffrak}{\mathfrak{F}}
\newcommand{\gfrak}{\mathfrak{G}}
\newcommand{\hfrak}{\mathfrak{H}}
\newcommand{\ifrak}{\mathfrak{I}}
\newcommand{\jfrak}{\mathfrak{J}}
\newcommand{\kfrak}{\mathfrak{K}}
\newcommand{\lfrak}{\mathfrak{L}}
\newcommand{\mfrak}{\mathfrak{M}}
\newcommand{\nfrak}{\mathfrak{N}}
\newcommand{\ofrak}{\mathfrak{O}}
\newcommand{\pfrak}{\mathfrak{P}}
\newcommand{\qfrak}{\mathfrak{Q}}
\newcommand{\rfrak}{\mathfrak{R}}
\newcommand{\sfrak}{\mathfrak{S}}
\newcommand{\tfrak}{\mathfrak{T}}
\newcommand{\ufrak}{\mathfrak{U}}
\newcommand{\vfrak}{\mathfrak{V}}
\newcommand{\wfrak}{\mathfrak{W}}
\newcommand{\xfrak}{\mathfrak{X}}
\newcommand{\yfrak}{\mathfrak{Y}}
\newcommand{\zfrak}{\mathfrak{Z}}

% Mathscr
\newcommand{\ascr}{\mathscr{A}}
\newcommand{\bscr}{\mathscr{B}}
\newcommand{\cscr}{\mathscr{C}}
\newcommand{\dscr}{\mathscr{D}}
\newcommand{\escr}{\mathscr{E}}
\newcommand{\fscr}{\mathscr{F}}
\newcommand{\gscr}{\mathscr{G}}
\newcommand{\hscr}{\mathscr{H}}
\newcommand{\iscr}{\mathscr{I}}
\newcommand{\jscr}{\mathscr{J}}
\newcommand{\kscr}{\mathscr{K}}
\newcommand{\lscr}{\mathscr{L}}
\newcommand{\mscr}{\mathscr{M}}
\newcommand{\nscr}{\mathscr{N}}
\newcommand{\oscr}{\mathscr{O}}
\newcommand{\pscr}{\mathscr{P}}
\newcommand{\qscr}{\mathscr{Q}}
\newcommand{\rscr}{\mathscr{R}}
\newcommand{\sscr}{\mathscr{S}}
\newcommand{\tscr}{\mathscr{T}}
\newcommand{\uscr}{\mathscr{U}}
\newcommand{\vscr}{\mathscr{V}}
\newcommand{\wscr}{\mathscr{W}}
\newcommand{\xscr}{\mathscr{X}}
\newcommand{\yscr}{\mathscr{Y}}
\newcommand{\zscr}{\mathscr{Z}}

% Mathbb
\newcommand{\abb}{\mathbb{A}}
\newcommand{\bbb}{\mathbb{B}}
\newcommand{\cbb}{\mathbb{C}}
\newcommand{\dbb}{\mathbb{D}}
\newcommand{\ebb}{\mathbb{E}}
\newcommand{\fbb}{\mathbb{F}}
\newcommand{\gbb}{\mathbb{G}}
\newcommand{\hbb}{\mathbb{H}}
\newcommand{\ibb}{\mathbb{I}}
\newcommand{\jbb}{\mathbb{J}}
\newcommand{\kbb}{\mathbb{K}}
\newcommand{\lbb}{\mathbb{L}}
\newcommand{\mbb}{\mathbb{M}}
\newcommand{\nbb}{\mathbb{N}}
\newcommand{\obb}{\mathbb{O}}
\newcommand{\pbb}{\mathbb{P}}
\newcommand{\qbb}{\mathbb{Q}}
\newcommand{\rbb}{\mathbb{R}}
\newcommand{\sbb}{\mathbb{S}}
\newcommand{\tbb}{\mathbb{T}}
\newcommand{\ubb}{\mathbb{U}}
\newcommand{\vbb}{\mathbb{V}}
\newcommand{\wbb}{\mathbb{W}}
\newcommand{\xbb}{\mathbb{X}}
\newcommand{\ybb}{\mathbb{Y}}
\newcommand{\zbb}{\mathbb{Z}}


%===============================================================================
%=============================== SPECIAL SYMBOLS ===============================
%===============================================================================


% Orbit (group theory)
\newcommand{\orbit}{\mathcal{O}}
% Normal group
\newcommand{\normal}{\mathcal{N}}
% Indicator function
\newcommand{\indicator}{\mathbbm{1}}
% Laplace transform
\newcommand{\laplace}[1]{\mathcal{L}}
% Epsilon shorthand
\newcommand{\eps}{\varepsilon}
% Omega
\newcommand{\om}{\omega}
\newcommand{\Om}{\Omega}

%===============================================================================
%================================== OPERATORS ==================================
%===============================================================================


% Inverse exponent
\newcommand{\inv}[0]{^{-1}}
% Overline bar
\newcommand{\olsi}[1]{\,\overline{\!{#1}}}
% Less than or equal slanted
\newcommand{\seqs}{\leqslant}
% Greater or equal slanted
\newcommand{\geqs}{\geqslant}
% Subset or equal
\newcommand{\sub}{\subseteq}
% Proper subset
\newcommand{\prosub}{\subset}
% from
\newcommand{\from}{\leftarrow}

% Parantheses
\newcommand{\para}[1]{\left( #1 \right)}
% Curly Braces
\newcommand{\curl}[1]{\left\{ #1 \right\}}
% Brackets
\newcommand{\brac}[1]{\left[ #1 \right]}
% Angled Brackets
\newcommand{\ang}[1]{\left\langle #1 \right\rangle}
% Norm
\newcommand{\norm}[1]{\left\| #1 \right\|}

% Piece wise (use \\ between cases)
\newcommand{\piecewise}[1]{\begin{cases} #1 \end{cases}}

% Bold symbol shorthand
\newcommand{\bl}{\boldsymbol}

% Vertical space
\newcommand{\vs}[1]{\vspace{#1 pt}}
% Horizontal ertical space
\newcommand{\hs}[1]{\hspace{#1 pt}}


%===============================================================================
%============================== TEXT BASED SYMBOLS =============================
%===============================================================================


% Radians
\newcommand{\rad}{\text{rad}}
% Least Common Multiple
\newcommand{\lcm}{\text{lcm}}
% Automorphism
\newcommand{\Aut}{\text{Aut}}
% Variance
\newcommand{\var}{\text{Var}}
% Covariance
\newcommand{\cov}{\text{Cov}}
% Cofactor (matrix)
\newcommand{\cof}{\text{Cof}}
% Adjugate (matrix)
\newcommand{\adj}{\text{Adj}}
% Trace (matrix)
\newcommand{\tr}{\text{tr}}
% Standard deviation
\newcommand{\std}{\text{Std}}
% Correlation coefficient
\newcommand{\corr}{\text{Corr}}
% Sign
\newcommand{\sign}{\text{sign}}

% And text
\newcommand{\AND}{\text{ and }}
% Or text 
\newcommand{\OR}{\text{ or }}
% If text
\newcommand{\IF}{\text{ if }}
% When text
\newcommand{\WHEN}{\text{ when }}
% Then text
\newcommand{\THEN}{\text{ then }}

% 1st
\newcommand{\st}[1]{#1^{\text{st}}}
% 2nd
\newcommand{\nd}[1]{#1^{\text{nd}}}
% 3rd
\newcommand{\rd}[1]{#1^{\text{rd}}}
% nth
\newcommand{\nth}[1]{#1^{\text{th}}}


%===============================================================================
%========================= PROBABILITY AND STATISTICS ==========================
%===============================================================================


% Permutation
\newcommand{\perm}[2]{{}^{#1}\!P_{#2}}
% Combination
\newcommand{\comb}[2]{{}^{#1}C_{#2}}

% Baye's risk
\newcommand{\risk}[1]{\mathscr{R}_{#1}}
% Baye's optimal risk
\newcommand{\riskOptimal}[1]{\mathscr{R}_{#1}^*}
% Baye's empirical risk
\newcommand{\riskEmpirical}[2]{\hat{\mathscr{R}}_{#1}^{#2}}


%===============================================================================
%=================================== CALCULUS ==================================
%===============================================================================


% d over d derivative
\newcommand{\dd}[2]{\dfrac{d#1}{d#2}}
% partial d over d derivative
\newcommand{\partialdd}[2]{\dfrac{\partial #1}{\partial #2}}
% delta d over d derivative
\newcommand{\deltadd}[2]{\dfrac{\Delta #1}{\Delta #2}}

% Integration between a and b
\newcommand{\integral}[4]{\int_{#1}^{#2} #3 \, #4}
% Integration in some space 
\newcommand{\boundIntegral}[2]{\int_{#1} #2 \, d#1}

% Limit
\newcommand{\limit}[3]{\lim_{#1 \to #2} #3}


%===============================================================================
%================================  BIG SYMBOLS  ================================
%===============================================================================


% Sum
\newcommand{\sumof}[2]{\sum_{#1}^{#2}}
% Product
\newcommand{\productof}[2]{\prod_{#1}^{#2}}
% Union
\newcommand{\unionof}[2]{\bigcup_{#1}^{#2}}
% Intersection
\newcommand{\intersectionof}[2]{\bigcap_{#1}^{#2}}
% Or
\newcommand{\orof}[2]{\bigvee_{#1}^{#2}}
% And
\newcommand{\andof}[2]{\bigwedge_{#1}^{#2}}


%===============================================================================
%=============================== LINEAR ALGEBRA ================================
%===============================================================================


% Bold vector arrow
\newcommand{\bv}[1]{\vec{\mathbf{#1}}}

% Matrix or vector (use // for column, & for row) 
% with brackets
\newcommand{\bmat}[1]{\begin{bmatrix} #1 \end{bmatrix}}
% Matrix or vector (use // for column, & for row) 
% with curved brackets
\newcommand{\pmat}[1]{\begin{pmatrix} #1 \end{pmatrix}}
% Matrix or vector (use // for column, & for row) 
% with lines on either side 
\newcommand{\lmat}[1]{\begin{vmatrix} #1 \end{vmatrix}}
% Matrix or vector (use // for column, & for row) 
% with curly braces
\newcommand{\cmat}[1]{\begin{Bmatrix} #1 \end{Bmatrix}}
% Matrix or vector (use // for column, & for row) 
% with no braces
\newcommand{\mat}[1]{\begin{matrix} #1 \end{matrix}}


%===============================================================================
%================================ LARGE OBJECTS ================================
%===============================================================================

% Multiple lines
\newcommand{\multiline}[1]{
\begin{align*}
    #1
\end{align*}
}

% Multiple lines with equation numbers
\newcommand{\eqmultiline}[1]{
\begin{align*}
    #1
\end{align*}
}

% Color
\newcommand{\colorText}[2]{
\begingroup
\color{#1}
    #2
\endgroup
}

% Centered figure
\newcommand{\centerFigure}[2]{
    \begin{figure}[h]
        \centering
            #1
        \caption{#2}
    \end{figure}
}

% Tikz figure
\newcommand{\tikzGraphic}[1]{
    \begin{center}
    \begin{tikzpicture}
        #1
    \end{tikzpicture}
    \end{center}
}

% Enumerate numbers (seperate by \item)
\newcommand{\numbers}{\textbf{\number*)}}

% Enumerate letters (seperate by \item)
\newcommand{\letters}{\textbf{\alph*)}}

\title{
    \Huge Elements of Machine Learning \\
    \huge Assignment II \\
    \LARGE Problem 1
}
\date{2025-11-15}
\author{
    Michael Saba (Matriculation No. 7076371) \\
    Chongbiao Wang (Matriculation No. 7076265)
}

\pagestyle{fancy}
\fancyhf{}
\fancyhead[L]{Michael Saba 7076371}
\fancyhead[R]{Chongbiao Wang 7076265}
\fancyfoot[C]{\thepage}

\begin{document}
\pagenumbering{gobble}
\maketitle
\newpage
\setlength{\parindent}{0pt}
\pagenumbering{arabic}

\subsection*{Problem 1}

\begin{enumerate}[label = \letters]
\item
    Logistic Regression is a discriminative approach to
    binary classification that attempts to estimate the conditional
    $P(Y=1 \mid X=x)$ using a linear classifier 
    $f(X) = \bf{X}\beta$, 
    hence why it has regression in its name. \\
    It uses the logistic function, $\sigma(a)$, to map
    the values of the linear model to $[0, 1]$, ensuring
    we have a probability. \\
    So we end up with:
    \[ P(Y = 1 \mid X=x) = \sigma(\bf{X}\beta)
    = \dfrac{\exp(\beta_0 + \sum_{k=1}^p\bf{X}_k\beta_k)}
    {1 + \exp(\beta_0 + \sum_{k=1}^p\bf{X}_k\beta_k)} \] 
    Here, $f(X) = \beta_0 + \sum_{k=1}^p\bf{X}_k\beta_k$ 
    is the linear model given that $X$ has $p$ features, 
    and $\sigma(f(X))$ is the mapping of that output to $[0, 1]$,
    which turns the result into a probability that the given
    $X$ has class output $Y=1$. \\
\item 
    The decision boundary is the set of points that are determined
    by the model to be equally likely in class $Y=0$ or class $Y=1$.
    That is, it is the set of all $x \in \rbb^p$
    such that:
    \[P(Y = 1 \mid X = x) = 0.5 \]
    We can find this boundary by using the odds and log-odds:
    \[ P(Y = 1 \mid X = x) = 0.5 \]
    \[ \text{odds ratio } = \dfrac{P(Y = 1 \mid X = x)}
    {1 - P(Y = 1 \mid X = x)} 
    = \dfrac{0.5}{0.5} = 1 \]
    \[ \log\para{\dfrac{P(Y = 1 \mid X = x)}
    {1 - P(Y = 1 \mid X = x)}} = \log(1) = 0 \]
    We know that:
    \multiline{
    \dfrac{P(Y = 1 \mid X = x)}{1 - P(Y = 1 \mid X = x)}
    &= \dfrac{\dfrac{\exp(\beta_0 + \sum_{k=1}^p\bf{X}_k\beta_k)}
    {1 + \exp(\beta_0 + \sum_{k=1}^p\bf{X}_k\beta_k)}}
    {1 - \dfrac{\exp(\beta_0 + \sum_{k=1}^p\bf{X}_k\beta_k)}
    {1 + \exp(\beta_0 + \sum_{k=1}^p\bf{X}_k\beta_k)}}
    = \dfrac{\dfrac{\exp(\beta_0 + \sum_{k=1}^p\bf{X}_k\beta_k)}
    {1 + \exp(\beta_0 + \sum_{k=1}^p\bf{X}_k\beta_k)}}
    {\dfrac{1}{1 + \exp(\beta_0 + \sum_{k=1}^p\bf{X}_k\beta_k)}} \\
    &= \exp(\beta_0 + \sum_{k=1}^p\bf{X}_k\beta_k)}
    So:
    \[ \log\para{\dfrac{P(Y = 1 \mid X = x)}
    {1 - P(Y = 1 \mid X = x)}} = \beta_0 + \sum_{k=1}^p\bf{X}_k\beta_k \]
    So our decision boundary now looks like:
    \[ \beta_0 + \sum_{k=1}^p\bf{X}_k\beta_k = 0 \]
    Which is a linear function, and is therefore a $p$-dimensional
    hyperplane (a linear decision boundary). \\
\item
    If we have $n$ independent datapoints $\{x_i, y_i\}_{i=1}^n$,
    then the likelihood that we will see this data given our
    logistic regression model $\sigma(f(x))$ is
    the joint probability that $Y = y_i$ given $X=x_i$
    and the parameters $\beta$, for all data points. \\
    Since the datapoints are independent, the joint probability
    is going to be a simple product:
    \[ p(y_1, \dots, y_n \mid x_1, \dots, x_n, \beta)
    = \prod_{i=1}^n p(y_i \mid x_i, \beta) \]
    Since we want to write our likelihood in terms of $\sigma(f(x))$,
    which is $p(y = 1 \mid x)$, we will split our product between
    all data points in class $Y = 1$ and class $Y = 0$:
    \[\prod_{y_i=1} p(y_i = 1 \mid x_i, \beta)
    \prod_{y_i=0} p(y_i = 0 \mid x_i, \beta) \]
    And, since the output is binary, we end up with:
    \[ \prod_{y_i=1} p(y_i = 1 \mid x_i, \beta_i)
    \prod_{y_i=0} (1 - p(y_i = 1 \mid x_i, \beta_i)) \]
    (We can actually write this as a single product
    over all points, which we will do in the next question). \\
    Since we want the log-likelihood, we can take
    the logarithm, which turns products into sums:
    \[ \log\para{\prod_{y_i=1} p(y_i = 1 \mid x_i, \beta_i)
    \prod_{y_i=0} (1 - p(y_i = 1 \mid x_i, \beta_i))} \]
    \[ = \sum_{y_i=1} \log(p(y_i = 1 \mid x_i, \beta_i))
    + \sum_{y_i=0} \log(1 - p(y_i = 1 \mid x_i, \beta_i)) \]
    We can then replace the conditional probability by our 
    logistic regression model:
    \[ \sum_{y_i=1} \log(\sigma(f(x_i)))
    + \sum_{y_i=0} \log(1 -\sigma(f(x_i))) \]
    Our goal is to maximize the likelihood of seeing
    the training data, given our parameters $\beta$.
    We can do this by maximizing the log-likelihood (since
    the logarithm is a monotonically increasing function),
    which can be done by differentiating with repsect
    to the vector of parameters $\beta$, and setting
    the derivative to be equal to $0$. \\
    We can do this using the partial derivative of each 
    individual parameter $\beta_i$ and then solve a system. \\
    This will give us the optimal parameters $\beta$ 
    for our model. \\
\item 
    \begin{enumerate}[label = \romans]
    \item 
        The cost function is:
        \[ \ell(\beta) = 
        -\sum_{i=1}^n y_i\log(p(y_i = 1 \mid x_i, \beta_i)) 
        + (1-y_i)\log(1 - p(y_i = 1 \mid x_i, \beta_i))  \]
        We want to show the relationship it has with the
        log-likelihood. \\
        Since $y_i$ is either $0$ or $1$, 
        we can write the likelihood:
        \[ \prod_{y_i=1} p(y_i = 1 \mid x_i, \beta_i)
        \prod_{y_i=0} (1 - p(y_i = 1 \mid x_i, \beta_i)) \]
        as a single product, using exponents:
        \[ \prod_{i=1}^n p(y_i = 1 \mid x_i, \beta_i)^{y_i}
        (1 - p(y_i = 1 \mid x_i, \beta_i))^{1-y_i} \]
        This works because when $y_i = 1$, the first term
        remains the same, and the second becomes $1$,
        and vice-versa. \\
        When we take the logaritm of the likelihood,
        we thus end up with:
        \[ \log\para{\prod_{i=1}^n p(y_i = 1 \mid x_i, \beta_i)^{y_i}
        (1 - p(y_i = 1 \mid x_i, \beta_i))^{1-y_i}} \]
        \[ = \sum_{i=1}^n y_i\log(p(y_i = 1 \mid x_i, \beta_i))
        + (1-y_i)\log(1 - p(y_i = 1 \mid x_i, \beta_i)) \]
        So the log-likelihood is the negation of the 
        cost function, which is the log-likelihood multiplied
        by $-1$. 
        Thus, finding the optimal parameters requires minimizing
        instead of maximizing the cost function. \\
    \item 
        First we can note that the logistic function
        can be rewritten:
        \[ \sigma(x) = \dfrac{e^x}{1+e^x}
        = \dfrac{e^x \cdot e^{-x}}{(1+e^x) \cdot e^{-x}}
        = \dfrac{1}{1+e^{-x}} \]
        And we note that:
        \[ (1 - \sigma(x)) = 1 - \dfrac{1}{1+e^{-x}}
        = \dfrac{e^{-x}}{1+e^{-x}} \]
        The derivative of the logistic function is:
        \[ \dd{}{x}\sigma(x)
        = \dd{}{x} \para{\dfrac{1}{1+e^{-x}}}
        = -\dfrac{1}{(1 + e^{-x})^2}\dd{}{x}(1+e^{-x}) 
        = \dfrac{e^{-x}}{(1 + e^{-x})^2}\]
        And we can factor it as:
        \[ \dfrac{e^{-x}}{(1 + e^{-x})^2} = 
        \dfrac{1}{1 + e^{-x}} \cdot \dfrac{e^{-x}}{1 + e^{-x}}
        = \sigma(x)(1 - \sigma(x)) \]
        Which is what we wanted to prove. \\
    \item 
        Assuming that all $x_i$ has been padded with a $1$,
        that is $x_{i0} = 1$:
        \[ \beta^\top x_i = 
        \beta_0 + \sum_{k = 1}^p \beta_k x_{ik}
        = \sum_{k = 0}^p \beta_k x_{ik} \]
        we want to calculate:
        \[ \partialdd{}{\beta_j} y_i\log(p(y_i = 1 \mid x_i, \beta)) \]
        We substitute the probability with our logistic
        regression model:
        \[ \partialdd{}{\beta_j} y_i\log(\sigma(\beta^\top x_i)) \]
        Using the chain rule, we get:
        \[ \dfrac{y_i}{\sigma(\beta^\top x_i)}
        \partialdd{\sigma(\beta^\top x_i)}{\beta_j} \]
        Then, using the property 
        $\dd{}{x}\sigma(x) = \sigma(x)(1-\sigma(x))$
        and the chain rule again, we get:
        \[ \dfrac{y_i}{\sigma(\beta^\top x_i)}
        \sigma(\beta^\top x_i)(1 - \sigma(\beta^\top x_i)) 
        \partialdd{\beta^\top x_i}{\beta_j}
        =y_i(1 - \sigma(\beta^\top x_i)) 
        \partialdd{\beta^\top x_i}{\beta_j} \]
        Since all $\beta_k$ except for $\beta_j$ are treated
        as constants, the derivative of $\beta^\top x_i$
        with respect to $\beta_j$ 
        is just the derivative of $\beta_jx_{ij}$,
        which is $x_{ij}$, so we end up with:
        \[ y_i(1 - \sigma(\beta^\top x_i)) x_{ij} \]
        Which is the form we wanted. \\
    \item 
        We can use a similar derivation for:
        \[ \partialdd{}{\beta_j} 
        (1-y_i)\log(1 - p(y_i = 1 \mid x_i, \beta)) \]
        First we substitute the probability with our model:
        \[ \partialdd{}{\beta_j} 
        (1-y_i)\log(1 - \sigma(\beta^\top x_i)) \]
        Then we use the chain rule and get:
        \[ \dfrac{1 - y_i}{1 - \sigma(\beta^\top x_i)}
        \partialdd{(1 - \sigma(\beta^\top x_i))}{\beta_j}
        = -\dfrac{1 - y_i}{1 - \sigma(\beta^\top x_i)}
        \partialdd{\sigma(\beta^\top x_i)}{\beta_j} \]
        Then, using the property 
        $\dd{}{x}\sigma(x) = \sigma(x)(1-\sigma(x))$
        and the chain rule again, we get:
        \[ -\dfrac{1 - y_i}{1 - \sigma(\beta^\top x_i)}
        \sigma(\beta^\top x_i)(1 - \sigma(\beta^\top x_i)) 
        \partialdd{\beta^\top x_i}{\beta_j}
        = -(1-y_i)\sigma(\beta^\top x_i)
        \partialdd{\beta^\top x_i}{\beta_j} \]
        Again, we know that the derivative
        with respect to $\beta_j$ of $\beta^\top x_i$
        is just the derivative of $\beta_jx_{ij}$,
        which is $x_{ij}$, so we end up with:
        \[ -(1-y_i)\sigma(\beta^\top x_i)x_{ij} \]
        Which is the form we want. \\
    \item 
        We want to find:
        \[ \partialdd{}{\beta_j} y_i\log(p(y_i = 1 \mid x_i, \beta))
        + \partialdd{}{\beta_j} 
        (1-y_i)\log(1 - p(y_i = 1 \mid x_i, \beta)) \]
        Substituting with the results we derived, we have:
        \[ y_i(1 - \sigma(\beta^\top x_i)) x_{ij}
        - (1-y_i)\sigma(\beta^\top x_i)x_{ij} \]
        Expanding, we get:
        \[ y_ix_{ij} - y_i\sigma(\beta^\top x_i)x_{ij}
        + y_i\sigma(\beta^\top x_i)x_{ij} 
        - \sigma(\beta^\top x_i)x_{ij} \]
        Which we can simplify to:
        \[ y_ix_{ij} - \sigma(\beta^\top x_i)x_{ij}
        = (y_i - \sigma(\beta^\top x_i))x_{ij} \]
        Which is the result we needed to get. \\
    \item 
        We want to find the partial derivative:
        \[ \partialdd{}{\beta_j} \ell(\beta) \]
        Which is:
        \[ \partialdd{}{\beta_j} \brac{-\sum_{i=1}^n 
        y_i\log(p(y_i = 1 \mid x_i, \beta_i)) 
        + \log(1 - p(y_i = 1 \mid x_i, \beta_i))} \]
        The derivative of a sum of terms if the sum of
        the derivatives of those terms, so we get:
        \[ -\sum_{i=1}^n \partialdd{}{\beta_j} 
        ( y_i\log(p(y_i = 1 \mid x_i, \beta_i)) 
        + \log(1 - p(y_i = 1 \mid x_i, \beta_i) )) \]
        Using the previous result, we get:
        \[ -\sum_{i=1}^n (y_i - \sigma(\beta^\top x_i))x_{ij}
        = \sum_{i=1}^n (\sigma(\beta^\top x_i) - y_i)x_{ij} \]
        Which is derivative of the cost function. \\
    \end{enumerate}
\item 
    \begin{enumerate}[label = \romans]
    \item 
        The calculated probablities are:
        \[ P(y_1 = 1 \mid x_1, \beta) \approx 0.0474 \]
        \[ P(y_2 = 1 \mid x_2, \beta) \approx 0.2689 \]
        \[ P(y_3 = 1 \mid x_3, \beta) \approx 0.3775 \]
        \[ P(y_4 = 1 \mid x_4, \beta) \approx 0.6225 \]
        \[ P(y_5 = 1 \mid x_5, \beta) \approx 0.6225 \]
        \[ P(y_6 = 1 \mid x_6, \beta) \approx 0.8808 \]
        We used a python script to expedite solving this
        part (since it is purely computational). \\
    \item 
        We just compare the probability from the last part
        with the threshold of $0.5$, and arrive at:
        \[ \hat{y}_1 = \hat{y}_2 = \hat{y}_3 = 0 \]
        \[ \hat{y}_4 = \hat{y}_5 = \hat{y}_6 = 1 \]
        Which are the model's predicted values. \\
    \item 
        All of the points except $(x_2, y_2)$, $(x_4, y_4)$,
        and $(x_5, y_5)$ are classified correctly, 
        so the accuracy is $\dfrac{3}{6} = 0.5$. \\
    \item 
        The loss is calculated as:
        \[\ell(\beta) = 
        -\sum_{i=1}^n y_i\log(P(y_i = 1 \mid x_i, \beta_i)) 
        + (1-y_i)\log(1 - P(y_i = 1 \mid x_i, \beta_i))  \]
        Using the given $\beta$ values, we get
        \[ \ell \para{ \bmat{-3 \\ 0.5 \\ -0.5} } \approx 3.911 \]
        For $\beta_0 = -3$, $\beta_1 = 0.5$, 
        and $\beta_2 = -0.5$. \\
    \end{enumerate}
\end{enumerate}

\newpage

\end{document}