

\documentclass[12pt]{article}

\usepackage[margin=1in]{geometry}


%===============================================================================
%================================== PACKAGES ===================================
%===============================================================================

% For using float option H that places figures 
% exatcly where we want them
\usepackage{float}
% makes figure font bold
\usepackage{caption}
\captionsetup[figure]{labelfont=bf}
% For text generation
\usepackage{lipsum}
% For drawing
\usepackage{tikz}
% For smaller or equal sign and not divide sign
\usepackage{amssymb}
% For the diagonal fraction
\usepackage{xfrac}
% For enumerating exercise parts with letters instead of numbers
\usepackage{enumitem}
% For dfrac, which forces the fraction to be in display mode (large) e
% even in math mode (small)
\usepackage{amsmath}
% For degree sign
\usepackage{gensymb}
% For "\mathbb" macro
\usepackage{amsfonts}
% For positioning 
\usepackage{indentfirst}
\usetikzlibrary{shapes,positioning,fit,calc}
% for adjustwidth environment
\usepackage{changepage}
% for arrow on top
\usepackage{esvect}
% for mathbb 1
\usepackage{bbm}
% for mathsrc
\usepackage[mathscr]{eucal}
% For degree sign
\usepackage{gensymb}
% For quotes
\usepackage{csquotes}
% For vertical lines
\usepackage{mathtools}
% For cols
\usepackage{multicol}

% for tikz
\usepackage{pgfplots}
\pgfplotsset{compat=1.18}
\usepackage{amsmath}
\usepgfplotslibrary{groupplots}


%===============================================================================
%==================================== FONTS ====================================
%===============================================================================


% Mathcal
\newcommand{\acal}{\mathcal{A}}
\newcommand{\bcal}{\mathcal{B}}
\newcommand{\ccal}{\mathcal{C}}
\newcommand{\dcal}{\mathcal{D}}
\newcommand{\ecal}{\mathcal{E}}
\newcommand{\fcal}{\mathcal{F}}
\newcommand{\gcal}{\mathcal{G}}
\newcommand{\hcal}{\mathcal{H}}
\newcommand{\ical}{\mathcal{I}}
\newcommand{\jcal}{\mathcal{J}}
\newcommand{\kcal}{\mathcal{K}}
\newcommand{\lcal}{\mathcal{L}}
\newcommand{\mcal}{\mathcal{M}}
\newcommand{\ncal}{\mathcal{N}}
\newcommand{\ocal}{\mathcal{O}}
\newcommand{\pcal}{\mathcal{P}}
\newcommand{\qcal}{\mathcal{Q}}
\newcommand{\rcal}{\mathcal{R}}
\newcommand{\scal}{\mathcal{S}}
\newcommand{\tcal}{\mathcal{T}}
\newcommand{\ucal}{\mathcal{U}}
\newcommand{\vcal}{\mathcal{V}}
\newcommand{\wcal}{\mathcal{W}}
\newcommand{\xcal}{\mathcal{X}}
\newcommand{\ycal}{\mathcal{Y}}
\newcommand{\zcal}{\mathcal{Z}}

% Mathfrak
\newcommand{\afrak}{\mathfrak{A}}
\newcommand{\bfrak}{\mathfrak{B}}
\newcommand{\cfrak}{\mathfrak{C}}
\newcommand{\dfrak}{\mathfrak{D}}
\newcommand{\efrak}{\mathfrak{E}}
\newcommand{\ffrak}{\mathfrak{F}}
\newcommand{\gfrak}{\mathfrak{G}}
\newcommand{\hfrak}{\mathfrak{H}}
\newcommand{\ifrak}{\mathfrak{I}}
\newcommand{\jfrak}{\mathfrak{J}}
\newcommand{\kfrak}{\mathfrak{K}}
\newcommand{\lfrak}{\mathfrak{L}}
\newcommand{\mfrak}{\mathfrak{M}}
\newcommand{\nfrak}{\mathfrak{N}}
\newcommand{\ofrak}{\mathfrak{O}}
\newcommand{\pfrak}{\mathfrak{P}}
\newcommand{\qfrak}{\mathfrak{Q}}
\newcommand{\rfrak}{\mathfrak{R}}
\newcommand{\sfrak}{\mathfrak{S}}
\newcommand{\tfrak}{\mathfrak{T}}
\newcommand{\ufrak}{\mathfrak{U}}
\newcommand{\vfrak}{\mathfrak{V}}
\newcommand{\wfrak}{\mathfrak{W}}
\newcommand{\xfrak}{\mathfrak{X}}
\newcommand{\yfrak}{\mathfrak{Y}}
\newcommand{\zfrak}{\mathfrak{Z}}

% Mathscr
\newcommand{\ascr}{\mathscr{A}}
\newcommand{\bscr}{\mathscr{B}}
\newcommand{\cscr}{\mathscr{C}}
\newcommand{\dscr}{\mathscr{D}}
\newcommand{\escr}{\mathscr{E}}
\newcommand{\fscr}{\mathscr{F}}
\newcommand{\gscr}{\mathscr{G}}
\newcommand{\hscr}{\mathscr{H}}
\newcommand{\iscr}{\mathscr{I}}
\newcommand{\jscr}{\mathscr{J}}
\newcommand{\kscr}{\mathscr{K}}
\newcommand{\lscr}{\mathscr{L}}
\newcommand{\mscr}{\mathscr{M}}
\newcommand{\nscr}{\mathscr{N}}
\newcommand{\oscr}{\mathscr{O}}
\newcommand{\pscr}{\mathscr{P}}
\newcommand{\qscr}{\mathscr{Q}}
\newcommand{\rscr}{\mathscr{R}}
\newcommand{\sscr}{\mathscr{S}}
\newcommand{\tscr}{\mathscr{T}}
\newcommand{\uscr}{\mathscr{U}}
\newcommand{\vscr}{\mathscr{V}}
\newcommand{\wscr}{\mathscr{W}}
\newcommand{\xscr}{\mathscr{X}}
\newcommand{\yscr}{\mathscr{Y}}
\newcommand{\zscr}{\mathscr{Z}}

% Mathbb
\newcommand{\abb}{\mathbb{A}}
\newcommand{\bbb}{\mathbb{B}}
\newcommand{\cbb}{\mathbb{C}}
\newcommand{\dbb}{\mathbb{D}}
\newcommand{\ebb}{\mathbb{E}}
\newcommand{\fbb}{\mathbb{F}}
\newcommand{\gbb}{\mathbb{G}}
\newcommand{\hbb}{\mathbb{H}}
\newcommand{\ibb}{\mathbb{I}}
\newcommand{\jbb}{\mathbb{J}}
\newcommand{\kbb}{\mathbb{K}}
\newcommand{\lbb}{\mathbb{L}}
\newcommand{\mbb}{\mathbb{M}}
\newcommand{\nbb}{\mathbb{N}}
\newcommand{\obb}{\mathbb{O}}
\newcommand{\pbb}{\mathbb{P}}
\newcommand{\qbb}{\mathbb{Q}}
\newcommand{\rbb}{\mathbb{R}}
\newcommand{\sbb}{\mathbb{S}}
\newcommand{\tbb}{\mathbb{T}}
\newcommand{\ubb}{\mathbb{U}}
\newcommand{\vbb}{\mathbb{V}}
\newcommand{\wbb}{\mathbb{W}}
\newcommand{\xbb}{\mathbb{X}}
\newcommand{\ybb}{\mathbb{Y}}
\newcommand{\zbb}{\mathbb{Z}}


%===============================================================================
%=============================== SPECIAL SYMBOLS ===============================
%===============================================================================


% Orbit (group theory)
\newcommand{\orbit}{\mathcal{O}}
% Normal group
\newcommand{\normal}{\mathcal{N}}
% Indicator function
\newcommand{\indicator}{\mathbbm{1}}
% Laplace transform
\newcommand{\laplace}[1]{\mathcal{L}}
% Epsilon shorthand
\newcommand{\eps}{\varepsilon}
% Omega
\newcommand{\om}{\omega}
\newcommand{\Om}{\Omega}

%===============================================================================
%================================== OPERATORS ==================================
%===============================================================================


% Inverse exponent
\newcommand{\inv}[0]{^{-1}}
% Overline bar
\newcommand{\olsi}[1]{\,\overline{\!{#1}}}
% Less than or equal slanted
\newcommand{\seqs}{\leqslant}
% Greater or equal slanted
\newcommand{\geqs}{\geqslant}
% Subset or equal
\newcommand{\sub}{\subseteq}
% Proper subset
\newcommand{\prosub}{\subset}
% from
\newcommand{\from}{\leftarrow}

% Parantheses
\newcommand{\para}[1]{\left( #1 \right)}
% Curly Braces
\newcommand{\curl}[1]{\left\{ #1 \right\}}
% Brackets
\newcommand{\brac}[1]{\left[ #1 \right]}
% Angled Brackets
\newcommand{\ang}[1]{\left\langle #1 \right\rangle}
% Norm
\newcommand{\norm}[1]{\left\| #1 \right\|}

% Piece wise (use \\ between cases)
\newcommand{\piecewise}[1]{\begin{cases} #1 \end{cases}}

% Bold symbol shorthand
\newcommand{\bl}{\boldsymbol}

% Vertical space
\newcommand{\vs}[1]{\vspace{#1 pt}}
% Horizontal ertical space
\newcommand{\hs}[1]{\hspace{#1 pt}}


%===============================================================================
%============================== TEXT BASED SYMBOLS =============================
%===============================================================================


% Radians
\newcommand{\rad}{\text{rad}}
% Least Common Multiple
\newcommand{\lcm}{\text{lcm}}
% Automorphism
\newcommand{\Aut}{\text{Aut}}
% Variance
\newcommand{\var}{\text{Var}}
% Covariance
\newcommand{\cov}{\text{Cov}}
% Cofactor (matrix)
\newcommand{\cof}{\text{Cof}}
% Adjugate (matrix)
\newcommand{\adj}{\text{Adj}}
% Trace (matrix)
\newcommand{\tr}{\text{tr}}
% Standard deviation
\newcommand{\std}{\text{Std}}
% Correlation coefficient
\newcommand{\corr}{\text{Corr}}
% Sign
\newcommand{\sign}{\text{sign}}

% And text
\newcommand{\AND}{\text{ and }}
% Or text 
\newcommand{\OR}{\text{ or }}
% If text
\newcommand{\IF}{\text{ if }}
% When text
\newcommand{\WHEN}{\text{ when }}
% Then text
\newcommand{\THEN}{\text{ then }}

% 1st
\newcommand{\st}[1]{#1^{\text{st}}}
% 2nd
\newcommand{\nd}[1]{#1^{\text{nd}}}
% 3rd
\newcommand{\rd}[1]{#1^{\text{rd}}}
% nth
\newcommand{\nth}[1]{#1^{\text{th}}}


%===============================================================================
%========================= PROBABILITY AND STATISTICS ==========================
%===============================================================================


% Permutation
\newcommand{\perm}[2]{{}^{#1}\!P_{#2}}
% Combination
\newcommand{\comb}[2]{{}^{#1}C_{#2}}

% Baye's risk
\newcommand{\risk}[1]{\mathscr{R}_{#1}}
% Baye's optimal risk
\newcommand{\riskOptimal}[1]{\mathscr{R}_{#1}^*}
% Baye's empirical risk
\newcommand{\riskEmpirical}[2]{\hat{\mathscr{R}}_{#1}^{#2}}


%===============================================================================
%=================================== CALCULUS ==================================
%===============================================================================


% d over d derivative
\newcommand{\dd}[2]{\dfrac{d#1}{d#2}}
% partial d over d derivative
\newcommand{\partialdd}[2]{\dfrac{\partial #1}{\partial #2}}
% delta d over d derivative
\newcommand{\deltadd}[2]{\dfrac{\Delta #1}{\Delta #2}}

% Integration between a and b
\newcommand{\integral}[4]{\int_{#1}^{#2} #3 \, #4}
% Integration in some space 
\newcommand{\boundIntegral}[2]{\int_{#1} #2 \, d#1}

% Limit
\newcommand{\limit}[3]{\lim_{#1 \to #2} #3}


%===============================================================================
%================================  BIG SYMBOLS  ================================
%===============================================================================


% Sum
\newcommand{\sumof}[2]{\sum_{#1}^{#2}}
% Product
\newcommand{\productof}[2]{\prod_{#1}^{#2}}
% Union
\newcommand{\unionof}[2]{\bigcup_{#1}^{#2}}
% Intersection
\newcommand{\intersectionof}[2]{\bigcap_{#1}^{#2}}
% Or
\newcommand{\orof}[2]{\bigvee_{#1}^{#2}}
% And
\newcommand{\andof}[2]{\bigwedge_{#1}^{#2}}


%===============================================================================
%=============================== LINEAR ALGEBRA ================================
%===============================================================================


% Bold vector arrow
\newcommand{\bv}[1]{\vec{\mathbf{#1}}}

% Matrix or vector (use // for column, & for row) 
% with brackets
\newcommand{\bmat}[1]{\begin{bmatrix} #1 \end{bmatrix}}
% Matrix or vector (use // for column, & for row) 
% with curved brackets
\newcommand{\pmat}[1]{\begin{pmatrix} #1 \end{pmatrix}}
% Matrix or vector (use // for column, & for row) 
% with lines on either side 
\newcommand{\lmat}[1]{\begin{vmatrix} #1 \end{vmatrix}}
% Matrix or vector (use // for column, & for row) 
% with curly braces
\newcommand{\cmat}[1]{\begin{Bmatrix} #1 \end{Bmatrix}}
% Matrix or vector (use // for column, & for row) 
% with no braces
\newcommand{\mat}[1]{\begin{matrix} #1 \end{matrix}}


%===============================================================================
%================================ LARGE OBJECTS ================================
%===============================================================================

% Multiple lines
\newcommand{\multiline}[1]{
\begin{align*}
    #1
\end{align*}
}

% Multiple lines with equation numbers
\newcommand{\eqmultiline}[1]{
\begin{align*}
    #1
\end{align*}
}

% Color
\newcommand{\colorText}[2]{
\begingroup
\color{#1}
    #2
\endgroup
}

% Centered figure
\newcommand{\centerFigure}[2]{
    \begin{figure}[h]
        \centering
            #1
        \caption{#2}
    \end{figure}
}

% Tikz figure
\newcommand{\tikzGraphic}[1]{
    \begin{center}
    \begin{tikzpicture}
        #1
    \end{tikzpicture}
    \end{center}
}

% Enumerate numbers (seperate by \item)
\newcommand{\numbers}{\textbf{\number*)}}

% Enumerate letters (seperate by \item)
\newcommand{\letters}{\textbf{\alph*)}}

\title{
    \Huge Elements of Machine Learning \\
    \huge Assignment II \\
    \LARGE Problem 2
}
\date{2025-11-15}
\author{
    Michael Saba (Matriculation No. 7076371) \\
    Chongbiao Wang (Matriculation No. 7076265)
}

\pagestyle{fancy}
\fancyhf{}
\fancyhead[L]{Michael Saba 7076371}
\fancyhead[R]{Chongbiao Wang 7076265}
\fancyfoot[C]{\thepage}

\begin{document}
\pagenumbering{gobble}
\maketitle
\newpage
\setlength{\parindent}{0pt}
\pagenumbering{arabic}

\subsection*{Problem 2}

\begin{enumerate}[label = \letters]
\item
    To estimate the mean of each class, 
    we can use the sample mean $\mu_k$ for the set of
    points with output $k$:
    \[ \mu_k = \dfrac{1}{n_k} \sum_{y_i = k} x_i \]
    Where $n_k$ is the number of points in class $k$. \\
    Note that since the input $x_i \in \rbb^2$ is a vector with
    components $x_{i1}$ and $x_{i2}$, the data points we sum
    are vectors. \\
    We have $4$ point in classes $1$ and $2$:
    \[ \mu_1 = \dfrac{1}{4}\para{\bmat{1 \\ 2} 
    + \bmat{2 \\ 1} + \bmat{2 \\ 3} + \bmat{3 \\ 2}}
    = \bmat{2 \\ 2} \]
    \[ \mu_2 = \dfrac{1}{4}\para{\bmat{3 \\ 4} 
    + \bmat{3 \\ 6} + \bmat{5 \\ 4} + \bmat{5 \\ 6}}
    = \bmat{4 \\ 5} \]
    So now we have an estimate for the class means. \\
\item 
    First we calculate the covariance matrices of each
    class on its own, using the formula (for some class $k$):
    \[ \cov(X) = \Sigma_k = \ebb[(X - \mu_k)(X - \mu_k)^\top] \]
    We can use an unbiased estimator for the covariance
    matrix (since we don't have access to the whole distribution):
    \[ \hat{\Sigma}_k = \dfrac{1}{n_k - 1} 
    \sum_{y_i = k}(x_i - \mu_k)(x_i - \mu_k)^\top \]
    Calculating the covariance matrices, we get:
    \[ \hat{\Sigma}_1 = \dfrac{1}{3}\bmat{2 & 0 \\ 0 & 2} \]
    \[ \hat{\Sigma}_2 = \dfrac{1}{3}\bmat{4 & 0 \\ 0 & 4} \]
    The pooled covariance matrix has a formula:
    \[ \Sigma = \dfrac{\sum_{k=1}^K (n_k-1) \cdot \Sigma_k}
    {\sum_{k=1}^K (n_k-1)} \]
    Which in our case we can estimate to be:
    \[ \Sigma = \dfrac{1}{3 + 3}
    \para{\bmat{2 & 0 \\ 0 & 2} + \bmat{4 & 0 \\ 0 & 4}}
    = \bmat{1 & 0 \\ 0 & 1} = I \]
    Which is the identity matrix. \\
\item 
    We know that LDA is a generative approach to classification;
    that is, we estimate the prior $P(Y)$ and
    the conditional $P(X \mid Y = k)$, which we can use along
    with Baye's rule to get an estimate on $P(Y = k \mid X)$. \\
    Our goal then is to pick the class $k$ 
    that maximizes the estimate of this probability.
    After simplifying for a bit, and taking the logarithm,
    we end up with the discriminant:
    \[ \delta_k(x) = x^\top \Sigma^{-1} \mu_k - 
    \frac{1}{2} \mu_k^\top \Sigma^{-1} \mu_k + \log(\pi_k) \]
    Since the logarithm is a monotonically increasing function,
    the class $k$ that maximizes the discriminant also
    maximizes the conditional probability estimate. \\
    We were given an input and priors:
    \[ x = \bmat{4 \\ 3} \qquad \pi_1 = \pi_2 = 0.5 \]
    So the discriminants are:
    \[ \delta_1\para{\bmat{4 \\ 3}} = 
    \bmat{4 & 3}\bmat{1 & 0 \\ 0 & 1}\inv \bmat{2 \\ 2}
    -\dfrac{1}{2}\bmat{2 & 2}\bmat{1 & 0 \\ 0 & 1}\inv \bmat{2 \\ 2}
    = 10 + \log(0.5) \]
    \[ \delta_2\para{\bmat{4 \\ 3}} = 
    \bmat{4 & 3}\bmat{1 & 0 \\ 0 & 1}\inv \bmat{4 \\ 5}
    -\dfrac{1}{2}\bmat{4 & 5}\bmat{1 & 0 \\ 0 & 1}\inv \bmat{4 \\ 5}
    = \dfrac{21}{2} + \log(0.5) \]
    The discriminant of class $2$ is larger, so we should
    classify the point into class $2$. \\
\item 
    The points $x$ on the decision boundary are those
    that are equally likely to be classified into
    classes $1$ and $2$, that is, the discriminants are equal:
    \[ \delta_1(x) = \delta_2(x)  \]
    \[ x^\top \Sigma^{-1} \mu_1 - 
    \frac{1}{2} \mu_1^\top \Sigma^{-1} \mu_1 + \log(\pi_1)
    = x^\top \Sigma^{-1} \mu_2 - 
    \frac{1}{2} \mu_2^\top \Sigma^{-1} \mu_2 + \log(\pi_2) \]
    We can then rearrange it such that $x$ is on one side,
    and we arrive at out linear decision boundary:
    \[ x^\top \para{ \Sigma^{-1} \mu_1 - \Sigma^{-1} \mu_2}
    = \frac{1}{2} \para{ \mu_1^\top \Sigma^{-1} \mu_1
    - \mu_2^\top \Sigma^{-1} \mu_2} + \log(\pi_2) - \log(\pi_1) \]
    Assuming equal priors, and substituting with the actual
    values, we get:
    \[ x^\top \para{\bmat{2 \\ 2} - \bmat{4 \\ 5} }
    = \frac{1}{2} \para{ 8 - 41} + \log(0.5) - \log(0.5) \]
    \[ 2x_1 + 3x_2 = \frac{33}{2} \]
    We can draw this as a linear boundary on the figure:
    \begin{center}
    \tikzGraphic{
        % Grid
        \draw[step=1cm,gray,very thin] (0,0) grid (7,7);

        % Axes
        \draw[thick,->] (0,0) -- (7.5,0) node[right] {$x_1$};
        \draw[thick,->] (0,0) -- (0,7.5) node[above] {$x_2$};

        % Points
        \coordinate (p1) at (1,2);
        \coordinate (p2) at (2,1);
        \coordinate (p3) at (2,3);
        \coordinate (p4) at (3,2);

        \coordinate (p5) at (3,4);
        \coordinate (p6) at (3,6);
        \coordinate (p7) at (5,4);
        \coordinate (p8) at (5,6);

        % Drawing
        \foreach \p in {p1,p2,p3,p4} {
            \fill[blue] (\p) circle (4pt);
        }
        \foreach \p in {p5,p6,p7,p8} {
            \fill[red] (\p) circle (4pt);
        }

        % Line
        \draw[thick, dashed, purple] (0,5.5) -- (7,0.1667);
    }
    \end{center}
    As we can see it correctly classifies the points. \\
\item 
    LDA assumes that the covariance matrix of the data
    is the same for all classes, 
    while QDA makes no assumptions about the
    covariance matrices. \\
    In practice, what this does is turn the linear boundary
    into a quadratic boundary (unless the covariance
    matrices happen to be equal, which degenerates to LDA). \\ 
    If we were to go back to
    the generative model that estimates the prior and conditional,
    we would notice that it looks to be quadratic in $x$,
    but simplifies to linear because the covariance matrix
    is the same for all classes, and the quadratic terms
    cancel out, leaving linear terms that lead to a 
    linear boundary. \\
    If the covariance matrices were different for each class,
    the quadratic terms would remain, and the decision boundary
    would become quadratic. \\
    Moreover, QDA has more parameters than LDA, since each
    class has its own covariance matrix. \\ 
\end{enumerate}

\newpage

\end{document}